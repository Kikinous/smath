% This is part of Un soupçon de mathématique sans être agressif pour autant
% Copyright (c) 2013
%   Laurent Claessens
% See the file fdl-1.3.txt for copying conditions.

\begin{exercice}\label{exosmath-0552}

    \( ABCD\) est un trapèze rectangle avec \( AB=8\), \( AD=5\) et \( BC=3\). Pour tout point \( M\) du segment \( [AB]\) nous notons \( x\) la distance \( AM\). Nous notons \( f\), \( g\) et \( h\) les fonctions qui à \( x\) associent l'aire des triangles \( AMD\), \( MBC\) et \( BCM\).

\begin{wrapfigure}{r}{5.0cm}
   \vspace{-0.5cm}        % à adapter.
   \centering
   \input{Fig_EDnzyzS.pstricks}
\end{wrapfigure}

\corrref{smath-0552}
\end{exercice}
