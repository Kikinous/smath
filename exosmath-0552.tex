% This is part of Un soupçon de mathématique sans être agressif pour autant
% Copyright (c) 2013
%   Laurent Claessens
% See the file fdl-1.3.txt for copying conditions.

\begin{exercice}\label{exosmath-0552}

\begin{wrapfigure}{r}{7.0cm}
   \vspace{-0.5cm}        % à adapter.
   \centering
   \input{Fig_EDnzyzS.pstricks}

   \input{Fig_BZRzIsR.pstricks}
\end{wrapfigure}

    Le quadrilatère \( ABCD\) est un trapèze rectangle avec \( AB=8\), \( AD=5\) et \( BC=3\). Pour tout point \( M\) du segment \( [AB]\) nous notons \( x\) la distance \( AM\). Nous notons \( f\), \( g\) et \( h\) les fonctions qui à \( x\) associent l'aire des triangles \( AMD\), \( MBC\) et \( BCM\).
    \begin{enumerate}
        \item
            À quel intervalle appartient \( x\) ?
        \item
            Montrer que \( f(x)=\frac{ 5 }{2}x\).
        \item
            Montrer que \( g(x)=-\frac{ 3 }{2}x+12\).
        \item
            En déduire l'expression de \( h(x)\).
    \end{enumerate}
    Sur le graphe ci-contre nous avons représenté deux droites. Chacune est soit \( f\), soit \( g\) soit \( h\). Associer la droite à la fonction et tracer celle qui manque.

%\begin{wrapfigure}{r}{10.cm}
%   \vspace{-0.5cm}        % à adapter.
%   \centering
%\end{wrapfigure}

%\begin{multicols}{2}

    \begin{enumerate}
        \item
            Peut-on trouver un point \( M\) tel que \( AMD\) et \( MCD\) aient la même aire ? Justifier graphiquement.
        \item
            Déterminer la le calcul la position du point \( M\) pour que l'aire de \( AMD\) soit égale à celle de \( BCM\).
    \end{enumerate}

%    \columnbreak

%\end{multicols}

\corrref{smath-0552}
\end{exercice}
