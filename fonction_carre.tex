%This is part of Un soupçon de mathématique sans être agressif pour autant
% Copyright (c) 2012-2013
%   Laurent Claessens, Pauline Klein
% See the file fdl-1.3.txt for copying conditions.

%+++++++++++++++++++++++++++++++++++++++++++++++++++++++++++++++++++++++++++++++++++++++++++++++++++++++++++++++++++++++++++ 
\section{Graphique et tableau de variation}
%+++++++++++++++++++++++++++++++++++++++++++++++++++++++++++++++++++++++++++++++++++++++++++++++++++++++++++++++++++++++++++

La fonction carré est la fonction définie sur \( \eR\) qui à \( x\) fait correspondre \( x^2\).

\begin{multicols}{2}

        \begin{equation*}
            \begin{array}[]{c|ccccc}
                x&-\infty&&0&&+\infty\\
                \hline
                &+\infty&&&&+\infty\\
                &&\searrow&&\nearrow&\\
                &&&0&&
            \end{array}
        \end{equation*}

        \columnbreak

        %The result is on figure \ref{LabelFigfigureXNAufCh}. % From file figureXNAufCh
        %\newcommand{\CaptionFigfigureXNAufCh}{<+Type your caption here+>}
        \begin{center}
\input{Fig_figureXNAufCh.pstricks}
        \end{center}
\end{multicols}

\begin{definition}
    Le graphe de la fonction carré est une \defe{parabole}{parabole}.
\end{definition}
%+++++++++++++++++++++++++++++++++++++++++++++++++++++++++++++++++++++++++++++++++++++++++++++++++++++++++++++++++++++++++++ 
\section{Double antécédent et solutions d'équations}
%+++++++++++++++++++++++++++++++++++++++++++++++++++++++++++++++++++++++++++++++++++++++++++++++++++++++++++++++++++++++++++

\begin{Aretenir}
    La fonction carré a pour principale caractéristique d'avoir \emph{deux} antécédents pour chaque image (à part pour zéro). Les antécédents du nombre \( a\) sont \( \sqrt{a}\) et \( -\sqrt{a}\).
\end{Aretenir}

\begin{example}
    Les carrés de \( 4\) et de \( -4\) sont tous les deux \( 16\).
\end{example}

Au niveau des inéquations, cela se ressent. La fonction \( x^2\) sera par exemple plus petite que \( 4\) non seulement pour les \( x\in\mathopen[ 0 , 3 \mathclose]\) mais aussi pour \( x\in\mathopen[ -3 , 0 \mathclose]\).

Nous voyons cela sur la figure \ref{LabelFigfigureEWDVDTS} qui montre les solution de \( f(x)\leq 4\). % From file figureEWDVDTS
\newcommand{\CaptionFigfigureEWDVDTS}{La résulution graphique d'une inéquation avec la fonction carré.}
\input{Fig_figureEWDVDTS.pstricks}

%+++++++++++++++++++++++++++++++++++++++++++++++++++++++++++++++++++++++++++++++++++++++++++++++++++++++++++++++++++++++++++ 
\section{Encadrement}
%+++++++++++++++++++++++++++++++++++++++++++++++++++++++++++++++++++++++++++++++++++++++++++++++++++++++++++++++++++++++++++

\begin{example}
    Le carré de \( 3\) est \( 9\) qui est plus grand que \( 3\). De même le carré de \( 10\) est \( 100\) qui est encore beaucoup plus grand.

    Le carré de \( 0.1\) est \( 0.01\) et donc plus petit que \( 0.1\).
\end{example}


\begin{Aretenir}
    Si \( 0<x<1\), alors \( x^2<x\). Si \( x>1\), alors \( x^2>x\).
\end{Aretenir}

\begin{Aretenir}
    Voici comment la fonction carré ordonne les nombres.
    \begin{enumerate}
        \item
            Si \( 0<x<y\), alors \( 0<x^2<y^2\).
        \item
            Si \( x<y<0\), alors \( x^2>y^2>0\).
    \end{enumerate}

    Dit en français :
    \begin{enumerate}
        \item
            les carrés de nombres négatifs sont rangés dans l’ordre contraire;
        \item
            les carrés de nombres positifs sont rangés dans le même ordre.
    \end{enumerate}
\end{Aretenir}

%+++++++++++++++++++++++++++++++++++++++++++++++++++++++++++++++++++++++++++++++++++++++++++++++++++++++++++++++++++++++++++ 
\section{Exercices : la fonction carré}
%+++++++++++++++++++++++++++++++++++++++++++++++++++++++++++++++++++++++++++++++++++++++++++++++++++++++++++++++++++++++++++

\Exo{smath-0176}
\Exo{smath-0178}
\Exo{smath-0252}
\Exo{smath-0179}
\Exo{smath-0177}
\Exo{smath-0141}
\Exo{smath-0254}
\Exo{smath-0139}
\Exo{smath-0180}
\Exo{smath-0140}
\Exo{smath-0174}
\Exo{smath-0175}
\Exo{smath-0253}
\Exo{smath-0255}
