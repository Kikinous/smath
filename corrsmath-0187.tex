% This is part of Un soupçon de mathématique sans être agressif pour autant
% Copyright (c) 2013
%   Laurent Claessens
% See the file fdl-1.3.txt for copying conditions.

\begin{corrige}{smath-0187}

    \begin{enumerate}
        \item
            L'univers est l'ensemble de toutes les cartes.
        \item
            \begin{enumerate}
                \item
                    Il y a une seule dame de cœur, donc \( 1/52\).
                \item
                    Il y a \( 26\) cartes rouges, donc \( 26/52=1/4\).
                \item
                    Il y a quatre dix, donc \( 1/13\).
                \item
                    Il y a \( 26 \) cartes rouges, et parmi les quatre as, seul l'as de trèfle et de pique ne sont pas encore comptés dans les rouges. Donc il y a \( 26+2\) bonnes cartes et la probabilité d'en tirer une est \( 28/52=7/13\).
            \end{enumerate}
    \end{enumerate}

\end{corrige}
