% This is part of Un soupçon de mathématique sans être agressif pour autant
% Copyright (c) 2013
%   Laurent Claessens
% See the file fdl-1.3.txt for copying conditions.

\begin{exercice}\label{exosmath-0400}

%The result is on figure \ref{LabelFigIXtyOnk}. % From file IXtyOnk
%\newcommand{\CaptionFigIXtyOnk}{<+Type your caption here+>}

\begin{minipage}{0.485\textwidth}
    Une grenouille effectue un bond dont nous modélisons la trajectoire par la parabole
    \begin{equation}
        f(x)=-5x^2+x.
    \end{equation}
    \begin{enumerate}
        \item
            Calculer l'angle à l'horizontale que la grenouille donne à son saut au départ. (penser à la dérivée au point \( x=0\))
        \item
            En quelle abscisse la grenouille est-elle au plus haut ?
        \item
            Quelle est la hauteur maximale atteinte ?
        \item
            Donner l'équation de la tangente à la trajectoire au sommet. Est-ce logique ?
        \item
            Quelle est la longueur \( l\) totale du saut ?
    \end{enumerate}
                    \end{minipage}
                    \hspace{1mm}    
                    \begin{minipage}{0.485\textwidth}
                            \begin{center}
                            \input{Fig_IXtyOnk.pstricks}
                            \end{center}
                    \end{minipage}

\end{exercice}
