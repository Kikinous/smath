% This is part of Un soupçon de mathématique sans être agressif pour autant
% Copyright (c) 2014
%   Laurent Claessens
% See the file fdl-1.3.txt for copying conditions.

%--------------------------------------------------------------------------------------------------------------------------- 
\subsection*{Activité : mesure astronomique}
%---------------------------------------------------------------------------------------------------------------------------

Des astronomes veulent mesurer la distance approximative entre la Terre et une comète. La technique consiste à mesurer à \( 6\) mois d'écart l'angle formé entre les droites Terre-comète et Terre-Soleil. Pour la facilité (nous ne sommes pas des astronomes professionnels) nous allons supposer que la comète ne se soit pas beaucoup déplacée en \( 6\) mois\footnote{En réalité la technique décrite ici est utilisée pour mesurer des distance avec des étoiles proches, mais les angles ne sont alors pas possible à dessiner avec un rapporteur, étant de l'ordre de \unit{89.999772}{\degree}. Les nombres donnés ici sont choisis pour que l'exercice soit possible, plutôt que pour le réalisme.}.

Lors de la première mesure, l'angle obtenu est \unit{80}{\degree} tandis que la seconde mesure a donné \unit{60}{\degree}. Quelle est la distance entre la comète et le Soleil ?
