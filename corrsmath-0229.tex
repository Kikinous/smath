% This is part of Un soupçon de mathématique sans être agressif pour autant
% Copyright (c) 2013
%   Laurent Claessens
% See the file fdl-1.3.txt for copying conditions.

\begin{corrige}{smath-0229}

\begin{enumerate}
    \item
        Il s'agit de trouver des équations de droites en connaissant leurs graphiques. Commençons par \( f\). La droite passe par les points \( (1;2)\) et \( (4;4)\), donc son coefficient directeur est \( \frac{ 4-1 }{ 4-2 }=\frac{ 3 }{ 2 }\). Son ordonnée à l'origine se lit graphiquement : environ \( 1.25\). Donc l'équation de la droite est
        \begin{equation}
            y=\frac{ 3 }{ 2 }x+1.25
        \end{equation}
        En ce qui concerne \( g\), nous faisons le même jeu et nous trouvons
        \begin{equation}
            y=-x+3.
        \end{equation}
        Attention au signe : le coefficient directeur est bien négatif parce que la droite descend.
    \item
        L'équation de la droite étant \( y=\frac{ 3 }{ 2 }x+1.25\), l'ordonnée du point d'abscisse \( 27\) est 
        \begin{equation}
            \frac{ 3 }{ 2 }\times 27+1.25=41.75.
        \end{equation}
        Le point recherché est donc \( (27;41.75)\).
    \item
    \item
        Les coordonnées \( (x;y)\) du point d'intersection vérifient
        \begin{subequations}
            \begin{numcases}{}
                y=\frac{ 3 }{ 2 }x+1.25\\
                y=-x+3,
            \end{numcases}
        \end{subequations}
        donc
        \begin{equation}
            \frac{ 3 }{ 2 }x+1.25=-x+3,
        \end{equation}
        ce qui a pour solution \( x=0.7\). Le point d'intersection a pour ordonnée \( g(0.7)=2.3\). 

        Notons que ce ne sont pas les valeurs exactes lues sur le graphique; cela est simplement dû au fait que nous avons pris \( 1.25\) comme ordonnée à l'origine de \( f\); cela n'est qu'une approximation.
\end{enumerate}

\end{corrige}
