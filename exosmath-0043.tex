% This is part of Un soupçon de mathématique sans être agressif pour autant
% Copyright (c) 2012
%   Laurent Claessens
% See the file fdl-1.3.txt for copying conditions.

\begin{exercice}\label{exosmath-0043}

    Calculer \info{f(3)}, \info{g(3)}, \info{f(-1)} et \info{g(-1)} pour les fonctions \info{f} et \info{g} données ci-dessous :
    \begin{multicols}{2}
        \lstinputlisting{ex_algo8.py}
        \columnbreak
        \lstinputlisting{ex_algo9.py}
    \end{multicols}
    Écrire ces deux fonctions sous forme algébrique. Sont-elles égales ? Si non, donner une valeur de \( \info{x}\) pour laquelle elles ne produisent pas le même résultat.

    Rappel : en python l'expression \info{x**2} est \( x^2\); c'est équivalent à \info{x*x}.

\corrref{smath-0043}
\end{exercice}
