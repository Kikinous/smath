% This is part of Un soupçon de mathématique sans être agressif pour autant
% Copyright (c) 2012-2014
%   Laurent Claessens
% See the file fdl-1.3.txt for copying conditions.


\usepackage{etex}
\usepackage{ifthen}
%\usepackage{pdfsync}       % This package is obsolete : compile with pdflatex -synctex=1 instead.

\usepackage{latexsym}
\usepackage{amsfonts}
\usepackage{amsmath}
\usepackage{amsthm}
\usepackage{amssymb}
\usepackage{bbm}
\usepackage{mathrsfs}           
\usepackage{mathabx}           % Pour \divides

\usepackage{framed} % pour oframed
\usepackage{wrapfig}

\usepackage{calc}   % Les dépendances de phystricks si on n'utilise que le pdf.
%\usepackage{pstricks,pst-eucl,pstricks-add,calc,pst-math}   % Les dépendances de phystricks. Peut être qu'il faut ajouter catchfile


% Les dépendances de phystricks en mode Tikz
\usepackage{tikz}
\usepackage{calc}
\usetikzlibrary{calc}
\usetikzlibrary{patterns}

\usetikzlibrary{external}
\tikzexternalize
%\newcommand{\tikzsetnextfilename}[1]{}
\newcounter{defHatch}
\newcounter{defPattern}
\setcounter{defHatch}{0}
\setcounter{defPattern}{0}


\usepackage{graphicx}                   % Pour l'inclusion d'image en pfd.

%\newcommand{\EpsOrPdfincludegraphics}[2][]{%
%        \ifpdf
%            \includegraphics[#1]{#2.png}
%        \else
%            \includegraphics[#1]{#2.eps}
%        \fi
%        }

\usepackage{subfigure}

\usepackage{fancyvrb}
\usepackage{stmaryrd}       % Pour le \obslash
\usepackage{xstring}        % Utilisé pour les références vers wikipédia
\usepackage{cases}
\usepackage{lscape}         % pour l'environnement landscape, utilisé dans la correction corr0076.tex
\usepackage{multicol}
\setlength{\columnseprule}{0.5pt}
\usepackage{import}         % Pour le hack qui sert à inclure GeomAnal

% TODO : n'en utiliser qu'un
\usepackage[normalem]{ulem}     % Pour le barré, commande \sout
\usepackage{soul}       % Pour le barré, commande \st

\usepackage[all]{xy}

\let\second\undefined      % le paquet amthabx définit \second
\let\degree\undefined       % le paquet amthabx définit \degree
%\usepackage[cdot,thinqspace,amssymb]{SIunits} 
\usepackage[parse-numbers=false,binary-units=true]{siunitx} 
\newcommand{\unit}[2]{\SI{#1}{#2}}
 % L'option amssymb sert à éviter un conflit avec la commande \square de amssymb. Note qu'elle n'est plus accessible. Si tu en as besoin, faudra RTFM
%ftp://ftp.belnet.be/packages/ctan/macros/latex/contrib/SIunits/SIunits.pdf

\usepackage[nottoc]{tocbibind}
\usepackage[numbers]{natbib}

%%%%%%%%%%%%%%%%%%%%%%%%%%
%
%   Trucs mathématiques
%
%%%%%%%%%%%%%%%%%%%%%%%%

% ENSEMBLES DE NOMBRES
\newcommand{\eA}{\mathbbm{A}}
\newcommand{\eC}{\mathbbm{C}}
\newcommand{\eD}{\mathbbm{D}}
\newcommand{\eE}{\mathbbm{E}}
\newcommand{\eF}{\mathbbm{F}}
\newcommand{\eG}{\mathbbm{G}}
\newcommand{\eH}{\mathbbm{H}}
\newcommand{\eK}{\mathbbm{K}}
\newcommand{\eL}{\mathbbm{L}}
\newcommand{\eM}{\mathbbm{M}}
\newcommand{\eN}{\mathbbm{N}}
\newcommand{\eP}{\mathbbm{P}}
\newcommand{\eQ}{\mathbbm{Q}}
\newcommand{\eR}{\mathbbm{R}}
\newcommand{\eZ}{\mathbbm{Z}}

% ENSEMBLES de fonctions
\newcommand{\aL}{\mathcal{L}}       % Les applications linéaires
\newcommand{\aC}{\mathcal{C}}       % Les fonctions C^1, C^2 etc

% AUTRES
\newcommand{\sdS}{\mathcal{S}}      % L'ensemble des subdivisions d'un intervalle.



\newcommand{\mF}{\mathcal{F}}
\newcommand{\mC}{\mathcal{C}}
\newcommand{\mG}{\mathcal{G}}
\newcommand{\mI}{\mathcal{I}}
\newcommand{\mL}{\mathcal{L}}
\newcommand{\mS}{\mathcal{S}}   % Utilisé pour l'espace des fonctions Schwartz
\newcommand{\mZ}{\mathcal{Z}}


\newcommand{\mtu}{\mathbbm{1}}              % La matrice unité
\newcommand{\caract}{\mathbbm{1}}    % Characteristic function of a set

\DeclareMathOperator{\val}{val}     % valuation d'un polynôme


%\newcommand{\efrac}[2]{\frac{ \displaystyle #1 }{\displaystyle #2 }}
%%%%%%%%%%%%%%%%%%%%%%%%%%
%
%   Numérotations en tout genre
%
%%%%%%%%%%%%%%%%%%%%%%%%

\setcounter{tocdepth}{2}        % Profondeur de la table des matières
\setcounter{secnumdepth}{2}     % Profondeur dans le texte

\renewcommand{\thesubsection}{\thesection.\alph{subsection}}

%%%%%%%%%%%%%%%%%%%%%%%%%%
%
%   Les lignes magiques pour le texte en français.
%
%%%%%%%%%%%%%%%%%%%%%%%%

\usepackage[utf8]{inputenc}
\usepackage[T1]{fontenc}

\usepackage{listingsutf8}
%\lstset{language=python,basicstyle=\footnotesize,tabsize=3,numbers=left,numberstyle=\tiny,frame=single,commentstyle=\ttfamily\color[rgb]{0,0,0.5},stringstyle=\color[rgb]{0,0.5,0},title=\lstname,inputencoding=utf8/latin1}
\lstset{language=python,basicstyle=\footnotesize,tabsize=3,frame=single,commentstyle=\ttfamily\color[rgb]{0,0,0.5},stringstyle=\color[rgb]{0,0.5,0},title=\lstname,inputencoding=utf8/latin1}

\usepackage[fr]{exocorr}
\usepackage{textcomp}
\usepackage{lmodern}
\usepackage[a4paper,margin=2cm]{geometry} 
\usepackage[english,frenchb]{babel}


\usepackage{hyperref}                           %Doit être appelé en dernier.
\hypersetup{
colorlinks=true,
linkcolor=blue,
urlcolor=magenta,     % couleur des url
filecolor=magenta   % couleur des textes qui sont des liens
}

% Il me semble que cette commande doit être définie après l'appel à Babel.
\newcommand{\Ieme}{\up{\lowercase{ième}}\xspace}

%%%%%%%%%%%%%%%%%%%%%%%%%%
%
%   Les théorèmes et choses attenantes
%
%%%%%%%%%%%%%%%%%%%%%%%%


\newcounter{numtho}
\newcounter{numprob}

\makeatletter
\@addtoreset{numtho}{chapter}
%\@addtoreset{CountExercice}{chapter}
\@addtoreset{chapter}{part}
\makeatother

\newlength{\EnvSpace}
\setlength{\EnvSpace}{9pt}      % C'est la distance que je veux mettre avant et après les théorèmes, remarques, etc.

\newtheoremstyle{MyTheorems}%
        {\EnvSpace}{\EnvSpace}%
        {\itshape}%
        {}%
        {\bfseries}{.}%
        {\newline}%
        {}%
\newtheoremstyle{MyExamples}%
        {\EnvSpace}{\EnvSpace}%
        {}%
        {}%
        {\bfseries}{.}%
        {\newline}%
        {}%
\newtheoremstyle{MyRemarks}%
        {\EnvSpace}{\EnvSpace}%
        {}%
        {}%
        {\bfseries}{.}%
        {\newline}%
        {}%

%\theoremstyle{MyExamples}   %\newtheorem{exemple}[numtho]{Exemple}      % Pour unification, ne plus utiliser
%                            \newtheorem{example}[numtho]{Exemple}
\newcounter{CounterExample}
\renewcommand{\theCounterExample}{\thechapter.\arabic{CounterExample}}

% J'ai décidé de ne plus numéroter les choses encadrées. 8 avril 2014
\newenvironment{example}{\vspace{\EnvSpace}\refstepcounter{numtho}\noindent{\bf Exemple}\\\nopagebreak}{\phantom{a}\hfill $\triangle$\vspace{\EnvSpace}}
\newenvironment{Aretenir}{\refstepcounter{numtho}\begin{oframed}\noindent{\bf À retenir}\newline}{\end{oframed}\vspace{\EnvSpace}}
\newenvironment{Aprojeter}{\clearpage\phantom{a}\vfill}{\vfill\newpage}
\newenvironment{definition}{\refstepcounter{numtho}\begin{oframed}\noindent{\bf Définition}\newline}{\end{oframed}\vspace{\EnvSpace}}
\newenvironment{propriete}{\refstepcounter{numtho}\begin{oframed}\noindent{\bf Propriété}\newline}{\end{oframed}\vspace{\EnvSpace}}

\newenvironment{Enmini}{\begin{oframed}\noindent{\bf Mini résumé}\newline}{\end{oframed}\vspace{\EnvSpace}}
% Ce bout de code provient de BrunoJ
% https://brunoj.wordpress.com/2009/10/08/latex-the-framed-minipage/
\newsavebox{\fmbox}
 \newenvironment{fmpage}[1]
 {\begin{lrbox}{\fmbox}\begin{minipage}{#1}}
     {\end{minipage}\end{lrbox}
     \fbox{\usebox{\fmbox}}
 }

\theoremstyle{MyRemarks}    \newtheorem{remark}[numtho]{Remarque}

                \newtheorem{amusement}[numtho]{Amusement}
                \newtheorem{erreur}[numtho]{Error}
                \newtheorem{probleme}[numprob]{\fbox{\bf Problèmes et choses à faire}}


\theoremstyle{MyTheorems}
\newtheorem{lemma}[numtho]{Lemme}
\newtheorem{corollary}[numtho]{Corollaire}
\newtheorem{theorem}[numtho]{Théorème}      
\newtheorem{proposition}[numtho]{Proposition}      


\renewcommand{\thenumtho}{\thechapter.\arabic{numtho}}
% La numérotation des équations change dans les corrigés
\renewcommand{\theequation}{\thechapter.\arabic{equation}}
\renewcommand{\theCountExercice}{\arabic{CountExercice}}       % Ce compteur est défini dans SystemeCorr.sty
\newcommand{\defe}[2]{\textbf{#1}\index{#2}}

\renewcommand{\theenumi}{(\alph{enumi})}
\renewcommand{\theenumii}{(\alph{enumi}\arabic{enumii})}

\renewcommand{\labelenumi}{\theenumi}
\renewcommand{\labelenumii}{\theenumii}

\newcommand{\justification}{ {\small \begin{center} Les réponses doivent être justifiées par un calcul et accompagnées d'un raisonnement.\end{center}}}

    % L'une des deux est avec le nom et l'autre sans.
    \newenvironment{feuilleDS}[1]{\noindent Nom, Prénom : \begin{center}\large #1\\\justification\end{center}\setcounter{CountExercice}{0}  }{\clearpage}
    %\newenvironment{feuilleDS}[1]{\begin{center}\large #1\\\justification\end{center}\setcounter{CountExercice}{0}  }{\clearpage}


    \newenvironment{feuilleExo}[1]{\newpage\begin{center}\large #1\\\justification\end{center}\setcounter{CountExercice}{0}  }{\clearpage}
        
        \newcounter{numactivmentale}
        \setcounter{numactivmentale}{0}
        \newcounter{numExoMental}
        \setcounter{numExoMental}{0}
        \newenvironment{MentalActivity}{\setcounter{numExoMental}{0}\newpage\refstepcounter{numactivmentale}\section{Activité mentale \arabic{numactivmentale}}}{\newpage\hphantom{jj}\vfill\large C'est tout pour aujourd'hui\vfill\newpage}
        \newenvironment{mental}{\refstepcounter{numExoMental}\newpage\begin{center}\fbox{\huge Activité mentale \arabic{numactivmentale}}\\\huge Question \arabic{numExoMental}\end{center}\huge\vfill}{\vfill}


\newcommand{\enteteInterro}[3]{
    \begin{center}
        #1\\
        Intérogation #2, sujet #3
    \end{center}
    Nom, prénom, classe : \ldots\\
    \setcounter{CountExercice}{0}
}


%%%%%%%%%%%%%%%%%%%%%%%%%%
%
%   Les macros qui font des choses
%
%%%%%%%%%%%%%%%%%%%%%%%%

\newcommand{\mA}{\mathcal{A}}
\newcommand{\mO}{\mathcal{O}}
\newcommand{\mR}{\mathcal{R}}
\newcommand{\mT}{\mathcal{T}}
\newcommand{\mU}{\mathcal{U}}

\newcommand{\scal}[2]{ \langle #1,#2\rangle }

\newcommand{\tq}{\text{ tel que }}
\newcommand{\tqs}{\text{ tels que }}
\newcommand{\quext}[1]{ \footnote{\textsf{#1}}  }
\newcommand{\info}[1]{\texttt{#1}}
\newcommand{\vect}[1]{\overrightarrow{#1}}    % Cette macro est codée en dur dans phystricksDefVecteurAXDDGP et dans d'autres

\newcommand{\VarAbs}{\text{Var}_{\text{abs}}}
\newcommand{\VarRel}{\text{Var}_{\text{rel}}}

\newcommand{\normal}{\lhd}
\newcommand{\swS}{\mathscr{S}}          % L'ensemble des fonctions Schwartz

%\newcommand{\defD}{\mathscr{D}}     % Ensemble de définition d'une fonction
\newcommand{\defD}{D}                % Le D avec des croles était impossible à comprendre pour les élèves.

\newcommand{\Borelien}{\mathcal{B}\text{or}}       % Les boréliens
\newcommand{\tribA}{\mathcal{A}}            % Une tribu A
\newcommand{\tribB}{\mathcal{B}}            
\newcommand{\tribF}{\mathcal{F}}            % Une tribu F

\newcommand{\affE}{\mathcal{E}}            % Un espace affine E
\newcommand{\affF}{\mathcal{F}}            
\newcommand{\affG}{\mathcal{G}}            

\newcommand{\statS}{\mathcal{S}}            % Un modèle statistique
\newcommand{\partP}{\mathcal{P}}            % L'ensemble des parties d'un ensemble

\newcommand{\polyP}{\mathcal{P}}            % L'ensemble des polynômes

\newcommand{\dB}{\mathscr{B}}       % la distribution de Bernoulli
\newcommand{\dE}{\mathscr{E}}       % la distribution exponentielle
\newcommand{\dG}{\mathscr{G}}       % la distribution géométrique.
\newcommand{\dM}{\mathscr{M}}       % la distribution multinomiale
\newcommand{\dN}{\mathscr{N}}       % la distribution normale.
\newcommand{\dP}{\mathscr{P}}       % la distribution de Poisson.
\newcommand{\dT}{\mathscr{T}}       % la distribution de Student
\newcommand{\dU}{\mathscr{U}}       % la distribution uniforme

\newcommand{\hL}{\mathscr{L}}       
\newcommand{\cL}{\hL}           % Pour la partie Chafai

\newcommand{\modE}{\mathcal{E}}         % Le E des modules
\newcommand{\modF}{\mathcal{F}}         % Le F des modules
\newcommand{\hH}{\mathscr{H}}           % Le H des espaces de Hilbert

%%%%%%%%%%%%%%%%%%%%%%%%%%
%
%   Bibliographie, index et liste des notations
%
%%%%%%%%%%%%%%%%%%%%%%%%

\usepackage{makeidx}
\usepackage[nottoc]{tocbibind}      % Le paquetage qui fait en sorte que la biblio soit inclue correctement dans la table des matières.
\usepackage[refpage]{nomencl}
\renewcommand{\nomname}{Liste des notations}
%
%   Comment introduire des éléments dans l'index des notations.
%
% La liste des tags à mettre pour bien classer mes notations est :
% T     pour la topologie et théorie des ensembles
%
% La syntaxe est facile, par exemple 
%       $\SL(2,\eR)$\nomenclature[G]{$\SL(2,\eR)$}{Le groupe de matrices deux par deux de déterminant 1.}
%\renewcommand{\nomgroup}[1]{%
%    \ifthenelse{\equal{#1}{A}}{\item[\textbf{Algèbre}]}{}%
%    \ifthenelse{\equal{#1}{G}}{\item[\textbf{Géométrie}]}{}%
%    \ifthenelse{\equal{#1}{R}}{\item[\textbf{Théorie des groupes}]}{}%
%    \ifthenelse{\equal{#1}{P}}{\item[\textbf{Probabilités et statistique}]}{}%
%    \ifthenelse{\equal{#1}{Y}}{\item[\textbf{Analyse}]}{}%
%    \ifthenelse{\equal{#1}{M}}{\item[\textbf{Chaînes de Markov}]}{}%
%}

%%%%%%%%%%%%%%%%%%%%%%%%%%
%
%   DeclareMathOperator
%
%%%%%%%%%%%%%%%%%%%%%%%%

\DeclareMathOperator{\signe}{sgn}
\DeclareMathOperator{\Vol}{Vol}
\DeclareMathOperator{\Int}{Int}     % Intérieur d'un ensemble.
\DeclareMathOperator{\Ind}{Ind}     % l'indice d'un chemin en analyse complexe
\DeclareMathOperator{\Diam}{Diam}   
\DeclareMathOperator{\id}{Id}   
\DeclareMathOperator{\Graph}{Graph} 
\DeclareMathOperator{\pr}{\texttt{proj}}
\DeclareMathOperator{\dom}{dom}

\DeclareMathOperator{\Graphe}{Gr}
\DeclareMathOperator{\Spec}{Spec}   % spectre d'un opérateur
\DeclareMathOperator{\arctg}{arctg}
\DeclareMathOperator{\cotg}{cotg}
\DeclareMathOperator{\cosec}{cosec}
\DeclareMathOperator{\arcsinh}{arcsinh}

\DeclareMathOperator{\GL}{GL}   % le groupe linéaire
\DeclareMathOperator{\PGL}{PGL}   % le groupe projectif
\DeclareMathOperator{\SO}{SO}           
\DeclareMathOperator{\SL}{SL}           
\DeclareMathOperator{\PSL}{PSL}   % Le groupe modulaire SL(2,Z)/Z2
\DeclareMathOperator{\gO}{O}           
\DeclareMathOperator{\SU}{SU}           
\DeclareMathOperator{\gU}{U}           

\DeclareMathOperator{\Reel}{Re}        % La partie réelle d'un nombre complexe

\DeclareMathOperator{\Image}{Image}        % ... avec \Image qui donne l'image d'une fonction ou d'un opérateur.
\DeclareMathOperator{\rang}{rg}   
\DeclareMathOperator{\Kernel}{Ker}
\DeclareMathOperator{\Domaine}{Dom}
\DeclareMathOperator{\Span}{Span}
\DeclareMathOperator{\Hom}{Hom}
\DeclareMathOperator{\End}{End}     % L'ensemble des endomorphismes
\DeclareMathOperator{\tr}{Tr}       % la trace
\DeclareMathOperator{\Majorant}{Maj}
\DeclareMathOperator{\codim}{codim} % pour la codimension.
\DeclareMathOperator{\diam}{diam} % le diamètre d'un ensemble.

\DeclareMathOperator{\Var}{Var}     % Variance d'une variable aléatoire.
\DeclareMathOperator{\Fun}{\texttt{Fun}}     % Ensemble des applications d'un ensemble vers l'autre.
\DeclareMathOperator{\Cov}{Cov}     % la covariance.
\DeclareMathOperator{\gr}{gr}     % le groupe engendré
\DeclareMathOperator{\pgcd}{pgcd}     
\DeclareMathOperator{\ppcm}{ppcm}     
\DeclareMathOperator{\Frob}{Frob}     
\DeclareMathOperator{\Card}{Card}       % Le cardinal d'un ensemble.
\DeclareMathOperator{\Stab}{Stab}       % Le stabilisateur d'un point sous l'action d'un groupe.

\DeclareMathOperator{\Frac}{Frac}       % le corps des fractions d'un anneau
\DeclareMathOperator{\Aff}{Aff}         %  l'espace affine engendré

\newenvironment{subproof}{\begin{description}}{\end{description}}

\newcommand{\telque}{\vert\,}
\newcommand{\donc}{\Rightarrow}

\usepackage{mathtools}
\mathtoolsset{showonlyrefs}
