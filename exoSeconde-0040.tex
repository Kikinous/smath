% This is part of Un soupçon de mathématique sans être agressif pour autant
% Copyright (c) 2012
%   Laurent Claessens
% See the file fdl-1.3.txt for copying conditions.

\begin{exercice}\label{exoSeconde-0040}


Comparaison d'histogrammes.
\begin{enumerate}
\item La structure de la population française est donnée par le
  tableau ci-dessous. Représenter cette distribution par un
  histogramme, en limitant la dernière classe à 110 ans.
  
  \begin{center}
    \begin{tabular}{|l|c|c|c|}
      \hline
      \textbf{Age} & $-$ $15$ ans & $15$--$65$ ans & $+$ de 65 ans \\
      \hline
      \textbf{Proportion} & 20\,\% & 65\,\% & 15\,\% \\
      \hline
    \end{tabular}      
  \end{center}
  
  On prendra comme unité d'aire : 3 carreaux pour 5 \%.

\item La structure de la population du Kenya est donnée par le tableau
  ci-dessous. Représenter cette distribution par un histogramme ayant
  la même échelle que celui du 1.

  \begin{center}
    \begin{tabular}{|l|c|c|c|}
      \hline
      \textbf{Age} & $-$ $15$ ans & $15$--$65$ ans & $+$ de 65 ans \\
      \hline
      \textbf{Proportion} & 52\,\% & 47\,\% & 2\,\% \\
      \hline
    \end{tabular}      
  \end{center}
  
\end{enumerate}



\corrref{Seconde-0040}
\end{exercice}
