% This is part of Un soupçon de mathématique sans être agressif pour autant
% Copyright (c) 2013
%   Laurent Claessens
% See the file fdl-1.3.txt for copying conditions.

\begin{exercice}\label{exosmath-0249}

    Voici les diagrammes en boites des salaires mensuels de cinq entreprises. 

%The result is on figure \ref{LabelFigfigureEHyIMRQ}. % From file figureEHyIMRQ
%\newcommand{\CaptionFigfigureEHyIMRQ}{<+Type your caption here+>}
    \begin{center}
\input{Fig_figureEHyIMRQ.pstricks}
    \end{center}

Pour chacune des affirmations suivantes, indiquer une entreprise (il peut cependant y en avoir plusieurs) pour laquelle l'affirmation est vraie. 
\begin{enumerate}
    \item
        Le plus haut salaire est au moins dix fois supérieur au plus bas.
    \item
        Au moins la moitié des employés ont un salaire ne dépassant pas \( 1500\) euros.
    \item
        Au moins \( 25\%\) des employés gagnent moins de \( 1000\) euros par mois.
    \item
        Au moins trois salariés sur quatre ont un salaire dépassant \( 1500\) euros.
    \item
        Le patron (le plus haut salaire) gagne trois fois plus qu'au moins la moitié des employés.
\end{enumerate}

Donner une répartition des salaires qui correspondrait à l'entreprise \( B\). Vous avez le choix du nombre de salariés.

\corrref{smath-0249}
\end{exercice}
