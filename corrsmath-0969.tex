% This is part of Un soupçon de mathématique sans être agressif pour autant
% Copyright (c) 2014
%   Laurent Claessens
% See the file fdl-1.3.txt for copying conditions.

\begin{corrige}{smath-0969}

    \begin{enumerate}
        \item
            La formule de distribution (développement) donne
            \begin{equation}
                30\times (10-x)=30\times x-30\times x=300-30x.
            \end{equation}
            C'est donc la possibilité \ref{IXCEooMPdVln} qui est correcte.
        \item
            Le développement donne
            \begin{equation}
                3\times (x+2)=3\times x+3\times 2=3x+6.
            \end{equation}
        \item
            Il y a plusieurs choses à faire. Une première est de remarquer que les deux termes sont multiples de \( 3\) et d'écrire
            \begin{equation}
                6\times a-3\times a^2=3\times (2a-a^2).
            \end{equation}
            Une autre possibilité est de remarquer que les deux termes sont multiples de \( a\) et d'écrire
            \begin{equation}
                6\times a-3\times a^2=a\times (6-3a).
            \end{equation}
            Le mieux est de faire les deux en même temps en mettant \( 3a\) en facteur :
            \begin{equation}
                6\times a-3\times a^2=3a(2-a).
            \end{equation}
    \end{enumerate}

\end{corrige}
