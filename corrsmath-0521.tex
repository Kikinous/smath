% This is part of Un soupçon de mathématique sans être agressif pour autant
% Copyright (c) 2013
%   Laurent Claessens
% See the file fdl-1.3.txt for copying conditions.

\begin{corrige}{smath-0521}

    Faire un dessin est fortement conseillé, bien qu'il soit possible de répondre sans.
    
    \begin{enumerate}
        \item
            
    Nous appelons \( M\) le milieu du segment \( [KR]\), qui sera égalent le milieu du parallélogramme. Ses coordonnées se calculent en utilisant la formule du milieu :
    \begin{equation}
        M=\left( \frac{ x_K+x_R }{2};\frac{ y_K+y_R }{2} \right)=\left( \frac{ -3+(-2) }{2};\frac{ 2+7 }{2} \right)=\left( -\frac{ 5 }{2};\frac{ 9 }{2} \right)=(-2.5;4.5).
    \end{equation}

        \item

            Le point \( M\) doit être le milieu de la diagonale \( [LQ]\); nous cherchons donc le point \( Q\) de telle sorte que \( M\) soit milieu de \( [LQ]\). Il s'agit d'utiliser la formule du milieu «à l'envers». En ce qui concerne les abscisses :
            \begin{equation}
                x_M=\frac{ x_Q+x_R }{ 2 },
            \end{equation}
            donc
            \begin{equation}
                -2.5=\frac{ x_Q+0 }{2},
            \end{equation}
            ce qui donne \( x_Q=-5\). Pour les ordonnées,
            \begin{equation}
                y_M=\frac{ y_Q+y_R }{ 2 },
            \end{equation}
            donc
            \begin{equation}
                4.5=\frac{ y_Q+4 }{2},
            \end{equation}
            ce qui donne \( 9=y_Q+4\) et donc \( y_Q=5\). Au final le point recherché est
            \begin{equation}
                Q=(-5;5).
            \end{equation}

        \item

            Pour qu'un parallélogramme soit un rectangle, il suffit de vérifier qu'il ait un angle droit. Vérifions par exemple si l'angle \( \hat L\) est droit en utilisant la réciproque du théorème de Pythagore dans le triangle \( KLR\). D'abord nous calculons les longueurs des trois côtés :
            \begin{subequations}
                \begin{align}
                    KL=\sqrt{(-3)^2+2^2}=\sqrt{13}\\
                    KR=\sqrt{1^2+5^2}=\sqrt{26}\\
                    LR=\sqrt{(-2)^2+3^2}=\sqrt{13}.
                \end{align}
            \end{subequations}
            Nous avons bien \( KR^2=KL^2+LR^2\), ce qui prouve que l'angle \( \hat L\) est droit. Le quadrilatère \( KLRQ\) est donc bien un rectangle.

            Il est de plus un carré parce qu'il a deux cotés adjacents de même longueurs : \( KL=LR\).

    \end{enumerate}
    
\end{corrige}
