% This is part of Un soupçon de mathématique sans être agressif pour autant
% Copyright (c) 2013
%   Laurent Claessens
% See the file fdl-1.3.txt for copying conditions.

\begin{exercice}\label{exosmath-0416}

    Soit la fonction \( f(x)=\frac{ \displaystyle  e^{-x}-2 e^{x} }{ \displaystyle 1+3 e^{2x} }\).
    \begin{enumerate}
        \item
            Laquelle des trois propositions suivantes est correcte ?
            \begin{enumerate}
                \item
                    \( f(x)=\frac{ 1-2 e^{2x} }{  e^{x} +3 e^{3x}  }\)
                \item
                    \( f(x)=\frac{  e^{-x}-2 }{ 1+3 e^{x} }\).
                \item
                    \( f(x)=\frac{ 1-2 e^{x} }{  e^{x}+3 }\)
            \end{enumerate}
        \item
            Donner une valeur de \( a\) pour laquelle \( f(a)<0\).
    \end{enumerate}

\corrref{smath-0416}
\end{exercice}
