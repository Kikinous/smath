% This is part of Un soupçon de mathématique sans être agressif pour autant
% Copyright (c) 2014
%   Laurent Claessens
% See the file fdl-1.3.txt for copying conditions.

\begin{corrige}{smath-0897}

    En ce qui concerne les phrases :
    \begin{enumerate}
        \item
            Si il pleut alors il y a des nuages.
        \item
            Si une quadrilatère est un parallélogramme, alors il a deux diagonales de même longueurs.
    \end{enumerate}
    Pour les questions :
    \begin{enumerate}
        \item
            Non, on ne peut pas affirmer qu'il ait plu le \( 5\) avril : il peut y avoir des nuages sans pluie.
        \item
            Il est possible d'avoir un quadrilatère dont les diagonales ont les mêmes longueurs sans qu'il soit un parallélogramme. Par exemple ceci :
\begin{center}
   \input{Fig_UPGooXSYGqo.pstricks}
\end{center}
Les segments \( [AC]\) et \( [BD]\) sont de même longueurs, mais le quadrilatère \( ABCD\) n'est pas un parallélogramme.

        Attention : un rectangle est un parallélogramme. Répondre «non parce que cela pourrait être un rectangle» est faux.

    \end{enumerate}

\end{corrige}
