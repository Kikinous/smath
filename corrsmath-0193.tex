% This is part of Un soupçon de mathématique sans être agressif pour autant
% Copyright (c) 2013
%   Laurent Claessens
% See the file fdl-1.3.txt for copying conditions.

\begin{corrige}{smath-0193}

    Les données de l'énoncé permettent de remplir le tableau suivant :
    \begin{equation*}
        \begin{array}[]{|c|c|c|c|}
            \hline
            &\text{internet}&\text{pas internet}&\text{total}\\
            \hline
            \text{DVD}&35&&65\\
            \hline
            \text{pas DVD}&&&\\
            \hline
            \text{total}&48&&100\\
            \hline
        \end{array}
    \end{equation*}
    Ensuite, il suffit de compter pour remplir les cases restantes :
    \begin{equation*}
        \begin{array}[]{|c|c|c|c|}
            \hline
            &\text{internet}&\text{pas internet}&\text{total}\\
            \hline
            \text{DVD}&35&30&65\\
            \hline
            \text{pas DVD}&13&22&35\\
            \hline
            \text{total}&48&52&100\\
            \hline
        \end{array}
    \end{equation*}
    Il y a donc \( 22\%\) des ménages qui n'ont ni lecteur DVD ni internet.

\end{corrige}
