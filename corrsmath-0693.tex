% This is part of Un soupçon de mathématique sans être agressif pour autant
% Copyright (c) 2014
%   Laurent Claessens
% See the file fdl-1.3.txt for copying conditions.

\begin{corrige}{smath-0693}

    \begin{enumerate}
        \item
            Le dénominateur commun est \( a\) :
            \begin{equation}
                2-\frac{ 2 }{ a }=\frac{ 2a }{ a }-\frac{ 2 }{ a }=\frac{ 2a-2 }{ a }.
            \end{equation}
        \item
            Les plus simples sont les droites \( y=10\) et \( x=4\) :  les droites horizontales et verticales se croisant au point \( A\). Sinon il y a aussi les droites \( y=2x+2\), \( y=x+6\) et de nombreuses autres possibilités.
        \item
            Il suffit de vérifier si \( f(-4)=15\). Le calcul est :
            \begin{equation}
                f(-4)=(-4)^2-1=16-1=15.
            \end{equation}
            Donc \( B\) est bien sur le graphe de \( f\).
        \item
            Il faut remplacer \( x\) par \( -3\) et voir si l'inéquation est satisfaite ou non. Lorsque \( x=-3\) nous avons \( x^2+1=(-3)^2+1=10\), qui est plus grand que \( -1\). Donc \( x=-3\) est bien solution.
        \item
            Technique usuelle : mettre tous les \( x\) d'un côté et tous les termes sans \( x\) de l'autre :
            \begin{equation}
                4x+x=10-3,
            \end{equation}
            donc \( 5x=7\) et au final \( x=\frac{ 7 }{ 5 }\).
    \end{enumerate}

\end{corrige}
