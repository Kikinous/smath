% This is part of Un soupçon de mathématique sans être agressif pour autant
% Copyright (c) 2012-2013
%   Laurent Claessens
% See the file fdl-1.3.txt for copying conditions.

\begin{exercice}\label{exosmath-0229}

%The result is on figure \ref{LabelFigKKEdcAR}. % From file KKEdcAR
%\newcommand{\CaptionFigKKEdcAR}{<+Type your caption here+>}
\begin{wrapfigure}{r}{7.0cm}
            \vspace{-3cm}        % Il était à -3cm pour le cours complet.
                \centering
                    \input{Fig_KKEdcAR.pstricks}
                \end{wrapfigure}

        À propos de la figure ci-contre, 
        \begin{enumerate}
            \item
                Déterminer graphiquement les équations des deux droites.
            \item
                À partir des équations de droites trouvées, calculer les coordonnées du point d'abscisse \( 27\) se trouvant sur la droite \( f\).
            \item
                À partir des équations de droites trouvées, calculer les coordonnées du point d'intersection par le calcul. (c'est à dire en résolvant un système d'équation)
        \end{enumerate}

\corrref{smath-0229}
\end{exercice}
