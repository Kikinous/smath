% This is part of Un soupçon de mathématique sans être agressif pour autant
% Copyright (c) 2012-2013
%   Laurent Claessens
% See the file fdl-1.3.txt for copying conditions.

\begin{exercice}\label{exosmath-0229}

%The result is on figure \ref{LabelFigKKEdcAR}. % From file KKEdcAR
%\newcommand{\CaptionFigKKEdcAR}{<+Type your caption here+>}
\begin{wrapfigure}{r}{7.0cm}
            \vspace{-3cm}        % à adapter.
                \centering
                    \input{Fig_KKEdcAR.pstricks}
                \end{wrapfigure}

        À propos de la figure ci-contre, 
        \begin{enumerate}
            \item
                Donner les équations des deux droites.
            \item
                Calculer les coordonnées du point d'abscisse \( 13\) se trouvant sur la droite \( f\).
            \item
                Montrer le point d'intersection sur le graphique.
            \item
                Retrouver le point d'intersection par le calcul. (c'est à dire en résolvant un système d'équation)
        \end{enumerate}

\corrref{smath-0229}
\end{exercice}
