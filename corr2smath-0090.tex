% This is part of Un soupçon de mathématique sans être agressif pour autant
% Copyright (c) 2015
%   Laurent Claessens
% See the file fdl-1.3.txt for copying conditions.

\begin{corrige}{2smath-0090}

    \begin{enumerate}
        \item
    Étant donné que \( (BC)\) est parallèle à \( (KL)\) et que la longueur du segment \( [KL]\) est la moitié de celle de \( [BC]\), la droite \( (KL)\) est une droite des milieux. Elle doit donc couper \( [AB]\) et \( [CA]\) en leurs milieux. Par conséquent \( CL=LA=5\).
\item
    Si \( BK=2\) alors \( BA=4\) et le triangle \( ABC\) aurait pour mesures \( 3\), \( 10\) et \( 4\). Cela n'est pas possible à cause de l'inégalité triangulaire (vue en cinquième).
    \end{enumerate}

\end{corrige}
