% This is part of Un soupçon de mathématique sans être agressif pour autant
% Copyright (c) 2013
%   Laurent Claessens
% See the file fdl-1.3.txt for copying conditions.

\begin{exercice}[\cite{FGMWzeK}]\label{exosmath-0322}

    \begin{minipage}{0.485\textwidth}
    Compléter le tableau ci-contre qui donne une correspondance entre les inégalités et les intervalles.

    Conseil : aidez-vous de dessins.
    \end{minipage}
    \begin{minipage}{0.485\textwidth}
    \begin{equation*}
        \begin{array}[]{|c||c|}
            \hline
            \text{Inégalité}&\text{Intervalle}\\
            \hline\hline
            x\leq 3&\\
            \hline
            x>-1&\\
            \hline
            &\mathopen[ 7 ; +\infty [\\
            \hline
            &\mathopen] -\infty , -2 \mathclose[\\
            \hline
            1\leq x<8&\\
            \hline
            x<-5\text{ ou }x\geq 3&\\
            \hline
            &\mathopen] -2 , 6 \mathclose]\\
            \hline
            &\mathopen] -\infty , 1 \mathclose]\\
            \hline
        \end{array}
    \end{equation*}
    \end{minipage}

\corrref{smath-0322}
\end{exercice}
