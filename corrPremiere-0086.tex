% This is part of Un soupçon de mathématique sans être agressif pour autant
% Copyright (c) 2012
%   Laurent Claessens
% See the file fdl-1.3.txt for copying conditions.

\begin{corrige}{Premiere-0086}

    \begin{enumerate}
        \item
            La probabilité de succès est \( 4/6=2/3\). Le nombre d'essais est \( 3\).
        \item
            L'arbre est un peu trop long à écrire ici, mais ressemble à celui de l'exemple \ref{ExVRZMlyN}.
        \item
            \begin{enumerate}
                \item
                    Il y a une seule façon d'avoir \( X=0\), c'est de rater les trois essais : c'est la case EEE. La probabilité est de \( (2/6)^3=1/27\).
                \item
                    Il y a trois cases correspondantes à un seul succès : \( SEE\), \( ESE\) et \( EES\). Chacune a une probabilité \( (2/6)^2\times (4/6)\). Au total \( P(X=1)=2/9\)
                \item
                    Il y a trois cases correspondantes à deux succès : \( ESS\), \( SES\) et \( SSE\). Chacune a une probabilité \( (4/6)^2\times (2/6)\). Au total, \( P(X=2)=4/9\).
                \item
                    Il y a une seule case avec \( X=3\) c'est \( EEE\). Donc \( P(X=3)=(4/6)^38/27\).
                \item
                    Il n'y a évidemment pas moyen de réussis quatre ou cinq fois en seulement 3 essais.
            \end{enumerate}
            
    \end{enumerate}

\end{corrige}
