% This is part of Un soupçon de mathématique sans être agressif pour autant
% Copyright (c) 2014
%   Laurent Claessens
% See the file fdl-1.3.txt for copying conditions.

\begin{exercice}[\ldots/5]\label{exosmath-0598}

    Type «calcul mental». Les réponses doivent être brièvement justifiées.
    \begin{enumerate}
        \item
            Développer \( (x-4)^2\).
        \item
            Simplifier et dire par combien vous avez simplifié :
            \begin{equation*}
                \frac{ 12x+50 }{ 2x }.
            \end{equation*}
        \item
            Si \( f(x)=-3x^2+7\), combien vaut \( f(-2)\) ?
        \item
            Donner une valeur exacte de la longueur de la diagonale d'un carré dont le côté mesure \unit{6}{\kilo\meter}.
        \item
            Les fonctions \( f(x)=7x-150\) et \( g(x)=-6x-150\) ont-elles des droites représentatives parallèles ?
    \end{enumerate}

\corrref{smath-0598}
\end{exercice}
