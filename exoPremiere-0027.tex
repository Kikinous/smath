% This is part of Un soupçon de mathématique sans être agressif pour autant
% Copyright (c) 2012
%   Laurent Claessens
% See the file fdl-1.3.txt for copying conditions.

\begin{exercice}\label{exoPremiere-0027}

    Écrire une fonction de deux variables \( a\) et \( b\) qui retourne la liste des carrés des entiers situés entre \( a\) et \( b\). Par exemple si \info{f} est la fonction, alors \info{f(4;8)} doit retourner la liste \info{[16,25,36,49,64]}.

    \begin{enumerate}
        \item
            Dans un premier temps, supposer que \( a<b\), et que \( a\) et \( b\) sont entiers.
        \item
            Traiter le cas où \( b<a\). Demander \info{f(8;4)} doit retourner la même liste \info{[16,25,36,19,64]}.
        \item
            Les puristes vérifierons que les entrées de la fonction sont des entiers, et choisiront ce qu'il convient de faire si ce n'est pas le cas. (lever une exception \info{ValueError}, prendre des arrondis, \ldots).
    \end{enumerate}

\corrref{Premiere-0027}
\end{exercice}
