% This is part of Un soupçon de mathématique sans être agressif pour autant
% Copyright (c) 2012
%   Laurent Claessens
% See the file fdl-1.3.txt for copying conditions.

\begin{exercice}\label{exoPremiere-0027}

    Écrire une fonction de deux variables \( a\) et \( b\) qui retourne la liste des carrés des entiers situés entre \( a\) et \( b\).

    Attention : nous ne supposons pas que \( a<b\) : la fonction doit trouver elle-même le plus petit. Par exemple si \info{f} est la fonction, alors \info{f(4,8)} doit retourner la liste \info{[16,25,36,49,64]}. Demander \info{f(8,4)} doit retourner la même chose.

    \begin{enumerate}
        \item
            Dans un premier temps, vous pouvez supposer que les nombres \( a\) et \( b\) sont entiers.
        \item
            Les puristes vérifierons que les entrées de la fonction sont des entiers, et choisiront ce qu'il convient de faire si ce n'est pas le cas. (lever une exception \info{ValueError}, prendre des arrondis, \ldots).
    \end{enumerate}

\corrref{Premiere-0027}
\end{exercice}
