% This is part of Un soupçon de mathématique sans être agressif pour autant
% Copyright (c) 2014
%   Laurent Claessens
% See the file fdl-1.3.txt for copying conditions.

%TODO : citer sésamath de seconde

\begin{exercice}\label{exosmath-0716}

    \begin{multicols}{2}

Dans un lycée de $1470$ élèves, \( 350\) élèves ont été vaccinés contre la grippe au début de l'hiver. \( 10\%\) des élèves ont contracté la grippe pendant l'épidémie annuelle, dont \( 4\%\) des élèves vaccinés. Voici un tableau résumant la situation :

\begin{center}
    \begin{tabular}[]{|c|c|c|c|}
        \hline
        &malade&pas malade&total\\
        \hline
        vaccinés&\( 14\)&&\( 350\)\\
        \hline
        non vacciné&&&\\
        \hline
        total&\( 147\)&&\( 1470\)\\
        \hline
    \end{tabular}
\end{center}

    Nous choisissons au hasard un des élèves du lycée et nous considérons les événements 
    \begin{itemize}
        \item \( V\) : «il a été vacciné»; 
        \item \( G\) : «il a eu la grippe». 
    \end{itemize}
        

\begin{enumerate}
    \item
        Expliquer comment le \( 14\) du tableau a été calculé, et compléter le tableau.
    \item
        Calculer la probabilité des événements \( V\cap G\) et \( V\cup G\).
    \item
        Décrire par une phrase l'événement \( \bar V\).
    \item
        Nous choisissons un élève au hasard parmi ceux qui ont été malades; quelle est la probabilité qu'il ait été vacciné ?
    \item
        Nous choisissons un élève au hasard parmi ceux qui n'ont pas été vaccinés. Quelle est la probabilité qu'il ait eu la grippe ?
    \item
        Expliquer pourquoi le vaccin est efficace.
\end{enumerate}

    \end{multicols}

\corrref{smath-0716}
\end{exercice}
