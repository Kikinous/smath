% This is part of Un soupçon de mathématique sans être agressif pour autant
% Copyright (c) 2015
%   Laurent Claessens
% See the file fdl-1.3.txt for copying conditions.

\begin{exercice}[\cite{HOWZooBuuoms}]\label{exo2smath-0181}

    Ceci est une pyramide à base rectangulaire.

\begin{wrapfigure}{r}{3.0cm}
   \vspace{-0.5cm}        % à adapter.
   \centering
   \input{Fig_HTGDooAVOWgZ.pstricks}
\end{wrapfigure}
À partir du dessin,
\begin{enumerate}
    \item
        nommer la base et les faces latérales,
    \item
        sachant que \( AB=8\), \( BC=6\) et \( SC=13\), calculer les longueurs de la hauteur et des quatre arrêtes latérales.
\end{enumerate}

Conseil : dessiner la hauteur et reporter les mesures données sur le dessin.


\corrref{2smath-0181}
\end{exercice}
