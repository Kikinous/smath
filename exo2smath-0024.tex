% This is part of Un soupçon de mathématique sans être agressif pour autant
% Copyright (c) 2014
%   Laurent Claessens
% See the file fdl-1.3.txt for copying conditions.

\begin{exercice}\label{exo2smath-0024}

Dans chaque cas, dire lequel des deux nombres est le plus grand.
\begin{multicols}{2}
    \begin{enumerate}
        \item
            \( 3\) et \( 5\)
        \item
            \( -3\) et \( -3\)
        \item
            \( \dfrac{ 1 }{ -5 }\) et \( \dfrac{ 1 }{ -7 }\)
        \item
            \( \dfrac{ 3 }{ 8.2 }\) et \( \dfrac{ 3 }{ -8.2 }\)
        \item
            \( -\dfrac{ 5.23 }{ 14.5 }\) et \( \dfrac{ -5.23 }{ 14.6 }\)
        \item
            \( \dfrac{ 7.5 }{ 0.23 }\) et \( \dfrac{ 75 }{ 2.4 }\).
        \item
            \( \dfrac{ -7.5 }{ 0.23 }\) et \( \dfrac{ 75 }{ -2.4 }\).
    \end{enumerate}
\end{multicols}

\corrref{2smath-0024}
\end{exercice}
