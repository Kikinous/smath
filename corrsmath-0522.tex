% This is part of Un soupçon de mathématique sans être agressif pour autant
% Copyright (c) 2013
%   Laurent Claessens
% See the file fdl-1.3.txt for copying conditions.

\begin{corrige}{smath-0522}

    \begin{enumerate}
        \item
            Il faut vérifier si l'inéquation donnée est correcte en remplaçant \( x\) par \( -5\). Si \( x=-5\), alors \( -x=5\) et l'inéquation proposée devient \( 5\leq 2\), ce qui est faux. Donc \( x=-5\) n'est pas solution de \( -x\leq 2\).

        \item
            Nous avons \( g(2)=12\), donc le graphe de la fonction \( g\) ne passe pas par le point \( A(2;0)\). Vu que \( g(1)=7\) il passe bien par le point \( (1;7)\). La réponse est donc faux : il ne passe pas par les deux.
        \item
            Il suffit de vérifier la réciproque de Pythagore en sachant que le plus long côté serait l'hypoténuse. Est-ce que \( 7^2=5^2+6^2\) ? Non. Donc il n'est pas rectangle.
        \item
            Il faut récrire sous une forme qui nous permet mieux de voir la fonction affine :
            \begin{equation}
                k(x)=\frac{ 1-3x }{ 2 }=\frac{ 1 }{2}-\frac{ 3 }{2}x.
            \end{equation}
            Le coefficient directeur de cette fonction vaut \( -\frac{ 3 }{2}\), qui est négatif. Donc la fonction est décroissante et pas croissant\item e. 
            \item
                Oui, elles sont parallèles parce que leurs coefficient directeur sont égaux (ils valent \( 25\)).
    \end{enumerate}
    <++>

\end{corrige}
