% This is part of Un soupçon de mathématique sans être agressif pour autant
% Copyright (c) 2012
%   Laurent Claessens
% See the file fdl-1.3.txt for copying conditions.

\begin{exercice}\label{exoPremiere-0030}


\begin{enumerate}
    \item
        Que vaut \( (x+5)(x-2)\) ?
    \item
        Compléter le tableau de signe
        \begin{center}
            \begin{tabular}[h]{|c||c|c|c|c|c|}
                \hline
                    \( x\)&\ldots&-5&\ldots&2&\( \ldots\)\\
                    \hline
                    \( x+5\)&&&&& \\
                    \hline
                    \( x-2\)&&&&&\\
                    \hline\hline
                    \(x^2+3x-10\)&&&&&\\
                    \hline
            \end{tabular}
        \end{center}
        Note : la dernière ligne se remplit sans calculs, juste en regardant les deux premières.
    \item
        À partir de ce tableau, donner les solutions des équations et inéquations suivantes :
        \begin{subequations}
            \begin{align}
                x^2+3x-10&=0\\
                x^2+3x-10&\leq0\\
                x^2+3x-10&\geq0
            \end{align}
        \end{subequations}
    \item
        Lequel des graphiques de la figure \ref{LabelFigSecond} correspond à la fonction \( y=f(x)=x^2+3x-10\) ?
\newcommand{\CaptionFigSecond}{Laquelle de ces trois courbes est \( x^2+3x-10\) ?}
\input{Fig_Second.pstricks}
\end{enumerate}

\corrref{Premiere-0030}
\end{exercice}
