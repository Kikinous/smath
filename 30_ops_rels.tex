% This is part of Un soupçon de mathématique sans être agressif pour autant
% Copyright (c) 2015
%   Laurent Claessens
% See the file fdl-1.3.txt for copying conditions.

% This is part of Un soupçon de mathématique sans être agressif pour autant
% Copyright (c) 2015
%   Laurent Claessens
% See the file fdl-1.3.txt for copying conditions.


%--------------------------------------------------------------------------------------------------------------------------- 
\subsection*{Activité : jeu de cartes}
%---------------------------------------------------------------------------------------------------------------------------

Découper des cartes numérotées de \( -6\) à \( +8\).

\begin{enumerate}
    \item
        Calculer la somme de toutes les cartes.
    \item
        Que devient cette somme si on retire la carte \( +5\) ?
    \item
        On remet la carte \( +5\) et on retire la carte \( -3\). Quelle est la nouvelle somme ?
    \item
        On reprend le paquet de départ, et on remplace la carte \( +4\) par une carte \( -7\). Quelle est la nouvelle somme ?
\end{enumerate}

De : \cite{VJNYooBKjymp}


%+++++++++++++++++++++++++++++++++++++++++++++++++++++++++++++++++++++++++++++++++++++++++++++++++++++++++++++++++++++++++++ 
\section{Addition}
%+++++++++++++++++++++++++++++++++++++++++++++++++++++++++++++++++++++++++++++++++++++++++++++++++++++++++++++++++++++++++++

\begin{definition}
    Pour \defe{additionner}{additionner!nombres relatifs} deux nombres relatifs de même signe,
    \begin{itemize}
        \item on prend le signe commun,
        \item on additionner les distances à zéro.
    \end{itemize}
    Pour \defe{additionner}{additionner!nombres relatifs} deux nombres relatifs de signes contraires,
    \begin{itemize}
        \item on prend le signe de celui qui a la plus grande distance à zéro,
        \item on soustrait les distances à zéro.
    \end{itemize}
\end{definition}

\begin{example}
    Pour calculer \( (-3)+(-7)\) :
    \begin{itemize}
        \item le signe est \( -\) (nombre négatif)
        \item la distance à zéro est \( 7+3=10\)
        \item donc \( (-3)+(-7)=-10\).
    \end{itemize}
    Pour calculer \( (+5)+(-9)\)
    \begin{itemize}
        \item le signe est \( -\) (c'est \( -9\) la plus grande distance à zéro)
        \item on soustrait : \( 9-5=4\)
        \item donc \( (+5)+(-9)=-4\).
    \end{itemize}
\end{example}

%+++++++++++++++++++++++++++++++++++++++++++++++++++++++++++++++++++++++++++++++++++++++++++++++++++++++++++++++++++++++++++ 
\section{Soustraction}
%+++++++++++++++++++++++++++++++++++++++++++++++++++++++++++++++++++++++++++++++++++++++++++++++++++++++++++++++++++++++++++

Question à la volée : combien fait \( (+5)-(-4)  \) ?

Une façon de faire : 
\begin{itemize}
    \item 
créer des cartes dont la somme est \( 5\) et contenant une carte \( -4\).
\item
    retirer la carte \( -4\)
\item 
    Calculer la nouvelle somme.
\end{itemize}
Exemple : le jeu \( -4\), \( 4\), \( 5\). Ou encore le jeu \( -4\), \( 6\), \( 3\).


\begin{definition}
    Pour \defe{soustraire}{soustraction!nombres relatifs} un nombre relatif, on additionne l'opposé.
\end{definition}

\begin{example}
    \( 3-5=3+(-5)=-2\)
\end{example}

Bilan : toutes les soustractions sont maintenant possibles.
