% This is part of Un soupçon de mathématique sans être agressif pour autant
% Copyright (c) 2012
%   Laurent Claessens
% See the file fdl-1.3.txt for copying conditions.

\begin{exercice}\label{exosmath-0044}

    Soient les fonctions \( f(x)=(x-3)^2-16\), \( g(x)=x^2-6x-7\) et \( h(x)=(x-7)(x+1)\).
    \begin{enumerate}
        \item
            Montrer que ces trois fonctions sont égales pour tout \( x\).
        \item
            Calculer \( f(0)\), \( f(-1)\), \( f(\sqrt{5})\) et \( f(3)\). Pour ce faire, vous pouvez utiliser au choix les trois formes données.
    \end{enumerate}

\corrref{smath-0044}
\end{exercice}
