% This is part of Un soupçon de mathématique sans être agressif pour autant
% Copyright (c) 2012
%   Laurent Claessens
% See the file fdl-1.3.txt for copying conditions.

\begin{corrige}{smath-0124}

    \begin{multicols}{2}
        Le dessin ci-contre n'est volontairement pas à l'échelle. Le quadrilatère sera un losange si les diagonales se coupent à angle droit. Autrement dit si le triangle \( AOB\) est rectangle en \( O\). Le théorème de Pythagore permet de vérifier cela : nous aurons un triangle rectangle si \( AO^2+OB^2=AB^2\). En remplaçant nous trouvons
        \begin{equation}
            AO^2+OB^2=64+16=80,
        \end{equation}
        tandis que \( AB^2=16\times 5=80\). Le compte y est donc le quadrilatère est un losange.

        Notons que le dessin n'est pas du tout à l'échelle.


        \columnbreak


        %The result is on figure \ref{LabelFigfigureFNkqWFE}. % From file figureFNkqWFE
        %\newcommand{\CaptionFigfigureFNkqWFE}{<+Type your caption here+>}
        \begin{center}
        \input{Fig_figureFNkqWFE.pstricks}
        \end{center}

    \end{multicols}
    

\end{corrige}
