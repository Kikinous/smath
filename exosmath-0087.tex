% This is part of Un soupçon de mathématique sans être agressif pour autant
% Copyright (c) 2012
%   Laurent Claessens
% See the file fdl-1.3.txt for copying conditions.

\begin{exercice}\label{exosmath-0087}

    \begin{multicols}{2}
Les moniteurs d'une colonie de vacances désirent délimiter une surface de baignade rectangulaire la plus grande possible, en installant une ligne de bouées de $80$ mètres de long. On modélise la situation de la façon suivante : le bord de la plage est supposé rectiligne, et $ABCD$ est un rectangle. On note $x$ la longueur $AB$ en mètres.

Nous voudrions déterminer pour quel \( x\) l'aire de baignade est maximale.

    \columnbreak

%The result is on figure \ref{LabelFigfigureERITfSy}. % From file figureERITfSy
%\newcommand{\CaptionFigfigureERITfSy}{<+Type your caption here+>}
\input{Fig_figureERITfSy.pstricks}

    \end{multicols}

\begin{enumerate}
\item Exprimer la longueur de chacun des côtés du rectangle $ABCD$ en
  fonction de $x$ et les reporter sur la figure.
\item En déduire l'expression $A(x)$ de l'aire du rectangle
  $ABCD$ en $m^2$ en fonction de $x$.
\item Quelle est la nature de la fonction $x\mapsto\mathscr{A}(x)$ ?
\item Tracer à main levée la courbe représentative de la fonction $A$.
\item Donner la valeur de $x$ pour laquelle cette aire est maximale. Quelle est la valeur de ce maximum ? 
\end{enumerate}

\corrref{smath-0087}
\end{exercice}
