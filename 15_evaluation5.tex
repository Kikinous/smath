% This is part of Un soupçon de mathématique sans être agressif pour autant
% Copyright (c) 2014-2015
%   Laurent Claessens
% See the file fdl-1.3.txt for copying conditions.

% LES ÉVALUATIONS DE CINQUIÈME

\begin{center}
    ceci sont les énoncés et les corrections des questions posées en évaluation. attention : certaines questions ont pu être modifiées par rapport à l'énoncé exact donné aux élèves; cela pour clarifier certaines questions qui n'ont pas été bien comprises.
\end{center}

%\begin{minipage}{0.485\textwidth}
    \enteteInterro{Le 19 septembre 2014}{1}{A}

    Calculer :
    \begin{enumerate}
        \item
            \( 8-3\times 2=\ldots\)
        \item
            \( \dfrac{ 16+4 }{ 5 }=\ldots\)
        \item
            \( \ldots \times 4+12=40\)
        \item
            \( (8-3)\times 2=\ldots\)
        \item
            \(  4\times 3-4=\ldots \)
    \end{enumerate}

\end{minipage}
\begin{minipage}{0.485\textwidth}
    \enteteInterro{Le 19 septembre 2014}{1}{B}

    Calculer :
    \begin{enumerate}
        \item
            \( 17-4\times 4=\ldots\)
        \item
            \( \dfrac{ 12+6 }{ 2 }=\ldots\)
        \item
            \( \ldots \times 6+5=41\)
        \item
            \( (10-6)\times 9=\ldots\)
        \item
            \(  9\times 3-7=\ldots \)
    \end{enumerate}

\end{minipage}
      % c'est ce fichier qui demandait Exosmath-0813 et exosmath-0814

%+++++++++++++++++++++++++++++++++++++++++++++++++++++++++++++++++++++++++++++++++++++++++++++++++++++++++++++++++++++++++++ 
\section{29 septembre 2014}
%+++++++++++++++++++++++++++++++++++++++++++++++++++++++++++++++++++++++++++++++++++++++++++++++++++++++++++++++++++++++++++

\Exo{smath-0813}
\Exo{smath-0814}
\Exo{smath-0829}
\Exo{smath-0830}

%+++++++++++++++++++++++++++++++++++++++++++++++++++++++++++++++++++++++++++++++++++++++++++++++++++++++++++++++++++++++++++ 
\section{3 octobre 2014}
%+++++++++++++++++++++++++++++++++++++++++++++++++++++++++++++++++++++++++++++++++++++++++++++++++++++++++++++++++++++++++++

\Exo{smath-0820}
\Exo{smath-0822}
\Exo{smath-0823}
\Exo{smath-0824}
\Exo{smath-0847}
\Exo{smath-0846}
\Exo{smath-0845}

%+++++++++++++++++++++++++++++++++++++++++++++++++++++++++++++++++++++++++++++++++++++++++++++++++++++++++++++++++++++++++++ 
\section{17 octobre 2014}
%+++++++++++++++++++++++++++++++++++++++++++++++++++++++++++++++++++++++++++++++++++++++++++++++++++++++++++++++++++++++++++

\Exo{smath-0886}
\Exo{smath-0885}
\Exo{smath-0893}    % gâteau au chocolat. il faut un peu le modifier avant de le reposer
                    % parce qu'il serait mieux qu'il n'y ait pas deux ingrédients ayant les mêmes quantités (100g de chocolat et de lait).
\Exo{smath-0907}
\Exo{smath-0889}
\Exo{smath-0903}

%+++++++++++++++++++++++++++++++++++++++++++++++++++++++++++++++++++++++++++++++++++++++++++++++++++++++++++++++++++++++++++ 
\section{Devoir maison}
%+++++++++++++++++++++++++++++++++++++++++++++++++++++++++++++++++++++++++++++++++++++++++++++++++++++++++++++++++++++++++++

\Exo{smath-0841}

%+++++++++++++++++++++++++++++++++++++++++++++++++++++++++++++++++++++++++++++++++++++++++++++++++++++++++++++++++++++++++++ 
\section{Devoir surveillé, 25 novembre 2014}
%+++++++++++++++++++++++++++++++++++++++++++++++++++++++++++++++++++++++++++++++++++++++++++++++++++++++++++++++++++++++++++

\Exo{smath-0971}
\Exo{smath-0972}
\Exo{smath-0976}
\Exo{smath-0974}
\Exo{smath-0975}
\Exo{smath-0978}
\Exo{smath-0977}

%+++++++++++++++++++++++++++++++++++++++++++++++++++++++++++++++++++++++++++++++++++++++++++++++++++++++++++++++++++++++++++ 
\section{Interrogation du 5 décembre 2014}
%+++++++++++++++++++++++++++++++++++++++++++++++++++++++++++++++++++++++++++++++++++++++++++++++++++++++++++++++++++++++++++

\Exo{smath-0994}
\Exo{smath-0995}

%+++++++++++++++++++++++++++++++++++++++++++++++++++++++++++++++++++++++++++++++++++++++++++++++++++++++++++++++++++++++++++ 
\section{Interrogation du 9 décembre 2014}
%+++++++++++++++++++++++++++++++++++++++++++++++++++++++++++++++++++++++++++++++++++++++++++++++++++++++++++++++++++++++++++

\Exo{2smath-0001}   % Attention : la réponse de celui-ci est négative !! à ne pas refaire.
\Exo{2smath-0002}  

%+++++++++++++++++++++++++++++++++++++++++++++++++++++++++++++++++++++++++++++++++++++++++++++++++++++++++++++++++++++++++++ 
\section{Devoir surveillé, 16 décembre 2014}
%+++++++++++++++++++++++++++++++++++++++++++++++++++++++++++++++++++++++++++++++++++++++++++++++++++++++++++++++++++++++++++

\Exo{smath-0973}    % Ce serait bien de changer les nombres pour que la part totale ne soit pas un demi.
\Exo{2smath-0009}  
\Exo{2smath-0006} 
\Exo{2smath-0007}  
\Exo{2smath-0003}  
\Exo{2smath-0017}

% Ces exercices-ci sont le rattrapage fait le 13 janvier 2015.
\Exo{2smath-0063}
\Exo{2smath-0064}
\Exo{2smath-0065}
\Exo{2smath-0005}  
\Exo{2smath-0066}

%+++++++++++++++++++++++++++++++++++++++++++++++++++++++++++++++++++++++++++++++++++++++++++++++++++++++++++++++++++++++++++ 
\section{Interrogation du 16 janvier 2015}
%+++++++++++++++++++++++++++++++++++++++++++++++++++++++++++++++++++++++++++++++++++++++++++++++++++++++++++++++++++++++++++

\Exo{2smath-0072}
\Exo{2smath-0077}
\Exo{2smath-0074}
\Exo{2smath-0079}

%+++++++++++++++++++++++++++++++++++++++++++++++++++++++++++++++++++++++++++++++++++++++++++++++++++++++++++++++++++++++++++ 
\section{Devoir surveillé du 23 janvier 2015}
%+++++++++++++++++++++++++++++++++++++++++++++++++++++++++++++++++++++++++++++++++++++++++++++++++++++++++++++++++++++++++++

\Exo{2smath-0081}
\Exo{2smath-0087}   % Changer les sachets et biscuits en biscuits et pommes.
\Exo{2smath-0088}
\Exo{2smath-0099}
\Exo{2smath-0100}
\Exo{2smath-0082}
\Exo{2smath-0083}
\Exo{2smath-0084}
\Exo{2smath-0086}

%+++++++++++++++++++++++++++++++++++++++++++++++++++++++++++++++++++++++++++++++++++++++++++++++++++++++++++++++++++++++++++ 
\section{Interrogation du 6 février 2015}
%+++++++++++++++++++++++++++++++++++++++++++++++++++++++++++++++++++++++++++++++++++++++++++++++++++++++++++++++++++++++++++

\Exo{2smath-0138}
\Exo{2smath-0140}       % Avant de le remettre, il faut changer la figure pour qu'elle soit plus proche de vraiment 10 degrés
\Exo{2smath-0143}
\Exo{2smath-0141}
\Exo{2smath-0145}
\Exo{2smath-0139}

%+++++++++++++++++++++++++++++++++++++++++++++++++++++++++++++++++++++++++++++++++++++++++++++++++++++++++++++++++++++++++++ 
\section{Devoir surveillé du 20 février 2015}
%+++++++++++++++++++++++++++++++++++++++++++++++++++++++++++++++++++++++++++++++++++++++++++++++++++++++++++++++++++++++++++

\Exo{2smath-0161}
\Exo{2smath-0162}
\Exo{2smath-0160}

Attention : l'énoncé ici a été modifié par rapport à celui donné en devoir. Dans le devoir il était question de \emph{demi}-cercles; pour simplifier nous parlons ici de cercles entiers.

\Exo{2smath-0156}
\Exo{2smath-0167}
\Exo{2smath-0164}
\Exo{2smath-0165}

%+++++++++++++++++++++++++++++++++++++++++++++++++++++++++++++++++++++++++++++++++++++++++++++++++++++++++++++++++++++++++++ 
\section{Interrogation du 10 avril 2015}
%+++++++++++++++++++++++++++++++++++++++++++++++++++++++++++++++++++++++++++++++++++++++++++++++++++++++++++++++++++++++++++

\Exo{2smath-0222}  % angles et parall
\Exo{2smath-0223}   % proportionnalité
\Exo{2smath-0224}  % angles et parall
\Exo{2smath-0225}  % proportionnalité
\Exo{2smath-0226} % exp litt
\Exo{2smath-0227} % exp litt

%+++++++++++++++++++++++++++++++++++++++++++++++++++++++++++++++++++++++++++++++++++++++++++++++++++++++++++++++++++++++++++ 
\section{Devoir surveillé du 15 mai 2015 (et son rattrapage)}
%+++++++++++++++++++++++++++++++++++++++++++++++++++++++++++++++++++++++++++++++++++++++++++++++++++++++++++++++++++++++++++

\Exo{2smath-0250}
\Exo{2smath-0251}
\Exo{2smath-0252}
\Exo{2smath-0253}
\Exo{2smath-0254}
\Exo{2smath-0283}
\Exo{2smath-0284}

%+++++++++++++++++++++++++++++++++++++++++++++++++++++++++++++++++++++++++++++++++++++++++++++++++++++++++++++++++++++++++++ 
\section{Interrogation du 26 mai 2015}
%+++++++++++++++++++++++++++++++++++++++++++++++++++++++++++++++++++++++++++++++++++++++++++++++++++++++++++++++++++++++++++

\Exo{2smath-0293}
\Exo{2smath-0294}
\Exo{2smath-0295}
\Exo{2smath-0296}
\Exo{2smath-0297}
\Exo{2smath-0298}

%+++++++++++++++++++++++++++++++++++++++++++++++++++++++++++++++++++++++++++++++++++++++++++++++++++++++++++++++++++++++++++ 
\section{DS 8, 5 juin 2015}
%+++++++++++++++++++++++++++++++++++++++++++++++++++++++++++++++++++++++++++++++++++++++++++++++++++++++++++++++++++++++++++

\Exo{2smath-0309}
\Exo{2smath-0311}
\Exo{2smath-0312}
\Exo{2smath-0313}
\Exo{2smath-0237}
\Exo{2smath-0238}
\Exo{2smath-0239}

%+++++++++++++++++++++++++++++++++++++++++++++++++++++++++++++++++++++++++++++++++++++++++++++++++++++++++++++++++++++++++++ 
\section{Interrogation 5A et 5B numéro 7, 12 juin 2015}
%+++++++++++++++++++++++++++++++++++++++++++++++++++++++++++++++++++++++++++++++++++++++++++++++++++++++++++++++++++++++++++

\Exo{2smath-0318}
\Exo{2smath-0317}
\Exo{2smath-0319}
\Exo{2smath-0320}
\Exo{2smath-0325}
\Exo{2smath-0326}

%+++++++++++++++++++++++++++++++++++++++++++++++++++++++++++++++++++++++++++++++++++++++++++++++++++++++++++++++++++++++++++ 
\section{DS 8 (rattrapage), 19 juin 2015}
%+++++++++++++++++++++++++++++++++++++++++++++++++++++++++++++++++++++++++++++++++++++++++++++++++++++++++++++++++++++++++++

Certains des exercices du rattrapage étaient dans le DS8. Ils ne sont pas répétés ici.

\Exo{2smath-0323}   % Proportionnalité
\Exo{2smath-0327}
\Exo{2smath-0321}   
\Exo{2smath-0324}   % QCM
\Exo{2smath-0322}   % aires

