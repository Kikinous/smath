% This is part of Un soupçon de mathématique sans être agressif pour autant
% Copyright (c) 2013
%   Laurent Claessens
% See the file fdl-1.3.txt for copying conditions.

\begin{exercice}\label{exosmath-0562}

    À la question «existe-t-il un nombre \( x\) tel que \( x^2-4=x\) ?», Suzie a répondu que non en justifiant de la façon suivante : 
    \begin{quote}
        Non. En prenant par exemple \( x=5\) on trouve \( 5^2-4=25-4=21\) donc \( x^2-4\) n'est pas \( x\).
    \end{quote}
    Que penser de cette réponse ?

    Arnaud par contre répond :
    \begin{quote}
        J'essaye de résoudre l'équation \( x^2-4=x\) :
        \begin{subequations}
            \begin{align*}
                x^2-x=4\\
                x(x-1)=4.
            \end{align*}
        \end{subequations}
        Je ne parviens pas à résoudre cette équation, donc il n'y a pas de \( x\) tels que \( x^2-4=x\).
    \end{quote}

\corrref{smath-0562}
\end{exercice}
