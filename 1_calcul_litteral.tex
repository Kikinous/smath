% This is part of Un soupçon de mathématique sans être agressif pour autant
% Copyright (c) 2014
%   Laurent Claessens
% See the file fdl-1.3.txt for copying conditions.

%--------------------------------------------------------------------------------------------------------------------------- 
\subsection*{Carré sans coins}
%---------------------------------------------------------------------------------------------------------------------------

% This is part of Un soupçon de mathématique sans être agressif pour autant
% Copyright (c) 2014
%   Laurent Claessens
% See the file fdl-1.3.txt for copying conditions.

Un carreleur veut créer le motif suivant :
\begin{center}
   \input{Fig_TCBLooKXvOaZ.pstricks}
\end{center}
\begin{enumerate}
    \item
Combien de carreaux de couleur lui faudra-t-il ? 
\item
Reproduire le dessin pour une fresque de \( 6\times 6\) au lieu de \( 5\times 5\). Combien de carreaux de couleur faut-il alors ? 
\item
    Même questions pour une fresque \( 7\times 7\) et \( 10\times 10\).
\item
Et pour une fresque de taille \( 100\times 100\) que l'on crée encore selon le même motif ?
\item
Le professeur appelle $x$ le nombre de carreaux d'un côté de la fresque et $G$ le nombre de cases roses. Des élèves ont obtenu les expressions suivantes :
\begin{multicols}{3}
    \begin{itemize}
        \item
            Anis : \( A=x\times 4-2\)
        \item
            Basile : \( B=x-2\times 4\)
        \item
            Chloé : \( C=4\times (x-2)\)
        \item
             Dalila : \( D=(x-2)\times 4 \)
         \item
             Enzo : \( E=4\times x-8\)
         \item
             François : \( F=4\times x-4\)
    \end{itemize}
\end{multicols}
Parmi ces expressions, lesquelles sont fausses ? Pourquoi ? Y a-t-il plusieurs bonnes réponses ? Justifier.
\end{enumerate}



De \cite{NRHooXFvgpp5}

Questions subsidiaires : combien de carreaux blancs ?

%+++++++++++++++++++++++++++++++++++++++++++++++++++++++++++++++++++++++++++++++++++++++++++++++++++++++++++++++++++++++++++ 
\section{Expression «en fonction de \( x\)» }
%+++++++++++++++++++++++++++++++++++++++++++++++++++++++++++++++++++++++++++++++++++++++++++++++++++++++++++++++++++++++++++

\begin{definition}
    Une expression dans laquelle certains nombres sont représentés par des lettres est une \defe{expression littérale}{expression littérale}.
\end{definition}

\begin{example}
    Le triple d'un nombre \( x\) est noté «\( 3\times x\)».
\end{example}

\begin{example}
    La longueur d'un wagon est de \( 7.35\) mètres. La longueur de \( x\) wagons est \( 7.35\times x\) mètres.
\end{example}

La triple de \( x\) et la longueur des \( x\) wagons ont été exprimés \defe{en fonction de \( x\)}{en fonction de}.

%+++++++++++++++++++++++++++++++++++++++++++++++++++++++++++++++++++++++++++++++++++++++++++++++++++++++++++++++++++++++++++ 
\section{Simplification d'écriture}
%+++++++++++++++++++++++++++++++++++++++++++++++++++++++++++++++++++++++++++++++++++++++++++++++++++++++++++++++++++++++++++

\begin{Aretenir}
    Afin de simplifier les notations, nous écrivons
    \begin{enumerate}
        \item
            \( 3a\) pour \( 3\times a\)
        \item
            \( ax\) pour \( a\times x\).
    \end{enumerate}
\end{Aretenir}

\begin{example}
    Si \( y=5\) alors
    \begin{enumerate}
        \item
            \( 2y=2\times y=10\)
        \item
            \( ay=5\times a\)
    \end{enumerate}
\end{example}

%+++++++++++++++++++++++++++++++++++++++++++++++++++++++++++++++++++++++++++++++++++++++++++++++++++++++++++++++++++++++++++ 
\section{Développement et simplification}
%+++++++++++++++++++++++++++++++++++++++++++++++++++++++++++++++++++++++++++++++++++++++++++++++++++++++++++++++++++++++++++

% This is part of Un soupçon de mathématique sans être agressif pour autant
% Copyright (c) 2014
%   Laurent Claessens
% See the file fdl-1.3.txt for copying conditions.

%--------------------------------------------------------------------------------------------------------------------------- 
\subsection*{Activité : vente de gâteaux}
%---------------------------------------------------------------------------------------------------------------------------

Bob et Alice vendent des gâteaux pour financer un voyage. Chaque gâteau coûte deux euros à produire et est vendu cinq euros. Bob fait le raisonnement suivant : «si on vend \( n\) gâteaux, cela nous a coûté \( 2n\) euros à produire et rapporté \( 5n\) euros. Donc le bénéfice est \( 5n-2n\)». Alice par contre dit : «chaque gâteau rapporte \( 3\) euros; donc si on en vend \( n\), on gagne \( 3n\) euros». 
\begin{enumerate}
    \item
        Quel est le bénéfice obtenu en vendant \( 10\) gâteaux suivant le raisonnement de Bob ?
    \item
        Quel est le bénéfice obtenu en vendant \( 10\) gâteaux suivant le raisonnement d'Alice ?
    \item
        Qui a raison ?
\end{enumerate}


\begin{Aretenir}
    Pour tout nombres \( k\), \( x\) et \( y\) nous avons la formule de \defe{factorisation}{factorisation}
    \begin{equation}
        k\times x+k\times y=k\times (x+y),
    \end{equation}
    notée en abrégé par
    \begin{equation}
        kx+ky=k(x+y).
    \end{equation}
\end{Aretenir}

\begin{Aretenir}
    Pour tout nombres \( k\), \( x\) et \( y\) nous avons la formule de \defe{développement}{développement}
    \begin{equation}
        k\times (x+y)=k\times x+k\times y
    \end{equation}
    notée en abrégé par
    \begin{equation}
        k(x+y)=kx+ky.
    \end{equation}
\end{Aretenir}

%+++++++++++++++++++++++++++++++++++++++++++++++++++++++++++++++++++++++++++++++++++++++++++++++++++++++++++++++++++++++++++ 
\section{Petites astuces}
%+++++++++++++++++++++++++++++++++++++++++++++++++++++++++++++++++++++++++++++++++++++++++++++++++++++++++++++++++++++++++++

Quelque «structures» à retenir :
\begin{enumerate}
    \item
        Si \( n\) est un nombre entier, alors \( n+1\) est «le suivant».
    \item
        Les nombres pairs sont les nombres de la forme \( 2n\) où \( n\) est un nombre entier.
    \item
        La table de \( 7\) est la liste des nombres \( 7n\) où \( n\) prend toutes les valeurs entières positives.
\end{enumerate}


 
