% This is part of Un soupçon de mathématique sans être agressif pour autant
% Copyright (c) 2014
%   Laurent Claessens
% See the file fdl-1.3.txt for copying conditions.

%+++++++++++++++++++++++++++++++++++++++++++++++++++++++++++++++++++++++++++++++++++++++++++++++++++++++++++++++++++++++++++ 
\section{Expression «en fonction de \( x\)» }
%+++++++++++++++++++++++++++++++++++++++++++++++++++++++++++++++++++++++++++++++++++++++++++++++++++++++++++++++++++++++++++

\begin{definition}
    Une expression dans laquelle certains nombres sont représentés par des lettres est une \defe{expression littérale}{expression littérale}.
\end{definition}

\begin{example}
    Le triple d'un nombre \( x\) est noté «\( 3\times x\)».
\end{example}

\begin{example}
    La longueur d'un wagon est de \( 7.35\) mètres. La longueur de \( x\) wagons est \( 7.35\times x\) mètres.
\end{example}

La triple de \( x\) et la longueur des \( x\) wagons ont été exprimés \defe{en fonction de \( x\)}{en fonction de}.
