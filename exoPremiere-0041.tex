% This is part of Un soupçon de mathématique sans être agressif pour autant
% Copyright (c) 2012
%   Laurent Claessens
% See the file fdl-1.3.txt for copying conditions.

\begin{exercice}\label{exoPremiere-0041}

    \begin{enumerate}
        \item\label{ItemlzLrrk}
            Donner un polynôme du second degré qui s'annule en \( x=3\) et en \( x=-1\). 
        \item
            En donner un autre.
        \item
            Donner un polynôme du second degré s'annulant en \( x=3\) et en \( x=-1\) et dont le sommet soit à la hauteur \( -2\). Conseil : calculer le sommet du polynôme que vous avez trouvé au point \ref{ItemlzLrrk}.
        \item
            Donner un polynôme du second degré s'annulant en \( x=3\) et en \( x=-1\) et dont le sommet soit à la hauteur \( 1\). 
    \end{enumerate}

\corrref{Premiere-0041}
\end{exercice}
