% This is part of Un soupçon de mathématique sans être agressif pour autant
% Copyright (c) 2014
%   Laurent Claessens
% See the file fdl-1.3.txt for copying conditions.

% This is part of Un soupçon de mathématique sans être agressif pour autant
% Copyright (c) 2014
%   Laurent Claessens
% See the file fdl-1.3.txt for copying conditions.

Nous considérons le nombre \( A=(-2)+(-2)+(-2)+(-2)+(-2)\).

\begin{enumerate}
    \item
        Combien vaut \( B\) ?
    \item
        Écrire \( B\) sous forme d'un produit.
    \item
        Écrire sous forme d'une somme et calculer :
        \begin{enumerate}
            \item \( (-6)\times 4\)
            \item \( (-21)\times 5\)
            \item\( (-1.5)\times 3\).
        \end{enumerate}
\end{enumerate}

Compléter le tableau de produits suivant:
\begin{equation*}
    \begin{array}[]{|c||c|c|c|c|c|c|c|c|c|}
        \hline
        \times&-4&-3&-2&-1&\hphantom{-}0&1\hphantom{-}&2\hphantom{-}&3\hphantom{-}&4\hphantom{-}\\
        \hline\hline
        -4&&&&&0&&&&\\
        \hline
        -3&&&&&&&&&\\
        \hline
        -2&&&&&&&&&\\
        \hline
        -1&&&&&0&&&&\\
        \hline
        0&&&&&&&&0&\\
        \hline
        1&&&&&&&&&\\
        \hline
        2&&&&&&&&&\\
        \hline
        3&&&&&&&&&12\\
        \hline
        4&&&&&&4&&&\\
        \hline
    \end{array}
\end{equation*}


%+++++++++++++++++++++++++++++++++++++++++++++++++++++++++++++++++++++++++++++++++++++++++++++++++++++++++++++++++++++++++++ 
\section{Somme et différence}
%+++++++++++++++++++++++++++++++++++++++++++++++++++++++++++++++++++++++++++++++++++++++++++++++++++++++++++++++++++++++++++

\begin{propriete}
    \begin{enumerate}
        \item
    Pour la somme de deux nombres relatifs {\bf de même signe},
    \begin{itemize}
        \item
            nous conservons le signe commun
        \item
            nous additionnons les distances à zéro.
    \end{itemize}

    \begin{example}
        \begin{enumerate}
            \item
        \( (+4)+(+8)=4+8=12\)
    \item
        \( (-5)+(-6)=-(5+6)=-11\)
        \end{enumerate}
    \end{example}
\item
    Pour la somme de deux nombres relatifs {\bf de signes différents},
    \begin{itemize}
        \item
            nous prenons le signe de celui qui a la plus grande distance à zéro,
        \item
            nous calculons la différence des distances à zéro.
    \end{itemize}

    \begin{example}
        \begin{enumerate}
            \item
                \( (-3)+(+7)=+(7-3)=4\)
            \item
                \( (-9)+(+1)=-(9-1)=-8\)
        \end{enumerate}
    \end{example}
\item
    {\bf Soustraire} un nombre relatif revient à additionner son opposé.

    \begin{example}
        \begin{enumerate}
            \item
                \( (-4)-(-8)=(-4)+(+8)=+(8-4)=4\)
            \item
                \( (+7)-(+3)=7+(-3)=+(7-3)=4\)
        \end{enumerate}
    \end{example}

    \end{enumerate}
\end{propriete}

%+++++++++++++++++++++++++++++++++++++++++++++++++++++++++++++++++++++++++++++++++++++++++++++++++++++++++++++++++++++++++++ 
\section{Produit}
%+++++++++++++++++++++++++++++++++++++++++++++++++++++++++++++++++++++++++++++++++++++++++++++++++++++++++++++++++++++++++++

\begin{propriete}
    Pour le produit de deux nombres relatifs {\bf de signes contraires} :
    \begin{itemize}
        \item nous prenons le signe \( -\)
        \item nous calculons le produit des distances à zéro.
    \end{itemize}

    \begin{example}
        \begin{enumerate}
            \item
                \( (-3)\times 4=-(3\times 4)=-12\)
            \item
                \( 6\times(-5.5)=-(6\times 5.5)=-33\)
        \end{enumerate}
    \end{example}

    Pour le produit de deux nombres relatifs {\bf de même signes} :
    \begin{itemize}
        \item nous prenons le signe \( +\)
        \item nous calculons le produit des distances à zéro.
    \end{itemize}
    
    \begin{example}
        \begin{enumerate}
            \item
                \( 2.5\times 3=4.5\)
            \item
                \( (-3)\times (-7)=21\)
        \end{enumerate}
    \end{example}

\end{propriete}

%+++++++++++++++++++++++++++++++++++++++++++++++++++++++++++++++++++++++++++++++++++++++++++++++++++++++++++++++++++++++++++ 
\section{Vocabulaire}
%+++++++++++++++++++++++++++++++++++++++++++++++++++++++++++++++++++++++++++++++++++++++++++++++++++++++++++++++++++++++++++

\begin{definition}
    \begin{itemize}
        \item L'\defe{opposé}{opposé} d'un nombre s'obtient en changeant son signe.
        \item L'\defe{inverse}{inverse} d'un nombre \( x\) est le quotient \( \dfrac{ 1 }{ x }\).
    \end{itemize}
\end{definition}

\begin{example}
    \begin{enumerate}
        \item
            L'opposé de \( 3\) est \( -3\).
        \item
            L'opposé de \( -\dfrac{ 3 }{ 7 }\) est \( \dfrac{ 3 }{ 7 }\)
        \item
            L'inverse de \( 4\) est \( \frac{1}{ 4 }\)
        \item
            L'inverse de \( -5\) est \( -\frac{1}{ 5 }\)
    \end{enumerate}
\end{example}
