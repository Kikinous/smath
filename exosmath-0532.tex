% This is part of Un soupçon de mathématique sans être agressif pour autant
% Copyright (c) 2013
%   Laurent Claessens
% See the file fdl-1.3.txt for copying conditions.

\begin{exercice}\label{exosmath-0532}

    Louis veut postuler pour un poste dans une entreprise, et il veut choisir la bonne. Pour l'entreprise \( A\), il connait la distribution des salaires :
    \begin{equation*}
        \begin{array}[]{|c||c|c|c|c|c|c|c|c|}
            \hline
            \text{Salaire}&850&1000&1100&1250&1350&3000&5000&10000\\
            \hline\hline
            \text{Effectifs}&1&4&9&12&19&3&1&1\\
            \hline
            \text{Fréquences}&&&&&&&&\\
            \hline
            &&&&&&&&\\
            \hline
        \end{array}
    \end{equation*}
    \begin{multicols}{2}
        \begin{enumerate}
            \item
                Calculer le salaire moyen dans l'entreprise \( A\).
            \item
                Compléter la ligne des fréquences de l'entreprise \( A\). (justifier une seule des cases suffit)
            \item
                Calculer la médiane et les quartiles des salaires dans cette entreprise.
            \item
                À propos de l'entreprise \( B\), Louis connaît les données suivantes :
                \begin{enumerate}
                    \item
                        Salaire moyen : \( 3970\)
                    \item
                        Salaire médian : \( 1000\)
                    \item
                        Premier quartile : \( 950\)
                    \item
                        Troisième quartile : \( 1100\)
                \end{enumerate}
                Louis n'ayant pas un emploi de cadre, quelle entreprise a l'air de lui donner le meilleur salaire ? Justifier.
        \end{enumerate}
    \end{multicols}

\corrref{smath-0532}
\end{exercice}
