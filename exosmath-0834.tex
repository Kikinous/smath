% This is part of Un soupçon de mathématique sans être agressif pour autant
% Copyright (c) 2014
%   Laurent Claessens
% See the file fdl-1.3.txt for copying conditions.

\begin{exercice}[\dots/3.5]\label{exosmath-0834}

    Ève l'élève doit calculer la fraction
    \begin{equation}
        \frac{ 4+5\times 2 }{ 6+1 }
    \end{equation}
    et tape la séquence suivante sur sa calculatrice :
    \begin{equation}
        \input{JPIooEhtUVn.calcul}
    \end{equation}
    Est-ce correct ? Si oui, donner la valeur obtenue; sinon, placer des parenthèses de telle sorte à obtenir le bon résultat.

\corrref{smath-0834}
\end{exercice}
