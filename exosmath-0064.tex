% This is part of Un soupçon de mathématique sans être agressif pour autant
% Copyright (c) 2012
%   Laurent Claessens
% See the file fdl-1.3.txt for copying conditions.

\begin{exercice}\label{exosmath-0064}

    Soient les points \( A=(0;0)\), \( B=(2;1)\), \( C=(-2;3)\), \( E=(-3;-2)\) et \( F=(1;5)\).
    \begin{enumerate}
        \item
            Calculer les coordonnées du point \( D\) tel que \( ABCD\) soit une parallélogramme.
        \item
            Prouver que \( \vect{ AE }=-\vect{ FC }\).
        \item
            Que peut-t-on dire du quadrilatère \( AECF\) ?
        \item
            Montrer que les segments \( [FE]\) et \( [BD]\) ont même milieu.
    \end{enumerate}

\corrref{smath-0064}
\end{exercice}
