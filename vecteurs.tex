% This is part of Un soupçon de mathématique sans être agressif pour autant
% Copyright (c) 2012
%   Laurent Claessens
% See the file fdl-1.3.txt for copying conditions.

%+++++++++++++++++++++++++++++++++++++++++++++++++++++++++++++++++++++++++++++++++++++++++++++++++++++++++++++++++++++++++++
\section{Translation}
%+++++++++++++++++++++++++++++++++++++++++++++++++++++++++++++++++++++++++++++++++++++++++++++++++++++++++++++++++++++++++++

\begin{multicols}{2}
\begin{definition}
    La \defe{translation}{translation} \( t_{A,B}\) est la transformation du plan qui à un point \( C\) fait correspondre l'unique point \( D\) tel que les segments \( [AD]\) et \( [BC]\) aient même milieu.
\end{definition}

\columnbreak

%The result is on figure \ref{LabelFigDefVecteurAXDDGP}. % From file DefVecteurAXDDGP
%\newcommand{\CaptionFigDefVecteurAXDDGP}{<+Type your caption here+>}
\input{Fig_DefVecteurAXDDGP.pstricks}
\end{multicols}
Nous notons \( \vect{ AB }\) le vecteur associé à cette translation.

\begin{definition}
    Nous disons que \( \vect{ AB }=\vect{ CD }\) si et seulement si \( t_{A,B}=t_{C,D}\), c'est à dire si pour tout point \( K\) dans le plan nous avons \( t_{A,B}(K)=t_{C,d}(K)\).
\end{definition}

\begin{Aretenir}
    Si \( K'=t_{A,B}(K)\) alors \( ABK'K\) est un parallélogramme.
\end{Aretenir}

\begin{propriete}
    Soient quatre points du plan \( A\), \( B\), \( C\) et \( D\). Le quadrilatère \( ABCD\) est un parallélogramme si et seulement si \( \vect{ AB }=\vect{ CD }\).
\end{propriete}

\begin{proof}
    Nous commençons par supposer que \( \vect{ AB }=\vect{ CD }\) et nous prouvons qu'alors \( ABCD\) est un parallélogramme. Pour cela nous considérons un point \( K\) quelconque du plan et nous nous souvenons que \( t_{A,B}(K)=t_{C,D}(K)\). Nous nommons \( K'\) l'image de \( K\) par \( t_{A,B}\) ou \( t_{C,D}\). Autrement dit :
    \begin{equation}
        K'=t_{A,B}(K)=t_{C,D}(K).
    \end{equation}
    Donc \( ABK'K\) et \( CDk'K\) sont des parallélogrammes.

    Le fait que \( ABK'K\) soit un parallélogramme implique que \( AB=K'K\) et que \( (AB)\parallel(K'K)\). De la même manière, le fait que \( CDK'K\) soit un parallélogramme implique que \( CD=K'K\) et que \( (CD)\parallel(K'K)\). Donc nous avons \( AB=CD\) et \( (AB)\parallel(CD)\), ce qui montre que \( ABCD\) est un parallélogramme.
\end{proof}
<++>

%+++++++++++++++++++++++++++++++++++++++++++++++++++++++++++++++++++++++++++++++++++++++++++++++++++++++++++++++++++++++++++
\section{Exercices}
%+++++++++++++++++++++++++++++++++++++++++++++++++++++++++++++++++++++++++++++++++++++++++++++++++++++++++++++++++++++++++++

\Exo{smath-0053}
\Exo{smath-0052}
\Exo{smath-0055}

%---------------------------------------------------------------------------------------------------------------------------
\subsection{Colinéarité}
%---------------------------------------------------------------------------------------------------------------------------

\Exo{smath-0054}
