% This is part of Un soupçon de mathématique sans être agressif pour autant
% Copyright (c) 2012
%   Laurent Claessens
% See the file fdl-1.3.txt for copying conditions.

%+++++++++++++++++++++++++++++++++++++++++++++++++++++++++++++++++++++++++++++++++++++++++++++++++++++++++++++++++++++++++++
\section{Translation}
%+++++++++++++++++++++++++++++++++++++++++++++++++++++++++++++++++++++++++++++++++++++++++++++++++++++++++++++++++++++++++++

\begin{multicols}{2}
    \begin{definition}  \label{DefAAJEuS}
    La \defe{translation}{translation} \( t_{A,B}\) est la transformation du plan qui à un point \( C\) fait correspondre l'unique point \( D\) tel que les segments \( [AD]\) et \( [BC]\) aient même milieu.
\end{definition}

\columnbreak

%The result is on figure \ref{LabelFigDefVecteurAXDDGP}. % From file DefVecteurAXDDGP
%\newcommand{\CaptionFigDefVecteurAXDDGP}{<+Type your caption here+>}
\input{Fig_DefVecteurAXDDGP.pstricks}
\end{multicols}
Nous notons \( \vect{ AB }\) le vecteur associé à cette translation.

Étant donné qu'un quadrilatère dont les diagonales se coupent en leur milieu est un parallélogramme, nous avons immédiatement le règle suivante :
\begin{Aretenir}
    Le quadrilatère \( ABDC\) est un parallélogramme si et seulement si \( D=t_{A,B}(C)\).

Attention : il s'agit bien de \( ABDC\) et non de \( ABCD\).
\end{Aretenir}

Le dessin à côté de la définition \ref{DefAAJEuS}, aplati, donne immédiatement aussi
\begin{equation}
    t_{A,B}(A)=B.
\end{equation}

\begin{definition}
    Nous disons que \( \vect{ AB }=\vect{ CD }\) si et seulement si \( t_{A,B}(C)=D\), c'est à dire si \( ABDC\) est un parallélogramme.
\end{definition}

\begin{propriete}
    Si \( \vect{ AB }=\vect{ CD }\) alors les translations \( t_{A,B}\) et \( t_{C,D}\) sont égales, c'est à dire que pour tout point \( K\) dans le plan, nous avons \( t_{A,B}(K)=t_{C,D}(K)\).
\end{propriete}

\begin{proof}
    L'égalité \( \vect{ AB }=\vect{ CD }\) nous dit immédiatement que \( ABDC\) est un parallélogramme. Soit \( K\) un point quelconque du plan et \( K'=t_{A,B}(K)\). Alors \( ABK'K\) est un parallélogramme. Nous devons prouver que \( CDK'K\) est également un parallélogramme. La situation est à la figure \ref{LabelFigVectoParallItzteT}. % From file VectoParallItzteT
\newcommand{\CaptionFigVectoParallItzteT}{Nous savons que \( ABDC\) et \( CDK'K\) sont des parallélogramme, et nous voulons déduire que \( CDK'K\) en est également un.}
\input{Fig_VectoParallItzteT.pstricks}

    Par les parallélogramme connus, nous avons \( (AB)\parallel (CD)\) et \( (AB)\parallel (KK')\), donc \( (CD)\parallel (KK')\). La difficulté est de prouver que \( (CK)\parallel (DK')\). 
    
    Pour cela nous considérons les triangles \( ACK\) et \( BDK'\), et nous allons montrer qu'ils sont isométriques. Vu que \( (KA)\parallel (K'B)\), les angles notés \( \alpha\) sur la figure \ref{LabelFigVectoParallelgjDlmD} sont égaux. Le fait que \( (DB)\parallel (CA)\) donne l'égalité des angles notés \( \beta\) sur cette même figure. De tout cela nous concluons que \( \hat A=\hat B\) pour les triangles \( ACK\) et \( BDK'\).
    % From file VectoParallelgjDlmD
\newcommand{\CaptionFigVectoParallelgjDlmD}{Égalité de quelque angles.}
\input{Fig_VectoParallelgjDlmD.pstricks}

    Par les parallélogramme connus nous avons aussi les égalités de longueurs \( AK=BK'\) et \( AC=BD\). En vertu de la propriété \ref{PropRtqqxJ} sur les triangles isométriques, les triangles \( ACK\) et \( BDK'\) sont isométriques. En particulier \( \hat K=\hat K'\).

    L'égalité des angles \( \hat K\) et \( \hat K'\) et le parallélisme \( (KA)\parallel (K'B)\) entraine que \( (KC)\parallel(K'D)\).

    Maintenant le quadrilatère \( CDK'K\) est formé de deux couples de droites parallèles. Les longueurs sont de plus égales parce que \( CD=AB=KK'\) (encore par les parallélogrammes) et \( KC=K'D\) par isométrie des triangles. Le quadrilatère \( CDK'K\) es estt donc un parallélogramme. Cela prouve que \( K'\) est l'image de \( K\) par la translation \( t_{C,D}\).
\end{proof}

\begin{definition}
    \begin{enumerate}
        \item
    Le vecteur \( \vect{ AB }\) est \defe{nul}{vecteur!nul} si les points \( A\) et \( B\) sont confondus. On le note alors \( \vect{ 0 }\).

        \item
    Nous définissons aussi \( -\vect{ AB }=\vect{ BA }\).

\item

    Le vecteur \defe{somme}{somme!de vecteur}\index{vecteur!somme} \( \vect{ AB }+\vect{ CD }\) est le vecteur qui correspond à la translation composée de \( t_{A,B}\) par \( t_{C,D}\)
    \end{enumerate}
\end{definition}

\begin{Aretenir}
    \begin{multicols}{2}
    Nous avons la \defe{relation de Chasles}{Chasles} qui permet de mettre des vecteurs «bout à bout» :
    \begin{equation}
        \vect{ AB }+\vect{ BC }=\vect{ AC }.
    \end{equation}

    \columnbreak

%Une illustration de la relation de Chasles est donnée à la figure \ref{LabelFigChaslesGTRtKR}. % From file ChaslesGTRtKR
%\newcommand{\CaptionFigChaslesGTRtKR}{La somme $ \vect{ AB }+\vect{ BC }$ est le vecteur $ \vect{ AC }$.}
\input{Fig_ChaslesGTRtKR.pstricks}

    \end{multicols}
\end{Aretenir}

%+++++++++++++++++++++++++++++++++++++++++++++++++++++++++++++++++++++++++++++++++++++++++++++++++++++++++++++++++++++++++++ 
\section{Milieu d'un segment}
%+++++++++++++++++++++++++++++++++++++++++++++++++++++++++++++++++++++++++++++++++++++++++++++++++++++++++++++++++++++++++++

\begin{propriete}
    \begin{multicols}{2}

        Le point \( M\) est milieu du segment \( [AB]\) si et seulement si \( \vect{ AM }=\vect{ MB }\).

        \columnbreak

%The result is on figure \ref{LabelFigVectoMilieuNuWgHW}. % From file VectoMilieuNuWgHW
%\newcommand{\CaptionFigVectoMilieuNuWgHW}{<+Type your caption here+>}
\input{Fig_VectoMilieuNuWgHW.pstricks}

    \end{multicols}
\end{propriete}

\begin{proof}
    Supposons pour commencer que \( M\) soit le milieu du segment \( [AB]\). Afin de prouver que \( \vect{ AM }=\vect{ MB }\), nous prouvons que \( AMBM\) est un parallélogramme. Vu que \( M\) est sur la droite \( (AB)\) nous avons évidemment le parallélisme des côtés : \( (AM)\parallel (MB)\).

    En ce qui concerne l'égalité des longueurs, \( AM=BM\) parce que \( M\) est milieu de \( [AB]\).

    Pour la réciproque, nous supposons que \( \vect{ AM }=\vect{ MB }\) et nous devons prouver que \( M\) est le milieu de \( [AB]\).  Par hypothèse, \( AMBM\) est un parallélogramme, et donc les diagonales se coupent en leur milieu. Ces diagonales sont \( [AB]\) et \( [MM]\). Le milieu de \( [MM]\) est évidemment \( M\), qui est alors le milieu de \( [AB]\), ce qu'il fallait prouver.
\end{proof}

\begin{theorem}[Théorème de milieux]

    \begin{multicols}{2}
    La droite joignant les milieux de deux côtés d'un triangle est parallèle au troisième côté.

    Sur la figure ci-contre \( I\) et \( J\) sont les milieux de \( [AB]\) et \( [AC]\) et nous avons l'égalité vectorielle
    \begin{equation}
        \vect{ BC }=2\vect{ IJ }.
    \end{equation}

    \columnbreak

%The result is on figure \ref{LabelFigfigureVNaHvXi}. % From file figureVNaHvXi
%\newcommand{\CaptionFigfigureVNaHvXi}{<+Type your caption here+>}
    \begin{center}
\input{Fig_figureVNaHvXi.pstricks}
    \end{center}

    \end{multicols}
\end{theorem}

\begin{proof}
    En utilisant les relations de Chasles : \( \vect{ IJ }=\vect{ IB }+\vect{ BC }+\vect{ CJ }\). Mais \( \vect{ IB }=\vect{ AI }\) et \( \vect{ CJ }=\vect{ JA }\), donc
    \begin{equation}
        \vect{ IJ }=\vect{ AI }+\vect{ BC }+\vect{ JA }=\vect{ JI }+\vect{ BC }.
    \end{equation}
    Étant donné que \( \vect{ IJ }=-\vect{ JI }\) nous trouvons \( 2\vect{ IJ }=\vect{ BC }\).
\end{proof}

On démontre de la même façon que si \( \vect{ AI }=k\vect{ AC }\) et \( \vect{ BJ }=k\vect{ BC }\), alors
\begin{equation}
    \vect{ IJ }=(1-k)\vect{ AB }.
\end{equation}

\begin{propriete}
    Si \( ABCD\) est un quadrilatère quelconque et si \( I\), \( J\), \( K\) et \( L\) sont les milieux des segments consécutifs de \( ABCD\), alors \( IJKL\) est un parallélogramme.
\end{propriete}

\begin{proof}

    \begin{multicols}{2}

        Nous considérons la diagonale \( [AC]\) et le triangle \( ABC\). Les points \( I\) et \( J\) étant les milieux des côtés \( [AB]\) et \( [BC]\), nous avons l'égalité vectorielle \( \vect{ AC }=2\vect{ IJ }\).

    De la même façon, dans le triangle \( ACD\), nous avons \( \vect{ AC }=2\vect{ LK }\). Par conséquent \( \vect{ LK }=\vect{ IJ }\), ce qui prouve que \( IJKL\) est un parallélogramme.

        \columnbreak

%The result is on figure \ref{LabelFigfigureBOuQJyj}. % From file figureBOuQJyj
%\newcommand{\CaptionFigfigureBOuQJyj}{<+Type your caption here+>}
   \begin{center}
\input{Fig_figureBOuQJyj.pstricks}
   \end{center}

    \end{multicols}


\end{proof}
 
%+++++++++++++++++++++++++++++++++++++++++++++++++++++++++++++++++++++++++++++++++++++++++++++++++++++++++++++++++++++++++++ 
\section{Coordonnées d'un vecteur dans un repère}
%+++++++++++++++++++++++++++++++++++++++++++++++++++++++++++++++++++++++++++++++++++++++++++++++++++++++++++++++++++++++++++

\begin{multicols}{2}
    \begin{definition}
        Les coordonnées du vecteur \( \vect{ u }\) dans un repère d'origine \( O\) sont les coordonnées du point \( M\) tel que \( \vect{ u }=\vect{ OM }\).
    \end{definition}

    \columnbreak

    %The result is on figure \ref{LabelFigfigureNNgEEzx}. % From file figureNNgEEzx
    %\newcommand{\CaptionFigfigureNNgEEzx}{<+Type your caption here+>}
    \begin{center}
\input{Fig_figureNNgEEzx.pstricks}
    \end{center}
\end{multicols}
Dans le cas ci-dessus, nous avons \( \vect{ AB }=\vect{ OM }\) et les coordonnées de \( M\) sont \( M=(-2;1)\). Nous notons
\begin{equation}
    \vect{ AB }=\begin{pmatrix}
        -2    \\ 
        1    
    \end{pmatrix}.
\end{equation}

\begin{Aretenir}
    \begin{multicols}{2}
    Si \( A=(x_A;x_B)\) et \( B=(x_B;y_B)\) alors
    \begin{equation}
        \vect{ AB }=\begin{pmatrix}
            x_B-x_A    \\ 
            y_B-y_A    
        \end{pmatrix}.
    \end{equation}

    \columnbreak

   %The result is on figure \ref{LabelFigfigureLEOvqez}. % From file figureLEOvqez
%\newcommand{\CaptionFigfigureLEOvqez}{<+Type your caption here+>}
    \begin{center}
\input{Fig_figureLEOvqez.pstricks}
    \end{center}

    \end{multicols}
   % TODO : lire le TODO qu'il y a dans le fichier phystricksfigureLEOvqez 
\end{Aretenir}

%--------------------------------------------------------------------------------------------------------------------------- 
\section{Règles de calcul}
%---------------------------------------------------------------------------------------------------------------------------

\begin{propriete}
    Si dans un repère \( \vect{ u }=\begin{pmatrix}
        x    \\ 
        y    
    \end{pmatrix}\) et \( \vect{ v }=\begin{pmatrix}
        x'    \\ 
        y'    
    \end{pmatrix}\), alors 
    \begin{equation}
        \vect{ u }+\vect{ v }=\begin{pmatrix}
            x+x'    \\ 
            y+y'    
        \end{pmatrix}.
    \end{equation}
    
\end{propriete}

\begin{proof}
    Soient \( A\) et \( B\) des points tels que \( \vect{ u }=\vect{ AB }\). Vu que les vecteurs peuvent être placés n'importe où, nous pouvons placer \( \vect{ v }\) au point \( B\) et dire \( \vect{ v }=\vect{ BC }\) pour un certain point \( C\). Par les relations de Chasles,
    \begin{equation}    \label{EqASjyYXs}
        \vect{ u }+\vect{ v }=\vect{ AC }=\begin{pmatrix}
            x_C-x_A    \\ 
            y_C-y_A    
        \end{pmatrix}.
    \end{equation}
    
    D'autre part étant donné que \( \vect{ u= }\vect{ AB }\), nous avons
    \begin{subequations}
        \begin{align}
            x=x_B-x_A\\
            y=y_B-y_A
        \end{align};
    \end{subequations}
    et vu que \( \vect{ v }=\vect{ BC }\)
    \begin{subequations}
        \begin{align}
            x'=x_C-x_B\\
            y'=y_C-y_B
        \end{align}
    \end{subequations}
    Du coup nous avons
    \begin{subequations}
        \begin{align}
            x+x'=(x_B-x_A)+(x_C-x_B)=x_C-x_A\\
            y+y'=(y_B-y_A)+(y_C-y_B)=y_C-y_A
        \end{align}
    \end{subequations}
    En comparant avec \eqref{EqASjyYXs}, nous avons bien
    \begin{equation}
        \vect{ u }+\vect{ v }=\vect{ AC }=\begin{pmatrix}
            x_C-x_A    \\ 
            y_C-y_A    
        \end{pmatrix}=\begin{pmatrix}
            x+x'    \\ 
            y+y'    
        \end{pmatrix},
    \end{equation}
    ce qu'il fallait démontrer.
\end{proof}

\begin{definition}
    Soit \( \lambda\in \eR\) et \( \vect{ u }\), un vecteur. Si dans un repère \( \vect{ u }=\begin{pmatrix}
        x    \\ 
        y    
    \end{pmatrix}\), alors le vecteur \( \lambda\vect{ u }\) est le vecteur de coordonnées \( \begin{pmatrix}
        \lambda x    \\ 
        \lambda y    
    \end{pmatrix}\) dans ce repère.
\end{definition}

\begin{Aretenir}
    Les règles de calcul vectoriel vis-à-vis de la multiplication et de l'addition sont essentiellement les mêmes que celles qui fonctionnent avec les réels :
    \begin{enumerate}
        \item
            \( (\lambda+\mu)\vect{ u }=\lambda\vect{ u }+\mu\vect{ u }\)
        \item
            \( \lambda(\mu\vect{ u })=(\lambda\mu)\vect{ u }\)
        \item
            \( \lambda(\vect{ u }+\vect{ v })=\lambda\vect{ u }+\lambda\vect{ v }\).
    \end{enumerate}
\end{Aretenir}

%+++++++++++++++++++++++++++++++++++++++++++++++++++++++++++++++++++++++++++++++++++++++++++++++++++++++++++++++++++++++++++ 
\section{Parallélisme et colinéarité}
%+++++++++++++++++++++++++++++++++++++++++++++++++++++++++++++++++++++++++++++++++++++++++++++++++++++++++++++++++++++++++++

\begin{definition}
    Deux vecteurs \( \vect{ u }\) et \( \vect{ v }\) sont \defe{colinéaires}{colinéaire (vecteurs)} si il existe \( \lambda\in \eR\) tel que \( \vect{ u }=\lambda\vect{ v }\).
\end{definition}

\begin{propriete}
    \begin{enumerate}
        \item
            Les droites \( (AB)\) et \( (CD)\) sont parallèles si et seulement si les vecteurs \( \vect{ AB }\) et \( \vect{ CD }\) sont colinéaires.
        \item
            Les points \( A\), \( B\) et \( C\) sont alignés si et seulement si les vecteurs \( \vect{ AB }\) et \( \vect{ AC }\) sont colinéaires.
    \end{enumerate}
\end{propriete}

Nous savons qu'un quadrilatère ayant deux côtés parallèles de même longueur est un parallélogramme. Donc nous avons le critère suivant pour savoir si \( ABCD\) est un parallélogramme :
\begin{equation}
    \vect{ AB }=\vect{ CD }.
\end{equation}
Bien entendu les autres côtés fonctionnent aussi :
\begin{equation}
    \vect{ AC }=\vect{ BD }.
\end{equation}
Si une de ces deux égalités vectorielle est satisfaite, alors \( ABCD\) est un parallélogramme.

\begin{example}
    Les points \( A=(-2;-3)\), \( B=(-1,-1)\), \( C=(2;-2)\) et \( D=(1;-4)\) forment un parallélogramme.
\end{example}

%+++++++++++++++++++++++++++++++++++++++++++++++++++++++++++++++++++++++++++++++++++++++++++++++++++++++++++++++++++++++++++
\section{Exercices}
%+++++++++++++++++++++++++++++++++++++++++++++++++++++++++++++++++++++++++++++++++++++++++++++++++++++++++++++++++++++++++++

%--------------------------------------------------------------------------------------------------------------------------- 
\subsection{Vecteurs dans le plan}
%---------------------------------------------------------------------------------------------------------------------------

\Exo{smath-0196}
\Exo{smath-0053}
\Exo{smath-0067}

\Exo{smath-0213}
\Exo{smath-0065}
\Exo{smath-0105}
\Exo{smath-0068}
\Exo{smath-0055}
\Exo{smath-0142}
\Exo{smath-0070}
\Exo{smath-0085}
\Exo{smath-0071}
\Exo{smath-0072}
\Exo{smath-0076}
\Exo{smath-0066}
\Exo{smath-0052}
\Exo{smath-0075}
\Exo{Seconde-0098}
\Exo{Seconde-0093}
\Exo{smath-0073}
\Exo{smath-0199}

%---------------------------------------------------------------------------------------------------------------------------
\subsection{Coordonnées}
%---------------------------------------------------------------------------------------------------------------------------

\Exo{smath-0103}
\Exo{smath-0059}
\Exo{smath-0104}
\Exo{smath-0063}
\Exo{smath-0107}
\Exo{smath-0064}

%---------------------------------------------------------------------------------------------------------------------------
\subsection{Colinéarité}
%---------------------------------------------------------------------------------------------------------------------------

\Exo{smath-0108}
\Exo{smath-0061}
\Exo{smath-0060}
\Exo{smath-0054}
\Exo{smath-0106}

\subsection{Droites}

\Exo{smath-0109}

\subsection{Problèmes}

\Exo{smath-0111}
\Exo{smath-0112}
\Exo{smath-0074}

%---------------------------------------------------------------------------------------------------------------------------
\subsection{Théorie}
%---------------------------------------------------------------------------------------------------------------------------

\Exo{smath-0056}
\Exo{smath-0057}
\Exo{smath-0058}
