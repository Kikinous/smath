% This is part of Un soupçon de mathématique sans être agressif pour autant
% Copyright (c) 2014
%   Laurent Claessens
% See the file fdl-1.3.txt for copying conditions.

\begin{exercice}\label{exosmath-0714}

    Questions de type «calcul mental» 
    \begin{multicols}{2}
    \begin{enumerate}
        \item
            Quel est l'ensemble de définition de la fonction définie par \( f(x)=\frac{ x+2 }{ 3x+5 }\) ?
        \item
            Donner l'équation d'une droite passant par le point \( A(4;7)\).
        \item
            Soit la suite de nombres \( 1\), \( 7\), \( 8\), \( 8\), \( 10\), \( 15\), \( 15\), \( 20\). Est-il possible d'augmenter la moyenne sans changer la médiane en changeant un seul nombre ?
        \item
            Résoudre l'équation \( 5x+7=-2x+4\).
        \item
            Donner les coordonnées du vecteur \( k\vect{ v }\) si \( k=3\) et \( \vect{ v }=\begin{pmatrix}
                -3    \\ 
                10    
            \end{pmatrix}\).
    \end{enumerate}
    \end{multicols}

\corrref{smath-0714}
\end{exercice}
