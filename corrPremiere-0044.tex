% This is part of Un soupçon de mathématique sans être agressif pour autant
% Copyright (c) 2012
%   Laurent Claessens
% See the file fdl-1.3.txt for copying conditions.

\begin{corrige}{Premiere-0044}

    Si nous ajoutons la longueur \( x\) aux trois côtés, nous trouvons un triangle de côtés de longueurs \( 3+x\), \( 4+x\) et \( 6+x\). Il sera rectangle si et seulement si
    \begin{equation}
        (3+x)^2+(4+x)^2=(6+x)^2
    \end{equation}
    En développant tous les carrés (par exemple \( (3+x)^2=9+6x+x^2\)), et en mettant tout d'un côté, nous tombons sur l'équation
    \begin{equation}
        x^2+2x-11=0
    \end{equation}
    dont le solutions sont \( x=-2\sqrt{3}-1\) et \( 2\sqrt{3}-1\). La première est exclue parce qu'elle rendrait négative la longueur d'un des côtés. Nous avons donc le triangle de côtés \( 2+2\sqrt{3}\), \( 3+2\sqrt{3}\) et \( 5+2\sqrt{3}\).

\end{corrige}
