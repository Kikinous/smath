% This is part of Un soupçon de mathématique sans être agressif pour autant
% Copyright (c) 2014
%   Laurent Claessens
% See the file fdl-1.3.txt for copying conditions.

Un carreleur veut créer le motif suivant :
\begin{center}
   \input{Fig_TCBLooKXvOaZ.pstricks}
\end{center}
\begin{enumerate}
    \item
Combien de carreaux de couleur lui faudra-t-il ? 
\item
Reproduire le dessin pour une fresque de \( 6\times 6\) au lieu de \( 5\times 5\). Combien de carreaux de couleur faut-il alors ? 
\item
Et pour une fresque de taille \( 100\times 100\) que l'on crée encore selon le même motif ?
\item
Le professeur appelle $x$ le nombre de carreaux d'un côté de la fresque et $G$ le nombre de cases roses. Des élèves ont obtenu les expressions suivantes :
\begin{multicols}{3}
    \begin{itemize}
        \item
            Anis : \( G=x\times 4-2\)
        \item
            Basile : \( G=x-2\times 4\)
        \item
            Chloé : \( G=4\times (x-2)\)
        \item
             Dalila : \( (x-2)\times 4 \)
         \item
             Enzo : \( G=4\times x-8\)
         \item
             René : \( G=4\times x-4\)
    \end{itemize}
\end{multicols}
Parmi ces expressions, lesquelles sont fausses ? Pourquoi ? Y a-t-il plusieurs bonnes réponses ? Justifier.
\end{enumerate}


