% This is part of Un soupçon de mathématique sans être agressif pour autant
% Copyright (c) 2014
%   Laurent Claessens
% See the file fdl-1.3.txt for copying conditions.

%--------------------------------------------------------------------------------------------------------------------------- 
\subsection*{Activité : carré sans coins}
%---------------------------------------------------------------------------------------------------------------------------

Un carreleur veut créer le motif suivant :
\begin{center}
   \input{Fig_TCBLooKXvOaZ.pstricks}
\end{center}
\begin{enumerate}
    \item
Combien de carreaux de couleur lui faudra-t-il ? 
\item
Reproduire le dessin pour une fresque de \( 6\times 6\) au lieu de \( 5\times 5\). Combien de carreaux de couleur faut-il alors ? 
\item
    Même questions pour une fresque \( 7\times 7\) et \( 10\times 10\).
\item
Et pour une fresque de taille \( 100\times 100\) que l'on crée encore selon le même motif ?
\item
Le professeur appelle $x$ le nombre de carreaux d'un côté de la fresque et $G$ le nombre de cases roses. Des élèves ont obtenu les expressions suivantes :
\begin{multicols}{3}
    \begin{itemize}
        \item
            Anis : \( A=x\times 4-2\)
        \item
            Basile : \( B=x-2\times 4\)
        \item
            Chloé : \( C=4\times (x-2)\)
        \item
             Dalila : \( D=(x-2)\times 4 \)
         \item
             Enzo : \( E=4\times x-8\)
         \item
             François : \( F=4\times x-4\)
    \end{itemize}
\end{multicols}
Parmi ces expressions, lesquelles sont fausses ? Pourquoi ? Y a-t-il plusieurs bonnes réponses ? Justifier.
\end{enumerate}


