% This is part of Un soupçon de mathématique sans être agressif pour autant
% Copyright (c) 2012
%   Laurent Claessens
% See the file fdl-1.3.txt for copying conditions.

\begin{exercice}\label{exoSeconde-0080}


    Soit une tour de hauteur \( h\). Nous voudrions savoir (en fonction de \( h\)) la distance à laquelle on voir l'horizon du sommet de la tour.
    \begin{enumerate}
        \item
            Dessiner un grand cercle qui représente la Terre. Nous notons \( R\) son rayon et \( O\) son centre.
        \item   \label{ItemOodDLD}
            Dessiner sur la surface de la Terre une tour de hauteur \( h\).
        \item
            À quelle distance du centre de la Terre se trouve le sommet de la tour ? Nous nommons \( S\) le point au sommet de la tour et \( A\) le point à la base de la tour.
        \item
            Tracer la tangente à la Terre passant par \( S\). Nous nommons \( B\) le point d'intersection. C'est en ce point que se trouve l'horizon. Les prochaines étapes nous permettront de calculer la longueur \( AB\).
        \item   \label{ItemdWmeVL}
            À quelle distance le point \( B\) se trouve-t-il du centre de la Terre ? (facile)
        \item
            Écrire le théorème de Pythagore pour le triangle \( SBO\) en utilisant les réponses aux questions \ref{ItemOodDLD} et \ref{ItemdWmeVL}. Pour ce faire, sachez que la droite \( (SB)\) est perpendiculaire à la droite \( OB\) parce que pour un cercle, la tangente est toujours perpendiculaire au rayon.
        \item
            Déduire la longueur du segment \( [SB]\) en fonction de \( R\) et \( h\).
        \item
            Nous supposons maintenant que les droites \( (BA)\) et \( (AS)\) soient perpendiculaires. C'est pas tout à fait vrai, mais en pratique, ça signifie que la Terre est plate, ce qui est une bonne approximation pour de petites distances. Écrire le théorème de Pythagore dans le triangle \( ASB\).
        \item
            Déduire la valeur de la longueur \( AB\). La réponse dépend de \( R\) et de \( h\).
        \item
            Calculer la réponse dans le cas de la tour Eiffel (\unit{324}{\meter}) en supposant que la Terre ait un rayon de \unit{6300}{\kilo\meter}.
    \end{enumerate}

\corrref{Seconde-0080}
\end{exercice}
