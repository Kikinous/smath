\vbox{1
\emph{Toutes les réponses doivent être justifiées par un calcul accompagné d'un raisonnement.}

\begin{wrapfigure}{r}{5.0cm}
   \vspace{-0.5cm}        % à adapter.
   \centering
   \input{Fig_OKTXHoc.pstricks}
\end{wrapfigure}

    La figure ci-contre est un cube de \unit{2}{\centi\meter}.

    \begin{enumerate}
    \item
    Quelle est la nature du triangle \(EGC\) ? 
    \item
    Quel est son périmètre ?
    \item
    Quelle est son aire ? (pour les rapides)
    \end{enumerate}
    

}
\vspace{2cm}
\vbox{2
\emph{Toutes les réponses doivent être justifiées par un calcul accompagné d'un raisonnement.}

\begin{wrapfigure}{r}{5.0cm}
   \vspace{-0.5cm}        % à adapter.
   \centering
   \input{Fig_OKTXHoc.pstricks}
\end{wrapfigure}

    La figure ci-contre est un cube de \unit{5}{\centi\meter}.

    \begin{enumerate}
    \item
    Quelle est la nature du triangle \(HEB\) ? 
    \item
    Quel est son périmètre ?
    \item
    Quelle est son aire ? (pour les rapides)
    \end{enumerate}
    

}
\vspace{2cm}
\vbox{3
\emph{Toutes les réponses doivent être justifiées par un calcul accompagné d'un raisonnement.}

\begin{wrapfigure}{r}{5.0cm}
   \vspace{-0.5cm}        % à adapter.
   \centering
   \input{Fig_OKTXHoc.pstricks}
\end{wrapfigure}

    La figure ci-contre est un cube de \unit{5}{\centi\meter}.

    \begin{enumerate}
    \item
    Quelle est la nature du triangle \(FBH\) ? 
    \item
    Quel est son périmètre ?
    \item
    Quelle est son aire ? (pour les rapides)
    \end{enumerate}
    

}
\vspace{2cm}
\vbox{4
\emph{Toutes les réponses doivent être justifiées par un calcul accompagné d'un raisonnement.}

\begin{wrapfigure}{r}{5.0cm}
   \vspace{-0.5cm}        % à adapter.
   \centering
   \input{Fig_OKTXHoc.pstricks}
\end{wrapfigure}

    La figure ci-contre est un cube de \unit{6}{\centi\meter}.

    \begin{enumerate}
    \item
    Quelle est la nature du triangle \(BGA\) ? 
    \item
    Quel est son périmètre ?
    \item
    Quelle est son aire ? (pour les rapides)
    \end{enumerate}
    

}
\vspace{2cm}
\vbox{5
\emph{Toutes les réponses doivent être justifiées par un calcul accompagné d'un raisonnement.}

\begin{wrapfigure}{r}{5.0cm}
   \vspace{-0.5cm}        % à adapter.
   \centering
   \input{Fig_OKTXHoc.pstricks}
\end{wrapfigure}

    La figure ci-contre est un cube de \unit{4}{\centi\meter}.

    \begin{enumerate}
    \item
    Quelle est la nature du triangle \(GFC\) ? 
    \item
    Quel est son périmètre ?
    \item
    Quelle est son aire ? (pour les rapides)
    \end{enumerate}
    

}
\vspace{2cm}
\vbox{6
\emph{Toutes les réponses doivent être justifiées par un calcul accompagné d'un raisonnement.}

\begin{wrapfigure}{r}{5.0cm}
   \vspace{-0.5cm}        % à adapter.
   \centering
   \input{Fig_OKTXHoc.pstricks}
\end{wrapfigure}

    La figure ci-contre est un cube de \unit{3}{\centi\meter}.

    \begin{enumerate}
    \item
    Quelle est la nature du triangle \(CAD\) ? 
    \item
    Quel est son périmètre ?
    \item
    Quelle est son aire ? (pour les rapides)
    \end{enumerate}
    

}
\vspace{2cm}
\vbox{7
\emph{Toutes les réponses doivent être justifiées par un calcul accompagné d'un raisonnement.}

\begin{wrapfigure}{r}{5.0cm}
   \vspace{-0.5cm}        % à adapter.
   \centering
   \input{Fig_OKTXHoc.pstricks}
\end{wrapfigure}

    La figure ci-contre est un cube de \unit{4}{\centi\meter}.

    \begin{enumerate}
    \item
    Quelle est la nature du triangle \(EAH\) ? 
    \item
    Quel est son périmètre ?
    \item
    Quelle est son aire ? (pour les rapides)
    \end{enumerate}
    

}
\vspace{2cm}
\vbox{8
\emph{Toutes les réponses doivent être justifiées par un calcul accompagné d'un raisonnement.}

\begin{wrapfigure}{r}{5.0cm}
   \vspace{-0.5cm}        % à adapter.
   \centering
   \input{Fig_OKTXHoc.pstricks}
\end{wrapfigure}

    La figure ci-contre est un cube de \unit{7}{\centi\meter}.

    \begin{enumerate}
    \item
    Quelle est la nature du triangle \(FBD\) ? 
    \item
    Quel est son périmètre ?
    \item
    Quelle est son aire ? (pour les rapides)
    \end{enumerate}
    

}
\vspace{2cm}
\vbox{9
\emph{Toutes les réponses doivent être justifiées par un calcul accompagné d'un raisonnement.}

\begin{wrapfigure}{r}{5.0cm}
   \vspace{-0.5cm}        % à adapter.
   \centering
   \input{Fig_OKTXHoc.pstricks}
\end{wrapfigure}

    La figure ci-contre est un cube de \unit{7}{\centi\meter}.

    \begin{enumerate}
    \item
    Quelle est la nature du triangle \(ACG\) ? 
    \item
    Quel est son périmètre ?
    \item
    Quelle est son aire ? (pour les rapides)
    \end{enumerate}
    

}
\vspace{2cm}
\vbox{10
\emph{Toutes les réponses doivent être justifiées par un calcul accompagné d'un raisonnement.}

\begin{wrapfigure}{r}{5.0cm}
   \vspace{-0.5cm}        % à adapter.
   \centering
   \input{Fig_OKTXHoc.pstricks}
\end{wrapfigure}

    La figure ci-contre est un cube de \unit{4}{\centi\meter}.

    \begin{enumerate}
    \item
    Quelle est la nature du triangle \(ECB\) ? 
    \item
    Quel est son périmètre ?
    \item
    Quelle est son aire ? (pour les rapides)
    \end{enumerate}
    

}
\vspace{2cm}
\vbox{11
\emph{Toutes les réponses doivent être justifiées par un calcul accompagné d'un raisonnement.}

\begin{wrapfigure}{r}{5.0cm}
   \vspace{-0.5cm}        % à adapter.
   \centering
   \input{Fig_OKTXHoc.pstricks}
\end{wrapfigure}

    La figure ci-contre est un cube de \unit{4}{\centi\meter}.

    \begin{enumerate}
    \item
    Quelle est la nature du triangle \(BFD\) ? 
    \item
    Quel est son périmètre ?
    \item
    Quelle est son aire ? (pour les rapides)
    \end{enumerate}
    

}
\vspace{2cm}
\vbox{12
\emph{Toutes les réponses doivent être justifiées par un calcul accompagné d'un raisonnement.}

\begin{wrapfigure}{r}{5.0cm}
   \vspace{-0.5cm}        % à adapter.
   \centering
   \input{Fig_OKTXHoc.pstricks}
\end{wrapfigure}

    La figure ci-contre est un cube de \unit{3}{\centi\meter}.

    \begin{enumerate}
    \item
    Quelle est la nature du triangle \(BEG\) ? 
    \item
    Quel est son périmètre ?
    \item
    Quelle est son aire ? (pour les rapides)
    \end{enumerate}
    

}
\vspace{2cm}
\vbox{13
\emph{Toutes les réponses doivent être justifiées par un calcul accompagné d'un raisonnement.}

\begin{wrapfigure}{r}{5.0cm}
   \vspace{-0.5cm}        % à adapter.
   \centering
   \input{Fig_OKTXHoc.pstricks}
\end{wrapfigure}

    La figure ci-contre est un cube de \unit{4}{\centi\meter}.

    \begin{enumerate}
    \item
    Quelle est la nature du triangle \(BAE\) ? 
    \item
    Quel est son périmètre ?
    \item
    Quelle est son aire ? (pour les rapides)
    \end{enumerate}
    

}
\vspace{2cm}
\vbox{14
\emph{Toutes les réponses doivent être justifiées par un calcul accompagné d'un raisonnement.}

\begin{wrapfigure}{r}{5.0cm}
   \vspace{-0.5cm}        % à adapter.
   \centering
   \input{Fig_OKTXHoc.pstricks}
\end{wrapfigure}

    La figure ci-contre est un cube de \unit{4}{\centi\meter}.

    \begin{enumerate}
    \item
    Quelle est la nature du triangle \(FCE\) ? 
    \item
    Quel est son périmètre ?
    \item
    Quelle est son aire ? (pour les rapides)
    \end{enumerate}
    

}
\vspace{2cm}
\vbox{15
\emph{Toutes les réponses doivent être justifiées par un calcul accompagné d'un raisonnement.}

\begin{wrapfigure}{r}{5.0cm}
   \vspace{-0.5cm}        % à adapter.
   \centering
   \input{Fig_OKTXHoc.pstricks}
\end{wrapfigure}

    La figure ci-contre est un cube de \unit{2}{\centi\meter}.

    \begin{enumerate}
    \item
    Quelle est la nature du triangle \(GBD\) ? 
    \item
    Quel est son périmètre ?
    \item
    Quelle est son aire ? (pour les rapides)
    \end{enumerate}
    

}
\vspace{2cm}
\vbox{16
\emph{Toutes les réponses doivent être justifiées par un calcul accompagné d'un raisonnement.}

\begin{wrapfigure}{r}{5.0cm}
   \vspace{-0.5cm}        % à adapter.
   \centering
   \input{Fig_OKTXHoc.pstricks}
\end{wrapfigure}

    La figure ci-contre est un cube de \unit{5}{\centi\meter}.

    \begin{enumerate}
    \item
    Quelle est la nature du triangle \(EHA\) ? 
    \item
    Quel est son périmètre ?
    \item
    Quelle est son aire ? (pour les rapides)
    \end{enumerate}
    

}
\vspace{2cm}
\vbox{17
\emph{Toutes les réponses doivent être justifiées par un calcul accompagné d'un raisonnement.}

\begin{wrapfigure}{r}{5.0cm}
   \vspace{-0.5cm}        % à adapter.
   \centering
   \input{Fig_OKTXHoc.pstricks}
\end{wrapfigure}

    La figure ci-contre est un cube de \unit{6}{\centi\meter}.

    \begin{enumerate}
    \item
    Quelle est la nature du triangle \(CFD\) ? 
    \item
    Quel est son périmètre ?
    \item
    Quelle est son aire ? (pour les rapides)
    \end{enumerate}
    

}
\vspace{2cm}
\vbox{18
\emph{Toutes les réponses doivent être justifiées par un calcul accompagné d'un raisonnement.}

\begin{wrapfigure}{r}{5.0cm}
   \vspace{-0.5cm}        % à adapter.
   \centering
   \input{Fig_OKTXHoc.pstricks}
\end{wrapfigure}

    La figure ci-contre est un cube de \unit{6}{\centi\meter}.

    \begin{enumerate}
    \item
    Quelle est la nature du triangle \(CFB\) ? 
    \item
    Quel est son périmètre ?
    \item
    Quelle est son aire ? (pour les rapides)
    \end{enumerate}
    

}
\vspace{2cm}
\vbox{19
\emph{Toutes les réponses doivent être justifiées par un calcul accompagné d'un raisonnement.}

\begin{wrapfigure}{r}{5.0cm}
   \vspace{-0.5cm}        % à adapter.
   \centering
   \input{Fig_OKTXHoc.pstricks}
\end{wrapfigure}

    La figure ci-contre est un cube de \unit{7}{\centi\meter}.

    \begin{enumerate}
    \item
    Quelle est la nature du triangle \(DAC\) ? 
    \item
    Quel est son périmètre ?
    \item
    Quelle est son aire ? (pour les rapides)
    \end{enumerate}
    

}
\vspace{2cm}
\vbox{20
\emph{Toutes les réponses doivent être justifiées par un calcul accompagné d'un raisonnement.}

\begin{wrapfigure}{r}{5.0cm}
   \vspace{-0.5cm}        % à adapter.
   \centering
   \input{Fig_OKTXHoc.pstricks}
\end{wrapfigure}

    La figure ci-contre est un cube de \unit{4}{\centi\meter}.

    \begin{enumerate}
    \item
    Quelle est la nature du triangle \(FEC\) ? 
    \item
    Quel est son périmètre ?
    \item
    Quelle est son aire ? (pour les rapides)
    \end{enumerate}
    

}
\vspace{2cm}
\vbox{21
\emph{Toutes les réponses doivent être justifiées par un calcul accompagné d'un raisonnement.}

\begin{wrapfigure}{r}{5.0cm}
   \vspace{-0.5cm}        % à adapter.
   \centering
   \input{Fig_OKTXHoc.pstricks}
\end{wrapfigure}

    La figure ci-contre est un cube de \unit{3}{\centi\meter}.

    \begin{enumerate}
    \item
    Quelle est la nature du triangle \(GCF\) ? 
    \item
    Quel est son périmètre ?
    \item
    Quelle est son aire ? (pour les rapides)
    \end{enumerate}
    

}
\vspace{2cm}
\vbox{22
\emph{Toutes les réponses doivent être justifiées par un calcul accompagné d'un raisonnement.}

\begin{wrapfigure}{r}{5.0cm}
   \vspace{-0.5cm}        % à adapter.
   \centering
   \input{Fig_OKTXHoc.pstricks}
\end{wrapfigure}

    La figure ci-contre est un cube de \unit{3}{\centi\meter}.

    \begin{enumerate}
    \item
    Quelle est la nature du triangle \(ADG\) ? 
    \item
    Quel est son périmètre ?
    \item
    Quelle est son aire ? (pour les rapides)
    \end{enumerate}
    

}
\vspace{2cm}
\vbox{23
\emph{Toutes les réponses doivent être justifiées par un calcul accompagné d'un raisonnement.}

\begin{wrapfigure}{r}{5.0cm}
   \vspace{-0.5cm}        % à adapter.
   \centering
   \input{Fig_OKTXHoc.pstricks}
\end{wrapfigure}

    La figure ci-contre est un cube de \unit{3}{\centi\meter}.

    \begin{enumerate}
    \item
    Quelle est la nature du triangle \(CHF\) ? 
    \item
    Quel est son périmètre ?
    \item
    Quelle est son aire ? (pour les rapides)
    \end{enumerate}
    

}
\vspace{2cm}
\vbox{24
\emph{Toutes les réponses doivent être justifiées par un calcul accompagné d'un raisonnement.}

\begin{wrapfigure}{r}{5.0cm}
   \vspace{-0.5cm}        % à adapter.
   \centering
   \input{Fig_OKTXHoc.pstricks}
\end{wrapfigure}

    La figure ci-contre est un cube de \unit{7}{\centi\meter}.

    \begin{enumerate}
    \item
    Quelle est la nature du triangle \(DBF\) ? 
    \item
    Quel est son périmètre ?
    \item
    Quelle est son aire ? (pour les rapides)
    \end{enumerate}
    

}
\vspace{2cm}
\vbox{25
\emph{Toutes les réponses doivent être justifiées par un calcul accompagné d'un raisonnement.}

\begin{wrapfigure}{r}{5.0cm}
   \vspace{-0.5cm}        % à adapter.
   \centering
   \input{Fig_OKTXHoc.pstricks}
\end{wrapfigure}

    La figure ci-contre est un cube de \unit{6}{\centi\meter}.

    \begin{enumerate}
    \item
    Quelle est la nature du triangle \(HDC\) ? 
    \item
    Quel est son périmètre ?
    \item
    Quelle est son aire ? (pour les rapides)
    \end{enumerate}
    

}
\vspace{2cm}
\vbox{26
\emph{Toutes les réponses doivent être justifiées par un calcul accompagné d'un raisonnement.}

\begin{wrapfigure}{r}{5.0cm}
   \vspace{-0.5cm}        % à adapter.
   \centering
   \input{Fig_OKTXHoc.pstricks}
\end{wrapfigure}

    La figure ci-contre est un cube de \unit{4}{\centi\meter}.

    \begin{enumerate}
    \item
    Quelle est la nature du triangle \(HCA\) ? 
    \item
    Quel est son périmètre ?
    \item
    Quelle est son aire ? (pour les rapides)
    \end{enumerate}
    

}
\vspace{2cm}
\vbox{27
\emph{Toutes les réponses doivent être justifiées par un calcul accompagné d'un raisonnement.}

\begin{wrapfigure}{r}{5.0cm}
   \vspace{-0.5cm}        % à adapter.
   \centering
   \input{Fig_OKTXHoc.pstricks}
\end{wrapfigure}

    La figure ci-contre est un cube de \unit{6}{\centi\meter}.

    \begin{enumerate}
    \item
    Quelle est la nature du triangle \(BEA\) ? 
    \item
    Quel est son périmètre ?
    \item
    Quelle est son aire ? (pour les rapides)
    \end{enumerate}
    

}
\vspace{2cm}
\vbox{28
\emph{Toutes les réponses doivent être justifiées par un calcul accompagné d'un raisonnement.}

\begin{wrapfigure}{r}{5.0cm}
   \vspace{-0.5cm}        % à adapter.
   \centering
   \input{Fig_OKTXHoc.pstricks}
\end{wrapfigure}

    La figure ci-contre est un cube de \unit{2}{\centi\meter}.

    \begin{enumerate}
    \item
    Quelle est la nature du triangle \(EAD\) ? 
    \item
    Quel est son périmètre ?
    \item
    Quelle est son aire ? (pour les rapides)
    \end{enumerate}
    

}
\vspace{2cm}
\vbox{29
\emph{Toutes les réponses doivent être justifiées par un calcul accompagné d'un raisonnement.}

\begin{wrapfigure}{r}{5.0cm}
   \vspace{-0.5cm}        % à adapter.
   \centering
   \input{Fig_OKTXHoc.pstricks}
\end{wrapfigure}

    La figure ci-contre est un cube de \unit{2}{\centi\meter}.

    \begin{enumerate}
    \item
    Quelle est la nature du triangle \(GFA\) ? 
    \item
    Quel est son périmètre ?
    \item
    Quelle est son aire ? (pour les rapides)
    \end{enumerate}
    

}
\vspace{2cm}
\vbox{30
\emph{Toutes les réponses doivent être justifiées par un calcul accompagné d'un raisonnement.}

\begin{wrapfigure}{r}{5.0cm}
   \vspace{-0.5cm}        % à adapter.
   \centering
   \input{Fig_OKTXHoc.pstricks}
\end{wrapfigure}

    La figure ci-contre est un cube de \unit{4}{\centi\meter}.

    \begin{enumerate}
    \item
    Quelle est la nature du triangle \(GDF\) ? 
    \item
    Quel est son périmètre ?
    \item
    Quelle est son aire ? (pour les rapides)
    \end{enumerate}
    

}
\vspace{2cm}
\vbox{31
\emph{Toutes les réponses doivent être justifiées par un calcul accompagné d'un raisonnement.}

\begin{wrapfigure}{r}{5.0cm}
   \vspace{-0.5cm}        % à adapter.
   \centering
   \input{Fig_OKTXHoc.pstricks}
\end{wrapfigure}

    La figure ci-contre est un cube de \unit{4}{\centi\meter}.

    \begin{enumerate}
    \item
    Quelle est la nature du triangle \(BDE\) ? 
    \item
    Quel est son périmètre ?
    \item
    Quelle est son aire ? (pour les rapides)
    \end{enumerate}
    

}
\vspace{2cm}
\vbox{32
\emph{Toutes les réponses doivent être justifiées par un calcul accompagné d'un raisonnement.}

\begin{wrapfigure}{r}{5.0cm}
   \vspace{-0.5cm}        % à adapter.
   \centering
   \input{Fig_OKTXHoc.pstricks}
\end{wrapfigure}

    La figure ci-contre est un cube de \unit{5}{\centi\meter}.

    \begin{enumerate}
    \item
    Quelle est la nature du triangle \(CDB\) ? 
    \item
    Quel est son périmètre ?
    \item
    Quelle est son aire ? (pour les rapides)
    \end{enumerate}
    

}
\vspace{2cm}
\vbox{33
\emph{Toutes les réponses doivent être justifiées par un calcul accompagné d'un raisonnement.}

\begin{wrapfigure}{r}{5.0cm}
   \vspace{-0.5cm}        % à adapter.
   \centering
   \input{Fig_OKTXHoc.pstricks}
\end{wrapfigure}

    La figure ci-contre est un cube de \unit{3}{\centi\meter}.

    \begin{enumerate}
    \item
    Quelle est la nature du triangle \(HBF\) ? 
    \item
    Quel est son périmètre ?
    \item
    Quelle est son aire ? (pour les rapides)
    \end{enumerate}
    

}
\vspace{2cm}
\vbox{34
\emph{Toutes les réponses doivent être justifiées par un calcul accompagné d'un raisonnement.}

\begin{wrapfigure}{r}{5.0cm}
   \vspace{-0.5cm}        % à adapter.
   \centering
   \input{Fig_OKTXHoc.pstricks}
\end{wrapfigure}

    La figure ci-contre est un cube de \unit{7}{\centi\meter}.

    \begin{enumerate}
    \item
    Quelle est la nature du triangle \(BAG\) ? 
    \item
    Quel est son périmètre ?
    \item
    Quelle est son aire ? (pour les rapides)
    \end{enumerate}
    

}
\vspace{2cm}
\vbox{35
\emph{Toutes les réponses doivent être justifiées par un calcul accompagné d'un raisonnement.}

\begin{wrapfigure}{r}{5.0cm}
   \vspace{-0.5cm}        % à adapter.
   \centering
   \input{Fig_OKTXHoc.pstricks}
\end{wrapfigure}

    La figure ci-contre est un cube de \unit{2}{\centi\meter}.

    \begin{enumerate}
    \item
    Quelle est la nature du triangle \(BHE\) ? 
    \item
    Quel est son périmètre ?
    \item
    Quelle est son aire ? (pour les rapides)
    \end{enumerate}
    

}
\vspace{2cm}
\vbox{36
\emph{Toutes les réponses doivent être justifiées par un calcul accompagné d'un raisonnement.}

\begin{wrapfigure}{r}{5.0cm}
   \vspace{-0.5cm}        % à adapter.
   \centering
   \input{Fig_OKTXHoc.pstricks}
\end{wrapfigure}

    La figure ci-contre est un cube de \unit{7}{\centi\meter}.

    \begin{enumerate}
    \item
    Quelle est la nature du triangle \(CEG\) ? 
    \item
    Quel est son périmètre ?
    \item
    Quelle est son aire ? (pour les rapides)
    \end{enumerate}
    

}
\vspace{2cm}
\vbox{37
\emph{Toutes les réponses doivent être justifiées par un calcul accompagné d'un raisonnement.}

\begin{wrapfigure}{r}{5.0cm}
   \vspace{-0.5cm}        % à adapter.
   \centering
   \input{Fig_OKTXHoc.pstricks}
\end{wrapfigure}

    La figure ci-contre est un cube de \unit{3}{\centi\meter}.

    \begin{enumerate}
    \item
    Quelle est la nature du triangle \(ADG\) ? 
    \item
    Quel est son périmètre ?
    \item
    Quelle est son aire ? (pour les rapides)
    \end{enumerate}
    

}
\vspace{2cm}
\vbox{38
\emph{Toutes les réponses doivent être justifiées par un calcul accompagné d'un raisonnement.}

\begin{wrapfigure}{r}{5.0cm}
   \vspace{-0.5cm}        % à adapter.
   \centering
   \input{Fig_OKTXHoc.pstricks}
\end{wrapfigure}

    La figure ci-contre est un cube de \unit{6}{\centi\meter}.

    \begin{enumerate}
    \item
    Quelle est la nature du triangle \(GHD\) ? 
    \item
    Quel est son périmètre ?
    \item
    Quelle est son aire ? (pour les rapides)
    \end{enumerate}
    

}
\vspace{2cm}
\vbox{39
\emph{Toutes les réponses doivent être justifiées par un calcul accompagné d'un raisonnement.}

\begin{wrapfigure}{r}{5.0cm}
   \vspace{-0.5cm}        % à adapter.
   \centering
   \input{Fig_OKTXHoc.pstricks}
\end{wrapfigure}

    La figure ci-contre est un cube de \unit{6}{\centi\meter}.

    \begin{enumerate}
    \item
    Quelle est la nature du triangle \(BCE\) ? 
    \item
    Quel est son périmètre ?
    \item
    Quelle est son aire ? (pour les rapides)
    \end{enumerate}
    

}
\vspace{2cm}
