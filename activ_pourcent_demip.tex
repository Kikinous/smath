% This is part of Un soupçon de mathématique sans être agressif pour autant
% Copyright (c) 2015
%   Laurent Claessens
% See the file fdl-1.3.txt for copying conditions.

Une ville possède deux collèges. Dans le premier, il y a $350$ élèves et $40\%$ d'entre eux sont des demi-pensionnaires.  Dans le deuxième, il y a $620$ élèves dont $124$ demi-pensionnaires.  
\begin{enumerate}
    \item
        Compléter le tableau suivant :
        \begin{equation*}
            \begin{array}[]{|c||c|c|c|}
                \hline
                &\text{nombre de demi-pensionnaires}&\text{pourcentage de demi-pensionnaires}&\text{total}\\
                \hline
                \text{premier collège}&&&\\
                \hline
                \text{deuxième collège}&&&\\
                \hline
                \text{total}&&&\\
                \hline
            \end{array}
        \end{equation*}
\end{enumerate}
Dans les deux établissements réunis, quel est le pourcentage de demi-pensionnaires ?  Quelle remarque peux-tu faire ?
