% This is part of Un soupçon de mathématique sans être agressif pour autant
% Copyright (c) 2014
%   Laurent Claessens
% See the file fdl-1.3.txt for copying conditions.

\begin{exercice}\label{exosmath-0596}

    Soient les points \( O=(0;0)\), \( A=(1;-2)\), \( B=(5;5)\), et \( C=(-3;0)\).
    \begin{enumerate}
        \item
            Donner les coordonnées des vecteurs \( \vect{ u }=\vect{ OB }\), \( \vect{ v }=\vect{ AC }\) et \( \vect{ w }=\vect{ AC }\).
        \item
            Donner les coordonnées des vecteurs \( \vect{ u }+\vect{ v }\), \( \vect{ u }+\vect{ w }\) et \( \vect{ u }+\vect{ v }+\vect{ w }\).
    \end{enumerate}

    Faire l'exercice 27 de la page 213 et 82, page 217.

\corrref{smath-0596}
\end{exercice}
