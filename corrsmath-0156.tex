% This is part of Un soupçon de mathématique sans être agressif pour autant
% Copyright (c) 2012-2013
%   Laurent Claessens
% See the file fdl-1.3.txt for copying conditions.

\begin{corrige}{smath-0156}

Le point \( B\) est aux coordonnées \( (4;3)\), et \( K=(2;3)\). Pour déterminer les points \( E\) et \( F\) nous pouvons calculer les équations des droites en présence et calculer les intersections. Nous avons
\begin{subequations}
    \begin{align}
        (AC)&\equiv y=-\frac{ 3 }{ 4 }x+3\\
        (OK)&\equiv y=\frac{ 3 }{2}x\\
        (OB)&\equiv y=\frac{ 3 }{ 4 }x\\
        (KA)&\equiv y=-\frac{ 3 }{2}x+6.
    \end{align}
\end{subequations}
Les points d'intersection sont 
\begin{equation}
    E=(AC)\cap (OK)=\big( \frac{ 4 }{ 3 };2 \big)
\end{equation}
et
\begin{equation}
    F=(OB)\cap (KA)=\big( \frac{ 8 }{ 3 };2 \big)
\end{equation}

La droite \( (EF) \) est donc horizontale et donc parallèle à \( (CB)\). En tant que vecteur, nous avons
\begin{equation}
    \vect{ EF }=\begin{pmatrix}
        \frac{ 8 }{ 3 }-\frac{ 4 }{ 3 }    \\ 
        2-2    
    \end{pmatrix}=\begin{pmatrix}
        \frac{ 4 }{ 3 }    \\ 
        0    
    \end{pmatrix}.
\end{equation}
En même temps nous avons
\begin{equation}
    \vect{ BC }=\begin{pmatrix}
        0-4    \\ 
        3-3    
    \end{pmatrix}=\begin{pmatrix}
        -4    \\ 
        0    
    \end{pmatrix}
\end{equation}
et nous avons donc bien \( \vect{ BC }=-3\vect{ EF }\).


\end{corrige}
