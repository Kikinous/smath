% This is part of Un soupçon de mathématique sans être agressif pour autant
% Copyright (c) 2012-2013
%   Laurent Claessens
% See the file fdl-1.3.txt for copying conditions.

\begin{corrige}{smath-0156}

Le point \( B\) est aux coordonnées \( (4;3)\). Le point \( K\) se calcule comme le milieu :
\begin{equation}
    K=\left( \frac{ 0+4 }{2};\frac{ 3+3 }{2} \right)=(2;3).
\end{equation}


Pour déterminer les coordonnées du point \( E\), nous allons suivre les étapes suivantes :
\begin{itemize}
    \item Trouver les équations des droites \( (OK)\) et \( CA\)
    \item Calculer leur intersection.
\end{itemize}

\begin{description}
    \item[L'équation de la droite $CA$]
        Un vecteur directeur est \( \vect{ CA }=\begin{pmatrix}
            4    \\ 
            -3    
        \end{pmatrix}\), donc le coefficient directeur est \( a=\frac{ -3 }{ 4 }\). La droite \( (CA)\) a pour équation \( y=-\frac{ 3 }{ 4 }x+b\) où \( b\) est encore inconnu. Pour le trouver, nous savons que la droite passe par les points \( (4;0)\) et \( (0;3)\), donc
        \begin{subequations}
            \begin{numcases}{}
                0=-\frac{ 3 }{ 4 }\times 4+b\\
                3=-\frac{ 3 }{ 4 }\times 0+b,
            \end{numcases}
        \end{subequations}
        dont nous tirons que \( b=3\). Au final, la droite \( (CA)\) a pour équation
        \begin{equation}
            y=-\frac{ 3 }{ 4 }x+3.
        \end{equation}
    \item[L'équation de la droite \( KO\)] Étant donné que cette droite passe par l'origine, nous savons qu'elle est le graphe d'une fonction linéaire (et non seulement affine), c'est à dire que nous savons d'emblée que \( b=0\). La vecteur directeur que nous trouvons est \( \vect{ OK }=\begin{pmatrix}
            2    \\ 
            3    
        \end{pmatrix}\) et donc le coefficient directeur est \( a=\frac{ 3 }{2}\). Donc la droite \( (OK)\) a pour équation
        \begin{equation}
            y=\frac{ 3 }{2}x.
        \end{equation}
    \item[Calcul de l'intersection]
        Nous savons que le point \( E\) est sur les deux droites en même temps, donc ses coordonnées \( (x;y)\) vérifient le système
        \begin{subequations}
            \begin{numcases}{}
                y=-\frac{ 3 }{ 4 }x+3\\
                y=\frac{ 3 }{ 2 }x,
            \end{numcases}
        \end{subequations}
        donc l'équation pour \( x\) est
        \begin{equation}
            =\frac{ 3 }{ 4 }x+3=\frac{ 3 }{2}x,
        \end{equation}
        donc nous tirons \( x=\frac{ 4 }{ 3 }\). Pour trouver le \( y\), nous récrivons les équations en remplaçant \( x\) par la valeur que nous venons de trouver :
        \begin{subequations}
            \begin{numcases}{}
                y=-\frac{ 3 }{ 4 }\times \frac{ 4 }{ 3 }+3=2\\
                y=\frac{ 3 }{ 2 }\times\frac{ 4 }{ 3 }=2.
            \end{numcases}
        \end{subequations}
        Au final nous avons 
        \begin{equation}
            E=(\frac{ 4 }{ 3 };2).
        \end{equation}

\end{description}

En ce qui concerne le point \( F\), nous suivons les mêmes étapes :
\begin{itemize}
    \item Trouver les équations des droites \( OB\) et \( KA\).
    \item Calculer les coordonnées de leur point d'intersection.
\end{itemize}
Nous ne donnons pas ici tous les détails (mais c'est un exercice à faire). Les résultats sont :
\begin{subequations}
    \begin{align}
        (AC)&\equiv y=-\frac{ 3 }{ 4 }x+3\\
        (OK)&\equiv y=\frac{ 3 }{2}x\\
        (OB)&\equiv y=\frac{ 3 }{ 4 }x\\
        (KA)&\equiv y=-\frac{ 3 }{2}x+6.
    \end{align}
\end{subequations}
Les points d'intersection sont 
\begin{equation}
    E=(AC)\cap (OK)=\big( \frac{ 4 }{ 3 };2 \big)
\end{equation}
et
\begin{equation}
    F=(OB)\cap (KA)=\big( \frac{ 8 }{ 3 };2 \big)
\end{equation}

En ce qui concerne les vecteurs :
\begin{equation}
    \vect{ EF }=\begin{pmatrix}
        \frac{ 8 }{ 3 }-\frac{ 4 }{ 3 }    \\ 
        2-2    
    \end{pmatrix}=\begin{pmatrix}
        \frac{ 4 }{ 3 }    \\ 
        0    
    \end{pmatrix}.
\end{equation}
En même temps nous avons
\begin{equation}
    \vect{ BC }=\begin{pmatrix}
        0-4    \\ 
        3-3    
    \end{pmatrix}=\begin{pmatrix}
        -4    \\ 
        0    
    \end{pmatrix}
\end{equation}
et nous avons donc bien \( \vect{ BC }=-3\vect{ EF }\).


\end{corrige}
