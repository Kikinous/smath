% This is part of Un soupçon de mathématique sans être agressif pour autant
% Copyright (c) 2015
%   Laurent Claessens
% See the file fdl-1.3.txt for copying conditions.


\begin{rituel}
    Le tableau suivant donne le périmètre d'un carré en fonction de la longueur de son côté :
    \begin{equation*}
        \begin{array}[]{|c||c|c|c|c|c|}
            \hline
            \text{côté}&0&4&5&10&\ldots\\
            \hline
            \text{périmètre}&\ldots&16&\ldots&\ldots&52\\
            \hline
        \end{array}
    \end{equation*}
    \begin{enumerate}
        \item
            Compléter ce tableau.
        \item
            En tracer une représentation graphique.
        \item
            Est-ce un tableau de proportionnalité ?
    \end{enumerate}
    Le tableau suivant donne l'aire d'un carré en fonction de la longueur de son côté :
    \begin{equation*}
        \begin{array}[]{|c||c|c|c|c|c|}
            \hline
            \text{côté}&0&2&3&5&\ldots\\
            \hline\hline
            \text{aire}&\ldots&4&\ldots&\ldots&49\\
            \hline
        \end{array}
    \end{equation*}
    \begin{enumerate}
        \item
            Compléter ce tableau.
        \item
            En tracer une représentation graphique.
        \item
            Est-ce un tableau de proportionnalité ?
    \end{enumerate}
\end{rituel}
