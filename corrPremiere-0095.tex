% This is part of Un soupçon de mathématique sans être agressif pour autant
% Copyright (c) 2012
%   Laurent Claessens
% See the file fdl-1.3.txt for copying conditions.

\begin{corrige}{Premiere-0095}

    Les paramètres de \( x^2+4x-5\) sont \( a=1\), \( b=4\) et \( c=-5\).
    \begin{enumerate}
        \item
            Le sommet est en \( x_0=-\frac{ b }{ 2a }=-\frac{ 4 }{ 2 }=-2\). Les coordonnées du point sont donc \( \big( -2;f(-2) \big)=(-2;-9)\).
        \item
            Il suffit de calculer \( f(-1)\), \( f(-5)\), \( f(0)\), \( f(1)\) et \( f(7)\). Les calculs sont
    \end{enumerate}
    <++>

\end{corrige}
