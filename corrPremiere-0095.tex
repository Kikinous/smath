% This is part of Un soupçon de mathématique sans être agressif pour autant
% Copyright (c) 2012
%   Laurent Claessens
% See the file fdl-1.3.txt for copying conditions.

\begin{corrige}{Premiere-0095}

    Les paramètres de \( x^2+4x-5\) sont \( a=1\), \( b=4\) et \( c=-5\).
    \begin{enumerate}
        \item
            Le sommet est en \( x_0=-\frac{ b }{ 2a }=-\frac{ 4 }{ 2 }=-2\). Les coordonnées du point sont donc \( \big( -2;f(-2) \big)=(-2;-9)\).
        \item
            Les calculs sont :
            \begin{multicols}{2}

                \begin{enumerate}
                    \item
                        $f(-1)=(-1)^2+4\times (-1)-5=-8$
                    \item
                        $f(-5)=(-5)^2+4\times (-5)-5=25-20-5=0$
                    \item
                        \( f(0)=-5\)
                    \item
                        \( f(1)=1^2+4-5=0\)
                    \item
                        \( f(7)=49+4\times 7-5=49+28-5=77-5=72\)
                \end{enumerate}
            \end{multicols}
            Attention à ne pas écrire \( -(A)^2\) lorsqu'il faut écrire \( (-1)^2\). Parmi les nombres proposés, \( -5\) et \( 1\) vérifient \( x^2+4x-5=0\).
        \item
            Nous savons que la factorisation d'un polynôme du second degré se fait avec les racines \( x_1\) et \( x_2\) en écrivant \( (x-x_1)(x-x_2)\). Ici les racines sont connues : ce sont les nombres qui satisfont à \( f(x)=0\), c'est à dire \( 1\) et \( -5\). Donc
            \begin{equation}
                f(x)=(x-1)(x+5).
            \end{equation}
            Attention aux signes.
        \item
            La vérification consiste à effectuer le produit :
            \begin{equation}
                (x-1)(x+5)=x^2+5x-x-5=x^2+4x-5.
            \end{equation}
            Étant donné que nous retrouvons le polynôme de départ, la factorisation est correcte.
    \end{enumerate}

\end{corrige}
