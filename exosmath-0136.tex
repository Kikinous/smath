% This is part of Un soupçon de mathématique sans être agressif pour autant
% Copyright (c) 2012
%   Laurent Claessens
% See the file fdl-1.3.txt for copying conditions.

\begin{exercice}\label{exosmath-0136}

    Un stylo coûte \( 2.5\) euros en comptant la TVA de \( 19.5\%\). Nous voulons savoir le prix du même stylo après une hausse de la TVA de \( 19.5\%\) à \( 20\%\).
    \begin{enumerate}
        \item
            Cécile propose le raisonnement suivant : étant donné que la TVA augmente de \( 0.5\%\), le nouveau prix sera
            \begin{equation}
                2.5\times\frac{ 100.5 }{ 100 }.
            \end{equation}
            Quel est l'idée de Cécile ? Est-elle correcte ?
        \item
            Calculer le prix du stylo 
    \end{enumerate}
    <++>



\corrref{smath-0136}
\end{exercice}
