% This is part of Un soupçon de mathématique sans être agressif pour autant
% Copyright (c) 2014
%   Laurent Claessens
% See the file fdl-1.3.txt for copying conditions.

\begin{exercice}\label{exosmath-0662}

    La trajectoire d'un boulet de canon est donnée par le graphe ci-dessous :

\begin{center}
   \input{Fig_QGaeERu.pstricks}
\end{center}

\begin{enumerate}
    \item
        Déterminer graphiquement pendant combien de temps la hauteur du boulet est-elle supérieure à \unit{85}{\meter} ?
    \item
        En admettant que \( h(x)=-x^2+20x+21\),
        \begin{enumerate}
            \item
                Montrer que \( h(x)-85=-(x-4)(x-16)\).
            \item
                Résoudre sur \( \mathopen[ 0 , 21 \mathclose]\) l'inéquation\\ \( -(x-4)(x-16)\geq 0\).
        \end{enumerate}
\end{enumerate}

\corrref{smath-0662}
\end{exercice}
