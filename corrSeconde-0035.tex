% This is part of Un soupçon de mathématique sans être agressif pour autant
% Copyright (c) 2012
%   Laurent Claessens
% See the file fdl-1.3.txt for copying conditions.

\begin{corrige}{Seconde-0035}

\begin{enumerate}
\item On nous dit que l'effectif de la classe $[17;19[$ est égal à
  2. Or, cette classe est représentée par un rectangle de 4
  carreaux. On en déduit que l'échelle est : 2 carreaux pour un
  effectif de 1 (autrement dit, l'unité d'aire correspond à 2
  carreaux).  On en déduit les effectifs des autres classes.
  \begin{center}
    \begin{tabular}{|c||c|c|c|c|c|c|}
        \hline
      \textbf{Classe} & $[4;10[$ & $[10;11][$ & $[11;15[$ & $[15;17[$ &
      $[17;19[$ & $[19;29[$  \\
          \hline
      \textbf{Aire en carreaux} & 18 & 12 & 24 & 16 & 4 & 10  \\
      \hline
      \textbf{Effectif} & 9 & 6 & 12 & 8 & \textbf{2} & 5  \\
      \hline
    \end{tabular}
  \end{center}
  Remarque  : l'effectif total de la série est égal à 42.

\item La classe modale est la classe qui a le plus grand effectif (autrement dit, celle qui a l'aire la plus grande). D'après le tableau, c'est la classe $[11;15[$.

        Attention : ce n'est pas la classe \( \mathopen[ 10 , 11 [\), malgré qu'elle ait le rectangle le plus haut.
\end{enumerate}


\end{corrige}
