% This is part of Un soupçon de mathématique sans être agressif pour autant
% Copyright (c) 2015
%   Laurent Claessens
% See the file fdl-1.3.txt for copying conditions.

\begin{exercice}[\ldots\ldots/4]\label{exo2smath-0082}

    Soient deux points \( A\) et \( B\); nous nommons \( M\) le milieu du segment \( [AB]\). Nous considérons un point \( I\) (hors de \( [AB]\)) tel que la droite \( (MI)\) soit perpendiculaire à \( (AB)\).

    \begin{enumerate}
        \item
            Tracer un dessin à main levée respectant ces conditions, et y ajouter les codages correspondants.
        \item       
            Si \( AM=\SI{17}{\centi\meter}\), quelle est la longueur de \( MB\) ? Citer une propriété vue au cours.
    \end{enumerate}

\corrref{2smath-0082}
\end{exercice}
