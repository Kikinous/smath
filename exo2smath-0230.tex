% This is part of Un soupçon de mathématique sans être agressif pour autant
% Copyright (c) 2015
%   Laurent Claessens
% See the file fdl-1.3.txt for copying conditions.

\begin{exercice}\label{exo2smath-0230}

    D'abord sur papier, ensuite sur tableur.
    \begin{enumerate}
        \item   \label{ItemWCZYooGAQGZf}
            Pour quelle valeur de \( x\), l'égalité \( 3x+2=17\) est-elle correcte ?
        \item
            Avec un tableur, faire calculer les valeurs de l'expression \( 3x+2\) pour une centaine de valeurs de \( x\) entre \( 0\) et \( 10\). Est-ce que vous pouvez ainsi confirmer la réponse donnée en \ref{ItemWCZYooGAQGZf} ?
        \item
            Faire calculer les valeurs de l'expression \( 16x\) pour une centaine de valeurs entre \( 0\) et \( 10\). Faire de même, dans une colonne séparée, pour l'expression \( 8x+3\).
        \item
            Trouver une valeur de \( x\) pour laquelle \( 16x=8x+3\).
    \end{enumerate}
    Êtes-vous capable de trouver une valeur approximative de la mesure de l'angle \( a\) ?

\begin{center}
    \input{Fig_MXGVooOFDqLq.pstricks}
\end{center}

\corrref{2smath-0230}
\end{exercice}
