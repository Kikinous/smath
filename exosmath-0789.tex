% This is part of Un soupçon de mathématique sans être agressif pour autant
% Copyright (c) 2014
%   Laurent Claessens
% See the file fdl-1.3.txt for copying conditions.

\begin{exercice}[\cite{NRHooXFvgpp4}]\label{exosmath-0789}

    « Si un quadrilatère possède deux angles droits alors c'est un rectangle.» Lesquelles des figures suivantes sont des contre-exemples à cette affirmation (fausse) ? Expliquer vos choix.

% Les figures sont à compiler d'un coup avec NOPooYlqjzW

    \begin{multicols}{2}
    \begin{enumerate}
        \item
            \begin{center}
               \input{Fig_TBYooGTZFxt.pstricks}
            \end{center}
        \item
        \begin{center}
            \input{Fig_SJZooODLIrs.pstricks}
        \end{center}
    \item
\begin{center}
   \input{Fig_KIZooDOXSDH.pstricks}
\end{center}
\item
\begin{center}
   \input{Fig_SQRooAWmMQL.pstricks}
\end{center}
    \end{enumerate}
    \end{multicols}
    
\corrref{smath-0789}
\end{exercice}
