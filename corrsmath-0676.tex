% This is part of Un soupçon de mathématique sans être agressif pour autant
% Copyright (c) 2014
%   Laurent Claessens
% See the file fdl-1.3.txt for copying conditions.

\begin{corrige}{smath-0676}

    \begin{center}
    \input{Fig_SUCoowlFdp.pstricks}
    \end{center}

\begin{enumerate}
    \item
        Pour calculer \( \vect{ OA }+\vect{ BC }\) nous mettons le vecteur \( \vect{ BC }\) «au bout de» \( \vect{ OA }\). C'est la construction en rouge. Le vecteur \( \vect{ v }\) est celui dessiné en bleu.
    \item
        Pour passer de \( E\) à \( F\) on bouge de \( 3\) vers la gauche et de \( 1\) vers le bas. Cela correspond à la translation de vecteur \( \vect{ AO }\).
    \item
        Étant donné que \( \vect{ EF }=\vect{ AO }\), le quadrilatère \( OAEF\) est un parallélogramme; les droite \( (AE)\) et \( (OF)\) sont donc également parallèles.
    \item
        Vérifions si l'angle \( \hat A\) est droit en vérifiant la réciproque du théorème de Pythagore sur le triangle \( OAE\). Les longueurs sont :
        \begin{subequations}
            \begin{align}
                OA^2&=3^2+1^2=10\\
                OE^2&=1^2+2^2=5\\
                AE^2&=2^2+1^2=5.
            \end{align}
        \end{subequations}
        Étant donné que \( OE^2\neq OA^2+AE^2\), l'angle \( \hat B\) n'est pas droit. Notons que l'angle \( \widehat{OEA}\) est droit, mais cela n'a aucun rapport avec le quadrilatère \( OAEF\).
\end{enumerate}

\end{corrige}
