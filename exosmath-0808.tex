% This is part of Un soupçon de mathématique sans être agressif pour autant
% Copyright (c) 2014
%   Laurent Claessens
% See the file fdl-1.3.txt for copying conditions.

\begin{exercice}[\cite{NRHooXFvgpp4}]\label{exosmath-0808}

 Une salle de concert peut contenir $600$ places. Il y a $x$ places assises et les autres sont debout. Les places debout coûtent $15$€ et les places assises $25$€.
 \begin{enumerate}
     \item
 Que représentent les expressions suivantes : $600-x$ ; $25x$ et $15(600-x)$ ?
 \item
 Exprime, en fonction de $x$, la recette totale en euros si toutes les places sont prises.
\item
 Calcule la recette pour la salle pleine contenant \( 200\) places assises.
 \end{enumerate}

\corrref{smath-0808}
\end{exercice}
