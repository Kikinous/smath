% This is part of Un soupçon de mathématique sans être agressif pour autant
% Copyright (c) 2015
%   Laurent Claessens
% See the file fdl-1.3.txt for copying conditions.

\begin{exercice}\label{exo2smath-0110}

Christoffer carreleur veut réaliser une fresque \( 10\times 20\) comme ceci :
\begin{center}
   \input{Fig_CMUTooFyLisx.pstricks}
\end{center}
Il s'agit d'un rectangle de carreaux blancs, dont les quatre coins sont rouges (carreaux grisés) et contenant deux bandes bleues au milieu (zone hachurée).

\begin{enumerate}
    \item
        Combien de carreaux de chaque couleur faut-il ?
    \item
        Sur ce dessin, chacune des bandes bleues a une longueur de \( 4\) carreaux. Christoffer veut savoir combien de carreaux de chaque couleur il aurait besoin si il faisait varier cette longueur. Donner le nombre de carreaux de chaque couleur en fonction de la longueur de la bande bleue.
\end{enumerate}

\corrref{2smath-0110}
\end{exercice}
