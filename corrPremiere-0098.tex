% This is part of Un soupçon de mathématique sans être agressif pour autant
% Copyright (c) 2012
%   Laurent Claessens
% See the file fdl-1.3.txt for copying conditions.

\begin{corrige}{Premiere-0098}

    \begin{enumerate}
        \item
            Connaissant la factorisation, les racines sont immédiates : c'est \( -1\) et \( 7\). Encore une fois, attention au signe.
        \item
            Le tableau de signe est :
            \begin{equation*}
                \begin{array}[]{|c||c|c|c|c|c|}
                    \hline
                    x&&-1&&7&\\
                    \hline\hline
                    x+1&-&0&+&+&+\\
                    \hline
                    x-7&-&-&-&0&+\\
                    \hline\hline
                    (x+1)(x-1)&+&0&-&0&+\\
                    \hline
                \end{array}
            \end{equation*}
        \item
            L'inéquation \( f(x)\geq 0\) consiste à regarder dans le tableau les cases pour lesquelles \( f(x)\) a un signe \( +\) ou un zéro. Les solutions sont donc
            \begin{equation}
                x\in\mathopen] -\infty ; -1 \mathclose]\cup\mathopen[ 7 , +\infty [.
            \end{equation}
            <++>
    \end{enumerate}
    <++>

\end{corrige}
