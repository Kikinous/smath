% This is part of Un soupçon de mathématique sans être agressif pour autant
% Copyright (c) 2012
%   Laurent Claessens
% See the file fdl-1.3.txt for copying conditions.

\begin{exercice}\label{exoPremiere-0062}

    Étudier le signe (tableau de signe) des expressions suivantes.
    \begin{multicols}{3}
        \begin{enumerate}
            \item
                \( f(x)=(x+1)(x+4)\)
            \item
                \( g(x)=-(x+1)(x+4)\)
            \item
                \( h(x)=(3-x)(x+7)\)
            \item
                \( i(x)=x(1-3x)\)
            \item
                \( j(x)=\frac{ 1 }{2}(x+1)(2x-3)\)
            \item
                \( k(x)=(\frac{ x }{ 2 }+1)(x-1)\)
            \item
                \( k(x)=(\frac{ x }{ 3 }+2)(x+6)\)
            \item
                \( l(x)=(x-1)^2\)
            \item
                \( m(x)=(2x+3)(5x-2)\)
        \end{enumerate}
    \end{multicols}
    
    À l'aide du travail fait ci-dessus, résoudre les inéquations
    \begin{multicols}{2}
        \begin{enumerate}
            \item
                \( (x+1)(x+4)\leq 0\)
            \item
                \( x(1-3x)\geq 0\)
            \item
                \( (\frac{ x }{2}+1)(x-1)<0\)
            \item
                \( (x-1)^2>0\)
        \end{enumerate}
    \end{multicols}

\corrref{Premiere-0062}
\end{exercice}
