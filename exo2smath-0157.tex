% This is part of Un soupçon de mathématique sans être agressif pour autant
% Copyright (c) 2015
%   Laurent Claessens
% See the file fdl-1.3.txt for copying conditions.

\begin{exercice}\label{exo2smath-0157}

Une ville possède deux collèges. Dans le premier, il y a $350$ élèves et $40\%$ d'entre eux sont des demi-pensionnaires.  Dans le deuxième, il y a $620$ élèves dont $124$ demi-pensionnaires.  
\begin{enumerate}
    \item
        Compléter le tableau suivant :
        \begin{equation*}
            \begin{array}[]{|c||c|c|}
                \hline
                &\text{nombre de demi-pensionnaires}&\text{pourcentage de demi-pensionnaires}\\
                \hline
                \text{premier collège}&&\\
                \hline
                \text{deuxième collège}&&\\
                \hline
            \end{array}
        \end{equation*}
\item
Quel est le pourcentage de demi-pensionnaires dans les deux établissements réunis ?
\end{enumerate}

\corrref{2smath-0157}
\end{exercice}
