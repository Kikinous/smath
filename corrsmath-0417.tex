% This is part of Un soupçon de mathématique sans être agressif pour autant
% Copyright (c) 2013
%   Laurent Claessens
% See the file fdl-1.3.txt for copying conditions.

\begin{corrige}{smath-0417}

    Un sondage effectué sur \( 500\) personne donnant un résultat de \( 37\%\) à un candidat signifie qu'il y a \( 95\%\) de chances que le \emph{vrai} résultat du candidat soit dans l'intervalle
    \begin{equation}
        \left[ 0.37-\frac{1}{ \sqrt{500} };0.37+\frac{1}{ \sqrt{500} } \right]=[0.325;0.414].
    \end{equation}
    Étant donné que le résultat effectif de K. est de \( 0.36\) et que ce nombre est bien dans l'intervalle, on ne peut pas parler de chance.

    En ce qui concerne le résultat du candidat J. l'intervalle de confiance du sondage est
    \begin{equation}
        \left[ 0.31+\frac{1}{ \sqrt{500} };0.13+\frac{1}{ \sqrt{500} } \right]=[0.265;0.354].
    \end{equation}
    Vu que le résultat de \( 36.5\%\) n'est pas dans cet intervalle, on peut conclure soit que le candidat a eu de la chance, soit que le sondage a été mal fait, soit qu'une promesse de dernière minute lui a rapporté des voix. En tout cas, il y a moins de \( 5\%\) de chances que le résultat soit dû fait fait qu'on ait interrogé les mauvaises personnes.

\end{corrige}
