% This is part of Un soupçon de mathématique sans être agressif pour autant
% Copyright (c) 2013
%   Laurent Claessens
% See the file fdl-1.3.txt for copying conditions.

\begin{exercice}\label{exosmath-0252}

    Faire correspondre à chaque graphique de la figure \ref{LabelFigfigureSIZwqIZ} l'expression correspondante parmi les propositions suivantes :
    \begin{multicols}{2}
        \begin{enumerate}
            \item
                \( h(x)=x+1\)
            \item
                \( i(x)=(x+2)^2\)
            \item
                $f(x)=x^2$
            \item
                \( g(x)=1-x^2\)
        \end{enumerate}
    \end{multicols}

    \newcommand{\CaptionFigfigureSIZwqIZ}{Les graphes de l'exercice \ref{exosmath-0252}.}
    \input{Fig_figureSIZwqIZ.pstricks}

    Expliquer les éléments qui ont permis l'identification.

\corrref{smath-0252}
\end{exercice}
