% This is part of Un soupçon de mathématique sans être agressif pour autant
% Copyright (c) 2013-2014
%   Laurent Claessens
% See the file fdl-1.3.txt for copying conditions.

\begin{exercice}\label{exosmath-0252}

\let\Oldtheenumi\theenumi
\renewcommand{\theenumi}{(\alph{enumi})}
    Faire correspondre à chaque graphique l'expression correspondante :
    \begin{multicols}{2}
        \begin{enumerate}
            \item
                \( f_a(x)=x+1\)
            \item
                \( f_b(x)=(x+2)^2\)
            \item
                $f_c(x)=x^2$
            \item
                \( f_d(x)=1-x^2\)
        \end{enumerate}
    \end{multicols}
\let\theenumi\Oldtheenumi

    \begin{center}
   \input{Fig_figureSIZwqIZ.pstricks}
    \end{center}

    Expliquer les éléments qui ont permis l'identification.

\corrref{smath-0252}
\end{exercice}
