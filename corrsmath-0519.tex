% This is part of Un soupçon de mathématique sans être agressif pour autant
% Copyright (c) 2013
%   Laurent Claessens
% See the file fdl-1.3.txt for copying conditions.

\begin{corrige}{smath-0519}

    \begin{enumerate}
        \item
            Attention : \( x=-1\) n'est pas solution parce que \( -(-1)+2=3\) alors que l'inéquation demandée est stricte. La bonne réponse est que \( x=-2\) est une solution parce que \( -(-2)+2=4\).
        \item
            La fonction \( x\mapsto -2x-7\) est décroissante parce que son coefficient directeur vaut \( -2\) qui est négatif.
        \item
            L'intervalle \( \mathopen] -12 ; 15 \mathclose]\) ne contient pas \( -12\) parce que le crochet du côté de \( -12\) l'exclu. La bonne réponse est que \( \mathopen] -12 ; 15 \mathclose]\) contient \( 0\) parce que \( 0\) est bien entre \( -12\) et \( 15\).
    \end{enumerate}

\end{corrige}
