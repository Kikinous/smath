% This is part of Un soupçon de mathématique sans être agressif pour autant
% Copyright (c) 2012
%   Laurent Claessens
% See the file fdl-1.3.txt for copying conditions.

\begin{example}

\begin{enumerate}
    \item
        Que vaut \( (x+5)(x-2)\) ?
    \item
        Compléter le tableau de signe
        \begin{center}
            \begin{tabular}[h]{|c||c|c|c|c|c|c|c|c|c|}
                \hline
                    \( x\)&\( \ldots\)&-10&\ldots&-5&\ldots&2&\ldots&5&\( \ldots\)\\
                    \hline
                    \( x+5\)& & & & & & & && \\
                    \hline
                    \( x-2\)&&&&&&&&&\\
                    \hline
                    \(x^2+3x-10\)&&&&&&&&&\\
                    \hline
            \end{tabular}
        \end{center}
    \item
        À partir de ce tableau, donner les solutions des équations et inéquations suivantes :
        \begin{subequations}
            \begin{align}
                x^2+3x-10&=0\\
                x^2+3x-10&\leq0\\
                x^2+3x-10&\geq0
            \end{align}
        \end{subequations}
    \item
        Lequel des graphiques de la figure \ref{LabelFigSecond} correspond à la fonction \( y=f(x)=x^2+3x-10\) ?
\newcommand{\CaptionFigSecond}{Laquelle de ces trois courbes est \( x^2+3x-10\) ?}
\input{Fig_Second.pstricks}

\end{enumerate}
    
\end{example}

\begin{example}
    Parmi les nombres \( -2,-1,0,1,2\), lesquels sont racines de \( x^2-x-2\) ?

    Compléter l'équation
    \begin{equation}
        x^2-x-2=(x+???)(x+???).
    \end{equation}
    Note : les points d'interrogation peuvent être des nombres négatifs.
\end{example}

% TODO : enlever ce newpage après avoir imprimé.
\newpage

\begin{example} \label{ExilbWQA}

À propos de la parabole de la figure \ref{LabelFigParabolesfTKFw}. 
\newcommand{\CaptionFigParabolesfTKFw}{La parabole de l'exemple \ref{ExilbWQA}.}
\input{Fig_ParabolesfTKFw.pstricks}
\begin{enumerate}
    \item
        Quel est l'antécédent de \( 1\) par la fonction ? Autrement dit : pour quelle(s) valeur(s) de \( x\) la courbe vaut-elle \( 1\) ?
    \item
        En quel \( x\) se trouve le sommet de la courbe ?
    \item
        Remarquer que le sommet se trouve «au milieu» des deux antécédents de \( 1\).
    \item
        Tracer les deux antécédents de \( 2\) (avec une règle). Est-ce que le sommet est encore au milieu des deux ?
\end{enumerate}
\end{example}
