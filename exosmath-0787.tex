% This is part of Un soupçon de mathématique sans être agressif pour autant
% Copyright (c) 2014
%   Laurent Claessens
% See the file fdl-1.3.txt for copying conditions.

\begin{exercice}\label{exosmath-0787}

    La somme des chiffres du nombre \( 42\) est égale à \( 6\). Le nombre \( 42\) lui-même est un multiple de \( 6\). De même le nombre \( 510\) a la somme de ses chiffres qui est égale à \( 6\), et est divisible en \( 6\). Est-il vrai que tout nombre dont la somme des chiffres est égale à \( 6\) soit un multiple de \( 6\) ?

\corrref{smath-0787}
\end{exercice}
