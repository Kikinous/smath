% This is part of Un soupçon de mathématique sans être agressif pour autant
% Copyright (c) 2012-2013
%   Laurent Claessens
% See the file fdl-1.3.txt for copying conditions.

Ce chapitre contient les choses tapées mais qui sont hors programmes.

%+++++++++++++++++++++++++++++++++++++++++++++++++++++++++++++++++++++++++++++++++++++++++++++++++++++++++++++++++++++++++++
\section{Dessiner en vraie grandeur}
%+++++++++++++++++++++++++++++++++++++++++++++++++++++++++++++++++++++++++++++++++++++++++++++++++++++++++++++++++++++++++++

La perspective cavalière n'est pas parfaite; aucune perspective n'est parfaite. Étant donné que les surfaces sont déformées, nous voudrions être capables de dessiner certaines surfaces sans déformations. C'est ce que nous appelons \emph{dessiner en vraie grandeur}.

\begin{Aretenir}
    Lorsqu'on demande de dessiner une surface en vraie grandeur, voici les mouvements «de base» que vous êtes autorisés à effectuer \emph{dans un plan}.
    \begin{enumerate}
        \item
            Tracer des segments de longueur \emph{entières} !
        \item
            Tracer des cercles dont le centre et un point sont donnés.
        \item
            Tracer les droites dont deux points sont donnés.
        \item
            Considérer les intersections des droites et cercles.
    \end{enumerate}
    À partir ce ces mouvements, il est possible de tracer de nombreuses choses, dont par exemple
    \begin{enumerate}
        \item
            Le milieu du segment \( [AB]\) lorsque les points \( A\) et \( B\) sont donnés.
        \item
            Si la droite \( d\) et le point \( I\) plan sont donnés, la perpendiculaire à \( d\) passant par \( I\).
        \item
            Si les points \( d\) du plan et le point \( I\) sont donnés, la parallèle à \( d\) passant par \( I\).
    \end{enumerate}
\end{Aretenir}

\begin{proof}
    Nous allons donner les constructions.
    \begin{enumerate}
        \item
            Soient \( A\) et \( B\) données. À construire : la médiatrice du segment \( [AB]\). Avec cela, nous connaîtrons le milieu. Le technique est simple : il suffit de tracer deux cercles de même rayons \( r\) centrés en \( A\) et en \( B\). Les points d'intersection \( I\) et \( J\) donnent la médiatrice. En effet, par construction \( I\) est équidistant de \( A\) et \( B\) parce que \( I\) est à la fois sur le cercle de rayon \( r\) autour de \( A\) et sur le cercle de rayon \( r\) autour de \( B\). Or nous savons que les points équidistants de \( A\) et \( B\) sont sur la médiatrice du segment \( [AB]\).

            Pour la même raison, le point \( J\) est également sur la médiatrice. Cela fait que \( (IJ)\) coupe \( [AB]\) en son milieu.

Cette construction est illustrée sur la figure \ref{LabelFigIsoceleVdviOE}. % From file IsoceleVdviOE
\newcommand{\CaptionFigIsoceleVdviOE}{Le triangle \( AJB\) est isocèle.}
\input{Fig_IsoceleVdviOE.pstricks}

        \item

            Soient donnés la droite \( d\) et le point \( I\). Nous construisons un cercle centré en \( I\) et nous nommons \( A\) et $B$ les points d'intersection entre ce cercle et la droite \( A\). Par construction le triangle \( AIB\) est isocèle. Donc \( I\) est sur la médiatrice du segment \( [AB]\). La construction précédente permet de construire cette médiatrice qui sera alors perpendiculaire à \( d\) passant par \( I\).

Un dessin de la situation est proposé à la figure \ref{LabelFigPerpSegqrbMBZ}. % From file PerpSegqrbMBZ
\newcommand{\CaptionFigPerpSegqrbMBZ}{Le triangle \( AIB\) est isocèle et la droite rouge est la médiatrice de \( [AB]\).}
\input{Fig_PerpSegqrbMBZ.pstricks}

    \item
        
        Si la droite \( d\) et le point \( I\) sont donnés, pour construire la parallèle à \( d\) passant par \( I\), il suffit de construire la perpendiculaire \( d'\) à \( d\) passant par \( I\) et ensuite la perpendiculaire à \( d'\) passant par \( I\).

    \end{enumerate}
\end{proof}

%+++++++++++++++++++++++++++++++++++++++++++++++++++++++++++++++++++++++++++++++++++++++++++++++++++++++++++++++++++++++++++ 
\section{Exercices}
%+++++++++++++++++++++++++++++++++++++++++++++++++++++++++++++++++++++++++++++++++++++++++++++++++++++++++++++++++++++++++++

\Exo{Seconde-0096}
\Exo{Seconde-0097}

%+++++++++++++++++++++++++++++++++++++++++++++++++++++++++++++++++++++++++++++++++++++++++++++++++++++++++++++++++++++++++++ 
\section{AP sur les problèmes climatiques et de pétrole}
%+++++++++++++++++++++++++++++++++++++++++++++++++++++++++++++++++++++++++++++++++++++++++++++++++++++++++++++++++++++++++++

Je voudrais faire une AP basée sur la vidéo de Jean-Marc Jancovici à l'ENS :
\url{http://savoirsenmultimedia.ens.fr/expose.php?id=572}

Montrer ça à des élèves de seconde est le super-défi :)

%---------------------------------------------------------------------------------------------------------------------------
\subsection{Orthogonalité}
%---------------------------------------------------------------------------------------------------------------------------

% Note : l'orthogonalité n'a pas l'air d'être au programme de seconde.
\Exo{smath-0077}
\Exo{smath-0082}

%--------------------------------------------------------------------------------------------------------------------------- 
\subsection{Exercices de révision pour les secondes}
%---------------------------------------------------------------------------------------------------------------------------

\Exo{smath-0433}
\Exo{smath-0448}
\Exo{smath-0449}
\Exo{smath-0450}
\Exo{smath-0452}
\Exo{smath-0453}
%AFAIRE : trier ces exos dans leurs chapitres respectifs.
