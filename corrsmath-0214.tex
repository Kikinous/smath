% This is part of Un soupçon de mathématique sans être agressif pour autant
% Copyright (c) 2013
%   Laurent Claessens
% See the file fdl-1.3.txt for copying conditions.

\begin{corrige}{smath-0214}

    Écrivons un tableau des différentes possibilités :
    \begin{equation*}
        \begin{array}[]{|c||c|c|c|c|c|c|}
            \hline
            &1&2&3&4&5&6\\
            \hline\hline
            1&&&&&&\\
            \hline
            2&&&&&&\\
            \hline
            3&&&&&&\\
            \hline
            4&&&&&&\\
            \hline
            5&&&&&&\\
            \hline
            6&&&&&&\\
            \hline
        \end{array}
    \end{equation*}
    Ce tableau contient \( 36\) cases qui sont les possibilités de résultats lorsqu'on jette deux dés. De ces cases, \( 11\) contiennent un six. La probabilité d'obtenir au mois un six est donc de \( \frac{ 11 }{ 36 }\).

\end{corrige}
