% This is part of Un soupçon de mathématique sans être agressif pour autant
% Copyright (c) 2014
%   Laurent Claessens
% See the file fdl-1.3.txt for copying conditions.

\begin{corrige}{smath-0630}

    \begin{enumerate}
        \item
            Si l'utilisateur rentre \( x=-1\) alors la condition \( x<10\) est satisfaire et \( k=-3\).

            Note : si l'utilisateur rentre \( x=10\), la condition \( x<10 \) n'est pas satisfaite et l'algorithme passe au «Sinon».
        \item
            Les solutions de l'équation \( (x+2)(x-10)\) sont \( x=-2\) et \( x=10\). Pour le vérifier on peut remplacer. Pour \( x=-2\) on obtient
            \begin{equation}
                (-2+2)(-2-10)=0.
            \end{equation}
            Pour \( x=10\) on obtient
            \begin{equation}
                (10+2)(10-10)=0
            \end{equation}
            Une autre façon de vérifier est de résoudre «pour de vrai» en utilisant la règle du produit nul.
        \item
            La fraction
            \begin{equation}
                \frac{ 9a }{ 3a^2+6 }
            \end{equation}
            peut être simplifiée par \( 3\) parce que \( 9\), \( 3\) et \( 6\) le sont. Elle ne peut pas être simplifiée par \( a\) parce qu'il n'y a pas de \( a\) dans \emph{tous} les termes du dénominateur.

            La bonne réponse est donc la \ref{ItemCEzoQCB}.
        \item
            La représentation graphique de \( g\) n'est pas une droite parce que ce n'est pas une fonction affine à cause du \( x^2\). La réponse est donc \ref{ItemEKKQLoz}.

    \end{enumerate}
    

\end{corrige}
