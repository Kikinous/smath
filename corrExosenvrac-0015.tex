\begin{corrige}{Exosenvrac-0015}


    \begin{enumerate}
        \item
            \( f(x)=x^3-1\).
            \begin{enumerate}
                \item
                    Le domaine est \( \eR\).
                  \item La fonction n'est pas p\'eriodique. 
                \item
                    La fonction n'est ni paire ni impaire :
                    \begin{equation}
                        f(-x)=(-x)^3-1=-x^3-1,
                    \end{equation}
                    qui n'est ni \( f(x)\) ni \( -f(x)\).
                \item
                    Les bords du domaine sont \( \pm\infty\). Les limites sont
                    \begin{subequations}
                        \begin{align}
                            \lim_{x\to \infty} x^3-1&=\infty\\
                            \lim_{x\to -\infty} x^3-1&=-\infty.
                        \end{align}
                    \end{subequations}
                \item
                    La dérivée est \( f'(x)=3x^2\).

            \end{enumerate}
        \item
            \( f(x)= e^{\cos(x)}\)
            \begin{enumerate}
                \item
                    Le domaine est \( \eR\) entier.
                \item
                   La fonction est p\'eriodique de p\'eriode \( 2\pi\).
                \item
                    Elle est paire parce que
                    \begin{equation}
                        f(-x)= e^{\cos(-x)}= e^{\cos(x)}=f(x).
                    \end{equation}
                    Ici ce qui joue est \( \cos(-x)=\cos(x)\).
                \item
                    Les bords du domaine sont \( x=\pm\infty\). La fonction cosinus oscillant sans fin entre \( 1\) et \( -1\), la fonction oscille entre \(  e^{1}\) et \(  e^{-1}\). Les limites n'existent donc pas.
                \item
                    La dérivée :
                    \begin{equation}
                        f'(x)=-\sin(x) e^{\cos(x)}.
                    \end{equation}
            \end{enumerate}
        \item
            \( f_3(x)=\frac{ x }{ x-2 }\).
            \begin{enumerate}
                \item
                    Le domaine est \( \eR\setminus\{ 2 \}\).
                \item
                    La fonction  n'est pas périodique.
                \item
                    Ni paire ni impaire :
                    \begin{equation}
                        f(-x)=\frac{ -x }{ -x-2 }=\frac{ x }{ x+2 },
                    \end{equation}
                    qui est ni \( f(x)\) ni \( f(-x)\).
                \item
            Les bords du domaine sont \( x=\pm\infty\) et \( x=2\). Nous avons les limites
            \begin{subequations}
                \begin{align}
                    \lim_{x\to -\infty} f_3(x)&=1\\
                    \lim_{x\to \infty} f_3(x)&=1\\
                    \lim_{x\to 2} f_3(x)&=\text{n'existe pas}.
                \end{align}
            \end{subequations}
            La troisième limite n'existe pas parce que la limite à gauche et à droite ne coïncident pas.
                \item
                    La dérivée est 
            \begin{equation}
                f'_3(x)=\frac{1}{ x-2 }-\frac{ x }{ (x-2)^2 }=\frac{ -2 }{ (x-2)^2 }.
            \end{equation}
                    
            \end{enumerate}
           
    \end{enumerate}

\end{corrige}
