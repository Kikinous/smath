% This is part of Un soupçon de mathématique sans être agressif pour autant
% Copyright (c) 2012
%   Laurent Claessens
% See the file fdl-1.3.txt for copying conditions.

\begin{exercice}\label{exosmath-0021}

\begin{example}
    Le tableau suivant recense différentes situations de points \( A\) et \( B\) dans le plan. Pour chaque cas, dessiner les points et trouver le milieu.

    \begin{center}
        \begin{tabular}[h]{|c||c|c|c|c|c|c|}
            \hline
            \( A\)&\( (3;0)\)&\( (2;7)\)&\( (1;1)\)&\( (0;0)\)&\( (-1;3)\)&\( (7;-8)\)\\
            \hline
            \( B\)&\( (7;0)\)&\( (2;5)\)&\( (3;3)\)&\( (3;4)\)&\( (1;-5)\)&\( (-6;1)\)\\
            \hline\hline
            milieu&&&&&&\\
            \hline
        \end{tabular}
    \end{center}
\end{example}

\corrref{smath-0021}
\end{exercice}
