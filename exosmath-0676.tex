% This is part of Un soupçon de mathématique sans être agressif pour autant
% Copyright (c) 2014
%   Laurent Claessens
% See the file fdl-1.3.txt for copying conditions.

\begin{exercice}[\ldots/4]\label{exosmath-0676}

\begin{wrapfigure}{r}{8.0cm}
   \vspace{-0.5cm}        % à adapter.                                                                                                      
   \centering                                                                                                                               
   \input{Fig_GJbvyTt.pstricks}                                                                                                             
\end{wrapfigure}

À partir du dessin ci-contre :
    \begin{enumerate}
        \item
            Construire sur le dessin ci-contre le vecteur \( \vect{ v }=\vect{ OA }+\vect{ BC }\) et placer le point \( E\) tel que \( \vect{ v }=\vect{ OE }\).
        \item
            Le point \( F\) est-il le translaté de \( E\) par \( \vect{ OA }\), \( \vect{ AO }\) ou \( \vect{ BC }\) ?
        \item
            Prouver que les droites \( (AE)\) et \( (OF)\) sont parallèles.
        \item
            Est-ce que \( OAEF\) est un rectangle ?
    \end{enumerate}

\corrref{smath-0676}
\end{exercice}
