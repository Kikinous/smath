% This is part of Un soupçon de mathématique sans être agressif pour autant
% Copyright (c) 2014
%   Laurent Claessens
% See the file fdl-1.3.txt for copying conditions.

\begin{exercice}\label{exo2smath-0036}

    Une étude portant sur les petites parcelles de blé a noté la production et la superficie d'un certain nombre d'exploitations. Voici les résultats :
    \begin{equation*}
        \begin{array}[]{|c|c|c|c| c |c|c|}
            \hline
            \text{Superficie (\si{\meter\squared})}&2&5      &  7       &  10   &  12  & 15 \\
              \hline\hline
              \text{Production (\si{\kilo\gram} par an)}&1.6&4      &  5.6     &  8     &  9.6& 12 \\ 
              \hline 
               \end{array}
    \end{equation*}
    (sources : invention personnelle et \href{ http://fr.wikipedia.org/wiki/Rendement_agricole }{ Wikipédia })
    \begin{enumerate}
        \item
            Représenter ces données sous forme d'un graphique.
        \item
            Est-ce une situation de proportionnalité ? Justification par le graphique ou par le calcul, au choix.
        \item
            Quelle superficie faudrait-il pour produire \( \SI{20}{\kilo\gram}\) de blé par an ?
    \end{enumerate}

\corrref{2smath-0036}
\end{exercice}
