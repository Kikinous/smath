% This is part of Un soupçon de mathématique sans être agressif pour autant
% Copyright (c) 2014
%   Laurent Claessens
% See the file fdl-1.3.txt for copying conditions.

\begin{exercice}[\cite{NRHooXFvgpp}]\label{exosmath-0781}

    Recopier et compléter le tableau suivant (pour gagner du temps, partager le travail avec son voisin)
    \begin{equation*}
        \begin{array}[]{|c||c|c|c|c|c|c|c|}
            \hline
            x&0&1&2&3&4&5&10\\
            \hline
            2x+3x&&&&&&&\\
            \hline
            2+3x&&&&&&&\\
            \hline
            5x&&&&&&&\\
            \hline
        \end{array}
    \end{equation*}

    Vrai ou faux ?

    \begin{enumerate}
        \item
            L'égalité \( 2+3x=5x\)
            \begin{enumerate}
                    
        \item
             est vraie pour une valeur de $x$.
         \item
            est vraie pour n'importe quelle valeur de $x$.
            \end{enumerate}
        \item
            L'égalité \( 3x+3x=2+3x\)
            \begin{enumerate}
                \item
 est vraie pour une valeur de $x$.
 \item
est vraie pour n'importe quelle valeur de $x$.
            \end{enumerate}
        \item
            L'égalité \( 2x+3x=5x\)
            \begin{enumerate}
                \item
                    
 est vraie pour une valeur de $x$.
 \item
     est vraie pour n'importe quelle valeur de $x$.          
     \end{enumerate}
        \item

Les calculs du tableau suffisent-ils pour répondre à cette question ? Pourquoi ?
Que dois-tu utiliser pour y répondre ?


    \end{enumerate}

\corrref{smath-0781}
\end{exercice}
