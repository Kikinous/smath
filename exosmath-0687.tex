% This is part of Un soupçon de mathématique sans être agressif pour autant
% Copyright (c) 2014
%   Laurent Claessens
% See the file fdl-1.3.txt for copying conditions.

\begin{exercice}\label{exosmath-0687}

    Donner les coordonnées du vecteur \( k\vect{ v }\) dans les cas suivants :
    \begin{multicols}{3}
        \begin{enumerate}
            \item
                \( k=4\) et \( \vect{ v }\begin{pmatrix}
                    3    \\ 
                    2    
                \end{pmatrix}\);
            \item
                \( k=-3\) et \( \vect{ v }\begin{pmatrix}
                    -1    \\ 
                    2    
                \end{pmatrix}\)
            \item
                \( k=-1\) et \( \vect{ v }\begin{pmatrix}
                    -6    \\ 
                    -4    
                \end{pmatrix}\).
        \end{enumerate}
    \end{multicols}

\corrref{smath-0687}
\end{exercice}
