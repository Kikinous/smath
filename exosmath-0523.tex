% This is part of Un soupçon de mathématique sans être agressif pour autant
% Copyright (c) 2013
%   Laurent Claessens
% See the file fdl-1.3.txt for copying conditions.

\begin{exercice}\label{exosmath-0523}

\begin{wrapfigure}{r}{5.0cm}
   \vspace{-0.5cm}        % à adapter.
   \centering
   \input{Fig_ONMRllE.pstricks}
\end{wrapfigure}
 
La figure \( ABCDEFGH\) est un cube de \unit{2}{\meter} de côté. Le point \( I\) est le milieu du côté \( [AE]\) et \( J\) est le milieu de \( [CG]\). Nous notons \( \alpha\) le plan \( (IJD)\).
\begin{enumerate}
    \item
        Est-ce que les points \( I\) et \( J\) sont dans le plan \( (AEH)\) ?
    \item
        Est-ce que les plans \( P\) et \( AEH\) sont sécants ? Que vaut \(  (AEH)\cap \alpha  \) ?
    \item
        Donner \( (IJ)\cap (DHC)\).
    \item
        Quelle est la position relative des droites \( (EF)\) et \( (AH)\) ?
    \item
        Donner une droite du plan \( (EFG)\) à être parallèle au plan \( \alpha\).
\end{enumerate}

\corrref{smath-0523}
\end{exercice}
