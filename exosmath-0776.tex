% This is part of Un soupçon de mathématique sans être agressif pour autant
% Copyright (c) 2014
%   Laurent Claessens
% See the file fdl-1.3.txt for copying conditions.

\begin{exercice}\label{exosmath-0776}

    Le triangle \( ABC\) est isocèle en \( C\); nous savons la longueur \( AB=\unit{5}{\centi\meter}\) et l'angle \( \widehat{CAB}=\unit{20}{\degree}\). Le dessiner en vraie grandeur.

\corrref{smath-0776}
\end{exercice}
