% This is part of Un soupçon de mathématique sans être agressif pour autant
% Copyright (c) 2014
%   Laurent Claessens
% See the file fdl-1.3.txt for copying conditions.

\begin{exercice}\label{exosmath-0720}


    Un fabriquant médicaments prétend que son nouveau médicament «Palox» aide à la digestion. Des enquêteurs ont demandé à \( 1000\) personnes si elles souffraient de mal de ventre et si elles prenaient du Palox. Les résultats sont les suivants :

    \begin{center}
    \begin{tabular}[]{|c|c|c|c|}
        \hline
        &mal au ventre&pas mal au ventre&total\\
        \hline
        prend du Palox&&234&312\\
        \hline
        ne prend pas de Palox&&551&\\
        \hline
        total&&&1000\\
        \hline
    \end{tabular}
    \end{center}

    \begin{multicols}{2}
    Nous choisissons un des sondés au hasard et nous considérons les événements
    \begin{itemize}
        \item \( P\) : «il prend du Palox» 
        \item   \( M\) : «il a mal au ventre».
    \end{itemize}
    \begin{enumerate}
        \item
            Compléter le tableau.
        \item
            Calculer la probabilité de \( P\cap M\) et \( P\cup M\).
        \item
            Décrire par une phrase l'événement \( \bar P\).
        \item
            Nous choisissons une personne au hasard parmi celles qui prennent du Palox. Quelle est la probabilité qu'elle souffre de mal au ventre ?
        \item
            Nous choisissons une personne au hasard parmi celles qui n'ont pas mal au ventre. Quelle est la probabilité qu'elle prenne du Palox ?
        \item
            Est-ce que le Palox semble efficace ? Expliquer.
    \end{enumerate}
    \end{multicols}

\corrref{smath-0720}
\end{exercice}
