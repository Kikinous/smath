% This is part of Un soupçon de mathématique sans être agressif pour autant
% Copyright (c) 2014
%   Laurent Claessens
% See the file fdl-1.3.txt for copying conditions.

\begin{exercice}\label{exosmath-0707}

    Le programme suivant demande les coordonnées d'un point \( A\), et répond «oui» si il est sur le graphe de la fonction \( f(x)=2x-4\) et «non» sinon.

    \begin{fmpage}{0.9\linewidth}

        demander \( x_A\)

        demander \( y_A\)

        Si \(  y_A=2x_A-4 \) :

        ~~ Écrire «Oui» 

        Sinon :

        ~~ «Non» 

    \end{fmpage}

    En utilisant une boucle «tant que», modifier ce programme pour que l'ordinateur redemande les coordonnées jusqu'à ce que le point \( A\) donné soit sur la droite. À chaque mauvais point, il devra écrire «non», et lorsqu'un point correct est donné, il devra écrire «oui» et d'arrêter.

\corrref{smath-0707}
\end{exercice}
