% This is part of Un soupçon de mathématique sans être agressif pour autant
% Copyright (c) 2015
%   Laurent Claessens
% See the file fdl-1.3.txt for copying conditions.

%--------------------------------------------------------------------------------------------------------------------------- 
\subsection*{Activité : soustraire une somme}
%---------------------------------------------------------------------------------------------------------------------------

Sur la figure suivante nous avons tracé un triangle \( ABC\) et la droite parallèle à \( (AB)\) passant par \( C\). Parmi les trois angles que l'on voit au point \( C\), lequel fait \SI{34}{\degree} ? Est-il possible de déterminer la mesure des deux autres ?

\begin{center}
   \input{Fig_UZOQooTSAQcl.pstricks}
\end{center}



% ATTENTION : ces deux figures sont reprise dans une autre activité (Activité : soustraire une somme). Ne pas les changer sans les doubler.




Sur le dessin suivant, indiquer quels sont les angles égaux à \( a\) et à \( b\). 
\begin{center}
    \input{Fig_QZABooEsqWaq.pstricks}
\end{center}
Si \( a\) et \( b\) étaient connus, comment feriez-vous pour calculer l'angle \( c\) ?
