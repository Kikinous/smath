% This is part of Un soupçon de mathématique sans être agressif pour autant
% Copyright (c) 2014
%   Laurent Claessens
% See the file fdl-1.3.txt for copying conditions.

\begin{exercice}\label{exosmath-0601}

    Type «calcul mental». Les réponses doivent être brièvement justifiées.
    \begin{enumerate}
        \item
            Résoudre \( (x+3)(x-15)=0\)
        \item
            Si \( f(x)=-2x+12\), que vaut \( f(7)\) ?
        \item
            Simplifier et dire par combien vous avez simplifié :
            \begin{equation*}
                \frac{ 6a+a^2 }{ 3a }.
            \end{equation*}
        \item
            Quelle est la distance entre les points \( A(0;5)\) et \( B(B;-4)\) ?
        \item
            Un rectangle a une longueur de \unit{7}{\centi\meter} et une aire de \unit{2}{\centi\meter\squared}. Quelle est la valeur exacte de la largeur de sa hauteur.
    \end{enumerate}

\corrref{smath-0601}
\end{exercice}
