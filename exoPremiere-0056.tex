% This is part of Un soupçon de mathématique sans être agressif pour autant
% Copyright (c) 2012
%   Laurent Claessens
% See the file fdl-1.3.txt for copying conditions.

\begin{exercice}\label{exoPremiere-0056}

    Écrire une fonction prenant une liste en argument et retournant le double du dernier élément. Par exemple en donnant la liste \info{[1,8,3,7]}, la fonction doit retourner \( 14\).

    Tester la fonction sur les listes suivantes :
    \begin{enumerate}
        \item
            \info{[1,2,4]},
        \item
            \info{range(1,10)}. Attention : ici il y a un effet à remarquer.
        \item
            \info{[``Maïlys'',``Anne'',``Estelle'']}
    \end{enumerate}

            Formellement, l'objet \info{range(1,5)} n'est pas une liste, mais en pratique, ça y ressemble beaucoup et votre fonction devrait retourner \( 4\), comme si \info{range(1,5)} était la liste \info{[1,2,3,4]}.

\corrref{Premiere-0056}
\end{exercice}
