% This is part of Un soupçon de mathématique sans être agressif pour autant
% Copyright (c) 2013
%   Laurent Claessens
% See the file fdl-1.3.txt for copying conditions.

\begin{corrige}{smath-0517}

\begin{wrapfigure}[9]{r}{6.0cm}
   \vspace{-0.5cm}        % à adapter.
   \centering
   \input{Fig_RLGQTQR.pstricks}
\end{wrapfigure}

Le graphe est sur la figure ci-contre.

    \begin{enumerate}
        \item
            L'image de \( 130\) par \( g\) est le nombre \( g(130)\) qui vaut \( 2\times 130-4=256\). Trouver un antécédent de \( 10\) revient à trouver une valeur de \( x\) pour laquelle \( g(x)=10\), c'est à dire résoudre l'équation
            \begin{equation}
                2x-4=10.
            \end{equation}
            La solution est donnée par \( x=7\). Le nombre \( 7\) est donc antécédent de \( 10\) par \( g\).

        \item
            Pour tracer le graphe d'une fonction affine il suffit de trouver deux points. Nous avons par exemple \( g(0)=-4\) et \( g(3)=2\), ce qui donne les deux points \( A(0;-4)\) et \( B(3;2)\). Nous les plaçons dans un repère et nous traçons la droite à la règle.

        \item

            Étudier le signe signifie tracer le tableau de signe. L'antécédent de zéro se trouve en résolvant l'équation \( g(x)=0\), c'est à dire
            \begin{equation}
                2x-4=0
            \end{equation}
            dont la solution est \( x=2\). C'est donc le nombre \( 2\) qui apparaît dans le tableau :
            \begin{equation*}
                \begin{array}[]{c|ccc}
                     x&&2&\\
                      \hline
                      f(x)&-&0&+\\ 
                       \end{array}
              \end{equation*}
              Le signe \( -\) est à gauche et le \( +\) à droite parce que la fonction est croissante.     
          \item
              Résoudre l'équation \( g(x)\geq 0\) revient à trouver pour quels \( x\) nous avons \( 2x-4\geq 0\). Il n'y a rien à faire : c'est déjà dans le tableau de signe : ce sont les \( x\geq 2\). Donc la réponse est
              \begin{equation}
                  x\in\mathopen[ 2; \infty [.
              \end{equation}
    \end{enumerate}

\end{corrige}
