% This is part of Un soupçon de mathématique sans être agressif pour autant
% Copyright (c) 2012
%   Laurent Claessens
% See the file fdl-1.3.txt for copying conditions.

\begin{exercice}\label{exosmath-0054}

    Soit un triangle \( ABC\). Nous considérons les points \( N\), \( M\) et \( P\) tels que \( \vect(AN)=\frac{ 3 }{ 4 }\vect{AB}\), \( \vect{BP}=\frac{ 1 }{2}\vect{BC}\) et \( \vect{AM}=\frac{ 3 }{2}\vect{AC}\).

    Prouver que les points \( N\), \( P\) et \( M\) sont alignés.

\corrref{smath-0054}
\end{exercice}
