% This is part of Un soupçon de mathématique sans être agressif pour autant
% Copyright (c) 2014
%   Laurent Claessens
% See the file fdl-1.3.txt for copying conditions.

\begin{corrige}{smath-0978}

    \begin{enumerate}
        \item
    Il fait d'abord tracer un segment \( [LK]\) de longueur \SI{6}{\centi\meter} et de mettre le point \( L\) où on veut.
\item
    La médiane issue de \( K\) est la droite passant par \( K\) et coupant \( [LM]\) en son milieu. La hauteur issue de \( L\) est la droite passant par \( L\) et coupant \( [KM]\) de façon perpendiculaire.
\item
    Au niveau des codages, il faut noter un angle droit entre la hauteur et le segment \( [KM]\), et montrer que les deux «morceaux» de \( [LM]\) délimités par la médiane sont de même longueurs.
    \end{enumerate}
    
    Un exemple de dessin que cela peut donner :

\begin{center}
   \input{Fig_JAQHooIhMeKp.pstricks}
\end{center}

    La hauteur est en bleu et la médiane en rouge.


    Attention à ne pas ajouter des codages inutiles. Sur ce dessin, la médiane semble être perpendiculaire au côté \( [L,M]\). Ce n'est cependant pas le cas. Ce serait le cas si le triangle était isocèle. En règle générale, n'ajoutez des codages que si vous êtes sûr de votre coup.

\end{corrige}
