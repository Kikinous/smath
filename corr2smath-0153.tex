% This is part of Un soupçon de mathématique sans être agressif pour autant
% Copyright (c) 2015
%   Laurent Claessens
% See the file fdl-1.3.txt for copying conditions.

\begin{corrige}{2smath-0153}

    \begin{enumerate}
        \item
            Il s'agit de la formule de double distributivité :
            \begin{subequations}
                \begin{align}
                    (x+3)(x-5)&=x^2+3x-5x-15\\
                    &=x^2-2x-15.
                \end{align}
            \end{subequations}
        \item
            Il est possible de mettre \( 4\) et \( x\) en facteur. Les trois réponses possibles sont donc, dans l'ordre de préférence : \( 4(x+3x^2)\), \( x(4+12x)\) et \( 4x(1+3x)\).
    \end{enumerate}

\end{corrige}
