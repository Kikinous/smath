% This is part of Un soupçon de mathématique sans être agressif pour autant
% Copyright (c) 2013
%   Laurent Claessens
% See the file fdl-1.3.txt for copying conditions.

\begin{exercice}\label{exosmath-0321}

    Compléter les tableaux de signe suivants.
    \begin{multicols}{2}
    \begin{equation*}
        \begin{array}[]{|c||ccccc|}
            \hline
             x&-\infty&&\ldots&&+\infty\\
              \hline
              -3x+1&&\ldots&0&\ldots&\\ 
              \hline 
               \end{array}
           \end{equation*}


    \begin{equation*}
        \begin{array}[]{|c||ccccc|}
            \hline
             x&-\infty&&\ldots&&+\infty\\
              \hline
              2x+4&&\ldots&0&\ldots&\\ 
              \hline 
               \end{array}
           \end{equation*}

    \end{multicols}
    Résoudre l'inéquation \( -3x+1>0\) et donner les solutions sous forme d'un intervalle.

\corrref{smath-0321}
\end{exercice}
