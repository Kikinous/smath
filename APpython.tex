% This is part of Un soupçon de mathématique sans être agressif pour autant
% Copyright (c) 2012
%   Laurent Claessens
% See the file fdl-1.3.txt for copying conditions.

\chapter{Structures de base}

%+++++++++++++++++++++++++++++++++++++++++++++++++++++++++++++++++++++++++++++++++++++++++++++++++++++++++++++++++++++++++++
\section{Listes}
%+++++++++++++++++++++++++++++++++++++++++++++++++++++++++++++++++++++++++++++++++++++++++++++++++++++++++++++++++++++++++++

Une liste est une collection ordonnée d'éléments. Elle se définit avec des crochets.

\lstinputlisting{ex_listes.py}

donne

\lstinputlisting[title=Résultat]{res_ex_listes.txt}

Nous pouvons ajouter un élément à une liste en utilisant la \emph{méthode} \info{append}, et retrouver un élément d'une liste par son numéro (attention : python numérote à partie de zéro). Cas spécial : le dernier élément de la liste est le numéro \( -1\).

\lstinputlisting{ex_listes2.py}

donne

\lstinputlisting[title=Résultat]{res_ex_listes2.txt}

La multiplication d'une liste par un nombre donne la liste contenant plusieurs fois la liste originale. Nous ajoutons une liste à une autre en utilisant la méthode \info{extend}.


\lstinputlisting{ex_listes3.py}

donne

\lstinputlisting[title=Résultat]{res_ex_listes3.txt}


%+++++++++++++++++++++++++++++++++++++++++++++++++++++++++++++++++++++++++++++++++++++++++++++++++++++++++++++++++++++++++++
\section{Suite définie par récurrence} 
%+++++++++++++++++++++++++++++++++++++++++++++++++++++++++++++++++++++++++++++++++++++++++++++++++++++++++++++++++++++++++++

Soit une suite définie par récurrence

\begin{subequations}
    \begin{numcases}{}
    u_0=10\\
    u_{n+1}=2u_n.
    \end{numcases}
\end{subequations}

Nous voudrions pouvoir répondre à deux types de questions :
\begin{enumerate}
    \item
        construire une liste contenant les \( 100\) premiers termes de la suite;
    \item
        savoir quel est le premier terme à dépasser un million.
\end{enumerate}

Avant de se lancer, nous devons nous poser une question de vocabulaire : est-ce que le centième terme de la suite \( u\) est \( u_{100}\) ou \( u_{99}\) ? Le premier terme étant \( u_0\), le centième est bien \( u_{99}\).

\lstinputlisting{recurrence1.py}

donne

\lstinputlisting{res_recurrence1.txt}

Notons que la ligne \info{print(u[100])} plante avec l'erreur \info{list index out of range}, c'est à dire que la liste \info{u} n'a pas d'élément \info{u[100]}, ce qui est normal parce qu'elle contient \( 100\) éléments et que la numérotation commence à zéro.

En ce qui concerne la possibilité de trouver le premier élément qui dépasse le million, le programme suivant donne deux méthodes.

\lstinputlisting{recurrence2.py}

donne

\lstinputlisting{res_recurrence2.txt}

Notons que la première méthode donne \( 17\) et la seconde donne \( 18\). Qui a raison ? Le \( 18\) de la seconde méthode est la longueur de la liste construite, donc il indique que le premier terme à passer le million est \( u_{17}\) (vu que le premier terme est \( u_0\), la longueur est toujours un plus grande que le numéro du dernier élément).

La réponse est donc que \( u_{17}\) est le premier élément à être plus grand que un million, mais \( u_{17}\) est le dix-huitième élément de la liste.


%+++++++++++++++++++++++++++++++++++++++++++++++++++++++++++++++++++++++++++++++++++++++++++++++++++++++++++++++++++++++++++
\section{Moyenne, médiane, quartiles}
%+++++++++++++++++++++++++++++++++++++++++++++++++++++++++++++++++++++++++++++++++++++++++++++++++++++++++++++++++++++++++++
\label{SecZYftar}

    % Note : la liste ci-dessous est codée en dur dans les scripts d'exemples. Si on la modifie, il faut modifier les scripts.
    Un magasin de chaussures a vendu des tailles entre \( 40\) et \( 45\) suivant la distribution suivante :
    \begin{center}
        \begin{tabular}{|l||c|c|c|c|c|c|}
            \hline
            taille \( x_i\)&40&41&42&43&44&45\\
            \hline
            effectifs \( n_i\)&3&5&10&8&2&4\\
            \hline
        \end{tabular}
    \end{center}

    Nous allons écrire quelque programmes pour calculer les moyennes, médianes et quartiles de ces chiffres. Le programme suivant écrit la moyenne de la liste.

    \lstinputlisting{chaussures.py}

    %\lstinputlisting{res_chaussures.txt}


Nous pouvons aussi créer des fonctions qui donnent la médiane et les quartiles d'une liste de nombres. Voici pour le premier quartile :

\lstinputlisting{premier_quartile.py}

Et voici pour la médiane :

\lstinputlisting{mediane.py}

Nous nous créons maintenant un petit module contenant les fonctions moyenne, médiane et quartiles.

