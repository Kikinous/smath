% This is part of Un soupçon de mathématique sans être agressif pour autant
% Copyright (c) 2013
%   Laurent Claessens
% See the file fdl-1.3.txt for copying conditions.

\begin{exercice}\label{exosmath-0510}

    Soit \( f\) la fonction affine donnée par la formule \( f(x)=-3x+7\).
    \begin{enumerate}
        \item
            Calculer \( f(1)\), \( f(-4)\) et \( f(\frac{ 4 }{ 9 })\).
        \item
            Donner un antécédent de \( 19\).
        \item
            Pour quelle valeur de \( x\) a-t-on \( f(x)=-32\) ?
    \end{enumerate}

\corrref{smath-0510}
\end{exercice}
