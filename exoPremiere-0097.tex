% This is part of Un soupçon de mathématique sans être agressif pour autant
% Copyright (c) 2012
%   Laurent Claessens
% See the file fdl-1.3.txt for copying conditions.

\begin{exercice}\label{exoPremiere-0097}

    Soit le polynôme du second degré \( f(x)=-2x^2+2x+4\).
    \begin{multicols}{2}
        \begin{enumerate}
            \item
                Donner les racines de \( f\).
            \item
                Donner la factorisation de \( f(x)\).
            \item
                Donner les coordonnées du sommet de \( f\).
            \item
                Écrire le tableau de signe de la fonction~\( f\).
            \item
                Calculer \( f(0)\), \( f(-2)\) et \( f(3)\).
            \item
                Avec les information obtenues dans les points précédents, tracer le graphe de la fonction \( f\) entre les abscisses \( -2\) et \( 1\).
        \end{enumerate}
    \end{multicols}

\corrref{Premiere-0097}
\end{exercice}
