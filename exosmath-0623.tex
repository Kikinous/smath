% This is part of Un soupçon de mathématique sans être agressif pour autant
% Copyright (c) 2014
%   Laurent Claessens
% See the file fdl-1.3.txt for copying conditions.

\begin{exercice}\label{exosmath-0623}

    \begin{enumerate}
        \item
            Est-ce que les points \( K(10;32)\), \( L(16;24)\) et \( E(30;22)\) sont alignés ?
        \item
    Un joueur situé à la position \( K(10;32)\) sur la carte d'un jeu de combat veut tirer sur un ennemi situé à la position \( E(30;22)\). Peut-il l'atteindre sans toucher son ami positionné en \( A(22;28)\) ?
            
    \end{enumerate}

    Faire les exercices 36 et 39 de la page 187.

\corrref{smath-0623}
\end{exercice}
