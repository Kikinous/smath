% This is part of Un soupçon de mathématique sans être agressif pour autant
% Copyright (c) 2012-2013
%   Laurent Claessens
% See the file fdl-1.3.txt for copying conditions.

\begin{exercice}\label{exosmath-0081}

\begin{wrapfigure}{r}{3.5cm}
   \vspace{-0.3cm}        % à adapter.
   \centering
   \input{Fig_figureTFaRFVd.pstricks}
\end{wrapfigure}

        Soit la pyramide régulière de base carrée dessinée ci-contre. Le point \( I\) est sur le segment \( AS\) et le point \( J\) est sur \( [SC]\).
        \begin{enumerate}
            \item
                Est-ce que le droites \( (AC)\) et \( (IJ)\) sont sécantes ?
            \item
                Déterminer l'intersection entre les plans \( (SAJ)\) et \( (BCD)\).
            \item
                Où placer le point \( I\) sur \( [AS]\) pour que les droites \( (AC)\) et \( (IJ)\) ne soient pas sécantes ?
        \end{enumerate}

        Faire à la maison l'exercice 11 page 235.

\corrref{smath-0081}
\end{exercice}
