% This is part of Un soupçon de mathématique sans être agressif pour autant
% Copyright (c) 2013
%   Laurent Claessens
% See the file fdl-1.3.txt for copying conditions.

\begin{corrige}{Seconde-0047}


\begin{wrapfigure}{r}{12.cm}
   \vspace{-1cm}        % à adapter.
   \centering
   \input{Fig_KGOAveW.pstricks}
\end{wrapfigure}

    Tout est lisible sur le graphique. Attention : si certains points ne tombent pas juste sur le quadrillage, il faut donner une approximation numérique.
    \begin{enumerate}
        \item
            Les solutions de \( f(x)=0\) sont les abscisses des points où la courbe «vaut zéro», c'est à dire des points où la courbe touche l'axe des abscisses. Ce sont les points \( A\), \( B\) et \( C\) sur le graphe. Les solutions de \( f(x)=0\) sont donc (approximativement) :
            \begin{equation}
                S=\{ -0.7;0.4;2.25 \}.
            \end{equation}
    \end{enumerate}
    <++>

\end{corrige}
