% This is part of Un soupçon de mathématique sans être agressif pour autant
% Copyright (c) 2014
%   Laurent Claessens
% See the file fdl-1.3.txt for copying conditions.

%+++++++++++++++++++++++++++++++++++++++++++++++++++++++++++++++++++++++++++++++++++++++++++++++++++++++++++++++++++++++++++ 
    \section{Repères, distance et milieu}
%+++++++++++++++++++++++++++++++++++++++++++++++++++++++++++++++++++++++++++++++++++++++++++++++++++++++++++++++++++++++++++
    \justification

    \begin{multicols}{2}
% Placer
\Exo{smath-0480}
\Exo{smath-0481}
\Exo{smath-0482}
\Exo{Seconde-0007}

% Distance
\Exo{Seconde-0008}
\Exo{smath-0483}
\Exo{smath-0475}
\Exo{smath-0627}

\Exo{smath-0293}
\Exo{Seconde-0006}
\Exo{Seconde-0009}


% Milieu
\Exo{smath-0484}
\Exo{smath-0485}
\Exo{smath-0019}
\Exo{smath-0486}

\Exo{smath-0487}
\Exo{smath-0624}
\Exo{Seconde-0012}
\Exo{smath-0488}
\Exo{Seconde-0004}
\Exo{Seconde-0056}
\Exo{smath-0476}
\Exo{smath-0478}


\Exo{Seconde-0055}
\Exo{Seconde-0011}

\Exo{smath-0125}
\Exo{smath-0410}

\Exo{smath-0124}
\Exo{Seconde-0005}
\Exo{Seconde-0010}
\Exo{Seconde-0020}


% Repère pas ON
\Exo{smath-0020}
\end{multicols}

\newpage

%+++++++++++++++++++++++++++++++++++++++++++++++++++++++++++++++++++++++++++++++++++++++++++++++++++++++++++++++++++++++++++ 
\section{Fonctions linéaires et affines}
%+++++++++++++++++++++++++++++++++++++++++++++++++++++++++++++++++++++++++++++++++++++++++++++++++++++++++++++++++++++++++++

\justification
    
%\corrDraft{1}

\begin{multicols}{2}
\Exo{smath-0514}
    \Exo{smath-0149}
\Exo{smath-0513}
    \Exo{smath-0500}
    \Exo{smath-0494}
    \Exo{smath-0491}
    \Exo{smath-0127}
    \Exo{smath-0490}
    \Exo{smath-0016}
    \Exo{smath-0135}
    \Exo{smath-0148}
    \Exo{smath-0001}
    \Exo{smath-0502}
\Exo{smath-0625}
    \Exo{smath-0492}
    \Exo{smath-0497}
    \Exo{smath-0496}
\Exo{smath-0626}

    \Exo{smath-0499}
    \Exo{smath-0498}
    \Exo{smath-0501}
    \Exo{smath-0504}
    \Exo{smath-0503}
\end{multicols}
    \Exo{smath-0134}


    \clearpage
%+++++++++++++++++++++++++++++++++++++++++++++++++++++++++++++++++++++++++++++++++++++++++++++++++++++++++++++++++++++++++++ 
\section{Représentation graphique de fonctions}
%+++++++++++++++++++++++++++++++++++++++++++++++++++++++++++++++++++++++++++++++++++++++++++++++++++++++++++++++++++++++++++

\justification

 %   \corrDraft{0}
\Exo{Seconde-0072}
\Exo{smath-0629}
\Exo{Seconde-0070}


\vspace{3cm}

\begin{multicols}{2}
    \Exo{smath-0209}
    \Exo{Seconde-0049}
    \Exo{smath-0298}

    \Exo{Seconde-0043}
    \Exo{smath-0489}
    \Exo{Seconde-0075}
    \Exo{smath-0439}
\Exo{smath-0505}
\end{multicols}
\Exo{Seconde-0069}

    \clearpage
%+++++++++++++++++++++++++++++++++++++++++++++++++++++++++++++++++++++++++++++++++++++++++++++++++++++++++++++++++++++++++++ 
\section{Statistique descriptive}
%+++++++++++++++++++++++++++++++++++++++++++++++++++++++++++++++++++++++++++++++++++++++++++++++++++++++++++++++++++++++++++

\justification

%\corrDraft{1}

\begin{multicols}{2}
\Exo{smath-0534}    %DS
\Exo{smath-0536}
\Exo{smath-0537}
\Exo{Seconde-0035}
\Exo{smath-0538}
\Exo{Seconde-0028}
\Exo{Seconde-0073}
\end{multicols}

\Exo{smath-0532}        %DS
\Exo{smath-0533}    %DS
\begin{multicols}{2}
    \Exo{smath-0530}    %DS
    \Exo{smath-0535}
\end{multicols}
\Exo{Seconde-0014}
\Exo{Seconde-0037}


    \clearpage
%+++++++++++++++++++++++++++++++++++++++++++++++++++++++++++++++++++++++++++++++++++++++++++++++++++++++++++++++++++++++++++ 
\section{Géométrie dans l'espace}
%+++++++++++++++++++++++++++++++++++++++++++++++++++++++++++++++++++++++++++++++++++++++++++++++++++++++++++++++++++++++++++

% TODO : il faut plus d'exercices avec des droites à prolonger hors du cube. Et aussi avec des volumes qui ne sont pas des cubes, genre des
%           pavés pas droits.

   %\corrDraft{1}
\justification

% Intersections de plans et co

\Exo{smath-0081}
\Exo{Seconde-0087}
\Exo{smath-0525}
\Exo{smath-0079}
\Exo{smath-0115}
\Exo{smath-0523}

\begin{multicols}{2}
\Exo{smath-0524}
\Exo{smath-0114}
\Exo{smath-0009}
\Exo{smath-0113}
\end{multicols}


\Exo{smath-0094}
\Exo{smath-0529}

\clearpage

%+++++++++++++++++++++++++++++++++++++++++++++++++++++++++++++++++++++++++++++++++++++++++++++++++++++++++++++++++++++++++++ 
\section{Variations de fonctions}
%+++++++++++++++++++++++++++++++++++++++++++++++++++++++++++++++++++++++++++++++++++++++++++++++++++++++++++++++++++++++++++

   %\corrDraft{1}
\justification

\Exo{smath-0547}    % De DS des autres 
\vspace{0.5cm}
\Exo{smath-0552}    % De DS des autres
\vspace{2cm}
\begin{multicols}{2}
\Exo{smath-0548}    % De DS des autres 
\Exo{smath-0549}    % De DS des autres 
\Exo{smath-0546}    % De DS des autres
\Exo{smath-0550}    % De DS des autres
\Exo{smath-0553}    % DS
\end{multicols}
\Exo{smath-0554}

%+++++++++++++++++++++++++++++++++++++++++++++++++++++++++++++++++++++++++++++++++++++++++++++++++++++++++++++++++++++++++++ 
\section{Vecteurs}
%+++++++++++++++++++++++++++++++++++++++++++++++++++++++++++++++++++++++++++++++++++++++++++++++++++++++++++++++++++++++++++

\Exo{smath-0594}
\Exo{smath-0593}

\begin{multicols}{2}
\Exo{smath-0592}
\Exo{smath-0588}    % de Haag
\Exo{smath-0595}
\Exo{smath-0590}    % Haag          % Celui-ci est un exercice à découper en deux.
\Exo{smath-0103}
\Exo{smath-0332}
\Exo{smath-0067}

\Exo{smath-0596}
\Exo{smath-0591}    % Haag
\Exo{smath-0106}
\Exo{smath-0063}
\Exo{smath-0064}
\Exo{smath-0111}
\Exo{smath-0142}
\Exo{smath-0085}
\Exo{smath-0073}
\Exo{smath-0661}    % Celui-ci n'est pas des feuilles, mais il faut le considérer parce qu'il vient d'une formation.
\end{multicols}

%+++++++++++++++++++++++++++++++++++++++++++++++++++++++++++++++++++++++++++++++++++++++++++++++++++++++++++++++++++++++++++ 
\section{Équation de droite (2D)}
%+++++++++++++++++++++++++++++++++++++++++++++++++++++++++++++++++++++++++++++++++++++++++++++++++++++++++++++++++++++++++++

\begin{multicols}{2}
\Exo{smath-0607}
\Exo{smath-0613}
\Exo{smath-0608}
\Exo{smath-0620}
\Exo{smath-0130}
\Exo{smath-0615}
\Exo{smath-0616}
\Exo{smath-0614}
    \Exo{smath-0603}    % Haag
\Exo{smath-0605}    % Haag
\Exo{smath-0200}
\end{multicols}


\Exo{smath-0606}    % Haag
\vspace{1cm}
\begin{multicols}{2}
\Exo{smath-0609}    % Haag
\Exo{smath-0610}    % Haag
\Exo{smath-0622}
\Exo{smath-0611}    % Haag
\Exo{smath-0612}    % Haag
\end{multicols}
\Exo{smath-0617}
\begin{multicols}{2}
\Exo{smath-0618}
\Exo{smath-0110}
\Exo{smath-0084}
\Exo{smath-0623}

\Exo{smath-0619}
\Exo{smath-0621}
\Exo{smath-0451}
\Exo{smath-0450}
\Exo{smath-0234}
\Exo{smath-0649}
\end{multicols}
\Exo{smath-0413}
