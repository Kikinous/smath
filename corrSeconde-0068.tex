% This is part of Un soupçon de mathématique sans être agressif pour autant
% Copyright (c) 2012
%   Laurent Claessens
% See the file fdl-1.3.txt for copying conditions.

\begin{corrige}{Seconde-0068}

    \begin{enumerate}
        \item
            Avec \( v=1\) nous avons \( E(3)=a\times 1^3=a\).
        \item
            Avec \( v=3\) nous avons \( E(3)=a\times 3^3=27a\).
        \item
            Les \( 13\) janvier, l'énergie produite était de \( E(v_0)=av_0^3\). Le \( 14\) janvier, la production était de \( E(2v_0)=a\times (2v_0)^2=8av_0^3\).
        \item
            Le rapport entre les deux est
            \begin{equation}
                \frac{ 8av_0^3 }{ av_0^3 }=8.
            \end{equation}
            <++>
    \end{enumerate}
    <++>

    L'énergie produite par l'éolienne en fonction de la vitesse du vent est donnée par \( E(v)=av^3\). Si la vitesse du vent est \( v_0\), alors l'énergie produite est \( E(v_0)=av_0^3\); si la vitesse du vent est \( 2v_0\), alors l'énergie produite sera \( a(2v_0)^2\).

\end{corrige}
