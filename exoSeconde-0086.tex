% This is part of Un soupçon de mathématique sans être agressif pour autant
% Copyright (c) 2012
%   Laurent Claessens
% See the file fdl-1.3.txt for copying conditions.

\begin{exercice}\label{exoSeconde-0086}

    Vérification d'une des réponses de l'exercice \ref{exoSeconde-0054}. Écrire un programme qui compte la quantité de nombres de \( 3\) chiffres. Pour cela :
    \begin{multicols}{2}
    \begin{enumerate}
        \item
            Tester les deux petits programmes suivants :
            \begin{verbatim}
            a="123"
            print(len(a))
            \end{verbatim}
            et
            \begin{verbatim}
            a=123
            print(len(a))
            \end{verbatim}
            Le premier donne la bonne réponse : \info{3}. Le second plante. Pourquoi ?
        \item
            Initialiser une variable \info{n} à zéro.
        \item
            Écrire une boucle pour \info{i} dans \info{range(1,1001)} (pourquoi \( 1001\) et non \( 1000\) ?)
        \item
            Pour chaque \info{i}, si \info{i} a \( 3\) chiffres, incrémenter \info{n} de \( 1\).
        \item
            Afficher la valeur finale de \info{n}.
    \end{enumerate}
    \end{multicols}

\corrref{Seconde-0086}
\end{exercice}
