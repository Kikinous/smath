% This is part of Un soupçon de mathématique sans être agressif pour autant
% Copyright (c) 2013
%   Laurent Claessens
% See the file fdl-1.3.txt for copying conditions.

\begin{exercice}[\cite{QSfMxlk}]\label{exosmath-0383}

    Un entraineur a étudié les statistiques des tirs au but de ses joueurs. Il a alors remarqué qu'un joueur pris au hasard dans l'équipe marque le but avec une probabilité \( 0.55\). On admet que les épreuves de tir au but sont des expériences aléatoires identiques (on néglige les effets psychologiques dus à la fatigue par exemple). 

    Durant un entrainement un joueur tire une série de \( 5\) ballons.
    \begin{enumerate}
        \item
            Calculer la probabilité qu'il réussisse tous ses tirs.
        \item
            Calculer la probabilité qu'il réussisse exactement \( 3\) tirs.
        \item
            Calculer la probabilité qu'il réussisse au moins un tir.
        \item
            Calculer le nombre de tirs au but que le joueur doit effectuer pour que la probabilité d'en marquer au moins un soit supérieure à \( 0.9999\).
    \end{enumerate}

\corrref{smath-0383}
\end{exercice}
