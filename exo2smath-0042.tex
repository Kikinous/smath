% This is part of Un soupçon de mathématique sans être agressif pour autant
% Copyright (c) 2014
%   Laurent Claessens
% See the file fdl-1.3.txt for copying conditions.

\begin{exercice}[Histoires de calendrier]\label{exo2smath-0042}

    Pour l'exercice suivant, les élèves les plus aventureux tiendront compte du fait que les calendriers commencent en l'an \( 1\) (et non zéro) et n'ont pas d'an zéro.
    \begin{enumerate}
        \item
            Le calendrier révolutionnaire commence en l'an \( 1792\). En quelle année sommes nous actuellement selon ce calendrier ?
        \item
            Le calendrier musulman commence en l'année \( 622\). Selon ce calendrier, en quelle année est mort Jules César (\( -44\) de notre calendrier) ?
        \item
            Le calendrier hébraïque commence en l'an \( \opp(3761)\) de notre calendrier. En quelle année sommes nous actuellement selon ce calendrier ?
    \end{enumerate}

\corrref{2smath-0042}
\end{exercice}
