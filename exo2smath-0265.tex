% This is part of Un soupçon de mathématique sans être agressif pour autant
% Copyright (c) 2015
%   Laurent Claessens
% See the file fdl-1.3.txt for copying conditions.

\begin{exercice}\label{exo2smath-0265}

    Calculer l'aire des figures suivantes :
    \begin{enumerate}
        \item
\begin{center}
   \input{Fig_RRWJooIPQhTnooZERO.pstricks}
\end{center}
\item
    \( DEFG\) est un parallélogramme, \( GF=3\) et \( HG=2\).
\begin{center}
   \input{Fig_RRWJooIPQhTnooONE.pstricks}
\end{center}
\item
    \( KM=\SI{3.5}{\centi\meter}\), \( MS=\SI{2}{\centi\meter}\) et \( KL=\SI{3}{\centi\meter}\).
\begin{center}
   \input{Fig_RRWJooIPQhTnooTWO.pstricks}
\end{center}
\item
    \( HI=\SI{3.3}{\milli\meter}\), \( IJ=\SI{3.6}{\milli\meter}\), \( JH=\SI{2.5}{\milli\meter}\) et \( TJ=\SI{2.4}{\milli\meter}\).
\begin{center}
   \input{Fig_RRWJooIPQhTnooTHREE.pstricks}
\end{center}

    \end{enumerate}

\corrref{2smath-0265}
\end{exercice}
