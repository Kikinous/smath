% This is part of Un soupçon de mathématique sans être agressif pour autant
% Copyright (c) 2012
%   Laurent Claessens
% See the file fdl-1.3.txt for copying conditions.

\begin{exercice}\label{exoPremiere-0054}

    Pour cet exercice nous reprenons un texte relativement long dans la variable \info{texte\_reference}. Écrire un programme qui affiche tous les enchaînement de deux lettres commençants par un «r».

    Inspirez vous de ceci :
    \begin{verbatim}
    >>> texte="la richesse est rarissime"
    >>> texte.split("r")
    ['la ', 'ichesse est ', 'a', 'issime']
    \end{verbatim}

    Notez en particulier que le premier éléments de la liste ne compte pas : «la» n'est pas précédé d'un «r».
    \begin{enumerate}
        \item
            Couper le texte avec «r» en utilisant \info{texte\_reference.split(``r'')}.
        \item
            Supprimer le premier élément de la liste obtenue
        \item
            Faire une boucle \info{for} sur la liste créée.
        \item
            À chaque étape, écrire la première lettre.
    \end{enumerate}

    Vous pouvez vous inspirer de ceci :
    \begin{verbatim}
    >>> A=texte.split("r")[1:]
    >>> for x in A:
    ...    print("r"+x[0])
    ... 
    ri
    ra
    ri
    \end{verbatim}

    Attention : si dans votre texte de référence il y a deux «r» d'affilée, un des éléments de la liste est vide. Il faut traiter ce cas. Faites des essais avec textes plus courts, par exemple étudiez comment se découpe «arrangement».

\corrref{Premiere-0054}
\end{exercice}
