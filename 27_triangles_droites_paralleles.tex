% This is part of Un soupçon de mathématique sans être agressif pour autant
% Copyright (c) 2014
%   Laurent Claessens
% See the file fdl-1.3.txt for copying conditions.


% This is part of Un soupçon de mathématique sans être agressif pour autant
% Copyright (c) 2014
%   Laurent Claessens
% See the file fdl-1.3.txt for copying conditions.

%--------------------------------------------------------------------------------------------------------------------------- 
\subsection*{Activité : droite des milieux}
%---------------------------------------------------------------------------------------------------------------------------

Tracer un triangle quelconque \( ABC\).  Nous nommons \( I\) le milieu de \( [AB]\), \( J\) le milieu de \( [BC]\) et \( K\) le milieu de \( [AC]\).

Tracer les droites \( (IJ)\), \( (JK)\) et \( (KI)\). Que remarquez-vous ? Comparer avec votre voisin.


\vspace{2cm}

Questions intermédiaires : 
\begin{itemize}
    \item  À quoi ressemble le quadrilatère \( KJBI\) ?
\end{itemize}

\begin{theorem}
    Si une droite passe par les milieux de deux côtés d'un triangle, alors elle est parallèle au troisième côté.
\end{theorem}

Sur le dessin suivant, \( (IJ)\parallel (AC)\).
\begin{center}
   \input{Fig_TBHUooKbohnF1.pstricks}
\end{center}

Toutes les droites.
\begin{center}
   \input{Fig_TBHUooKbohnF0.pstricks}
\end{center}

\begin{theorem}
Si, dans un triangle, un segment joint les milieux de deux côtés alors sa longueur est égale à la moitié de celle du troisième côté.
\end{theorem}

\begin{example}
    Calculer la longueur \( JK\) dans le cas suivant :
\begin{center}
   \input{Fig_NWVMooKshXYo.pstricks}
\end{center}
\end{example}

Le segment \( [JK]\) joint les milieux de deux côtés du triangle \( AND\). Le théorème des milieux affirme alors que la longueur \( JK\) vaut la moitié de celle du troisième côté. Donc
\begin{equation}
    JK=\frac{ AN }{2}=\frac{ 7.8 }{ 2 }=\SI{3.9}{\centi\meter}.
\end{equation}

% This is part of Un soupçon de mathématique sans être agressif pour autant
% Copyright (c) 2014
%   Laurent Claessens
% See the file fdl-1.3.txt for copying conditions.

%--------------------------------------------------------------------------------------------------------------------------- 
\subsection*{Activité : troisième théorème du milieu}
%---------------------------------------------------------------------------------------------------------------------------

Soit un triangle \( ABC\) et \( I\) le milieu de \( [AC]\).
\begin{enumerate}
    \item
        Combien de droites passant par \( I\) sont parallèles à \( (AB)\) ?
    \item
        Si nous notons \( K\) le milieu de \( [BC]\), que dire de la droite \( (IK)\) par rapport à \( (AB)\) ?
    \item
        Si une droite parallèle à \( (AB)\) passe par le point \( I\), est-ce qu'elle doit obligatoirement passer par \( K\) ?
\end{enumerate}


\begin{theorem}
    Si, dans un triangle, une droite passe par le milieu d'un côté et est parallèle à un deuxième côté alors elle passe par le milieu du troisième côté.
\end{theorem}

\begin{example}
Soit $TOR$ un triangle tel que $M$ soit le milieu du côté $[RO]$. La parallèle à $(TR)$ passant par $M$ coupe le côté $[OT]$ en $N$. Démontrer que $N$ est le milieu du côté $[OT]$.    

\begin{center}
   \input{Fig_DIPLooHILUUs.pstricks}
\end{center}

La droite \( (MN)\) passe par le milieu du côté \( [RO]\) et est parallèle au côté \( [RT]\), donc le troisième théorème des milieux implique que la droite \( (MN)\) coupe \( [TO]\) en son milieu.

Or \( (MN)\) coupe \( [TO]\) en \( N\). Donc \( N\) est le milieu de \( [TO]\).

\end{example}
