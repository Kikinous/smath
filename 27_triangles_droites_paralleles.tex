% This is part of Un soupçon de mathématique sans être agressif pour autant
% Copyright (c) 2014
%   Laurent Claessens
% See the file fdl-1.3.txt for copying conditions.


% This is part of Un soupçon de mathématique sans être agressif pour autant
% Copyright (c) 2014
%   Laurent Claessens
% See the file fdl-1.3.txt for copying conditions.

%--------------------------------------------------------------------------------------------------------------------------- 
\subsection*{Activité : droite des milieux}
%---------------------------------------------------------------------------------------------------------------------------

Tracer un triangle quelconque \( ABC\).  Nous nommons \( I\) le milieu de \( [AB]\), \( J\) le milieu de \( [BC]\) et \( K\) le milieu de \( [AC]\).

Tracer les droites \( (IJ)\), \( (JK)\) et \( (KI)\). Que remarquez-vous ? Comparer avec votre voisin.


\vspace{2cm}

Questions intermédiaires : 
\begin{itemize}
    \item  À quoi ressemble le quadrilatère \( KJBI\) ?
\end{itemize}

%+++++++++++++++++++++++++++++++++++++++++++++++++++++++++++++++++++++++++++++++++++++++++++++++++++++++++++++++++++++++++++ 
\section{Théorème des milieux}
%+++++++++++++++++++++++++++++++++++++++++++++++++++++++++++++++++++++++++++++++++++++++++++++++++++++++++++++++++++++++++++

\begin{theorem}
    Si une droite passe par les milieux de deux côtés d'un triangle, alors elle est parallèle au troisième côté.
\end{theorem}

Sur le dessin suivant, \( (IJ)\parallel (AC)\).
\begin{center}
   \input{Fig_TBHUooKbohnF1.pstricks}
\end{center}

Toutes les droites.
\begin{center}
   \input{Fig_TBHUooKbohnF0.pstricks}
\end{center}

\begin{theorem}
Si, dans un triangle, un segment joint les milieux de deux côtés alors sa longueur est égale à la moitié de celle du troisième côté.
\end{theorem}

\begin{example}
    Calculer la longueur \( JK\) dans le cas suivant :
\begin{center}
   \input{Fig_NWVMooKshXYo.pstricks}
\end{center}
\end{example}

Le segment \( [JK]\) joint les milieux de deux côtés du triangle \( AND\). Le théorème des milieux affirme alors que la longueur \( JK\) vaut la moitié de celle du troisième côté. Donc
\begin{equation}
    JK=\frac{ AN }{2}=\frac{ 7.8 }{ 2 }=\SI{3.9}{\centi\meter}.
\end{equation}
