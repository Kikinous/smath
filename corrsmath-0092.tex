% This is part of Un soupçon de mathématique sans être agressif pour autant
% Copyright (c) 2012
%   Laurent Claessens
% See the file fdl-1.3.txt for copying conditions.

\begin{corrige}{smath-0092}

    \begin{enumerate}
        \item
            Il suffit de remplacer \( a=6\), \( b=15\), \( c=-9\) dans les formules. On a
            \begin{equation}
                \Delta=(15)^2-4\times 6\times (-9)=441.
            \end{equation}
            Ensuite
            \begin{subequations}
                \begin{align}
                    x_1&=\frac{ -15+21 }{ 12 }=\frac{ 1 }{2}\\
                    x_2&=\frac{ -15-21 }{ 12 }=-3
                \end{align}
            \end{subequations}
        \item
            \begin{equation*}
                \begin{array}[]{|c||c|c|c|c|c|}
                    \hline
                    x&&-3&&\frac{ 1 }{2}&\\
                      \hline\hline
                      f(x)&+&0&-&0&+\\ 
                      \hline 
                       \end{array}
                   \end{equation*}
                   Dans ce tableau, les \( 0\) se mettent sur les racines, et nous savons que les \( +\) se mettent à l'extérieur des racines parce que \( a<0\).
               \item

                   La factorisation est
                   \begin{equation}
                       f(x)=6(x+3)(x-\frac{ 1 }{2}).
                   \end{equation}
                   
    \end{enumerate}

\end{corrige}
