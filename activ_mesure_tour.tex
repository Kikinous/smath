% This is part of Un soupçon de mathématique sans être agressif pour autant
% Copyright (c) 2015
%   Laurent Claessens
% See the file fdl-1.3.txt for copying conditions.

%--------------------------------------------------------------------------------------------------------------------------- 
\subsection*{Activité : mesure de la hauteur d'une tour}
%---------------------------------------------------------------------------------------------------------------------------

Un architecte veut mesurer la hauteur d'une tour en ruine dont l'entrée est interdite. Il ne lui est donc pas possible d'accéder ni au pied de la tour ni au sommet de cette tour. À neuf heures, l'ombre de la tour arrive en un point \( O\) situé à \SI{14}{\meter} de la base de la tour. L'architecte prend alors un bâton de \SI{1}{\meter} de haut et remarque qu'en avançant de deux mètres, le haut du bâton reçoit tout juste les rayons du Soleil.

Le dessin suivant illustre la situation sans être à l'échelle.

\begin{center}
   \input{Fig_OHDIooDdupWC.pstricks}
\end{center}

Quelle est la hauteur de la tour ?
