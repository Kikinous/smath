%This is part of Un soupçon de mathématique sans être agressif pour autant
% Copyright (c) 2012-2014
%   Laurent Claessens
% See the file fdl-1.3.txt for copying conditions.

%+++++++++++++++++++++++++++++++++++++++++++++++++++++++++++++++++++++++++++++++++++++++++++++++++++++++++++++++++++++++++++ 
\section{La fonction carré}
%+++++++++++++++++++++++++++++++++++++++++++++++++++++++++++++++++++++++++++++++++++++++++++++++++++++++++++++++++++++++++++

La fonction carré est la fonction définie sur \( \eR\) qui à \( x\) fait correspondre \( x^2\).

\begin{multicols}{2}

        \begin{equation*}
            \begin{array}[]{c|ccccc}
                x&-\infty&&0&&+\infty\\
                \hline
                &+\infty&&&&+\infty\\
                &&\searrow&&\nearrow&\\
                &&&0&&
            \end{array}
        \end{equation*}

        \columnbreak

        %The result is on figure \ref{LabelFigfigureXNAufCh}. % From file figureXNAufCh
        %\newcommand{\CaptionFigfigureXNAufCh}{<+Type your caption here+>}
        \begin{center}
\input{Fig_figureXNAufCh.pstricks}
        \end{center}
\end{multicols}

\begin{definition}
    Le graphe de la fonction carré est une \defe{parabole}{parabole}.
\end{definition}
%+++++++++++++++++++++++++++++++++++++++++++++++++++++++++++++++++++++++++++++++++++++++++++++++++++++++++++++++++++++++++++ 
\section{Double antécédent et solutions d'équations}
%+++++++++++++++++++++++++++++++++++++++++++++++++++++++++++++++++++++++++++++++++++++++++++++++++++++++++++++++++++++++++++

\begin{Aretenir}
    La fonction carré a pour principale caractéristique d'avoir \emph{deux} antécédents pour chaque image (à part pour zéro). Les antécédents du nombre \( a\) sont \( \sqrt{a}\) et \( -\sqrt{a}\).
\end{Aretenir}

\begin{example}
    Les carrés de \( 4\) et de \( -4\) sont tous les deux \( 16\).
\end{example}

Au niveau des inéquations, cela se ressent. La fonction \( x^2\) sera par exemple plus petite que \( 4\) non seulement pour les \( x\in\mathopen[ 0 , 3 \mathclose]\) mais aussi pour \( x\in\mathopen[ -3 , 0 \mathclose]\).

Nous voyons cela sur la figure \ref{LabelFigfigureEWDVDTS} qui montre les solution de \( f(x)\leq 4\). % From file figureEWDVDTS
\newcommand{\CaptionFigfigureEWDVDTS}{La résulution graphique d'une inéquation avec la fonction carré.}
\input{Fig_figureEWDVDTS.pstricks}

%+++++++++++++++++++++++++++++++++++++++++++++++++++++++++++++++++++++++++++++++++++++++++++++++++++++++++++++++++++++++++++ 
\section{Encadrement}
%+++++++++++++++++++++++++++++++++++++++++++++++++++++++++++++++++++++++++++++++++++++++++++++++++++++++++++++++++++++++++++

\begin{Aretenir}
    La fonction carré est décroissante sur \( \mathopen] -\infty , 0 \mathclose]\) et croissante sur \( \mathopen[ 0 , \infty [\).
    \begin{enumerate}
        \item
            Si \( 0<x<y\), alors \( 0<x^2<y^2\).
        \item
            Si \( x<y<0\), alors \( x^2>y^2>0\).
    \end{enumerate}
\end{Aretenir}
