%This is part of Un soupçon de mathématique sans être agressif pour autant
% Copyright (c) 2012-2014
%   Laurent Claessens
% See the file fdl-1.3.txt for copying conditions.

%+++++++++++++++++++++++++++++++++++++++++++++++++++++++++++++++++++++++++++++++++++++++++++++++++++++++++++++++++++++++++++ 
\section*{Activité}
%+++++++++++++++++++++++++++++++++++++++++++++++++++++++++++++++++++++++++++++++++++++++++++++++++++++++++++++++++++++++++++

%This is part of Un soupçon de mathématique sans être agressif pour autant
% Copyright (c) 2014
%   Laurent Claessens
% See the file fdl-1.3.txt for copying conditions.

Bertrand l'artisan vend des pots de terre cuite sur le marché. Chaque pot lui coûte \( 2\)€ de matériel. Au début de sa carrière il avait fixé le prix à \( 10\)€ et il vendait \( 8\) pots par semaine. Chaque semaine Bertrand baisse son prix de \( 0.5\)€ et vend alors \( 4\) pots supplémentaires.

\begin{enumerate}
    \item
        Donner le nombre de pots vendus ainsi que son bénéfice au début de sa carrière ainsi qu'après \( 1\), \( 2\) et \( 3\) semaines.
    \item
        Exprimer le nombre de pots vendus ainsi que le bénéfice après \( x\) semaines.
    \item
        Tracer un graphique.
    \item
        Après combien de baisses de prix Bertrand aura-t-il intérêt à cesser de baisser le prix ?
\end{enumerate}


%+++++++++++++++++++++++++++++++++++++++++++++++++++++++++++++++++++++++++++++++++++++++++++++++++++++++++++++++++++++++++++ 
\section{Paraboles}
%+++++++++++++++++++++++++++++++++++++++++++++++++++++++++++++++++++++++++++++++++++++++++++++++++++++++++++++++++++++++++++

\begin{definition}
    Une fonction \defe{polynôme du second degré}{polynôme!second degré} est une fonction définie sur \( \eR\) de la forme
    \begin{equation}
        f(x)=ax^2+bx+c
    \end{equation}
    avec \( a\neq 0\). Le graphe d'un polynôme du second degré est une \defe{paraboles}{parabole}. 

    Ce graphe sera noté \( y=ax^2+bx+c\).
\end{definition}

\newcommand{\CaptionFigLSaSLoS}{Quelque paraboles.}
\input{Fig_LSaSLoS.pstricks}


%--------------------------------------------------------------------------------------------------------------------------- 
%\subsection{Axe de symétrie}
%---------------------------------------------------------------------------------------------------------------------------

%L'axe de symétrie de la courbe représentative de \( f(x)=ax^2+bx+c\) se trouve en résolvant par rapport à \( s\) l'équation
%\begin{equation}
%    f(s+h)=f(s-h).
%\end{equation}

\begin{Aretenir}
    La parabole \( y=ax^2+bx+x\) est tournée vers le haut si \( a>0\) et tournée vers le bas si \( a<0\).

    Le sommet de la parabole \( y=ax^2+bx+c\) se trouve à l'abscisse \( x_S=-\frac{ b }{ 2a }\) et la droite verticale \( x=-\frac{ -b }{ 2a }\) est un axe de symétrie de la parabole.
\end{Aretenir}
Quelque unes sont dessinées à la figure \ref{LabelFigLSaSLoS}. % From file LSaSLoS

\clearpage

%+++++++++++++++++++++++++++++++++++++++++++++++++++++++++++++++++++++++++++++++++++++++++++++++++++++++++++++++++++++++++++ 
\section{Tableau de variation}
%+++++++++++++++++++++++++++++++++++++++++++++++++++++++++++++++++++++++++++++++++++++++++++++++++++++++++++++++++++++++++++

Les tableaux de variations sont
\begin{multicols}{2}

    \begin{center}
        Si \( a>0\)

        \begin{equation*}
            \begin{array}[]{c|ccccc}
                x&&&-\frac{ b }{ 2a }&&\\
                \hline
                &\infty&&&&\infty\\
                f(x)&&\searrow&&\nearrow&\\
                &&&f\left( -\frac{ b }{ 2a } \right)&&\\
            \end{array}
        \end{equation*}
        
    \end{center}

    \columnbreak

    \begin{center}
        Si \( a<0\)


        \begin{equation*}
            \begin{array}[]{c|ccccc}
                x&&&-\frac{ b }{ 2a }&&\\
                \hline
                &&&f\left( -\frac{ b }{ 2a } \right)&&\\
                f(x)&&\nearrow&&\searrow&\\
                &-\infty&&&&-\infty\\
            \end{array}
        \end{equation*}
    \end{center}
\end{multicols}

%+++++++++++++++++++++++++++++++++++++++++++++++++++++++++++++++++++++++++++++++++++++++++++++++++++++++++++++++++++++++++++ 
\section{La fonction carré}
%+++++++++++++++++++++++++++++++++++++++++++++++++++++++++++++++++++++++++++++++++++++++++++++++++++++++++++++++++++++++++++

La fonction carré est la fonction définie sur \( \eR\) qui à \( x\) fait correspondre \( x^2\).

\begin{multicols}{2}

        \begin{equation*}
            \begin{array}[]{c|ccccc}
                x&-\infty&&0&&+\infty\\
                \hline
                &+\infty&&&&+\infty\\
                &&\searrow&&\nearrow&\\
                &&&0&&
            \end{array}
        \end{equation*}

        \columnbreak

        %The result is on figure \ref{LabelFigfigureXNAufCh}. % From file figureXNAufCh
        %\newcommand{\CaptionFigfigureXNAufCh}{<+Type your caption here+>}
        \begin{center}
\input{Fig_figureXNAufCh.pstricks}
        \end{center}
\end{multicols}

% Plus besoin de ça parce que la parabole en générale est vue avant la fonction carré.
%\begin{definition}
%    Le graphe de la fonction carré est une \defe{parabole}{parabole}.
%\end{definition}

%+++++++++++++++++++++++++++++++++++++++++++++++++++++++++++++++++++++++++++++++++++++++++++++++++++++++++++++++++++++++++++ 
\section{Double antécédent et solutions d'équations}
%+++++++++++++++++++++++++++++++++++++++++++++++++++++++++++++++++++++++++++++++++++++++++++++++++++++++++++++++++++++++++++

\begin{Aretenir}
    La fonction carré a pour principale caractéristique d'avoir \emph{deux} antécédents pour chaque image (à part pour zéro). Les antécédents du nombre \( a\) sont \( \sqrt{a}\) et \( -\sqrt{a}\).
\end{Aretenir}

\begin{example}
    Les carrés de \( 4\) et de \( -4\) sont tous les deux \( 16\).
\end{example}

Au niveau des inéquations, cela se ressent. La fonction \( x^2\) sera par exemple plus petite que \( 4\) non seulement pour les \( x\in\mathopen[ 0 , 3 \mathclose]\) mais aussi pour \( x\in\mathopen[ -3 , 0 \mathclose]\).

Nous voyons cela sur la figure \ref{LabelFigfigureEWDVDTS} qui montre les solution de \( f(x)\leq 4\). % From file figureEWDVDTS
\newcommand{\CaptionFigfigureEWDVDTS}{La résulution graphique d'une inéquation avec la fonction carré.}
\input{Fig_figureEWDVDTS.pstricks}

%+++++++++++++++++++++++++++++++++++++++++++++++++++++++++++++++++++++++++++++++++++++++++++++++++++++++++++++++++++++++++++ 
\section{Encadrement}
%+++++++++++++++++++++++++++++++++++++++++++++++++++++++++++++++++++++++++++++++++++++++++++++++++++++++++++++++++++++++++++

\begin{Aretenir}
    La fonction carré est décroissante sur \( \mathopen] -\infty , 0 \mathclose]\) et croissante sur \( \mathopen[ 0 , \infty [\).
    \begin{enumerate}
        \item
            Si \( 0<x<y\), alors \( 0<x^2<y^2\).
        \item
            Si \( x<y<0\), alors \( x^2>y^2>0\).
    \end{enumerate}
\end{Aretenir}

