% This is part of Un soupçon de mathématique sans être agressif pour autant
% Copyright (c) 2014
%   Laurent Claessens
% See the file fdl-1.3.txt for copying conditions.

\begin{corrige}{smath-0896}

    Il suffit de remplacer tous les \( a\) par \( 3\) et tous les \( b\) par \( -2\) en faisant attention aux multiplications sous-entendues :
        \begin{enumerate}
            \item
                Attention : la notation «$2a$» signifie \( 2\times a\). Donc \( 2a=2\times 3=6\)
            \item
                \( 7b=7\times (-2)=-14\)
            \item
                \( 3\times a+4=3\times 3+4=13\)
            \item
                \(  -4\times a=-4\times 3=-12\)
            \item 
                \( \dfrac{ a\times b }{ -12 }=\dfrac{ 3\times (-2) }{ -12 }=\dfrac{ -6 }{ -12 }=\dfrac{ 1 }{2}\).

                La dernière simplification mérite une explication. D'abord \( \dfrac{ -6 }{ -12 }\) est un quotient de deux nombres de même signe; le résultat sera donc positif. Nous pouvons donc écrire
                \begin{equation}
                    \frac{ -6 }{ -12 }=\frac{ 6 }{ 12 }.
                \end{equation}
                Ensuite \( 6\) et \( 12\) peuvent tous les deux être divisé par \( 6\). Nous écrivons la simplification
                \begin{equation}
                    \frac{ 6 }{ 12 }=\frac{ 6\div 6 }{ 12\div 6 }=\frac{1}{ 2 }.
                \end{equation}
        \end{enumerate}

\end{corrige}
