% This is part of Un soupçon de mathématique sans être agressif pour autant
% Copyright (c) 2015
%   Laurent Claessens
% See the file fdl-1.3.txt for copying conditions.

\begin{corrige}{2smath-0237}

    \begin{enumerate}
        \item
            Un triangle équilatéral possède toujours trois angles de \SI{60}{\degree}. Il est donc impossible qu'il soit un triangle rectangle.
        \item
            Cela est possible. Il suffit par exemple de faire un angle droit entre deux segments de \SI{5}{\centi\meter} et de rejoindre les bouts :

\begin{center}
   \input{Fig_OAHLooOwLezW.pstricks}
\end{center}

\item
     
    Si un triangle possède des angles de \SI{60}{\degree} et \SI{70}{\degree}, le troisième angle doit être \( 180-60-70=50\). Or un triangle isocèle doit posséder au moins deux angles égaux. Donc impossible.

    \end{enumerate}

\end{corrige}
