% This is part of Un soupçon de mathématique sans être agressif pour autant
% Copyright (c) 2014
%   Laurent Claessens
% See the file fdl-1.3.txt for copying conditions.

\begin{exercice}\label{exosmath-0891}

    L'énoncé suivant est faux : «Si la somme des chiffres d'un nombre vaut \( 7\) alors il est divisible par \( 7\)».  Lequel des nombres suivants en est un contre-exemple ?
    \begin{multicols}{2}
        \begin{enumerate}
            \item
                \( 49\)
            \item
                \( 214\)
            \item
                \( 77\) 
            \item
                \( 322\)
        \end{enumerate}
    \end{multicols}
    Donner un autre contre-exemple; expliquer.

\corrref{smath-0891}
\end{exercice}
