% This is part of Un soupçon de mathématique sans être agressif pour autant
% Copyright (c) 2013
%   Laurent Claessens
% See the file fdl-1.3.txt for copying conditions.

\begin{exercice}\label{exosmath-0495}

    Un amateur de vin possédant \( 200\) bouteilles vient de s'acheter une seconde cave et veut y mettre la moitié de sa collection. Sa méthode de travail est un peu particulière : chaque semaine il prend \( 10\) bouteilles dans la première cave, en boit une, et met les neuf autres dans la seconde cave. Après combien de semaines les deux caves seront à égalité ? (la réponse n'est peut être pas un nombre entier; ici une approximation numérique a un sens)

\corrref{smath-0495}
\end{exercice}
