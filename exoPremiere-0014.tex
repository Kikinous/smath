% This is part of Un soupçon de mathématique sans être agressif pour autant
% Copyright (c) 2012
%   Laurent Claessens
% See the file fdl-1.3.txt for copying conditions.

\begin{exercice}\label{exoPremiere-0014}

    Chacune des questions possède une, deux ou trois bonnes réponses. Lesquelles ?
    \begin{enumerate}
        \item
        Dans une classe de \( 30\) élèves, \( 21\) ont \( 17\) ans, \( 20\%\) ont \( 18\) ans et le reste a \( 16\) ans.  Alors
        \begin{enumerate}
            \item

         \( 3\%\) des élèves ont \( 16\) ans
     \item
         \( 75\%\) des élèves ont \( 17\) ans.
     \item
         \( 3\) élèves ont \( 16\) ans
                
        \end{enumerate}
    \item
        Un bracelet de \( \unit{80}{\gram}\) contient \( \unit{73.6}{\gram}\) d'or pur. Quelle est la proportion d'or pur dans le bracelet ?
        \begin{enumerate}
            \item
        \( \frac{ 23 }{ 25 }\)
    \item
        \( 0.92\%\)
    \item
        \( 0.92\)

        \end{enumerate}

    \item
        \( 60\) personnes, soit \( 30\%\) d'une population \( P\) vont travailler en bus. Combien de personnes comprend la population ?
        \begin{enumerate}
            \item
                $90$
            \item
                \( 200\)
            \item
                \( 125\).
        \end{enumerate}
    \item
        Un lycée a \( 400\) élèves. Parmi ces élèves, \( 30\%\) vont en premières STMG et \( 60\%\) de ces STMG étudient l'anglais. Combien d'élèves de STMG étudient une autre langue ?
        \begin{enumerate}
            \item
                72
            \item
                144
            \item
                48
        \end{enumerate}
    \item
        Un article coûte \( n\) euros et augmente de \( 5\%\). Quel est son nouveau prix ?
        \begin{enumerate}
            \item
                \( 1.05n\)
            \item
                \( 0.05+n\)
            \item
                \( 105n/100\).
        \end{enumerate}
    \item
        L'ancien prix d'un objet était \( Q\). Le nouveau prix est \( 0.75Q\). Qu'a fait le prix ?
        \begin{enumerate}
            \item
                Augmenté de \( 75\%\).
            \item
                Diminué de \( 75\%\).
            \item
                Diminué de \( 25\%\).
        \end{enumerate}
    \item
        Une somme de \( 2000\) euros est placée un an et devient \( 2100\) euros. Quel était le taux ?
        \begin{enumerate}
            \item
                \( 5\%\)
            \item
                \( 10\%\)
            \item
                \( 4.76\%\).
        \end{enumerate}
        Le prix d'une matière première a baissé de \( 5\%\) et vaut maintenant \( 1000\) euros la tonne. Quel était (en arrondi) le prix original ?
        \begin{enumerate}
            \item
                \( 1050\) euros.
            \item
                \( 1053\) euros.
            \item
                \( 950\) euros.
        \end{enumerate}
    \end{enumerate}
        
\corrref{Premiere-0014}
\end{exercice}
