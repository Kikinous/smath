% This is part of Un soupçon de mathématique sans être agressif pour autant
% Copyright (c) 2015
%   Laurent Claessens
% See the file fdl-1.3.txt for copying conditions.

\begin{corrige}{2smath-0313}

    \begin{enumerate}
        \item
            Faux. Par exemple l'aire d'un cercle de rayon \( 3\) est de \( \pi\times 3\times 3=9\pi\) alors que si le rayon est le double (c'est à dire \( 6\)), l'aire devient \( \pi\times 6\times 6=36\pi\). Clairement \( 36\pi\) n'est pas le double de \( 9\pi\).
        \item
            Vrai. Si on veut une justification par un exemple (ce qui n'est pas une preuve), on peut calculer le périmètre d'un cercle de rayon \( 3\) (égal à \( 2\times 3\times\pi=6\pi\)) et celui d'un cercle de rayon \( 6\), c'est à dire \( 2\times 6\times \pi=12\pi\). On voit que c'est le double.

            Pour une justification complète, le périmètre d'un cercle de rayon \( r\) est \( 2\pi r\). Celui d'un cercle \( 2r\) est \( 2\times \pi\times 2r=4\pi r\), c'est à dire le double.
            
        \item
            Faux. Si \( x=-10\) par exemple, \( 4+x=-6\) qui n'est pas plus grand que \( 4\). Plus généralement dès que \( x\) est négatif, \( x+4\) est plus petit que \( 4\).
    \end{enumerate}

\end{corrige}
