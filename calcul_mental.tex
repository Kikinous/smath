% This is part of Un soupçon de mathématique sans être agressif pour autant
% Copyright (c) 2012
%   Laurent Claessens
% See the file fdl-1.3.txt for copying conditions.

%\Exo{Premiere-0020}                                                                                                                                    
%\Exo{Premiere-0021}                                                                                                                                    
%\Exo{Premiere-0022}                                                                                                                                
%\Exo{Seconde-0018}
%\Exo{Seconde-0041}
%\Exo{Premiere-0064}
%\Exo{Seconde-0065}
%\Exo{smath-0014}
%\Exo{smath-0017}

%TODO : vider ces fichiers et les remettre en disponibilité parce qu'ils sont tous dans exosmath-0018.tex
% (18 octobre 2012)

%---------------------------------------------------------------------------------------------------------------------------
\subsection{Choses basiques}
%---------------------------------------------------------------------------------------------------------------------------

\Exo{smath-0018}
\Exo{Premiere-0019}                                                                                                                              

%---------------------------------------------------------------------------------------------------------------------------
\subsection{Égalité de fonctions}
%---------------------------------------------------------------------------------------------------------------------------

Voir livre math'x aux pages 85 et 87.

Les identités remarquables :
\begin{subequations}
    \begin{align}
        (a+b)^2=a^2+2ab+b^2\\
        (a-b)^2=a^2-2ab+b^2\\
        (a+b)(a-b)=a^2-b^2.
    \end{align}
\end{subequations}

\Exo{smath-0042}
\Exo{smath-0043}
\Exo{smath-0044}
\Exo{smath-0045}
\Exo{smath-0046}

%---------------------------------------------------------------------------------------------------------------------------
\subsection{Factorisation}
%---------------------------------------------------------------------------------------------------------------------------

\Exo{smath-0047}
\Exo{smath-0048}
