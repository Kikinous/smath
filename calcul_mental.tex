% This is part of Un soupçon de mathématique sans être agressif pour autant
% Copyright (c) 2013
%   Laurent Claessens
% See the file fdl-1.3.txt for copying conditions.


\documentclass{beamer}

\usepackage[utf8]{inputenc}
\usepackage[T1]{fontenc}

\usepackage{textcomp}
\usepackage{lmodern}
\usepackage[english,frenchb]{babel}

\usepackage{calc}   % Les dépendances de phystricks si on n'utilise que le pdf.
\usepackage[cdot,thinqspace,amssymb]{SIunits} 
\usepackage{multicol}
\usepackage{wrapfig}
\usepackage{framed}
% Ce bout de code provient de BrunoJ
% https://brunoj.wordpress.com/2009/10/08/latex-the-framed-minipage/
\newsavebox{\fmbox}
 \newenvironment{fmpage}[1]
 {\begin{lrbox}{\fmbox}\begin{minipage}{#1}}
     {\end{minipage}\end{lrbox}
     \fbox{\usebox{\fmbox}}
 }


\usetheme{default}
\begin{document}



\begin{frame}{Calcul mental 6}
    \begin{enumerate}
        \item
            \pause
            Résoudre \( \frac{1}{ x }=4\)
        \item
            \pause
            Simplifier
            \begin{equation*}
                \frac{ x+2 }{ (x+2)(x-3) }.
            \end{equation*}
        \item
            \pause
            Qu'affiche l'algorithme suivant si on entre la valeur \( 4\) ?

\begin{fmpage}{0.9\linewidth}

    Demander \( x\)

    Si \( x < 2\) alors

    \hspace{1cm} \( P\) prend la valeur \( 2x-7 \)

    Si \( n >= 2 \) alors

    \hspace{1cm} \( P\) prend la valeur \( \frac{ 3 }{ 8 }x\)

    Afficher \( P\)

\end{fmpage}
        \item
            \pause
                Résoudre \( 2x+60=4\).

    \end{enumerate}
\end{frame}

\begin{frame}{Calcul mental 5}

    \begin{enumerate}
        \item

    \pause
            Écrire sous forme d'une somme : \( (x-4)^2\)

        \item
    \pause
            Simplifier :
            \begin{equation}
                \frac{ 4bx+b^2 }{ 2b }.
            \end{equation}
        \item

    \pause
            Résoudre : \( (x+2)(x-1)=0\).

        \item
    \pause
            Quelle est la longueur du segment joignant \( A(3;7)\) et \( B(4;10)\) ?
        \item
    \pause
            Si \( f(x)=x^2-2\), que vaut \( f(-1)\) ?
            
    \end{enumerate}
\end{frame}

    

\begin{frame}{Calcul mental 4}

    \begin{enumerate}
        \item

    \pause
            \begin{multicols}{2}

                La figure ci-contre est un cube. Quelle est la nature du triangle \( AHD\) ?

                \columnbreak

                \input{Fig_MUriGyU.pstricks}
            \end{multicols}

        \item
    \pause
            Écrire sous forme d'une somme : $(x+3)^2$.
        \item
            \pause
            L'aire d'un carré est de \unit{11}{\centi\meter\squared}. Quelle est la longueur exacte de son côté ?
        \item
            \pause
            Simplifier 
            \begin{equation}
                \frac{ 8ax+a^2 }{ 2a }.
            \end{equation}
            
    \end{enumerate}
\end{frame}


\begin{frame}{Calcul mental 3}
    \pause
    \begin{enumerate}
        \item
            Résoudre \( (x+1)(x-17)=0\)
            \pause
        \item
            Simplifier
            \begin{equation}
                \frac{ 4x+8a }{ 2 }.
            \end{equation}
            \pause
        \item
            Si \( f(x)=x^2-x\), que vaut \( f(4)\) ?
            \pause
        \item
            Quelle est la distance entre \( (0;4)\) et \( (2;-2)\) ?
            \pause
        \item
            La droite représentative de la fonction \( f(x)=3x+2\) est-elle parallèle à celle de \( g(x)=-3x+7\) ?
    \end{enumerate}
\end{frame}

\begin{frame}{Calcul mental 2}
    \pause
    \begin{enumerate}
        \item
            Simplifier \( \sqrt{72}\)
            \pause
        \item
            Calculer la distance entre \( (0;10)\) et \( (2;0)\).
            \pause
        \item
            Résoudre \( \frac{1}{ x }=\frac{ 2 }{ 3 }\).
            \pause
        \item
            Simplifier 
            \begin{equation*}
                \frac{ 12x+3b }{ 21 }.
            \end{equation*}
            \pause
        \item
            Calculer le milieu du segment entre \( (1;3)\) et \( (-3;8)\).
            \pause
        \item
            Résoudre
            \begin{equation*}
                (x+2)(x-1)=0.
            \end{equation*}
            \pause
    \end{enumerate}
    \begin{center}
        FIN pour aujourd'hui.
    \end{center}
\end{frame}

\begin{frame}{Calcul mental 1}

    \pause
    \begin{enumerate}
        \item
            Simplifier : \( \sqrt{12}\)

            \phantom{\( 2\sqrt{3}\)}

            \pause
        \item
            Les coordonnées du milieu entre \( (1;-2)\) et \( (6;4)\)
            \pause
        \item
            Simplifier 
            \begin{equation*}
                \frac{ 2xy }{ 4 }
            \end{equation*}
            \pause
        \item
            Simplifier
            \begin{equation*}
                \frac{ 8x+4a }{ 2 }
            \end{equation*}
            \pause
        \item
            Résoudre : \( 3x=12\).
            \pause
        \item
            Résoudre : \( 2x-6=4\).
    \end{enumerate}

\end{frame}

\end{document}
