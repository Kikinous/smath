% This is part of Un soupçon de mathématique sans être agressif pour autant
% Copyright (c) 2015
%   Laurent Claessens
% See the file fdl-1.3.txt for copying conditions.

% This is part of Un soupçon de mathématique sans être agressif pour autant
% Copyright (c) 2015
%   Laurent Claessens
% See the file fdl-1.3.txt for copying conditions.

%--------------------------------------------------------------------------------------------------------------------------- 
\subsection*{Activité : triangle de Sierpiński}
%---------------------------------------------------------------------------------------------------------------------------

Les triangles de Sierpiński se construisent de la façon suivante : le triangle de Sierpiński numéro zéro est un simple triangle gris. Les suivants s'obtiennent en supprimant à chaque étape le triangle «central» de chacun de triangles gris formant l'étape précédente.

En voici quelque uns :

%The result is on figure \ref{LabelFigMOCGooKjSrVV}. % From file MOCGooKjSrVV
\newcommand{\CaptionFigMOCGooKjSrVV}{Quelque triangles de Sierpiński}
\input{Fig_MOCGooKjSrVV.pstricks}

\begin{enumerate}
    \item
        De combien de triangles gris est formé le triangle de Sierpiński numéro \( 1\), \( 2\), \( 3\) ?
    \item
        De combien de triangles gris est formé le triangle de Sierpiński numéro \( 5\), \( 10\), \( 20\) ?
\end{enumerate}


%+++++++++++++++++++++++++++++++++++++++++++++++++++++++++++++++++++++++++++++++++++++++++++++++++++++++++++++++++++++++++++ 
\section{Puissances positives}
%+++++++++++++++++++++++++++++++++++++++++++++++++++++++++++++++++++++++++++++++++++++++++++++++++++++++++++++++++++++++++++

\begin{definition}
    Pour tout nombre entier positif \( n\) et tout nombre relatif \( a\), on note
    \begin{equation}
        a^n=\underbrace{a\times a \times a\times\ldots\times a}_{\text{\( n\) facteurs}}.
    \end{equation}
\end{definition}


\begin{propriete}
    Si \( a\) est un nombre relatif et si \( n\) et \( m\) sont positifs,
    \begin{equation}
        a^n\times a^m=a^{m+n}.
    \end{equation}
\end{propriete}

% This is part of Un soupçon de mathématique sans être agressif pour autant
% Copyright (c) 2015
%   Laurent Claessens
% See the file fdl-1.3.txt for copying conditions.

%--------------------------------------------------------------------------------------------------------------------------- 
\subsection*{Activité : puissances négatives}
%---------------------------------------------------------------------------------------------------------------------------

\begin{enumerate}
    \item
        Par combien doit-on diviser \( 5^8\) pour obtenir \( 5^7\) ?
    \item
        Combien vaut \( 3^{12}\div 3^{4}\) ?
    \item
        Combien vaut \( 7^{6}\div 7^6\) ?
\end{enumerate}


%+++++++++++++++++++++++++++++++++++++++++++++++++++++++++++++++++++++++++++++++++++++++++++++++++++++++++++++++++++++++++++ 
\section{Puissances négatives}
%+++++++++++++++++++++++++++++++++++++++++++++++++++++++++++++++++++++++++++++++++++++++++++++++++++++++++++++++++++++++++++

Pour la cohérence, on définit :
\begin{definition}
    Pour tout nombre \( a\), on définit
    \begin{equation}
        a^0=1.
    \end{equation}
    et pour tout entier positif \( n\),
    \begin{equation}
        a^{-n}=\frac{1}{ a^n }.
    \end{equation}
\end{definition}

