% This is part of Un soupçon de mathématique sans être agressif pour autant
% Copyright (c) 2013
%   Laurent Claessens
% See the file fdl-1.3.txt for copying conditions.

\begin{exercice}\label{exosmath-0411}

\begin{wrapfigure}{r}{5.0cm}
            \vspace{-0.5cm}        % à adapter.
                \centering
                    \input{Fig_VMNerGf.pstricks}
                \end{wrapfigure}

    Deux joueurs de jeu de combat tirent des rayons de plasma. Si les rayons s'intersectent, il se produit une explosion qui tue tout le monde dans un rayon de \unit{4}{\meter}. 

    Les joueurs \( A\) et \( B\) sont dans la situation ci-contre (le grillage est donné en mètres) et tirent dans les directions indiquées par les vecteurs dessinés. Vont-ils survivre ? Répondre par le calcul (un dessin est le bienvenu, mais ne consitue pas une réponse à part entière). 

    Indice : pensez au théorème de Pythagore pour conclure.

\corrref{smath-0411}
\end{exercice}
