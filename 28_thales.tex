% This is part of Un soupçon de mathématique sans être agressif pour autant
% Copyright (c) 2015
%   Laurent Claessens
% See the file fdl-1.3.txt for copying conditions.


% This is part of Un soupçon de mathématique sans être agressif pour autant
% Copyright (c) 2014-2015
%   Laurent Claessens
% See the file fdl-1.3.txt for copying conditions.

%--------------------------------------------------------------------------------------------------------------------------- 
\subsection*{Activité : la barre du A}
%---------------------------------------------------------------------------------------------------------------------------

    Baptiste le graphiste veut dessiner un «A» ayant à peu près la forme suivante :

\begin{center}
   \input{Fig_XFMTooSCJlTh.pstricks}
\end{center}

    À quelle hauteur doit-il placer la barre horizontale de son «A» de telle sorte que sa longueur soit le tiers de la base du «A» ?



\vspace{1cm}

Note : mettre les parallèles en couleur.

\begin{theorem}[Théorème de Thalès]
    Les droites \( (IJ)\) et \( (CB)\) sont parallèles et coupent les droites \( (AC)\) et \( (AB)\).

\begin{center}
   \input{Fig_IQKJooUmvFBs.pstricks}
\end{center}

Alors nous avons les rapports de longueurs suivants :
\begin{equation}
    \frac{ AJ }{ AC }=\frac{ AI }{ AB }=\frac{ JI }{ CB }.
\end{equation}

\end{theorem}

Ce théorème permet de trouver quelque longueurs en en connaissant d'autres.

\begin{example}
    
    Sur le dessin suivant, \( AB=12\) et la droite \( (KL)\) est parallèle à la droite \( (BC)\).
\begin{center}
   \input{Fig_UDKYooKlVbjh.pstricks}
\end{center}
Quelle est la longueur du segment \( [CB]\) ?

Nous écrivons les égalités du théorème de Thalès :
\begin{equation}
    \frac{ BA }{ AL }=\frac{ CB }{ KL }=\frac{ AB }{ AL }
\end{equation}
et nous remplaçons les données :
\begin{equation}
    \frac{ BA }{ AL }=\frac{ CB }{ 3 }=\frac{ 12 }{ 2 }.
\end{equation}
Donc \( \frac{ CB }{ 3 }=6\), ce qui donne \( CB=18\).

\end{example}

% This is part of Un soupçon de mathématique sans être agressif pour autant
% Copyright (c) 2015
%   Laurent Claessens
% See the file fdl-1.3.txt for copying conditions.

%--------------------------------------------------------------------------------------------------------------------------- 
\subsection*{Activité : mesure de la hauteur d'une tour}
%---------------------------------------------------------------------------------------------------------------------------

Un architecte veut mesurer la hauteur d'une tour en ruine dont l'entrée est interdite. Il ne lui est donc pas possible d'accéder ni au pied de la tour ni au sommet de cette tour. À neuf heures, l'ombre de la tour arrive en un point \( O\) situé à \SI{14}{\meter} de la base de la tour. L'architecte prend alors un bâton de \SI{1}{\meter} de haut et remarque qu'en avançant de trois mètres, le haut du bâton reçoit tout juste les rayons du Soleil.

Le dessin suivant illustre la situation sans être à l'échelle.

\begin{center}
   \input{Fig_OHDIooDdupWC.pstricks}
\end{center}

Quelle est la hauteur de la tour ?



