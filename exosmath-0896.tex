% This is part of Un soupçon de mathématique sans être agressif pour autant
% Copyright (c) 2014
%   Laurent Claessens
% See the file fdl-1.3.txt for copying conditions.

\begin{exercice}[\ldots\ldots/4]\label{exosmath-0896}

    Que valent les expressions suivantes lorsque \( a=3\) et \( b=-2\) ?
    \begin{multicols}{2}
        \begin{enumerate}
            \item
                \( 2a\)
            \item
                \( 3\times a+4\)
            \item
                \(  -4\times a\)
            \item
                \( (a+1)\times (b-1)\)
            \item   \label{ItemMMFooWmiRCP}
                \( \dfrac{ a\times b }{ -12 }\)
        \end{enumerate}
    \end{multicols}
    Pour la question \ref{ItemMMFooWmiRCP}, simplifier la fraction obtenue.

\corrref{smath-0896}
\end{exercice}
