% This is part of Un soupçon de mathématique sans être agressif pour autant
% Copyright (c) 2014
%   Laurent Claessens
% See the file fdl-1.3.txt for copying conditions.

\begin{exercice}\label{exosmath-0965}

    Norbert le propriétaire possède un terrain de \SI{3}{\meter} de la large et \SI{10}{\meter} de long. Il décide de consacrer \( x\) mètres à une terrasse et le reste à une piscine. Le but de cet exercice est de déterminer \( x\) de telle sorte à avoir une piscine d'au moins \SI{12}{\meter\squared}.

\begin{center}
   \input{Fig_FIOJooSmlhXR.pstricks}
\end{center}

Répondre en justifiant aux questions suivantes.
\begin{enumerate}
    \item
        Déterminer la longueur de la piscine en fonction de \( x\).
    \item
        Déterminer l'aire de la piscine en fonction de \( x\).
    \item
        Pour quelle valeur de \( x\), l'aire de la piscine vaut \SI{12}{\meter\squared} ?
\end{enumerate}

\corrref{smath-0965}
\end{exercice}
