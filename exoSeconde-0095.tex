% This is part of Un soupçon de mathématique sans être agressif pour autant
% Copyright (c) 2012
%   Laurent Claessens
% See the file fdl-1.3.txt for copying conditions.

\begin{exercice}\label{exoSeconde-0095}

    \begin{multicols}{2}

        Le dessin ci-contre représente un cube de \unit{7}{\centi\meter} de côté. Le point \( I\) est le centre du carré \( ABCD\), le point \( J\) est le milieu du segment \( [AD]\), et le point \( K\) est le milieu du segment \( [EH]\).

        \begin{enumerate}
    \item
        Dessiner le triangle \( IJK\) en vraie grandeur.
            \item
        Quelle est l'aire du triangle \( IJK\) ?
        \end{enumerate}

        \columnbreak

        \begin{center}
%The result is on figure \ref{LabelFigSurfaceCubeXlLEEy}. % From file SurfaceCubeXlLEEy
%\newcommand{\CaptionFigSurfaceCubeXlLEEy}{<+Type your caption here+>}
\input{Fig_SurfaceCubeXlLEEy.pstricks}
        \end{center}

    \end{multicols}
    
    Quelle est l'aire du triangle \( CGF\) ?

\corrref{Seconde-0095}
\end{exercice}
