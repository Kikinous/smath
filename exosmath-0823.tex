% This is part of Un soupçon de mathématique sans être agressif pour autant
% Copyright (c) 2014
%   Laurent Claessens
% See the file fdl-1.3.txt for copying conditions.

\begin{exercice}[\ldots\ldots/4]\label{exosmath-0823}

    \begin{enumerate}
        \item
            Tracer, avec la règle et un compas, un triangle \( KLM\) dont les longueurs sont \( KL=\SI{10}{\centi\meter}\), \( LM=\SI{6}{\centi\meter}\) et \( LM=\unit{6}{\centi\meter}\). Laisser les traits de construction.
        \item
            Mesurer les angles avec le rapporteur, et les noter sur le dessin.
        \item
            Pour un devoir Alysée doit calculer la somme des angles d'un triangle \( ABC\) dont les mesures sont \( AB=\SI{12}{\centi\meter}\), \( AC=\SI{25}{\centi\meter}\) et \( BC=\SI{10}{\centi\meter}\). Que pensez-vous de ce devoir ?
    \end{enumerate}

\corrref{smath-0823}
\end{exercice}
