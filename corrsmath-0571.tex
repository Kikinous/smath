% This is part of Un soupçon de mathématique sans être agressif pour autant
% Copyright (c) 2013
%   Laurent Claessens
% See the file fdl-1.3.txt for copying conditions.

\begin{corrige}{smath-0571}

    \begin{enumerate}
        \item
            Le seul facteur commun à \( 3x^2\), \( 7x\) et \( 18x\) est \( x\). Nous pouvons donc simplifier par \( x\) :
            \begin{equation}
                \frac{ 3x^2+7x }{ 18x }=\frac{ 3x+7 }{ 18 }.
            \end{equation}
        \item
            L'aire d'un rectangle est donné par le produit des deux longueurs. La longueur de notre rectangle est \( \frac{ 23 }{ 6 }\)\centi\meter. Attention : c'est une valeur exacte. La valeur \( 3.8333\) n'est pas exacte.
        \item 
            Pour le vérifier il suffit de remplacer \( x\) par \( 1\) dans l'expression \( 5x^2(x-1)\) et voir si ça fait zéro. La réponse est oui parce que \( x-1=0\) lorsque \( x=1\).
        \item
            Il faut prendre quelque nombres très petits pour «plomber» la moyenne : \( 0\), \( 1000\), \( 1000\). La médiane est \( 1000\) et la moyenne est environ \( 666\).
        \item
            Il faut remplacer \( x\) par \( -4\) en n'oubliant pas que \( -(-4)=4\). Nous avons
            \begin{equation}
                f(-4)=\frac{ -4+3 }{ -(-4) }=\frac{ -1 }{ 4 }=-\frac{1}{ 4 }.
            \end{equation}
    \end{enumerate}

\end{corrige}
