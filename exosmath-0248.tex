% This is part of Un soupçon de mathématique sans être agressif pour autant
% Copyright (c) 2013
%   Laurent Claessens
% See the file fdl-1.3.txt for copying conditions.

\begin{exercice}[\cite{HYlEAxq}]\label{exosmath-0248}

Une entreprise veut revoir sa politique salariale. Actuellement, la moyenne des salaires mensuels est de 1700, l'écart-type est 110, le maximum 6000, le minimum 1200.  
Le but de la révision est de réduire les écarts de salaire tout en augmentant le salaire moyen. Plus précisément, elle prévoir de ramener l'écart-type à 80 et d'augmenter la moyenne à 1800. 

Supposons pour simplifier que l'entreprise comprenne 15 employés.

\begin{enumerate}
    \item
        Proposer une distribution des salaires satisfaisant aux conditions avant et après la révision des salaires.
    \item
        Tous les salaires auront-ils été augmentés ?
\end{enumerate}

%TODO : les nombres donnés dans cet exercice sont trop difficiles; il faut trouver autre chose.

\corrref{smath-0248}
\end{exercice}
