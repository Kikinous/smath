% This is part of Un soupçon de mathématique sans être agressif pour autant
% Copyright (c) 2013
%   Laurent Claessens
% See the file fdl-1.3.txt for copying conditions.

\begin{exercice}[\cite{HYlEAxq}]\label{exosmath-0248}

Une entreprise veut revoir sa politique salariale. Actuellement, la direction a remarqué que la moyenne des salaires est situé dans le premier quartile des salaires, situation qu'elle n'estime pas très logique. Le salaire minimum est de \( 1200\) et le maximum est de \( 6000\). Le but de la révision est de rationaliser les salaires tout en augmentant le salaire médian.

Supposons pour simplifier que l'entreprise comprenne entre \( 15\) et \( 20\) employés (vous pouvez choisir combien exactement).

\begin{enumerate}
    \item
        Proposer une distribution des salaires satisfaisant aux conditions d'avant la réforme.
    \item
        Proposer une réforme des salaires qui fasse en sorte que la moyenne devienne égale à la médiane tout en augmentant la médiane par rapport à la distribution proposée au premier point.
\end{enumerate}

\corrref{smath-0248}
\end{exercice}
