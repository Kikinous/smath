% This is part of Un soupçon de mathématique sans être agressif pour autant
% Copyright (c) 2013
%   Laurent Claessens
% See the file fdl-1.3.txt for copying conditions.

\begin{exercice}[\cite{HYlEAxq}]\label{exosmath-0248}


On considère, dans une entreprise, la série des salaires mensuels de tous les employés.  La moyenne de cette série est 1500, l'écart-type est 90, le maximum 5000, le minimum 1300.  L'entreprise voudrait mettre en œuvre un plan de revalorisation des salaires et de réduction des écarts.  Elle prévoit de réduire l'écart-type à 60 et d'augmenter la moyenne à 1700.  Proposer une transformation des salaires qui conduise à ce résultat. Si cette transformation est effectuée, tous les salaires seront-ils augmentés ?

    %AFAIRE : si j'obtiens pas l'autorisation du gars, il faudra retirer cet exercice.

\corrref{smath-0248}
\end{exercice}
