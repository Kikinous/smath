% This is part of Un soupçon de mathématique sans être agressif pour autant
% Copyright (c) 2013
%   Laurent Claessens
% See the file fdl-1.3.txt for copying conditions.

\begin{exercice}\label{exosmath-0534}

    Un boucher doit faire des tranches de \unit{100}{\gram}. Voici le relevé des poids réels des tranches :
    \begin{equation*}
        \begin{array}[]{|c||c|c|c|c|c|c|}
            \hline
            \text{Valeurs}&\unit{90}{\gram}&\unit{95}{\gram}&\unit{100}{\gram}&\unit{105}{\gram}&\unit{110}{\gram}&\unit{115}{\gram}\\
            \hline\hline
            \text{Effectifs}&25&30&10&15&15&5\\
            \hline
        \end{array}
    \end{equation*}
    Combien vaut le troisième quartile ? Tracer le graphe des effectifs cumulées croissantes.

    Faire les exercices 13 et 14 de la page 257.

\corrref{smath-0534}
\end{exercice}
