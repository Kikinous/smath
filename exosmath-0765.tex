% This is part of Un soupçon de mathématique sans être agressif pour autant
% Copyright (c) 2014
%   Laurent Claessens
% See the file fdl-1.3.txt for copying conditions.

\begin{exercice}[\cite{JQYooIYHiaH}]\label{exosmath-0765}

Avec les neuf chiffres $1$, $2$, $3$, $4$, $5$, $6$, $7$, $8$ et $9$ utilisés chacun une fois et une seule, on écrit trois nombres entiers,
puis on les additionne.

Par exemple : $251 + 3 798 + 46 = 4 095$ mais ce n'est pas la plus petite somme.

Quelle est la plus petite somme que l'on peut obtenir ?

\corrref{smath-0765}
\end{exercice}
