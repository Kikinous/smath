% This is part of Un soupçon de mathématique sans être agressif pour autant
% Copyright (c) 2012
%   Laurent Claessens
% See the file fdl-1.3.txt for copying conditions.

\begin{exercice}\label{exoPremiere-0069}

    Trouver les solutions des équations suivantes.
    \begin{multicols}{3}
        \begin{enumerate}
            \item
                \( x^2-8x+15\)
            \item
                \( x^2+2x-15\)
            \item
                \( 2x^2+12x+10\)
            \item
                \( 2x^2-13x-7\)
            \item
                \( 4x^2-24x+37\)
            \item
                \( 9x^2-12x+4\)
            \item
                \( x^2-2x+2\)
            \item
                \( x^2+1\)
            \item
                \( x^2-8x+16\)
        \end{enumerate}
    \end{multicols}

    Rappel\footnote{Un tel rappel ne sera pas présent dans les interrogations et les DS} : les racines de \( ax^2+bx+c\) sont données par
    \begin{equation}
        \begin{aligned}[]
            x_1&=\frac{ -b+\sqrt{b^2-4ac} }{ 2a },&x_2&=\frac{ -b-\sqrt{b^2-4ac} }{ 2a }.
        \end{aligned}
    \end{equation}

\corrref{Premiere-0069}
\end{exercice}
