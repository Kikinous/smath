% This is part of Un soupçon de mathématique sans être agressif pour autant
% Copyright (c) 2013
%   Laurent Claessens
% See the file fdl-1.3.txt for copying conditions.

\begin{exercice}\label{exosmath-0440}

    Un joueur soupçonneux veut vérifier si un dé est truqué parce qu'il a l'impression que les nombres \( 1\) et \( 2\) arrivent trop souvent à son goût. Il décide donc de lancer \( 100\) fois le dé et de compter combien de fois il obtient \( 1\) ou \( 2\). Résultat : sur \( 100\) lancers, il a obtient \( 44\) fois un un ou un deux.

    Peut-il accuser le dé, au péril de se faire passer pour un mauvais perdant ?

\corrref{smath-0440}
\end{exercice}
