% This is part of Un soupçon de mathématique sans être agressif pour autant
% Copyright (c) 2015
%   Laurent Claessens
% See the file fdl-1.3.txt for copying conditions.

%--------------------------------------------------------------------------------------------------------------------------- 
\subsection*{Activité : triangle de Sierpiński}
%---------------------------------------------------------------------------------------------------------------------------

Les triangles de Sierpiński se construisent de la façon suivante : le triangle de Sierpiński numéro zéro est un simple triangle gris. Les suivants s'obtiennent en supprimant à chaque étape le triangle «central» de chacun de triangles gris formant l'étape précédente.

En voici quelque uns :

%The result is on figure \ref{LabelFigMOCGooKjSrVV}. % From file MOCGooKjSrVV
\newcommand{\CaptionFigMOCGooKjSrVV}{Quelque triangles de Sierpiński}
\input{Fig_MOCGooKjSrVV.pstricks}

\begin{enumerate}
    \item
        De combien de triangles gris est formé le triangle de Sierpiński numéro \( 1\), \( 2\), \( 3\) ?
    \item
        De combien de triangles gris est formé le triangle de Sierpiński numéro \( 5\), \( 10\), \( 20\) ?
\end{enumerate}
