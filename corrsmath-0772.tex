% This is part of Un soupçon de mathématique sans être agressif pour autant
% Copyright (c) 2014
%   Laurent Claessens
% See the file fdl-1.3.txt for copying conditions.

\begin{corrige}{smath-0772}

    Les expressions égales à \( -\dfrac{ 6 }{ 7 }\) sont \( \dfrac{ 6 }{ -7 }\) et \( -(5\div 7)\).

    Le nombre \( 0.87\) n'est pas égal à \( -6/7\) d'une part pour son signe, et d'autre part, en multipliant \( 0.86\) par \( 7\), on n'obtient pas \( 6\).

\end{corrige}
