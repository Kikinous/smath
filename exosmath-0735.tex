% This is part of Un soupçon de mathématique sans être agressif pour autant
% Copyright (c) 2014
%   Laurent Claessens
% See the file fdl-1.3.txt for copying conditions.

\begin{exercice}\label{exosmath-0735}

Une sortie théâtre est organisée pour les $47$ élèves de \( 6\)\ieme et les $32$ élèves de \( 5\)\ieme du collège. Chaque place coûte 6 €.

\begin{enumerate}
    \item
        Lucas a tapé la séquence suivante sur sa calculatrice :
\begin{equation}
    \boxed{47}\,\boxed{+}\,\boxed{32}\,\boxed{\times}\,\boxed{6}\,\boxed{=}
\end{equation}
        Est-ce correct ?
    \item
        Quelle séquence de touches de calculatrice faut-il faire pour calculer le coût total à payer pour la collège ?
    \item
        Finalement, combien coûte cette sortie ?
\end{enumerate}

\corrref{smath-0735}
\end{exercice}
