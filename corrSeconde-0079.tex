% This is part of Un soupçon de mathématique sans être agressif pour autant
% Copyright (c) 2012
%   Laurent Claessens
% See the file fdl-1.3.txt for copying conditions.

\begin{corrige}{Seconde-0079}

    \begin{enumerate}
        \item
            La distance entre \( (0;0)\) et \( (1;3)\) est \( \sqrt{1^2+3^2}=\sqrt{10}\). C'est le rayon du cercle.
        \item
            Le point \( (3;1)\) est à distance \( \sqrt{10}\) du centre du cercle.
        \item
            Le point \( (3;1)\) est donc sur le cercle parce qu'il est à distance du centre égale au rayon.
        \item
            Le point \( (2;1)\) n'est pas sur le cercle parce que sa distance à \( (0;0)\) est \( \sqrt{2^2+1^2}=\sqrt{10}\) au lieu de \( \sqrt{10}\).
        \item
            Le point \( (-3;1)\) est sur le cercle (le point \( (0;\sqrt{10})\) aussi par exemple). Le point \( (10,10)\) est hors du cercle et le point \( (0;0)\) est à l'intérieur.
    \end{enumerate}

\end{corrige}
