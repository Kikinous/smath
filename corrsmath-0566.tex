% This is part of Un soupçon de mathématique sans être agressif pour autant
% Copyright (c) 2013
%   Laurent Claessens
% See the file fdl-1.3.txt for copying conditions.

\begin{corrige}{smath-0566}

    Pour cet exercice nous ne donnons que les réponses. Les justifications sont les mêmes que pour les autres.

    \begin{enumerate}
        \item
            Nous avons \( f(x)=2\) pour \( x=-2.5\) et \( x=-1\). Donc 
            \begin{equation}
                S=\{ -2.5;-1 \}.
            \end{equation}
        \item
            Non. Le graphe monte puis descend; la fonction n'est ni croissante ni décroissante sur cet intervalle.
        \item
            Il faut repérer les abscisses de tous les points pour lesquels le graphe de la courbe coupe l'axe horizontal :
            \begin{equation}
                S=\{ -3;0;1 \}
            \end{equation}
        \item
            L'image de \( -1\) est \( 2\).
        \item
            \begin{equation*}
                \begin{array}[]{c|ccccccc}
                    x&-3.5&&-1.9&&0.5&&1.5\\
                    \hline
                    &&&3&&&&1.7\\
                    f(x)&&\nearrow&&\searrow&&\nearrow&\\
                    &-4&&&&-0.5&&\\
                \end{array}
            \end{equation*}
        \item
            Albert est complètement dans l'erreur : le graphe présenté n'est même pas une droite; la fonction ne peut donc être linéaire (ni affine en général).
    \end{enumerate}

\end{corrige}
