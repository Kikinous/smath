% This is part of Un soupçon de mathématique sans être agressif pour autant
% Copyright (c) 2015
%   Laurent Claessens
% See the file fdl-1.3.txt for copying conditions.

\begin{corrige}{2smath-0096}

    \begin{enumerate}
        \item
            Le bon graphe est le deuxième parce qu'il est le seul à être une droite passant par l'origine.
        \item
            Parmi les nombres demandés dans le tableau, le plus simple à aller chercher sur le graphique est le point \( P\) qui dit que \SI{90}{\minute} correspondent à \SI{100}{\kilo\meter}.
            \begin{center}
   \input{Fig_QXYWooOWHshl3.pstricks}
            \end{center}
            Nous pouvons donc compléter ceci :
\begin{center}
$
    \begin{array}[]{|c|c|c|c|}
          \hline
          \text{Durée (\si{\minute})}&27&90&\\
        \hline
        \text{Distance (\si{\kilo\meter})}&&100&130\\
          \hline
    \end{array}
    $
\end{center}

Pour le reste, il faut faire jouer la proportionnalité, ou des produits en croix. D'abord
\begin{equation}
    \frac{ x }{ 27 }=\frac{ 100 }{ 90 },
\end{equation}
qui donne \( 90\times x=27\times 100\) et donc \( x=\dfrac{ 27\times 100 }{ 90 }=30\) et ensuite
\begin{equation}
    \frac{ x }{ 130 }=\frac{ 90 }{ 100 },
\end{equation}
qui donne \( x=\dfrac{ 90\times 130 }{ 100 }=117\). Au final le tableau est :
\begin{center}
$
    \begin{array}[]{|c|c|c|c|}
          \hline
          \text{Durée (\si{\minute})}&27&90&117\\
        \hline
        \text{Distance (\si{\kilo\meter})}&30&100&130\\
          \hline
    \end{array}
    $
\end{center}
    \end{enumerate}

\end{corrige}
