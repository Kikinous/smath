% This is part of Un soupçon de mathématique sans être agressif pour autant
% Copyright (c) 2014
%   Laurent Claessens
% See the file fdl-1.3.txt for copying conditions.

\begin{corrige}{smath-0598}

    \begin{enumerate}
        \item
            Il s'agit d'un produit remarquable :
            \begin{equation}
                (x-4)^2=x^2-8x+16.
            \end{equation}
        \item
            Le seul facteur commun entre \( 12x\), \( 50\) et \( 2x\) est \( 2\) (et surtout pas \( x\) parce qu'il n'y a pas de \( x\) dans tous les termes du numérateur). Nous pouvons donc simplifier par \( 2\) :
            \begin{equation}
                \frac{ 12x+50 }{ 2x }=\frac{ 6c+25 }{ x }.
            \end{equation}
        \item
            Il suffit de remplacer \( x\) par \( -2\) en faisant attention aux signes :
            \begin{equation}
                f(-2)=-3\times (-2)^2+7=-3\times 4+7=-12+7=-5.
            \end{equation}
        \item
            Si le côté fait \unit{6}{\kilo\meter}, alors le théorème de Pythagore nous donne la diagonale :
            \begin{equation}
                \sqrt{6^2+6^2}=\sqrt{72}=6\sqrt{2}.
            \end{equation}
        \item
            Non parce que leurs coefficients directeurs ne sont pas égaux. Le premier vaut \( 7\) et le second vaut \( -6\).
    \end{enumerate}

\end{corrige}
