% This is part of Exercices de mathématique pour SVT
% Copyright (C) 2010
%   Laurent Claessens et Carlotta Donadello
% See the file fdl-1.3.txt for copying conditions.

\begin{corrige}{TD3-0008}

	\begin{enumerate}
		\item
			Commençons par dire quelque chose de valable dans tous les cas: si $u_n=0$, alors $u_{n+1}=0$ parce que
			\begin{equation}
				u_{n+1}=u_nf(u_n)
			\end{equation}
			qui vaut zéro lorsque $u_n=0$.

			D'autre part, si $u_n\geq 2$, alors $u_{n+1}=u_nf(u_n)=0$. Par conséquent, dès que la suite dépasse $2$, elle retombe immédiatement à zéro.
			\begin{enumerate}
				\item
					Les cas $x=0$ et $x\geq 2$ sont traités par la remarque que nous venons de faire.

			Il reste à voir ce qui se passe si $x\in\mathopen] 0 , 2 \mathclose[$. Si $u_n\in\mathopen] 0 , 1 \mathclose[$, alors $f(u_n)=\frac{ 1 }{2}$, et donc
			\begin{equation}
				u_{n+1}=\frac{ u_n }{ 2 }.
			\end{equation}
			Par conséquent, dès que la suite entre dans l'intervalle $\mathopen] 0 , 2 \mathclose[$, la suite y reste (parce que dans ce cas, $\frac{ u_n }{2}$ est encore entre $0$ et $2$) et est décroissante (la moitié à chaque pas). Nous en déduisons que dans ce cas, la suite tend vers $0$.

			En conclusion, pour tous les $x\geq 0$, la suite va tendre vers $0$.

				\item
					Nous avons
					\begin{equation}
						f(y)=\begin{cases}
							1	&	\text{si $y<2$}\\
							0	&	 \text{si $y\geq 2$.}
						\end{cases}
					\end{equation}
					Le calcul des premiers termes est immédiat : $u_0=1$, $u_1=1\cdot f(1)=1$, et puis la suite reste en fait constante égale à $1$.
				\item
					Cette fois, nous partons de $u_0=3$. Donc $f(u_0)=0$, et la suite tombe immédiatement à zéro et y reste. Elle converge donc vers zéro.
				\item
					Si $u_0\geq 2$, alors $f(u_0)=0$ et la suite est immédiatement zéro.

					Si $u_0<2$, alors $u_1=2u_0>u_1$. Tant que la suite reste plus petite que $2$, la valeur est doublée à chaque pas. À un moment donné, elle va dépasser $2$, et à ce moment, elle retombe immédiatement à zéro. La suite tend donc vers zéro.
					
			\end{enumerate}
			
		\item
			Dans cette question, nous partons avec $u_0=1$, et on demande pour que $n$ nous aurons $u_n\leq 0.1$. (cent mille, c'est zéro virgule un million) Afin de comprendre comment les choses se passent, calculons les premiers termes. La fonction $f$ est donnée par
			\begin{equation}		\label{EqLaFnfcandaad}
				f(y)=\begin{cases}
					\frac{ 9 }{ 10 }	&	\text{si $y<2$}\\
					0	&	 \text{si $y\geq 2$,}
				\end{cases}
			\end{equation}
			donc
			\begin{equation}
				\begin{aligned}[]
					u_0&=1\\
					u_1&=\frac{ 9 }{ 10 }u_0=\frac{ 9 }{ 10 }\\
					u_2&=\frac{ 9 }{ 10 }u_1=\frac{ 9 }{ 10 }\cdot\frac{ 9 }{ 10 }=\left( \frac{ 9 }{ 10 } \right)^2\\
					u_3&=\frac{ 9 }{ 10 }u_2=\left( \frac{ 9 }{ 10 } \right)^3,
				\end{aligned}
			\end{equation}
			etc. Nous voyons que
			\begin{equation}
				u_n=\left( \frac{ 9 }{ 10 } \right)^n.
			\end{equation}
			La question revient à trouver pour quel $n$ nous avons
			\begin{equation}
				\left( \frac{ 9 }{ 10 } \right)^n=0.1,
			\end{equation}
			la réponse est donnée par
			\begin{equation}
				n=\log_{9/10}(0.1)=21.8,
			\end{equation}
			et il faut donc $22$ mois pour passer en dessous de la limite des $100$ mille.

			Si par contre nous prenons $a=11/10$, alors nous avons, pour les premiers termes la suite
			\begin{equation}
				u_n=\left( \frac{ 11 }{ 10 } \right)^n,
			\end{equation}
			et la suite est croissante. Mais dès que la suite dépasse la valeur $2$, elle retombe à zéro parce que 
			\begin{equation}
				f(y)=\begin{cases}
					\frac{ 11 }{ 10 }	&	\text{si $y<2$}\\
					0	&	 \text{si $y\geq 2$.}
				\end{cases}
			\end{equation}
			En calculant
			\begin{equation}
				n=\log_{11/10}(2),
			\end{equation}
			nous trouvons que la suite dépasse $2$ en $u_8$. Par conséquent $u_9=0$.

	\end{enumerate}

\end{corrige}
