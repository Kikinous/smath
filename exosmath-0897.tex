% This is part of Un soupçon de mathématique sans être agressif pour autant
% Copyright (c) 2014
%   Laurent Claessens
% See the file fdl-1.3.txt for copying conditions.

\begin{exercice}\label{exosmath-0897}

    Exprimer les phrases suivantes sous la forme «si \ldots alors \ldots» 
    \begin{enumerate}
        \item
            Lorsqu'il pleut, il y a des nuages.
        \item
            Un parallélogramme possède deux diagonales de même longueurs.
    \end{enumerate}
    En s'appuyant sur les affirmations précédentes, répondre si possible aux questions suivantes.
    \begin{enumerate}
        \item   \label{ItemSSCooAYmeCZ}
            Le \( 5\) avril, il y a eu des nuages. Est-on certain qu'il ait plu ?
        \item
            Les segments \( [AC]\) et \( [BD]\) sont de même longueur. Est-ce que le quadrilatère \( ABCD\) est forcément un parallélogramme ?
    \end{enumerate}

    Expliquer les réponses soit en citant une propriété, soit par un contre-exemple ou une phrase en français (pour la \ref{ItemSSCooAYmeCZ}).

\corrref{smath-0897}
\end{exercice}
