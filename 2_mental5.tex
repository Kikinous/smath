% This is part of Un soupçon de mathématique sans être agressif pour autant
% Copyright (c) 2014
%   Laurent Claessens
% See the file fdl-1.3.txt for copying conditions.



\begin{MentalActivity}
    \begin{mental}
        Calculer :
        \begin{enumerate}
            \item
                \( 18-12+3\)
            \item
                \( 3\times 8+2\)
            \item
                \( 3\times (8+2)\)
        \end{enumerate}
    \end{mental}

    \begin{mental}
        Vrai ou faux ?
        \begin{enumerate}
            \item
                \( 3\times (8+57)=3\times 8+57\)
            \item
                \( 12\times (32+3)=12\times 32+12\times 3\)
        \end{enumerate}
    \end{mental}
    
    \begin{mental}

        \begin{center}
   \input{Fig_RYDooDLpToB.pstricks}
        \end{center}
        
        Le périmètre du rectangle est
        \begin{enumerate}
            \item
                \( 2\times 15+\ell\)
            \item
                \( 15\times \ell\)
            \item
                \( 2\times (15+\ell)\)
        \end{enumerate}

    \end{mental}

    \begin{mental}
        \begin{enumerate}
            \item
                $\SI{250}{\centi\meter}=\ldots\si{\meter}$
            \item
                $\SI{1}{\liter}=\ldots\si{\deci\cubic\meter}$
            \item
                \( \SI{1}{\hour}=\ldots\si{\second}\)
        \end{enumerate}
    \end{mental}

\end{MentalActivity}

%%%%%%%%%%%%%%%%%%%%%%%%%%%%%%%%%%%%%%%%%%%%%%%%%%%%%%%%%%%%
\begin{MentalActivity}

    \begin{mental}
        Dire si les mesures suivantes sont les mesures d'un triangle (préciser si il est plat).
        \begin{enumerate}
            \item
                \( AB=5\), \( BC=7\), \( AC=10\)
            \item
                \( KL=9\), \( KT=1\),  \( LT=11\)
        \end{enumerate}
    \end{mental}

    \begin{mental}
        Quelle est la circonférence d'une cercle de rayon \( \SI{2}{\centi\meter}\) ? (rappel : la formule est \( 2\times\pi\times R\)) ?
    \end{mental}

    \begin{mental}
        Compléter :
        \begin{enumerate}
            \item
                \( 3\times 7-\ldots=15\)
            \item
                \( \dfrac{ 5\times 5\times 67 }{ 25 }=\ldots\)
            \item
                \( (100-5)\times 8=\ldots\times 100-\ldots\times 5\)
            \item
                \( \dfrac{ \ldots }{ 4+6 }=100\)
        \end{enumerate}
    \end{mental}

    \begin{mental}
        Trouver deux façons de compléter les mesures pour que les points \( A\), \( B\) et \( C\) soient alignés :
        \begin{subequations}
            \begin{align}
                AB&=\SI{5}{\centi\meter}\\
                BC&=\ldots\si{\centi\meter}\\
                AC&=\SI{4}{\centi\meter}
            \end{align}
        \end{subequations}
    \end{mental}
\end{MentalActivity}

\begin{MentalActivity}
    
    \begin{mental}
        Soit le programme de calcul
        \begin{framed}
            \begin{itemize}
                \item Choisir un nombre;
                \item soustraire \( 32\);
                \item diviser par 3
            \end{itemize}
        \end{framed}
        \begin{enumerate}
            \item
                De combien faut-il partir pour obtenir \( 6\) ?
            \item
                Combien obtient-t-on en partant de \( 101\) ?
        \end{enumerate}
    \end{mental}

    \begin{mental}
        Compléter les pointillés :
        \begin{enumerate}
            \item
                \( 5\times \ldots +5=30\)
            \item
                \( \dfrac{ 12\times 124 }{ \ldots }=124\)
            \item
                Le triangle de côtés \( AB=12\), \( BC=1\), \( AC=\ldots\) est isocèle.
        \end{enumerate}
    \end{mental}

    \begin{mental}
        Calculs 
        \begin{enumerate}
            \item
                Combien vaut \( 7\times a\) si \( a=9\) ?
            \item
                Combien vaut \( 2\times b+5\) si \( b=3\) ?
            \item
                Combien vaut \( 7\times a+b\) si \( a=2\) et \( b=6\) ?
        \end{enumerate}
    \end{mental}
\end{MentalActivity}

\begin{MentalActivity}
    
    \begin{mental}
        \begin{enumerate}
            \item
                L'aire de ce rectangle est de \(\SI{20}{\centi\meter\squared}\).
        \begin{center}
           \input{Fig_LCUooNGZJFk.pstricks}
        \end{center}
        Que vaut sa hauteur \( h\) ?
            \item
                L'aire de ce rectangle est de \(\SI{20}{\centi\meter\squared}\).
                \begin{center}
                    \input{Fig_KXXooKBoqAY.pstricks}
                \end{center}
                Que vaut sa hauteur \( h\) en fonction de sa longueur \( l\) ?
        \end{enumerate}
    \end{mental}
    
\end{MentalActivity}<++>

%%%%%%%%%%%%%%%%%%%%%%%%%%%%%%%%%%%%%%%%%%%%%%%%%%%%%%%%%%

% Les exercices suivants contienent des vrai/faux sur le second degré. À mettre dans les prochains calculs mental.
% M'est avis qu'il faut les ajouter aussi à ``autres exercices de seconde''.
%\Exo{smath-0652}    % Haag; c'est mon vrai ou faux en vrac à compléter.
%\Exo{smath-0254}
