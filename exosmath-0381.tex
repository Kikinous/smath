% This is part of Un soupçon de mathématique sans être agressif pour autant
% Copyright (c) 2013
%   Laurent Claessens
% See the file fdl-1.3.txt for copying conditions.

\begin{exercice}[\cite{QSfMxlk}]\label{exosmath-0381}

    Un ivrogne part d'un point \( D\) dans un espace assez grand et marche en zigzaguant. Il se déplacce en avant en restant toujours sur le maillage de la figure \ref{LabelFigJHrkjFz}.  La probabilité qu'il fasse un pas à droite est égale à \( \frac{ 1 }{2}\).

%The result is on figure \ref{LabelFigJHrkjFz}. % From file JHrkjFz
\newcommand{\CaptionFigJHrkjFz}{La grille sur laquelle se déplace l'ivrogne des exercices \ref{exosmath-0381} et \ref{exosmath-0382}.}
\input{Fig_JHrkjFz.pstricks}
Nous supposons que l'ivrogne fait quatre pas.
\begin{enumerate}
    \item
        Quelle est la probabilité qu'il tombe dans la bouche d'égout \( E\) ?
    \item
        Quelle est la probabilité qu'il tombe dans la bouche d'égout \( E'\) ?
    \item
        Quelle est la probabilité qu'il tombe dans la bouche d'égout \( F\) ?
\end{enumerate}

\corrref{smath-0381}
\end{exercice}
