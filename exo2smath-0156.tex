% This is part of Un soupçon de mathématique sans être agressif pour autant
% Copyright (c) 2015
%   Laurent Claessens
% See the file fdl-1.3.txt for copying conditions.

\begin{exercice}\label{exo2smath-0156}

    Trois demi-cercles sont représentés sur le dessin suivant.
\begin{center}
   \input{Fig_YMMDooFDCWPN.pstricks}
\end{center}
    Le plus grand a pour diamètre \( 10\), le plus petit a pour diamètre \( x\).
    \begin{enumerate}
        \item
            Quel est le diamètre du troisième cercle ?
        \item
            Donner en fonction de \( x\) la somme des longueurs des trois demi-cercles tracés.
        \item
            Réduire et simplifier au maximum la réponse.
    \end{enumerate}
    Pour rappel, la circonférence d'un cercle de rayon \( r\) est \( 2\times \pi\times r\) ou encore \( \pi\times \text{diamètre}\).

\corrref{2smath-0156}
\end{exercice}
