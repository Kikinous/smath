% This is part of Un soupçon de mathématique sans être agressif pour autant
% Copyright (c) 2015
%   Laurent Claessens
% See the file fdl-1.3.txt for copying conditions.

\begin{exercice}\label{exo2smath-0156}

%\begin{wrapfigure}{r}{6.0cm}
%   \vspace{-0.5cm}        % à adapter.
%   \centering
%   \input{Fig_YMMDooFDCWPN.pstricks}
%\end{wrapfigure}

    \begin{multicols}{2}
    Trois demi-cercles sont représentés sur le dessin ci-contre. Le cercle \( \mC_3\) a pour diamètre \( 10\).

    \begin{enumerate}
        \item
            Quels sont les diamètres des cercles \( \mC_1\) et \( \mC_2\) ?
        \item
            Donner en fonction de \( x\) les longueurs des trois demi-cercles tracés.
        \item   \label{ItemCUFLooIdJOVd}
            Calculer la somme des longueurs de \( \mC_1\) et \( \mC_2\). Qu'est-ce que l'on remarque ?
    \end{enumerate}

    \columnbreak

    \begin{center}
   \input{Fig_YMMDooFDCWPN.pstricks}
    \end{center}
    \end{multicols}
    Pour rappel, la circonférence d'un cercle de rayon \( r\) est \( 2\times \pi\times r\) ou encore \( \pi\times \text{diamètre}\).

\corrref{2smath-0156}
\end{exercice}
