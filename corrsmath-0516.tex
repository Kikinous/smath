% This is part of Un soupçon de mathématique sans être agressif pour autant
% Copyright (c) 2013
%   Laurent Claessens
% See the file fdl-1.3.txt for copying conditions.

\begin{corrige}{smath-0516}


    Tout est lisible sur le dessin. Ce dessin n'étant pas très précis autour de la bosse entre \( x=3\) et \( x=3.5\), plusieurs réponses différentes ont été acceptées.

    \begin{center}
   \input{Fig_LQGzkvL.pstricks}
    \end{center}

    \begin{enumerate}
        \item
            Une fonction est décroissante sur un intervalle si le graphe est une courbe qui descend. Sur notre dessin ce sont les morceaux repassés en rouge.
        \item
            Les antécédents de \( 2\) sont les abscisses pour lesquelles l'image vaut \( 2\). Nous les repérons à partir des points \( A_1\) et \( A_2\) sur le graphe. Les solutions sont : \( x=3\) et \( x=3.5\). 

        \item

            Le nombre \( f(0)\) est donnée par la hauteur du graphe au-dessus de \( x=0\). Le graphe passant par \( (0;0)\) nous avons \( f(0)=0\).
        \item
            Résoudre \( f(x)=1\) revient à trouver les antécédents de \( 1\). Ce sont les abscisses des points de hauteur \( 1\), c'est à dire des points \( B_1\) et \( B_2\). Les solutions de l'équation sont donc environ
            \begin{equation}
                S=\{ 1.4;4.6 \}
            \end{equation}
        \item
            Parmi les points d'abscisse entre \( -1\) et \( 2\), le point le plus haut est le point \( (2;1.5)\). Le maximum de la fonction sur cet intervalle est donc \( 2\). Pour rappel, le maximum est la valeur atteinte et non le point lui-même.
        \item
            Étudier les variations signifie écrire le tableau de variations :
            \begin{equation*}
                \begin{array}[]{c|ccccccc}
                    x&-1.5&&-0.5&&3.2&&5.3\\
                    \hline
                    &1.4&&&&2.1&&\\
                    f(x)&&\searrow&&\nearrow&&\searrow&\\
                    &&&-0.2&&&&-0.5\\
                \end{array}
            \end{equation*}
    \end{enumerate}

\end{corrige}
