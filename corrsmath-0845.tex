% This is part of Un soupçon de mathématique sans être agressif pour autant
% Copyright (c) 2014
%   Laurent Claessens
% See the file fdl-1.3.txt for copying conditions.

\begin{corrige}{smath-0845}

    \begin{enumerate}
        \item
            Nous passons par l'étape intermédiaire \( 7\times \ldots=21\). Il suffit alors de chercher \( 21\) dans la table de \( 7\) et de trouver \( 3\) :
            \begin{equation}
                7\times 3+7=21.
            \end{equation}
        \item
            Nous utilisons la formule de distribution (ou de factorisation, suivant le point de vue) en remarquant que \( 95=100-5\) :
            \begin{subequations}
                \begin{align}
                    95\times 67&=(100-5)\times 67\\
                    &=100\times 67-5\times 67.
                \end{align}
            \end{subequations}
            Pour cet exercice, il ne fallait surtout pas essayer de calculer \( 95\times 67\) ni mentalement ni sur une feuille de brouillon.
        \item
            Pour calculer \( \dfrac{ 9+ 3 }{ 2\times 3 }\), il faut calculer d'abord (séparément) le numérateur et le dénominateur :
            \begin{subequations}
                \begin{align}
                    \frac{ 9+3 }{ 2\times 3 }&=\frac{ 12 }{ 6 }\\
                    &=2.
                \end{align}
            \end{subequations}
            Ne pas oublier que \( \dfrac{ 12 }{ 6 }\) représente le nombre \( 12\div 6\); lorsque la division «tombe juste», il faut l'effectuer.
        \item
            \( \dfrac{ 15\times 32 }{ 15 }=32\).
    \end{enumerate}

\end{corrige}
