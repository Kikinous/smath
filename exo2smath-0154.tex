% This is part of Un soupçon de mathématique sans être agressif pour autant
% Copyright (c) 2015
%   Laurent Claessens
% See the file fdl-1.3.txt for copying conditions.

\begin{exercice}\label{exo2smath-0154}

    Vrai ou faux ? (expliquer)
    \begin{enumerate}
        \item
            Un triangle dont les côtés mesurent \SI{6}{\centi\meter}, \SI{3}{\centi\meter} et \SI{4}{\centi\meter} est rectangle.
        \item
            Quel que soit le nombre \( a\), le nombre \( a\times (a+4)\) est plus grand que \( a\).
        \item
            Sur le dessin suivant, sachant que \( (BC)\) est parallèle à \( (AD)\) nous pouvons déduire que \( BC=2\times AD\) 
            \begin{center}
                \input{Fig_GHKHooEnlVFf.pstricks}
            \end{center}

    \end{enumerate}

\corrref{2smath-0154}
\end{exercice}
