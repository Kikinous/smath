% This is part of Un soupçon de mathématique sans être agressif pour autant
% Copyright (c) 2013
%   Laurent Claessens
% See the file fdl-1.3.txt for copying conditions.

\begin{multicols}{2}
    \begin{enumerate}
        \item
Le TER 894258 a pour horaire :
\begin{description}
    \item[14h56] Besançon-Viotte
    \item[15h06] St-Vit
    \item[15h21] Dole-Ville
    \item[15h30] Auxonne
    \item[15h39] Genlis
    \item[15h51] Dijon-Ville
\end{description}
Dessiner le trajet sur une ligne du temps en indiquant les durées entre les stations.
        \item
            En supposant les mêmes temps de parcours, quelles sont les heures d'arrivées dans les différentes gares du TER partant à 20h23 pour le même trajet ?
        \item
             Les points \( A(1;2)\), \( B(3;3)\), \( C(2;5)\) et \( D(0;4)\) forment un carré. Donner les coordonnées du «même» carré \( A'B'C'D'\) partant de \( A'(4;0)\).
         \item
             Soient les points \( K(0;0)\), \( L(4;1)\) et \( M(3;3)\). Donner les coordonnées du point \( N\) tel que \( KLMN\) soit un parallélogramme.
    \end{enumerate}
\end{multicols}
