% This is part of Un soupçon de mathématique sans être agressif pour autant
% Copyright (c) 2012
%   Laurent Claessens
% See the file fdl-1.3.txt for copying conditions.

\begin{exercice}\label{exoSeconde-0063}

    Nous voulons savoir la longueur du côté d'un triangle équilatéral dont la surface vaut \( \sqrt{3}\).
    \begin{enumerate}
        \item
            Dessiner un triangle équilatéral \( ABC\) et la hauteur issue de \( A\). Nous nommons \( H\) le point d'intersection avec le côté \( [BC]\).
        \item
            Nommons \( x\) la longueur du côté du triangle.
        \item
            Quelle est la longueur du segment \( [BH]\) ? Quelle est la longueur du segment \( [AB]\) ?
        \item
            Écrire le théorème Pythagore dans le triangle \( AHB\).
        \item
            Exprimer la longueur \( AH\) en fonction de \( x\). Soit \( h(x)\) cette fonction.
        \item
            Donner la fonction \( S(x)\) qui donne la surface du triangle en fonction de \( x\).
        \item
            Résoudre \( S(x)=\sqrt{3}\).
    \end{enumerate}

\corrref{Seconde-0063}
\end{exercice}
