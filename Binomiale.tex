% This is part of Un soupçon de mathématique sans être agressif pour autant
% Copyright (c) 2012
%   Laurent Claessens
% See the file fdl-1.3.txt for copying conditions.

%+++++++++++++++++++++++++++++++++++++++++++++++++++++++++++++++++++++++++++++++++++++++++++++++++++++++++++++++++++++++++++
\section{Épreuve de Bernoulli}
%+++++++++++++++++++++++++++++++++++++++++++++++++++++++++++++++++++++++++++++++++++++++++++++++++++++++++++++++++++++++++++

Source : \cite{YDRqwN}
% TODO : ajouter la citation du fichier de ressource de eduscol Ressource_Statistiques_Probabilites_1eres_208596.pdf

Nous jouons à pile ou face avec une pièce non truquée. Nous avons donc une chance sur deux d'obtenir pile et une chance sur deux d'obtenir face. Nous sommes intéressé par le nombre de fois que nous obtenons pile en jouant souvent.

\begin{example}
Commençons par jouer deux fois et dessinons un arbre.
            \begin{equation*}
            \xymatrix{%
                &&&\fbox{\text{Début}}\ar[lld]_{\frac{ 1 }{2}}\ar[rrd]^{\frac{ 1 }{2}}\\
                &\fbox{P}\ar[ld]_{\frac{ 1 }{2}}\ar[rd]^{\frac{ 1 }{2}}&&&&\fbox{F}\ar[ld]_{\frac{ 1 }{2}}\ar[rd]^{\frac{ 1 }{2}}\\
                \fbox{PP}&&\fbox{PF}&&\fbox{FP}&&\fbox{FF}
               }
            \end{equation*}

            Quelque questions :
            \begin{enumerate}
                \item
                    Quelle est la probabilité d'avoir deux fois pile ?
                \item
                    Quelle est la probabilité d'obtenir une fois pile et une fois face ?
                \item
                    Quelle est la probabilité d'obtenir au moins un face ?
            \end{enumerate}
            Les réponses :
            \begin{enumerate}
                \item
                    Sur les quatre issues possibles, une seule a deux fois face. La probabilité est \( 1/4\).
                \item
                    Sur les quatre issues possibles, deux ont un pile et un face : \( PF\) et \( FP\). La probabilité de tomber sur l'un des deux est \( 1/2\).
                \item
                    Sur les quatre issues possibles, trois ont au moins un face : \( PF\), \( FP\) et \( FF\). La probabilité d'avoir un des trois est \( 3/4\).
            \end{enumerate}

            Nous pouvons écrire le tableau
            \begin{equation}
                \begin{array}[]{|c|c|c|c|}
                    \hline
                    k&0&1&2\\
                    \hline\hline
                    P(X=k)&\frac{1}{ 4 }&\frac{1}{ 2 }&\frac{1}{ 4 }\\
                    \hline
                \end{array}
            \end{equation}

\end{example}



\begin{definition}
    Toute expérience aléatoire conduisant à deux issues possibles (appelées \emph{échec} ou \emph{succès}) est une \defe{épreuve de Bernoulli}{Bernoulli!épreuve}. Nous noterons \( p\) la probabilité du succès. La probabilité de l'échec sera alors \( 1-p\).
\end{definition}

\begin{example}
    Nous lançons un dé et nous nommons «succès» l'événement «le \( 6\) sort». Cela est une épreuve de Bernoulli avec \( p=1/6\).
\end{example}


%---------------------------------------------------------------------------------------------------------------------------
\subsection{La probabilité n'est en pratique pas toujours connue}
%---------------------------------------------------------------------------------------------------------------------------

\begin{example}
    Demandez à un de vos amis si il aime réciter des vers de Corneille en réunion de famille.

    Vous noterez que cela est bien une épreuve de Bernoulli, mais qu'il vous est plus ou moins impossible d'en estimer la probabilité de succès a priori. Au mieux vous pouvez dire «j'espère bien qu'aucun de mes amis n'a ce genre de hobbies».
    
\end{example}

\begin{example}
    Choisissez une voiture au hasard dans une file, et considérer l'événement «le conducteur est en train téléphoner». Ici encore il s'agit d'une épreuve de Bernoulli, mais il n'est pas possible d'en donner la probabilité.
\end{example}


%---------------------------------------------------------------------------------------------------------------------------
\subsection{À propos de l'hypothèse de remise}
%---------------------------------------------------------------------------------------------------------------------------

\begin{example}
    Dans une classe de \( 31\) élèves, \( 19\) sont nés en \( 1996\). Si nous en prenons un au hasard, nous avons donc une probabilité environ \( 0.61\) qu'il soit né en \( 1996\). Que penser de cet arbre où l'on note «O»  l'événement «avoir tiré un élève de \( 1996\)» et «N» l'événement contraire ?
   
            \begin{equation*}
            \xymatrix{%
                &&&\fbox{\text{Début}}\ar[lld]_{\frac{ 19 }{31}}\ar[rrd]^{\frac{ 12 }{31}}\\
                &\fbox{O}\ar[ld]_{\frac{ 19 }{31}}\ar[rd]^{\frac{ 12 }{31}}&&&&\fbox{N}\ar[ld]_{\frac{ 19 }{31}}\ar[rd]^{\frac{ 12 }{31}}\\
                \fbox{OO}&&\fbox{ON}&&\fbox{NO}&&\fbox{NN}
               }
            \end{equation*}

            Nous pouvons écrire le tableau
            \begin{equation}
                \begin{array}[]{|c|c|c|c|}
                    \hline
                    k&0&1&2\\
                    \hline\hline
                    P(X=k)&0.375&0.474&0.15\\
                    \hline
                \end{array}
            \end{equation}


    Si il n'y a pas de remise, alors l'arbre devrait ressembler plutôt à ceci :
            \begin{equation*}
            \xymatrix{%
                &&&\fbox{\text{Début}}\ar[lld]_{\frac{ 19 }{31}}\ar[rrd]^{\frac{ 12 }{31}}\\
                &\fbox{O}\ar[ld]_{\frac{ 18 }{30}}\ar[rd]^{\frac{ 12 }{30}}&&&&\fbox{N}\ar[ld]_{\frac{ 19 }{30}}\ar[rd]^{\frac{ 11 }{30}}\\
                \fbox{OO}&&\fbox{ON}&&\fbox{NO}&&\fbox{NN}
               }
            \end{equation*}
            et le tableau devient
            \begin{equation}
                \begin{array}[]{|c|c|c|c|}
                    \hline
                    k&0&1&2\\
                    \hline\hline
                    P(X=k)&0.368&0.49&0.142\\
                    \hline
                \end{array}
            \end{equation}

            Les nombres dans le tableaux ne sont pas tellement différents. Dans certains cas, l'approximation «avec remise» est justifiée.
\end{example}

\begin{example}
    Supposons que nous jouons au poker avec un paquet de \( 52\) cartes. Si nous avons déjà trois as en main, il reste \( 49\) cartes dans le paquet, et parmi ces \( 49\) cartes, un seul as. Nous avons donc une chance sur \( 49\) de réussir un carré en échangeant une carte.

    Si par contre nous jouons avec des dés, la probabilité de lancer un quatrième as est de une sur six; la différence est énorme.
\end{example}



%+++++++++++++++++++++++++++++++++++++++++++++++++++++++++++++++++++++++++++++++++++++++++++++++++++++++++++++++++++++++++++
\section{Schéma de Bernoulli}
%+++++++++++++++++++++++++++++++++++++++++++++++++++++++++++++++++++++++++++++++++++++++++++++++++++++++++++++++++++++++++++

\begin{definition}
    Un \defe{schéma de Bernoulli}{schéma (de Bernoulli)} de paramètres \( n\) et \( p\) est la répétition de \( n\) épreuves de Bernoulli de probabilité \( p\) identiques et indépendants.
\end{definition}
Nous allons nous intéresser au nombre de succès d'un schéma de Bernoulli de probabilité \( p\).

Reprenons notre arbre de choix d'élèves en fonction de l'année de naissance.

\begin{equation*}
\xymatrix{%
    &&&\fbox{\text{Début}}\ar[lld]_{\frac{ 19 }{31}}\ar[rrd]^{\frac{ 12 }{31}}\\
    &\fbox{O}\ar[ld]_{\frac{ 19 }{31}}\ar[rd]^{\frac{ 12 }{31}}&&&&\fbox{N}\ar[ld]_{\frac{ 19 }{31}}\ar[rd]^{\frac{ 12 }{31}}\\
    \fbox{OO}&&\fbox{ON}&&\fbox{NO}&&\fbox{NN}
   }
\end{equation*}
Nous notons \( P(X=0)\) la probabilité de tirer zéro élèves nés en \( 1996\); \( P(X=1)\) la probabilité d'en tirer (exactement \( 1\)); \( P(X=2)\) est la probabilité d'en tirer deux. En suivant la logique, \( P(X=3)\) est nul parce que de toutes façons, on n'en tire que deux. Étudions les cas.

Nous commençons par calculer \( P(X=0)\). Pour obtenir aucun étudiant nés en \( 1996\) en en tirant deux, il faut d'abord en tirer un qui n'est pas né en \( 1996\) (probabilité : \( \frac{ 12 }{ 13 }\)) puis en tirer un second (probabilité : \( \frac{ 12 }{ 13 }\) encore). Du coup la probabilité est
\begin{equation}
    P(X=0)=\left( \frac{ 12 }{ 31 } \right)^2,
\end{equation}
c'est à dire environ\footnote{Encore une fois : méfiez vous des approximations numériques. Vous n'accepteriez pas de confier votre vie à un calcul qui a arrondi la seconde décimale d'une probabilité.} \( 0.15\).

Calculons maintenant \( P(X=1)\). Nous repérons dans le tableau les cases qui correspondent à un élève né en \( 1996\) et un élève non né en \( 1996\). Ce sont les cases \( ON\) et \( NO\). En suivant les branches de l'arbre, la probabilité d'obtenir le \( ON\) est 
\begin{equation}
    P(ON)=\frac{ 19 }{ 31 }\times \frac{ 12 }{ 31 },
\end{equation}
tandis que l'autre est
\begin{equation}
    P(NO)=\frac{ 12 }{ 31 }\times \frac{ 19 }{ 31 }.
\end{equation}
Notons que c'est la même. La probabilité totale de pêcher \emph{exactement} un élève de \( 1996\) en en pêchant deux au hasard dans la classe est
\begin{equation}
    P(X=1)=2\times\frac{ 12 }{ 31 }\times \frac{ 19 }{ 31 }=\frac{ 456 }{ 961 },
\end{equation}
qui est environ \( 0.47\). Presque une chance sur deux.

Enfin, calculons la probabilité d'en tirer deux est donnée par la case \( OO\) de l'arbre et sa probabilité est
\begin{equation}
    P(X=2)=\left( \frac{ 19 }{ 31 } \right)^2
\end{equation}
qui est environ \( 0.37\).


\begin{example}
Nous effectuons \( 50\) lancers de pile ou face et nous comptons le nombre de réussite.

Par exemple :\\
P F P P P P F F F F F F F P P P P P F P F F P P F F F F F F P P F F F F P F F F F P P P P F F F F P \\
 total : 29 F et  21 P
 
Un autre exemple :

F F P P P P F P P P F P P P F F F F F P F F P P P P P F F F P P P F F P F F F P F P P P F P F P P F \\
 total : 23 F et  27 P

 Intuitivement, nous imaginons que nous devrions toujours obtenir autour de \( 25\) piles et \( 25\) faces.
    
\end{example}

\begin{example}

Effectuons dix mille fois les cent lancers et traçons un diagramme en bâtons décrivant le nombre de fois que nous avons obtenu chaque quantité de «piles».

Les diagrammes en bâtons de la figure \ref{LabelFigSimulBinNWxfTN} % From file SimulBinNWxfTN
représentent, en fonction de \( k\), le nombre de fois qu'on a obtenu exactement \( k\) fois pile en jouant \( 10\), \( 100\) et \( 10000\) fois à pile ou face.
\newcommand{\CaptionFigSimulBinNWxfTN}{Jouer à pile ou face un certain nombre de fois, et s'intéresser au nombre de fois que le «pile» sort.}
\input{Fig_SimulBinNWxfTN.pstricks}

%See also the subfigure \ref{LabelFigSimulBinNWxfTNssLabelSubFigSimulBinNWxfTN0}
%See also the subfigure \ref{LabelFigSimulBinNWxfTNssLabelSubFigSimulBinNWxfTN1}
%See also the subfigure \ref{LabelFigSimulBinNWxfTNssLabelSubFigSimulBinNWxfTN2}
%See also the subfigure \ref{LabelFigSimulBinNWxfTNssLabelSubFigSimulBinNWxfTN3}

Quelque propriétés intuitives.
\begin{enumerate}
    \item
        Il y a un maximum sur \( 50\) parce que lorsqu'on tire \( 100\) fois une pièce, on a un maximum de chances d'avoir \( 50\) fois pile et \( 50\) fois face.
    \item
        Le dessin doit être symétrique. Il y a a priori autant de chances d'avoir \( 48\) faces et \( 52\) pile que le contraire.

        En réalité, la courbe n'est pas exactement symétrique, mais plus les nombres sont grands, plus ça y ressemble.
    \item
        La courbe doit valoir à peu près zéro pour les petits et les grands nombres : nous sommes d'accord que la probabilité d'obtenir \( 95\) fois face sur \( 100\) lancers est à peu près le vide interstellaire.
\end{enumerate}
    
\end{example}

\newpage
\begin{example}
    Nous jouons au poker avec des dés. Un dé a \( 6\) face : as, roi, dame, valet, \( 10\), \( 9\). Nous voulons obtenir une paire de valet en lançant trois dés (nous estimons qu'un brelan de valet est une réussite). Voici l'arbre.

    \begin{equation}
    \xymatrix{%
        &&&\boxed{VVV}\\
        &&\boxed{VV}\ar[ur]^{\frac{1}{ 6 }}\ar[rd]_{\frac{ 5 }{ 6 }}\\
        &&&\boxed{VV \bar V}\\
        &\boxed{V}\ar[uur]^{\frac{1}{ 6 }}\ar[ddr]_{\frac{ 5 }{ 6 }}\\
        &&&\boxed{V \bar V  V}\\
        &&\boxed{V \bar V}\ar[ur]^{\frac{1}{ 6 }}\ar[dr]_{\frac{ 5 }{ 6 }}\\
        &&&\boxed{V \bar V\bar V}\\
        \boxed{\text{Début}}\ar[ruuuu]^{\frac{1}{ 6 }}\ar[rdddd]_{\frac{ 5 }{ 6 }}\\
        &&&\boxed{\bar  V V V}\\
        &&\boxed{ \bar V V}\ar[ur]^{\frac{1}{ 6 }}\ar[dr]_{\frac{ 5 }{ 6 }}\\
        &&&\boxed{ \bar V \bar V V}\\
        &\boxed{ \bar V}\ar[uur]^{\frac{1}{ 6 }}\ar[ddr]_{\frac{ 5 }{ 6 }}\\
        &&&\boxed{ \bar V \bar V V}\\
        &&\boxed{\bar V\bar V}\ar[ur]^{\frac{1}{ 6 }}\ar[dr]_{\frac{ 5 }{ 6 }}\\
        &&&\boxed{ \bar V \bar V  \bar V}\\
       }
    \end{equation}

            %\begin{equation*}
            %\xymatrix{%
            %    o&o&o&o&o&o&o&\fbox{\text{Début}}\ar[lllld]_{\frac{ 1 }{6}}\ar[rrrrd]^{\frac{ 5 }{6}}\\
            %    o&o&o&\fbox{A}\ar[lld]_{\frac{ 1 }{ 6 }}\ar[rrd]^{\frac{ 5 }{ 6 }}&o&o&o&o&o&o&\fbox{B}\ar[lld]_{\frac{ 1 }{ 6 }}\ar[rrd]^{\frac{ 5 }{ 6 }}&o&o&o\\
            %    o&\fbox{C}\ar[ld]_{\frac{ 5 }{ 6 }}\ar[rd]^{\frac{ 1 }{ 6 }}&o&o&o&\fbox{D}\ar[ld]_{\frac{ 5 }{ 6 }}\ar[rd]^{\frac{ 1 }{ 6 }}&o&o&o&\fbox{E}&o&o&o&\fbox{F}\ar[ld]_{\frac{ 5 }{ 6 }}\ar[rd]^{\frac{ 1 }{ 6 }}&o\\
            %    \fbox{F}&o&\fbox{F}&o&\fbox{F}&o&\fbox{F}&o&\fbox{F}&o&\fbox{F}&o&\fbox{F}&o&\fbox{F}
            %   }
            %\end{equation*}
    
    Nous pouvons écrire un tableau donnant les probabilités d'obtenir \( k\) fois le valet.
    \begin{equation}
        \begin{array}[]{|c|c|c|c|c|}
            \hline
            k&0&1&2&3\\
            \hline\hline
            P(X=k)&\left( \frac{ 5 }{ 6 } \right)^3\approx 0.58& 3\left( \frac{ 5 }{ 6 } \right)^2\left( \frac{1}{ 6 } \right)\approx 0.347 &3\left( \frac{1}{ 6 } \right)^2\left( \frac{ 5 }{ 6 } \right)\approx 0.07&\left( \frac{1}{ 6 } \right)^3\approx0.0046\\
            \hline
        \end{array}
    \end{equation}

\end{example}

%+++++++++++++++++++++++++++++++++++++++++++++++++++++++++++++++++++++++++++++++++++++++++++++++++++++++++++++++++++++++++++
\section{La loi binomiale}
%+++++++++++++++++++++++++++++++++++++++++++++++++++++++++++++++++++++++++++++++++++++++++++++++++++++++++++++++++++++++++++

%---------------------------------------------------------------------------------------------------------------------------
\subsection{Définition et formule}
%---------------------------------------------------------------------------------------------------------------------------

Nous considérons un schéma de Bernoulli de paramètres \( n\) et \( p\).
\begin{Aretenir}
    Si \( X\) est la «variable aléatoire» donnant le nombre de succès du schéma (qui est un nombre entier entre \( 0\) et \( n\)), nous disons que \( X\) suit la \defe{loi binomiale}{loi binomiale} de paramètres \( n\) et \( p\), et nous notons \( X\sim B(n,p)\).

    Nous disons que l'événement \( \{ X=k \}\) est réalisé lorsque le schéma de Bernoulli a obtenu exactement \( k\) succès sur les \( n\) épreuves. La probabilité de cet événement est notée \( P(X=k)\).

    Nous disons que l'événement \( \{ X<k \}\) est réalisé lorsque le nombre de succès obtenu est strictement inférieur à \( k\). La probabilité que cela arrive est notée \( P(X<k)\).

    Les événements et probabilités \( \{ X\geq k \}\), \( P(X>k)\) etc. sont définies de façon analogue.
\end{Aretenir}


Nous voudrions établir une formule donnant \( P(X=k)\) pour une variable aléatoire \( X\) suivant la loi binomiale de paramètres \( n\) et \( p\). Nous allons raisonner sur \( n=100\) et \( p=\frac{ 7 }{ 10 }\). Dans ce cas, le nombre \( P(X=k)\) est la probabilité d'obtenir exactement \( k\) succès en \( 100\) essais d'une épreuve ayant une probabilité \( 0.7\) de réussite.

L'arbre de \( 100\) tirages est évidemment compliqué à dessiner, mais nous comprenons que les chemins qui mènent à \( X=k\) sont les chemins qui prennent \( k\) fois la bifurcation de la réussite et \( 100-k\) fois la bifurcation de l'échec. Chacun de ces chemins a une probabilité de
\begin{equation}
    \left( \frac{ 7 }{ 10 } \right)^{k}\left( \frac{ 3 }{ 10 } \right)^{100-k}.
\end{equation}
La probabilité totale est donnée par
\begin{equation}
    P(X=k)=(\text{nombre de chemins passant \( k\) fois par la réussite})\times \left( \frac{ 7 }{ 10 } \right)^k\left( \frac{ 3 }{ 10 } \right)^{100-k}.
\end{equation}
Le nombre de chemins est passablement compliqué à calculer. Heureusement, il y a la calculatrice\footnote{Si vous voulez un avis personnel, utiliser une calculatrice pour débloquer une difficulté en mathématique est aussi intelligent qu'utiliser un anti-douleur pour améliorer ses performances sportives.}. Le nombre de chemin en question est nommé le \defe{coefficient binomial}{coefficient binomial}\footnote{C'est le moment d'ouvrir le mode d'emploi de ladite calculatrice.} et est noté \( {100\choose n}\); c'est le nombre de moyens de placer \( k\) «S» dans une suite de \( 100\) «S» ou «E».

\begin{Aretenir}
    Pour un schéma de Bernoulli de paramètres \( n\) et \( p\) nous avons
    \begin{equation}
        P(X=k)={n\choose k}p^{k}(1-p)^{n-k}.
    \end{equation}
\end{Aretenir}

Nous notons \( P(X\leq k)\) la probabilité que le nombre de succès soit plus petite ou égale à \( k\); nous avons par exemple
\begin{equation}
    P(X\leq 3)=P(X=0)+P(X=1)+P(X=2)+P(X=3).
\end{equation}
Pas de panique : si vous voulez calculer \( P(X\leq 137)\), il ne faudra pas faire \( 137\) calculs : il existe une touche sur votre calculatrice qui calcule la probabilité cumulée.

Notons la formule sympa
\begin{equation}
    P(X\leq k+1)=P(X\leq k)+P(X=k+1).
\end{equation}
En particulier nous avons
\begin{equation}
    P(X=k)=P(X\leq k)-P(X\leq k-1).
\end{equation}
Cela permet de calculer des probabilités en ayant seulement à disposition une table de probabilité cumulée.

\begin{Aretenir}
     La somme des probabilités de deux événements contraire doit faire \( 1\). Donc
    \begin{equation}
        P(x>k)=1-P(x\leq k).
    \end{equation}
\end{Aretenir}

\begin{example}
    Pour calculer à moindres frais la probabilité \( p\) qu'un événement se produise au moins une fois, il suffit de calculer la probabilité \( q\) qu'il ne se produise pas du tout. La probabilité cherchée est alors \( p=1-q\).
\end{example}

%---------------------------------------------------------------------------------------------------------------------------
\subsection{Espérance}
%---------------------------------------------------------------------------------------------------------------------------

\begin{example}
    Si nous lançons \( 18\) fois un dé à \( 6\) faces, il est «raisonnable» d'espérer obtenir trois fois le \( 6\) parce qu'on devrait l'obtenir en moyenne une fois sur six.
\end{example}
Plus généralement, si nous répétons \( n\) fois le lancer de dé, nous espérons obtenir \( \frac{ n }{ 6 }\) fois le six.

\begin{definition}
    L'\defe{espérance}{espérance} d'une variable aléatoire de Bernoulli \( X\sim B(n,p)\) est le nombre
    \begin{equation}
        E(X)=np.
    \end{equation}
\end{definition}
\begin{Aretenir}
    L'espérance s'interprète comme la valeur moyenne du nombre de succès obtenus si on répète un grand nombre de fois le schéma de Bernoulli.
\end{Aretenir}

\begin{example}
    Nous lançons \( 100\) fois un dé à \( 6\) faces et nous obtenons \( 20\) fois la face un. Ensuite nous relançons \( 100\) fois la pièce et nous obtenons \( 16\) fois la face un; maintenant la moyenne du nombre de succès est \( \frac{ 20+16 }{ 2 }=18\). Nous recommençons encore une fois, etc.

Le programme suivant simule \( 20\) schémas de \( 100\) lancer de pièces, affiche le nombre de succès à chaque schéma et la moyenne de nombre de succès.
\lstinputlisting{ex_bernoulli5.py}
donne
\lstinputlisting[title=Résultat]{res_ex_bernoulli5.txt}
Note : la valeur vers laquelle la moyenne doit tendre est \( \frac{ 100 }{ 6 }\approx 16.67\).

\end{example}
L'espérance est la tendance à long terme de la moyenne.

%+++++++++++++++++++++++++++++++++++++++++++++++++++++++++++++++++++++++++++++++++++++++++++++++++++++++++++++++++++++++++++
\section{Simuler un schéma de Bernoulli avec un ordinateur}
%+++++++++++++++++++++++++++++++++++++++++++++++++++++++++++++++++++++++++++++++++++++++++++++++++++++++++++++++++++++++++++

Pour simuler une expérience de Bernoulli de probabilité \( p\), il suffit de prendre un nombre aléatoire entre \( 0\) et \( 1\) puis de vérifier si il est plus petit que \( p\).

En effet, si nous prenons un nombre au hasard entre \( 0\) et \( 1\), quelle est la probabilité qu'il soit plus petit que, par exemple, \( 0.8\) ?

La majorité des tableurs comme \wikipedia{fr}{Libreoffice}{LibreOffice} ou \wikipedia{fr}{Gnumeric}{Gnumeric} possèdent des fonctions toutes faites pour simuler.

Il est cependant instructif d'en construire un par nous même. Par exemple ceci écrit «succès» avec une probabilité \( 0.6\).
\lstinputlisting{ex_random_binomiale.py}

Le programme suivant est celui écrit par votre serviteur pour produire les résultats et graphiques présentés plus haut. Attention : il est écrit pour \href{http://sagemath.org}{sage} et utilise donc la syntaxe de Python 2.x. La principale différence est qu'il n'y a pas de parenthèse pour \info{print}.
\lstinputlisting{simul_Bernoulli.py}

Une simulation pas à pas en python pour construire un histogramme de schéma de Bernoulli est donné à la section \ref{SecSimulBernrfNskC}.

%---------------------------------------------------------------------------------------------------------------------------
\subsection{Les commandes toutes faites}
%---------------------------------------------------------------------------------------------------------------------------

Nous construisons une variable aléatoire \( X\) suivant une loi binomiale de paramètres \( n\) et \( p\) avec

\lstinputlisting{ex_binomiale1.py}
donne
\lstinputlisting[title=Résultat]{res_ex_binomiale1.txt}

%+++++++++++++++++++++++++++++++++++++++++++++++++++++++++++++++++++++++++++++++++++++++++++++++++++++++++++++++++++++++++++
\section{Exercices }
%+++++++++++++++++++++++++++++++++++++++++++++++++++++++++++++++++++++++++++++++++++++++++++++++++++++++++++++++++++++++++++

%---------------------------------------------------------------------------------------------------------------------------
\subsection{Reconnaître une situation de Bernoulli}
%---------------------------------------------------------------------------------------------------------------------------

% Exercice à remettre dans un truc de proba plus généraliste.
%\Exo{Premiere-0078}
\Exo{Premiere-0071}
\Exo{Premiere-0094}

%---------------------------------------------------------------------------------------------------------------------------
\subsection{Calculer des probabilités avec un arbre}
%---------------------------------------------------------------------------------------------------------------------------

\Exo{Premiere-0074}
\Exo{Premiere-0075}
\Exo{Premiere-0083}
\Exo{Premiere-0086}
\Exo{Premiere-0090}

%---------------------------------------------------------------------------------------------------------------------------
\subsection{Espérance}
%---------------------------------------------------------------------------------------------------------------------------

\Exo{Premiere-0093}
\Exo{Premiere-0076}
\Exo{Premiere-0082}
\Exo{Premiere-0077}

%---------------------------------------------------------------------------------------------------------------------------
\subsection{Utilisation de la loi binomiale}
%---------------------------------------------------------------------------------------------------------------------------

\Exo{Premiere-0087}
\Exo{Premiere-0088}
\Exo{Premiere-0091}
\Exo{Premiere-0092}

