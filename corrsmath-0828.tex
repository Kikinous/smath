% This is part of Un soupçon de mathématique sans être agressif pour autant
% Copyright (c) 2014
%   Laurent Claessens
% See the file fdl-1.3.txt for copying conditions.

\begin{corrige}{smath-0828}

    Le plus simple pour trouver la réciproque d'un énoncé est de commencer par l'écrire sous la forme «Si \ldots alors \ldots» .
    \begin{enumerate}
        \item
            La première affirmation, exprimée sous forme «Si \ldots alors \ldots» est :
            \begin{quote}
                Si un quadrilatère a ses deux diagonales perpendiculaires, alors c'est un carré.
            \end{quote}
            La réciproque est :
            \begin{quote}
                Si un quadrilatère est un carré alors ses deux diagonales sont perpendiculaires.
            \end{quote}
            Cette réciproque est vraie : c'est la propriété comme quoi tout carré a ses diagonales qui se coupent perpendiculairement.
        \item
            La réciproque est :
            \begin{quote}
                Si un nombre peut être écrit sous forme fractionnaire, alors il est décimal.
            \end{quote}
            Cette réciproque est fausse comme le montre le contre-exemple suivant : \( \dfrac{1}{ 3 }\) est fractionnaire, mais n'est pas décimal parce qu'il possède une infinité de chiffres derrière la virgule.

            Note : à l'inverse une fraction comme \( \dfrac{ 1 }{ 3 }\) est un nombre décimal : \( \dfrac{ 1 }{ 4 }=0.25\). Il n'y a que deux chiffres derrière la virgule.
    \end{enumerate}

\end{corrige}
