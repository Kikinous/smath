% This is part of Un soupçon de mathématique sans être agressif pour autant
% Copyright (c) 2014
%   Laurent Claessens
% See the file fdl-1.3.txt for copying conditions.

\begin{exercice}\label{exosmath-0648}

    Dans le parallélépipède suivant, \( AD=2\), \( CD=3\) et \( AE=4\).
\begin{center}
   \input{Fig_CFjmTFb.pstricks}
\end{center}

\begin{enumerate}
    \item
        Calculer la longueur de \(  [DG]\) et \( [DF]\).
    \item
        Quelle est l'aire du triangle \( AEF\) ?
    \item
        Placer un point \( I\) au milieu du segment \( [CG]\). Est-ce que la droite \( (IF)\) intersecte la droite \( (BC)\) ?
    \item
        Est-ce que la droite \( (FC)\) intersecte la droite \( (AD)\) ?
\end{enumerate}

\corrref{smath-0648}
\end{exercice}
