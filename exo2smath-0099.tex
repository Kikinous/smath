% This is part of Un soupçon de mathématique sans être agressif pour autant
% Copyright (c) 2015
%   Laurent Claessens
% See the file fdl-1.3.txt for copying conditions.

\begin{exercice}[\ldots\ldots/5]\label{exo2smath-0099}

    \begin{enumerate}
        \item
            
    Sébastien dit que le nombre \( 7\times n\) n'est dans la table de \( 3\) pour aucune valeur entière de \( n\). A-t-il raison ? Pourquoi ?

\item

    Max veut faire un tour de magie et demande à Sophie : « choisis un nombre quelconque, multiplie-le par 4, et ajoute \( 2\) au résultat». Solphie dit alors avoir obtenu \( 4\). Max accuse alors Sophie de s'être trompée dans ses calculs, ce que Sophie dément. Est-il possible que Sophie ne se soit pas trompée ? Qu'a-t-elle fait ?
    
    \end{enumerate}

\corrref{2smath-0099}
\end{exercice}
