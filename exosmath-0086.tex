% This is part of Un soupçon de mathématique sans être agressif pour autant
% Copyright (c) 2012
%   Laurent Claessens
% See the file fdl-1.3.txt for copying conditions.

\begin{exercice}\label{exosmath-0086}

Un taxi divise son prix en deux paries : $0.2$ euros de frais de prise en charge plus un euro par km parcouru.

Un autre taxi divise son prix en $1$ euro de frais de prise en charge plus $0.8$ euros par kilomètre parcouru.

\begin{enumerate}
    \item
        Combien coûte un trajet de \unit{5}{\kilo\meter} avec le premier taxi ?
    \item
        Combien de kilomètres peut-t-on effectuer dans le second taxi avec \( 10\) euros ?
    \item
        À partir de combien de kilomètres parcourus le second taxi est-il avantageux ?
\end{enumerate}

\corrref{smath-0086}
\end{exercice}
