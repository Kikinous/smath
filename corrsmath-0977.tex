% This is part of Un soupçon de mathématique sans être agressif pour autant
% Copyright (c) 2014
%   Laurent Claessens
% See the file fdl-1.3.txt for copying conditions.

\begin{corrige}{smath-0977}

    \begin{enumerate}
        \item
        \item
            Le second triangle demandé n'est pas possible à dessiner. En effet la somme des longueurs des deux petits côtés fait \( 5+7=12\) alors que le plus long côté fait \( 15\). L'inégalité triangulaire n'est pas respectée.

            Voici un dessin de ce qui arrive lorsqu'on essaye de le dessiner. Nous commençons par dessiner un segment \( [LM]\) de \SI{15}{\centi\meter} et ensuite un cercle de rayon \SI{5}{\centi\meter} autour de \( L\) et un de rayon \SI{7}{\centi\meter} centré en \( M\).


\begin{center}
   \input{Fig_YMGAooVCIBvE.pstricks}
\end{center}
    

    \end{enumerate}
    <++>

\end{corrige}
