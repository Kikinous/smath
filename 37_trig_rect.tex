% This is part of Un soupçon de mathématique sans être agressif pour autant
% Copyright (c) 2015
%   Laurent Claessens
% See the file fdl-1.3.txt for copying conditions.


% This is part of Un soupçon de mathématique sans être agressif pour autant
% Copyright (c) 2015
%   Laurent Claessens
% See the file fdl-1.3.txt for copying conditions.

%--------------------------------------------------------------------------------------------------------------------------- 
\subsection*{Activité : des triangles rectangles}
%---------------------------------------------------------------------------------------------------------------------------

Tracer un segment sur une feuille de papier.  Êtes-vous capable de tracer un triangle rectangle dont ce segment est l'hypoténuse ? Un autre ? Encore un autre ? Et encore et encore \ldots

% This is part of Un soupçon de mathématique sans être agressif pour autant
% Copyright (c) 2015
%   Laurent Claessens
% See the file fdl-1.3.txt for copying conditions.

%--------------------------------------------------------------------------------------------------------------------------- 
\subsection*{Activité : la démonstration pour ceux que ça intéresse}
%---------------------------------------------------------------------------------------------------------------------------

Voici un triangle rectangle en \( B\) :

\begin{center}
\input{Fig_NHGWooBOAkGS.pstricks}
\end{center}
Le point \( K\) est le milieu de l'hypoténuse \( [AC]\). Nous donnons les longueurs \( AC=2\), \( LK=x\) et \( LB=y\).

Le but est de montrer que \( m\) est le moitié de l'hypoténuse, c'est à dire \( m=1\).

\begin{enumerate}
    \item
        Exprimer les longueurs \( AL\) et \( KC\) en fonction de \( x\) et \( y\).
    \item
        Quel segment a pour carré de la longueur le nombre \( x^2+y^2\) ?
    \item
        Exprimer les longueurs \( AB\) et \( BC\) en termes de \( x\) et \( y\). (écrire les égalités de Pythagore dans \( ALB\) et \( LCB\))
    \item
        Écrire l'égalité de Pythagore dans le triangle \( ABC\).
    \item
        Remplacer dans l'égalité de Pythagore toutes les mesures en termes de \( x\) et \( y\).
    \item
        Calculer beaucoup et déduire que \( m=1\).
\end{enumerate}
Aide : on peut utiliser l'égalité \( (1-x)^2+(1+x)^2=2+2x^2\).



%+++++++++++++++++++++++++++++++++++++++++++++++++++++++++++++++++++++++++++++++++++++++++++++++++++++++++++++++++++++++++++ 
\section{Cercle circonscrit}
%+++++++++++++++++++++++++++++++++++++++++++++++++++++++++++++++++++++++++++++++++++++++++++++++++++++++++++++++++++++++++++

\begin{propriete}
    Un triangle \( ABC\) est rectangle en \( A\) si et seulement si le point \( A\) est sur le cercle de diamètre \( [BC]\).
\end{propriete}
Autrement dit :
\begin{itemize}
    \item
        Un triangle \( ABC\) est rectangle en \( A\) si le point \( A\) est sur le cercle de diamètre \( [BC]\).

        Si on a
 \input{Fig_CSFRooVynjkfooZERO.pstricks}
 alors on a 
        \input{Fig_CSFRooVynjkfooONE.pstricks} 

    \item
        Un triangle \( ABC\) est rectangle en \( A\) seulement si le point \( A\) est sur le cercle de diamètre \( [BC]\).

        Si on a
 \input{Fig_CSFRooVynjkfooONE.pstricks}
 alors on a 
        \input{Fig_CSFRooVynjkfooZERO.pstricks} 

\end{itemize}

Au final on a le dessin très synthétique suivant :
\begin{center}
   \input{Fig_QCARooMunXfF.pstricks}
\end{center}


