% This is part of Un soupçon de mathématique sans être agressif pour autant
% Copyright (c) 2014
%   Laurent Claessens
% See the file fdl-1.3.txt for copying conditions.

\begin{exercice}\label{exosmath-0712}

    Un étalage de yaourts se compose comme suit :
    \begin{equation*}
        \begin{array}[]{|c|c|c|c|c|}
            \hline
            &\text{cerise}&\text{fraise}&\text{nature}&\text{total}\\
            \hline
            \text{bio}&20&50&50&120\\
            \hline
            \text{pas bio}&70&150&200&420\\
            \hline
            \text{total}&90&200&250&540\\
            \hline
        \end{array}
    \end{equation*}
    Nous nommons \( C\) l'événement «prendre un yaourt à la cerise», \( F\) l'événement «prendre un yaourt à la fraise», \( N\) l'événement «prendre un yaourt nature» ainsi que \( B\) l'événement «prendre un yaourt bio» et \( P\) l'événement «prendre un yaourt non bio».

    \begin{enumerate}
        \item
            
            Nous supposons être pressé et prendre un yaourt au hasard dans le rayon.
            \begin{enumerate}
                \item
                    Quelle est la probabilité qu'il soit à la fraise ?
                \item
                    Quelle est la probabilité qu'il soit non bio ?
            \end{enumerate}

        \item
            Nous sommes maintenant (un peu) plus attentif.

            \begin{enumerate}
                \item
                    En sachant que le yaourt pris est à la cerise, quelle est la probabilité qu'il soit à bio ?
                \item
                    En sachant que le yaourt pris est bio, quelle est la probabilité qu'il soit nature ?
            \end{enumerate}

        \item

            Calculer \( P(F\cap B)\) et \( P(C\cup P)\).
        \item
            Soit \( K\) l'événement «Prendre un yaourt bio à la fraise ou à la cerise». Décrire par une phrase l'événement contraire, et l'exprimer à l'aide de \( B\), \( P\), \( F\), \( C\), \( N\) et de leurs contraires.
        \item

    \end{enumerate}

\corrref{smath-0712}
\end{exercice}
