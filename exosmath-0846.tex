% This is part of Un soupçon de mathématique sans être agressif pour autant
% Copyright (c) 2014,2016
%   Laurent Claessens
% See the file fdl-1.3.txt for copying conditions.

\begin{exercice}\label{exosmath-0846}

    Aux États-Unis et dans quelques autres pays, on utilise les degrés Fahrenheit (°F) plutôt que des degrés Celsius (\si{\degreecelsius}) pour mesurer des températures. Il faut soustraire $32$ à une température en °F puis diviser par $1,8$ pour la connaître en °C.  

\begin{enumerate}
    \item
        À quelle température en \si{\degreecelsius} correspond \( 100 \) °F ?
    \item
        Adèle, en voyage aux USA veut régler son thermostat sur \SI{20}{\degreecelsius}. Hélas, le thermostat est gradué en Fahrenheit. À quelle température doit-elle le régler ?
    \item
        Adèle essaye de suivre une recette dans un terrible livre de cuisine ramené de son voyage aux USA. Il y est demandé de chauffer de l'eau à \( 212\)°F. À quelle température doit-elle chauffer son eau ?
\end{enumerate}

\corrref{smath-0846}
\end{exercice}
