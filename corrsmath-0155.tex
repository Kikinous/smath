% This is part of Un soupçon de mathématique sans être agressif pour autant
% Copyright (c) 2012-2013
%   Laurent Claessens
% See the file fdl-1.3.txt for copying conditions.

\begin{corrige}{smath-0155}

    \begin{enumerate}
        \item
            \( y=5x-7\).
        \item
            Un vecteur directeur de la droite passant par les points $(1,2)$ et $(4,4)$ est \( \begin{pmatrix}
                4-1    \\ 
                4-2    
            \end{pmatrix}=\begin{pmatrix}
                3    \\ 
                2    
            \end{pmatrix}\). Donc dans l'équation de la droite \( y=axb\) nous avons \( a=\frac{ 2 }{ 3 }\). Nous devons donc chercher une équation de droite 
            \begin{equation}
                f(x)=\frac{ 2 }{ 3 }x+b
            \end{equation}
            qui passe par les points \( (1;2)\) et \( (4;4)\). Nous devons donc fixer \( b\) de telle sorte que \( f(1)=2\) et \( f(4)=4\). Nous calculons
            \begin{equation}
                f(1)=\frac{ 2 }{ 3 }+b,
            \end{equation}
            et cela donne l'équation suivante pour \( b\) :
            \begin{equation}
                \frac{ 2 }{ 3 }+b=2,
            \end{equation}
            et donc \( b=2-\frac{ 2 }{ 3 }=\frac{ 4 }{ 3 }\). Au final l'équation de la droite est
            \begin{equation}
                f(x)=\frac{ 2 }{ 3 }x+\frac{ 4 }{ 3 }.
            \end{equation}
        \item
            Vu le vecteur directeur, nous avons \( f(x)=ax+b\) avec \( a=3\). Il reste à fixer le \( b\) en sachant que la droite doit passer par \( (-1;0)\). Nous devons avoir \( f(-1)=0\) et donc
            \begin{equation}
                3\times (-1)+b=0,
            \end{equation}
            c'est à dire \( b=3\). Donc au final \( f(x)=3x+3\). Un autre point sur la droite est par exemple \( B=(0;3)\).
            
    \end{enumerate}

\end{corrige}
