% This is part of Un soupçon de mathématique sans être agressif pour autant
% Copyright (c) 2012
%   Laurent Claessens
% See the file fdl-1.3.txt for copying conditions.

\begin{exercice}\label{exosmath-0077}

    Soient les points \( A=(-2;1)\) et \( B=(2;-1)\), et la fonction \( f(x)=2x\). Nous nommons \( d\) la droite d'équation \( y=f(x)\).

    Problème 1 : calculer la position d'un point \( C\) sur la droite \( d\) tel que le triangle \( ABC\) soit rectangle en \( C\)

    Problème 2 : calculer la position d'un point \( C\) sur la droite \( d\) tel que le triangle \( ABC\) soit équilatéral.

    \begin{enumerate}
        \item
            Faire un dessin et émettre une conjecture pour le premier problème.
        \item
            Prouver que pour tout point \( K\) sur la droite \( d\), le triangle \( ABK\) est isocèle en \( K\).
        \item
            Donner les coordonnées du point \( C\) en fonction de son abscisse \( x\).
        \item
            Écrire une équation en \( x\) exprimant que le triangle \( ABC\) est rectangle. (il y a plusieurs possibilités). Résoudre cette équation.
        \item
            Écrire une équation en \( x\) exprimant que le triangle \( ABC\) est équilatéral. Résoudre cette équation.
    \end{enumerate}

\corrref{smath-0077}
\end{exercice}
