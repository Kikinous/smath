% This is part of Un soupçon de mathématique sans être agressif pour autant
% Copyright (c) 2014
%   Laurent Claessens
% See the file fdl-1.3.txt for copying conditions.

\begin{exercice}\label{exo2smath-0004}

    Pour chacune des deux situations suivantes, tracer le triangle demandé et son cercle circonscrit, ou expliquer pourquoi vous n'y parvenez pas. Laisser les traits de construction.
    \begin{enumerate}
        \item
            le triangle \( ABC\) dont les mesures sont \( AB=\SI{5}{\centi\meter}\), \( BC=\SI{10}{\centi\meter}\) et \( AC=\SI{4}{\centi\meter}\).
        \item
            le triangle \( KLM\) dont les mesures sont \( KL=\SI{5}{\centi\meter}\), \( LM=\SI{10}{\centi\meter}\) et \( KM=\SI{10}{\centi\meter}\).
    \end{enumerate}


\corrref{2smath-0004}
\end{exercice}
