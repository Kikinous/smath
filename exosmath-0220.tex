% This is part of Un soupçon de mathématique sans être agressif pour autant
% Copyright (c) 2012
%   Laurent Claessens
% See the file fdl-1.3.txt for copying conditions.

\begin{exercice}\label{exosmath-0220}

Soit la fonction \( f\) définie par \( f(x)=4x^2+16x-9\).
\begin{enumerate}
    \item
        Donner l'ensemble de définition de \( f\).
    \item
        Compléter le tableau de variation :
        \begin{equation*}
            \begin{array}[]{c|ccccc}
                x&-\infty&&\ldots&&+\infty\\
                \hline
                &&&&&\\
                f(x)&&\searrow&&\nearrow&\\
                &&&\ldots&&\\
            \end{array}
        \end{equation*}
    \item
        Calculer \( f(-5)\) et \( f(3)\)
    \item
        Tracer approximativement la courbe entre \( -7\) et \( 3\). Indiquer sur le graphique les éléments-clefs qui ont permis de tracer. Aidez-vous des questions précédentes.
    \item
        Calculer les images de \( -10\) et de \( 10\) par la fonction \( f\).
    \item
        Donner un encadrement de \( f(x)\) pour \( 3<x<10\).
    \item
        Donner un encadrement de \( 3f(x)+5\) pour \( -10\leq x\leq -5\).
    \item
        Donner un encadrement de \( -2f(x)\) pour \( -5\leq x\leq -2\).
        
\end{enumerate}
%AFAIRE : cet exercice vient de Guillaume Connan

\corrref{smath-0220}
\end{exercice}
