% This is part of Un soupçon de mathématique sans être agressif pour autant
% Copyright (c) 2012
%   Laurent Claessens
% See the file fdl-1.3.txt for copying conditions.

\begin{exercice}\label{exoSeconde-0039}

Pierre a des notes de trois types : des devoirs à la maison, comptés
avec un coefficient 1 ; des contrôles de cours, comptés coefficient 2
; et des devoirs surveillés, comptés coefficient 3. Au deuxième
trimestre, il a obtenu les notes suivantes :

\begin{center}
  \begin{tabular}{|c||c|c|c|c|c|c|}
      \hline
    \textbf{Note} & 9 & 8 & 15 & 12 & 7 & 16  \\
    \hline
    \textbf{Coefficient} & 3 & 2 & 1 & 2 & 3 & 1 \\
    \hline
  \end{tabular}      
\end{center}

\begin{enumerate}
\item Quelle est la moyenne de Pierre ce trimestre ?

\item Si les notes n'étaient pas pondérés par des coefficients (c'est-à-dire s'ils avaient tous le même coefficient), quelle serait la moyenne de Pierre ?

\item Proposer un jeu de coefficients pour les devoirs à la maison, les contrôles de cours et les devoirs surveillés qui permette à Pierre d'obtenir la moyenne.

  Les règles à respecter sont les suivantes :
  \begin{itemize}
  \item les coefficients sont des entiers ;
  \item aucun coefficient ne peut être nul ;
  \item le coefficient des devoirs surveillés doit être plus grand que
    celui des contrôles de cours, qui doit être plus grand que celui
    des devoirs à la maison.
  \end{itemize}
  En revanche, les coefficients peuvent être aussi grands que nécessaire.    

\item On reprend les coefficients initiaux.

  Le prochain devoir est un devoir surveillé de deux heures, comptés
  coefficient 6. Les notes sont toutes entières.

  Quelle est la note minimale qui permettrait à Pierre d'obtenir la
  moyenne ce trimestre ? D'obtenir au moins 12 de moyenne ? Y a-t-il
  une note qui lui permette d'obtenir au moins 14 de moyenne ? 

  Justifier toutes les réponses.
\end{enumerate}

\corrref{Seconde-0039}
\end{exercice}
