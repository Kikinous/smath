% This is part of Un soupçon de mathématique sans être agressif pour autant
% Copyright (c) 2014
%   Laurent Claessens
% See the file fdl-1.3.txt for copying conditions.

\begin{exercice}[\cite{NRHooXFvgpp}]\label{exosmath-0771}

Marie a recopié l'exercice de mathématiques à faire pour demain. En voici l'énoncé :
\begin{quote}
    ABCD est un quadrilatère tel que : $AB$ = \unit{3}{\centi\meter} ; $BC $= \unit{5}{\centi\meter} ; $AC$ =\unit{7}{\centi\meter} ; $CD$ = \unit{3}{\centi\meter} et $BD$ = \unit{1}{\centi\meter}. 
\end{quote}
Après plusieurs essais sans succès, Marie réalise qu'une des longueurs est fausse.  Laquelle ? La modifier pour qu'il soit possible de placer les quatre points.

\corrref{smath-0771}
\end{exercice}
