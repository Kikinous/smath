% This is part of Un soupçon de mathématique sans être agressif pour autant
% Copyright (c) 2015
%   Laurent Claessens
% See the file fdl-1.3.txt for copying conditions.

\begin{corrige}{smath-0964}

    Justifier pas à pas pourquoi chacune des propositions fausses est fausse est un peu longuet. Le mieux est de trouver la bonne solution et de la justifier :
    \begin{equation}
        \frac{ 15 }{ 21 }=\frac{ 15\div 3 }{ 21\div 3 }=\frac{ 5 }{ 7 }.
    \end{equation}
    La bonne solution est donc la \ref{ItemSEATooZFmNGgd}.

    Note : la proposition $ 0.71428571428571$ n'est pas égale à \( \dfrac{ 5 }{ 7 }\). Pour le justifier, il faut poser et effectuer la division \( 5\div 7\) et remarquer que les décimales \( 571428\) se répètent sans s'arrêter. Le fait que certaines calculatrices écrivent \( 5/7\) lorsqu'on leur donne ce nombre doit inciter à la prudence lorsqu'on a une calculatrice en main.

\end{corrige}
