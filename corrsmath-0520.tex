% This is part of Un soupçon de mathématique sans être agressif pour autant
% Copyright (c) 2013
%   Laurent Claessens
% See the file fdl-1.3.txt for copying conditions.

\begin{corrige}{smath-0520}

\begin{wrapfigure}{r}{7.0cm}
   \vspace{-0.5cm}        % à adapter.
   \centering
   \input{Fig_VXFxyni.pstricks}
\end{wrapfigure}

    Voir le dessin ci-contre.
    \begin{enumerate}
        \item
            \( f(-1)=2\), \( f(0)=1\) et \( f(\frac{ 5 }{2})=-1.5\).
        \item
            Commençons par la fonction \( f\). La question revient à trouver pour quel \( m\) et \( p\) nous avons \( f(x)=mx+p\). L'ordonnée à l'origine est l'ordonnée de la droite pour \( x=0\), c'est à dire \( 1\). Nous avons donc \( f(x)=mx+1\). Il reste à trouver pour quelle valeur de \( m\) ça colle au graphe. 
            
            Pour ce faire nous savons par exemple que la droite passe par le point de coordonnées \( (2;-1)\), donc \( f(2)=-1\), c'est à dire
            \begin{equation}
                m\times 2+1=-1
            \end{equation}
            ce qui donne \( m=-1\). Au final \( f(x)=-x+1\); l'ordonnée à l'origine est \( 1\) et le coefficient directeur est \( -1\).

            Pour la fonction \( g\) nous faisons de même. Nous posons \( g(x)=mx+p\). Du premier coup d'œil nous savons que \( p=-3\) : c'est l'ordonnée du point où la droite \( g\) coupe l'axe vertical. Il suffit maintenant de trouver pour quel \( m\) nous avons \( g(x)=mx-3\). Pour cela nous prenons par exemple le point \( K(2;1)\) par lequel passe le graphe de \( g\). Par la règle d'or des graphiques nous avons \( g(2)=1\), c'est à dire
            \begin{equation}
                m\times 2-3=1
            \end{equation}
            ou encore \( m=2\).
        \item
            La fonction \( h\) est celle dessinée en rouge.
    \end{enumerate}

\end{corrige}
