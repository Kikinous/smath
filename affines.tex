%This is part of Un soupçon de mathématique sans être agressif pour autant
% Copyright (c) 2012-2013
%   Laurent Claessens, Pauline Klein
% See the file fdl-1.3.txt for copying conditions.

%+++++++++++++++++++++++++++++++++++++++++++++++++++++++++++++++++++++++++++++++++++++++++++++++++++++++++++++++++++++++++++ 
\section{Définitions}
%+++++++++++++++++++++++++++++++++++++++++++++++++++++++++++++++++++++++++++++++++++++++++++++++++++++++++++++++++++++++++++

\begin{definition}
    Une \defe{fonction affine}{affine}\index{fonction!affine} est une fonction définie sur \( \eR\) par
    \begin{equation}
        f(x)=ax+b
    \end{equation}
    où \( a\) et \( b\) sont deux nombres réels fixés.
\end{definition}

Cas particuliers :
\begin{enumerate}
    \item
        Si \( b=0\) alors \( f(x)=ax\) et nous disons que \( f\) est une fonction \defe{linéaire}{fonction!linéaire}.
    \item
        Si \( a=0\) alors \( f(x)=b\) et nous disons que \( f\) est une fonction \defe{constante}{fonction!constante}.
\end{enumerate}

%+++++++++++++++++++++++++++++++++++++++++++++++++++++++++++++++++++++++++++++++++++++++++++++++++++++++++++++++++++++++++++ 
\section{Représentation graphique}
%+++++++++++++++++++++++++++++++++++++++++++++++++++++++++++++++++++++++++++++++++++++++++++++++++++++++++++++++++++++++++++

\begin{Aretenir}
    La représentation graphique de la fonction \( f(x)=ax+b\) est une droite non parallèle à l'axe des ordonnées. Réciproquement toute droite non parallèle est la représentation graphique d'une fonction affine.
\end{Aretenir}

Si \( f(x)=ax+b\), alors nous avons \( f(0)=b\), ce qui signifie que \( b\) est la hauteur de la droite au-dessus de l'origine.

En ce qui concerne le signification du coefficient \( a\), nous traçons quelque droites.


\begin{remark}
    Demander aux élèves de faire ces dessins eux-mêmes.
\end{remark}

\newpage

\begin{multicols}{2}
    Sur la figure ci-contre, nous avons tracé les droites
    \begin{enumerate}
        \item
            \( f(x)=2x+1\)
        \item
            \( g(x)=2x-2\)
        \item
            \( h(x)=3x-1\)
    \end{enumerate}

    \columnbreak

%The result is on figure \ref{LabelFigfigureCFoZCYe}. % From file figureCFoZCYe
%\newcommand{\CaptionFigfigureCFoZCYe}{<+Type your caption here+>}
    \begin{center}
\input{Fig_figureCFoZCYe.pstricks}
    \end{center}

\end{multicols}

Nous constatons que
\begin{enumerate}
    \item
        Les droites représentatives de \( f\) et \( g\) sont parallèles.
    \item
        La droite \( h\) est plus pendue que les deux autres.
\end{enumerate}

\begin{definition}
    Soit une droite \( d\) d'équation \( y=ax+b\). Le nombre \( a\) est appelé le \defe{coefficient directeur}{coefficient directeur} de la droite.
\end{definition}



Un exemple des éléments de \( ax+b\) est donné à la figure \ref{LabelFigfigureUERGVgS}. % From file figureUERGVgS
\newcommand{\CaptionFigfigureUERGVgS}{Une droite et quelque éléments de son équation.}
\input{Fig_figureUERGVgS.pstricks}

\begin{propriete}
    Soit \( f(x)=ax+b\) une fonction affine.
    \begin{enumerate}
        \item
            Si \( a>0\) alors \( f\) est croissante sur \( \eR\).
        \item
            Si \( a<0\) alors \( f\) est décroissante sur \( \eR\).
        \item
            Si \( a=0\) alors \( f\) est constante sur \( \eR\) (et vaut \( b\)).
    \end{enumerate}
\end{propriete}


\begin{theorem}
    Soit la fonction affine \( f(x)=ax+b\). Alors pour tout nombres distincts \( u\) et \( v\) nous avons
    \begin{equation}
        \frac{ f(u)-f(v) }{ u-v }=a.
    \end{equation}
\end{theorem}

\begin{proof}
    Soient \( u\) et \( v\), deux réels distincts. Nous calculons
    \begin{equation}
        f(u)-f(v)=au+b-(av+b)=au-av=a(u-v).
    \end{equation}
    En divisant par \( u-v\) on trouve le résultat annoncé.
\end{proof}

Cela donne une méthode graphique pour déterminer le coefficient directeur d'une droite. Si les points \( A=(x_A;y_A)\) et \( B=(x_B;y_B)\) sont sur la droite \( d\) d'équation \( y=ax+b\), alors
\begin{equation}
    a=\frac{ y_B-y_A }{ x_B-x_A }.
\end{equation}
Notez que cela vient du vecteur directeur
\begin{equation}
    \begin{pmatrix}
        x_B-x_A    \\ 
        y_B-y_A    
    \end{pmatrix}
\end{equation}

%+++++++++++++++++++++++++++++++++++++++++++++++++++++++++++++++++++++++++++++++++++++++++++++++++++++++++++++++++++++++++++ 
\section{Tableau de variation}
%+++++++++++++++++++++++++++++++++++++++++++++++++++++++++++++++++++++++++++++++++++++++++++++++++++++++++++++++++++++++++++

Nous savons qu'une fonction affine est croissante ou décroissante suivant le signe de \( a\). Il y a donc deux tableaux possibles. Les tableaux de signes s'en déduisent immédiatement.

\begin{equation}
    \begin{aligned}[]
        &a<0&a>0&\\
&\includegraphics[width=6.0cm]{Picture_FIGLabelFigfigureXUQaJjFssLabelSubFigfigureXUQaJjF0PICTfigureXUQaJjFpspict0-for_eps.png}&
\includegraphics[width=6.0cm]{Picture_FIGLabelFigfigureXUQaJjFssLabelSubFigfigureXUQaJjF1PICTfigureXUQaJjFpspict1-for_eps.png}&\\
        &\begin{array}[]{c|ccccc}
             x&-\infty&&-b/a&&\infty\\
              \hline
              ax+b&&-&0&+&\\ 
               \end{array}&
        \begin{array}[]{c|ccccc}
             x&-\infty&&-b/a&&\infty\\
              \hline
              ax+b&&+&0&-&\\ 
               \end{array}&
    \end{aligned}
\end{equation}

% Au cas où les figures sont modifiées, il faut décommenter ce deux lignes pour que la compilation se passe bien.
%\newcommand{\CaptionFigfigureXUQaJjF}{Hein}
%\input{Fig_figureXUQaJjF.pstricks}

%The result is on figure \ref{LabelFigfigureXUQaJjF}. % From file figureXUQaJjF
%See also the subfigure \ref{LabelFigfigureXUQaJjFssLabelSubFigfigureXUQaJjF0}
%See also the subfigure \ref{LabelFigfigureXUQaJjFssLabelSubFigfigureXUQaJjF1}

%+++++++++++++++++++++++++++++++++++++++++++++++++++++++++++++++++++++++++++++++++++++++++++++++++++++++++++++++++++++++++++ 
\section{Droites parallèles et droites sécantes}
%+++++++++++++++++++++++++++++++++++++++++++++++++++++++++++++++++++++++++++++++++++++++++++++++++++++++++++++++++++++++++++

%--------------------------------------------------------------------------------------------------------------------------- 
\subsection{Critère avec le coefficient directeur}
%---------------------------------------------------------------------------------------------------------------------------

\begin{example}
    Les droites \( y=-3x+4\) et \( y=2x+1\) sont sécantes, comme nous pouvons le voir sur un dessin. Comment trouver les coordonnées du point d'intersection ?

    Le point d'intersection \( (x_0;y_0)\) doit satisfaire \( y_0=-3x_0+4\) et \( y_0=2x_0+1\) en même temps. Donc
    \begin{equation}
        -3x_0+4=2x_0+1,
    \end{equation}
    ce qui donne \( x_0=\frac{ 3 }{ 5 }\). Pour trouver \( y_0\) il suffit de remplacer dans l'une ou l'autre équation :
    \begin{equation}
        y_0=-3x_0+4=-3\frac{ 3 }{ 5 }+4=\frac{ 11 }{ 5 }.
    \end{equation}
    Vérification : dans l'autre on obtient le même résultat :
    \begin{equation}
        y_0=2x_0+1=2\frac{ 3 }{ 5 }+1=\frac{ 11 }{ 5 }.
    \end{equation}
    Donc le point d'intersection des deux droites est le point de coordonnées
    \begin{equation}
        \big( \frac{ 3 }{ 5 };\frac{ 11 }{ 5 } \big).
    \end{equation}
\end{example}

\begin{theorem}
    Dans un repère, les droites d'équations \( y=ax+b\) et \( y=a'x+b'\) sont parallèles si et seulement si elles ont même coefficient directeur, c'est à dire si et seulement si \( a=a'\).

    Les droites sont sécantes si et seulement si \( a\neq a'\).
\end{theorem}

\begin{proof}
    Nous allons seulement démontrer que si \( a\neq a'\), alors les droites s'intersectent. Pour cela nous allons montrer qu'il existe un point \( (x_0,y_0)\) qui se trouve sur le graphe des deux fonctions, c'est à dire qui vérifie
    \begin{subequations}
        \begin{numcases}{}
            y_0=ax_0+b\\
            y_0=a'x_0+b'.
        \end{numcases}
    \end{subequations}
    Le nombre \( x_0\) doit donc satisfaire
    \begin{equation}
        ax_0+b=a'x_0+b',
    \end{equation}
    c'est à dire \( (a-a')x_0=b'-b\) et donc
    \begin{equation}
        x_0=\frac{ b'-b }{ a-a' }.
    \end{equation}
\end{proof}

%--------------------------------------------------------------------------------------------------------------------------- 
\subsection{Et le vecteur directeur}
%---------------------------------------------------------------------------------------------------------------------------

Si \( A\) et \( B\) sont des points du plan, nous avons déjà vu que le vecteur \( \vect{ AB }\) était un critère de parallélisme de la droite \( AB\). Plus précisément, nous avons vu qu'une droite \( (CD)\) était parallèle à \( AB\) si et seulement si \( \vect{ CD }\) est un multiple de \( \vect{ AB }\). Voyons cela de plus près.

Soit la droite \( y=ax+b\). Les points \( A=(0;b)\) et \( B=(1,a+b)\) sont sur cette droite et donc le vecteur directeur est
\begin{equation}
    \vect{ AB }=\begin{pmatrix}
        1    \\ 
        a    
    \end{pmatrix}
\end{equation}
Si les points \( C\) et \( D\) sont tels que \( (CD)\parallel (AB)\), alors
\begin{equation}
    \vect{ CD }=\lambda\begin{pmatrix}
        1    \\ 
        a    
    \end{pmatrix}=\begin{pmatrix}
        \lambda    \\ 
            \lambda a
    \end{pmatrix}.
\end{equation}
Si le point \( C\) a pour coordonnées \( C=(x_C;y_C)\), alors le point \( D\) doit avoir comme coordonnées
\begin{equation}
    D=\big( x_C+\lambda;y_C+\lambda a \big).
\end{equation}
Le coefficient directeur de la droite \( (CD)\) est alors
\begin{equation}
    \frac{ (y_C+\lambda a)-y_C }{ (x_C+\lambda)-x_C }=a.
\end{equation}
Nous avons donc confirmé que le critère du vecteur proportionnel et le critère du coefficient directeur sont en réalité les mêmes.

%+++++++++++++++++++++++++++++++++++++++++++++++++++++++++++++++++++++++++++++++++++++++++++++++++++++++++++++++++++++++++++ 
\section{Colinéarité}
%+++++++++++++++++++++++++++++++++++++++++++++++++++++++++++++++++++++++++++++++++++++++++++++++++++++++++++++++++++++++++++

\begin{theorem}
    Trois points distincts \( A\), \( B\) et \( C\) sont alignés si et seulement si les droites \( (AB)\) et \( (AC)\) ont même coefficient directeur.
\end{theorem}

\begin{proof}
    Si les droites \( (AB)\) et \( (AC)\) ont même coefficient directeur, alors nous pouvons écrire
    \begin{equation}
        (AB)\equiv y=ax+b
    \end{equation}
    et
    \begin{equation}
        (AC)\equiv ax+b'.
    \end{equation}
    Cependant \( A\) est un point commun aux deux droites : \( y_A=ax_A+b=ax_A+b'\) et donc \( b=b'\), ce qui signifie que les droites \( (AB)\) et \( (AC)\) sont les mêmes.
\end{proof}

\begin{example}
    Vérifions que les points \( A=(1;-1)\), \( B=(3;5)\) et \( C=(4;8)\) sont alignés.

    D'abord le coefficient directeur de la droite \( (AB)\) est 
    \begin{equation}
        \frac{ y_B-y_A }{ x_B-x_A }=\frac{ 5-(-1) }{ 3-1 }=3
    \end{equation}
    ensuite le coefficient directeur de la droite \( (AC)\) est
    \begin{equation}
        \frac{ y_C-y_A }{ x_C-x_A }=\frac{ 8+1 }{ 4-1 }=3
    \end{equation}
    Donc les points \( A\), \( B\) et \( C\) sont alignés.
\end{example}

%+++++++++++++++++++++++++++++++++++++++++++++++++++++++++++++++++++++++++++++++++++++++++++++++++++++++++++++++++++++++++++ 
\section{Exercices}
%+++++++++++++++++++++++++++++++++++++++++++++++++++++++++++++++++++++++++++++++++++++++++++++++++++++++++++++++++++++++++++

%---------------------------------------------------------------------------------------------------------------------------
\subsection{Équations linéaires et affines}
%---------------------------------------------------------------------------------------------------------------------------

\Exo{smath-0016}

%--------------------------------------------------------------------------------------------------------------------------- 
\subsection{Équations de droites}
%---------------------------------------------------------------------------------------------------------------------------

\Exo{smath-0128}
\Exo{smath-0149}
\Exo{smath-0130}
\Exo{smath-0069}
\Exo{smath-0135}
\Exo{smath-0229}
\Exo{smath-0150}
\Exo{smath-0155}
\Exo{smath-0151}
\Exo{smath-0134}
\Exo{smath-0086}
\Exo{smath-0138}
\Exo{smath-0200}
\Exo{smath-0233}
\Exo{smath-0234}

%--------------------------------------------------------------------------------------------------------------------------- 
\subsection{Tableaux de variation et de signe}
%---------------------------------------------------------------------------------------------------------------------------

\Exo{smath-0132}
\Exo{smath-0148}

%---------------------------------------------------------------------------------------------------------------------------
\subsection{Parallélisme et intersection}
%---------------------------------------------------------------------------------------------------------------------------

\Exo{smath-0083}
\Exo{smath-0129}
\Exo{smath-0110}    % TODO: il faut plus d'exercices comme celui-ci.
\Exo{smath-0084}
\Exo{smath-0153}
\Exo{smath-0154}
\Exo{smath-0330}        % Cet exo est le même que le smath-155, mais en plus court pour rentrer dans un devoir.
\Exo{smath-0156}
\Exo{smath-0157}
\Exo{smath-0185}
\Exo{smath-0000}
\Exo{smath-0235}
\Exo{smath-0236}
\Exo{smath-0237}
\Exo{smath-0239}
\Exo{smath-0339}

%TODO : écrire un exercice avec un rebond sur une table de billard.

%--------------------------------------------------------------------------------------------------------------------------- 
\subsection{Systèmes d'équation}
%---------------------------------------------------------------------------------------------------------------------------

\Exo{smath-0226}
\Exo{smath-0227}
\Exo{smath-0228}
\Exo{smath-0240}

%---------------------------------------------------------------------------------------------------------------------------
\subsection{Orthogonalité}
%---------------------------------------------------------------------------------------------------------------------------

% Note : l'orthogonalité n'a pas l'air d'être au programme.

\Exo{smath-0077}
\Exo{smath-0082}

%--------------------------------------------------------------------------------------------------------------------------- 
\subsection{Problèmes}
%---------------------------------------------------------------------------------------------------------------------------

\Exo{smath-0241}
\Exo{smath-0242}
