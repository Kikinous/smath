% This is part of Un soupçon de mathématique sans être agressif pour autant
% Copyright (c) 2013
%   Laurent Claessens
% See the file fdl-1.3.txt for copying conditions.

\begin{exercice}\label{exosmath-0582}

    Nous voulons donner une approximation du nombre \( \pi\) de la façon suivante : nous tirons \( 500\) points au hasard dans le carré centré à l'origine et de côté \( 2\).
    \begin{enumerate}
        \item
            Quelle est l'aire de ce carré ?
        \item
            Quelle est l'aire du cercle de rayon \( 1\) centré en \( (0;0)\) ?
        \item
            Faire un dessin des deux.
        \item
            Quelle est (en théorie) la probabilité pour qu'un point au hasard dans le carré soit dans le cercle ?
    \end{enumerate}

\corrref{smath-0582}
\end{exercice}
