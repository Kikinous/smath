% This is part of Un soupçon de mathématique sans être agressif pour autant
% Copyright (c) 2012
%   Laurent Claessens
% See the file fdl-1.3.txt for copying conditions.

\begin{exercice}\label{exoSeconde-0034}

    Le tableau suivant donne le nombre de demandeurs d'emplois en fonction de l'âge et du sexe dans un village.

    \begin{center}
        \begin{tabular}{|c|c|c|}
            \hline
            âge&hommes&femmes\\
            \hline\hline
            \( \mathopen[ 16 , 26 [\)&\( 280\)&\( 160\)\\
                \hline
                \( \mathopen[ 26 , 40 [\)&\( 310\)&\( 360\)\\
                    \hline
            \( \mathopen[ 40 , 50 [\)&\( 240\)&\( 120\)\\
                \hline
             \( \mathopen[ 50 , 60 [\)&\( 420\)&\( 530\)\\
                 \hline
              \( \mathopen[ 60 , 65 [\)&$70$&$50$\\
                  \hline
        \end{tabular}
    \end{center}

    \begin{enumerate}
        \item
    Dans quelle case sont les demandeurs d'emplois ont exactement 40 ans ?
\item 
    Tracer les deux courbes de fréquences cumulées. 
\item
    Parmi les demandeurs d'emplois de \( 16\) à \( 50\) à (non compris), quelle est la proportion d'hommes ?
            
    \end{enumerate}

\corrref{Seconde-0034}
\end{exercice}
