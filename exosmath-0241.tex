% This is part of Un soupçon de mathématique sans être agressif pour autant
% Copyright (c) 2012
%   Laurent Claessens
% See the file fdl-1.3.txt for copying conditions.

\begin{exercice}\label{exosmath-0241}

    Soient les points \( A=(\frac{1}{ 2 };\frac{ 3 }{ 2 })\), \( B=(\frac{ 9 }{2});\frac{ 5 }{2}\) et \( C=3;6\). Nous notons $A'$, \( B'\) et \( C'\) les milieux respectifs de \( [BC]\), \( [AC]\) et \( [AB]\).
    \begin{enumerate}
        \item
            Calculer les coordonnées des points \( A'\), \( B'\), \( C'\).
        \item
            Déterminer les équations des droites \( (AA')\), \( BB'\) et \( CC'\).
        \item
            Montrer que les droites \( (AA')\) et \( (BB')\) sont sécantes, et calculer les coordonnées du point d'intersection. Nous notons \( G\) ce point.
        \item
            Montrer que \( G\) appartient également à la droite \( (CC')\).
        \item
            À quelle propriété des triangles vue au collège tout ceci vous fait penser ?
    \end{enumerate}

%AFAIRE : cet exercice vient de Guillaume Connan

\corrref{smath-0241}
\end{exercice}
