% This is part of Un soupçon de mathématique sans être agressif pour autant
% Copyright (c) 2015
%   Laurent Claessens
% See the file fdl-1.3.txt for copying conditions.

\begin{exercice}\label{exo2smath-0084}

\begin{center}
   \input{Fig_GLKFooKwvxSl.pstricks}
\end{center}

\begin{enumerate}
    \item
        Quelles sont les coordonnées du point \( A\) ?
    \item       \label{ItemKNMKooPlguMAii}
        Placer les points \( B(4;1)\) et \( C(2;-3)\) dans ce repère :
    \item
        Tracer la hauteur issue du sommet \( C \).
    \item
        Donner les coordonnées du point d'intersection entre cette hauteur et la droite \( (AB)\).
\end{enumerate}
Note : si vous ne parvenez pas à placer les points \( B\) et \( C\) comme demandé en \ref{ItemKNMKooPlguMAii}, placez-les où vous voulez et continuez l'exercice.

\corrref{2smath-0084}
\end{exercice}
