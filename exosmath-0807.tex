% This is part of Un soupçon de mathématique sans être agressif pour autant
% Copyright (c) 2014
%   Laurent Claessens
% See the file fdl-1.3.txt for copying conditions.

\begin{exercice}[\cite{NRHooXFvgpp4}]\label{exosmath-0807}

Soient les deux programmes de calculs suivants :

\begin{framed}
    {\bf Programme 1 :}
\begin{enumerate}
    \item
 Choisir un nombre ;
\item
 Ajouter $6$ à ce nombre ;
\item
 Multiplier le résultat par $-2$ ;
\item
 Ajouter le quadruple du nombre choisi au départ.
\end{enumerate}
\end{framed}
\begin{framed}
    {\bf Programme 2 :}
\begin{enumerate}
    \item
 Choisir un nombre ;
    \item
 Soustraire $3$ à ce nombre ;
    \item
 Multiplier le résultat par $-4$ ;
    \item
 Ajouter le double du nombre choisi au départ.
\end{enumerate}
\end{framed}

\begin{enumerate}
    \item
 Tester ces deux programmes de calculs pour $x = 2$ ; pour $x = -3$ et enfin pour $x = 4$.
\item   \label{ItemAMPooMNffsb}
    Qu'est-ce qu'on remarque ?
\item
 Si l'on note $x$ le nombre choisi au départ, écrire une expression $A$ qui traduit le programme $1$.
\item
 De la même manière, écris une expression $B$ pour le programme $2$.
\item
    Est-ce que l'on peut expliquer le résultat de la question \ref{ItemAMPooMNffsb}.
\end{enumerate}

\corrref{smath-0807}
\end{exercice}
