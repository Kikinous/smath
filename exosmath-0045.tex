% This is part of Un soupçon de mathématique sans être agressif pour autant
% Copyright (c) 2012
%   Laurent Claessens
% See the file fdl-1.3.txt for copying conditions.

\begin{exercice}\label{exosmath-0045}

    Soit \( f(x)=x^2+3x-1\) et \( g(x)=15-3x\).
    \begin{enumerate}
        \item
            Calculer \( f(2)\) et \( g(2)\).
        \item
            Est-ce que \( f\) et \( g\) sont égales ?
        \item
            Pour quelle valeur de \( a\) peut-t-on écrire \( f(x)-g(x)=(x-2)(x-a)\) ?
        \item
            Résoudre l'équation \( x^2+3x-1=15-3x\), c'est à dire \( f(x)-g(x)=0\).
    \end{enumerate}
    Les deux premières questions sont (devraient être) faciles. Les deux suivantes sont plus compliquées.

\corrref{smath-0045}
\end{exercice}
