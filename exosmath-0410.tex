% This is part of Un soupçon de mathématique sans être agressif pour autant
% Copyright (c) 2013
%   Laurent Claessens
% See the file fdl-1.3.txt for copying conditions.

\begin{exercice}\label{exosmath-0410}

    Les points \( (1;1)\), \( (4;-1)\) et \( (-1;5)\) forment-ils un triangle isocèle ? Répondre en utilisant la définition d'un triangle isocèle et le théorème de Pythagore.

\corrref{smath-0410}
\end{exercice}
