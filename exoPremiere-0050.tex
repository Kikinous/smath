% This is part of Un soupçon de mathématique sans être agressif pour autant
% Copyright (c) 2012
%   Laurent Claessens
% See the file fdl-1.3.txt for copying conditions.

\begin{exercice}\label{exoPremiere-0050}

    Écrire un programme qui donne un nombre au hasard entre \( 1\) et \( 100\). Pour ce faire, il y a deux méthodes.
    \begin{enumerate}
        \item
            Prendre un élément au hasard dans la liste \info{range(1,101)}.
        \item
            Utiliser la commande \info{random.randint}.
    \end{enumerate}

\corrref{Premiere-0050}
\end{exercice}
