% This is part of Un soupçon de mathématique sans être agressif pour autant
% Copyright (c) 2013
%   Laurent Claessens
% See the file fdl-1.3.txt for copying conditions.

\begin{exercice}\label{exosmath-0338}

    Un jeu consiste à lancer \( 100\) fois un dé. Chaque fois que le joueur fait \( 5\) ou \( 6\), il gagne un euro. Le but de l'exercice est de déterminer si nous pouvons accuser l'organisateur du jeu d'avoir truqué le dé si nous ne gagnons que 10 euros.

    Ouvrir le fichier \info{IF\_TD\_seconde.py}, et répondre aux questions suivantes.
    \begin{enumerate}
        \item
            Est-ce raisonnable d'espérer gagner \( 95\) euros à ce jeu ? En moyenne, combien peut-on espérer gagner ?
        \item

          La ligne \info{x=random.randint(1,6)} donne à \info{x} une valeur entière aléatoire entre \( 1\) et \( 6\). Sachant cela, que fait la fonction \info{lancer} ?
        \item
            Ajouter au bout du programme les lignes 
            \begin{verbatim}
            a=lancer()
            print(a)
            \end{verbatim}
            et tester quelque fois. Quel est le nombre que vous obtenez le plus souvent ? Est-ce normal ?
        \item
            Supprimer ces lignes, et regarder la fonction \info{jeu}. Que fait-elle ? 
        \item
            Ajouter au bout du programme les lignes 
            \begin{verbatim}
            a=jeu()
            print(a)
            \end{verbatim}
            et tester quelque fois. Quels sont les nombres affichés ? Quelle est approximativement la taille des nombres auxquels vous vous attendez ?
        \item
            Que fait la fonction \info{plusieurs\_jeux} ? Testez-la en ajoutant la ligne suivante au bout du programme :
            \begin{verbatim}
            print(plusieurs_jeux())
            \end{verbatim}
        \item
            Si on vous proposait soit \( 19\)€, soit le droit de jouer à ce jeu, que choisiriez-vous ? Motivez votre réponse à partir des résultats de la fonction \info{plusieurs\_jeux}.
        \item
            Modifier la fonction \info{plusieurs\_jeux} pour qu'elle affiche la \emph{fréquence} des lancers gagnants au lieu du \emph{nombre} de lancers gagnants.
    \end{enumerate}

\corrref{smath-0338}
\end{exercice}
