% This is part of Un soupçon de mathématique sans être agressif pour autant
% Copyright (c) 2013
%   Laurent Claessens
% See the file fdl-1.3.txt for copying conditions.

\begin{exercice}\label{exosmath-0247}

    Le temps d'attente au téléphone lors d'un appel au service après vente est un facteur important de satisfaction du client. On estime qu'un client est satisfait si il attends \( 5\) minutes ou moins; il sera très insatisfait d'attendre \( 10\) minutes ou plus.

    Sur le diagramme suivant, nous avons mis en bleu les effectifs en fonction du temps d'attente du premier centre d'appel et en rouge ceux du second centre d'appel :

%The result is on figure \ref{LabelFigfigureKAzSlQr}. % From file figureKAzSlQr
%\newcommand{\CaptionFigfigureKAzSlQr}{<+Type your caption here+>}
    \begin{center}
\input{Fig_figureKAzSlQr.pstricks}
    \end{center}
    Les données ne sont pas très réalistes; elles sont choies pour illustrer «clairement» un phénomène.

    \begin{enumerate}
        \item
            Donner la moyenne et l'écart-type du temps d'attente des deux centres.
        \item
            Calculer la médiane et les quartiles pour les deux centres d'appels.
        \item
            Quel est le centre d'appel qui évite au mieux d'avoir des clients très insatisfaits ? (est-ce qu'il est préférable d'utiliser le couple moyenne--écart-type ou le couple médiane-quartile pour répondre à cette question ?)
        \item
            Quel est le centre d'appel ayant le nombre de clients satisfaits le plus élevé ?
    \end{enumerate}

\corrref{smath-0247}
\end{exercice}
