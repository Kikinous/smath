% This is part of Un soupçon de mathématique sans être agressif pour autant
% Copyright (c) 2014
%   Laurent Claessens
% See the file fdl-1.3.txt for copying conditions.

\begin{exercice}\label{exosmath-0715}

    Sarah, paysagiste doit créer un parterre de fleurs dans un jardin rectangulaire aux dimensions de \unit{8}{\meter} par \unit{12}{\meter}. Il propose de planter des fleurs dans la partie grisée du dessin suivant :

\begin{center}
   \input{Fig_SZYooRuSplc.pstricks}
\end{center}

Les longueurs \( MB\) et \( ND\) sont identiques et nommées \( x\). Le point \( P\) est placé de telle sorte que \( PQCR\) soit un carré. Sarah nomme \( S\) la fonction qui à \( x\) fait correspondre l'aire du parterre.

\begin{enumerate}
    \item
        Pour quelles valeurs de \( x\) la fonction \( S\) est-elle définie ?
    \item
        Sarah demande l'aide de deux amis pour avoir une formule pour \( S\). Sam lui répond \( S(x)=x^2-10x+30\) tandis que Jacques lui répond \( S(x)=x^2-12x+96\). Laquelle est la bonne ?
    \item
        Dans un premier temps, Sarah présente son projet à la mairie.
        \begin{enumerate}
            \item
                Dresser un tableau de variations de \( S\).
            \item
                Pour quelle valeur de \( x\) le parterre est le plus petit ? Quelle est l'aire correspondante ?
            \item
                Pour quelle valeur de \( x\) le parterre est le plus grand ? Quelle est l'aire correspondante ?
        \end{enumerate} 
    \item
        Hélas le budget «pesticide, désherbant et engrais chimique» est limité et la mairie ne peut pas s'offrir d'entretenir plus de \unit{96}{\squared\meter} de parterre fleuri. Sarah doit donc résoudre \( S(x)\leq 96\).
        \begin{enumerate}
            \item
                Montrer que l'équation \( S(x)\leq 96\) est équivalente à \( (x-6)(x+1)<0\).
            \item
                Dresser le tableau de signe de l'expression \( (x-6)(x+1)\).
            \item
                Conclure : quelle valeur de \( x\) Sarah peut-elle choisir pour que l'aire du parterre n'excède pas \unit{96}{\squared\meter} ?
        \end{enumerate}
\end{enumerate}

\corrref{smath-0715}
\end{exercice}
