% This is part of Un soupçon de mathématique sans être agressif pour autant
% Copyright (c) 2014
%   Laurent Claessens
% See the file fdl-1.3.txt for copying conditions.

\begin{exercice}\label{exosmath-0709}

    Un lycée contient \( 500\) élèves dont \( 350\) sont inscrits à une activité sportive et \( 200\) à une activité musicale. En outre, \( 100\) élèves sont inscrits au deux. Nous prenons un élève au hasard dans le lycée et nous considérons les événements suivants :
    \begin{itemize}
        \item A : «l'élève choisi fait du sport» 
        \item B : «l'élève choisi fait de la musique».
    \end{itemize}
    \begin{enumerate}
        \item
            Déterminer les probabilités de \( A\) et \( B\).
        \item
            Décrire en français l'événement \( A\cap B\). Calculer \( P(A\cap B)\).
        \item
            Décrire en français l'événement \( A\cup B\). Calculer \( P(A\cup B)\).
        \item
            Sachant que l'élève choisit ne joue pas de musique, calculer la probabilité qu'il pratique un sport.
    \end{enumerate}

\corrref{smath-0709}
\end{exercice}
