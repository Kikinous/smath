% This is part of Un soupçon de mathématique sans être agressif pour autant
% Copyright (c) 2015
%   Laurent Claessens
% See the file fdl-1.3.txt for copying conditions.

\begin{exercice}\label{exo2smath-0070}

    Reporter les angles donnés sur le dessin et calculer les angles manquants.
    \begin{enumerate}
        \item
   \input{Fig_SHSRooHgvofo0.pstricks}sachant
             \( \hat C=\SI{120}{\degree}\) et \( \hat{M}=\SI{40}{\degree}\)
\item
    \input{Fig_SHSRooHgvofo1.pstricks}sachant \( \hat{K}=\SI{50}{\degree}\)
\item
    \input{Fig_SHSRooHgvofo2.pstricks}sachant \( \hat{R}=\SI{100}{\degree}\) et \( \hat{T}=\SI{20}{\degree}\).
\item
\begin{center}
   \input{Fig_SHSRooHgvofo3.pstricks}
\end{center}
\item
\begin{center}
   \input{Fig_SHSRooHgvofo4.pstricks}
\end{center}
\item
\begin{center}
   \input{Fig_SHSRooHgvofo5.pstricks}
\end{center}
            
    \end{enumerate}
    <++>

\corrref{2smath-0070}
\end{exercice}
