% This is part of Un soupçon de mathématique sans être agressif pour autant
% Copyright (c) 2014
%   Laurent Claessens
% See the file fdl-1.3.txt for copying conditions.

% This is part of Un soupçon de mathématique sans être agressif pour autant
% Copyright (c) 2014
%   Laurent Claessens
% See the file fdl-1.3.txt for copying conditions.

%--------------------------------------------------------------------------------------------------------------------------- 
\subsection*{Partage d'un carré}
%---------------------------------------------------------------------------------------------------------------------------

\begin{wrapfigure}[2]{r}{4.0cm}
   \vspace{-1cm}        % à adapter.
   \centering
   \input{Fig_LXQooZPbZml.pstricks}
\end{wrapfigure}

Répondre aux questions à partir du dessin ci-contre.
\begin{enumerate}
    \item
                L'aire de la région hachurée représente \( \dfrac{ 1 }{ \ldots }\) de l'aire totale.
            \item
                L'aire de la région remplie représente \( \dfrac{ 3 }{ \ldots }\) de l'aire totale.
    \item
        Ensemble, ces deux régions forment \( \dfrac{ \ldots }{ \ldots }\) de l'aire totale.
\end{enumerate}

De \cite{NRHooXFvgpp5}

% This is part of Un soupçon de mathématique sans être agressif pour autant
% Copyright (c) 2014
%   Laurent Claessens
% See the file fdl-1.3.txt for copying conditions.

%--------------------------------------------------------------------------------------------------------------------------- 
\subsection*{Confiture sucrée}
%---------------------------------------------------------------------------------------------------------------------------

Après un bel été bien ensoleillé, Philippe souhaite faire de la confiture pas trop sucrée. En regardant sur Internet, il trouve trois recettes.

\begin{center}
    \begin{tabular}[]{|c|c|}
        \hline
        Confiture de fraises&«\unit{450}{\gram} de sucre pour \unit{750}{\gram} de fraises» \\
        \hline
        Confiture d'abricots& «\unit{500}{\gram} de sucre pour \unit{1}{\kilo\gram} de confiture» \\
        \hline
        Confiture de cerises&  «\unit{800}{\gram} de sucre pour \unit{2400}{\gram} de cerises» \\ 
        \hline
    \end{tabular}
\end{center}


\begin{enumerate}
    \item
Pour chaque recette, exprimer la proportion de sucre ajouté dans la confiture sous forme de fraction.
\item
    Simplifier le plus possible les fractions obtenues à la question précédente.
\item
    Que signifie une proportion de sucre ajouté supérieure à \( \dfrac{ 1 }{ 2 }\) ?
\end{enumerate}


Philippe cherche à savoir quelle est la recette avec le moins de sucre ajouté. Il fait le raisonnement suivant : « C'est dans la confiture de fraises qu'on retrouve la masse de sucre ajouté la moins importante (\unit{450}{\gram}), c'est donc dans la confiture de fraises qu'il y a le moins de sucre ajouté. ». 

\begin{enumerate}
    \item
        
Que penser de ce raisonnement ?
\item
Pour aider Philippe dans son choix, récrire les ingrédients nécessaires à la réalisation de \unit{1}{\kilo} de confiture.
\item
Quelle est la confiture qui contient le moins de sucre ajouté en proportion ?
\end{enumerate}

De \cite{NRHooXFvgpp5}

%+++++++++++++++++++++++++++++++++++++++++++++++++++++++++++++++++++++++++++++++++++++++++++++++++++++++++++++++++++++++++++ 
\section{Fraction et proportion}
%+++++++++++++++++++++++++++++++++++++++++++++++++++++++++++++++++++++++++++++++++++++++++++++++++++++++++++++++++++++++++++

\begin{Aretenir}
    Une fraction indique une proportion. 
\end{Aretenir}

\begin{example}
    La fraction \( \dfrac{ 3 }{ 4 }\) représente soit
    \begin{enumerate}
        \item
            la situation où nous avons \emph{un} kilo de confiture que nous partageons en \( 4\) et dont nous prenons trois morceaux.
        \item
            la situation où nous avons \emph{trois} kilos de confiture que nous partageons en \( 4\).
    \end{enumerate}
\end{example}

\begin{Aretenir}
Si le numérateur d'une fraction est supérieur à son dénominateur alors le nombre représenté est supérieur à $1$.

Si son numérateur est inférieur à son dénominateur alors le nombre représenté est inférieur à $1$.
\end{Aretenir}

\begin{example}
    \begin{equation}
        \frac{ 3 }{ 2 }>1
    \end{equation}
    Manger trois demi-pizza revient à en manger plus d'une entière. De plus
    \begin{equation}
        \frac{ 3 }{ 2 }=3\times \frac{ 1 }{2}=3\times 0.5=1.5.
    \end{equation}
\end{example}

%+++++++++++++++++++++++++++++++++++++++++++++++++++++++++++++++++++++++++++++++++++++++++++++++++++++++++++++++++++++++++++ 
\section{Simplification}
%+++++++++++++++++++++++++++++++++++++++++++++++++++++++++++++++++++++++++++++++++++++++++++++++++++++++++++++++++++++++++++

\begin{Aretenir}
    La valeur d'une fraction ne change pas si on divise le numérateur et le dénominateur par un même nombre.
\end{Aretenir}

\begin{example}
    Des pizzas sont coupées en douze. J'en mange \( 3\) morceaux. J'ai mangé
    \begin{equation}
        \frac{ 3 }{ 12 }
    \end{equation}
    pizza. On peut simplifier par \( 3\) :
    \begin{equation}
        \frac{ 3 }{ 12 }=\frac{1}{ 4 }.
    \end{equation}
    J'ai mangé un quart de pizza.
\end{example}

\begin{example}
    Un tiers d'une année est composée de \( 4\) mois. En effet \( 4\) mois font
    \begin{equation}
        \frac{ 4 }{ 12 }\,\text{années}=\frac{ 1 }{ 3 }\,\text{années}
    \end{equation}
\end{example}

\begin{example}
    \begin{equation}
        \frac{ 12 }{ 9 }=\frac{ 4 }{ 3 }.
    \end{equation}
\end{example}

%+++++++++++++++++++++++++++++++++++++++++++++++++++++++++++++++++++++++++++++++++++++++++++++++++++++++++++++++++++++++++++ 
\section{Approximation décimale}
%+++++++++++++++++++++++++++++++++++++++++++++++++++++++++++++++++++++++++++++++++++++++++++++++++++++++++++++++++++++++++++

\begin{Aretenir}
    La fraction \( \dfrac{ a }{ b }\) représente le quotient \( a\div b\). Effectuer la division donne une approximation de la valeur de la fraction.
\end{Aretenir}

\begin{example}
    Une approximation de \( \dfrac{ 9563 }{ 123 }\) est
    \begin{equation}
     9563\div 123\simeq 77.74796.
    \end{equation}
    Cette fraction vaut donc environ \( 77\) et trois quart.
\end{example}

\begin{example}
    La fraction
    \begin{equation}
        \frac{ 3 }{ 4 }
    \end{equation}
    vaut \( 0.75\) et ce n'est pas une approximation.
\end{example}
