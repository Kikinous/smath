% This is part of Un soupçon de mathématique sans être agressif pour autant
% Copyright (c) 2012-2013
%   Laurent Claessens
% See the file fdl-1.3.txt for copying conditions.

\begin{exercice}\label{exosmath-0221}

        Sur ce dessin, \( MNBP\) est un rectangle, \( C\) est le milieu de \( [MN]\) et $A$ est le milieu de \( [BP]\). Nous avons les mesures \( AB=\)\unit{4}{\centi\meter} et \( AC=\)\unit{3}{\centi\meter}.

    \begin{center}
\input{Fig_figureFGgTGJA.pstricks}
    \end{center}

    Compléter les égalités suivantes de telles façon à ce qu'elles soient correctes.
    \begin{multicols}{2}
        \begin{enumerate}
            \item
                \( \vect{ BA }+\vect{ CM }=\vect{ B? }\) 
            \item
                \( \vect{ PB }+\vect{ PM }=\vect{ P? }\)
            \item
                \( \vect{ CP }+\vect{ CM }=\vect{ ?? }\)
            \item
                \( \vect{ NB }+\vect{ CA }-\vect{ NA }=\vect{ ?? }\)
            \item
                \( \vect{ AM }+\vect{ AB }=\vect{ A? }\)
            \item
                \( \vect{ MC }+\vect{ AB }=\vect{ P? }\)
            \item
                \( \vect{ AC }+\vect{ AB }=\vect{ ?? }\)
            %\item
            %    \( AB+BN=\)\unit{?}{\centi\meter}.
            \item
                \( \vect{ NC }+\vect{ BC }=\vect{ B? }\)
            \item
                \( \vect{ BC }-\vect{ PM }=\vect{ C? }\)
            \item
                \( NA-BA\)
        \end{enumerate}
    \end{multicols}

\corrref{smath-0221}
\end{exercice}
