% This is part of Un soupçon de mathématique sans être agressif pour autant
% Copyright (c) 2015
%   Laurent Claessens
% See the file fdl-1.3.txt for copying conditions.

\begin{exercice}\label{exo2smath-0047}

    Donner les distances entre les points suivants :
    \begin{enumerate}
        \item
            \( A\) d'abscisse \( 4\) et \( B\) d'abscisse \( 7\)
        \item
            \( C \) d'abscisse \( 4.7\) et \( D\) d'abscisse \( 4.9\)
        \item
            \( E \) d'abscisse \( 5\) et \( F\) d'abscisse \( 2.5\)
        \item
            \( G \) d'abscisse \( 2\) et \( H\) d'abscisse \( \opp(3)\)
        \item
            \( K \) d'abscisse \( 2.4\) et \( L\) d'abscisse \( \opp(10)\)
        \item
            \( M \) d'abscisse \( \opp(7)\) et \( N\) d'abscisse \( \opp(4.5)\)
    \end{enumerate}

\corrref{2smath-0047}
\end{exercice}
