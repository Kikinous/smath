% This is part of Un soupçon de mathématique sans être agressif pour autant
% Copyright (c) 2014
%   Laurent Claessens
% See the file fdl-1.3.txt for copying conditions.

\begin{corrige}{smath-0635}

    \begin{enumerate}
        \item
            Si l'utilisateur rentre \( x=12\), le programme ne fait rien parce que \( x=12\) ne vérifier aucune des deux conditions \( x<10\) et \( x>15\).
        \item
            En posant \( x=1\) dans l'expression proposée on obtient \( (1-7)^2=(-6)^2=36\). Donc c'est la possibilité \ref{ItemPMxYdYQ} qui est correcte.
        \item
            La fraction 
            \begin{equation}
                \frac{ 6a^2 }{ 8a^2+4a }
            \end{equation}
            peut être simplifiée par \( 2\) et par \( a\) parce que tous les termes présents dans la fraction sont divisibles par \( 2\) et par \( a\).
        \item
            Seules les fonctions affines sont représentées par des droites. Ici les fonctions \( f\) et \( g\) ne sont pas affines (à cause des \( x^2\)), donc les représentations graphiques de \( f\) et \( g\) ne sont pas des droites.
    \end{enumerate}

\end{corrige}
