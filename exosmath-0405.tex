% This is part of Un soupçon de mathématique sans être agressif pour autant
% Copyright (c) 2013
%   Laurent Claessens
% See the file fdl-1.3.txt for copying conditions.

\begin{exercice}\label{exosmath-0405}

    Le dessin ci-contre montre le graphe de la fonction \( f(x)=-\frac{ 1 }{ 2 }x^2+3\). Tracer la tangente à ce graphe au point d'abscisse \( x=3\).  Justifier le dessin par un calcul de dérivée.


FLnDVHh

\corrref{smath-0405}
\end{exercice}
