% This is part of Un soupçon de mathématique sans être agressif pour autant
% Copyright (c) 2014
%   Laurent Claessens
% See the file fdl-1.3.txt for copying conditions.

\begin{corrige}{smath-0956}

    Le côté le plus long est \( [PR]\). Donc si le triangle est rectangle, il doit être rectangle en \( Q\). D'après le théorème de Pythagore, \( QRP\) sera rectangle si 
    \begin{equation}
        PR^2=QP^2+QR^2,
    \end{equation}
    et ne sera pas rectangle sinon. Nous calculons :
    \begin{equation}
        PR^2=15^2=225
    \end{equation}
    et
    \begin{equation}
        QP^2+QR^2=8^2+12^2=208.
    \end{equation}
    La réciproque du théorème de Pythagore nous indique alors que le triangle \( QPR\) n'est pas rectangle.


\end{corrige}
