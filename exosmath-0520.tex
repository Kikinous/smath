% This is part of Un soupçon de mathématique sans être agressif pour autant
% Copyright (c) 2013
%   Laurent Claessens
% See the file fdl-1.3.txt for copying conditions.

\begin{exercice}\label{exosmath-0520}

\begin{wrapfigure}{r}{8.0cm}
   \vspace{-2cm}        % à adapter.
   \centering
   \input{Fig_XPqLRCF.pstricks}
\end{wrapfigure}

À partir du dessin ci-contre,
\begin{enumerate}
    \item
        lire les valeurs \( f(-1)\), \( f(0)\) et \( f(\frac{ 5 }{2})\);
    \item
        donner le coefficient directeur et l'ordonnée à l'origine des deux droites représentées;
    \item
       sur le même graphique, dessiner le graphe de la fonction affine \( h\colon x\mapsto -3x+1\).
\end{enumerate}

\corrref{smath-0520}
\end{exercice}
