% This is part of Un soupçon de mathématique sans être agressif pour autant
% Copyright (c) 2014
%   Laurent Claessens
% See the file fdl-1.3.txt for copying conditions.

\begin{exercice}\label{exosmath-0694}

    Pour cet exercice, faire un dessin qui sera complété au fur et à mesure.

    Soient les points \( A(0;3)\) et \( B(1;1)\). Nous considérons la droite \( d_1\) d'équation \( y=2x+3\).
    \begin{multicols}{2}
    \begin{enumerate}
        \item
            Vérifier que le point \( A\) appartient à la droite \( d_1\).
        \item
            Donner l'équation de la droite parallèle à \( d_1\) passant par \( B\). Nous nommons \( d_2\) cette droite.
        \item
            Tracer la droite \( d_2\).
        \item
            Donner les coordonnées du vecteur \( \vect{ AB }\).
        \item
            Donner les coordonnées du point \( D\) d'abscisse \( 2\) sur la droite \( d_1\).
        \item
            Donner les coordonnées du point \( C\) tel que \( \vect{ AB }=\vect{ DC }\).
        \item
            Est-ce que \( ABCD\) est un parallélogramme ? Un rectangle ? Un carré ?
    \end{enumerate}
    \end{multicols}

\corrref{smath-0694}
\end{exercice}
