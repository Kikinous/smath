% This is part of Un soupçon de mathématique sans être agressif pour autant
% Copyright (c) 2015
%   Laurent Claessens
% See the file fdl-1.3.txt for copying conditions.

\begin{corrige}{2smath-0152}

    Le dessin :
\begin{center}
   \input{Fig_UMQMooOlJYbZ.pstricks}
\end{center}
Le coefficient de proportionnalité entre le petit et le grand triangle rectangle est de 
\begin{equation}
    \frac{ 600 }{ 25 }=24.
\end{equation}
D'autre part le théorème de Pythagore nous permet de trouver la longueur entre le départ et \( A\) qui vaut \SI{800}{\meter}. Donc la distance entre le point de départ et \( P\) est
\begin{equation}
    \frac{ 800 }{ 25 }\simeq \SI{33.3}{\meter}.
\end{equation}

\end{corrige}
