% This is part of Un soupçon de mathématique sans être agressif pour autant
% Copyright (c) 2014
%   Laurent Claessens
% See the file fdl-1.3.txt for copying conditions.

\begin{corrige}{smath-0690}

    \begin{enumerate}
        \item
            Nous calculons \( \vect{ AC }=\begin{pmatrix}
                6    \\ 
                4    
            \end{pmatrix}\), donc il faut placer \( M\) de telle sorte que \( \vect{ MC }=\begin{pmatrix}
                2    \\ 
                4/3    
            \end{pmatrix}\).
            Si nous notons \( (x_M;y_M)\) les coordonnées de \( M\), nous avons les équations
            \begin{subequations}
                \begin{align}
                    x_C-x_M=2\\
                    y_C-y_M=\frac{ 4 }{ 3 }
                \end{align}
            \end{subequations}
            ou encore \( 5-x_M=2\) et \( 5-y_M=\frac{ 4 }{ 3 }\). Nous trouvons
            \begin{equation}
                M(3;\frac{ 11 }{ 3 }).
            \end{equation}
        \item
            Le point \( D\) est le translaté du point \( A\) par le vecteur \( \vect{ BC }=\begin{pmatrix}
                0    \\ 
                2    
            \end{pmatrix}\), c'est à dire \( D=(-1;3)\).
        \item
            Le milieu de \( [CD]\) se calcule avec la formule du milieu :
            \begin{equation}
                I=\big( \frac{ x_C+x_D }{2};\frac{ y_C+y_D }{2} \big)=(2;4).
            \end{equation}
        \item
            Pour prouver que \( I\), \( B\) et \( C\) sont alignés, nous calculons les vecteurs \( \vect{ IB }\) et \( \vect{ IM }\) :
            \begin{subequations}
                \begin{align}
                    \vect{ IB }=\begin{pmatrix}
                        3    \\ 
                        -1    
                    \end{pmatrix}\\
                    \vect{ IM }=\begin{pmatrix}
                        1    \\ 
                        -1/3    
                    \end{pmatrix}.
                \end{align}
            \end{subequations}
            Nous voyons que \( \vect{ IM }=\frac{1}{ 3 }\vect{ IB }\), donc les points sont alignés.
    \end{enumerate}
\end{corrige}
