% This is part of Un soupçon de mathématique sans être agressif pour autant
% Copyright (c) 2013
%   Laurent Claessens, Pauline Klein
% See the file fdl-1.3.txt for copying conditions.

\chapter{Systèmes linéaires}

On s'intéresse à des systèmes linéaires de deux équations à deux
inconnues, c'est-à-dire de la forme
    \begin{subequations}
        \begin{numcases}{}
            ax+by=c\\
a'x+b'y=c'
        \end{numcases}
    \end{subequations}
où $a$, $b$, $c$, $a'$, $b'$, $c'$ sont des coefficients réels. Les
solutions sont des couples $(x;y)$.

%+++++++++++++++++++++++++++++++++++++++++++++++++++++++++++++++++++++++++++++++++++++++++++++++++++++++++++++++++++++++++++ 
\section{Méthode par substitution}
%+++++++++++++++++++++++++++++++++++++++++++++++++++++++++++++++++++++++++++++++++++++++++++++++++++++++++++++++++++++++++++


À l'aide de la première équation, on exprime l'une des
  variables en fonction de l'autre, puis on remplace dans la deuxième
  équation par l'expression obtenue.


  \begin{example}
Résoudre par substitution le système
    \begin{subequations}
        \begin{numcases}{}
            x+2y=4\\
-3x+4y=18
        \end{numcases}
    \end{subequations}

  \medskip

  \begin{enumerate}
  \item La première équation permet d'exprimer facilement $x$ en
    fonction de $y$ : \\ $x=4-2y$.

  \item Dans la deuxième équation, on remplace $x$ par $4-2y$. On
    obtient une équation qui ne dépend que de $y$, qu'on résout.
    \begin{gather*}
      -3(4-2y) + 4y = 18 \\
      -12 + 6y + 4 y = 18 \\
      10 y =30 \\
      y = 3
    \end{gather*}

  \item Une fois trouvé $y$, on reprend l'expression de $x$ en
    fonction de $y$, et on remplace $y$ par la valeur trouvée :
    \[
    x = 4-2y = 4-2\times 3 = 4-6=-2
    \]

  \item On donne l'ensemble solution, toujours dans l'ordre $(x;y)$.
    \[
    \boxed{ \mathscr{S} = \left\{ (-2;3) \right\} }
    \]
  \end{enumerate}
  
      
  \end{example}
  


  %+++++++++++++++++++++++++++++++++++++++++++++++++++++++++++++++++++++++++++++++++++++++++++++++++++++++++++++++++++++++++++ 
  \section{Méthode par addition, ou par combinaisons linéaires}
  %+++++++++++++++++++++++++++++++++++++++++++++++++++++++++++++++++++++++++++++++++++++++++++++++++++++++++++++++++++++++++++


On multiplie l'une des équations (ou éventuellement
  les deux) par un facteur, positif ou négatif, de manière à éliminer
  l'une des deux variables lorsqu'on fait la somme des deux équations.


  \begin{example}
Résoudre
\begin{subequations}
    \begin{numcases}{}
        x+2y=4\\
        -3x+4y=18
    \end{numcases}
\end{subequations}


  \begin{enumerate}
  \item On cherche à éliminer $y$. Pour cela, il faut faire en sorte
    que le coefficient devant $y$ dans la première équation soit égal
    à $-4$. On doit donc multiplier la première équation par $-2$.
    
    \underline{\textbf{Attention}} : Il faut multiplier
    \underline{\textbf{tous}} les coefficients de l'équation par $-2$.
    \medskip

  \item Le système est équivalent à :
    \begin{subequations}
        \begin{numcases}{}
            -2x-4y=-8\\
-3x+4y=18
        \end{numcases}
    \end{subequations}

  \item On fait la somme, membre à membre, des deux équations. On
    obtient une équation dans laquelle n'intervient plus que $x$,
    qu'on résout.
    \begin{equation}
        -5x=10
    \end{equation}
    et don
    \begin{equation}
        x=-2
    \end{equation}
    
  \item On reprend l'une des deux équations du système d'origine pour
    exprimer $y$ en fonction de $x$ (choisir si possible une équation
    dans laquelle $y$ intervient avec le coefficient $1$, ou $-1$) et
    calculer $y$.
    \begin{gather*}
      x+2y = 4 \\
      2y  = 4-x \\
      2y = 4-(-2) \\
      2y = 6 \\
      y = 3
    \end{gather*}

  \item On donne l'ensemble solution :
    \[
    \boxed{ \mathscr{S} = \left\{ (-2;3) \right\} }
    \]
  \end{enumerate}

  \end{example}

%+++++++++++++++++++++++++++++++++++++++++++++++++++++++++++++++++++++++++++++++++++++++++++++++++++++++++++++++++++++++++++ 
\section{Exercices}
%+++++++++++++++++++++++++++++++++++++++++++++++++++++++++++++++++++++++++++++++++++++++++++++++++++++++++++++++++++++++++++

\Exo{smath-0323}
\Exo{smath-0327}
