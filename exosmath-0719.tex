% This is part of Un soupçon de mathématique sans être agressif pour autant
% Copyright (c) 2014
%   Laurent Claessens
% See the file fdl-1.3.txt for copying conditions.

\begin{exercice}\label{exosmath-0719}

    \begin{multicols}{2}

    Sarah, paysagiste doit créer un parterre de fleurs dans un jardin rectangulaire aux dimensions de \unit{7}{\meter} par \unit{12}{\meter}. Il propose de planter des fleurs dans la partie grisée du dessin suivant :

\begin{center}
   \input{Fig_JSFooDBSHFo.pstricks}
\end{center}

Le quadrilatère \( MBNP\) est un carré dont la longueur du côté est nommée \( x\); \( AD=7\) et \( AB=12\). Sarah nomme \( S\) la fonction qui à \( x\) fait correspondre l'aire du parterre.

\begin{enumerate}
    \item
        Pour quelles valeurs de \( x\) la fonction \( S\) est-elle définie ?
    \item
        Sarah demande l'aide de deux amis pour avoir une formule pour \( S\). Sam lui répond \( S(x)=2x^2-8x+30\) tandis que Jacques lui répond \( S(x)=x^2-8x+49\). Laquelle est la bonne ?
            \item
                Dresser un tableau de variations de \( S\).
            \item
                Pour quelle valeur de \( x\) le parterre est le plus petit ? Quelle est l'aire correspondante ?
            \item
                Pour quelle valeur de \( x\) le parterre est le plus grand ? Quelle est l'aire correspondante ?
    \item
        Hélas le budget «pesticide, désherbant et engrais chimique» est limité et la mairie ne peut pas s'offrir d'entretenir plus de \unit{37}{\squared\meter} de parterre fleuri. Sarah doit donc résoudre \( S(x)\leq 37\).
        \begin{enumerate}
            \item
                Montrer que l'équation \( S(x)\leq 37\) est équivalente à \( (x-2)(x-6)\leq 0\).
            \item
                Dresser le tableau de signe de l'expression \( (x-2)(x-6)\).
            \item
                Conclure : quelles valeurs de \( x\) Sarah peut-elle choisir pour que l'aire du parterre n'excède pas \unit{37}{\squared\meter} ?
        \end{enumerate}
\end{enumerate}
    \end{multicols}


\corrref{smath-0719}
\end{exercice}
