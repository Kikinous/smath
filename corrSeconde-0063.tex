% This is part of Un soupçon de mathématique sans être agressif pour autant
% Copyright (c) 2012
%   Laurent Claessens
% See the file fdl-1.3.txt for copying conditions.

\begin{corrige}{Seconde-0063}

Le dessin est sur la figure \ref{LabelFigCorSSTXPQVjn}. % From file CorSSTXPQVjn
\newcommand{\CaptionFigCorSSTXPQVjn}{Le triangle équilatéral de l'exercice \ref{exoSeconde-0063}.}
\input{Fig_CorSSTXPQVjn.pstricks}

Le segment \( [BH]\) est la moitié du segment \( [BC]\) et est donc de longueur \( \frac{ x }{2}\). Le triangle étant équilatéral, le côté \( [AB]\) a la même longueur que le côté \( [BC]\) et donc \( AB=x\).

Le triangle \( ABH\) est rectangle en \( H\); le théorème de Pythagore s'écrit donc
\begin{equation}
    x^2+\left( \frac{ x }{2} \right)^2=AH^2.
\end{equation}

La surface d'un triangle est donnée par la formule
\begin{equation}
    \text{surface}=\frac{ \text{base}\times\text{hauteur} }{ 2 }.
\end{equation}
Ici la base vaut \( x\) et la hauteur est \( AH\) qui vaut \( \sqrt{x^2+\left( \frac{ x }{2} \right)^2}\). La formule de l'aire du triangle donne
\begin{subequations}
    \begin{align}
        S(x)&=\frac{ 1 }{2}x\sqrt{x^2+\left( \frac{ x }{2} \right)^2}\\
        &=\frac{ x }{2}\sqrt{\frac{ 5x^2 }{ 4 }}\\
        &=\frac{ \sqrt{5} }{ 2 }x^2.
    \end{align}
\end{subequations}
La résolution de \( S(x)=\sqrt{5}\) donne \( x=\pm\sqrt{2}\). Mais attention : ici \( x\) doit être positif parce que c'est une longueur. Donc l'unique solution acceptable est \( x=\sqrt{2}\).

\end{corrige}
