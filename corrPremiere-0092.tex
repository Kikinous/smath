% This is part of Un soupçon de mathématique sans être agressif pour autant
% Copyright (c) 2012
%   Laurent Claessens
% See the file fdl-1.3.txt for copying conditions.

\begin{corrige}{Premiere-0092}

    \begin{enumerate}
        \item
            Chaque question est une expérience aléatoire ayant une probabilité \( \frac{1}{ 4 }\) de succès. Les différentes questions étant indépendantes, le nombre total de succès après \( 5\) tels expériences suit une loi binomiale \( B(5,0.25)\).
        \item
            Le programme suivant donne les probabilités \( P(X<k)\) pour \( k\) entre \( 0\) et \( 5\) :
\lstinputlisting{ex_binomiale2.py}
Le résultat est :
\lstinputlisting[title=Résultat]{res_ex_binomiale2.txt}
            Nous voyons que la probabilité d'obtenir zéro réponses correctes est \( 0.23\). La probabilité d'en obtenir au maximum une est de \( 0.63\). Cela signifie qu'avec une probabilité \( 0.63\), un élève ne peut pas obtenir deux réponses correctes.

            Enfin, la ligne qui nous intéresse est celle qui dit que \( P(X\leq 3)\approx 0.98\). C'est à dire qu'avec une probabilité \( 0.98\) l'élève ne pourra pas dépasser les \( 3/5\).

        \item
\lstinputlisting{ex_binomiale3.py}
donne
\lstinputlisting[title=Résultat]{res_ex_binomiale3.txt}

        Étudions les lignes correspondantes au questionnaire à \( 6\) questions. Ce que nous voulons est d'avoir une probabilité \( 0.95\) que l'élève réussisse deux questions ou moins. Or avec le questionnaire à \( 6\) questions, nous voyons qu'il n'y a que environ \( 83\%\) de chances de faire \( 2/6\) ou moins. Pour le questionnaire à \( 7\) questions, le professeur veut une probabilité \( 0.95\) que l'élève ne réussisse que au maximum \( 3\) questions; mais nous voyons que cette probabilité n'est que de $0.92$.

        Nous continuons ainsi jusqu'au questionnaire à \( 9\) questions. Pour celui-là nous voyons que \( P(X\leq 4)\approx 0.951\). Cela nous convient.

    \end{enumerate}
    Conclusion : si un QCM avec \( 4\) choix par questions fait \( 9\) questions ou plus, alors en tapant au hasard, il n'y a pas \( 5\%\) de réussir.

\end{corrige}
