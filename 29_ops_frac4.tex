% This is part of Un soupçon de mathématique sans être agressif pour autant
% Copyright (c) 2014
%   Laurent Claessens
% See the file fdl-1.3.txt for copying conditions.

% This is part of Un soupçon de mathématique sans être agressif pour autant
% Copyright (c) 2014
%   Laurent Claessens
% See the file fdl-1.3.txt for copying conditions.

%--------------------------------------------------------------------------------------------------------------------------- 
\subsection*{Activité : somme de fractions}
%---------------------------------------------------------------------------------------------------------------------------

Compléter les pointillés
\begin{center}
   \input{Fig_UKRHooEvocBg.pstricks}
\end{center}
\begin{enumerate}
    \item
        La partie grisée représente \( \dfrac{ 2 }{ \ldots }\) de l'aire totale.
    \item
        La partie hachurée représente \( \dfrac{ 1 }{ \ldots }\) de l'aire totale.
    \item
        En s'aidant du dessin, calculer la somme
        \begin{equation}
            \frac{ 2 }{ 3 }+\frac{1}{ 4 }=\ldots
        \end{equation}
\end{enumerate}

De \cite{NRHooXFvgpp4}

%+++++++++++++++++++++++++++++++++++++++++++++++++++++++++++++++++++++++++++++++++++++++++++++++++++++++++++++++++++++++++++ 
\section{Addition et soustraction}
%+++++++++++++++++++++++++++++++++++++++++++++++++++++++++++++++++++++++++++++++++++++++++++++++++++++++++++++++++++++++++++

\begin{Aretenir}
    Pour additionner (ou soustraire) des nombres en écriture fractionnaire :
    \begin{itemize}
        \item 
            on écrit les nombres avec le même dénominateur ;
        \item 
            on additionne (ou on soustrait) les numérateurs et on garde le dénominateur commun.
    \end{itemize}
\end{Aretenir}

\begin{example}
    Parfois le dénominateur commun n'est aucun des deux dénominateurs :
    \begin{equation}
        B=\frac{5}{ 6 }-\frac{ 3 }{ 14 }.
    \end{equation}
    Le plus petit dénominateur commun est \( 42\) : c'est le plus petit nombre à être multiple en même temps de \( 6\) et \( 14\). En choisissant cela,
    \begin{equation}
        B=\frac{ 5\times 7 }{ 6\times 7 }-\frac{ 3\times 3 }{ 14\times 3 }=\frac{ 35 }{ 42 }-\frac{ 9 }{ 42 }=\frac{ 26 }{ 42 }=\frac{ 13 }{ 21 }.
    \end{equation}
    
    Dans ce cas, il est peut-être plus simple de choisir le produit \( 14\times 6\) comme dénominateur commun (ça marche toujours) :
    \begin{equation}
        B=\frac{ 5\times 14 }{ 6\times 14 }-\frac{ 3\times 6 }{ 14\times 6 }=\frac{ 70 }{ 84 }-\frac{ 18 }{ 84 }=\frac{ 52 }{ 84 }=\frac{ 13 }{ 21 }.
    \end{equation}
    Prendre le produit des dénominateurs fonctionne toujours, mais mène à des calculs sur de plus grands nombres.    
\end{example}

%+++++++++++++++++++++++++++++++++++++++++++++++++++++++++++++++++++++++++++++++++++++++++++++++++++++++++++++++++++++++++++ 
\section{Multiplication}
%+++++++++++++++++++++++++++++++++++++++++++++++++++++++++++++++++++++++++++++++++++++++++++++++++++++++++++++++++++++++++++

% This is part of Un soupçon de mathématique sans être agressif pour autant
% Copyright (c) 2014
%   Laurent Claessens
% See the file fdl-1.3.txt for copying conditions.

%--------------------------------------------------------------------------------------------------------------------------- 
\subsection*{Activité : partie d'un paquet de pâtes}
%---------------------------------------------------------------------------------------------------------------------------

Fabrice veut cuisiner les trois quarts d'un paquet d'un demi-kilo de pâtes.
\begin{enumerate}
    \item
        Combien de grammes de pâtes cela fait ?
    \item
        Compléter les fractions :
        \begin{equation}
            \frac{ 3 }{ 4 }\times \frac{1}{ 2 }\si{\kilo\gram}=\frac{ \ldots }{ \ldots }\si{\kilo\gram}=\frac{ \ldots }{ \ldots }\times\ldots\si{\gram}=\ldots\si{\gram}
        \end{equation}
    \item
        Vérifier que les deux résultats sont identiques.
\end{enumerate}


\begin{Aretenir}
Pour multiplier des nombres en écriture fractionnaire, on multiplie les numérateurs entre eux et les dénominateurs entre eux.
\end{Aretenir}

\begin{example}
    Calculer : \( A=\dfrac{ 8 }{ 7 }\times \dfrac{ 5 }{ 3 }\). On fait :
    \begin{subequations}
        \begin{align}
            A&=\frac{ 8 }{ 7 }\times \frac{ 5 }{ 3 }\\
            &=\frac{ 8\times 5 }{ 7\times 3 }\\
            &=\frac{ 40 }{ 21 }.
        \end{align}
    \end{subequations}
\end{example}

%+++++++++++++++++++++++++++++++++++++++++++++++++++++++++++++++++++++++++++++++++++++++++++++++++++++++++++++++++++++++++++ 
\section{Division}
%+++++++++++++++++++++++++++++++++++++++++++++++++++++++++++++++++++++++++++++++++++++++++++++++++++++++++++++++++++++++++++

% This is part of Un soupçon de mathématique sans être agressif pour autant
% Copyright (c) 2014-2015
%   Laurent Claessens
% See the file fdl-1.3.txt for copying conditions.

%--------------------------------------------------------------------------------------------------------------------------- 
\subsection*{Activité : division par une fraction}
%---------------------------------------------------------------------------------------------------------------------------

\begin{enumerate}
    \item
        Il faut partager \( 45\) œufs en chocolat en \( 5\) personnes. Combien d'œufs par personne ? Écrire le calcul effectué sous forme de fraction.
    \item
        Nous avons \( 45\) œufs en chocolat avec lesquels il faut remplir des sachets. Chaque sachet peut contenir \( 5\) œufs. Combien de sachets faut-il ? Écrire le calcul effectué sous forme de fraction.
    \item
        Nous avons \SI{15}{\liter} d'eau pour remplir des bouteilles de $\SI{\dfrac{ 1 }{ 2 }}{\liter}$. Combien de bouteilles faudra-t-il ?
    \item
        Combien de fois \( 5\) rentre-t-il dans \( 15\) ?
    \item
        Combien de fois \( 1/6\) rentre-t-il dans \( 2/3\) ?

\begin{center}
   \input{Fig_MXHCooEnhHYeZ.pstricks}
\end{center}
\begin{center}
   \input{Fig_MXHCooEnhHYeO.pstricks}
\end{center}



    \item
        Combien peut-on faire de huitièmes de pizzas à partir de trois quarts de pizza ? Pour vous aider, voici un dessin de trois quart de pizzas et un dessin d'une pizza coupée en \( 8\) :
        \begin{center}
           \input{Fig_PJUIooZuEnPkZ.pstricks}
           \input{Fig_PJUIooZuEnPkO.pstricks}
        \end{center}
    \item
        Calculer
        \begin{equation}
            \dfrac{ 1 }{ 2 }\div\frac{1}{ 8 }.
        \end{equation}
\end{enumerate}


