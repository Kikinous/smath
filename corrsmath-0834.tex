% This is part of Un soupçon de mathématique sans être agressif pour autant
% Copyright (c) 2014
%   Laurent Claessens
% See the file fdl-1.3.txt for copying conditions.

\begin{corrige}{smath-0834}

    Non : la séquence n'est pas bonne, entre autres parce que le signe de division ne fait que \( \dfrac{ 2 }{ 6 }\) au lieu de diviser tout le numérateur \( 4+5\times 2\) par le dénominateur \( 6+1\).

    En ce qui concerne les parenthèses, il faut faire ceci :
    \begin{equation}
        \input{TIHooZFjIJl.calcul}
    \end{equation}

    Note : il ne faut pas mettre \( 5\times 2\) entre parenthèses parce que la multiplication est prioritaire.

\end{corrige}
