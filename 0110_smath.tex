% This is part of Un soupçon de mathématique sans être agressif pour autant
% Copyright (c) 2012-2013
%   Laurent Claessens
% See the file fdl-1.3.txt for copying conditions.


%+++++++++++++++++++++++++++++++++++++++++++++++++++++++++++++++++++++++++++++++++++++++++++++++++++++++++++++++++++++++++++ 
\section{Parallélisme et colinéarité}
%+++++++++++++++++++++++++++++++++++++++++++++++++++++++++++++++++++++++++++++++++++++++++++++++++++++++++++++++++++++++++++

\begin{definition}
    Deux vecteurs \( \vect{ u }\) et \( \vect{ v }\) sont \defe{colinéaires}{colinéaire (vecteurs)} si il existe \( \lambda\in \eR\) tel que \( \vect{ u }=\lambda\vect{ v }\).
\end{definition}

\begin{propriete}
    \begin{enumerate}
        \item
            Les droites \( (AB)\) et \( (CD)\) sont parallèles si et seulement si les vecteurs \( \vect{ AB }\) et \( \vect{ CD }\) sont colinéaires.
        \item
            Les points \( A\), \( B\) et \( C\) sont alignés si et seulement si les vecteurs \( \vect{ AB }\) et \( \vect{ AC }\) sont colinéaires.
    \end{enumerate}
\end{propriete}

Nous savons qu'un quadrilatère ayant deux côtés parallèles de même longueur est un parallélogramme. Donc nous avons le critère suivant pour savoir si \( ABCD\) est un parallélogramme :
\begin{equation}
    \vect{ AB }=\vect{ CD }.
\end{equation}
Bien entendu les autres côtés fonctionnent aussi :
\begin{equation}
    \vect{ AC }=\vect{ BD }.
\end{equation}
Si une de ces deux égalités vectorielle est satisfaite, alors \( ABCD\) est un parallélogramme.

\begin{example}
    Les points \( A=(-2;-3)\), \( B=(-1,-1)\), \( C=(2;-2)\) et \( D=(1;-4)\) forment un parallélogramme.
\end{example}

%--------------------------------------------------------------------------------------------------------------------------- 
\subsection{Le milieu revisité}
%---------------------------------------------------------------------------------------------------------------------------

Notre travail sur les coordonnées de vecteurs nous permet de donner une preuve alternative à la propriété \ref{PropFHznUfJ}.

\begin{propriete}
    Soient les points \( A\), \( B\) et \( K\) tels que \( K\) soit le milieu du segment \( [AB]\). Alors nous avons \( \vect{ AK }=\vect{ BK }\).
\end{propriete}

\begin{proof}
    Nous divisons la preuve en petits pas.
    \begin{subproof}
        \item[Création du repère]
            Nous considérons un repère orthonormé \footnote{En réalité il n'est pas obligatoire qu'il soit orthonormé, mais ça ne coûte rien qu'il le soit.} dont \( A\) est l'origine.
        \item[Coordonnées des points]
            Dans le repère choisi, nous considérons les coordonnées des points \( K=(x_K;y_K)\) et \( B=(x_B,y_B)\). Nous allons aussi nommer \( I\) et \( J\) les points \( I=(x_K;0)\) et \( J=(x_B;0)\). Le dessin est maintenant comme suit :

            \begin{center}
%The result is on figure \ref{LabelFigfigureSZyxsvp}. % From file figureSZyxsvp
%\newcommand{\CaptionFigfigureSZyxsvp}{<+Type your caption here+>}
\input{Fig_figureSZyxsvp.pstricks}
            \end{center}
        \item[Utilisation du théorème de Thalès]
            Vu que les droites \( (KI)\) et \( (BJ)\) sont parallèles, nous pouvons utiliser le théorème de Thalès :
            \begin{equation}
                \frac{ AB }{ AK }=\frac{ AJ }{ AI }=\frac{ BJ }{ KI }.
            \end{equation}
            Nous remplaçons dans ces égalités les longueurs par ce qu'on connait. Vu que \( K\) est le milieu de \( [AB] \), nous avons \( \frac{ AB }{ AK }=2\). D'autre part, $AJ=x_B$, \( AI=x_K\), \( BJ=y_B\) et \( KI=y_K\), donc
            \begin{equation}
                2=\frac{ x_B }{ x_K }
            \end{equation}
            et
            \begin{equation}
                2=\frac{ y_B }{ y_K }.
            \end{equation}
            Autrement dit,
            \begin{equation}
                x_B=2x_K
            \end{equation}
            et
            \begin{equation}
                y_B=2y_K.
            \end{equation}
        \item[Les vecteurs en présence]
            Les vecteurs \( \vect{ AB }\) et \( \vect{ AK }\) ont pour coordonnées
            \begin{subequations}
                \begin{align}
                    \vect{ AB }=\begin{pmatrix}
                        x_B    \\ 
                        y_B    
                    \end{pmatrix}\\
                    \vect{ AK }=\begin{pmatrix}
                        x_K    \\ 
                        y_K    
                    \end{pmatrix},
                \end{align}
            \end{subequations}
            Donc, étant donné que \( x_B=2x_K\) et \( y_B=2y_K\) nous avons
            \begin{equation}    \label{EqNGxKxaY}
                \vect{ AB }=2\vect{ AK }.
            \end{equation}
        \item[Utilisation de la loi de Chasles et conclusion]
            Nous savons que \( \vect{ AB }=\vect{ AK }+\vect{ KB }\), donc en replaçant \( \vect{ AB }\) par \( 2\vect{ AK }\) nous avons
            \begin{equation}
                2\vect{ AK }=\vect{ AK }+\vect{ KB },
            \end{equation}
            ce qui implique que
            \begin{equation}
                \vect{ AK }=\vect{ KB },
            \end{equation}
            ce qu'il fallait.
    \end{subproof}
\end{proof}
Nous notons aussi au passage l'intéressante formule \eqref{EqNGxKxaY} :
\begin{equation}
    \vect{ AB }=2\vect{ AK }.
\end{equation}

