% This is part of Un soupçon de mathématique sans être agressif pour autant
% Copyright (c) 2014
%   Laurent Claessens
% See the file fdl-1.3.txt for copying conditions.

\begin{exercice}\label{exosmath-0888}

    L'école de Poudlard organise \( 7\) niveaux (de la sixième à la terminale), et contient \( 4\) classes de chaque niveau. 
    
    \begin{enumerate}
        \item
            Cette école contient \( 250\) filles et \( 200\) garçons. Quelle est la proportion de filles dans l'école ? Donner la réponse sous forme de fraction.
        \item
            
    Quel(s) diagramme(s) exprime(nt) la proportion de classes de troisième à Poudlard ?

\begin{center}
   \input{Fig_KUUooRcCjkQooZERO.pstricks}
   \input{Fig_KUUooRcCjkQooONE.pstricks}
   \input{Fig_KUUooRcCjkQooTWO.pstricks}
\end{center}
Justifier vos choix. (le second diagramme est divisé en \( 28\) secteurs)
\item
    Parmi les classes de sixième à terminale, lesquelles sont celles du collège ? Laquelle des fractions suivantes donne la proportion de classes de collège à Poudlard ?  \( \dfrac{ 3 }{ 7 }\), \( \dfrac{ 4 }{ 7 }\), \( \dfrac{ 2 }{ 6 }\), \( \dfrac{ 1 }{ 2 }\) Justifier.

\item
    Donner une approximation numérique de cette proportion.
    \end{enumerate}

\corrref{smath-0888}
\end{exercice}
