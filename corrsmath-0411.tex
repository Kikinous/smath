% This is part of Un soupçon de mathématique sans être agressif pour autant
% Copyright (c) 2013
%   Laurent Claessens
% See the file fdl-1.3.txt for copying conditions.

\begin{corrige}{smath-0411}

    Choisissons un repère orthonormé dans lequel le point \( A\) se situe en \( (0;0)\). L'équation de son tir de plasma est
    \begin{equation}
        y=x
    \end{equation}
    parce que d'une part le dessin montre que le coefficient directeur est \( 1\) et que la fonction doit être affine vu que le graphe passe par l'origine.

    L'autre rayon de plasma est une droite qui passe par le point \( B=(4;0)\) et de coefficient directeur \( -3\). L'équation de la droite doit être de la forme
    \begin{equation}
        y=-3x+b
    \end{equation}
    et pour fixer le \( b\), on impose que le point \( (4;0)\) soit sur la droite :
    \begin{equation}
        0=-3\times 4+b,
    \end{equation}
    c'est à dire \( b=12\). Le rayon de plasma du joueur \( B\) a donc pour équation
    \begin{equation}
        y=-3x+12.
    \end{equation}
    
    Le point d'intersection de ces deux droites est donné par la résolution du système
    \begin{subequations}
        \begin{numcases}{}
            y=x\\
            y=-3x+12.
        \end{numcases}
    \end{subequations}
    L'abscisse du point d'intersection est alors donné par l'équation \( x=-3x+12\), c'est à dire \( x=3\). Le point d'intersection est donc le point d'abscisse \( 3\) sur l'une ou l'autre des deux droites, à savoir \( (3;3)\).

    L'exercice consiste maintenant à savoir qui des joueurs \( A\) et \( B\) sont dans le cercle de centre \( (3;3)\) et de rayon \( 4\). 

    La distance entre \( A=(0;0)\) et le centre \( (3;3)\) est de \( \sqrt{3^2+3^2}=\sqrt{18}\) qui est grand que \( 4\). Donc le joueur \( A\) survit.

    Par contre la distance entre \( B=(4;0)\) et le centre de L'explosion est \( \sqrt{1^2+3^2}=\sqrt{10}\) qui est plus petit que \( 4\). Donc le joueur \( B\) meurt.

\end{corrige}
