% This is part of Un soupçon de mathématique sans être agressif pour autant
% Copyright (c) 2014
%   Laurent Claessens
% See the file fdl-1.3.txt for copying conditions.

\begin{exercice}\label{exosmath-0841}

    Quelque questions à propos de division, de fractions et de nombres décimaux. Rappel : un nombre est \defe{décimal}{décimal} lorsque son écriture s'arrête. Il faut que la suite des chiffres derrière la virgule s'arrête.
    \begin{enumerate}
        \item
            Poser la division \( 25\div 33\) et calculer \( 7\) chiffres derrière la virgule. Est-ce que le nombre \( \dfrac{ 25 }{ 33 }\) est décimal ?
        \item
            Poser la division \( 17\div 50\) et l'effectuer. Est-ce que le nombre \( \dfrac{ 17 }{ 50 }\) est décimal.
        \item
            Quel est le \( 34\)\ieme\ chiffre derrière la virgule dans le nombre \( \dfrac{ 123 }{ 999 }\) ?
    \end{enumerate}

\corrref{smath-0841}
\end{exercice}
