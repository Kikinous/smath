% This is part of Un soupçon de mathématique sans être agressif pour autant
% Copyright (c) 2012
%   Laurent Claessens
% See the file fdl-1.3.txt for copying conditions.

\begin{exercice}\label{exosmath-0138}

    Une entreprise de location de véhicules propose trois tarifs en fonction du nombre de jours de location.
    \begin{itemize}
        \item Première option : \( 150\)€ pour la première semaine puis \( 25\)€ par jour supplémentaire.
        \item
            Seconde option : \( 30\)€ par jour quelle que soit la durée.
        \item
            Troisième option : \( 50\)€ de frais de dossier et \( 25\)€ par jour.
    \end{itemize}
    \begin{enumerate}
        \item
            Évelyne propose la fonction suivante pour exprimer le prix à payer pour la première option en fonction du nombre de jours de location :
            \begin{equation}
                f_1(x)=\begin{cases}
                    150    &   \text{si \( 0\leq x\leq 7\)}\\
                    150+25x    &    \text{si \( x>7\).}
                \end{cases}
            \end{equation}
            Est-ce correct ?
        \item
            Avec la première option, combien coûte \( 3\) jours de location ? Et \( 10\) jours ?
        \item
            Écrire les fonctions \( f_2\) et \( f_3\) qui expriment le coût de location suivant les options deux et trois  en fonction du nombre de jours de location du véhicule.
        \item
            Tracer les trois fonctions dans un même repère.
        \item
            Tracer la fonction qui donne le meilleur prix en fonction du nombre de jours de location.
    \end{enumerate}

\corrref{smath-0138}
\end{exercice}
