% This is part of Un soupçon de mathématique sans être agressif pour autant
% Copyright (c) 2013,2017
%   Laurent Claessens
% See the file fdl-1.3.txt for copying conditions.

\begin{exercice}[\cite{QSfMxlk}]\label{exosmath-0384}

Nous avons représenté ci-dessous la distribution de probabilité d'une variable aléatoire suivant une loi binomiale de paramètres \( n=30\) et \( p=0.91\). 

\begin{center}
   \input{Fig_GCxOEgb.pstricks}
\end{center}

\begin{enumerate}
    \item
Déterminer l'intervalle de fluctuation bilatéral au seuil \( 95\%\).
        \item
            Un ingénieur présente une machine en disant qu'elle crée une pièce défectueuse dans \( 9\%\) des cas. Sur un échantillon de \( 30\) pièces, vous en observez \( 21\) sans défauts. Que pensez-vous de l'affirmation de l'ingénieur ?
\end{enumerate}

% Les nombres donnés pour n et p ici sont codés en dur dans le fichier de la figure.

\corrref{smath-0384}
\end{exercice}
