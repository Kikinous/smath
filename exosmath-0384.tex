% This is part of Un soupçon de mathématique sans être agressif pour autant
% Copyright (c) 2013
%   Laurent Claessens
% See the file fdl-1.3.txt for copying conditions.

\begin{exercice}[\cite{QSfMxlk}]\label{exosmath-0384}

    Nous avons représenté à la figure \ref{LabelFigGCxOEgb} la distribution de probabilité d'une variable aléatoire suivant une loi binomiale de paramètres \( n=30\) et \( p=0.91\). Déterminer l'intervalle de fluctuation bilatéral au seuil \( 95\%\).

% Les nombres donnés pour n et p ici sont codés en dur dans le fichier de la figure.

%The result is on figure \ref{LabelFigGCxOEgb}. % From file GCxOEgb
\newcommand{\CaptionFigGCxOEgb}{La distribution des probabilités pour une variable aléatoire de paramètres \( n=30\) et \( p=0.91\).}
\input{Fig_GCxOEgb.pstricks}

% TODO : il faut que l'axe Y soit gradué normalement et non avec des fractions.

\corrref{smath-0384}
\end{exercice}
