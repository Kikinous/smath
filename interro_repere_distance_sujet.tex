\vbox{1
\emph{Toutes les réponses doivent être justifiées par un calcul accompagné d'un raisonnement.}
\begin{enumerate}\item
Placer les points \( A=(6;-1)\), \( B=(7;2)\), \( C=(1;0)\) et \( D=(-10;10)\) dans un repère orthonormé. Calculer la longueur du segment \( [AB]\) et les coordonnées du milieu du segment \( [CD]\).
\item
Est-ce que le triangle formé par les points \( A(8;8)\), \( B(9;7)\) et \( C(17;7)\) est rectangle ?

\end{enumerate}
} 
\vspace{2cm}
\vbox{2
\emph{Toutes les réponses doivent être justifiées par un calcul accompagné d'un raisonnement.}
\begin{enumerate}\item
Placer les points \( A=(9;-5)\), \( B=(9;7)\), \( C=(-9;4)\) et \( D=(1;-3)\) dans un repère orthonormé. Calculer la longueur du segment \( [AB]\) et les coordonnées du milieu du segment \( [CD]\).
\item
Est-ce que le triangle formé par les points \( A(-8;7)\), \( B(-2;1)\) et \( C(-9;0)\) est isocèle ?

\end{enumerate}
} 
\vspace{2cm}
\vbox{3
\emph{Toutes les réponses doivent être justifiées par un calcul accompagné d'un raisonnement.}
\begin{enumerate}\item
Placer les points \( A=(-5;-9)\), \( B=(-3;-1)\), \( C=(8;5)\) et \( D=(-1;-8)\) dans un repère orthonormé. Calculer la longueur du segment \( [AB]\) et les coordonnées du milieu du segment \( [CD]\).
\item
Est-ce que le triangle formé par les points \( A(-7;-3)\), \( B(-3;-7)\) et \( C(1;-5)\) est rectangle ?

\end{enumerate}
} 
\vspace{2cm}
\vbox{4
\emph{Toutes les réponses doivent être justifiées par un calcul accompagné d'un raisonnement.}
\begin{enumerate}\item
Placer les points \( A=(-10;7)\), \( B=(4;3)\), \( C=(0;3)\) et \( D=(10;-1)\) dans un repère orthonormé. Calculer la longueur du segment \( [AB]\) et les coordonnées du milieu du segment \( [CD]\).
\item
Est-ce que le triangle formé par les points \( A(-8;-5)\), \( B(-3;-10)\) et \( C(3;-4)\) est rectangle ?

\end{enumerate}
} 
\vspace{2cm}
\vbox{5
\emph{Toutes les réponses doivent être justifiées par un calcul accompagné d'un raisonnement.}
\begin{enumerate}\item
Placer les points \( A=(-4;2)\), \( B=(4;7)\), \( C=(9;8)\) et \( D=(9;0)\) dans un repère orthonormé. Calculer la longueur du segment \( [AB]\) et les coordonnées du milieu du segment \( [CD]\).
\item
Est-ce que le triangle formé par les points \( A(0;-6)\), \( B(2;-8)\) et \( C(1;-7)\) est isocèle ?

\end{enumerate}
} 
\vspace{2cm}
\vbox{6
\emph{Toutes les réponses doivent être justifiées par un calcul accompagné d'un raisonnement.}
\begin{enumerate}\item
Placer les points \( A=(-10;7)\), \( B=(6;-2)\), \( C=(9;-3)\) et \( D=(7;-3)\) dans un repère orthonormé. Calculer la longueur du segment \( [AB]\) et les coordonnées du milieu du segment \( [CD]\).
\item
Est-ce que le triangle formé par les points \( A(0;3)\), \( B(4;-1)\) et \( C(13;2)\) est isocèle ?

\end{enumerate}
} 
\vspace{2cm}
\vbox{7
\emph{Toutes les réponses doivent être justifiées par un calcul accompagné d'un raisonnement.}
\begin{enumerate}\item
Placer les points \( A=(-4;1)\), \( B=(4;-6)\), \( C=(-9;-3)\) et \( D=(5;-4)\) dans un repère orthonormé. Calculer la longueur du segment \( [AB]\) et les coordonnées du milieu du segment \( [CD]\).
\item
Est-ce que le triangle formé par les points \( A(-3;-10)\), \( B(-7;-6)\) et \( C(10;-7)\) est rectangle ?

\end{enumerate}
} 
\vspace{2cm}
\vbox{8
\emph{Toutes les réponses doivent être justifiées par un calcul accompagné d'un raisonnement.}
\begin{enumerate}\item
Placer les points \( A=(9;1)\), \( B=(-6;-8)\), \( C=(-1;-7)\) et \( D=(-5;-1)\) dans un repère orthonormé. Calculer la longueur du segment \( [AB]\) et les coordonnées du milieu du segment \( [CD]\).
\item
Est-ce que le triangle formé par les points \( A(1;6)\), \( B(-5;12)\) et \( C(9;10)\) est isocèle ?

\end{enumerate}
} 
\vspace{2cm}
\vbox{9
\emph{Toutes les réponses doivent être justifiées par un calcul accompagné d'un raisonnement.}
\begin{enumerate}\item
Placer les points \( A=(-6;-4)\), \( B=(6;10)\), \( C=(0;9)\) et \( D=(-6;8)\) dans un repère orthonormé. Calculer la longueur du segment \( [AB]\) et les coordonnées du milieu du segment \( [CD]\).
\item
Est-ce que le triangle formé par les points \( A(1;3)\), \( B(3;1)\) et \( C(8;0)\) est rectangle ?

\end{enumerate}
} 
\vspace{2cm}
\vbox{10
\emph{Toutes les réponses doivent être justifiées par un calcul accompagné d'un raisonnement.}
\begin{enumerate}\item
Placer les points \( A=(10;-5)\), \( B=(-1;-4)\), \( C=(6;2)\) et \( D=(6;10)\) dans un repère orthonormé. Calculer la longueur du segment \( [AB]\) et les coordonnées du milieu du segment \( [CD]\).
\item
Est-ce que le triangle formé par les points \( A(-10;-10)\), \( B(-7;-13)\) et \( C(0;-10)\) est rectangle ?

\end{enumerate}
} 
\vspace{2cm}
\vbox{11
\emph{Toutes les réponses doivent être justifiées par un calcul accompagné d'un raisonnement.}
\begin{enumerate}\item
Placer les points \( A=(-3;-5)\), \( B=(-6;-2)\), \( C=(-7;-10)\) et \( D=(-8;0)\) dans un repère orthonormé. Calculer la longueur du segment \( [AB]\) et les coordonnées du milieu du segment \( [CD]\).
\item
Est-ce que le triangle formé par les points \( A(6;-5)\), \( B(6;-5)\) et \( C(13;-8)\) est rectangle ?

\end{enumerate}
} 
\vspace{2cm}
\vbox{12
\emph{Toutes les réponses doivent être justifiées par un calcul accompagné d'un raisonnement.}
\begin{enumerate}\item
Placer les points \( A=(2;-1)\), \( B=(-8;3)\), \( C=(5;-8)\) et \( D=(6;6)\) dans un repère orthonormé. Calculer la longueur du segment \( [AB]\) et les coordonnées du milieu du segment \( [CD]\).
\item
Est-ce que le triangle formé par les points \( A(-9;3)\), \( B(-7;1)\) et \( C(-11;1)\) est rectangle ?

\end{enumerate}
} 
\vspace{2cm}
\vbox{13
\emph{Toutes les réponses doivent être justifiées par un calcul accompagné d'un raisonnement.}
\begin{enumerate}\item
Placer les points \( A=(7;2)\), \( B=(-8;-7)\), \( C=(-6;-2)\) et \( D=(-8;-8)\) dans un repère orthonormé. Calculer la longueur du segment \( [AB]\) et les coordonnées du milieu du segment \( [CD]\).
\item
Est-ce que le triangle formé par les points \( A(-6;-1)\), \( B(-8;1)\) et \( C(4;1)\) est isocèle ?

\end{enumerate}
} 
\vspace{2cm}
\vbox{14
\emph{Toutes les réponses doivent être justifiées par un calcul accompagné d'un raisonnement.}
\begin{enumerate}\item
Placer les points \( A=(-2;-5)\), \( B=(6;9)\), \( C=(8;-9)\) et \( D=(-10;4)\) dans un repère orthonormé. Calculer la longueur du segment \( [AB]\) et les coordonnées du milieu du segment \( [CD]\).
\item
Est-ce que le triangle formé par les points \( A(-10;7)\), \( B(-6;3)\) et \( C(4;7)\) est isocèle ?

\end{enumerate}
} 
\vspace{2cm}
\vbox{15
\emph{Toutes les réponses doivent être justifiées par un calcul accompagné d'un raisonnement.}
\begin{enumerate}\item
Placer les points \( A=(8;-4)\), \( B=(-7;-5)\), \( C=(-2;-10)\) et \( D=(3;-9)\) dans un repère orthonormé. Calculer la longueur du segment \( [AB]\) et les coordonnées du milieu du segment \( [CD]\).
\item
Est-ce que le triangle formé par les points \( A(9;-10)\), \( B(5;-6)\) et \( C(12;-7)\) est rectangle ?

\end{enumerate}
} 
\vspace{2cm}
\vbox{16
\emph{Toutes les réponses doivent être justifiées par un calcul accompagné d'un raisonnement.}
\begin{enumerate}\item
Placer les points \( A=(8;-3)\), \( B=(9;-1)\), \( C=(2;0)\) et \( D=(-3;-7)\) dans un repère orthonormé. Calculer la longueur du segment \( [AB]\) et les coordonnées du milieu du segment \( [CD]\).
\item
Est-ce que le triangle formé par les points \( A(-3;-2)\), \( B(-1;-4)\) et \( C(1;0)\) est isocèle ?

\end{enumerate}
} 
\vspace{2cm}
\vbox{17
\emph{Toutes les réponses doivent être justifiées par un calcul accompagné d'un raisonnement.}
\begin{enumerate}\item
Placer les points \( A=(-7;3)\), \( B=(0;-1)\), \( C=(-6;-8)\) et \( D=(5;-4)\) dans un repère orthonormé. Calculer la longueur du segment \( [AB]\) et les coordonnées du milieu du segment \( [CD]\).
\item
Est-ce que le triangle formé par les points \( A(-4;8)\), \( B(0;4)\) et \( C(-4;8)\) est rectangle ?

\end{enumerate}
} 
\vspace{2cm}
\vbox{18
\emph{Toutes les réponses doivent être justifiées par un calcul accompagné d'un raisonnement.}
\begin{enumerate}\item
Placer les points \( A=(1;-3)\), \( B=(5;-9)\), \( C=(-2;-5)\) et \( D=(-6;-6)\) dans un repère orthonormé. Calculer la longueur du segment \( [AB]\) et les coordonnées du milieu du segment \( [CD]\).
\item
Est-ce que le triangle formé par les points \( A(4;-9)\), \( B(10;-15)\) et \( C(19;-10)\) est isocèle ?

\end{enumerate}
} 
\vspace{2cm}
\vbox{19
\emph{Toutes les réponses doivent être justifiées par un calcul accompagné d'un raisonnement.}
\begin{enumerate}\item
Placer les points \( A=(-3;-7)\), \( B=(-6;-5)\), \( C=(4;3)\) et \( D=(3;-10)\) dans un repère orthonormé. Calculer la longueur du segment \( [AB]\) et les coordonnées du milieu du segment \( [CD]\).
\item
Est-ce que le triangle formé par les points \( A(8;0)\), \( B(2;6)\) et \( C(19;7)\) est isocèle ?

\end{enumerate}
} 
\vspace{2cm}
\vbox{20
\emph{Toutes les réponses doivent être justifiées par un calcul accompagné d'un raisonnement.}
\begin{enumerate}\item
Placer les points \( A=(10;5)\), \( B=(2;0)\), \( C=(-10;-5)\) et \( D=(-1;2)\) dans un repère orthonormé. Calculer la longueur du segment \( [AB]\) et les coordonnées du milieu du segment \( [CD]\).
\item
Est-ce que le triangle formé par les points \( A(2;5)\), \( B(0;7)\) et \( C(-3;2)\) est isocèle ?

\end{enumerate}
} 
\vspace{2cm}
\vbox{21
\emph{Toutes les réponses doivent être justifiées par un calcul accompagné d'un raisonnement.}
\begin{enumerate}\item
Placer les points \( A=(4;9)\), \( B=(-5;2)\), \( C=(0;6)\) et \( D=(-6;-5)\) dans un repère orthonormé. Calculer la longueur du segment \( [AB]\) et les coordonnées du milieu du segment \( [CD]\).
\item
Est-ce que le triangle formé par les points \( A(-10;6)\), \( B(-15;11)\) et \( C(-3;3)\) est rectangle ?

\end{enumerate}
} 
\vspace{2cm}
\vbox{22
\emph{Toutes les réponses doivent être justifiées par un calcul accompagné d'un raisonnement.}
\begin{enumerate}\item
Placer les points \( A=(1;3)\), \( B=(6;0)\), \( C=(-10;-6)\) et \( D=(9;-10)\) dans un repère orthonormé. Calculer la longueur du segment \( [AB]\) et les coordonnées du milieu du segment \( [CD]\).
\item
Est-ce que le triangle formé par les points \( A(4;6)\), \( B(7;3)\) et \( C(3;5)\) est rectangle ?

\end{enumerate}
} 
\vspace{2cm}
\vbox{23
\emph{Toutes les réponses doivent être justifiées par un calcul accompagné d'un raisonnement.}
\begin{enumerate}\item
Placer les points \( A=(-6;5)\), \( B=(7;-4)\), \( C=(6;-2)\) et \( D=(7;-10)\) dans un repère orthonormé. Calculer la longueur du segment \( [AB]\) et les coordonnées du milieu du segment \( [CD]\).
\item
Est-ce que le triangle formé par les points \( A(-7;-1)\), \( B(-12;4)\) et \( C(5;1)\) est rectangle ?

\end{enumerate}
} 
\vspace{2cm}
\vbox{24
\emph{Toutes les réponses doivent être justifiées par un calcul accompagné d'un raisonnement.}
\begin{enumerate}\item
Placer les points \( A=(9;-7)\), \( B=(-3;-6)\), \( C=(-10;7)\) et \( D=(-7;9)\) dans un repère orthonormé. Calculer la longueur du segment \( [AB]\) et les coordonnées du milieu du segment \( [CD]\).
\item
Est-ce que le triangle formé par les points \( A(10;7)\), \( B(8;9)\) et \( C(10;9)\) est isocèle ?

\end{enumerate}
} 
\vspace{2cm}
\vbox{25
\emph{Toutes les réponses doivent être justifiées par un calcul accompagné d'un raisonnement.}
\begin{enumerate}\item
Placer les points \( A=(-4;-9)\), \( B=(7;1)\), \( C=(9;-9)\) et \( D=(1;3)\) dans un repère orthonormé. Calculer la longueur du segment \( [AB]\) et les coordonnées du milieu du segment \( [CD]\).
\item
Est-ce que le triangle formé par les points \( A(-2;10)\), \( B(4;4)\) et \( C(4;10)\) est isocèle ?

\end{enumerate}
} 
\vspace{2cm}
\vbox{26
\emph{Toutes les réponses doivent être justifiées par un calcul accompagné d'un raisonnement.}
\begin{enumerate}\item
Placer les points \( A=(4;-1)\), \( B=(6;5)\), \( C=(3;10)\) et \( D=(5;8)\) dans un repère orthonormé. Calculer la longueur du segment \( [AB]\) et les coordonnées du milieu du segment \( [CD]\).
\item
Est-ce que le triangle formé par les points \( A(-5;-8)\), \( B(-9;-4)\) et \( C(-1;-10)\) est isocèle ?

\end{enumerate}
} 
\vspace{2cm}
\vbox{27
\emph{Toutes les réponses doivent être justifiées par un calcul accompagné d'un raisonnement.}
\begin{enumerate}\item
Placer les points \( A=(-4;5)\), \( B=(10;-10)\), \( C=(-8;6)\) et \( D=(6;-10)\) dans un repère orthonormé. Calculer la longueur du segment \( [AB]\) et les coordonnées du milieu du segment \( [CD]\).
\item
Est-ce que le triangle formé par les points \( A(3;4)\), \( B(-1;8)\) et \( C(8;3)\) est isocèle ?

\end{enumerate}
} 
\vspace{2cm}
\vbox{28
\emph{Toutes les réponses doivent être justifiées par un calcul accompagné d'un raisonnement.}
\begin{enumerate}\item
Placer les points \( A=(6;-4)\), \( B=(6;9)\), \( C=(8;9)\) et \( D=(8;-7)\) dans un repère orthonormé. Calculer la longueur du segment \( [AB]\) et les coordonnées du milieu du segment \( [CD]\).
\item
Est-ce que le triangle formé par les points \( A(-6;10)\), \( B(-4;8)\) et \( C(-6;8)\) est isocèle ?

\end{enumerate}
} 
\vspace{2cm}
\vbox{29
\emph{Toutes les réponses doivent être justifiées par un calcul accompagné d'un raisonnement.}
\begin{enumerate}\item
Placer les points \( A=(0;0)\), \( B=(5;10)\), \( C=(9;-10)\) et \( D=(-4;-8)\) dans un repère orthonormé. Calculer la longueur du segment \( [AB]\) et les coordonnées du milieu du segment \( [CD]\).
\item
Est-ce que le triangle formé par les points \( A(0;-10)\), \( B(0;-10)\) et \( C(-3;-13)\) est isocèle ?

\end{enumerate}
} 
\vspace{2cm}
\vbox{30
\emph{Toutes les réponses doivent être justifiées par un calcul accompagné d'un raisonnement.}
\begin{enumerate}\item
Placer les points \( A=(-1;6)\), \( B=(0;10)\), \( C=(0;-8)\) et \( D=(1;-1)\) dans un repère orthonormé. Calculer la longueur du segment \( [AB]\) et les coordonnées du milieu du segment \( [CD]\).
\item
Est-ce que le triangle formé par les points \( A(-6;1)\), \( B(-2;-3)\) et \( C(-2;1)\) est isocèle ?

\end{enumerate}
} 
\vspace{2cm}
\vbox{31
\emph{Toutes les réponses doivent être justifiées par un calcul accompagné d'un raisonnement.}
\begin{enumerate}\item
Placer les points \( A=(6;-3)\), \( B=(-5;-5)\), \( C=(-3;4)\) et \( D=(8;6)\) dans un repère orthonormé. Calculer la longueur du segment \( [AB]\) et les coordonnées du milieu du segment \( [CD]\).
\item
Est-ce que le triangle formé par les points \( A(-5;-1)\), \( B(-7;1)\) et \( C(0;-4)\) est isocèle ?

\end{enumerate}
} 
\vspace{2cm}
\vbox{32
\emph{Toutes les réponses doivent être justifiées par un calcul accompagné d'un raisonnement.}
\begin{enumerate}\item
Placer les points \( A=(-8;4)\), \( B=(1;-9)\), \( C=(-8;2)\) et \( D=(7;5)\) dans un repère orthonormé. Calculer la longueur du segment \( [AB]\) et les coordonnées du milieu du segment \( [CD]\).
\item
Est-ce que le triangle formé par les points \( A(6;1)\), \( B(11;-4)\) et \( C(19;4)\) est rectangle ?

\end{enumerate}
} 
\vspace{2cm}
\vbox{33
\emph{Toutes les réponses doivent être justifiées par un calcul accompagné d'un raisonnement.}
\begin{enumerate}\item
Placer les points \( A=(-10;4)\), \( B=(0;-8)\), \( C=(5;-2)\) et \( D=(-6;-10)\) dans un repère orthonormé. Calculer la longueur du segment \( [AB]\) et les coordonnées du milieu du segment \( [CD]\).
\item
Est-ce que le triangle formé par les points \( A(-1;7)\), \( B(-2;8)\) et \( C(0;8)\) est rectangle ?

\end{enumerate}
} 
\vspace{2cm}
\vbox{34
\emph{Toutes les réponses doivent être justifiées par un calcul accompagné d'un raisonnement.}
\begin{enumerate}\item
Placer les points \( A=(-2;5)\), \( B=(-4;-4)\), \( C=(1;10)\) et \( D=(-5;-6)\) dans un repère orthonormé. Calculer la longueur du segment \( [AB]\) et les coordonnées du milieu du segment \( [CD]\).
\item
Est-ce que le triangle formé par les points \( A(-1;-7)\), \( B(1;-9)\) et \( C(2;-6)\) est isocèle ?

\end{enumerate}
} 
\vspace{2cm}
\vbox{35
\emph{Toutes les réponses doivent être justifiées par un calcul accompagné d'un raisonnement.}
\begin{enumerate}\item
Placer les points \( A=(3;-1)\), \( B=(8;-8)\), \( C=(-3;2)\) et \( D=(-6;-2)\) dans un repère orthonormé. Calculer la longueur du segment \( [AB]\) et les coordonnées du milieu du segment \( [CD]\).
\item
Est-ce que le triangle formé par les points \( A(5;6)\), \( B(3;8)\) et \( C(15;8)\) est isocèle ?

\end{enumerate}
} 
\vspace{2cm}
\vbox{36
\emph{Toutes les réponses doivent être justifiées par un calcul accompagné d'un raisonnement.}
\begin{enumerate}\item
Placer les points \( A=(1;-6)\), \( B=(1;-9)\), \( C=(7;8)\) et \( D=(4;5)\) dans un repère orthonormé. Calculer la longueur du segment \( [AB]\) et les coordonnées du milieu du segment \( [CD]\).
\item
Est-ce que le triangle formé par les points \( A(-7;10)\), \( B(-3;6)\) et \( C(2;5)\) est isocèle ?

\end{enumerate}
} 
\vspace{2cm}
\vbox{37
\emph{Toutes les réponses doivent être justifiées par un calcul accompagné d'un raisonnement.}
\begin{enumerate}\item
Placer les points \( A=(-7;-2)\), \( B=(-5;-10)\), \( C=(-5;10)\) et \( D=(-4;-5)\) dans un repère orthonormé. Calculer la longueur du segment \( [AB]\) et les coordonnées du milieu du segment \( [CD]\).
\item
Est-ce que le triangle formé par les points \( A(1;-7)\), \( B(-3;-3)\) et \( C(2;-6)\) est rectangle ?

\end{enumerate}
} 
\vspace{2cm}
\vbox{38
\emph{Toutes les réponses doivent être justifiées par un calcul accompagné d'un raisonnement.}
\begin{enumerate}\item
Placer les points \( A=(-6;9)\), \( B=(5;-6)\), \( C=(7;0)\) et \( D=(3;7)\) dans un repère orthonormé. Calculer la longueur du segment \( [AB]\) et les coordonnées du milieu du segment \( [CD]\).
\item
Est-ce que le triangle formé par les points \( A(-1;5)\), \( B(4;0)\) et \( C(12;8)\) est rectangle ?

\end{enumerate}
} 
\vspace{2cm}
\vbox{39
\emph{Toutes les réponses doivent être justifiées par un calcul accompagné d'un raisonnement.}
\begin{enumerate}\item
Placer les points \( A=(-2;-5)\), \( B=(6;7)\), \( C=(8;2)\) et \( D=(0;6)\) dans un repère orthonormé. Calculer la longueur du segment \( [AB]\) et les coordonnées du milieu du segment \( [CD]\).
\item
Est-ce que le triangle formé par les points \( A(-9;-1)\), \( B(-15;5)\) et \( C(-5;-1)\) est isocèle ?

\end{enumerate}
} 
\vspace{2cm}
\vbox{40
\emph{Toutes les réponses doivent être justifiées par un calcul accompagné d'un raisonnement.}
\begin{enumerate}\item
Placer les points \( A=(0;5)\), \( B=(2;0)\), \( C=(-2;1)\) et \( D=(10;-5)\) dans un repère orthonormé. Calculer la longueur du segment \( [AB]\) et les coordonnées du milieu du segment \( [CD]\).
\item
Est-ce que le triangle formé par les points \( A(-9;-5)\), \( B(-11;-3)\) et \( C(-1;-5)\) est isocèle ?

\end{enumerate}
} 
\vspace{2cm}
\vbox{41
\emph{Toutes les réponses doivent être justifiées par un calcul accompagné d'un raisonnement.}
\begin{enumerate}\item
Placer les points \( A=(4;-3)\), \( B=(4;4)\), \( C=(9;-5)\) et \( D=(-4;-8)\) dans un repère orthonormé. Calculer la longueur du segment \( [AB]\) et les coordonnées du milieu du segment \( [CD]\).
\item
Est-ce que le triangle formé par les points \( A(-2;2)\), \( B(-4;4)\) et \( C(-1;3)\) est rectangle ?

\end{enumerate}
} 
\vspace{2cm}
\vbox{42
\emph{Toutes les réponses doivent être justifiées par un calcul accompagné d'un raisonnement.}
\begin{enumerate}\item
Placer les points \( A=(4;-1)\), \( B=(0;-5)\), \( C=(2;-1)\) et \( D=(9;7)\) dans un repère orthonormé. Calculer la longueur du segment \( [AB]\) et les coordonnées du milieu du segment \( [CD]\).
\item
Est-ce que le triangle formé par les points \( A(8;7)\), \( B(10;5)\) et \( C(5;2)\) est isocèle ?

\end{enumerate}
} 
\vspace{2cm}
\vbox{43
\emph{Toutes les réponses doivent être justifiées par un calcul accompagné d'un raisonnement.}
\begin{enumerate}\item
Placer les points \( A=(3;8)\), \( B=(-1;6)\), \( C=(2;-7)\) et \( D=(0;-4)\) dans un repère orthonormé. Calculer la longueur du segment \( [AB]\) et les coordonnées du milieu du segment \( [CD]\).
\item
Est-ce que le triangle formé par les points \( A(7;6)\), \( B(11;2)\) et \( C(7;6)\) est rectangle ?

\end{enumerate}
} 
\vspace{2cm}
\vbox{44
\emph{Toutes les réponses doivent être justifiées par un calcul accompagné d'un raisonnement.}
\begin{enumerate}\item
Placer les points \( A=(8;9)\), \( B=(0;10)\), \( C=(-5;10)\) et \( D=(-10;10)\) dans un repère orthonormé. Calculer la longueur du segment \( [AB]\) et les coordonnées du milieu du segment \( [CD]\).
\item
Est-ce que le triangle formé par les points \( A(8;-10)\), \( B(4;-6)\) et \( C(19;-5)\) est isocèle ?

\end{enumerate}
} 
\vspace{2cm}
\vbox{45
\emph{Toutes les réponses doivent être justifiées par un calcul accompagné d'un raisonnement.}
\begin{enumerate}\item
Placer les points \( A=(-1;-10)\), \( B=(-8;2)\), \( C=(-2;0)\) et \( D=(9;-7)\) dans un repère orthonormé. Calculer la longueur du segment \( [AB]\) et les coordonnées du milieu du segment \( [CD]\).
\item
Est-ce que le triangle formé par les points \( A(10;-5)\), \( B(10;-5)\) et \( C(12;-3)\) est rectangle ?

\end{enumerate}
} 
\vspace{2cm}
\vbox{46
\emph{Toutes les réponses doivent être justifiées par un calcul accompagné d'un raisonnement.}
\begin{enumerate}\item
Placer les points \( A=(4;0)\), \( B=(0;-10)\), \( C=(3;8)\) et \( D=(-6;-9)\) dans un repère orthonormé. Calculer la longueur du segment \( [AB]\) et les coordonnées du milieu du segment \( [CD]\).
\item
Est-ce que le triangle formé par les points \( A(3;9)\), \( B(7;5)\) et \( C(2;8)\) est rectangle ?

\end{enumerate}
} 
\vspace{2cm}
\vbox{47
\emph{Toutes les réponses doivent être justifiées par un calcul accompagné d'un raisonnement.}
\begin{enumerate}\item
Placer les points \( A=(2;9)\), \( B=(-6;-2)\), \( C=(0;-6)\) et \( D=(2;-3)\) dans un repère orthonormé. Calculer la longueur du segment \( [AB]\) et les coordonnées du milieu du segment \( [CD]\).
\item
Est-ce que le triangle formé par les points \( A(-8;-8)\), \( B(-10;-6)\) et \( C(0;-8)\) est isocèle ?

\end{enumerate}
} 
\vspace{2cm}
\vbox{48
\emph{Toutes les réponses doivent être justifiées par un calcul accompagné d'un raisonnement.}
\begin{enumerate}\item
Placer les points \( A=(3;5)\), \( B=(2;-1)\), \( C=(1;9)\) et \( D=(5;8)\) dans un repère orthonormé. Calculer la longueur du segment \( [AB]\) et les coordonnées du milieu du segment \( [CD]\).
\item
Est-ce que le triangle formé par les points \( A(8;-5)\), \( B(4;-1)\) et \( C(4;-5)\) est isocèle ?

\end{enumerate}
} 
\vspace{2cm}
\vbox{49
\emph{Toutes les réponses doivent être justifiées par un calcul accompagné d'un raisonnement.}
\begin{enumerate}\item
Placer les points \( A=(8;6)\), \( B=(-8;-7)\), \( C=(9;-5)\) et \( D=(-5;-2)\) dans un repère orthonormé. Calculer la longueur du segment \( [AB]\) et les coordonnées du milieu du segment \( [CD]\).
\item
Est-ce que le triangle formé par les points \( A(4;1)\), \( B(0;5)\) et \( C(10;1)\) est isocèle ?

\end{enumerate}
} 
\vspace{2cm}
\vbox{50
\emph{Toutes les réponses doivent être justifiées par un calcul accompagné d'un raisonnement.}
\begin{enumerate}\item
Placer les points \( A=(2;-10)\), \( B=(-1;2)\), \( C=(0;1)\) et \( D=(-4;-4)\) dans un repère orthonormé. Calculer la longueur du segment \( [AB]\) et les coordonnées du milieu du segment \( [CD]\).
\item
Est-ce que le triangle formé par les points \( A(-9;4)\), \( B(-9;4)\) et \( C(-10;3)\) est isocèle ?

\end{enumerate}
} 
\vspace{2cm}
\vbox{51
\emph{Toutes les réponses doivent être justifiées par un calcul accompagné d'un raisonnement.}
\begin{enumerate}\item
Placer les points \( A=(-9;9)\), \( B=(6;-10)\), \( C=(-8;10)\) et \( D=(5;9)\) dans un repère orthonormé. Calculer la longueur du segment \( [AB]\) et les coordonnées du milieu du segment \( [CD]\).
\item
Est-ce que le triangle formé par les points \( A(-4;0)\), \( B(-10;6)\) et \( C(-7;3)\) est isocèle ?

\end{enumerate}
} 
\vspace{2cm}
\vbox{52
\emph{Toutes les réponses doivent être justifiées par un calcul accompagné d'un raisonnement.}
\begin{enumerate}\item
Placer les points \( A=(-5;0)\), \( B=(8;3)\), \( C=(1;-2)\) et \( D=(-4;5)\) dans un repère orthonormé. Calculer la longueur du segment \( [AB]\) et les coordonnées du milieu du segment \( [CD]\).
\item
Est-ce que le triangle formé par les points \( A(7;-6)\), \( B(1;0)\) et \( C(1;-6)\) est isocèle ?

\end{enumerate}
} 
\vspace{2cm}
\vbox{53
\emph{Toutes les réponses doivent être justifiées par un calcul accompagné d'un raisonnement.}
\begin{enumerate}\item
Placer les points \( A=(0;8)\), \( B=(7;6)\), \( C=(8;4)\) et \( D=(0;6)\) dans un repère orthonormé. Calculer la longueur du segment \( [AB]\) et les coordonnées du milieu du segment \( [CD]\).
\item
Est-ce que le triangle formé par les points \( A(10;-8)\), \( B(10;-8)\) et \( C(11;-7)\) est rectangle ?

\end{enumerate}
} 
\vspace{2cm}
\vbox{54
\emph{Toutes les réponses doivent être justifiées par un calcul accompagné d'un raisonnement.}
\begin{enumerate}\item
Placer les points \( A=(9;-8)\), \( B=(7;-8)\), \( C=(-5;4)\) et \( D=(-9;-5)\) dans un repère orthonormé. Calculer la longueur du segment \( [AB]\) et les coordonnées du milieu du segment \( [CD]\).
\item
Est-ce que le triangle formé par les points \( A(3;-9)\), \( B(-3;-3)\) et \( C(11;-5)\) est isocèle ?

\end{enumerate}
} 
\vspace{2cm}
\vbox{55
\emph{Toutes les réponses doivent être justifiées par un calcul accompagné d'un raisonnement.}
\begin{enumerate}\item
Placer les points \( A=(9;9)\), \( B=(8;9)\), \( C=(9;-4)\) et \( D=(4;-8)\) dans un repère orthonormé. Calculer la longueur du segment \( [AB]\) et les coordonnées du milieu du segment \( [CD]\).
\item
Est-ce que le triangle formé par les points \( A(4;5)\), \( B(5;4)\) et \( C(4;5)\) est rectangle ?

\end{enumerate}
} 
\vspace{2cm}
\vbox{56
\emph{Toutes les réponses doivent être justifiées par un calcul accompagné d'un raisonnement.}
\begin{enumerate}\item
Placer les points \( A=(2;10)\), \( B=(10;-1)\), \( C=(-9;4)\) et \( D=(1;7)\) dans un repère orthonormé. Calculer la longueur du segment \( [AB]\) et les coordonnées du milieu du segment \( [CD]\).
\item
Est-ce que le triangle formé par les points \( A(-7;3)\), \( B(-6;2)\) et \( C(6;6)\) est rectangle ?

\end{enumerate}
} 
\vspace{2cm}
\vbox{57
\emph{Toutes les réponses doivent être justifiées par un calcul accompagné d'un raisonnement.}
\begin{enumerate}\item
Placer les points \( A=(-4;-1)\), \( B=(-8;-3)\), \( C=(-3;10)\) et \( D=(-1;-2)\) dans un repère orthonormé. Calculer la longueur du segment \( [AB]\) et les coordonnées du milieu du segment \( [CD]\).
\item
Est-ce que le triangle formé par les points \( A(-4;8)\), \( B(-6;10)\) et \( C(9;11)\) est rectangle ?

\end{enumerate}
} 
\vspace{2cm}
\vbox{58
\emph{Toutes les réponses doivent être justifiées par un calcul accompagné d'un raisonnement.}
\begin{enumerate}\item
Placer les points \( A=(9;-10)\), \( B=(-4;0)\), \( C=(8;-2)\) et \( D=(4;4)\) dans un repère orthonormé. Calculer la longueur du segment \( [AB]\) et les coordonnées du milieu du segment \( [CD]\).
\item
Est-ce que le triangle formé par les points \( A(-5;7)\), \( B(-3;5)\) et \( C(5;7)\) est rectangle ?

\end{enumerate}
} 
\vspace{2cm}
\vbox{59
\emph{Toutes les réponses doivent être justifiées par un calcul accompagné d'un raisonnement.}
\begin{enumerate}\item
Placer les points \( A=(6;-6)\), \( B=(2;2)\), \( C=(4;4)\) et \( D=(1;6)\) dans un repère orthonormé. Calculer la longueur du segment \( [AB]\) et les coordonnées du milieu du segment \( [CD]\).
\item
Est-ce que le triangle formé par les points \( A(2;9)\), \( B(-3;14)\) et \( C(3;10)\) est rectangle ?

\end{enumerate}
} 
\vspace{2cm}
\vbox{60
\emph{Toutes les réponses doivent être justifiées par un calcul accompagné d'un raisonnement.}
\begin{enumerate}\item
Placer les points \( A=(-10;-10)\), \( B=(5;-8)\), \( C=(10;7)\) et \( D=(3;-4)\) dans un repère orthonormé. Calculer la longueur du segment \( [AB]\) et les coordonnées du milieu du segment \( [CD]\).
\item
Est-ce que le triangle formé par les points \( A(3;6)\), \( B(1;8)\) et \( C(0;5)\) est isocèle ?

\end{enumerate}
} 
\vspace{2cm}
\vbox{61
\emph{Toutes les réponses doivent être justifiées par un calcul accompagné d'un raisonnement.}
\begin{enumerate}\item
Placer les points \( A=(5;-2)\), \( B=(9;-10)\), \( C=(4;-1)\) et \( D=(-10;-9)\) dans un repère orthonormé. Calculer la longueur du segment \( [AB]\) et les coordonnées du milieu du segment \( [CD]\).
\item
Est-ce que le triangle formé par les points \( A(-3;9)\), \( B(1;5)\) et \( C(5;3)\) est isocèle ?

\end{enumerate}
} 
\vspace{2cm}
\vbox{62
\emph{Toutes les réponses doivent être justifiées par un calcul accompagné d'un raisonnement.}
\begin{enumerate}\item
Placer les points \( A=(5;3)\), \( B=(-10;9)\), \( C=(2;10)\) et \( D=(-5;-2)\) dans un repère orthonormé. Calculer la longueur du segment \( [AB]\) et les coordonnées du milieu du segment \( [CD]\).
\item
Est-ce que le triangle formé par les points \( A(-10;-6)\), \( B(-14;-2)\) et \( C(1;-5)\) est rectangle ?

\end{enumerate}
} 
\vspace{2cm}
\vbox{63
\emph{Toutes les réponses doivent être justifiées par un calcul accompagné d'un raisonnement.}
\begin{enumerate}\item
Placer les points \( A=(-8;10)\), \( B=(6;3)\), \( C=(-5;-2)\) et \( D=(6;10)\) dans un repère orthonormé. Calculer la longueur du segment \( [AB]\) et les coordonnées du milieu du segment \( [CD]\).
\item
Est-ce que le triangle formé par les points \( A(6;-8)\), \( B(10;-12)\) et \( C(6;-12)\) est isocèle ?

\end{enumerate}
} 
\vspace{2cm}
\vbox{64
\emph{Toutes les réponses doivent être justifiées par un calcul accompagné d'un raisonnement.}
\begin{enumerate}\item
Placer les points \( A=(9;2)\), \( B=(-5;0)\), \( C=(4;-8)\) et \( D=(8;-5)\) dans un repère orthonormé. Calculer la longueur du segment \( [AB]\) et les coordonnées du milieu du segment \( [CD]\).
\item
Est-ce que le triangle formé par les points \( A(-1;4)\), \( B(-1;4)\) et \( C(12;7)\) est isocèle ?

\end{enumerate}
} 
\vspace{2cm}
\vbox{65
\emph{Toutes les réponses doivent être justifiées par un calcul accompagné d'un raisonnement.}
\begin{enumerate}\item
Placer les points \( A=(-5;3)\), \( B=(-5;-4)\), \( C=(0;-2)\) et \( D=(3;-5)\) dans un repère orthonormé. Calculer la longueur du segment \( [AB]\) et les coordonnées du milieu du segment \( [CD]\).
\item
Est-ce que le triangle formé par les points \( A(3;-2)\), \( B(7;-6)\) et \( C(1;-8)\) est isocèle ?

\end{enumerate}
} 
\vspace{2cm}
\vbox{66
\emph{Toutes les réponses doivent être justifiées par un calcul accompagné d'un raisonnement.}
\begin{enumerate}\item
Placer les points \( A=(2;2)\), \( B=(4;-9)\), \( C=(-10;7)\) et \( D=(0;-7)\) dans un repère orthonormé. Calculer la longueur du segment \( [AB]\) et les coordonnées du milieu du segment \( [CD]\).
\item
Est-ce que le triangle formé par les points \( A(-4;1)\), \( B(-2;-1)\) et \( C(4;-3)\) est isocèle ?

\end{enumerate}
} 
\vspace{2cm}
\vbox{67
\emph{Toutes les réponses doivent être justifiées par un calcul accompagné d'un raisonnement.}
\begin{enumerate}\item
Placer les points \( A=(-2;-9)\), \( B=(-6;8)\), \( C=(-9;6)\) et \( D=(4;-8)\) dans un repère orthonormé. Calculer la longueur du segment \( [AB]\) et les coordonnées du milieu du segment \( [CD]\).
\item
Est-ce que le triangle formé par les points \( A(3;-8)\), \( B(7;-12)\) et \( C(4;-11)\) est isocèle ?

\end{enumerate}
} 
\vspace{2cm}
\vbox{68
\emph{Toutes les réponses doivent être justifiées par un calcul accompagné d'un raisonnement.}
\begin{enumerate}\item
Placer les points \( A=(3;-8)\), \( B=(-1;6)\), \( C=(-5;4)\) et \( D=(10;-9)\) dans un repère orthonormé. Calculer la longueur du segment \( [AB]\) et les coordonnées du milieu du segment \( [CD]\).
\item
Est-ce que le triangle formé par les points \( A(2;-1)\), \( B(4;-3)\) et \( C(6;1)\) est isocèle ?

\end{enumerate}
} 
\vspace{2cm}
\vbox{69
\emph{Toutes les réponses doivent être justifiées par un calcul accompagné d'un raisonnement.}
\begin{enumerate}\item
Placer les points \( A=(5;1)\), \( B=(6;-2)\), \( C=(-9;10)\) et \( D=(2;2)\) dans un repère orthonormé. Calculer la longueur du segment \( [AB]\) et les coordonnées du milieu du segment \( [CD]\).
\item
Est-ce que le triangle formé par les points \( A(-7;-4)\), \( B(-8;-3)\) et \( C(6;-1)\) est rectangle ?

\end{enumerate}
} 
\vspace{2cm}
\vbox{70
\emph{Toutes les réponses doivent être justifiées par un calcul accompagné d'un raisonnement.}
\begin{enumerate}\item
Placer les points \( A=(-6;-6)\), \( B=(-6;-5)\), \( C=(10;-3)\) et \( D=(-9;-2)\) dans un repère orthonormé. Calculer la longueur du segment \( [AB]\) et les coordonnées du milieu du segment \( [CD]\).
\item
Est-ce que le triangle formé par les points \( A(-6;-5)\), \( B(-6;-5)\) et \( C(7;-2)\) est rectangle ?

\end{enumerate}
} 
\vspace{2cm}
\vbox{71
\emph{Toutes les réponses doivent être justifiées par un calcul accompagné d'un raisonnement.}
\begin{enumerate}\item
Placer les points \( A=(-2;-7)\), \( B=(2;-2)\), \( C=(-10;6)\) et \( D=(-7;6)\) dans un repère orthonormé. Calculer la longueur du segment \( [AB]\) et les coordonnées du milieu du segment \( [CD]\).
\item
Est-ce que le triangle formé par les points \( A(10;0)\), \( B(8;2)\) et \( C(19;-1)\) est rectangle ?

\end{enumerate}
} 
\vspace{2cm}
\vbox{72
\emph{Toutes les réponses doivent être justifiées par un calcul accompagné d'un raisonnement.}
\begin{enumerate}\item
Placer les points \( A=(1;6)\), \( B=(-1;9)\), \( C=(3;5)\) et \( D=(0;8)\) dans un repère orthonormé. Calculer la longueur du segment \( [AB]\) et les coordonnées du milieu du segment \( [CD]\).
\item
Est-ce que le triangle formé par les points \( A(10;10)\), \( B(14;6)\) et \( C(10;10)\) est rectangle ?

\end{enumerate}
} 
\vspace{2cm}
\vbox{73
\emph{Toutes les réponses doivent être justifiées par un calcul accompagné d'un raisonnement.}
\begin{enumerate}\item
Placer les points \( A=(-6;-10)\), \( B=(1;8)\), \( C=(-3;8)\) et \( D=(2;-3)\) dans un repère orthonormé. Calculer la longueur du segment \( [AB]\) et les coordonnées du milieu du segment \( [CD]\).
\item
Est-ce que le triangle formé par les points \( A(-7;-6)\), \( B(-13;0)\) et \( C(0;-3)\) est isocèle ?

\end{enumerate}
} 
\vspace{2cm}
\vbox{74
\emph{Toutes les réponses doivent être justifiées par un calcul accompagné d'un raisonnement.}
\begin{enumerate}\item
Placer les points \( A=(1;-1)\), \( B=(4;0)\), \( C=(6;-5)\) et \( D=(5;-4)\) dans un repère orthonormé. Calculer la longueur du segment \( [AB]\) et les coordonnées du milieu du segment \( [CD]\).
\item
Est-ce que le triangle formé par les points \( A(6;1)\), \( B(4;3)\) et \( C(11;-2)\) est isocèle ?

\end{enumerate}
} 
\vspace{2cm}
\vbox{75
\emph{Toutes les réponses doivent être justifiées par un calcul accompagné d'un raisonnement.}
\begin{enumerate}\item
Placer les points \( A=(0;1)\), \( B=(-7;-7)\), \( C=(3;8)\) et \( D=(7;-3)\) dans un repère orthonormé. Calculer la longueur du segment \( [AB]\) et les coordonnées du milieu du segment \( [CD]\).
\item
Est-ce que le triangle formé par les points \( A(10;5)\), \( B(14;1)\) et \( C(14;5)\) est isocèle ?

\end{enumerate}
} 
\vspace{2cm}
\vbox{76
\emph{Toutes les réponses doivent être justifiées par un calcul accompagné d'un raisonnement.}
\begin{enumerate}\item
Placer les points \( A=(10;-4)\), \( B=(-2;-1)\), \( C=(-3;3)\) et \( D=(-3;1)\) dans un repère orthonormé. Calculer la longueur du segment \( [AB]\) et les coordonnées du milieu du segment \( [CD]\).
\item
Est-ce que le triangle formé par les points \( A(-1;6)\), \( B(-6;11)\) et \( C(-3;4)\) est rectangle ?

\end{enumerate}
} 
\vspace{2cm}
\vbox{77
\emph{Toutes les réponses doivent être justifiées par un calcul accompagné d'un raisonnement.}
\begin{enumerate}\item
Placer les points \( A=(-7;-2)\), \( B=(6;10)\), \( C=(0;1)\) et \( D=(1;5)\) dans un repère orthonormé. Calculer la longueur du segment \( [AB]\) et les coordonnées du milieu du segment \( [CD]\).
\item
Est-ce que le triangle formé par les points \( A(-7;-3)\), \( B(-5;-5)\) et \( C(4;-4)\) est isocèle ?

\end{enumerate}
} 
\vspace{2cm}
\vbox{78
\emph{Toutes les réponses doivent être justifiées par un calcul accompagné d'un raisonnement.}
\begin{enumerate}\item
Placer les points \( A=(-7;7)\), \( B=(1;-10)\), \( C=(-7;0)\) et \( D=(-1;-3)\) dans un repère orthonormé. Calculer la longueur du segment \( [AB]\) et les coordonnées du milieu du segment \( [CD]\).
\item
Est-ce que le triangle formé par les points \( A(0;-8)\), \( B(2;-10)\) et \( C(14;-6)\) est isocèle ?

\end{enumerate}
} 
\vspace{2cm}
\vbox{79
\emph{Toutes les réponses doivent être justifiées par un calcul accompagné d'un raisonnement.}
\begin{enumerate}\item
Placer les points \( A=(-4;8)\), \( B=(-7;1)\), \( C=(10;8)\) et \( D=(-7;-1)\) dans un repère orthonormé. Calculer la longueur du segment \( [AB]\) et les coordonnées du milieu du segment \( [CD]\).
\item
Est-ce que le triangle formé par les points \( A(-4;-10)\), \( B(0;-14)\) et \( C(-7;-13)\) est rectangle ?

\end{enumerate}
} 
\vspace{2cm}
\vbox{80
\emph{Toutes les réponses doivent être justifiées par un calcul accompagné d'un raisonnement.}
\begin{enumerate}\item
Placer les points \( A=(-10;-10)\), \( B=(-1;2)\), \( C=(-6;-5)\) et \( D=(0;8)\) dans un repère orthonormé. Calculer la longueur du segment \( [AB]\) et les coordonnées du milieu du segment \( [CD]\).
\item
Est-ce que le triangle formé par les points \( A(-10;4)\), \( B(-11;5)\) et \( C(2;6)\) est rectangle ?

\end{enumerate}
} 
\vspace{2cm}
\vbox{81
\emph{Toutes les réponses doivent être justifiées par un calcul accompagné d'un raisonnement.}
\begin{enumerate}\item
Placer les points \( A=(0;0)\), \( B=(6;9)\), \( C=(-8;-9)\) et \( D=(-8;4)\) dans un repère orthonormé. Calculer la longueur du segment \( [AB]\) et les coordonnées du milieu du segment \( [CD]\).
\item
Est-ce que le triangle formé par les points \( A(2;-1)\), \( B(-4;5)\) et \( C(12;5)\) est isocèle ?

\end{enumerate}
} 
\vspace{2cm}
\vbox{82
\emph{Toutes les réponses doivent être justifiées par un calcul accompagné d'un raisonnement.}
\begin{enumerate}\item
Placer les points \( A=(-3;-9)\), \( B=(1;7)\), \( C=(5;-1)\) et \( D=(7;9)\) dans un repère orthonormé. Calculer la longueur du segment \( [AB]\) et les coordonnées du milieu du segment \( [CD]\).
\item
Est-ce que le triangle formé par les points \( A(-7;-2)\), \( B(-5;-4)\) et \( C(-8;-3)\) est rectangle ?

\end{enumerate}
} 
\vspace{2cm}
\vbox{83
\emph{Toutes les réponses doivent être justifiées par un calcul accompagné d'un raisonnement.}
\begin{enumerate}\item
Placer les points \( A=(4;4)\), \( B=(10;7)\), \( C=(-10;-7)\) et \( D=(-3;-5)\) dans un repère orthonormé. Calculer la longueur du segment \( [AB]\) et les coordonnées du milieu du segment \( [CD]\).
\item
Est-ce que le triangle formé par les points \( A(0;8)\), \( B(0;8)\) et \( C(10;8)\) est rectangle ?

\end{enumerate}
} 
\vspace{2cm}
\vbox{84
\emph{Toutes les réponses doivent être justifiées par un calcul accompagné d'un raisonnement.}
\begin{enumerate}\item
Placer les points \( A=(-1;5)\), \( B=(-10;-9)\), \( C=(10;9)\) et \( D=(-6;-8)\) dans un repère orthonormé. Calculer la longueur du segment \( [AB]\) et les coordonnées du milieu du segment \( [CD]\).
\item
Est-ce que le triangle formé par les points \( A(9;-3)\), \( B(6;0)\) et \( C(12;0)\) est rectangle ?

\end{enumerate}
} 
\vspace{2cm}
