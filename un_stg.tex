% This is part of Un soupçon de mathématique sans être agressif pour autant
% Copyright (c) 2012
%   Laurent Claessens
% See the file fdl-1.3.txt for copying conditions.

\chapter{Proportions, pourcentages}

%+++++++++++++++++++++++++++++++++++++++++++++++++++++++++++++++++++++++++++++++++++++++++++++++++++++++++++++++++++++++++++
\section{Théorie}
%+++++++++++++++++++++++++++++++++++++++++++++++++++++++++++++++++++++++++++++++++++++++++++++++++++++++++++++++++++++++++++

Nous disons qu'un gaz est en concentration de une \defe{partie par million}{partie par million} si un million de grammes d'air contient un gramme du gaz. Voici quelque chiffre concernant l'évolution de la concentration de \( CO_2\) dans l'atmosphère; les chiffres sont en \( \unit{}{ppm}\) :
\begin{center}
\begin{tabular}{|c|c|c|}
    \hline
    1750    &   2005    &   1012\\
    \hline
    280&380&395\\
    \hline
\end{tabular}
\end{center}
Pour information, cette concentration n'a pas dépassé les \unit{300}{ppm} depuis au moins \( 600.000\) ans.

\begin{enumerate}
    \item
        De combien de pourcent la concentration de \( CO_2\) a augmenté entre 1750 et 2005 ?
    \item
        Sur un kilo d'air, combien de grammes de \( CO_2\) ?
    \item 
        Quelle est la vitesse (en ppm par an) d'augmentation de la concentration entre 1750 et 2005 ? Même question entre 2005 et 2012.
\end{enumerate}

\begin{definition}
    Une \defe{population}{population} est un ensemble fini. Si \( E\) est une population, une \defe{sous-population}{sous-population} est un sous-ensemble \( A\subset E\). L'\defe{effectif}{effectif} d'une population est le nombre de ses éléments.

    La \defe{proportion}{proportion} de \( A\) dans \( E\) est le rapport
    \begin{equation}
        p_A=\frac{ n_A }{ n_E }
    \end{equation}
    où \( n_A\) et \( n_E\) sont les effectifs de \( A\) et \( E\).
\end{definition}
Note : une proportion est un nombre compris entre zéro et un.

\begin{example}
    Demander combien il y a de gauchers dans la classe. Quelle en est la proportion ?
\end{example}

\begin{example}
    Diviser la classe en \( 4\) groupes suivant que l'élève habite ou non à Dole et qu'il utilise ou non un cahier. Remplir le tableau suivant :

    \begin{center}
    \begin{tabular}{|l||c|c||c|}
        \hline\hline
        & habite à Dole&n'habite pas à Dole&total\\
        \hline
        utilise un cahier&&&\\
        \hline
        n'utilise pas de cahier&&&\\
        \hline\hline
        total&&&\\
        \hline
    \end{tabular}
    \end{center}

    Soit \( E\) la population totale : toute la classe; soit $A$ la population de ceux qui vivent à Dole; et \( B\) celle de ceux qui utilisent un cahier.
    \begin{enumerate}
        \item
            Trouver les effectifs des populations \( A\cup B\) et \( A\cap B\).
        \item
            Quelle est la proportion de \( A\) dans \( E\) ? Et celle du complémentaire \( \bar A\) ?
        \item
            Exprimer les proportions \( p_{A\cap B}\) et \( p_{A\cup B}\).
    \end{enumerate}
\end{example}

Nous avons l'égalité
\begin{equation}
    n_{A\cup B}=n_A+n_B-n_{A\cap B}
\end{equation}
parce que dans le compte \( n_A+n_B\), nous comptons deux fois les individus qui sont dans \( A\) et dans \( B\). En passant aux proportions (c'est à dire en divisant tout par \( n_E\)), nous avons la formule
\begin{equation}
    p_{A\cup B}=p_A+p_B-p_{A\cap B}.
\end{equation}

\begin{remark}
    Si les populations \( A\) et \( B\) sont disjointes, alors \( n_{A\cap B}=0\) et nous trouvons la formule
    \begin{equation}
        p_{A\cup B}=p_A+p_B.
    \end{equation}
\end{remark}


%+++++++++++++++++++++++++++++++++++++++++++++++++++++++++++++++++++++++++++++++++++++++++++++++++++++++++++++++++++++++++++
\section{Exercices}
%+++++++++++++++++++++++++++++++++++++++++++++++++++++++++++++++++++++++++++++++++++++++++++++++++++++++++++++++++++++++++++

\Exo{Seconde-0022}
\Exo{Premiere-0001}
\Exo{Premiere-0002}
\Exo{Premiere-0003}
\Exo{Premiere-0004}
\Exo{Premiere-0005}
\Exo{Premiere-0006}

\Exo{Premiere-0007}
\Exo{Premiere-0008}
\Exo{Premiere-0009}
\Exo{Premiere-0010}
\Exo{Premiere-0011}
\Exo{Premiere-0012}
\Exo{Seconde-0026}
\Exo{Seconde-0024}
\Exo{Premiere-0014}                                                                                                                                     
\Exo{Premiere-0015}                                               
\Exo{Premiere-0016}                                                                       
\Exo{Premiere-0017}                                                                                                                      
\Exo{Seconde-0034}

\Exo{Premiere-0023}                                                                                                                                    


\chapter{Second degré}

%+++++++++++++++++++++++++++++++++++++++++++++++++++++++++++++++++++++++++++++++++++++++++++++++++++++++++++++++++++++++++++
\section{Définitions}
%+++++++++++++++++++++++++++++++++++++++++++++++++++++++++++++++++++++++++++++++++++++++++++++++++++++++++++++++++++++++++++

\begin{definition}
    Une fonction \defe{polynôme de degré $2$}{polynôme (de degré $2$)} est une fonction s'exprimant sous la forme 
    \begin{equation}
    f(x)=ax^2+bx+c
    \end{equation}
    où \( a\), \( b\) et \( c\) sont des nombres réels avec \( a\neq 0\). 
    
    La courbe représentative d'un polynôme de degré deux dans un plan orthonormé est une \defe{parabole}{parabole}.
\end{definition}

%---------------------------------------------------------------------------------------------------------------------------
\subsection{Symétries}
%---------------------------------------------------------------------------------------------------------------------------

La parabole de la fonction \( f(x)=ax^2+bx+c\) (\( a\neq 0\)) est symétrique par rapport à la droite d'équation \( x=-\frac{ b }{ 2a }\), c'est à dire par rapport à la droite verticale en \( x=-b/2a\). Le \defe{sommet}{sommet (d'une parabole)} est le point d'abscisse 
\begin{equation}
x=-b/2a
\end{equation}
situé sur la parabole.

Deux paraboles avec leurs axes de symétries sont dessinées à la figure \ref{LabelFigParaboles}.
\newcommand{\CaptionFigParaboles}{Deux paraboles}
\input{Fig_Paraboles.pstricks}
See also the subfigure \ref{LabelFigParabolesssLabelSubFigParaboles0}
See also the subfigure \ref{LabelFigParabolesssLabelSubFigParaboles1}

Comment savoir si les branches de la paraboles \( ax^2+bx+c\) sont orientées vers le haut ou vers le bas ? La règle est simple : 
\begin{enumerate}
    \item
        si \( a>0\), alors elles sont tournées vers le haut;
    \item
        si \( a<0\), alors elles sont tournées vers le bas.
\end{enumerate}
Cette règle est facile à retenir en pensant par exemple à la parabole \( x^2+bx+c\) où \( b\) et \( c\) sont raisonnables. Si nous prenons \( x=1000\), alors \( x^2\) vaut un million alors que les deux autres termes ne valent que de l'ordre du mille. Il est alors clair que la branche part vers l'infini.

%+++++++++++++++++++++++++++++++++++++++++++++++++++++++++++++++++++++++++++++++++++++++++++++++++++++++++++++++++++++++++++
\section{Exercices}
%+++++++++++++++++++++++++++++++++++++++++++++++++++++++++++++++++++++++++++++++++++++++++++++++++++++++++++++++++++++++++++

\Exo{Premiere-0030}
\Exo{Premiere-0031}
\Exo{Premiere-0032}
\Exo{Premiere-0024}                                                                                                                                  
\Exo{Premiere-0025}                                                                                                                                   
\Exo{Premiere-0026}                                                                                               

%---------------------------------------------------------------------------------------------------------------------------
\subsection{Plus compliqués}
%---------------------------------------------------------------------------------------------------------------------------

\Exo{Premiere-0028}
\Exo{Premiere-0029}
