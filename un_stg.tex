% This is part of Un soupçon de mathématique sans être agressif pour autant
% Copyright (c) 2012
%   Laurent Claessens
% See the file fdl-1.3.txt for copying conditions.

\chapter{Proportions, pourcentages}

Nous disons qu'un gaz est en concentration de une \defe{partie par million}{partie par million} si un million de grammes d'air contient un gramme du gaz. Voici quelque chiffre concernant l'évolution de la concentration de \( CO_2\) dans l'atmosphère; les chiffres sont en \( \unit{}{ppm}\) :
\begin{center}
\begin{tabular}{|c|c|c|}
    \hline
    1750    &   2005    &   1012\\
    \hline
    280&380&395\\
    \hline
\end{tabular}
\end{center}
Pour information, cette concentration n'a pas dépassé les \unit{300}{ppm} depuis au moins \( 600.000\) ans.

\begin{enumerate}
    \item
        De combien de pourcent la concentration de \( CO_2\) a augmenté entre 1750 et 2005 ?
    \item
        Sur un kilo d'air, combien de grammes de \( CO_2\) ?
    \item 
        Quelle est la vitesse (en ppm par an) d'augmentation de la concentration entre 1750 et 2005 ? Même question entre 2005 et 2012.
\end{enumerate}

\begin{definition}
    Une \defe{population}{population} est un ensemble fini. Si \( E\) est une population, une \defe{sous-population}{sous-population} est un sous-ensemble \( A\subset E\). L'\defe{effectif}{effectif} d'une population est le nombre de ses éléments.

    La \defe{proportion}{proportion} de \( A\) dans \( E\) est le rapport
    \begin{equation}
        p_A=\frac{ n_A }{ n_E }
    \end{equation}
    où \( n_A\) et \( n_E\) sont les effectifs de \( A\) et \( E\).
\end{definition}
Note : une proportion est un nombre compris entre zéro et un.

\begin{example}
    Demander combien il y a de gauchers dans la classe. Quelle en est la proportion ?
\end{example}

\begin{example}
    Diviser la classe en \( 4\) groupes suivant que l'élève habite ou non à Dole et qu'il utilise ou non un cahier. Remplir le tableau suivant :

    \begin{center}
    \begin{tabular}{|l||c|c|c|}
        \hline
        & habite à Dole&n'habite pas à Dole&total\\
        \hline
        utilise un cahier&&&\\
        \hline
        n'utilise pas de cahier&&&\\
        \hline
        total&&&\\
        \hline
    \end{tabular}
    \end{center}

    Soit \( E\) la population totale : toute la classe; soit $A$ la population de ceux qui vivent à Dole; et \( B\) celle de ceux qui utilisent un cahier.
    \begin{enumerate}
        \item
            Trouver les effectifs des populations \( A\cup B\) et \( A\cap B\).
        \item
            Quelle est la proportion de \( A\) dans \( E\) ? Et celle du complémentaire \( \bar A\) ?
        \item
            Exprimer les proportions \( p_{A\cap B}\) et \( p_{A\cup B}\).
    \end{enumerate}
\end{example}

Nous avons l'égalité
\begin{equation}
    n_{A\cup B}=n_A+n_B-n_{A\cap B}
\end{equation}
parce que dans le compte \( n_A+n_B\), nous comptons deux fois les individus qui sont dans \( A\) et dans \( B\). En passant aux proportions (c'est à dire en divisant tout par \( n_E\)), nous avons la formule
\begin{equation}
    p_{A\cup B}=p_A+p_B-p_{A\cap B}.
\end{equation}

\begin{remark}
    Si les populations \( A\) et \( B\) sont disjointes, alors \( n_{A\cap B}=0\) et nous trouvons la formule
    \begin{equation}
        p_{A\cup B}=p_A+p_B.
    \end{equation}
\end{remark}


%+++++++++++++++++++++++++++++++++++++++++++++++++++++++++++++++++++++++++++++++++++++++++++++++++++++++++++++++++++++++++++
\section{Exercices}
%+++++++++++++++++++++++++++++++++++++++++++++++++++++++++++++++++++++++++++++++++++++++++++++++++++++++++++++++++++++++++++

\Exo{Seconde-0022}
\Exo{Premiere-0001}
\Exo{Premiere-0002}
\Exo{Premiere-0003}
\Exo{Premiere-0004}
\Exo{Premiere-0005}
\Exo{Premiere-0006}

\Exo{Premiere-0007}
\Exo{Premiere-0008}
\Exo{Premiere-0009}
\Exo{Premiere-0010}
\Exo{Premiere-0011}
\Exo{Premiere-0012}
