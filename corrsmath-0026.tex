% This is part of Un soupçon de mathématique sans être agressif pour autant
% Copyright (c) 2012
%   Laurent Claessens
% See the file fdl-1.3.txt for copying conditions.

\begin{corrige}{smath-0026}

    \begin{enumerate}
        \item
            La formule est
            \begin{equation}
                AB=\sqrt{  (x_A-x_B)^2+(y_A+y_B)^2 }.
            \end{equation}
        \item
            Il suffit d'utiliser la formule \emph{en faisant attention aux signes} :
            \begin{equation}
                AB=\sqrt{  (-1-5)^2+(7-1)^2 }=\sqrt{36+36}=\sqrt{72}=6\sqrt{2}.
            \end{equation}
        \item
            Le dessin est à la figure \ref{LabelFigRectanglegHuBZs}. % From file RectanglegHuBZs
            \newcommand{\CaptionFigRectanglegHuBZs}{Le triangle de l'exercice \ref{exosmath-0026}.}
            \input{Fig_RectanglegHuBZs.pstricks}


    \end{enumerate}
    <++>

\end{corrige}
