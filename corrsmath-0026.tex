% This is part of Un soupçon de mathématique sans être agressif pour autant
% Copyright (c) 2012
%   Laurent Claessens
% See the file fdl-1.3.txt for copying conditions.

\begin{corrige}{smath-0026}

    \begin{enumerate}
        \item
            La formule est
            \begin{equation}
                AB=\sqrt{  (x_A-x_B)^2+(y_A+y_B)^2 }.
            \end{equation}
        \item
            Il suffit d'utiliser la formule \emph{en faisant attention aux signes} :
            \begin{equation}
                AB=\sqrt{  (-1-5)^2+(7-1)^2 }=\sqrt{36+36}=\sqrt{72}=6\sqrt{2}.
            \end{equation}
        \item
            Le dessin est à la figure \ref{LabelFigRectanglegHuBZs}. % From file RectanglegHuBZs
            \newcommand{\CaptionFigRectanglegHuBZs}{Le triangle de l'exercice \ref{exosmath-0026}.}
            \input{Fig_RectanglegHuBZs.pstricks}
            Nous cherchons la longueur \( x\). Par le théorème de Pythagore (et le fait que le triangle soit rectangle en \( B\)), nous avons
            \begin{subequations}
                \begin{align}
                17^2=x^2+8^2,\\
                289=x^2+64\\
                x^2=289-64    \\
                x^2=225,
                \end{align}
            \end{subequations}
            et donc \( x=\sqrt{225}=15\).
    \end{enumerate}

\end{corrige}
