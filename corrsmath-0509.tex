% This is part of Un soupçon de mathématique sans être agressif pour autant
% Copyright (c) 2013
%   Laurent Claessens
% See the file fdl-1.3.txt for copying conditions.

\begin{corrige}{smath-0509}

    \begin{enumerate}
        \item
            Une fonction affine est croissante si et seulement si son coefficient directeur est positif. C'est le cas de la fonction \( f\) dont le coefficient directeur vaut \( 3\). Donc elle est bien croissante.
        \item
            Le point \( (7;17)\) sera sur le graphe de \( g\) si et seulement si \( g(7)=17\). Il suffit donc de calculer :
            \begin{equation}
                g(7)=-14+3=-11.
            \end{equation}
            Le point \( (7;17)\) est donc bien sur le graphe de la fonction \( g\).
        \item
            La fonction \( h \) n'est pas une fonction affine. Il ne peut pas y avoir de \( x\) en dénominateur dans une fonction affine. Donc faux.
    \end{enumerate}

\end{corrige}
