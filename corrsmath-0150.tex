% This is part of Un soupçon de mathématique sans être agressif pour autant
% Copyright (c) 2013
%   Laurent Claessens
% See the file fdl-1.3.txt for copying conditions.

\begin{corrige}{smath-0150}

    Un vecteur directeur est le vecteur
    \begin{equation}
        \vect{ AB }=\begin{pmatrix}
            2    \\ 
            6    
        \end{pmatrix}.
    \end{equation}
    Donc le coefficient directeur de la droite cherchée est \( \frac{ 6 }{2}=3\). La fonction cherchée est donc de la forme 
    \begin{equation}
        f(x)=3x+b
    \end{equation}
    pour un certain \( b\) encore inconnu.

    Vu les points par lesquels la droite doit passer, il faut avoir \( f(1)=4\) et \( f(3)=10\), c'est à dire
    \begin{subequations}
        \begin{numcases}{}
            3+b=4\\
            9+b=10.
        \end{numcases}
    \end{subequations}
    Nous voyons alors que \( b=1\) fait l'affaire. Au final, l'équation de la droite demandée est
    \begin{equation}
        f(x)=3x+1.
    \end{equation}

\end{corrige}
