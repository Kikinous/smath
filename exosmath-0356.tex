% This is part of Un soupçon de mathématique sans être agressif pour autant
% Copyright (c) 2013
%   Laurent Claessens
% See the file fdl-1.3.txt for copying conditions.

\begin{exercice}\label{exosmath-0356}


Une agence de voyage propose trois formules : \( A\), \( B\) et \( C\). Sur les \( 100\) derniers clients, \( 35\) ont choisi l'option \( A\), \( 20\) la formule \( B\) et \( 35\) la formule \( C\). Parmi ces mêmes \( 100\) clients, \( 8\) ont pris les formules \( A\) et \( B\), \( 9\) ont pris les formules \( B\) et \( C\) et enfin \( 19\) ont pris \( A\) et \( C\).

\begin{enumerate}
    \item
        Dessiner un diagramme de Venn décrivant la situation.
    \item
        Combien de clients (parmi les \( 100\) derniers) n'ont choisi aucune des trois formules ?
\end{enumerate}

% AFAIRE : ceci provient d'un corrigé d'un DS du jeudi 31 mai 2012 de M.Lenzen. Il faut le retrouver et le citer.

\corrref{smath-0356}
\end{exercice}
