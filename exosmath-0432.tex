% This is part of Un soupçon de mathématique sans être agressif pour autant
% Copyright (c) 2013
%   Laurent Claessens
% See the file fdl-1.3.txt for copying conditions.

\begin{exercice}\label{exosmath-0432}

    Soient les points \( A=(2;1)\), \( B=(4;0)\) et \( C=(1;-2)\).
    \begin{enumerate}
        \item
            Déterminer les coordonnées des vecteurs \( \vect{ AB }\) et \( \vect{ CB }\). 
        \item
            Donner les coordonnées d'un point \( N\) tel que
            \begin{equation}
                \vect{ 3AB }+\vect{ AC }=\vect{ AN }.
            \end{equation}
        \item
            Représenter les points \( A\), \( B\), \( C\) et \( N\) dans un repère orthonormé.
        \item
            Soit \( Q\), le point de coordonnées \( (8;-2)\). Donner la valeur \( \lambda\) pour laquelle
            \begin{equation}
                \vect{ AB }=\lambda\vect{ AQ }.
            \end{equation}
        \item
            Montrer que le quadrilatère \( AQNC\) est un parallélogramme. Donner en le centre. 
        \item
            Est-ce que le quadrilatère \( AQNC\) est un rectangle ?
    \end{enumerate}

\corrref{smath-0432}
\end{exercice}
