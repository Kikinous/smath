% This is part of Un soupçon de mathématique sans être agressif pour autant
% Copyright (c) 2014
%   Laurent Claessens
% See the file fdl-1.3.txt for copying conditions.

\begin{corrige}{smath-0597}

    \begin{enumerate}
        \item
            Le bûcheron peut au mieux essayer de le mettre en diagonale; la diagonale (le segment \( [DF]\) du dessin plus bas) du camion fait selon Pythagore :
            \begin{equation}
                \sqrt{13^2+5^2}=\sqrt{194}\simeq \unit{13.9}{\meter}.
            \end{equation}
            Il ne peut donc pas y rentrer le tronc d'arbre de \unit{15}{\meter}.
        \item
            En essayant de placer le tronc le long de la \emph{grande} diagonale :
            \begin{center}
   \input{Fig_WTVlzUE.pstricks}
            \end{center}
            Dans le triangle \( DFG\) nous savons que \( GF=\unit{6}{\meter}\) et \( DG=\unit{\sqrt{196}}{\meter}\). Donc
            \begin{equation}
                DF^2=DG^2+GF^2=194+36=230.
            \end{equation}
            La grande diagonale du camion a donc une longueur de \( \sqrt{230}\simeq\unit{15.16}{\meter}\). Le tronc y rentre donc sans devoir couper.
    \end{enumerate}

\end{corrige}
