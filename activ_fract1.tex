% This is part of Un soupçon de mathématique sans être agressif pour autant
% Copyright (c) 2014
%   Laurent Claessens
% See the file fdl-1.3.txt for copying conditions.

%--------------------------------------------------------------------------------------------------------------------------- 
\subsection*{Partage d'un carré}
%---------------------------------------------------------------------------------------------------------------------------

\begin{wrapfigure}[2]{r}{4.0cm}
   \vspace{-0.5cm}        % à adapter.
   \centering
   \input{Fig_LXQooZPbZml.pstricks}
\end{wrapfigure}

Répondre aux questions à partir du dessin ci-contre.
\begin{enumerate}
    \item
                L'aire de la région hachurée représente \( \dfrac{ 1 }{ \ldots }\) de l'aire totale.
            \item
                L'aire de la région remplie représente \( \dfrac{ 3 }{ \ldots }\) de l'aire totale.
    \item
        Ensemble, ces deux régions forment \( \dfrac{ \ldots }{ \ldots }\) de l'aire totale.
\end{enumerate}
