% This is part of Un soupçon de mathématique sans être agressif pour autant
% Copyright (c) 2013
%   Laurent Claessens
% See the file fdl-1.3.txt for copying conditions.

% Ce fichier est celui pour les secondes.


%+++++++++++++++++++++++++++++++++++++++++++++++++++++++++++++++++++++++++++++++++++++++++++++++++++++++++++++++++++++++++++ 
\section{Comptons des petites boules}
%+++++++++++++++++++++++++++++++++++++++++++++++++++++++++++++++++++++++++++++++++++++++++++++++++++++++++++++++++++++++++++

% This is part of Un soupçon de mathématique sans être agressif pour autant
% Copyright (c) 2013
%   Laurent Claessens
% See the file fdl-1.3.txt for copying conditions.

La classe est divisée en groupes, chacun contenant des perles turquoises et des perles blanches. Les sacs sont identiques mais inconnus. Le but de l'activité est de déterminer ce contenu sans compter toutes les perles. Voici le protocole :
\begin{itemize}
    \item Tirer une perle au hasard dans le sac,
    \item Noter sa couleur.
    \item La remettre dans le sac.
\end{itemize}
Nous parlons de tirage \defe{avec remise}{avec remise!tirage}.

%--------------------------------------------------------------------------------------------------------------------------- 
\subsection*{Après 10 tirages}
%---------------------------------------------------------------------------------------------------------------------------

Reporter les résultats de vos tirages en notant T ou B dans le tableau suivant :

\begin{tabular}[]{|c|c|c|c|c|c|c|c|c|}
    \hline
    &&&&&&&&\\
    \hline
\end{tabular}

Ce tirage forme un \defe{échantillon}{échantillon} du contenu du sac.
\begin{enumerate}
    \item
        Quelle est la fréquence d'apparition de la couleur turquoise ?
     \item
        Reporter les fréquences des autres groupes :
        \begin{tabular}[]{|c|c|c|c|c|c|c|c|c|c|}
        \hline
        &&&&&&&&\\
        \hline
    \end{tabular}
\item
    Que constate-t-on ?
\end{enumerate}

%--------------------------------------------------------------------------------------------------------------------------- 
\subsection*{Après 50 tirages}
%---------------------------------------------------------------------------------------------------------------------------

Effectuer encore \( 40\) tirages.
\begin{enumerate}
    \item
        Calculer la fréquence d'apparition de la couleur turquoise.
    \item
        Compléter le tableau de fréquences d'apparition du turquoise suivant :
        \begin{equation*}
            \begin{array}[]{|c|c|c|c|}
                \hline
                &\text{échantillon de taille 10}&\text{échantillon de taille 40}&\text{Échantillon de taille 50}\\
                  \hline
                  \text{Fréq.}&&&\\ 
                  \hline 
                   \end{array}
               \end{equation*}
               
\end{enumerate}

%--------------------------------------------------------------------------------------------------------------------------- 
\subsection*{Après 100 tirages}
%---------------------------------------------------------------------------------------------------------------------------

Effectuer \( 50\) nouveaux tirages et reporter les résultats :

\begin{equation*}
    \begin{array}[]{|c|c|c|c|}
      \hline
        &\text{Taille 10}&\text{Taille 50}&\text{Taille 100}\\
       \hline
       \text{Fréq.}&&&\\ 
       \hline 
  \end{array}
\end{equation*}
               
Écrire les résultats des autres groupes :
        \begin{tabular}[]{|c|c|c|c|c|c|c|c|c|c|}
        \hline
        &&&&&&&&\\
        \hline
    \end{tabular}

Combien de tirages ont été faits en tout dans la classe ? Quelle est la fréquence observée des boules turquoises ?



%+++++++++++++++++++++++++++++++++++++++++++++++++++++++++++++++++++++++++++++++++++++++++++++++++++++++++++++++++++++++++++ 
\section{Intervalle de confiance}
%+++++++++++++++++++++++++++++++++++++++++++++++++++++++++++++++++++++++++++++++++++++++++++++++++++++++++++++++++++++++++++

Nous avons vu que plus la taille de l'échantillon était grande, plus la fréquence estimée était précise. Précisons le mode opératoire pour fixer le vocabulaire :
\begin{Aretenir}
    Nous avons une \defe{population}{population} de perles à étudier. La \defe{proportion}{proportion} des perles turquoises est inconnue, mais en extrayant un \defe{échantillon}{échantillon} nous pouvons calculer une \defe{fréquence}{fréquence} qui sera une approximation de la proportion.
\end{Aretenir}

Le résultat suivant donne une version précise de la phrase «Plus l'échantillon est grand, plus l'approximation est bonne».
\begin{Aretenir}
    Nous mesurons la fréquence \( f\) d'apparition d'un caractère dans un échantillon de taille \( n\) issu d'une population dont la proportion (inconnue) d'apparition du caractère est \( p\). Alors il y a \( 95\%\) de chances que 
    \begin{equation}
        p\in \mathopen[ f-\frac{1}{ \sqrt{n} } , f+\frac{1}{ \sqrt{n} } \mathclose].
    \end{equation}
    Cet intervalle est l'\defe{intervalle de confiance}{intervalle!de confiance} de \( p\) au seuil \( 95\%\).
\end{Aretenir}

\begin{example}
Lors d’une élection, un sondage portant sur un échantillon aléatoire de $1000$ personnes donne $400$ votants en faveur d’un candidat \( L\). Au risque d’erreur de 5\%, quelle information peut-on obtenir sur la proportion réelle d’électeurs envisageant de voter pour $L$ ?    

La fréquence du vote pour le candidat \( L\) dans l'échantillon est de \( f=0.4\), et la taille de l'échantillon est \( n=400\). Donc il y a une probabilité \( 0.95\) que la proportion réelle de votants pour \( L\) soit dans
\begin{equation}
    \mathopen[ f-\frac{1}{ \sqrt{n} } , f+\frac{1}{ \sqrt{n} } \mathclose]=\mathopen[ 0.4-0.05 , 0.5+0.05 \mathclose]=\mathopen[ 0.35 , 0.55 \mathclose].
\end{equation}
Ce que dit ce résultat est que il devrait y avoir entre \( 35\%\) et \( 55\%\) de votants pour le candidat \( L\).

Autrement dit, si le résultat de l'élection donne moins de \( 35\%\) ou plus de \( 55\%\) au candidat \( L\), nous pouvons dire qu'il n'y a que \( 5\%\) de chances que le résultat soit dû au hasard. Il y a une probabilité \( 0.95\) que soit le sondage avait été mal fait, soit qu'un événement imprévu ait changé des votes au dernier moment.

\end{example}

%+++++++++++++++++++++++++++++++++++++++++++++++++++++++++++++++++++++++++++++++++++++++++++++++++++++++++++++++++++++++++++ 
\section{Intervalle de fluctuation}
%+++++++++++++++++++++++++++++++++++++++++++++++++++++++++++++++++++++++++++++++++++++++++++++++++++++++++++++++++++++++++++

Nous nous intéressons maintenant à un problème un peu différent.

Nous avons une population dont nous étudions un caractère pour lequel nous croyons déjà savoir la fréquence. Nous voulons confirmer l'idée en analysant un échantillon.

\begin{example}
    Un charlatan(?) prétend avoir le pouvoir lire à travers des cartes. Pour vérifier ses dires, on construit un jeu de cartes de façon à ce que chaque carte représente soit un rond, soit un carré soit un triangle.

    On lui présente une centaine de ces cartes successivement, face cachée et on lui demande si elle représente un rond, un carré ou un triangle.

    Si, comme le pensent certains sceptiques, la personne se contente de répondre au hasard alors il devrait répondre correctement environ \( 33\) fois sur la centaine d'essais (une fois sur trois). Nous observons \( 37\) réussites. Est-ce qu'on peut dire que la personne a un réel pouvoir ?

\end{example}

Nous ne savons pas quelle est la proportion réelle de réussite de la personne, mais sa fréquence de réussite est de \( 0.37\) sur \( 100\) essais. Donc au seuil de \( 95\%\) nous pouvons dire que sa proportion de réussite est comprise dans
\begin{equation}
    \mathopen[ 0.37-\frac{1}{ \sqrt{100} } ;0.37+\frac{1}{ \sqrt{100} } \mathclose]=\mathopen[ 0.27 ;0.47 \mathclose].
\end{equation}
Ce résultat est très compatible avec la proportion de réussite attendue en cas de réponse au hasard (parce que \( 0.33\) est dans l'intervalle).

Le résultat-clef est le suivant.
\begin{Aretenir}
    Soit une population dans laquelle une proportion \( p\) d'individus présentent une certaine caractéristique. Soit aussi un échantillon de taille \( n\) prélevé au hasard dans cette population. Nous notons \( f\) la fréquence des individus de l'échantillon présentant la caractéristique. 

    Si \( p\) est entre \( 0.2\) et \( 0.8\) et si \( n\geq 30\), alors il y a \( 95\%\) de chances que 
    \begin{equation}
        f\in\mathopen[ p-\frac{1}{ \sqrt{n} } ; p+\frac{1}{ \sqrt{n} } \mathclose].
    \end{equation}
    Cet intervalle est appelé \defe{intervalle de fluctuation}{intervalle!de fluctuation} de l'échantillon.
\end{Aretenir}

Cet intervalle peut être justifié à partir de l'autre :
\begin{equation}
\xymatrix{%
    &   p\in\mathopen[ f-\frac{1}{ \sqrt{n}} , f+\frac{1}{ \sqrt{n} } \mathclose]\ar[rd]\ar[ld] &  \\
    p\geq f-\frac{1}{ \sqrt{n}}\ar[d]&&p\leq f+\frac{1}{ \sqrt{n}}  \ar[d]\\
    f\leq p+\frac{1}{ \sqrt{n}}\ar[rd] && p-\frac{1}{ \sqrt{n}}\leq f\ar[ld]\\
    & f\in\mathopen[ p-\frac{1}{ \sqrt{n} } , p+\frac{1}{ \sqrt{n}} \mathclose]  &
   }
\end{equation}


\begin{example}
    Si \( 25\%\) de la population a des yeux bleus et qu'on pêche \( 500\) personnes au hasard, il y a environ \( 95\%\) de chances que parmi ces \( 500\) la proportion d'yeux bleus sera comprise dans l'intervalle
    \begin{equation}
        \mathopen[ 0.25-\frac{1}{ \sqrt{500} } ; 0.25+\frac{1}{ \sqrt{500} } \mathclose]\simeq\mathopen[ 0.25-0.045 ; 0.25+0.045 \mathclose]=\mathopen[ 0.205 ; 0.294 \mathclose].
    \end{equation}
    En termes de nombre de personnes, cela signifie qu'on aura entre \( 102\) et \( 148\) personnes à yeux bleus parmi les \( 500\) personnes tirées au hasard.

    Il n'y a que \( 2.5\%\) de chance d'obtenir moins de \( 102\) personnes à yeux bleus sur \( 500\), et environ \( 2.5\%\) de chances d'en avoir plus de \( 148\).

    Donc si on tire un échantillon de \( 500\) personnes et qu'on y voit $170$ avec des yeux bleus, on peut se dire trois choses :
    \begin{itemize}
        \item Le sondage a été mal fait, on n'a pas bien tiré les personnes au hasard, on a mal compté, \ldots
        \item Dans la population, en fait il n'y a pas vraiment \( 25\%\) de personnes aux yeux bleus, mais plus.
        \item On a été victime du hasard, et en réalité il n'y a aucune faute; on avait moins de deux chances et demi sur \( 100\) que ça arrive, mais c'est toujours possible.
    \end{itemize}
    Notons que \( 5\) fois sur \( 100\), un sondage donnera une proportion non conforme à ce à quoi on s'attend. Cela fait environ une fois sur \( 20\).
\end{example}
