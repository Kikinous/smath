% This is part of Un soupçon de mathématique sans être agressif pour autant
% Copyright (c) 2012
%   Laurent Claessens
% See the file fdl-1.3.txt for copying conditions.

\begin{exercice}\label{exoPremiere-0031}

    Parmi les nombres \( -2,-1,0,1,2\), lesquels sont racines de \( x^2-x-2\) ?

    Quelle valeurs \( A\) et \( B\) doivent prendre pour que l'égalité suivante soit satisfaite ?
    \begin{equation}
        x^2-x-2=(x+A)(x+B).
    \end{equation}
    Note : \( A\) et \( B\) pourraient être négatifs des nombres négatifs.


\corrref{Premiere-0031}
\end{exercice}
