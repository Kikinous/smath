% This is part of Un soupçon de mathématique sans être agressif pour autant
% Copyright (c) 2013
%   Laurent Claessens
% See the file fdl-1.3.txt for copying conditions.

\begin{exercice}\label{exosmath-0491}

    À l'aide d'un petit dessin, dire si les couples de fonctions suivants sont représentés par des droites parallèles ou non.
    \begin{enumerate}
        \item
            \( f\colon x\mapsto 2x+6\) et \( g\colon x\mapsto 4x\).
        \item
            \( r\colon x\mapsto 2x+3\) et \( g\colon x\mapsto 2x-2\).
        \item
            \( h\colon x\mapsto -7x+9\) et \( k\colon x\mapsto -9x+6\).
        \item
            \( l\colon x\mapsto -3x+6\) et \( q\colon x\mapsto -3x-2\).
    \end{enumerate}

\corrref{smath-0491}
\end{exercice}
