% This is part of Un soupçon de mathématique sans être agressif pour autant
% Copyright (c) 2015
%   Laurent Claessens
% See the file fdl-1.3.txt for copying conditions.

\begin{exercice}[\cite{NRHooXFvgpp5}]\label{exo2smath-0308}

Lors de vendanges, chaque tombereau est pesé à la cave coopérative avant d'être déversée dans les cuves à raisins. Voila ce qu'à relevé le caviste le premier jour (en kilogrammes) :

\begin{equation*}
    \begin{array}[]{cccccc}
        740&1243&827&327&977&352\\
        685&1025&1221&690&475&605\\
        401&893&799&723&469&552\\
        717&985&799&581&787&898\\
        361&963&1213&752&804&605\\
        293&473&677&313&520&732\\
        963&522&1209&993&928&547
    \end{array}
\end{equation*}

Regrouper ces données en quatre classes même amplitude et réaliser l'histogramme correspondant.

\corrref{2smath-0308}
\end{exercice}
