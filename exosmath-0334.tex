% This is part of Un soupçon de mathématique sans être agressif pour autant
% Copyright (c) 2013-2014
%   Laurent Claessens
% See the file fdl-1.3.txt for copying conditions.

\begin{exercice}\label{exosmath-0334}

    Associer les fonctions suivantes au graphes.

\begin{center}
   \input{Fig_RWZooOBMHZ.pstricks}
\end{center}

\let\Oldtheenumi\theenumi
\renewcommand{\theenumi}{(\alph{enumi})}
\begin{multicols}{2}
    \begin{enumerate}
        \item
            \( f_a(x)=\frac{ 0.5x+1 }{ x-1 }\)
        \item
            \( f_b(x)=\frac{2-x}{ x-1 }\)
        \item
            \( f_c(x)=\frac{ 1 }{ x }\)
        \item
            \( f_d(x)=\frac{1}{ x+2 }\)
    \end{enumerate}
\end{multicols}
\let\theenumi\Oldtheenumi
Il y a deux branches par fonction : une croissante et une décroissante.

\corrref{smath-0334}
\end{exercice}
