% This is part of Un soupçon de mathématique sans être agressif pour autant
% Copyright (c) 2013
%   Laurent Claessens
% See the file fdl-1.3.txt for copying conditions.

\begin{exercice}\label{exosmath-0334}

    Associer les fonctions suivantes au graphes de la figure \ref{LabelFigOHfdEZt} :
    \newcommand{\CaptionFigOHfdEZt}{Les graphes de l'exercice \ref{exosmath-0334}.}
\input{Fig_OHfdEZt.pstricks}

\begin{multicols}{3}
    \begin{enumerate}
        \item
            \( f(x)=\frac{1}{ x+2 }\)
        \item
            \( g(x)=\frac{ 2x+3 }{ x-1 }\)
        \item
            \( h(x)=\frac{ 1-x }{ 1+x }\)
        \item
            \( k(x)=\frac{ 1 }{ x }\)
    \end{enumerate}
\end{multicols}

%AFAIRE : continuer, surtout mettre bien les graphiques.

\corrref{smath-0334}
\end{exercice}
