% This is part of Un soupçon de mathématique sans être agressif pour autant
% Copyright (c) 2014
%   Laurent Claessens
% See the file fdl-1.3.txt for copying conditions.

\begin{corrige}{smath-0953}

    Le côté le plus long est \( [QR]\). Donc si le triangle est rectangle, il doit être rectangle en \( P\). D'après le théorème de Pythagore, \( QRP\) sera rectangle si 
    \begin{equation}
        QR^2=QP^2+RP^2,
    \end{equation}
    et ne sera pas rectangle sinon. Nous calculons :
    \begin{equation}
        QR^2=289
    \end{equation}
    et
    \begin{equation}
        QP^2+RP^2=64+225=289.
    \end{equation}
    La réciproque du théorème de Pythagore nous indique alors que le triangle \( QPR\) est rectangle.

\end{corrige}
