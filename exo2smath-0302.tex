% This is part of Un soupçon de mathématique sans être agressif pour autant
% Copyright (c) 2015
%   Laurent Claessens
% See the file fdl-1.3.txt for copying conditions.

\begin{exercice}[\cite{NRHooXFvgpp5}]\label{exo2smath-0302}


Voici le relevé des quatre tarifs appliqués aux visiteurs de la Tour Eiffel au cours de la première heure d'un jour donné.

\begin{center}
\begin{tabular}[]{|c|c|c|c|c|}
    \hline
    Tarif&Adulte&Enfant&Étudiant&Groupes\\
    \hline
    Fréquence&\( 0.45\)&&\( 0.1\)&0.2\\
    \hline
\end{tabular}
    
\end{center}

\begin{enumerate}
    \item
        
 Reproduire et compléter ce tableau.
 \item
Ajouter une ligne pour indiquer la fréquence en pourcentage.
\item
 Ajoute une nouvelle ligne et calcule l'effectif de chaque catégorie sachant qu'il y a eu $1 700$ visiteurs au total.
\end{enumerate}

\corrref{2smath-0302}
\end{exercice}
