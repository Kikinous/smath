% This is part of Un soupçon de mathématique sans être agressif pour autant
% Copyright (c) 2014
%   Laurent Claessens
% See the file fdl-1.3.txt for copying conditions.

\begin{exercice}[\cite{LEPOooBzqKhm}]\label{exosmath-0993}

    Soit un triangle \( ABC\) rectangle en \( A\). Nous nommons \( I\), \( J\) et \( K\) les milieux respectifs des côtés \( [BC]\), \( [AC]\) et \( [AB]\).
    \begin{enumerate}
        \item
            Faire un dessin.
        \item
            Que peut-on dire des droites \( (IJ)\) et \( (AB)\) ?
        \item
            Que peut-on dire des droites \( (AC)\) et \( (IJ)\) ?
        \item
            Quelle est la nature du quadrilatère \( AJIK\) ?
    \end{enumerate}

\corrref{smath-0993}
\end{exercice}
