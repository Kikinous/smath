% This is part of Un soupçon de mathématique sans être agressif pour autant
% Copyright (c) 2014
%   Laurent Claessens
% See the file fdl-1.3.txt for copying conditions.

\begin{exercice}[\ldots/3]\label{exosmath-0599}

    \begin{multicols}{2}

    Le programme suivant sert à vérifier si la moyenne d'un élève à 3 devoirs est plus grande que \( 10/20\) :

    \begin{fmpage}{0.9\linewidth}

    Écrire « Première note :» ; demander \( a\)

    Écrire « Seconde note :»;   demander \( b\)

    Écrire « Troisième note :»; demander \( c\)

    Si \ldots\ldots\ldots\ldots\ldots\ldots\ldots\ldots\ldots\ldots\ldots , alors :

    \hspace{0.5cm} Écrire « Moyenne plus grande que \( 10/20\)» 

    Sinon :

    \hspace{0.5cm} Écrire « Moyenne plus petite que \( 10/20\)» 

\end{fmpage}

\begin{enumerate}
    \item
        Compléter les pointillés. 
    \item
        Qu'affiche le programme \emph{tel que vous l'avez complété} si on entre les valeurs \( a=10\), \( b=8\) et \( c=12\) ?        
\end{enumerate}

\columnbreak

    Alice veut se faire un aide mémoire qui lui donne le volume d'un cylindre en fonction du rayon de la base et de sa hauteur :

    \begin{fmpage}{0.9\linewidth}

    Écrire «Combien vaut \ldots\ldots\ldots\ldots ?»; Demander \ldots\ldots\ldots

    Écrire «Combien vaut \ldots\ldots\ldots\ldots ?»; Demander \ldots\ldots\ldots

    \( V=\cdots\cdots\cdots\cdots\cdots\cdots\)

    Écrire \( V\)

\end{fmpage}

Compléter les pointillés.

    \end{multicols}

\corrref{smath-0599}
\end{exercice}
