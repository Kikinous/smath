% This is part of Un soupçon de mathématique sans être agressif pour autant
% Copyright (c) 2012-2014
%   Laurent Claessens
% See the file fdl-1.3.txt for copying conditions.


% Les DS sont à faire dans DSs.tex pour être compilés avec lst_DS.py




%\documentclass[a4paper,10pt]{book}
\documentclass[a4paper,10pt]{article}
\usepackage{multido}
% This is part of Un soupçon de mathématique sans être agressif pour autant
% Copyright (c) 2012-2013
%   Laurent Claessens
% See the file fdl-1.3.txt for copying conditions.


\usepackage{etex}
\usepackage{ifthen}
%\usepackage{pdfsync}       % This package is obsolete : compile with pdflatex -synctex=1 instead.

\usepackage{latexsym}
\usepackage{amsfonts}
\usepackage{amsmath}
\usepackage{amsthm}
\usepackage{amssymb}
\usepackage{bbm}
\usepackage{mathrsfs}           
\usepackage{mathabx}           % Pour \divides

\usepackage{framed}

\usepackage{calc}   % Les dépendances de phystricks si on n'utilise que le pdf.
%\usepackage{pstricks,pst-eucl,pstricks-add,calc,pst-math}   % Les dépendances de phystricks. Peut être qu'il faut ajouter catchfile
\usepackage{graphicx}                   % Pour l'inclusion d'image en pfd.

\newcommand{\EpsOrPdfincludegraphics}[2][]{%
        \ifpdf
            \includegraphics[#1]{#2.png}
        \else
            \includegraphics[#1]{#2.eps}
        \fi
        }

\usepackage{subfigure}

\usepackage{fancyvrb}
\usepackage{stmaryrd}       % Pour le \obslash
\usepackage{xstring}        % Utilisé pour les références vers wikipédia
\usepackage{cases}
\usepackage{lscape}         % pour l'environnement landscape, utilisé dans la correction corr0076.tex
\usepackage{multicol}
\usepackage{import}         % Pour le hack qui sert à inclure GeomAnal

% TODO : n'en utiliser qu'un
\usepackage[normalem]{ulem}		% Pour le barré, commande \sout
\usepackage{soul}		% Pour le barré, commande \st

\usepackage[all]{xy}

\let\second\undefined      % le paquet amthabx définit \second
\let\degree\undefined       % le paquet amthabx définit \degree
\usepackage[cdot,thinqspace,amssymb]{SIunits} 
 % L'option amssymb sert à éviter un conflit avec la commande \square de amssymb. Note qu'elle n'est plus accessible. Si tu en as besoin, faudra RTFM
%ftp://ftp.belnet.be/packages/ctan/macros/latex/contrib/SIunits/SIunits.pdf

\usepackage[nottoc]{tocbibind}

%%%%%%%%%%%%%%%%%%%%%%%%%%
%
%   Trucs mathématiques
%
%%%%%%%%%%%%%%%%%%%%%%%%

% ENSEMBLES DE NOMBRES
\newcommand{\eA}{\mathbbm{A}}
\newcommand{\eC}{\mathbbm{C}}
\newcommand{\eD}{\mathbbm{D}}
\newcommand{\eE}{\mathbbm{E}}
\newcommand{\eF}{\mathbbm{F}}
\newcommand{\eG}{\mathbbm{G}}
\newcommand{\eH}{\mathbbm{H}}
\newcommand{\eK}{\mathbbm{K}}
\newcommand{\eL}{\mathbbm{L}}
\newcommand{\eM}{\mathbbm{M}}
\newcommand{\eN}{\mathbbm{N}}
\newcommand{\eP}{\mathbbm{P}}
\newcommand{\eQ}{\mathbbm{Q}}
\newcommand{\eR}{\mathbbm{R}}
\newcommand{\eZ}{\mathbbm{Z}}

% ENSEMBLES de fonctions
\newcommand{\aL}{\mathcal{L}}       % Les applications linéaires
\newcommand{\aC}{\mathcal{C}}       % Les fonctions C^1, C^2 etc

% AUTRES
\newcommand{\sdS}{\mathcal{S}}      % L'ensemble des subdivisions d'un intervalle.



\newcommand{\mF}{\mathcal{F}}
\newcommand{\mC}{\mathcal{C}}
\newcommand{\mG}{\mathcal{G}}
\newcommand{\mI}{\mathcal{I}}
\newcommand{\mL}{\mathcal{L}}
\newcommand{\mS}{\mathcal{S}}   % Utilisé pour l'espace des fonctions Schwartz
\newcommand{\mZ}{\mathcal{Z}}


\newcommand{\mtu}{\mathbbm{1}}              % La matrice unité
\newcommand{\caract}{\mathbbm{1}}    % Characteristic function of a set

\DeclareMathOperator{\val}{val}     % valuation d'un polynôme


%\newcommand{\efrac}[2]{\frac{ \displaystyle #1 }{\displaystyle #2 }}
%%%%%%%%%%%%%%%%%%%%%%%%%%
%
%   Numérotations en tout genre
%
%%%%%%%%%%%%%%%%%%%%%%%%

\setcounter{tocdepth}{2}        % Profondeur de la table des matières
\setcounter{secnumdepth}{2}     % Profondeur dans le texte

%%%%%%%%%%%%%%%%%%%%%%%%%%
%
%   Les lignes magiques pour le texte en français.
%
%%%%%%%%%%%%%%%%%%%%%%%%

\usepackage[utf8]{inputenc}
\usepackage[T1]{fontenc}

\usepackage{listingsutf8}
\lstset{language=python,basicstyle=\footnotesize,tabsize=3,numbers=left,numberstyle=\tiny,frame=single,commentstyle=\ttfamily\color[rgb]{0,0,0.5},stringstyle=\color[rgb]{0,0.5,0},title=\lstname,inputencoding=utf8/latin1}

\usepackage[fr]{exocorr}
\usepackage{textcomp}
\usepackage{lmodern}
\usepackage[a4paper,margin=2cm]{geometry} 
\usepackage[english,frenchb]{babel}


\usepackage{hyperref}                           %Doit être appelé en dernier.
\hypersetup{
colorlinks=true,
linkcolor=blue,
urlcolor=magenta,     % couleur des url
filecolor=magenta   % couleur des textes qui sont des liens
}

%%%%%%%%%%%%%%%%%%%%%%%%%%
%
%   Les théorèmes et choses attenantes
%
%%%%%%%%%%%%%%%%%%%%%%%%


\newcounter{numtho}
\newcounter{numprob}

\makeatletter
\@addtoreset{numtho}{chapter}
%\@addtoreset{CountExercice}{chapter}
\@addtoreset{chapter}{part}
\makeatother

\newlength{\EnvSpace}
\setlength{\EnvSpace}{9pt}      % C'est la distance que je veux mettre avant et après les théorèmes, remarques, etc.

\newtheoremstyle{MyTheorems}%
        {\EnvSpace}{\EnvSpace}%
        {\itshape}%
        {}%
        {\bfseries}{.}%
        {\newline}%
        {}%
\newtheoremstyle{MyExamples}%
        {\EnvSpace}{\EnvSpace}%
        {}%
        {}%
        {\bfseries}{.}%
        {\newline}%
        {}%
\newtheoremstyle{MyRemarks}%
        {\EnvSpace}{\EnvSpace}%
        {}%
        {}%
        {\bfseries}{.}%
        {\newline}%
        {}%

%\theoremstyle{MyExamples}   %\newtheorem{exemple}[numtho]{Exemple}      % Pour unification, ne plus utiliser
%                            \newtheorem{example}[numtho]{Exemple}
\newcounter{CounterExample}
\renewcommand{\theCounterExample}{\thechapter.\arabic{CounterExample}}

\newenvironment{example}{\vspace{\EnvSpace}\refstepcounter{numtho}\noindent{\bf Exemple \thenumtho}\newline}{\phantom{a}\hfill $\triangle$\vspace{\EnvSpace}}
\newenvironment{Aretenir}{\refstepcounter{numtho}\begin{oframed}\noindent{\bf À retenir \thenumtho}\newline}{\end{oframed}\vspace{\EnvSpace}}
\newenvironment{Enmini}{\begin{oframed}\noindent{\bf Mini résumé}\newline}{\end{oframed}\vspace{\EnvSpace}}
\newenvironment{definition}{\refstepcounter{numtho}\begin{oframed}\noindent{\bf Définition \thenumtho}\newline}{\end{oframed}\vspace{\EnvSpace}}
\newenvironment{propriete}{\refstepcounter{numtho}\begin{oframed}\noindent{\bf Propriété \thenumtho}\newline}{\end{oframed}\vspace{\EnvSpace}}

\theoremstyle{MyRemarks}    \newtheorem{remark}[numtho]{Remarque}

                \newtheorem{amusement}[numtho]{Amusement}
                \newtheorem{erreur}[numtho]{Error}
                \newtheorem{probleme}[numprob]{\fbox{\bf Problèmes et choses à faire}}


\theoremstyle{MyTheorems}
            \newtheorem{lemma}[numtho]{Lemme}
            \newtheorem{corollary}[numtho]{Corollaire}
            \newtheorem{theorem}[numtho]{Théorème}      
            \newtheorem{proposition}[numtho]{Proposition}      

            %\newtheorem{exo}[CountExercice]{Exercice}       % C'est provisoire, pour Chafaï

\renewcommand{\thenumtho}{\thechapter.\arabic{numtho}}
% La numérotation des équations change dans les corrigés
\renewcommand{\theequation}{\thechapter.\arabic{equation}}
\renewcommand{\theCountExercice}{\arabic{CountExercice}}       % Ce compteur est défini dans SystemeCorr.sty
\newcommand{\defe}[2]{\textbf{#1}\index{#2}}

\renewcommand{\labelenumi}{\theenumi}
\renewcommand{\theenumi}{(\arabic{enumi})}


%%%%%%%%%%%%%%%%%%%%%%%%%%
%
%   Les macros qui font des choses
%
%%%%%%%%%%%%%%%%%%%%%%%%

\newcommand{\mA}{\mathcal{A}}
\newcommand{\mO}{\mathcal{O}}
\newcommand{\mR}{\mathcal{R}}
\newcommand{\mT}{\mathcal{T}}
\newcommand{\mU}{\mathcal{U}}

\newcommand{\scal}[2]{ \langle #1,#2\rangle }

\newcommand{\tq}{\text{ tel que }}
\newcommand{\tqs}{\text{ tels que }}
\newcommand{\quext}[1]{ \footnote{\textsf{#1}}  }
\newcommand{\info}[1]{\texttt{#1}}
\newcommand{\vect}[1]{\overrightarrow{#1}}    % Cette macro est codée en dur dans phystricksDefVecteurAXDDGP et dans d'autres

\newcommand{\VarAbs}{\text{Var}_{\text{abs}}}
\newcommand{\VarRel}{\text{Var}_{\text{rel}}}

\newcommand{\normal}{\lhd}
\newcommand{\swS}{\mathscr{S}}          % L'ensemble des fonctions Schwartz

%\newcommand{\defD}{\mathscr{D}}     % Ensemble de définition d'une fonction
\newcommand{\defD}{D}                % Le D avec des croles était impossible à comprendre pour les élèves.

\newcommand{\Borelien}{\mathcal{B}\text{or}}       % Les boréliens
\newcommand{\tribA}{\mathcal{A}}            % Une tribu A
\newcommand{\tribB}{\mathcal{B}}            
\newcommand{\tribF}{\mathcal{F}}            % Une tribu F

\newcommand{\affE}{\mathcal{E}}            % Un espace affine E
\newcommand{\affF}{\mathcal{F}}            
\newcommand{\affG}{\mathcal{G}}            

\newcommand{\statS}{\mathcal{S}}            % Un modèle statistique
\newcommand{\partP}{\mathcal{P}}            % L'ensemble des parties d'un ensemble

\newcommand{\polyP}{\mathcal{P}}            % L'ensemble des polynômes

\newcommand{\dB}{\mathscr{B}}       % la distribution de Bernoulli
\newcommand{\dE}{\mathscr{E}}       % la distribution exponentielle
\newcommand{\dG}{\mathscr{G}}       % la distribution géométrique.
\newcommand{\dM}{\mathscr{M}}       % la distribution multinomiale
\newcommand{\dN}{\mathscr{N}}       % la distribution normale.
\newcommand{\dP}{\mathscr{P}}       % la distribution de Poisson.
\newcommand{\dT}{\mathscr{T}}       % la distribution de Student
\newcommand{\dU}{\mathscr{U}}       % la distribution uniforme

\newcommand{\hL}{\mathscr{L}}       
\newcommand{\cL}{\hL}           % Pour la partie Chafai

\newcommand{\modE}{\mathcal{E}}         % Le E des modules
\newcommand{\modF}{\mathcal{F}}         % Le F des modules
\newcommand{\hH}{\mathscr{H}}           % Le H des espaces de Hilbert

%%%%%%%%%%%%%%%%%%%%%%%%%%
%
%   Bibliographie, index et liste des notations
%
%%%%%%%%%%%%%%%%%%%%%%%%

\usepackage{makeidx}
\usepackage[nottoc]{tocbibind}      % Le paquetage qui fait en sorte que la biblio soit inclue correctement dans la table des matières.
\usepackage[refpage]{nomencl}
\renewcommand{\nomname}{Liste des notations}
%
%   Comment introduire des éléments dans l'index des notations.
%
% La liste des tags à mettre pour bien classer mes notations est :
% T     pour la topologie et théorie des ensembles
%
% La syntaxe est facile, par exemple 
%       $\SL(2,\eR)$\nomenclature[G]{$\SL(2,\eR)$}{Le groupe de matrices deux par deux de déterminant 1.}
%\renewcommand{\nomgroup}[1]{%
%    \ifthenelse{\equal{#1}{A}}{\item[\textbf{Algèbre}]}{}%
%    \ifthenelse{\equal{#1}{G}}{\item[\textbf{Géométrie}]}{}%
%    \ifthenelse{\equal{#1}{R}}{\item[\textbf{Théorie des groupes}]}{}%
%    \ifthenelse{\equal{#1}{P}}{\item[\textbf{Probabilités et statistique}]}{}%
%    \ifthenelse{\equal{#1}{Y}}{\item[\textbf{Analyse}]}{}%
%    \ifthenelse{\equal{#1}{M}}{\item[\textbf{Chaînes de Markov}]}{}%
%}

%%%%%%%%%%%%%%%%%%%%%%%%%%
%
%   DeclareMathOperator
%
%%%%%%%%%%%%%%%%%%%%%%%%

\DeclareMathOperator{\signe}{sgn}
\DeclareMathOperator{\Vol}{Vol}
\DeclareMathOperator{\Int}{Int}     % Intérieur d'un ensemble.
\DeclareMathOperator{\Ind}{Ind}     % l'indice d'un chemin en analyse complexe
\DeclareMathOperator{\Diam}{Diam}   
\DeclareMathOperator{\id}{Id}   
\DeclareMathOperator{\Graph}{Graph} 
\DeclareMathOperator{\pr}{\texttt{proj}}
\DeclareMathOperator{\dom}{dom}

\DeclareMathOperator{\Graphe}{Gr}
\DeclareMathOperator{\Spec}{Spec}   % spectre d'un opérateur
\DeclareMathOperator{\arctg}{arctg}
\DeclareMathOperator{\cotg}{cotg}
\DeclareMathOperator{\cosec}{cosec}
\DeclareMathOperator{\arcsinh}{arcsinh}

\DeclareMathOperator{\GL}{GL}   % le groupe linéaire
\DeclareMathOperator{\PGL}{PGL}   % le groupe projectif
\DeclareMathOperator{\SO}{SO}           
\DeclareMathOperator{\SL}{SL}           
\DeclareMathOperator{\PSL}{PSL}   % Le groupe modulaire SL(2,Z)/Z2
\DeclareMathOperator{\gO}{O}           
\DeclareMathOperator{\SU}{SU}           
\DeclareMathOperator{\gU}{U}           

\DeclareMathOperator{\Reel}{Re}        % La partie réelle d'un nombre complexe

\DeclareMathOperator{\Image}{Image}        % ... avec \Image qui donne l'image d'une fonction ou d'un opérateur.
\DeclareMathOperator{\rang}{rg}   
\DeclareMathOperator{\Kernel}{Ker}
\DeclareMathOperator{\Domaine}{Dom}
\DeclareMathOperator{\Span}{Span}
\DeclareMathOperator{\Hom}{Hom}
\DeclareMathOperator{\End}{End}     % L'ensemble des endomorphismes
\DeclareMathOperator{\tr}{Tr}       % la trace
\DeclareMathOperator{\Majorant}{Maj}
\DeclareMathOperator{\codim}{codim} % pour la codimension.
\DeclareMathOperator{\diam}{diam} % le diamètre d'un ensemble.

\DeclareMathOperator{\Var}{Var}     % Variance d'une variable aléatoire.
\DeclareMathOperator{\Fun}{\texttt{Fun}}     % Ensemble des applications d'un ensemble vers l'autre.
\DeclareMathOperator{\Cov}{Cov}     % la covariance.
\DeclareMathOperator{\gr}{gr}     % le groupe engendré
\DeclareMathOperator{\pgcd}{pgcd}     
\DeclareMathOperator{\ppcm}{ppcm}     
\DeclareMathOperator{\Frob}{Frob}     
\DeclareMathOperator{\Card}{Card}       % Le cardinal d'un ensemble.
\DeclareMathOperator{\Stab}{Stab}       % Le stabilisateur d'un point sous l'action d'un groupe.

\DeclareMathOperator{\Frac}{Frac}       % le corps des fractions d'un anneau
\DeclareMathOperator{\Aff}{Aff}         %  l'espace affine engendré

\newenvironment{subproof}{\begin{description}}{\end{description}}

%%%%%%%%%%%%% TRUCS DE YVIK POUR FAIRE FONCTIONNER CdI1 %%%%%%%%%%%%%%%%%%%%%%
%

%\newcommand{\proofend}{\hspace*{\fill} $\Box$\\}
%\newcommand{\diam}{\hspace*{\fill} $\Diamond$\\}
%\def\s{\smallskip}
%\def\m{\medskip}
%\def\my{\bf}
\newcommand{\eps}{\varepsilon}
\newcommand{\Ker}{\operatorname{Ker}}
\newcommand{\IM}{\operatorname {Im}}
\newcommand{\cat}{\operatorname{cat}}
\newcommand{\crit}{\operatorname{crit}}
\newcommand{\Crit}{\operatorname{Crit}}
\newcommand{\Rest}{\operatorname{Rest}}
\newcommand{\grad}{\operatorname{grad}}
\newcommand{\sgrad}{\operatorname{sgrad}}
\newcommand{\Fix}{\operatorname{Fix}}
\newcommand{\pt}{\operatorname{pt}}
\newcommand{\cl}{\operatorname{cl}}
\newcommand{\B}{\operatorname {B}}
\newcommand{\C}{\operatorname {C}}
%\newcommand{\S}{\operatorname {S}}
\newcommand{\Gr}{\operatorname {Gr\;\!}}
%\def\dim{\operatorname {dim}}
\newcommand{\inj}{\operatorname {inj}}
%\newcommand{\Vol}{\operatorname {Vol}\:\!}
%\newcommand{\Int}{\operatorname {Int}\:\!}
\newcommand{\dist}{\operatorname {dist}}
%\def\inter{\operatorname {int}}
\newcommand{\ext}{\operatorname {ext}}
%\newcommand{\diameter}{\operatorname {diam}\:\!}
\newcommand{\Emb}{\operatorname {Emb}}
\newcommand{\can}{\operatorname {can}}
\newcommand{\euler}{\mbox{\rm e}}
\newcommand{\sii}{\mbox{\rm \scriptsize i}}
\newcommand{\VB}{\mbox{V}_{\!\!B}}   
\newcommand{\VC}{\mbox{V}_{\!\!C}}   
\newcommand{\VS}{\mbox{V}_{\!\!S}}   
\newcommand{\f}{\frac}
\newcommand{\ga}{\alpha}
\newcommand{\gb}{\beta}
%\newcommand{\gg}{\gamma}
\newcommand{\gd}{\delta}
\newcommand{\gve}{\varepsilon}
\newcommand{\gf}{\varphi}
\newcommand{\gk}{\kappa}
\newcommand{\gkk}{\varkappa}
\newcommand{\gl}{\lambda}
\newcommand{\go}{\omega}
\newcommand{\gs}{\sigma}
\newcommand{\gt}{\vartheta}
\newcommand{\gy}{\upsilon}
\newcommand{\gv}{\varrho}
\newcommand{\gz}{\zeta}
\newcommand{\gD}{\Delta}
\newcommand{\gF}{\Phi}
\newcommand{\gG}{\Gamma}
\newcommand{\gL}{\Lambda}
%\newcommand{\gO}{\Omega}
\newcommand{\gS}{\Sigma}

%\long\def\forget#1\forgotten{} %
%\def\end{center}{{\mathfrak C}}
%\def\ea{{\mathfrak A}}

\newcommand{\ca}{{\mathcal A}}
\newcommand{\cb}{{\mathcal B}}
\newcommand{\cc}{{\mathcal C}}
\newcommand{\cd}{{\mathcal D}}
\newcommand{\ce}{{\mathcal E}}
\newcommand{\cf}{{\mathcal F}}
\newcommand{\cg}{{\mathcal G}}
\newcommand{\ch}{{\mathcal H}}
\newcommand{\cj}{{\mathcal J}}
\newcommand{\ck}{{\mathcal K}}
\newcommand{\cn}{{\mathcal N}}
\newcommand{\co}{{\mathcal O}}
\newcommand{\cp}{{\mathcal P}}
\newcommand{\cq}{{\mathcal Q}}
\newcommand{\cs}{{\mathcal S}}
\newcommand{\ct}{{\mathcal T}}
\newcommand{\cu}{{\mathcal U}}
\newcommand{\cv}{{\mathcal V}}
\newcommand{\cw}{{\mathcal W}}
\newcommand{\eb}{{\mathfrak B}}
\newcommand{\ed}{{\mathfrak D}}
\newcommand{\ee}{{\mathfrak E}}
\newcommand{\ef}{{\mathfrak F}}
\newcommand{\eg}{{\mathfrak G}}
\newcommand{\ej}{{\mathfrak J}}
\newcommand{\eh}{{\mathfrak H}}
\newcommand{\en}{{\mathfrak N}}
\newcommand{\eo}{{\mathfrak O}}
\newcommand{\ep}{{\mathfrak P}}
\newcommand{\eq}{{\mathfrak Q}}
\newcommand{\es}{{\mathfrak S}}
\newcommand{\et}{{\mathfrak T}}
\newcommand{\eu}{{\mathfrak U}}
\newcommand{\ev}{{\mathfrak V}}
\newcommand{\ew}{{\mathfrak W}}



%\def\NN{\mathbbm{N}}
%\def\QQ{\mathbbm{Q}}
%\def\RR{\mathbbm{R}}
%\def\SS{\mathbbm{S}}
%\def\11{\mathbbm{1}}
%\def\ZZ{\mathbbm{Z}}
%\def\TT{\mathbbm{T}}
\newcommand{\RR}{\eR}
\newcommand{\DD}{\mathbbm{D}}
\newcommand{\HH}{\mathbbm{H}}
\newcommand{\II}{\mathbbm{I}}
\newcommand{\N}{\mathbbm{N}}
\newcommand{\PP}{\mathbbm{P}}
\newcommand{\Q}{\mathbbm{Q}}
\newcommand{\RRR}{\mathbbm{R}_+}
\newcommand{\Z}{\mathbbm{Z}}
\newcommand{\RP}{{\RR\PP}} 
%\newcommand{\CP}{{\CC\PP}} 
\newcommand{\pp}{\partial}
\newcommand{\ww}{\wedge}
%\newcommand{\dc}{d^\CC}
\newcommand{\sym}{Sp(n;\RR)}
\newcommand{\ha}{\hookrightarrow}
\newcommand{\Ra}{\Rightarrow}
\newcommand{\Lra}{\Leftrightarrow} 

%\def\ni{\noindent}
%\def\b{\bigskip}
%\def\m{\medskip}
%\def\im{\mbox{Im}\,}

\newcommand{\de}{\stackrel{\mbox{\scriptsize{def}}}{=}}
%\newcommand{\id}{\mbox{id}}

%\def\sq{\square}
%\def\tr{\triangle}
%\def\trd{\bigtriangledown}
%\def\proof{\noindent {\it Proof. \;}}


%	La num\'erotation des exercices


\newcounter{exoNico}
\setcounter{exoNico}{1}
\newcommand{\exerNico}{\stepcounter{exoNico}{\bf Exercice }\arabic{exoNico}. }


%++++++++++ACCENTS++++++++++++++++++
\newcommand{\e}{\'{e}}
%\newcommand{\esp}{\'{e }}
%\newcommand{\eg}{\`{e}}
\newcommand{\ac}{\`{a} }
%\newcommand{\meme}{m\^{e}me }
\newcommand{\ou}{o\`{u} }

%+++++++++++NEWCOMMANDS+++++++++++
\newcommand{\dst}{\displaystyle}
\newcommand{\ba}{\begin{array}}
%\newcommand{\ea}{\end{array}}
%++++++++++FORMULAS+++++++++++++
\newcommand{\hs}{\hspace{0.3cm}}
%\newcommand{\eps}{\epsilon}
%\newcommand{\f}{\frac}
\newcommand{\arcth}{{\rm arctanh}}
\newcommand{\arcsh}{{\rm arcsinh}}
\newcommand{\arcch}{{\rm arccosh}}
\newcommand{\csec}{{\rm cosec}}
\newcommand{\cotan}{{\rm cotg}}
\newcommand{\cis}{(\cos+i\sin)( }
%\newcommand{\ra}{\rightarrow}
\newcommand{\lra}{\longrightarrow}
\newcommand{\ceil}{\rm plafond(}
\newcommand{\dfdu}{\frac{\partial f}{\partial u}}
\newcommand{\dfdw}{\frac{\partial f}{\partial w}}
\newcommand{\dfdx}{\frac{\partial f}{\partial x}}
\newcommand{\dfdy}{\frac{\partial f}{\partial y}}
\newcommand{\dudx}{\frac{\partial u}{\partial x}}
\newcommand{\dvdx}{\frac{\partial v}{\partial x}}
\newcommand{\dUdx}{\dfrac{\partial U}{\partial x}}
\newcommand{\dVdx}{\dfrac{\partial V}{\partial x}}
\newcommand{\dhdx}{\frac{\partial h}{\partial x}}
\newcommand{\dhdy}{\frac{\partial h}{\partial y}}
\newcommand{\dgdu}{\frac{\partial g}{\partial u}}
\newcommand{\dgdv}{\frac{\partial g}{\partial v}}
\newcommand{\dgudu}{\frac{\partial g_1}{\partial u}}
\newcommand{\dgudv}{\frac{\partial g_1}{\partial v}}
\newcommand{\dgddu}{\frac{\partial g_2}{\partial u}}
\newcommand{\dgddv}{\frac{\partial g_2}{\partial v}}
\newcommand{\dhdu}{\frac{\partial h}{\partial u}}
\newcommand{\dhdv}{\frac{\partial h}{\partial v}}
\newcommand{\dldu}{\frac{\partial l}{\partial u}}
\newcommand{\dldv}{\frac{\partial l}{\partial v}}
\newcommand{\dgudr}{\frac{\partial g_1}{\partial r}}
\newcommand{\dgudth}{\frac{\partial g_1}{\partial \theta}}
\newcommand{\dgddr}{\frac{\partial g_2}{\partial r}}
\newcommand{\dgddth}{\frac{\partial g_2}{\partial \theta}}
\newcommand{\dfdv}{\frac{\partial f}{\partial v}}
\newcommand{\dfdr}{\frac{\partial f}{\partial r}}

\newcommand{\dfdth}{\frac{\partial f}{\partial \theta}}
\newcommand{\ddfdx}{\frac{\partial^2 f}{\partial x^2}}
\newcommand{\ddfdy}{\frac{\partial^2 f}{\partial y^2}}
\newcommand{\ddfdxy}{\frac{\partial^2 f}{\partial y\partial x}}
\newcommand{\ddfdt}{\frac{\partial^2 f}{\partial^2 t}}

\newcommand{\ud}{\underline}

 % *** Blackboard math symbols ***
 %\newcommand{\N}{\mathbb{N}}
 %\newcommand{\Z}{\mathbb{Z}}
 %\newcommand{\Q}{\mathbb{R}}
 %\newcommand{\K}{\mathbb{K}}
 %\newcommand{\R}{\mathbb{R}}
 %\newcommand{\C}{\mathbb{C}}
 %\newcommand{\F}{\mathbb{F}}
 %\newcommand{\J}{\mathbb{J}}
\newcommand{\Qn}{\mathbb{Q}}

\newcommand{\Rn}{\eR} 
\newcommand{\Nn}{\eN}


\newtheorem{theo}{Th{\'e}or{\`e}me}[section]
\newtheorem{defn}{D{\'e}finition}
\newtheorem{prop}{Proposition}     % redef encore dans Chafaï
%\newtheorem{rem}{Remarque}[section]
\newtheorem{lem}{Lemme}[section]
\newcommand{\R}{\mathbb{R}}
\newcommand{\dem}{\textbf{D{\'e}monstration.}}
\newcommand{\vc}[1]{\boldsymbol{#1}}
\newcommand{\p}{\textrm{P}}
%\newcommand{\e}{\textrm{E}}
\newcommand{\mbt}{arbre binaire markovien}
\newcommand{\mbts}{arbres binaires markoviens}

\newcommand{\ea}{\end{array}}


%%%%%%%%%%%%%%%%%%%%%%%%%%%%%%%%%%%%%%
%
% les petis yeux 
%
%%%%%%%%%%%%%%%%%%%%%%%%%%%%%%%%%%%%%%%%%%%%%

\newcommand{\coolexo}{$\circledast\circledast$}
\newcommand{\boringexo}{$\circleddash\circleddash$}
\newcommand{\minsyndical}{$\odot\odot$}
\newcommand{\mortelexo}{$\obslash\oslash$}


%%%%%%%%%%%%%% FIN TRUCS DE YVIK %%%%%%%%%%%%%%%%%%%%%%

%%%%%%%%%%%%%% TRUCS DE PIERRE %%%%%%%%%%%%%%%%%%%%%%


% Le paquet array est là pour faire fonctionner l'environement arrowcases dans les trucs de Pierre.
\usepackage{array}

%\documentclass[11pt,a4paper,openany]{book}
%\usepackage[ansinew]{inputenc}
%\usepackage{pstricks, pst-node, array, ifpdf, comment, pst-plot}
%\usepackage[marginparwidth=2cm]{geometry}
%\usepackage[dvips,colorlinks]{hyperref}
%\usepackage[frenchb]{entetes}

%\usepackage{bigcenter}
%%%% debut macro %%%%
%%% ----------debut de bigcenter.sty--------------

%%% nouvel environnement bigcenter
%%% pour centrer sur toute la page (sans overfull)
%\makeatletter
%\newskip\@bigflushglue \@bigflushglue = -100pt plus 1fil

%\def\bigcenter{\trivlist \bigcentering\item\relax}
%\def\bigcentering{\let\\\@centercr\rightskip\@bigflushglue%
%\def\endbigcenter{\endtrivlist}

%\leftskip\@bigflushglue
%\parindent\z@\parfillskip\z@skip}
%\makeatother

%%% ----------fin de bigcenter.sty--------------
%%%% fin macro %%%%

%\input{mfpic}

% À régler par l'utilisateur
\newlength{\arrowsep}\setlength{\arrowsep}{3pt}
\newlength{\arrowlength}\setlength{\arrowlength}{1cm}

% Cet environnement est sympa, mais il dépend trop de ps; en tout cas il ne passe pas dans pdflatex
% 15 mars 1012
%\newenvironment{arrowcases}	{%
%			\pnode(\arrowsep,0.5ex){A}%
%			\hspace{\arrowlength}%
%			\begin{array}{>{\displaystyle\pnode(-\arrowsep,0.5ex){B}}l<{\ncline{A}{B}}@{}}
%				}
%			{
%			  \end{array}
%			}

\newenvironment{arrowcases}%
{\begin{cases}}
{\end{cases}}



\makeatletter %% \limite[condition]x x_0
\newcommand*{\limite}[3][\@empty]{\lim_{\substack{#2\rightarrow#3\\#1}}}
\makeatother

% \newenvironment{split+justif}{%
% \begin{split}%
% \let\ampori&
% \def&#1&#2\\{}
% }{%

% \end{split}}%

%\def\ncov{\tilde\nabla} % Nouvelle dérivée covariante
\newcommand*\sev{<} % 

%\newcommand{\hgot}{\mathfrak{h}} % h gothique (ss algebre de Lie)
%\def\var#1{{\mathbf #1}} % \var <-> Une variété
%\def\pardef{\stackrel{def}{=}} % = par définition.
%\newcommand{\bbar#1}{\bar{\bar{#1}}}
%\def\cov{\nabla} % Derivee covariante / connexion
%\newcommand{\gl}{\mathfrak{gl}} % algèbre linéaire
%\def\doubleprime{{\prime\prime}} % Isomorphique 
%\def\scal(#1,#2){\langle #1,#2\rangle}
%\def\agit(#1,#2){\langle #1,#2^\vee\rangle}
%\let\phiori\phi
%\let\phi\varphi
\let\ssi\iff
%\def\iddc{\mathcal I}
\newcommand*{\ideal}[1]{\{#1\}}
\newcommand*{\fleche}[1]{\stackrel{#1}\longrightarrow}

%\newcounter{exercice}
%\setcounter{exercice}{0}
\setcounter{CountExercice}{0}

% \newenvironment{exo}[1][\relax]{%
% \stepcounter{exercice}%
% \par\medskip%
% #1{\textbf{Exercice}~\arabic{exercice}.}\quad}%
% {\par}

% \newenvironment{rep}{\hspace{1em}\par\textbf{Solution
%     proposée.\quad}}{\par\noindent\hrulefill\par}

%\date{}
\newcommand{\Acplx}{A_\cdot}
\newcommand{\Bcplx}{B_\cdot}
\newcommand{\toisom}{\fleche\simeq}
\newcommand{\D}{\partial}
%\newcommand{\cat}[1]{{\bf #1}}
%\newcommand{\donc}{\Rightarrow}
%\newcommand{\im}{\text{im}}
%\newcommand{\coker}{\text{coker}}
\newcommand{\lied}{\mathcal L}
\newcommand*{\nom}[1]{\textsc{#1}}
\makeatletter
\newcommand*{\attention}[1]{\@latex@warning{#1}{!\small\bf #1!}\marginpar{Warning}}
\makeatother
\newcommand*{\inner}{\imath}
\newcommand*{\newexo}{}
\newcommand*{\principe}{}
\newcommand*{\etape}{}
\newcommand*{\preuve}{}
\newcommand*{\exr}{\item}
%\def\prim#1\expandafter\d#2 {\int #1\d#2}

\newcommand*{\crochets}[1]{\Bigl[ #1 \Bigr]}
\newcommand*{\llbrack}[1]{\left\lbrack #1 \right\lbrack}
\newcommand*{\rlbrack}[1]{\left\rbrack #1 \right\lbrack}
\newcommand*{\lrbrack}[1]{\left\lbrack #1 \right\rbrack}
\newcommand*{\rrbrack}[1]{\left\rbrack #1 \right\rbrack}
\newcommand*{\vecteur}[1]{\mathbf{#1}}


%% Maths : Les ensembles
\newcommand*{\ens}[1]{\mathbb{#1}} % Ensemble de nombres
\newcommand*{\var}[1]{\mathbf{#1}} % Variété
\newcommand*{\alg}[1]{\mathcal{#1}} % Algèbre
%\newcommand*{\RR}{\ens R}%
\newcommand*{\TT}{\ens T}% Tore !
%\expandafter\show\csname SS \endcsname
%\renewcommand*{\SS}{\var S}% 
%\newcommand*{\CC}{\ens C}%
\newcommand*{\ZZ}{\ens Z}%
\newcommand*{\QQ}{\ens Q}%
\newcommand*{\NN}{\ens N}%
\newcommand{\schwartz}{\mathcal S} % Espace de Schwartz
\newcommand*{\topologie}{\mathscr{T}}
\newcommand*{\Topologie}{\textcursive{T}}
\newcommand{\LL}{\text{\textup{L}}} %% Espace de Lebesgue droit
\newcommand{\Ll}{\mathcal{L}} %% Lebesgue ronde
\newcommand{\fronde}{\mathcal{F}} %% Transformée de Fourier.
\newcommand{\sigmaalgebre}[1]{\mathcal{#1}} %% Une sigma algèbre...
\DeclareMathOperator{\SymMatrix}{Sym}
\DeclareMathOperator{\ASymMatrix}{ASym}
\newcommand{\Sym}{\SymMatrix}
\newcommand{\ASym}{\ASymMatrix}
\newcommand{\transpose}[1]{{\vphantom{#1}}^{\mathit t}{\/#1}}
\newcommand*{\Sp}{\textup{Sp}}
\newcommand*{\Gl}{\textup{GL}}
%\renewcommand*{\sp}{\textup{sp}}
\newcommand*{\dprime}{{\prime\prime}}
%\show\span
%\newcommand*{\Span}[1]{\mathopen> #1 \mathclose<}

%% Maths : Symboles divers
%\newcommand{\pp}{\text{\textup{~p.p.}}} %% Presque partout
%\PackageWarning{entetes}{Redefining command \d}
%\renewcommand{\d}{\mbox{$\,$\textrm{d}}}
\newcommand{\surj}{\vers}
\newcommand{\isom}{\simeq}
\newcommand*{\Tau}{\alg T}
\newcommand{\cdv}{\mathfrak{X}} % Champs de vecteurs


%% Maths : Constructions

%\let\Exp\exp
%\renewcommand{\exp}[1]{e^{#1}} % On préfère e^{} que exp{}

%\renewcommand{\exp}[1]{e^{#1}} % On préfère e^{} que exp{}
%\renewcommand{\vec}[1]{\mathbf{#1}} % Désigner un vecteur
\newcommand{\set}[1]{\left\{#1\right\}} % Un ensemble { }
\newcommand*{\abs}[1]{\left\vert#1\right\vert} % Valeur absolue.
\newcommand*{\module}[1]{\left\vert#1\right\vert} % Valeur absolue.
\newcommand*{\norme}[1]{\left\Vert#1\right\Vert} % norme
\newcommand*{\ordre}[1]{\left\vert#1\right\vert} % L'ordre d'un élément.
%\def\scal(#1,#2){% Produit scalaire.
%  \PackageWarning{entetes}{Obsolete command \string\scal}%
%  \scalprod{#1}{#2}%
%}
\newcommand*{\scalprod}[2]{\left\langle #1,#2\right\rangle}
\let\dual\ast

\newcommand*{\pardef}{\stackrel{\text{def}}{=}} % Par définition.
\newcommand*{\iffdefn}{\stackrel{\text{def}}{\iff}} % Par définition.
%\newcommand*{\telque}{\mbox{~\entetes@name@telque~}} % tel que, dans un ensemble.
\newcommand*{\Defn}[1]{\emph{#1}} %
\newcommand*{\tensor}{\otimes}
\newcommand*{\pder}[2]{\frac{\partial #1}{\partial #2}}

%{{{ Fraction in-line plus jolie
% \DeclareRobustCommand\sfrac[1]{\@ifnextchar/{\@sfrac{#1}}%
%                                             {\@sfrac{#1}/}}
% \def\@sfrac#1/#2{\leavevmode\kern.1em\raise.5ex
%          \hbox{$\m@th{\fontsize\sf@size\z@
%                            \selectfont#1}$}\kern-.1em
%          /\kern-.15em\lower.25ex
%           \hbox{$\m@th{\fontsize\sf@size\z@
%                             \selectfont#2}$}}
%}}} 

\DeclareRobustCommand{\sfrac}[3][\mathrm]{\hspace{0.1em}%
  \raisebox{0.4ex}{$#1{\scriptstyle
#2}$}\hspace{-0.1em}/\hspace{-0.07em}%
  \mbox{$#1{\scriptstyle #3}$}}



%% Maths : Opérateurs
%\DeclareMathOperator{\tr}{Tr}
%\DeclareMathOperator{\pr}{\texttt{pr}}
\DeclareMathOperator{\supp}{supp}
\DeclareMathOperator{\adh}{adh}
\DeclareMathOperator{\interior}{int}
\DeclareMathOperator{\im}{Im}
\DeclareMathOperator{\Id}{Id}
\DeclareMathOperator{\Aut}{Aut}
\DeclareMathOperator{\Iso}{Iso}
\DeclareMathOperator{\Jac}{Jac} % jacobienne
\DeclareMathOperator{\coker}{coker}
\DeclareMathOperator{\interieur}{int}
\DeclareMathOperator{\Tor}{Tor}
\DeclareMathOperator{\divg}{div}
\DeclareMathOperator{\rot}{rot}
%\DeclareMathOperator{\cosec}{cosec}


%% Pour obtenir le \Sha cyrillique...
% \RequirePackage[OT2,T1]{fontenc}
% \DeclareSymbolFont{cyrletters}{OT2}{wncyr}{m}{n}
% \DeclareMathSymbol{\Sha}{\mathalpha}{cyrletters}{"58}


% \newcounter{@institute}
% \let\authorori\author
% \def\@institute{}\def\@auteurs{}
% %\newcommand{\institute}[2]{\refstepcounter{@institute}\label{#1}\def\@institute{\@institute\small
% %#2}\set@authors}
% \newcommand{\institute}[2]{\refstepcounter{@institute}\label{#1}%
%   \let\maketitleori\maketitle%
%   \renewcommand\maketitle{\footnote{#2}\maketitleori}%
% }%
% \renewcommand{\author}[1]{\def\@auteurs{#1}\set@authors}
% \def\the@institute{${}^{(\roman{@institute})}$}
% \newcommand{\inst}[1]{\ref{#1}}

% \newcommand{\set@authors}{\authorori{\@auteurs}}% \\ \@institute}}


% \newcounter{@institute}
% \newcommand{\institute}[1]{
%   \let\labelori\label
%   \renewcommand{\label}[1]{%
%     \refstepcounter{@institute}\labelori{##1}
%     \begin{tabular}{cc}%format ?!
%     \begin{minipage}[t]
      


%   }

% }%

\newcommand*{\conclusion}{\emph{Conclusion~:~}}
\newcommand{\hint}{\par\emph{Aide~:~}\hspace{1em}}
\newcommand{\rappel}{\par\emph{Rappels~:~}\hspace{1em}}

%\newcounter{enumarray} 
%\newenvironment{enumarray}[1]{% Merci Ulrike Fischer
% \setcounter{enumarray}{0}%
% \begin{array}{% motif
%     >{% Au début de chaque ligne
%       \stepcounter{enumarray}%
%       (\alph{enumarray})%
%       \hspace{2em}
%     }% 
%     #1%
%   }% fin motif
% }{%
% \end{array}%
%}
\newenvironment{displayinline}{% displaystyle + inline.
  $\displaystyle%
}{%
  $%
}

\newcommand{\telque}{\vert\,}
\newcommand{\donc}{\Rightarrow}


\pagestyle{empty}   % Pour éviter les numéros de page.
\renewcommand{\cite}[1]{}           % Les citations ne servent à rien dans les feuilles distribuées.
\corrPosition{0}

\newcounter{bidon}

\newcommand{\multiinput}[2]%
{%
    \setcounter{bidon}{#2}
    \addtocounter{bidon}{-1}
    \multido{\i=1+1}{\value{bidon}}{\input{#1}\vspace{0.8cm}}%
    \input{#1}
}

% 1 on utilise external et 0 on ne l'utilise pas
\newcounter{useexternal}
\setcounter{useexternal}{1}
\ifthenelse{\value{useexternal}=1}{ \usetikzlibrary{external} \tikzexternalize }{ \newcommand{\tikzsetnextfilename}[1]{} }


% MINI INTERRO NUMÉRO 3 POUR 5A : EXPRESSIONS LITTÉRALES ET NOMBRES RELATIFS
\begin{document}
\multiinput{interro_8}{2}
\end{document}

% MINI INTERRO NUMÉRO 3 POUR 5B : EXPRESSIONS LITTÉRALES ET SYMÉTRIE CENTRALE
\begin{document}
\multiinput{interro_9}{2}
\end{document}


% QUESTIONS POUR RÉFLÉCHIR UN PEU PLUS, CINQUIÈME ET QUATRIÈME, 3
\begin{document}
% Ne pas oublier de mettre dans 0090_tout_fait.tex losque c'est fini
\begin{feuilleExo}{Questions pour réfléchir un peu plus, cinquième et quatrième, 3}
\Exo{smath-0915}
\Exo{smath-0913}
\Exo{smath-0933}
\Exo{2smath-0042}
\end{feuilleExo}
\end{document}


% QUESTIONS POUR RÉFLÉCHIR UN PEU PLUS, CINQUIÈME ET QUATRIÈME, 2
\begin{document}
\begin{feuilleExo}{Questions pour réfléchir un peu plus, cinquième et quatrième, 2}
\Exo{smath-0904}
\Exo{smath-0905}
\Exo{smath-0906}
\Exo{2smath-0015}  
\Exo{2smath-0016}  
\Exo{smath-0797}
\end{feuilleExo}
\end{document}



% IMPRESSION CHAPITRE NOMBRES RELATIFS (cinquième)
\begin{document}
\multiinput{activ_nombres_neg}{5}
\newpage
\multiinput{activ_relatifs_droite_grad}{5}
\newpage
\multiinput{activ_ranger_relatif}{4}
\newpage
\multiinput{activ_repere_plan}{2}
\end{document}



% IMPRESSION CHAPITRE EXPRESSIONS LITTÉRALES (cinquième)
\begin{document}
\multiinput{activ_carre_sans_coins}{2}
\newpage
\multiinput{activ_benef_facto}{5}
\end{document}


% IMPRESSION CHAPITRE SYMÉTRIE CENTRALE
\begin{document}
\multiinput{activ_sym_axiale}{4}
\newpage
\multiinput{activ_sym_centrale}{3}
\newpage
\multiinput{activ_sym_centrale_droite}{8}
\end{document}



% IMPRESSION CHAPITRE OPÉRATIONS FRACTIONS (QUATRIÈME)
\begin{document}
\multiinput{activ_add_fract2}{2}
\newpage
\multiinput{activ_prod_frac4}{5}
\newpage
\multiinput{activ_div_frac2}{5}
\newpage
\multiinput{activ_div_frac}{2}
\end{document}



% IMPRESSION CHAPITRE SYMÉTRIE CENTRALE
\begin{document}
\multiinput{activ_sym_axiale}{4}
\newpage
\multiinput{activ_sym_centrale}{3}
\newpage
\multiinput{activ_sym_centrale_droite}{8}
\end{document}


% MINI INTERRO OPÉRATIONS SUR LES FRACTIONS
\begin{document}
\multiinput{interro_6}{2}
\end{document}


% MINI INTERRO TABLEAU DE PROPORTIONNALITÉ
\begin{document}
\multiinput{interro_7}{2}
\end{document}

% IMPRESSION CHAPITRE THÉORÈMES DU MILEUX
\begin{document}
\multiinput{activ_tho_milieux}{7}
\newpage
\multiinput{activ_tho_milieux_trois}{6}
\end{document}


% IMPRESSION CHAPITRE SYMÉTRIE CENTRALE (CINQUIÈME)
\begin{document}
\multiinput{activ_sym_axiale}{4}
\newpage
\multiinput{activ_sym_centrale}{3}
\end{document}


% MINI-INTERRO DROITES REMARQUABLES
\begin{document}
%
\vfill

\begin{multicols}{2}
    \enteteInterro{Le 5 décembre 2014}{2}{A}
\Exo{smath-0994}

    \enteteInterro{Le 5 décembre 2014}{2}{B}
\Exo{smath-0995}
\end{multicols}

\vfill


\multiinput{interro_5}{2}
\end{document}


% POUR LA CORRECTION DU DS3 des CINQUIÈMES
\begin{document}
\Exo{smath-0971}
\Exo{smath-0972}
\Exo{smath-0976}
\Exo{smath-0974}
\Exo{smath-0975}
\Exo{smath-0978}
\Exo{smath-0977}
\end{document}

% IMPRESSION CHAPITRE OPÉRATIONS FRACTIONS
\begin{document}
% This is part of Un soupçon de mathématique sans être agressif pour autant
% Copyright (c) 2014
%   Laurent Claessens
% See the file fdl-1.3.txt for copying conditions.

%+++++++++++++++++++++++++++++++++++++++++++++++++++++++++++++++++++++++++++++++++++++++++++++++++++++++++++++++++++++++++++ 
\section*{Activité : secteurs d'émissions (1)}
%+++++++++++++++++++++++++++++++++++++++++++++++++++++++++++++++++++++++++++++++++++++++++++++++++++++++++++++++++++++++++++

Sur cette planète, un cinquième des émissions de dioxyde de carbone sont dus aux processus industriels, un autre cinquième aux transports et un dixième aux bâtiments. Quelle fraction du total des émissions de \( CO_2\) est due à ces trois activités ?

Colorier l'égalité suivante :
\begin{center}
    \input{Fig_GLVooNqioTo0.pstricks}+\input{Fig_GLVooNqioTo1.pstricks}+\input{Fig_GLVooNqioTo2.pstricks}=\input{Fig_GLVooNqioTo3.pstricks}.
\end{center}

Les centrales énergétiques sont responsables d'un tiers des émissions de dioxyde de carbone. Quel est le total de ces quatre activités ?

\noindent {\scriptsize Pour plus d'informations, voir\ \url{http://savoirsenmultimedia.ens.fr/uploads/videos//diffusion/2012_02_09_jancovici.mp4}}

\vspace{1cm}
% This is part of Un soupçon de mathématique sans être agressif pour autant
% Copyright (c) 2014
%   Laurent Claessens
% See the file fdl-1.3.txt for copying conditions.

%+++++++++++++++++++++++++++++++++++++++++++++++++++++++++++++++++++++++++++++++++++++++++++++++++++++++++++++++++++++++++++ 
\section*{Activité : secteurs d'émissions (1)}
%+++++++++++++++++++++++++++++++++++++++++++++++++++++++++++++++++++++++++++++++++++++++++++++++++++++++++++++++++++++++++++

Sur cette planète, un cinquième des émissions de dioxyde de carbone sont dus aux processus industriels, un autre cinquième aux transports et un dixième aux bâtiments. Quelle fraction du total des émissions de \( CO_2\) est due à ces trois activités ?

Colorier l'égalité suivante :
\begin{center}
    \input{Fig_GLVooNqioTo0.pstricks}+\input{Fig_GLVooNqioTo1.pstricks}+\input{Fig_GLVooNqioTo2.pstricks}=\input{Fig_GLVooNqioTo3.pstricks}.
\end{center}

Les centrales énergétiques sont responsables d'un tiers des émissions de dioxyde de carbone. Quel est le total de ces quatre activités ?

\noindent {\scriptsize Pour plus d'informations, voir\ \url{http://savoirsenmultimedia.ens.fr/uploads/videos//diffusion/2012_02_09_jancovici.mp4}}

\vspace{1cm}
% This is part of Un soupçon de mathématique sans être agressif pour autant
% Copyright (c) 2014
%   Laurent Claessens
% See the file fdl-1.3.txt for copying conditions.

%+++++++++++++++++++++++++++++++++++++++++++++++++++++++++++++++++++++++++++++++++++++++++++++++++++++++++++++++++++++++++++ 
\section*{Activité : secteurs d'émissions (1)}
%+++++++++++++++++++++++++++++++++++++++++++++++++++++++++++++++++++++++++++++++++++++++++++++++++++++++++++++++++++++++++++

Sur cette planète, un cinquième des émissions de dioxyde de carbone sont dus aux processus industriels, un autre cinquième aux transports et un dixième aux bâtiments. Quelle fraction du total des émissions de \( CO_2\) est due à ces trois activités ?

Colorier l'égalité suivante :
\begin{center}
    \input{Fig_GLVooNqioTo0.pstricks}+\input{Fig_GLVooNqioTo1.pstricks}+\input{Fig_GLVooNqioTo2.pstricks}=\input{Fig_GLVooNqioTo3.pstricks}.
\end{center}

Les centrales énergétiques sont responsables d'un tiers des émissions de dioxyde de carbone. Quel est le total de ces quatre activités ?

\noindent {\scriptsize Pour plus d'informations, voir\ \url{http://savoirsenmultimedia.ens.fr/uploads/videos//diffusion/2012_02_09_jancovici.mp4}}

\newpage
% This is part of Un soupçon de mathématique sans être agressif pour autant
% Copyright (c) 2014
%   Laurent Claessens
% See the file fdl-1.3.txt for copying conditions.

%--------------------------------------------------------------------------------------------------------------------------- 
\subsection*{Activité : fraction d'un grand rectangle}
%---------------------------------------------------------------------------------------------------------------------------

Quel est le périmètre du rectangle grisé ? Quelle est son aire ?
\begin{center}
   \input{Fig_FDOooRCCWGn.pstricks}
\end{center}

\vspace{1cm}
% This is part of Un soupçon de mathématique sans être agressif pour autant
% Copyright (c) 2014
%   Laurent Claessens
% See the file fdl-1.3.txt for copying conditions.

%--------------------------------------------------------------------------------------------------------------------------- 
\subsection*{Activité : fraction d'un grand rectangle}
%---------------------------------------------------------------------------------------------------------------------------

Quel est le périmètre du rectangle grisé ? Quelle est son aire ?
\begin{center}
   \input{Fig_FDOooRCCWGn.pstricks}
\end{center}

\vspace{1cm}
% This is part of Un soupçon de mathématique sans être agressif pour autant
% Copyright (c) 2014
%   Laurent Claessens
% See the file fdl-1.3.txt for copying conditions.

%--------------------------------------------------------------------------------------------------------------------------- 
\subsection*{Activité : fraction d'un grand rectangle}
%---------------------------------------------------------------------------------------------------------------------------

Quel est le périmètre du rectangle grisé ? Quelle est son aire ?
\begin{center}
   \input{Fig_FDOooRCCWGn.pstricks}
\end{center}

\vspace{1cm}
% This is part of Un soupçon de mathématique sans être agressif pour autant
% Copyright (c) 2014
%   Laurent Claessens
% See the file fdl-1.3.txt for copying conditions.

%--------------------------------------------------------------------------------------------------------------------------- 
\subsection*{Activité : fraction d'un grand rectangle}
%---------------------------------------------------------------------------------------------------------------------------

Quel est le périmètre du rectangle grisé ? Quelle est son aire ?
\begin{center}
   \input{Fig_FDOooRCCWGn.pstricks}
\end{center}

\end{document}

% IMPRESSION CHAPITRE GRANDEURS PROPORTIONNELLES
\begin{document}
% This is part of Un soupçon de mathématique sans être agressif pour autant
% Copyright (c) 2014
%   Laurent Claessens
% See the file fdl-1.3.txt for copying conditions.

%--------------------------------------------------------------------------------------------------------------------------- 
\subsection*{Tableaux et représentation graphique}
%---------------------------------------------------------------------------------------------------------------------------

Associer à chaque tableau le graphique qui correspond. Quels sont ceux qui correspondent à des situations de proportionnalité ?

\begin{equation}
    \input{Fig_ARKZooKZOuAkconvprix.latex}\quad
    \input{ARKZooKZOuAkfar.latex}
\end{equation}
\begin{equation}
    \input{ARKZooKZOuAkmilm.latex}\quad
    \input{ARKZooKZOuAkctlibre.latex}
\end{equation}

\begin{center}
   \input{Fig_ARKZooKZOuAk0.pstricks}
   \input{Fig_ARKZooKZOuAk3.pstricks}
   \input{Fig_ARKZooKZOuAk2.pstricks}
   \input{Fig_ARKZooKZOuAk1.pstricks}
\end{center}

% This is part of Un soupçon de mathématique sans être agressif pour autant
% Copyright (c) 2014
%   Laurent Claessens
% See the file fdl-1.3.txt for copying conditions.

%--------------------------------------------------------------------------------------------------------------------------- 
\subsection*{Tableaux et représentation graphique}
%---------------------------------------------------------------------------------------------------------------------------

Associer à chaque tableau le graphique qui correspond. Quels sont ceux qui correspondent à des situations de proportionnalité ?

\begin{equation}
    \input{Fig_ARKZooKZOuAkconvprix.latex}\quad
    \input{ARKZooKZOuAkfar.latex}
\end{equation}
\begin{equation}
    \input{ARKZooKZOuAkmilm.latex}\quad
    \input{ARKZooKZOuAkctlibre.latex}
\end{equation}

\begin{center}
   \input{Fig_ARKZooKZOuAk0.pstricks}
   \input{Fig_ARKZooKZOuAk3.pstricks}
   \input{Fig_ARKZooKZOuAk2.pstricks}
   \input{Fig_ARKZooKZOuAk1.pstricks}
\end{center}

\newpage
% This is part of Un soupçon de mathématique sans être agressif pour autant
% Copyright (c) 2014
%   Laurent Claessens
% See the file fdl-1.3.txt for copying conditions.
% et by the way, ce texte est recopié du sésamath de quatrième.

%--------------------------------------------------------------------------------------------------------------------------- 
\subsection{Activité : Pourcentages de deux groupes}
%---------------------------------------------------------------------------------------------------------------------------

Un groupe de $40$ filles et \( 14\) garçons de 4\ieme\ ont effectué un devoir commun. Les filles ont obtenu $60$\% de réussite et les garçons ont obtenu $50$\% de réussite. Calculer le pourcentage total d'élèves ayant réussi.

\vspace{1cm}
% This is part of Un soupçon de mathématique sans être agressif pour autant
% Copyright (c) 2014
%   Laurent Claessens
% See the file fdl-1.3.txt for copying conditions.
% et by the way, ce texte est recopié du sésamath de quatrième.

%--------------------------------------------------------------------------------------------------------------------------- 
\subsection{Activité : Pourcentages de deux groupes}
%---------------------------------------------------------------------------------------------------------------------------

Un groupe de $40$ filles et \( 14\) garçons de 4\ieme\ ont effectué un devoir commun. Les filles ont obtenu $60$\% de réussite et les garçons ont obtenu $50$\% de réussite. Calculer le pourcentage total d'élèves ayant réussi.

\vspace{1cm}
% This is part of Un soupçon de mathématique sans être agressif pour autant
% Copyright (c) 2014
%   Laurent Claessens
% See the file fdl-1.3.txt for copying conditions.
% et by the way, ce texte est recopié du sésamath de quatrième.

%--------------------------------------------------------------------------------------------------------------------------- 
\subsection{Activité : Pourcentages de deux groupes}
%---------------------------------------------------------------------------------------------------------------------------

Un groupe de $40$ filles et \( 14\) garçons de 4\ieme\ ont effectué un devoir commun. Les filles ont obtenu $60$\% de réussite et les garçons ont obtenu $50$\% de réussite. Calculer le pourcentage total d'élèves ayant réussi.

\vspace{1cm}
% This is part of Un soupçon de mathématique sans être agressif pour autant
% Copyright (c) 2014
%   Laurent Claessens
% See the file fdl-1.3.txt for copying conditions.
% et by the way, ce texte est recopié du sésamath de quatrième.

%--------------------------------------------------------------------------------------------------------------------------- 
\subsection{Activité : Pourcentages de deux groupes}
%---------------------------------------------------------------------------------------------------------------------------

Un groupe de $40$ filles et \( 14\) garçons de 4\ieme\ ont effectué un devoir commun. Les filles ont obtenu $60$\% de réussite et les garçons ont obtenu $50$\% de réussite. Calculer le pourcentage total d'élèves ayant réussi.

\vspace{1cm}
% This is part of Un soupçon de mathématique sans être agressif pour autant
% Copyright (c) 2014
%   Laurent Claessens
% See the file fdl-1.3.txt for copying conditions.
% et by the way, ce texte est recopié du sésamath de quatrième.

%--------------------------------------------------------------------------------------------------------------------------- 
\subsection{Activité : Pourcentages de deux groupes}
%---------------------------------------------------------------------------------------------------------------------------

Un groupe de $40$ filles et \( 14\) garçons de 4\ieme\ ont effectué un devoir commun. Les filles ont obtenu $60$\% de réussite et les garçons ont obtenu $50$\% de réussite. Calculer le pourcentage total d'élèves ayant réussi.

\vspace{1cm}
% This is part of Un soupçon de mathématique sans être agressif pour autant
% Copyright (c) 2014
%   Laurent Claessens
% See the file fdl-1.3.txt for copying conditions.
% et by the way, ce texte est recopié du sésamath de quatrième.

%--------------------------------------------------------------------------------------------------------------------------- 
\subsection{Activité : Pourcentages de deux groupes}
%---------------------------------------------------------------------------------------------------------------------------

Un groupe de $40$ filles et \( 14\) garçons de 4\ieme\ ont effectué un devoir commun. Les filles ont obtenu $60$\% de réussite et les garçons ont obtenu $50$\% de réussite. Calculer le pourcentage total d'élèves ayant réussi.

\vspace{1cm}
% This is part of Un soupçon de mathématique sans être agressif pour autant
% Copyright (c) 2014
%   Laurent Claessens
% See the file fdl-1.3.txt for copying conditions.
% et by the way, ce texte est recopié du sésamath de quatrième.

%--------------------------------------------------------------------------------------------------------------------------- 
\subsection{Activité : Pourcentages de deux groupes}
%---------------------------------------------------------------------------------------------------------------------------

Un groupe de $40$ filles et \( 14\) garçons de 4\ieme\ ont effectué un devoir commun. Les filles ont obtenu $60$\% de réussite et les garçons ont obtenu $50$\% de réussite. Calculer le pourcentage total d'élèves ayant réussi.

\vspace{1cm}
% This is part of Un soupçon de mathématique sans être agressif pour autant
% Copyright (c) 2014
%   Laurent Claessens
% See the file fdl-1.3.txt for copying conditions.
% et by the way, ce texte est recopié du sésamath de quatrième.

%--------------------------------------------------------------------------------------------------------------------------- 
\subsection{Activité : Pourcentages de deux groupes}
%---------------------------------------------------------------------------------------------------------------------------

Un groupe de $40$ filles et \( 14\) garçons de 4\ieme\ ont effectué un devoir commun. Les filles ont obtenu $60$\% de réussite et les garçons ont obtenu $50$\% de réussite. Calculer le pourcentage total d'élèves ayant réussi.

\vspace{1cm}
% This is part of Un soupçon de mathématique sans être agressif pour autant
% Copyright (c) 2014
%   Laurent Claessens
% See the file fdl-1.3.txt for copying conditions.
% et by the way, ce texte est recopié du sésamath de quatrième.

%--------------------------------------------------------------------------------------------------------------------------- 
\subsection{Activité : Pourcentages de deux groupes}
%---------------------------------------------------------------------------------------------------------------------------

Un groupe de $40$ filles et \( 14\) garçons de 4\ieme\ ont effectué un devoir commun. Les filles ont obtenu $60$\% de réussite et les garçons ont obtenu $50$\% de réussite. Calculer le pourcentage total d'élèves ayant réussi.

\vspace{1cm}
% This is part of Un soupçon de mathématique sans être agressif pour autant
% Copyright (c) 2014
%   Laurent Claessens
% See the file fdl-1.3.txt for copying conditions.
% et by the way, ce texte est recopié du sésamath de quatrième.

%--------------------------------------------------------------------------------------------------------------------------- 
\subsection{Activité : Pourcentages de deux groupes}
%---------------------------------------------------------------------------------------------------------------------------

Un groupe de $40$ filles et \( 14\) garçons de 4\ieme\ ont effectué un devoir commun. Les filles ont obtenu $60$\% de réussite et les garçons ont obtenu $50$\% de réussite. Calculer le pourcentage total d'élèves ayant réussi.

\vspace{1cm}
% This is part of Un soupçon de mathématique sans être agressif pour autant
% Copyright (c) 2014
%   Laurent Claessens
% See the file fdl-1.3.txt for copying conditions.
% et by the way, ce texte est recopié du sésamath de quatrième.

%--------------------------------------------------------------------------------------------------------------------------- 
\subsection{Activité : Pourcentages de deux groupes}
%---------------------------------------------------------------------------------------------------------------------------

Un groupe de $40$ filles et \( 14\) garçons de 4\ieme\ ont effectué un devoir commun. Les filles ont obtenu $60$\% de réussite et les garçons ont obtenu $50$\% de réussite. Calculer le pourcentage total d'élèves ayant réussi.

\newpage
% This is part of Un soupçon de mathématique sans être agressif pour autant
% Copyright (c) 2014
%   Laurent Claessens
% See the file fdl-1.3.txt for copying conditions.

%--------------------------------------------------------------------------------------------------------------------------- 
\subsection*{Activité : quatrième proportionnelle}
%---------------------------------------------------------------------------------------------------------------------------

Compléter les tableaux de proportionnalité suivants (on pourra appeler \( x\) la valeur manquante)
\begin{multicols}{3}
    \begin{enumerate}
        \item
            \begin{equation*}
                \begin{array}[]{|c|c|}
                    \hline
                    6&10\\
                    \hline
                    \ldots&30\\
                    \hline
                \end{array}
            \end{equation*}
        \item
            \begin{equation*}
                \begin{array}[]{|c|c|}
                    \hline
                    10&\ldots\\
                    \hline
                    30&45\\
                    \hline
                \end{array}
            \end{equation*}
        \item
            \begin{equation*}
                \begin{array}[]{|c|c|}
                    \hline
                    3&\ldots\\
                    \hline
                    4&10\\
                    \hline
                \end{array}
            \end{equation*}
    \end{enumerate}
\end{multicols}



\vspace{1cm}
% This is part of Un soupçon de mathématique sans être agressif pour autant
% Copyright (c) 2014
%   Laurent Claessens
% See the file fdl-1.3.txt for copying conditions.

%--------------------------------------------------------------------------------------------------------------------------- 
\subsection*{Activité : quatrième proportionnelle}
%---------------------------------------------------------------------------------------------------------------------------

Compléter les tableaux de proportionnalité suivants (on pourra appeler \( x\) la valeur manquante)
\begin{multicols}{3}
    \begin{enumerate}
        \item
            \begin{equation*}
                \begin{array}[]{|c|c|}
                    \hline
                    6&10\\
                    \hline
                    \ldots&30\\
                    \hline
                \end{array}
            \end{equation*}
        \item
            \begin{equation*}
                \begin{array}[]{|c|c|}
                    \hline
                    10&\ldots\\
                    \hline
                    30&45\\
                    \hline
                \end{array}
            \end{equation*}
        \item
            \begin{equation*}
                \begin{array}[]{|c|c|}
                    \hline
                    3&\ldots\\
                    \hline
                    4&10\\
                    \hline
                \end{array}
            \end{equation*}
    \end{enumerate}
\end{multicols}



\vspace{1cm}
% This is part of Un soupçon de mathématique sans être agressif pour autant
% Copyright (c) 2014
%   Laurent Claessens
% See the file fdl-1.3.txt for copying conditions.

%--------------------------------------------------------------------------------------------------------------------------- 
\subsection*{Activité : quatrième proportionnelle}
%---------------------------------------------------------------------------------------------------------------------------

Compléter les tableaux de proportionnalité suivants (on pourra appeler \( x\) la valeur manquante)
\begin{multicols}{3}
    \begin{enumerate}
        \item
            \begin{equation*}
                \begin{array}[]{|c|c|}
                    \hline
                    6&10\\
                    \hline
                    \ldots&30\\
                    \hline
                \end{array}
            \end{equation*}
        \item
            \begin{equation*}
                \begin{array}[]{|c|c|}
                    \hline
                    10&\ldots\\
                    \hline
                    30&45\\
                    \hline
                \end{array}
            \end{equation*}
        \item
            \begin{equation*}
                \begin{array}[]{|c|c|}
                    \hline
                    3&\ldots\\
                    \hline
                    4&10\\
                    \hline
                \end{array}
            \end{equation*}
    \end{enumerate}
\end{multicols}



\vspace{1cm}
% This is part of Un soupçon de mathématique sans être agressif pour autant
% Copyright (c) 2014
%   Laurent Claessens
% See the file fdl-1.3.txt for copying conditions.

%--------------------------------------------------------------------------------------------------------------------------- 
\subsection*{Activité : quatrième proportionnelle}
%---------------------------------------------------------------------------------------------------------------------------

Compléter les tableaux de proportionnalité suivants (on pourra appeler \( x\) la valeur manquante)
\begin{multicols}{3}
    \begin{enumerate}
        \item
            \begin{equation*}
                \begin{array}[]{|c|c|}
                    \hline
                    6&10\\
                    \hline
                    \ldots&30\\
                    \hline
                \end{array}
            \end{equation*}
        \item
            \begin{equation*}
                \begin{array}[]{|c|c|}
                    \hline
                    10&\ldots\\
                    \hline
                    30&45\\
                    \hline
                \end{array}
            \end{equation*}
        \item
            \begin{equation*}
                \begin{array}[]{|c|c|}
                    \hline
                    3&\ldots\\
                    \hline
                    4&10\\
                    \hline
                \end{array}
            \end{equation*}
    \end{enumerate}
\end{multicols}



\vspace{1cm}
% This is part of Un soupçon de mathématique sans être agressif pour autant
% Copyright (c) 2014
%   Laurent Claessens
% See the file fdl-1.3.txt for copying conditions.

%--------------------------------------------------------------------------------------------------------------------------- 
\subsection*{Activité : quatrième proportionnelle}
%---------------------------------------------------------------------------------------------------------------------------

Compléter les tableaux de proportionnalité suivants (on pourra appeler \( x\) la valeur manquante)
\begin{multicols}{3}
    \begin{enumerate}
        \item
            \begin{equation*}
                \begin{array}[]{|c|c|}
                    \hline
                    6&10\\
                    \hline
                    \ldots&30\\
                    \hline
                \end{array}
            \end{equation*}
        \item
            \begin{equation*}
                \begin{array}[]{|c|c|}
                    \hline
                    10&\ldots\\
                    \hline
                    30&45\\
                    \hline
                \end{array}
            \end{equation*}
        \item
            \begin{equation*}
                \begin{array}[]{|c|c|}
                    \hline
                    3&\ldots\\
                    \hline
                    4&10\\
                    \hline
                \end{array}
            \end{equation*}
    \end{enumerate}
\end{multicols}



\end{document}


% POUR LA CORRECTION DE LA MINI-INTERRO PYTHAGORE
\begin{document}
\Exo{smath-0953}
\Exo{smath-0954}
\Exo{smath-0955}
\Exo{smath-0956}
\end{document}


% MINI-INTERRO PYTHAGORE
\begin{document}
\begin{multicols}{2}
    \enteteInterro{Le 17 novembre 2014}{2}{A}
\Exo{smath-0953}

    \enteteInterro{Le 17 novembre 2014}{2}{B}
\Exo{smath-0954}
\end{multicols}

\vfill

\begin{multicols}{2}
    \enteteInterro{Le 17 novembre 2014}{2}{C}
\Exo{smath-0955}
    \enteteInterro{Le 17 novembre 2014}{2}{D}
\Exo{smath-0956}
\end{multicols}
\vfill

\end{document}



\begin{document}
\Exo{smath-0966}
\Exo{smath-0969}
\Exo{smath-0970}
\Exo{smath-0963}
\Exo{smath-0965}
\end{document}


% SALLE MULTIMÉDIA, 5B, 21 NOVEMBRE 2014
\begin{document}
\begin{feuilleExo}{Salle multimédia 5B, 21 novembre}
\Exo{smath-0961}
\Exo{smath-0962}
\end{feuilleExo}
\vspace{1cm}
\begin{feuilleExo}{Salle multimédia 5B, 21 novembre}
\Exo{smath-0961}
\Exo{smath-0962}
\end{feuilleExo}
\vspace{1cm}
\begin{feuilleExo}{Salle multimédia 5B, 21 novembre}
\Exo{smath-0961}
\Exo{smath-0962}
\end{feuilleExo}
\end{document}


% IMPRESSION CHAPITRE DROITES REMARQUABLES DANS UN TRIANGLE
\begin{document}
% This is part of Un soupçon de mathématique sans être agressif pour autant
% Copyright (c) 2014
%   Laurent Claessens
% See the file fdl-1.3.txt for copying conditions.

%--------------------------------------------------------------------------------------------------------------------------- 
\subsection*{Activité : Napoléon se place}
%---------------------------------------------------------------------------------------------------------------------------

Les Anglais et les Autrichiens ont pris position en deux points \( A\) et \( B\) distants de \( \SI{10}{\kilo\meter}\). Napoléon veut pouvoir être à même de les attaquer tous les deux d'égale manière et décide donc de se positionner en un point $N$ qui serait à égale distance de \( A\) que de \( B\).

Bien entendu il pourrait se placer au milieu du segment \( [AB]\), mais ainsi il se ferait trop facilement attaquer des deux côtés à la fois (pas fou le Corse!). Où peut-il se placer ? Faire un dessin pour l'aider.

\vspace{1cm}
% This is part of Un soupçon de mathématique sans être agressif pour autant
% Copyright (c) 2014
%   Laurent Claessens
% See the file fdl-1.3.txt for copying conditions.

%--------------------------------------------------------------------------------------------------------------------------- 
\subsection*{Activité : chevaliers de la table ronde}
%---------------------------------------------------------------------------------------------------------------------------

Les chevaliers de la table ronde se sont disputés; Tristan, Perceval et Lancelot sont allés mettre leurs chaises un peu à l'écart et refusent de bouger. Sur le dessin suivant, l'unité est en mètres. Quel doit être le rayon de la nouvelle table ronde de telle sorte à ce que ces trois chevaliers puissent siéger ?

\begin{center}
   \input{Fig_DTFZooTlciUT.pstricks}
\end{center}

\newpage
% This is part of Un soupçon de mathématique sans être agressif pour autant
% Copyright (c) 2014
%   Laurent Claessens
% See the file fdl-1.3.txt for copying conditions.

%--------------------------------------------------------------------------------------------------------------------------- 
\subsection{Activité : mêmes aires}
%---------------------------------------------------------------------------------------------------------------------------

\begin{wrapfigure}[1]{r}{5.0cm}
   \vspace{-0.5cm}        % à adapter.
   \centering
   \input{Fig_VUOCooYkrktO.pstricks}
\end{wrapfigure}

Les triangles \( ASB\) et \( SCB\) ont même aire. Ajouter les codages.

\vspace{3cm}
--------------------------------

\vspace{1cm}
% This is part of Un soupçon de mathématique sans être agressif pour autant
% Copyright (c) 2014
%   Laurent Claessens
% See the file fdl-1.3.txt for copying conditions.

%--------------------------------------------------------------------------------------------------------------------------- 
\subsection{Activité : mêmes aires}
%---------------------------------------------------------------------------------------------------------------------------

\begin{wrapfigure}[1]{r}{5.0cm}
   \vspace{-0.5cm}        % à adapter.
   \centering
   \input{Fig_VUOCooYkrktO.pstricks}
\end{wrapfigure}

Les triangles \( ASB\) et \( SCB\) ont même aire. Ajouter les codages.

\vspace{3cm}
--------------------------------

\vspace{1cm}
% This is part of Un soupçon de mathématique sans être agressif pour autant
% Copyright (c) 2014
%   Laurent Claessens
% See the file fdl-1.3.txt for copying conditions.

%--------------------------------------------------------------------------------------------------------------------------- 
\subsection{Activité : mêmes aires}
%---------------------------------------------------------------------------------------------------------------------------

\begin{wrapfigure}[1]{r}{5.0cm}
   \vspace{-0.5cm}        % à adapter.
   \centering
   \input{Fig_VUOCooYkrktO.pstricks}
\end{wrapfigure}

Les triangles \( ASB\) et \( SCB\) ont même aire. Ajouter les codages.

\vspace{3cm}
--------------------------------

\vspace{1cm}
% This is part of Un soupçon de mathématique sans être agressif pour autant
% Copyright (c) 2014
%   Laurent Claessens
% See the file fdl-1.3.txt for copying conditions.

%--------------------------------------------------------------------------------------------------------------------------- 
\subsection{Activité : mêmes aires}
%---------------------------------------------------------------------------------------------------------------------------

\begin{wrapfigure}[1]{r}{5.0cm}
   \vspace{-0.5cm}        % à adapter.
   \centering
   \input{Fig_VUOCooYkrktO.pstricks}
\end{wrapfigure}

Les triangles \( ASB\) et \( SCB\) ont même aire. Ajouter les codages.

\vspace{3cm}
--------------------------------

% Les figures à coder :
\newpage
\begin{center}
\input{Fig_QTCQooFtDgwk.pstricks}
\input{Fig_QTCQooFtDgwk.pstricks}
\vspace{1cm}
\input{Fig_QTCQooFtDgwk.pstricks}
\input{Fig_QTCQooFtDgwk.pstricks}
\vspace{1cm}
\input{Fig_QTCQooFtDgwk.pstricks}
\input{Fig_QTCQooFtDgwk.pstricks}
\vspace{1cm}
\input{Fig_QTCQooFtDgwk.pstricks}
\input{Fig_QTCQooFtDgwk.pstricks}
\vspace{1cm}
\input{Fig_QTCQooFtDgwk.pstricks}
\input{Fig_QTCQooFtDgwk.pstricks}
\newpage
\input{Fig_QZPMooIiOQpy0.pstricks}
\input{Fig_QZPMooIiOQpy0.pstricks}
\vspace{1cm}
\input{Fig_QZPMooIiOQpy0.pstricks}
\input{Fig_QZPMooIiOQpy0.pstricks}
\vspace{1cm}
\input{Fig_QZPMooIiOQpy0.pstricks}
\input{Fig_QZPMooIiOQpy0.pstricks}
\vspace{1cm}
\input{Fig_QZPMooIiOQpy0.pstricks}
\input{Fig_QZPMooIiOQpy0.pstricks}
\vspace{1cm}
\input{Fig_QZPMooIiOQpy0.pstricks}
\input{Fig_QZPMooIiOQpy0.pstricks}
\newpage
\input{Fig_GARYooJCnpFS0.pstricks}
\input{Fig_GARYooJCnpFS0.pstricks}
\vspace{1cm}
\input{Fig_GARYooJCnpFS0.pstricks}
\input{Fig_GARYooJCnpFS0.pstricks}
\vspace{1cm}
\input{Fig_GARYooJCnpFS0.pstricks}
\input{Fig_GARYooJCnpFS0.pstricks}
\vspace{1cm}
\input{Fig_GARYooJCnpFS0.pstricks}
\input{Fig_GARYooJCnpFS0.pstricks}
\vspace{1cm}
\input{Fig_GARYooJCnpFS0.pstricks}
\input{Fig_GARYooJCnpFS0.pstricks}
\vspace{1cm}
\end{center}
\end{document}



% IMPRESSION CHAPITRE OPÉRATIONS FRACTIONS
\begin{document}
% This is part of Un soupçon de mathématique sans être agressif pour autant
% Copyright (c) 2014
%   Laurent Claessens
% See the file fdl-1.3.txt for copying conditions.

%+++++++++++++++++++++++++++++++++++++++++++++++++++++++++++++++++++++++++++++++++++++++++++++++++++++++++++++++++++++++++++ 
\section*{Activité : secteurs d'émissions (1)}
%+++++++++++++++++++++++++++++++++++++++++++++++++++++++++++++++++++++++++++++++++++++++++++++++++++++++++++++++++++++++++++

Sur cette planète, un cinquième des émissions de dioxyde de carbone sont dus aux processus industriels, un autre cinquième aux transports et un dixième aux bâtiments. Quelle fraction du total des émissions de \( CO_2\) est due à ces trois activités ?

Colorier l'égalité suivante :
\begin{center}
    \input{Fig_GLVooNqioTo0.pstricks}+\input{Fig_GLVooNqioTo1.pstricks}+\input{Fig_GLVooNqioTo2.pstricks}=\input{Fig_GLVooNqioTo3.pstricks}.
\end{center}

Les centrales énergétiques sont responsables d'un tiers des émissions de dioxyde de carbone. Quel est le total de ces quatre activités ?

\noindent {\scriptsize Pour plus d'informations, voir\ \url{http://savoirsenmultimedia.ens.fr/uploads/videos//diffusion/2012_02_09_jancovici.mp4}}

\vspace{1cm}
% This is part of Un soupçon de mathématique sans être agressif pour autant
% Copyright (c) 2014
%   Laurent Claessens
% See the file fdl-1.3.txt for copying conditions.

%--------------------------------------------------------------------------------------------------------------------------- 
\subsection*{Activité : fraction d'un grand rectangle}
%---------------------------------------------------------------------------------------------------------------------------

Quel est le périmètre du rectangle grisé ? Quelle est son aire ?
\begin{center}
   \input{Fig_FDOooRCCWGn.pstricks}
\end{center}

\vspace{1cm}
% This is part of Un soupçon de mathématique sans être agressif pour autant
% Copyright (c) 2014
%   Laurent Claessens
% See the file fdl-1.3.txt for copying conditions.

%+++++++++++++++++++++++++++++++++++++++++++++++++++++++++++++++++++++++++++++++++++++++++++++++++++++++++++++++++++++++++++ 
\section*{Activité : secteurs d'émissions (1)}
%+++++++++++++++++++++++++++++++++++++++++++++++++++++++++++++++++++++++++++++++++++++++++++++++++++++++++++++++++++++++++++

Sur cette planète, un cinquième des émissions de dioxyde de carbone sont dus aux processus industriels, un autre cinquième aux transports et un dixième aux bâtiments. Quelle fraction du total des émissions de \( CO_2\) est due à ces trois activités ?

Colorier l'égalité suivante :
\begin{center}
    \input{Fig_GLVooNqioTo0.pstricks}+\input{Fig_GLVooNqioTo1.pstricks}+\input{Fig_GLVooNqioTo2.pstricks}=\input{Fig_GLVooNqioTo3.pstricks}.
\end{center}

Les centrales énergétiques sont responsables d'un tiers des émissions de dioxyde de carbone. Quel est le total de ces quatre activités ?

\noindent {\scriptsize Pour plus d'informations, voir\ \url{http://savoirsenmultimedia.ens.fr/uploads/videos//diffusion/2012_02_09_jancovici.mp4}}

\vspace{1cm}
% This is part of Un soupçon de mathématique sans être agressif pour autant
% Copyright (c) 2014
%   Laurent Claessens
% See the file fdl-1.3.txt for copying conditions.

%--------------------------------------------------------------------------------------------------------------------------- 
\subsection*{Activité : fraction d'un grand rectangle}
%---------------------------------------------------------------------------------------------------------------------------

Quel est le périmètre du rectangle grisé ? Quelle est son aire ?
\begin{center}
   \input{Fig_FDOooRCCWGn.pstricks}
\end{center}

\end{document}


% CORRECTION DU DS_4A2
\begin{document}
\title{Correction du DS numéro 2, 4A}
\maketitle
\Exo{smath-0898}
\Exo{smath-0896}
\Exo{smath-0897}
\Exo{smath-0899}
\Exo{smath-0895}
\end{document}


% CORRECTION DU DS_5AB2
\begin{document}

\title{Correction du DS numéro 2, 5AB}
\maketitle

\Exo{smath-0886}
\Exo{smath-0885}
\Exo{smath-0893}    % gâteau au chocolat. Il faut un peu le modifier avant de le reposer
                    % parce qu'il serait mieux qu'il n'y ait pas deux ingrédients ayant les mêmes quantités (100g de chocolat et de lait).
\Exo{smath-0907}
\Exo{smath-0889}
\Exo{smath-0903}
\end{document}

% IMPRESSION CHAPITRE OPÉRATIONS FRACTIONS
\begin{document}
% This is part of Un soupçon de mathématique sans être agressif pour autant
% Copyright (c) 2014
%   Laurent Claessens
% See the file fdl-1.3.txt for copying conditions.

%+++++++++++++++++++++++++++++++++++++++++++++++++++++++++++++++++++++++++++++++++++++++++++++++++++++++++++++++++++++++++++ 
\section*{Activité : secteurs d'émissions (1)}
%+++++++++++++++++++++++++++++++++++++++++++++++++++++++++++++++++++++++++++++++++++++++++++++++++++++++++++++++++++++++++++

Sur cette planète, un cinquième des émissions de dioxyde de carbone sont dus aux processus industriels, un autre cinquième aux transports et un dixième aux bâtiments. Quelle fraction du total des émissions de \( CO_2\) est due à ces trois activités ?

Colorier l'égalité suivante :
\begin{center}
    \input{Fig_GLVooNqioTo0.pstricks}+\input{Fig_GLVooNqioTo1.pstricks}+\input{Fig_GLVooNqioTo2.pstricks}=\input{Fig_GLVooNqioTo3.pstricks}.
\end{center}

Les centrales énergétiques sont responsables d'un tiers des émissions de dioxyde de carbone. Quel est le total de ces quatre activités ?

\noindent {\scriptsize Pour plus d'informations, voir\ \url{http://savoirsenmultimedia.ens.fr/uploads/videos//diffusion/2012_02_09_jancovici.mp4}}

\vspace{1cm}
% This is part of Un soupçon de mathématique sans être agressif pour autant
% Copyright (c) 2014
%   Laurent Claessens
% See the file fdl-1.3.txt for copying conditions.

%--------------------------------------------------------------------------------------------------------------------------- 
\subsection*{Activité : fraction d'un grand rectangle}
%---------------------------------------------------------------------------------------------------------------------------

Quel est le périmètre du rectangle grisé ? Quelle est son aire ?
\begin{center}
   \input{Fig_FDOooRCCWGn.pstricks}
\end{center}

\vspace{1cm}
% This is part of Un soupçon de mathématique sans être agressif pour autant
% Copyright (c) 2014
%   Laurent Claessens
% See the file fdl-1.3.txt for copying conditions.

%+++++++++++++++++++++++++++++++++++++++++++++++++++++++++++++++++++++++++++++++++++++++++++++++++++++++++++++++++++++++++++ 
\section*{Activité : secteurs d'émissions (1)}
%+++++++++++++++++++++++++++++++++++++++++++++++++++++++++++++++++++++++++++++++++++++++++++++++++++++++++++++++++++++++++++

Sur cette planète, un cinquième des émissions de dioxyde de carbone sont dus aux processus industriels, un autre cinquième aux transports et un dixième aux bâtiments. Quelle fraction du total des émissions de \( CO_2\) est due à ces trois activités ?

Colorier l'égalité suivante :
\begin{center}
    \input{Fig_GLVooNqioTo0.pstricks}+\input{Fig_GLVooNqioTo1.pstricks}+\input{Fig_GLVooNqioTo2.pstricks}=\input{Fig_GLVooNqioTo3.pstricks}.
\end{center}

Les centrales énergétiques sont responsables d'un tiers des émissions de dioxyde de carbone. Quel est le total de ces quatre activités ?

\noindent {\scriptsize Pour plus d'informations, voir\ \url{http://savoirsenmultimedia.ens.fr/uploads/videos//diffusion/2012_02_09_jancovici.mp4}}

\vspace{1cm}
% This is part of Un soupçon de mathématique sans être agressif pour autant
% Copyright (c) 2014
%   Laurent Claessens
% See the file fdl-1.3.txt for copying conditions.

%--------------------------------------------------------------------------------------------------------------------------- 
\subsection*{Activité : fraction d'un grand rectangle}
%---------------------------------------------------------------------------------------------------------------------------

Quel est le périmètre du rectangle grisé ? Quelle est son aire ?
\begin{center}
   \input{Fig_FDOooRCCWGn.pstricks}
\end{center}

\end{document}

% ACTIVITÉ GRANDEURS PROPORTIONNELLES
\begin{document}
\vspace{-0.5cm}
% This is part of Un soupçon de mathématique sans être agressif pour autant
% Copyright (c) 2014
%   Laurent Claessens
% See the file fdl-1.3.txt for copying conditions.

%--------------------------------------------------------------------------------------------------------------------------- 
\subsection*{Tableaux et représentation graphique}
%---------------------------------------------------------------------------------------------------------------------------

Associer à chaque tableau le graphique qui correspond. Quels sont ceux qui correspondent à des situations de proportionnalité ?

\begin{equation}
    \input{Fig_ARKZooKZOuAkconvprix.latex}\quad
    \input{ARKZooKZOuAkfar.latex}
\end{equation}
\begin{equation}
    \input{ARKZooKZOuAkmilm.latex}\quad
    \input{ARKZooKZOuAkctlibre.latex}
\end{equation}

\begin{center}
   \input{Fig_ARKZooKZOuAk0.pstricks}
   \input{Fig_ARKZooKZOuAk3.pstricks}
   \input{Fig_ARKZooKZOuAk2.pstricks}
   \input{Fig_ARKZooKZOuAk1.pstricks}
\end{center}

\vspace{-0.5cm}
% This is part of Un soupçon de mathématique sans être agressif pour autant
% Copyright (c) 2014
%   Laurent Claessens
% See the file fdl-1.3.txt for copying conditions.

%--------------------------------------------------------------------------------------------------------------------------- 
\subsection*{Tableaux et représentation graphique}
%---------------------------------------------------------------------------------------------------------------------------

Associer à chaque tableau le graphique qui correspond. Quels sont ceux qui correspondent à des situations de proportionnalité ?

\begin{equation}
    \input{Fig_ARKZooKZOuAkconvprix.latex}\quad
    \input{ARKZooKZOuAkfar.latex}
\end{equation}
\begin{equation}
    \input{ARKZooKZOuAkmilm.latex}\quad
    \input{ARKZooKZOuAkctlibre.latex}
\end{equation}

\begin{center}
   \input{Fig_ARKZooKZOuAk0.pstricks}
   \input{Fig_ARKZooKZOuAk3.pstricks}
   \input{Fig_ARKZooKZOuAk2.pstricks}
   \input{Fig_ARKZooKZOuAk1.pstricks}
\end{center}

\end{document}

% LA FIGURE CERCLE CIRCONSCRIT
\newcommand{\uneligne}{
\begin{center}
   \input{Fig_QTCQooFtDgwk.pstricks}
   \hfill
   \input{Fig_QTCQooFtDgwk.pstricks}
\end{center}
}
\begin{document}
\uneligne
\vspace{2cm}
\uneligne
\vspace{2cm}
\uneligne
\vspace{2cm}
\uneligne
\end{document}



% ACTIV NAPOLÉON
\begin{document}
% This is part of Un soupçon de mathématique sans être agressif pour autant
% Copyright (c) 2014
%   Laurent Claessens
% See the file fdl-1.3.txt for copying conditions.

%--------------------------------------------------------------------------------------------------------------------------- 
\subsection*{Activité : chevaliers de la table ronde}
%---------------------------------------------------------------------------------------------------------------------------

Les chevaliers de la table ronde se sont disputés; Tristan, Perceval et Lancelot sont allés mettre leurs chaises un peu à l'écart et refusent de bouger. Sur le dessin suivant, l'unité est en mètres. Quel doit être le rayon de la nouvelle table ronde de telle sorte à ce que ces trois chevaliers puissent siéger ?

\begin{center}
   \input{Fig_DTFZooTlciUT.pstricks}
\end{center}

\vspace{2cm}
% This is part of Un soupçon de mathématique sans être agressif pour autant
% Copyright (c) 2014
%   Laurent Claessens
% See the file fdl-1.3.txt for copying conditions.

%--------------------------------------------------------------------------------------------------------------------------- 
\subsection*{Activité : chevaliers de la table ronde}
%---------------------------------------------------------------------------------------------------------------------------

Les chevaliers de la table ronde se sont disputés; Tristan, Perceval et Lancelot sont allés mettre leurs chaises un peu à l'écart et refusent de bouger. Sur le dessin suivant, l'unité est en mètres. Quel doit être le rayon de la nouvelle table ronde de telle sorte à ce que ces trois chevaliers puissent siéger ?

\begin{center}
   \input{Fig_DTFZooTlciUT.pstricks}
\end{center}

\end{document}


% ACTIV NAPOLÉON
\begin{document}
% This is part of Un soupçon de mathématique sans être agressif pour autant
% Copyright (c) 2014
%   Laurent Claessens
% See the file fdl-1.3.txt for copying conditions.

%--------------------------------------------------------------------------------------------------------------------------- 
\subsection*{Activité : Napoléon se place}
%---------------------------------------------------------------------------------------------------------------------------

Les Anglais et les Autrichiens ont pris position en deux points \( A\) et \( B\) distants de \( \SI{10}{\kilo\meter}\). Napoléon veut pouvoir être à même de les attaquer tous les deux d'égale manière et décide donc de se positionner en un point $N$ qui serait à égale distance de \( A\) que de \( B\).

Bien entendu il pourrait se placer au milieu du segment \( [AB]\), mais ainsi il se ferait trop facilement attaquer des deux côtés à la fois (pas fou le Corse!). Où peut-il se placer ? Faire un dessin pour l'aider.

\vspace{2cm}
% This is part of Un soupçon de mathématique sans être agressif pour autant
% Copyright (c) 2014
%   Laurent Claessens
% See the file fdl-1.3.txt for copying conditions.

%--------------------------------------------------------------------------------------------------------------------------- 
\subsection*{Activité : Napoléon se place}
%---------------------------------------------------------------------------------------------------------------------------

Les Anglais et les Autrichiens ont pris position en deux points \( A\) et \( B\) distants de \( \SI{10}{\kilo\meter}\). Napoléon veut pouvoir être à même de les attaquer tous les deux d'égale manière et décide donc de se positionner en un point $N$ qui serait à égale distance de \( A\) que de \( B\).

Bien entendu il pourrait se placer au milieu du segment \( [AB]\), mais ainsi il se ferait trop facilement attaquer des deux côtés à la fois (pas fou le Corse!). Où peut-il se placer ? Faire un dessin pour l'aider.

\vspace{2cm}
% This is part of Un soupçon de mathématique sans être agressif pour autant
% Copyright (c) 2014
%   Laurent Claessens
% See the file fdl-1.3.txt for copying conditions.

%--------------------------------------------------------------------------------------------------------------------------- 
\subsection*{Activité : Napoléon se place}
%---------------------------------------------------------------------------------------------------------------------------

Les Anglais et les Autrichiens ont pris position en deux points \( A\) et \( B\) distants de \( \SI{10}{\kilo\meter}\). Napoléon veut pouvoir être à même de les attaquer tous les deux d'égale manière et décide donc de se positionner en un point $N$ qui serait à égale distance de \( A\) que de \( B\).

Bien entendu il pourrait se placer au milieu du segment \( [AB]\), mais ainsi il se ferait trop facilement attaquer des deux côtés à la fois (pas fou le Corse!). Où peut-il se placer ? Faire un dessin pour l'aider.

\vspace{2cm}
% This is part of Un soupçon de mathématique sans être agressif pour autant
% Copyright (c) 2014
%   Laurent Claessens
% See the file fdl-1.3.txt for copying conditions.

%--------------------------------------------------------------------------------------------------------------------------- 
\subsection*{Activité : Napoléon se place}
%---------------------------------------------------------------------------------------------------------------------------

Les Anglais et les Autrichiens ont pris position en deux points \( A\) et \( B\) distants de \( \SI{10}{\kilo\meter}\). Napoléon veut pouvoir être à même de les attaquer tous les deux d'égale manière et décide donc de se positionner en un point $N$ qui serait à égale distance de \( A\) que de \( B\).

Bien entendu il pourrait se placer au milieu du segment \( [AB]\), mais ainsi il se ferait trop facilement attaquer des deux côtés à la fois (pas fou le Corse!). Où peut-il se placer ? Faire un dessin pour l'aider.

\vspace{2cm}
% This is part of Un soupçon de mathématique sans être agressif pour autant
% Copyright (c) 2014
%   Laurent Claessens
% See the file fdl-1.3.txt for copying conditions.

%--------------------------------------------------------------------------------------------------------------------------- 
\subsection*{Activité : Napoléon se place}
%---------------------------------------------------------------------------------------------------------------------------

Les Anglais et les Autrichiens ont pris position en deux points \( A\) et \( B\) distants de \( \SI{10}{\kilo\meter}\). Napoléon veut pouvoir être à même de les attaquer tous les deux d'égale manière et décide donc de se positionner en un point $N$ qui serait à égale distance de \( A\) que de \( B\).

Bien entendu il pourrait se placer au milieu du segment \( [AB]\), mais ainsi il se ferait trop facilement attaquer des deux côtés à la fois (pas fou le Corse!). Où peut-il se placer ? Faire un dessin pour l'aider.

\end{document}

% ACTIV MULTIPLICATION FRACTION
\begin{document}
% This is part of Un soupçon de mathématique sans être agressif pour autant
% Copyright (c) 2014
%   Laurent Claessens
% See the file fdl-1.3.txt for copying conditions.

%--------------------------------------------------------------------------------------------------------------------------- 
\subsection*{Activité : fraction d'un grand rectangle}
%---------------------------------------------------------------------------------------------------------------------------

Quel est le périmètre du rectangle grisé ? Quelle est son aire ?
\begin{center}
   \input{Fig_FDOooRCCWGn.pstricks}
\end{center}

\vspace{2cm}
% This is part of Un soupçon de mathématique sans être agressif pour autant
% Copyright (c) 2014
%   Laurent Claessens
% See the file fdl-1.3.txt for copying conditions.

%--------------------------------------------------------------------------------------------------------------------------- 
\subsection*{Activité : fraction d'un grand rectangle}
%---------------------------------------------------------------------------------------------------------------------------

Quel est le périmètre du rectangle grisé ? Quelle est son aire ?
\begin{center}
   \input{Fig_FDOooRCCWGn.pstricks}
\end{center}

\vspace{2cm}
% This is part of Un soupçon de mathématique sans être agressif pour autant
% Copyright (c) 2014
%   Laurent Claessens
% See the file fdl-1.3.txt for copying conditions.

%--------------------------------------------------------------------------------------------------------------------------- 
\subsection*{Activité : fraction d'un grand rectangle}
%---------------------------------------------------------------------------------------------------------------------------

Quel est le périmètre du rectangle grisé ? Quelle est son aire ?
\begin{center}
   \input{Fig_FDOooRCCWGn.pstricks}
\end{center}

\vspace{2cm}
% This is part of Un soupçon de mathématique sans être agressif pour autant
% Copyright (c) 2014
%   Laurent Claessens
% See the file fdl-1.3.txt for copying conditions.

%--------------------------------------------------------------------------------------------------------------------------- 
\subsection*{Activité : fraction d'un grand rectangle}
%---------------------------------------------------------------------------------------------------------------------------

Quel est le périmètre du rectangle grisé ? Quelle est son aire ?
\begin{center}
   \input{Fig_FDOooRCCWGn.pstricks}
\end{center}

\end{document}



% ACTIV FRACTION 5B
\pagestyle{empty}
\begin{document}



\end{document}

\newpage



%\begin{feuilleDS}{Devoir surveillé numéro 2, 4A\\ \small Vendredi 17 octobre 2014}
%\end{feuilleDS}

% ACTIVITÉ OPÉRATIONS SUR LES FRACTIONS

% This is part of Un soupçon de mathématique sans être agressif pour autant
% Copyright (c) 2014
%   Laurent Claessens
% See the file fdl-1.3.txt for copying conditions.

%+++++++++++++++++++++++++++++++++++++++++++++++++++++++++++++++++++++++++++++++++++++++++++++++++++++++++++++++++++++++++++ 
\section*{Activité : secteurs d'émissions (1)}
%+++++++++++++++++++++++++++++++++++++++++++++++++++++++++++++++++++++++++++++++++++++++++++++++++++++++++++++++++++++++++++

Sur cette planète, un cinquième des émissions de dioxyde de carbone sont dus aux processus industriels, un autre cinquième aux transports et un dixième aux bâtiments. Quelle fraction du total des émissions de \( CO_2\) est due à ces trois activités ?

Colorier l'égalité suivante :
\begin{center}
    \input{Fig_GLVooNqioTo0.pstricks}+\input{Fig_GLVooNqioTo1.pstricks}+\input{Fig_GLVooNqioTo2.pstricks}=\input{Fig_GLVooNqioTo3.pstricks}.
\end{center}

Les centrales énergétiques sont responsables d'un tiers des émissions de dioxyde de carbone. Quel est le total de ces quatre activités ?

\noindent {\scriptsize Pour plus d'informations, voir\ \url{http://savoirsenmultimedia.ens.fr/uploads/videos//diffusion/2012_02_09_jancovici.mp4}}

\vspace{2cm}
% This is part of Un soupçon de mathématique sans être agressif pour autant
% Copyright (c) 2014
%   Laurent Claessens
% See the file fdl-1.3.txt for copying conditions.

%+++++++++++++++++++++++++++++++++++++++++++++++++++++++++++++++++++++++++++++++++++++++++++++++++++++++++++++++++++++++++++ 
\section*{Activité : secteurs d'émissions (1)}
%+++++++++++++++++++++++++++++++++++++++++++++++++++++++++++++++++++++++++++++++++++++++++++++++++++++++++++++++++++++++++++

Sur cette planète, un cinquième des émissions de dioxyde de carbone sont dus aux processus industriels, un autre cinquième aux transports et un dixième aux bâtiments. Quelle fraction du total des émissions de \( CO_2\) est due à ces trois activités ?

Colorier l'égalité suivante :
\begin{center}
    \input{Fig_GLVooNqioTo0.pstricks}+\input{Fig_GLVooNqioTo1.pstricks}+\input{Fig_GLVooNqioTo2.pstricks}=\input{Fig_GLVooNqioTo3.pstricks}.
\end{center}

Les centrales énergétiques sont responsables d'un tiers des émissions de dioxyde de carbone. Quel est le total de ces quatre activités ?

\noindent {\scriptsize Pour plus d'informations, voir\ \url{http://savoirsenmultimedia.ens.fr/uploads/videos//diffusion/2012_02_09_jancovici.mp4}}

\vspace{2cm}
% This is part of Un soupçon de mathématique sans être agressif pour autant
% Copyright (c) 2014
%   Laurent Claessens
% See the file fdl-1.3.txt for copying conditions.

%+++++++++++++++++++++++++++++++++++++++++++++++++++++++++++++++++++++++++++++++++++++++++++++++++++++++++++++++++++++++++++ 
\section*{Activité : secteurs d'émissions (1)}
%+++++++++++++++++++++++++++++++++++++++++++++++++++++++++++++++++++++++++++++++++++++++++++++++++++++++++++++++++++++++++++

Sur cette planète, un cinquième des émissions de dioxyde de carbone sont dus aux processus industriels, un autre cinquième aux transports et un dixième aux bâtiments. Quelle fraction du total des émissions de \( CO_2\) est due à ces trois activités ?

Colorier l'égalité suivante :
\begin{center}
    \input{Fig_GLVooNqioTo0.pstricks}+\input{Fig_GLVooNqioTo1.pstricks}+\input{Fig_GLVooNqioTo2.pstricks}=\input{Fig_GLVooNqioTo3.pstricks}.
\end{center}

Les centrales énergétiques sont responsables d'un tiers des émissions de dioxyde de carbone. Quel est le total de ces quatre activités ?

\noindent {\scriptsize Pour plus d'informations, voir\ \url{http://savoirsenmultimedia.ens.fr/uploads/videos//diffusion/2012_02_09_jancovici.mp4}}


\end{document}



% ACTIVITÉ FRACTIONS

% This is part of Un soupçon de mathématique sans être agressif pour autant
% Copyright (c) 2014
%   Laurent Claessens
% See the file fdl-1.3.txt for copying conditions.

%--------------------------------------------------------------------------------------------------------------------------- 
\subsection*{Partage d'un carré}
%---------------------------------------------------------------------------------------------------------------------------

\begin{wrapfigure}[2]{r}{4.0cm}
   \vspace{-1cm}        % à adapter.
   \centering
   \input{Fig_LXQooZPbZml.pstricks}
\end{wrapfigure}

Répondre aux questions à partir du dessin ci-contre.
\begin{enumerate}
    \item
                L'aire de la région hachurée représente \( \dfrac{ 1 }{ \ldots }\) de l'aire totale.
            \item
                L'aire de la région remplie représente \( \dfrac{ 3 }{ \ldots }\) de l'aire totale.
    \item
        Ensemble, ces deux régions forment \( \dfrac{ \ldots }{ \ldots }\) de l'aire totale.
\end{enumerate}

% This is part of Un soupçon de mathématique sans être agressif pour autant
% Copyright (c) 2014
%   Laurent Claessens
% See the file fdl-1.3.txt for copying conditions.

%--------------------------------------------------------------------------------------------------------------------------- 
\subsection*{Confiture sucrée}
%---------------------------------------------------------------------------------------------------------------------------

Après un bel été bien ensoleillé, Philippe souhaite faire de la confiture pas trop sucrée. En regardant sur Internet, il trouve trois recettes.

\begin{center}
    \begin{tabular}[]{|c|c|}
        \hline
        Confiture de fraises&«\unit{450}{\gram} de sucre pour \unit{750}{\gram} de fraises» \\
        \hline
        Confiture d'abricots& «\unit{500}{\gram} de sucre pour \unit{1}{\kilo\gram} de confiture» \\
        \hline
        Confiture de cerises&  «\unit{800}{\gram} de sucre pour \unit{2400}{\gram} de cerises» \\ 
        \hline
    \end{tabular}
\end{center}


\begin{enumerate}
    \item
Pour chaque recette, exprimer la proportion de sucre ajouté dans la confiture sous forme de fraction.
\item
    Simplifier le plus possible les fractions obtenues à la question précédente.
\item
    Que signifie une proportion de sucre ajouté supérieure à \( \dfrac{ 1 }{ 2 }\) ?
\end{enumerate}


Philippe cherche à savoir quelle est la recette avec le moins de sucre ajouté. Il fait le raisonnement suivant : « C'est dans la confiture de fraises qu'on retrouve la masse de sucre ajouté la moins importante (\unit{450}{\gram}), c'est donc dans la confiture de fraises qu'il y a le moins de sucre ajouté. ». 

\begin{enumerate}
    \item
        
Que penser de ce raisonnement ?
\item
Pour aider Philippe dans son choix, récrire les ingrédients nécessaires à la réalisation de \unit{1}{\kilo} de confiture.
\item
Quelle est la confiture qui contient le moins de sucre ajouté en proportion ?
\end{enumerate}


\vspace{1cm}
% This is part of Un soupçon de mathématique sans être agressif pour autant
% Copyright (c) 2014
%   Laurent Claessens
% See the file fdl-1.3.txt for copying conditions.

%--------------------------------------------------------------------------------------------------------------------------- 
\subsection*{Partage d'un carré}
%---------------------------------------------------------------------------------------------------------------------------

\begin{wrapfigure}[2]{r}{4.0cm}
   \vspace{-1cm}        % à adapter.
   \centering
   \input{Fig_LXQooZPbZml.pstricks}
\end{wrapfigure}

Répondre aux questions à partir du dessin ci-contre.
\begin{enumerate}
    \item
                L'aire de la région hachurée représente \( \dfrac{ 1 }{ \ldots }\) de l'aire totale.
            \item
                L'aire de la région remplie représente \( \dfrac{ 3 }{ \ldots }\) de l'aire totale.
    \item
        Ensemble, ces deux régions forment \( \dfrac{ \ldots }{ \ldots }\) de l'aire totale.
\end{enumerate}

% This is part of Un soupçon de mathématique sans être agressif pour autant
% Copyright (c) 2014
%   Laurent Claessens
% See the file fdl-1.3.txt for copying conditions.

%--------------------------------------------------------------------------------------------------------------------------- 
\subsection*{Confiture sucrée}
%---------------------------------------------------------------------------------------------------------------------------

Après un bel été bien ensoleillé, Philippe souhaite faire de la confiture pas trop sucrée. En regardant sur Internet, il trouve trois recettes.

\begin{center}
    \begin{tabular}[]{|c|c|}
        \hline
        Confiture de fraises&«\unit{450}{\gram} de sucre pour \unit{750}{\gram} de fraises» \\
        \hline
        Confiture d'abricots& «\unit{500}{\gram} de sucre pour \unit{1}{\kilo\gram} de confiture» \\
        \hline
        Confiture de cerises&  «\unit{800}{\gram} de sucre pour \unit{2400}{\gram} de cerises» \\ 
        \hline
    \end{tabular}
\end{center}


\begin{enumerate}
    \item
Pour chaque recette, exprimer la proportion de sucre ajouté dans la confiture sous forme de fraction.
\item
    Simplifier le plus possible les fractions obtenues à la question précédente.
\item
    Que signifie une proportion de sucre ajouté supérieure à \( \dfrac{ 1 }{ 2 }\) ?
\end{enumerate}


Philippe cherche à savoir quelle est la recette avec le moins de sucre ajouté. Il fait le raisonnement suivant : « C'est dans la confiture de fraises qu'on retrouve la masse de sucre ajouté la moins importante (\unit{450}{\gram}), c'est donc dans la confiture de fraises qu'il y a le moins de sucre ajouté. ». 

\begin{enumerate}
    \item
        
Que penser de ce raisonnement ?
\item
Pour aider Philippe dans son choix, récrire les ingrédients nécessaires à la réalisation de \unit{1}{\kilo} de confiture.
\item
Quelle est la confiture qui contient le moins de sucre ajouté en proportion ?
\end{enumerate}


\end{document}


% RATTRAPAGE DS1 POUR CINQUIÈME.
\begin{feuilleDS}{Devoir surveillé numéro 1 (rattrapage), 5A,5B\\ \small Mardi 8 octobre 2014}
\Exo{smath-0847}
\Exo{smath-0824}
\Exo{smath-0846}
\Exo{smath-0845}
\Exo{smath-0823}
\end{feuilleDS}


% DS 5A et 5B VENDREDI 3 NOVEMBRE 2014
\begin{feuilleDS}{Devoir surveillé numéro 1, 5A,5B\\ \small Vendredi 3 octobre 2014}
\Exo{smath-0820}
\Exo{smath-0821}
\Exo{smath-0822}
\Exo{smath-0823}
\Exo{smath-0824}
\end{feuilleDS}
\end{document}

% DS 4A LUNDI 29 SEPTEMBRE 2014

\begin{feuilleDS}{Devoir surveillé numéro 1, 4A\\ \small Lundi 29 septembre 2014}

\Exo{smath-0772}
\Exo{smath-0778}

\Exo{smath-0827}
\Exo{smath-0828}
\Exo{smath-0831}
\Exo{smath-0834}

\end{feuilleDS}


% LE RATTRAPAGE

\begin{minipage}{0.485\textwidth}
    \enteteInterro{Le 19 septembre 2014}{1}{A}

\Exo{smath-0829}
\end{minipage}
\begin{minipage}{0.485\textwidth}
    \enteteInterro{Le 19 septembre 2014}{1}{B}
\Exo{smath-0830}

\end{minipage}



\end{document}


% ACTIVITÉ FRACTIONS

% This is part of Un soupçon de mathématique sans être agressif pour autant
% Copyright (c) 2014
%   Laurent Claessens
% See the file fdl-1.3.txt for copying conditions.

%--------------------------------------------------------------------------------------------------------------------------- 
\subsection*{Partage d'un carré}
%---------------------------------------------------------------------------------------------------------------------------

\begin{wrapfigure}[2]{r}{4.0cm}
   \vspace{-1cm}        % à adapter.
   \centering
   \input{Fig_LXQooZPbZml.pstricks}
\end{wrapfigure}

Répondre aux questions à partir du dessin ci-contre.
\begin{enumerate}
    \item
                L'aire de la région hachurée représente \( \dfrac{ 1 }{ \ldots }\) de l'aire totale.
            \item
                L'aire de la région remplie représente \( \dfrac{ 3 }{ \ldots }\) de l'aire totale.
    \item
        Ensemble, ces deux régions forment \( \dfrac{ \ldots }{ \ldots }\) de l'aire totale.
\end{enumerate}

% This is part of Un soupçon de mathématique sans être agressif pour autant
% Copyright (c) 2014
%   Laurent Claessens
% See the file fdl-1.3.txt for copying conditions.

%--------------------------------------------------------------------------------------------------------------------------- 
\subsection*{Confiture sucrée}
%---------------------------------------------------------------------------------------------------------------------------

Après un bel été bien ensoleillé, Philippe souhaite faire de la confiture pas trop sucrée. En regardant sur Internet, il trouve trois recettes.

\begin{center}
    \begin{tabular}[]{|c|c|}
        \hline
        Confiture de fraises&«\unit{450}{\gram} de sucre pour \unit{750}{\gram} de fraises» \\
        \hline
        Confiture d'abricots& «\unit{500}{\gram} de sucre pour \unit{1}{\kilo\gram} de confiture» \\
        \hline
        Confiture de cerises&  «\unit{800}{\gram} de sucre pour \unit{2400}{\gram} de cerises» \\ 
        \hline
    \end{tabular}
\end{center}


\begin{enumerate}
    \item
Pour chaque recette, exprimer la proportion de sucre ajouté dans la confiture sous forme de fraction.
\item
    Simplifier le plus possible les fractions obtenues à la question précédente.
\item
    Que signifie une proportion de sucre ajouté supérieure à \( \dfrac{ 1 }{ 2 }\) ?
\end{enumerate}


Philippe cherche à savoir quelle est la recette avec le moins de sucre ajouté. Il fait le raisonnement suivant : « C'est dans la confiture de fraises qu'on retrouve la masse de sucre ajouté la moins importante (\unit{450}{\gram}), c'est donc dans la confiture de fraises qu'il y a le moins de sucre ajouté. ». 

\begin{enumerate}
    \item
        
Que penser de ce raisonnement ?
\item
Pour aider Philippe dans son choix, récrire les ingrédients nécessaires à la réalisation de \unit{1}{\kilo} de confiture.
\item
Quelle est la confiture qui contient le moins de sucre ajouté en proportion ?
\end{enumerate}


\vspace{1cm}
% This is part of Un soupçon de mathématique sans être agressif pour autant
% Copyright (c) 2014
%   Laurent Claessens
% See the file fdl-1.3.txt for copying conditions.

%--------------------------------------------------------------------------------------------------------------------------- 
\subsection*{Partage d'un carré}
%---------------------------------------------------------------------------------------------------------------------------

\begin{wrapfigure}[2]{r}{4.0cm}
   \vspace{-1cm}        % à adapter.
   \centering
   \input{Fig_LXQooZPbZml.pstricks}
\end{wrapfigure}

Répondre aux questions à partir du dessin ci-contre.
\begin{enumerate}
    \item
                L'aire de la région hachurée représente \( \dfrac{ 1 }{ \ldots }\) de l'aire totale.
            \item
                L'aire de la région remplie représente \( \dfrac{ 3 }{ \ldots }\) de l'aire totale.
    \item
        Ensemble, ces deux régions forment \( \dfrac{ \ldots }{ \ldots }\) de l'aire totale.
\end{enumerate}

% This is part of Un soupçon de mathématique sans être agressif pour autant
% Copyright (c) 2014
%   Laurent Claessens
% See the file fdl-1.3.txt for copying conditions.

%--------------------------------------------------------------------------------------------------------------------------- 
\subsection*{Confiture sucrée}
%---------------------------------------------------------------------------------------------------------------------------

Après un bel été bien ensoleillé, Philippe souhaite faire de la confiture pas trop sucrée. En regardant sur Internet, il trouve trois recettes.

\begin{center}
    \begin{tabular}[]{|c|c|}
        \hline
        Confiture de fraises&«\unit{450}{\gram} de sucre pour \unit{750}{\gram} de fraises» \\
        \hline
        Confiture d'abricots& «\unit{500}{\gram} de sucre pour \unit{1}{\kilo\gram} de confiture» \\
        \hline
        Confiture de cerises&  «\unit{800}{\gram} de sucre pour \unit{2400}{\gram} de cerises» \\ 
        \hline
    \end{tabular}
\end{center}


\begin{enumerate}
    \item
Pour chaque recette, exprimer la proportion de sucre ajouté dans la confiture sous forme de fraction.
\item
    Simplifier le plus possible les fractions obtenues à la question précédente.
\item
    Que signifie une proportion de sucre ajouté supérieure à \( \dfrac{ 1 }{ 2 }\) ?
\end{enumerate}


Philippe cherche à savoir quelle est la recette avec le moins de sucre ajouté. Il fait le raisonnement suivant : « C'est dans la confiture de fraises qu'on retrouve la masse de sucre ajouté la moins importante (\unit{450}{\gram}), c'est donc dans la confiture de fraises qu'il y a le moins de sucre ajouté. ». 

\begin{enumerate}
    \item
        
Que penser de ce raisonnement ?
\item
Pour aider Philippe dans son choix, récrire les ingrédients nécessaires à la réalisation de \unit{1}{\kilo} de confiture.
\item
Quelle est la confiture qui contient le moins de sucre ajouté en proportion ?
\end{enumerate}


\end{document}

% ACTIVITÉ CALCUL LITTÉRAL

% This is part of Un soupçon de mathématique sans être agressif pour autant
% Copyright (c) 2014
%   Laurent Claessens
% See the file fdl-1.3.txt for copying conditions.

%--------------------------------------------------------------------------------------------------------------------------- 
\subsection*{Petits et grands carrés}
%---------------------------------------------------------------------------------------------------------------------------

Nous découpons le bord d'un grand carré en petits carrés comme indiqué sur les dessins :

% Les figures sont de phystricksHVXooRtjPkd.py

\begin{center}
   \input{Fig_XYWooOPFwaca.pstricks} \input{Fig_BPZooBCyuyK.pstricks}
\end{center}

\begin{enumerate}
    \item
Réaliser une figure avec cinq petits carrés sur un côté et indiquer le nombre total de carrés coloriés. Recommencer avec une figure de six petits carrés de côté.

\item
S'il y a $100$ petits carrés sur le côté, combien y-a-t-il de carrés coloriés au total ?
\item
    Nous appelons \( n\) le nombre de petits carrés d'un côté du grand carré, et nous voulons trouver une formule donnant le nombre total de carrés coloriés. Valérian dit :
    \begin{quote}
        « Il y a \( 4\) côtés et \( n\) carrés par côtés, donc \( 4n\) petits carrés.» 
    \end{quote}
    Laureline n'est pas d'accord :
    \begin{quote}
       « Tu en as trop !»
    \end{quote}
    Qui a raison ? Pourquoi ? Donner une formule correcte.
\end{enumerate}

\vspace{2.5cm}
% This is part of Un soupçon de mathématique sans être agressif pour autant
% Copyright (c) 2014
%   Laurent Claessens
% See the file fdl-1.3.txt for copying conditions.

%--------------------------------------------------------------------------------------------------------------------------- 
\subsection*{Petits et grands carrés}
%---------------------------------------------------------------------------------------------------------------------------

Nous découpons le bord d'un grand carré en petits carrés comme indiqué sur les dessins :

% Les figures sont de phystricksHVXooRtjPkd.py

\begin{center}
   \input{Fig_XYWooOPFwaca.pstricks} \input{Fig_BPZooBCyuyK.pstricks}
\end{center}

\begin{enumerate}
    \item
Réaliser une figure avec cinq petits carrés sur un côté et indiquer le nombre total de carrés coloriés. Recommencer avec une figure de six petits carrés de côté.

\item
S'il y a $100$ petits carrés sur le côté, combien y-a-t-il de carrés coloriés au total ?
\item
    Nous appelons \( n\) le nombre de petits carrés d'un côté du grand carré, et nous voulons trouver une formule donnant le nombre total de carrés coloriés. Valérian dit :
    \begin{quote}
        « Il y a \( 4\) côtés et \( n\) carrés par côtés, donc \( 4n\) petits carrés.» 
    \end{quote}
    Laureline n'est pas d'accord :
    \begin{quote}
       « Tu en as trop !»
    \end{quote}
    Qui a raison ? Pourquoi ? Donner une formule correcte.
\end{enumerate}

\vspace{2.5cm}
% This is part of Un soupçon de mathématique sans être agressif pour autant
% Copyright (c) 2014
%   Laurent Claessens
% See the file fdl-1.3.txt for copying conditions.

%--------------------------------------------------------------------------------------------------------------------------- 
\subsection*{Petits et grands carrés}
%---------------------------------------------------------------------------------------------------------------------------

Nous découpons le bord d'un grand carré en petits carrés comme indiqué sur les dessins :

% Les figures sont de phystricksHVXooRtjPkd.py

\begin{center}
   \input{Fig_XYWooOPFwaca.pstricks} \input{Fig_BPZooBCyuyK.pstricks}
\end{center}

\begin{enumerate}
    \item
Réaliser une figure avec cinq petits carrés sur un côté et indiquer le nombre total de carrés coloriés. Recommencer avec une figure de six petits carrés de côté.

\item
S'il y a $100$ petits carrés sur le côté, combien y-a-t-il de carrés coloriés au total ?
\item
    Nous appelons \( n\) le nombre de petits carrés d'un côté du grand carré, et nous voulons trouver une formule donnant le nombre total de carrés coloriés. Valérian dit :
    \begin{quote}
        « Il y a \( 4\) côtés et \( n\) carrés par côtés, donc \( 4n\) petits carrés.» 
    \end{quote}
    Laureline n'est pas d'accord :
    \begin{quote}
       « Tu en as trop !»
    \end{quote}
    Qui a raison ? Pourquoi ? Donner une formule correcte.
\end{enumerate}


\end{document}

% INTERRO DU VENDREDI 19 sept 2014


\begin{minipage}{0.485\textwidth}
    \enteteInterro{Le 19 septembre 2014}{1}{A}

    Calculer :
    \begin{enumerate}
        \item
            \( 8-3\times 2=\ldots\)
        \item
            \( \dfrac{ 16+4 }{ 5 }=\ldots\)
        \item
            \( \ldots \times 4+12=40\)
        \item
            \( (8-3)\times 2=\ldots\)
        \item
            \(  4\times 3-4=\ldots \)
    \end{enumerate}

\end{minipage}
\begin{minipage}{0.485\textwidth}
    \enteteInterro{Le 19 septembre 2014}{1}{B}

    Calculer :
    \begin{enumerate}
        \item
            \( 17-4\times 4=\ldots\)
        \item
            \( \dfrac{ 12+6 }{ 2 }=\ldots\)
        \item
            \( \ldots \times 6+5=41\)
        \item
            \( (10-6)\times 9=\ldots\)
        \item
            \(  9\times 3-7=\ldots \)
    \end{enumerate}

\end{minipage}


\vspace{2.5cm}
\begin{minipage}{0.485\textwidth}
    \enteteInterro{Le 19 septembre 2014}{1}{A}

    Calculer :
    \begin{enumerate}
        \item
            \( 8-3\times 2=\ldots\)
        \item
            \( \dfrac{ 16+4 }{ 5 }=\ldots\)
        \item
            \( \ldots \times 4+12=40\)
        \item
            \( (8-3)\times 2=\ldots\)
        \item
            \(  4\times 3-4=\ldots \)
    \end{enumerate}

\end{minipage}
\begin{minipage}{0.485\textwidth}
    \enteteInterro{Le 19 septembre 2014}{1}{B}

    Calculer :
    \begin{enumerate}
        \item
            \( 17-4\times 4=\ldots\)
        \item
            \( \dfrac{ 12+6 }{ 2 }=\ldots\)
        \item
            \( \ldots \times 6+5=41\)
        \item
            \( (10-6)\times 9=\ldots\)
        \item
            \(  9\times 3-7=\ldots \)
    \end{enumerate}

\end{minipage}


\vspace{2.5cm}
\begin{minipage}{0.485\textwidth}
    \enteteInterro{Le 19 septembre 2014}{1}{A}

    Calculer :
    \begin{enumerate}
        \item
            \( 8-3\times 2=\ldots\)
        \item
            \( \dfrac{ 16+4 }{ 5 }=\ldots\)
        \item
            \( \ldots \times 4+12=40\)
        \item
            \( (8-3)\times 2=\ldots\)
        \item
            \(  4\times 3-4=\ldots \)
    \end{enumerate}

\end{minipage}
\begin{minipage}{0.485\textwidth}
    \enteteInterro{Le 19 septembre 2014}{1}{B}

    Calculer :
    \begin{enumerate}
        \item
            \( 17-4\times 4=\ldots\)
        \item
            \( \dfrac{ 12+6 }{ 2 }=\ldots\)
        \item
            \( \ldots \times 6+5=41\)
        \item
            \( (10-6)\times 9=\ldots\)
        \item
            \(  9\times 3-7=\ldots \)
    \end{enumerate}

\end{minipage}


\newpage

\begin{minipage}{0.485\textwidth}
    \enteteInterro{Le 19 septembre 2014}{1}{A}

\Exo{smath-0815}

\end{minipage}
\begin{minipage}{0.485\textwidth}
    \enteteInterro{Le 19 septembre 2014}{1}{B}

\Exo{smath-0816}
\end{minipage}


\vspace{2.5cm}
\begin{minipage}{0.485\textwidth}
    \enteteInterro{Le 19 septembre 2014}{1}{A}

\Exo{smath-0815}

\end{minipage}
\begin{minipage}{0.485\textwidth}
    \enteteInterro{Le 19 septembre 2014}{1}{B}

\Exo{smath-0816}
\end{minipage}


\vspace{2.5cm}
\begin{minipage}{0.485\textwidth}
    \enteteInterro{Le 19 septembre 2014}{1}{A}

\Exo{smath-0815}

\end{minipage}
\begin{minipage}{0.485\textwidth}
    \enteteInterro{Le 19 septembre 2014}{1}{B}

\Exo{smath-0816}
\end{minipage}



\end{document}

% ACTIVITÉ ASTRONOMIE

% This is part of Un soupçon de mathématique sans être agressif pour autant
% Copyright (c) 2014
%   Laurent Claessens
% See the file fdl-1.3.txt for copying conditions.

%--------------------------------------------------------------------------------------------------------------------------- 
\subsection*{Activité : mesure astronomique}
%---------------------------------------------------------------------------------------------------------------------------

Des astronomes veulent mesurer la distance approximative entre la Terre et une comète. La technique consiste à mesurer à \( 6\) mois d'écart l'angle formé entre les droites Terre-comète et Terre-Soleil. Pour la facilité (nous ne sommes pas des astronomes professionnels) nous allons supposer que la comète ne se soit pas beaucoup déplacée en \( 6\) mois%\footnote{En réalité la technique décrite ici est utilisée pour mesurer des distance avec des étoiles proches, mais les angles ne sont alors pas possible à dessiner avec un rapporteur, étant de l'ordre de \unit{89.999772}{\degree}. Les nombres donnés ici sont choisis pour que l'exercice soit possible, plutôt que pour le réalisme.}. 
(En réalité la technique décrite ici est utilisée pour mesurer des distance avec des étoiles proches, mais les angles ne sont alors pas possible à dessiner avec un rapporteur, étant de l'ordre de \unit{89.999772}{\degree}. Les nombres donnés ici sont choisis pour que l'exercice soit possible, plutôt que pour le réalisme.)

Lors de la première mesure, l'angle obtenu est \unit{80}{\degree} tandis que la seconde mesure a donné \unit{60}{\degree}. Quelle est la distance entre la comète et la Terre ?

\vspace{2.5cm}
% This is part of Un soupçon de mathématique sans être agressif pour autant
% Copyright (c) 2014
%   Laurent Claessens
% See the file fdl-1.3.txt for copying conditions.

%--------------------------------------------------------------------------------------------------------------------------- 
\subsection*{Activité : mesure astronomique}
%---------------------------------------------------------------------------------------------------------------------------

Des astronomes veulent mesurer la distance approximative entre la Terre et une comète. La technique consiste à mesurer à \( 6\) mois d'écart l'angle formé entre les droites Terre-comète et Terre-Soleil. Pour la facilité (nous ne sommes pas des astronomes professionnels) nous allons supposer que la comète ne se soit pas beaucoup déplacée en \( 6\) mois%\footnote{En réalité la technique décrite ici est utilisée pour mesurer des distance avec des étoiles proches, mais les angles ne sont alors pas possible à dessiner avec un rapporteur, étant de l'ordre de \unit{89.999772}{\degree}. Les nombres donnés ici sont choisis pour que l'exercice soit possible, plutôt que pour le réalisme.}. 
(En réalité la technique décrite ici est utilisée pour mesurer des distance avec des étoiles proches, mais les angles ne sont alors pas possible à dessiner avec un rapporteur, étant de l'ordre de \unit{89.999772}{\degree}. Les nombres donnés ici sont choisis pour que l'exercice soit possible, plutôt que pour le réalisme.)

Lors de la première mesure, l'angle obtenu est \unit{80}{\degree} tandis que la seconde mesure a donné \unit{60}{\degree}. Quelle est la distance entre la comète et la Terre ?

\vspace{2.5cm}
% This is part of Un soupçon de mathématique sans être agressif pour autant
% Copyright (c) 2014
%   Laurent Claessens
% See the file fdl-1.3.txt for copying conditions.

%--------------------------------------------------------------------------------------------------------------------------- 
\subsection*{Activité : mesure astronomique}
%---------------------------------------------------------------------------------------------------------------------------

Des astronomes veulent mesurer la distance approximative entre la Terre et une comète. La technique consiste à mesurer à \( 6\) mois d'écart l'angle formé entre les droites Terre-comète et Terre-Soleil. Pour la facilité (nous ne sommes pas des astronomes professionnels) nous allons supposer que la comète ne se soit pas beaucoup déplacée en \( 6\) mois%\footnote{En réalité la technique décrite ici est utilisée pour mesurer des distance avec des étoiles proches, mais les angles ne sont alors pas possible à dessiner avec un rapporteur, étant de l'ordre de \unit{89.999772}{\degree}. Les nombres donnés ici sont choisis pour que l'exercice soit possible, plutôt que pour le réalisme.}. 
(En réalité la technique décrite ici est utilisée pour mesurer des distance avec des étoiles proches, mais les angles ne sont alors pas possible à dessiner avec un rapporteur, étant de l'ordre de \unit{89.999772}{\degree}. Les nombres donnés ici sont choisis pour que l'exercice soit possible, plutôt que pour le réalisme.)

Lors de la première mesure, l'angle obtenu est \unit{80}{\degree} tandis que la seconde mesure a donné \unit{60}{\degree}. Quelle est la distance entre la comète et la Terre ?

\vspace{2.5cm}
% This is part of Un soupçon de mathématique sans être agressif pour autant
% Copyright (c) 2014
%   Laurent Claessens
% See the file fdl-1.3.txt for copying conditions.

%--------------------------------------------------------------------------------------------------------------------------- 
\subsection*{Activité : mesure astronomique}
%---------------------------------------------------------------------------------------------------------------------------

Des astronomes veulent mesurer la distance approximative entre la Terre et une comète. La technique consiste à mesurer à \( 6\) mois d'écart l'angle formé entre les droites Terre-comète et Terre-Soleil. Pour la facilité (nous ne sommes pas des astronomes professionnels) nous allons supposer que la comète ne se soit pas beaucoup déplacée en \( 6\) mois%\footnote{En réalité la technique décrite ici est utilisée pour mesurer des distance avec des étoiles proches, mais les angles ne sont alors pas possible à dessiner avec un rapporteur, étant de l'ordre de \unit{89.999772}{\degree}. Les nombres donnés ici sont choisis pour que l'exercice soit possible, plutôt que pour le réalisme.}. 
(En réalité la technique décrite ici est utilisée pour mesurer des distance avec des étoiles proches, mais les angles ne sont alors pas possible à dessiner avec un rapporteur, étant de l'ordre de \unit{89.999772}{\degree}. Les nombres donnés ici sont choisis pour que l'exercice soit possible, plutôt que pour le réalisme.)

Lors de la première mesure, l'angle obtenu est \unit{80}{\degree} tandis que la seconde mesure a donné \unit{60}{\degree}. Quelle est la distance entre la comète et la Terre ?


\end{document}

% ACTIVITÉ TRIANGLE

% This is part of Un soupçon de mathématique sans être agressif pour autant
% Copyright (c) 2014
%   Laurent Claessens
% See the file fdl-1.3.txt for copying conditions.



\begin{wrapfigure}{r}{2.5cm}
   \vspace{-0.5cm}        % à adapter.
   \centering
   \input{Fig_HHOooQUedri.pstricks}
\end{wrapfigure}

Tentatives de construire des triangles de longueurs imposées.

\begin{enumerate}
    \item
        
Choisir trois nombres compris entre $2$ et $15$ et tenter de tracer un triangle dont les côtés ont ces mesures (règle, rapporteur, compas, équerre).

\item

    Voyant le triangle ci-contre, Louise s'est exclamée «il est complètement faux !». Pourquoi ? Essayer de dessiner un triangle correct ayant ces mesures.

\end{enumerate}

\vspace{2.5cm}


% This is part of Un soupçon de mathématique sans être agressif pour autant
% Copyright (c) 2014
%   Laurent Claessens
% See the file fdl-1.3.txt for copying conditions.



\begin{wrapfigure}{r}{2.5cm}
   \vspace{-0.5cm}        % à adapter.
   \centering
   \input{Fig_HHOooQUedri.pstricks}
\end{wrapfigure}

Tentatives de construire des triangles de longueurs imposées.

\begin{enumerate}
    \item
        
Choisir trois nombres compris entre $2$ et $15$ et tenter de tracer un triangle dont les côtés ont ces mesures (règle, rapporteur, compas, équerre).

\item

    Voyant le triangle ci-contre, Louise s'est exclamée «il est complètement faux !». Pourquoi ? Essayer de dessiner un triangle correct ayant ces mesures.

\end{enumerate}

\vspace{2.5cm}

% This is part of Un soupçon de mathématique sans être agressif pour autant
% Copyright (c) 2014
%   Laurent Claessens
% See the file fdl-1.3.txt for copying conditions.



\begin{wrapfigure}{r}{2.5cm}
   \vspace{-0.5cm}        % à adapter.
   \centering
   \input{Fig_HHOooQUedri.pstricks}
\end{wrapfigure}

Tentatives de construire des triangles de longueurs imposées.

\begin{enumerate}
    \item
        
Choisir trois nombres compris entre $2$ et $15$ et tenter de tracer un triangle dont les côtés ont ces mesures (règle, rapporteur, compas, équerre).

\item

    Voyant le triangle ci-contre, Louise s'est exclamée «il est complètement faux !». Pourquoi ? Essayer de dessiner un triangle correct ayant ces mesures.

\end{enumerate}

\vspace{2.5cm}
% This is part of Un soupçon de mathématique sans être agressif pour autant
% Copyright (c) 2014
%   Laurent Claessens
% See the file fdl-1.3.txt for copying conditions.



\begin{wrapfigure}{r}{2.5cm}
   \vspace{-0.5cm}        % à adapter.
   \centering
   \input{Fig_HHOooQUedri.pstricks}
\end{wrapfigure}

Tentatives de construire des triangles de longueurs imposées.

\begin{enumerate}
    \item
        
Choisir trois nombres compris entre $2$ et $15$ et tenter de tracer un triangle dont les côtés ont ces mesures (règle, rapporteur, compas, équerre).

\item

    Voyant le triangle ci-contre, Louise s'est exclamée «il est complètement faux !». Pourquoi ? Essayer de dessiner un triangle correct ayant ces mesures.

\end{enumerate}

\end{document}

% ACTIVITÉ BERTRAND L'ARTISAN

% This is part of Un soupçon de mathématique sans être agressif pour autant
% Copyright (c) 2014
%   Laurent Claessens
% See the file fdl-1.3.txt for copying conditions.

%--------------------------------------------------------------------------------------------------------------------------- 
\subsection*{Activité : Bertrand vend des pots}
%---------------------------------------------------------------------------------------------------------------------------

Bertrand l'artisan vend des pots sur le marché. Chaque pot lui coûte \( 2\)€ de matériel et est revendu \( 7\)€.
\begin{enumerate}
    \item
        Pour savoir quel sera son gain en vendant \( 13\) pots, Bertrand fait l'opération suivante :
        \begin{equation}
            \input{GKJooPMzPdG.calcul}
        \end{equation}
        Calculer cette valeur.
    \item
        Son ami Josef lui fait remarquer qu'il peut plus facilement calculer son bénéfice en calculant d'abord le bénéfice d'un pot et en multipliant ensuite par le nombre de pot.

        Proposer, en suivant cette idée, une expression donnant le bénéfice de Bertrand lorsqu'il vend \( 13\) pots.

        Vérifier qu'elle fonctionne en recalculant le bénéfice réalisé par la vente de \( 13\) pots.
    \item
        Calculer mentalement le bénéfice réalisé lors de la vente de \( 20\) pots.
\end{enumerate}

\vspace{2cm}
% This is part of Un soupçon de mathématique sans être agressif pour autant
% Copyright (c) 2014
%   Laurent Claessens
% See the file fdl-1.3.txt for copying conditions.

%--------------------------------------------------------------------------------------------------------------------------- 
\subsection*{Activité : Bertrand vend des pots}
%---------------------------------------------------------------------------------------------------------------------------

Bertrand l'artisan vend des pots sur le marché. Chaque pot lui coûte \( 2\)€ de matériel et est revendu \( 7\)€.
\begin{enumerate}
    \item
        Pour savoir quel sera son gain en vendant \( 13\) pots, Bertrand fait l'opération suivante :
        \begin{equation}
            \input{GKJooPMzPdG.calcul}
        \end{equation}
        Calculer cette valeur.
    \item
        Son ami Josef lui fait remarquer qu'il peut plus facilement calculer son bénéfice en calculant d'abord le bénéfice d'un pot et en multipliant ensuite par le nombre de pot.

        Proposer, en suivant cette idée, une expression donnant le bénéfice de Bertrand lorsqu'il vend \( 13\) pots.

        Vérifier qu'elle fonctionne en recalculant le bénéfice réalisé par la vente de \( 13\) pots.
    \item
        Calculer mentalement le bénéfice réalisé lors de la vente de \( 20\) pots.
\end{enumerate}

\vspace{2cm}
% This is part of Un soupçon de mathématique sans être agressif pour autant
% Copyright (c) 2014
%   Laurent Claessens
% See the file fdl-1.3.txt for copying conditions.

%--------------------------------------------------------------------------------------------------------------------------- 
\subsection*{Activité : Bertrand vend des pots}
%---------------------------------------------------------------------------------------------------------------------------

Bertrand l'artisan vend des pots sur le marché. Chaque pot lui coûte \( 2\)€ de matériel et est revendu \( 7\)€.
\begin{enumerate}
    \item
        Pour savoir quel sera son gain en vendant \( 13\) pots, Bertrand fait l'opération suivante :
        \begin{equation}
            \input{GKJooPMzPdG.calcul}
        \end{equation}
        Calculer cette valeur.
    \item
        Son ami Josef lui fait remarquer qu'il peut plus facilement calculer son bénéfice en calculant d'abord le bénéfice d'un pot et en multipliant ensuite par le nombre de pot.

        Proposer, en suivant cette idée, une expression donnant le bénéfice de Bertrand lorsqu'il vend \( 13\) pots.

        Vérifier qu'elle fonctionne en recalculant le bénéfice réalisé par la vente de \( 13\) pots.
    \item
        Calculer mentalement le bénéfice réalisé lors de la vente de \( 20\) pots.
\end{enumerate}


\end{document}

% POUR QUI A FINI PLUS VITE
\title{Questions pour réfléchir un peu plus}
\maketitle

\Exo{smath-0764}
\Exo{smath-0765}
\Exo{smath-0766}
\Exo{smath-0763}
\Exo{smath-0767}

% Avant de virer ceci, il faut la mettre dans smath.

\end{document}

% ACTIVITÉ PRODUIT DE NOMBRES RELATIFS
% This is part of Un soupçon de mathématique sans être agressif pour autant
% Copyright (c) 2014
%   Laurent Claessens
% See the file fdl-1.3.txt for copying conditions.

Nous considérons le nombre \( A=(-2)+(-2)+(-2)+(-2)+(-2)\).

\begin{enumerate}
    \item
        Combien vaut \( B\) ?
    \item
        Écrire \( B\) sous forme d'un produit.
    \item
        Écrire sous forme d'une somme et calculer :
        \begin{enumerate}
            \item \( (-6)\times 4\)
            \item \( (-21)\times 5\)
            \item\( (-1.5)\times 3\).
        \end{enumerate}
\end{enumerate}

Compléter le tableau de produits suivant:
\begin{equation*}
    \begin{array}[]{|c||c|c|c|c|c|c|c|c|c|}
        \hline
        \times&-4&-3&-2&-1&\hphantom{-}0&1\hphantom{-}&2\hphantom{-}&3\hphantom{-}&4\hphantom{-}\\
        \hline\hline
        -4&&&&&0&&&&\\
        \hline
        -3&&&&&&&&&\\
        \hline
        -2&&&&&&&&&\\
        \hline
        -1&&&&&0&&&&\\
        \hline
        0&&&&&&&&0&\\
        \hline
        1&&&&&&&&&\\
        \hline
        2&&&&&&&&&\\
        \hline
        3&&&&&&&&&12\\
        \hline
        4&&&&&&4&&&\\
        \hline
    \end{array}
\end{equation*}

\vspace{4cm}
% This is part of Un soupçon de mathématique sans être agressif pour autant
% Copyright (c) 2014
%   Laurent Claessens
% See the file fdl-1.3.txt for copying conditions.

Nous considérons le nombre \( A=(-2)+(-2)+(-2)+(-2)+(-2)\).

\begin{enumerate}
    \item
        Combien vaut \( B\) ?
    \item
        Écrire \( B\) sous forme d'un produit.
    \item
        Écrire sous forme d'une somme et calculer :
        \begin{enumerate}
            \item \( (-6)\times 4\)
            \item \( (-21)\times 5\)
            \item\( (-1.5)\times 3\).
        \end{enumerate}
\end{enumerate}

Compléter le tableau de produits suivant:
\begin{equation*}
    \begin{array}[]{|c||c|c|c|c|c|c|c|c|c|}
        \hline
        \times&-4&-3&-2&-1&\hphantom{-}0&1\hphantom{-}&2\hphantom{-}&3\hphantom{-}&4\hphantom{-}\\
        \hline\hline
        -4&&&&&0&&&&\\
        \hline
        -3&&&&&&&&&\\
        \hline
        -2&&&&&&&&&\\
        \hline
        -1&&&&&0&&&&\\
        \hline
        0&&&&&&&&0&\\
        \hline
        1&&&&&&&&&\\
        \hline
        2&&&&&&&&&\\
        \hline
        3&&&&&&&&&12\\
        \hline
        4&&&&&&4&&&\\
        \hline
    \end{array}
\end{equation*}


\end{document}



% ACTIVITÉ ORDRE DES OPÉRATIONS
% This is part of Un soupçon de mathématique sans être agressif pour autant
% Copyright (c) 2014
%   Laurent Claessens
% See the file fdl-1.3.txt for copying conditions.

%--------------------------------------------------------------------------------------------------------------------------- 
\subsection*{Sans multiplications}
%---------------------------------------------------------------------------------------------------------------------------

\begin{enumerate}
    \item
Calculer \( A=12+5-4\), \( B=12-4+5\) et \( C=5-4+12\).
\item
    Qu'observe-t-on ?
\end{enumerate}

%--------------------------------------------------------------------------------------------------------------------------- 
\subsection*{Avec multiplications}
%---------------------------------------------------------------------------------------------------------------------------

\begin{enumerate}
    \item
        Calculer mentalement \( D=4\times 5+2\), \( E=5\times 4+2\) et \( F=2\times 4+5\).
    \item
        Recalculer ces expressions en les tapant à la calculatrice exactement comme elles sont écrites.
    \item
        Qu'observe-t-on ?
\end{enumerate}

%--------------------------------------------------------------------------------------------------------------------------- 
\subsection*{Avec des parenthèses}
%---------------------------------------------------------------------------------------------------------------------------

\begin{enumerate}
    \item
        Calculer \( G=(7+3)\times 3\), \( H=4\times (30-21)\) et \( K=(3\times 4)\times (7-2)\).
    \item
        Dans quel ordre faut-il faire les calculs ?
\end{enumerate}

\vspace{2cm}
% This is part of Un soupçon de mathématique sans être agressif pour autant
% Copyright (c) 2014
%   Laurent Claessens
% See the file fdl-1.3.txt for copying conditions.

%--------------------------------------------------------------------------------------------------------------------------- 
\subsection*{Sans multiplications}
%---------------------------------------------------------------------------------------------------------------------------

\begin{enumerate}
    \item
Calculer \( A=12+5-4\), \( B=12-4+5\) et \( C=5-4+12\).
\item
    Qu'observe-t-on ?
\end{enumerate}

%--------------------------------------------------------------------------------------------------------------------------- 
\subsection*{Avec multiplications}
%---------------------------------------------------------------------------------------------------------------------------

\begin{enumerate}
    \item
        Calculer mentalement \( D=4\times 5+2\), \( E=5\times 4+2\) et \( F=2\times 4+5\).
    \item
        Recalculer ces expressions en les tapant à la calculatrice exactement comme elles sont écrites.
    \item
        Qu'observe-t-on ?
\end{enumerate}

%--------------------------------------------------------------------------------------------------------------------------- 
\subsection*{Avec des parenthèses}
%---------------------------------------------------------------------------------------------------------------------------

\begin{enumerate}
    \item
        Calculer \( G=(7+3)\times 3\), \( H=4\times (30-21)\) et \( K=(3\times 4)\times (7-2)\).
    \item
        Dans quel ordre faut-il faire les calculs ?
\end{enumerate}

\vspace{2cm}
% This is part of Un soupçon de mathématique sans être agressif pour autant
% Copyright (c) 2014
%   Laurent Claessens
% See the file fdl-1.3.txt for copying conditions.

%--------------------------------------------------------------------------------------------------------------------------- 
\subsection*{Sans multiplications}
%---------------------------------------------------------------------------------------------------------------------------

\begin{enumerate}
    \item
Calculer \( A=12+5-4\), \( B=12-4+5\) et \( C=5-4+12\).
\item
    Qu'observe-t-on ?
\end{enumerate}

%--------------------------------------------------------------------------------------------------------------------------- 
\subsection*{Avec multiplications}
%---------------------------------------------------------------------------------------------------------------------------

\begin{enumerate}
    \item
        Calculer mentalement \( D=4\times 5+2\), \( E=5\times 4+2\) et \( F=2\times 4+5\).
    \item
        Recalculer ces expressions en les tapant à la calculatrice exactement comme elles sont écrites.
    \item
        Qu'observe-t-on ?
\end{enumerate}

%--------------------------------------------------------------------------------------------------------------------------- 
\subsection*{Avec des parenthèses}
%---------------------------------------------------------------------------------------------------------------------------

\begin{enumerate}
    \item
        Calculer \( G=(7+3)\times 3\), \( H=4\times (30-21)\) et \( K=(3\times 4)\times (7-2)\).
    \item
        Dans quel ordre faut-il faire les calculs ?
\end{enumerate}


\end{document}

\Exo{smath-0714}
\Exo{smath-0715}
\Exo{smath-0716}
\Exo{smath-0717}
\Exo{smath-0718}
\Exo{smath-0719}
\Exo{smath-0720}
\Exo{smath-0721}

\end{document}

%DS NUMÉRO 8 MARDI 3 JUIN 2014 -- RATTRAPAGE
\begin{feuilleDS}{DS numéro 8\\ \small mardi 3 juin 2014}
\Exo{smath-0718}
\Exo{smath-0719}
\Exo{smath-0720}

\begin{center}
    TOURNER LA PAGE.
\end{center}

\clearpage

\Exo{smath-0721}

\end{feuilleDS}

\end{document}


%DS NUMÉRO 8 MARDI 3 JUIN 2014
\begin{feuilleDS}{DS numéro 8\\ \small mardi 3 juin 2014}
\Exo{smath-0714}
\Exo{smath-0715}
\Exo{smath-0716}

\begin{center}
    TOURNER LA PAGE.
\end{center}
\clearpage

\Exo{smath-0717}
\end{feuilleDS}

\end{document}


%ACTIVITÉ TRIGONOMÉTRIE
    La SNCF veut enrouler un fil de cuivre de \SI{100}{\meter} de long autour d'une grande bobine de \SI{1}{\meter} de diamètre. Combien de tours seront nécessaires ?

    À la moitié du deuxième tour, nous mettons une marque sur le fil. À quelle distance du début du fil se trouve la marque ?


\vspace{2cm}
    La SNCF veut enrouler un fil de cuivre de \SI{100}{\meter} de long autour d'une grande bobine de \SI{1}{\meter} de diamètre. Combien de tours seront nécessaires ?

    À la moitié du deuxième tour, nous mettons une marque sur le fil. À quelle distance du début du fil se trouve la marque ?


\vspace{2cm}
    La SNCF veut enrouler un fil de cuivre de \SI{100}{\meter} de long autour d'une grande bobine de \SI{1}{\meter} de diamètre. Combien de tours seront nécessaires ?

    À la moitié du deuxième tour, nous mettons une marque sur le fil. À quelle distance du début du fil se trouve la marque ?


\vspace{2cm}
    La SNCF veut enrouler un fil de cuivre de \SI{100}{\meter} de long autour d'une grande bobine de \SI{1}{\meter} de diamètre. Combien de tours seront nécessaires ?

    À la moitié du deuxième tour, nous mettons une marque sur le fil. À quelle distance du début du fil se trouve la marque ?


\vspace{2cm}
    La SNCF veut enrouler un fil de cuivre de \SI{100}{\meter} de long autour d'une grande bobine de \SI{1}{\meter} de diamètre. Combien de tours seront nécessaires ?

    À la moitié du deuxième tour, nous mettons une marque sur le fil. À quelle distance du début du fil se trouve la marque ?


\vspace{2cm}
    La SNCF veut enrouler un fil de cuivre de \SI{100}{\meter} de long autour d'une grande bobine de \SI{1}{\meter} de diamètre. Combien de tours seront nécessaires ?

    À la moitié du deuxième tour, nous mettons une marque sur le fil. À quelle distance du début du fil se trouve la marque ?


\vspace{2cm}
    La SNCF veut enrouler un fil de cuivre de \SI{100}{\meter} de long autour d'une grande bobine de \SI{1}{\meter} de diamètre. Combien de tours seront nécessaires ?

    À la moitié du deuxième tour, nous mettons une marque sur le fil. À quelle distance du début du fil se trouve la marque ?


\vspace{2cm}
\end{document}

%ACTIVITÉ HOMOGRAPHIQUE
%This is part of Un soupçon de mathématique sans être agressif pour autant
% Copyright (c) 2014
%   Laurent Claessens
%See the file fdl-1.3.txt for copying conditions.


    \begin{enumerate}
        \item
    Richard veut parcourir \unit{60}{\kilo\meter} à vélo en deux heures et demie. À quelle vitesse doit-il pédaler ? À quelle vitesse doit-il pédaler pour effectuer ce trajet en deux heures et dix minutes ?

\item
    Écrire la fonction qui à \( x\) fait correspondre la vitesse à laquelle il faut se déplacer pour effectuer \unit{60}{\kilo\meter} en une heure plus \( x\) minutes.
\item
    Retrouver les réponses de la première question en utilisant la fonction trouvée à la seconde question.            
\item
    Toujours en utilisant cette fonction, à quelle vitesse faut-il avancer pour faire le voyage en \( 50\) minutes ? \( 10\) minutes ? \( 1\) minute ?
    \end{enumerate}




%Un muret vertical de deux mètres de haut est placé trois mètres devant la façade d'une maison. Nous voulons placer un projecteur au sol de telle sorte que l'ombre du muret couvre la fenêtre de la chambre des enfants. Le sommet de cette fenêtre se situe à \unit{2.5}{\meter}.
    
%Où placer le projecteur ?

%Voir la figure \ref{LabelFigBIlgjwy}. % From file BIlgjwy
%\newcommand{\CaptionFigBIlgjwy}{La figure de l'exercice \ref{exosmqth-0379}.}
%    \begin{center}
%\input{Fig_BIlgjwy.pstricks}
%    \end{center}


\vspace{2cm}
%This is part of Un soupçon de mathématique sans être agressif pour autant
% Copyright (c) 2014
%   Laurent Claessens
%See the file fdl-1.3.txt for copying conditions.


    \begin{enumerate}
        \item
    Richard veut parcourir \unit{60}{\kilo\meter} à vélo en deux heures et demie. À quelle vitesse doit-il pédaler ? À quelle vitesse doit-il pédaler pour effectuer ce trajet en deux heures et dix minutes ?

\item
    Écrire la fonction qui à \( x\) fait correspondre la vitesse à laquelle il faut se déplacer pour effectuer \unit{60}{\kilo\meter} en une heure plus \( x\) minutes.
\item
    Retrouver les réponses de la première question en utilisant la fonction trouvée à la seconde question.            
\item
    Toujours en utilisant cette fonction, à quelle vitesse faut-il avancer pour faire le voyage en \( 50\) minutes ? \( 10\) minutes ? \( 1\) minute ?
    \end{enumerate}




%Un muret vertical de deux mètres de haut est placé trois mètres devant la façade d'une maison. Nous voulons placer un projecteur au sol de telle sorte que l'ombre du muret couvre la fenêtre de la chambre des enfants. Le sommet de cette fenêtre se situe à \unit{2.5}{\meter}.
    
%Où placer le projecteur ?

%Voir la figure \ref{LabelFigBIlgjwy}. % From file BIlgjwy
%\newcommand{\CaptionFigBIlgjwy}{La figure de l'exercice \ref{exosmqth-0379}.}
%    \begin{center}
%\input{Fig_BIlgjwy.pstricks}
%    \end{center}


\vspace{2cm}
%This is part of Un soupçon de mathématique sans être agressif pour autant
% Copyright (c) 2014
%   Laurent Claessens
%See the file fdl-1.3.txt for copying conditions.


    \begin{enumerate}
        \item
    Richard veut parcourir \unit{60}{\kilo\meter} à vélo en deux heures et demie. À quelle vitesse doit-il pédaler ? À quelle vitesse doit-il pédaler pour effectuer ce trajet en deux heures et dix minutes ?

\item
    Écrire la fonction qui à \( x\) fait correspondre la vitesse à laquelle il faut se déplacer pour effectuer \unit{60}{\kilo\meter} en une heure plus \( x\) minutes.
\item
    Retrouver les réponses de la première question en utilisant la fonction trouvée à la seconde question.            
\item
    Toujours en utilisant cette fonction, à quelle vitesse faut-il avancer pour faire le voyage en \( 50\) minutes ? \( 10\) minutes ? \( 1\) minute ?
    \end{enumerate}




%Un muret vertical de deux mètres de haut est placé trois mètres devant la façade d'une maison. Nous voulons placer un projecteur au sol de telle sorte que l'ombre du muret couvre la fenêtre de la chambre des enfants. Le sommet de cette fenêtre se situe à \unit{2.5}{\meter}.
    
%Où placer le projecteur ?

%Voir la figure \ref{LabelFigBIlgjwy}. % From file BIlgjwy
%\newcommand{\CaptionFigBIlgjwy}{La figure de l'exercice \ref{exosmqth-0379}.}
%    \begin{center}
%\input{Fig_BIlgjwy.pstricks}
%    \end{center}


\vspace{2cm}
%This is part of Un soupçon de mathématique sans être agressif pour autant
% Copyright (c) 2014
%   Laurent Claessens
%See the file fdl-1.3.txt for copying conditions.


    \begin{enumerate}
        \item
    Richard veut parcourir \unit{60}{\kilo\meter} à vélo en deux heures et demie. À quelle vitesse doit-il pédaler ? À quelle vitesse doit-il pédaler pour effectuer ce trajet en deux heures et dix minutes ?

\item
    Écrire la fonction qui à \( x\) fait correspondre la vitesse à laquelle il faut se déplacer pour effectuer \unit{60}{\kilo\meter} en une heure plus \( x\) minutes.
\item
    Retrouver les réponses de la première question en utilisant la fonction trouvée à la seconde question.            
\item
    Toujours en utilisant cette fonction, à quelle vitesse faut-il avancer pour faire le voyage en \( 50\) minutes ? \( 10\) minutes ? \( 1\) minute ?
    \end{enumerate}




%Un muret vertical de deux mètres de haut est placé trois mètres devant la façade d'une maison. Nous voulons placer un projecteur au sol de telle sorte que l'ombre du muret couvre la fenêtre de la chambre des enfants. Le sommet de cette fenêtre se situe à \unit{2.5}{\meter}.
    
%Où placer le projecteur ?

%Voir la figure \ref{LabelFigBIlgjwy}. % From file BIlgjwy
%\newcommand{\CaptionFigBIlgjwy}{La figure de l'exercice \ref{exosmqth-0379}.}
%    \begin{center}
%\input{Fig_BIlgjwy.pstricks}
%    \end{center}


\vspace{2cm}
%This is part of Un soupçon de mathématique sans être agressif pour autant
% Copyright (c) 2014
%   Laurent Claessens
%See the file fdl-1.3.txt for copying conditions.


    \begin{enumerate}
        \item
    Richard veut parcourir \unit{60}{\kilo\meter} à vélo en deux heures et demie. À quelle vitesse doit-il pédaler ? À quelle vitesse doit-il pédaler pour effectuer ce trajet en deux heures et dix minutes ?

\item
    Écrire la fonction qui à \( x\) fait correspondre la vitesse à laquelle il faut se déplacer pour effectuer \unit{60}{\kilo\meter} en une heure plus \( x\) minutes.
\item
    Retrouver les réponses de la première question en utilisant la fonction trouvée à la seconde question.            
\item
    Toujours en utilisant cette fonction, à quelle vitesse faut-il avancer pour faire le voyage en \( 50\) minutes ? \( 10\) minutes ? \( 1\) minute ?
    \end{enumerate}




%Un muret vertical de deux mètres de haut est placé trois mètres devant la façade d'une maison. Nous voulons placer un projecteur au sol de telle sorte que l'ombre du muret couvre la fenêtre de la chambre des enfants. Le sommet de cette fenêtre se situe à \unit{2.5}{\meter}.
    
%Où placer le projecteur ?

%Voir la figure \ref{LabelFigBIlgjwy}. % From file BIlgjwy
%\newcommand{\CaptionFigBIlgjwy}{La figure de l'exercice \ref{exosmqth-0379}.}
%    \begin{center}
%\input{Fig_BIlgjwy.pstricks}
%    \end{center}


\vspace{2cm}
\end{document}

\corrPosition{0}

% FEUILLE D'AP POUR PREMIÈRE
\begin{feuilleExo}{Pour l'AP en première}
\Exo{smath-0713}
\Exo{smath-0381}
\Exo{smath-0382}
\Exo{smath-0049}
\end{feuilleExo}


\end{document}


%POUR LES CORRECTIONS DU DS7
\Exo{smath-0695}
\Exo{smath-0707}
\Exo{smath-0698}
\Exo{smath-0697}

\Exo{smath-0694}
\Exo{smath-0699}
\Exo{smath-0700}
\Exo{smath-0701}
\Exo{smath-0703}


\end{document}

% FEUILLE DE RÉVISION EN VUE DU DEVOIR COMMUN
\begin{feuilleExo}{Feuille de révisions}
%    \begin{multicols}{2}
        \Exo{smath-0547}    % De DS des autres 
        \Exo{smath-0654}  
        \Exo{smath-0710}    % Haag
        \Exo{smath-0064}
        \Exo{smath-0574}
 %   \end{multicols}
\end{feuilleExo}

\corrChapitre{Corrections des exercices}
\end{document}



%This is part of Un soupçon de mathématique sans être agressif pour autant
% Copyright (c) 2014
%   Laurent Claessens
% See the file fdl-1.3.txt for copying conditions.

Bertrand l'artisan vend des pots de terre cuite sur le marché. Chaque pot lui coûte \( 2\)€ de matériel. Au début de sa carrière il avait fixé le prix à \( 10\)€ et il vendait \( 8\) pots par semaine. Chaque semaine Bertrand baisse son prix de \( 0.5\)€ et vend alors \( 4\) pots supplémentaires.

\begin{enumerate}
    \item
        Donner le nombre de pots vendus ainsi que son bénéfice au début de sa carrière ainsi qu'après \( 1\), \( 2\) et \( 3\) semaines.
    \item
        Exprimer le nombre de pots vendus ainsi que le bénéfice après \( x\) semaines.
    \item
        Tracer un graphique.
    \item
        Après combien de baisses de prix Bertrand aura-t-il intérêt à cesser de baisser le prix ?
\end{enumerate}

\vspace{2cm}
%This is part of Un soupçon de mathématique sans être agressif pour autant
% Copyright (c) 2014
%   Laurent Claessens
% See the file fdl-1.3.txt for copying conditions.

Bertrand l'artisan vend des pots de terre cuite sur le marché. Chaque pot lui coûte \( 2\)€ de matériel. Au début de sa carrière il avait fixé le prix à \( 10\)€ et il vendait \( 8\) pots par semaine. Chaque semaine Bertrand baisse son prix de \( 0.5\)€ et vend alors \( 4\) pots supplémentaires.

\begin{enumerate}
    \item
        Donner le nombre de pots vendus ainsi que son bénéfice au début de sa carrière ainsi qu'après \( 1\), \( 2\) et \( 3\) semaines.
    \item
        Exprimer le nombre de pots vendus ainsi que le bénéfice après \( x\) semaines.
    \item
        Tracer un graphique.
    \item
        Après combien de baisses de prix Bertrand aura-t-il intérêt à cesser de baisser le prix ?
\end{enumerate}

\vspace{2cm}
%This is part of Un soupçon de mathématique sans être agressif pour autant
% Copyright (c) 2014
%   Laurent Claessens
% See the file fdl-1.3.txt for copying conditions.

Bertrand l'artisan vend des pots de terre cuite sur le marché. Chaque pot lui coûte \( 2\)€ de matériel. Au début de sa carrière il avait fixé le prix à \( 10\)€ et il vendait \( 8\) pots par semaine. Chaque semaine Bertrand baisse son prix de \( 0.5\)€ et vend alors \( 4\) pots supplémentaires.

\begin{enumerate}
    \item
        Donner le nombre de pots vendus ainsi que son bénéfice au début de sa carrière ainsi qu'après \( 1\), \( 2\) et \( 3\) semaines.
    \item
        Exprimer le nombre de pots vendus ainsi que le bénéfice après \( x\) semaines.
    \item
        Tracer un graphique.
    \item
        Après combien de baisses de prix Bertrand aura-t-il intérêt à cesser de baisser le prix ?
\end{enumerate}

\vspace{2cm}
%This is part of Un soupçon de mathématique sans être agressif pour autant
% Copyright (c) 2014
%   Laurent Claessens
% See the file fdl-1.3.txt for copying conditions.

Bertrand l'artisan vend des pots de terre cuite sur le marché. Chaque pot lui coûte \( 2\)€ de matériel. Au début de sa carrière il avait fixé le prix à \( 10\)€ et il vendait \( 8\) pots par semaine. Chaque semaine Bertrand baisse son prix de \( 0.5\)€ et vend alors \( 4\) pots supplémentaires.

\begin{enumerate}
    \item
        Donner le nombre de pots vendus ainsi que son bénéfice au début de sa carrière ainsi qu'après \( 1\), \( 2\) et \( 3\) semaines.
    \item
        Exprimer le nombre de pots vendus ainsi que le bénéfice après \( x\) semaines.
    \item
        Tracer un graphique.
    \item
        Après combien de baisses de prix Bertrand aura-t-il intérêt à cesser de baisser le prix ?
\end{enumerate}


\end{document}


%DS NUMÉRO 7 MARDI 8 AVRIL 2014, 2A
\begin{feuilleDS}{Seconde A, DS numéro 7\\ \small mardi 8 avril 2014}
\Exo{smath-0693}
\Exo{smath-0695}
\Exo{smath-0707}
\Exo{smath-0698}
\Exo{smath-0697}
\end{feuilleDS}

\begin{feuilleDS}{Seconde D, DS numéro 7\\ \small mardi 8 avril 2014}
\Exo{smath-0694}
\Exo{smath-0699}
\Exo{smath-0700}
\Exo{smath-0701}
\vspace{0.5cm}
\Exo{smath-0703}
\end{feuilleDS}


\end{document}


% FEUILLE DE COLLE avril 2014
\begin{feuilleDS}{Feuille de colle, avril 2014}
\Exo{smath-0670}
\Exo{smath-0671}
\Exo{smath-0668}
\Exo{smath-0679}
\Exo{smath-0674}
\end{feuilleDS}

\end{document}


%DS NUMÉRO 4 MARDI 14 JANVIER 2014, 2A
\begin{feuilleDS}{Seconde A, DS numéro 4\\ \small mardi 14 janvier 2014}
\Exo{smath-0602}
    \Exo{smath-0597}
    \Exo{smath-0598}
    \Exo{smath-0599}
\end{feuilleDS}

%DS NUMÉRO 4 MARDI 14 JANVIER 2014, 2D
\begin{feuilleDS}{Seconde D, DS numéro 4\\ \small mardi 14 janvier 2014}
\Exo{smath-0602}
    \Exo{smath-0600}
    \Exo{smath-0601}
    \Exo{smath-0599}
\end{feuilleDS}

\end{document}



%DS NUMÉRO 6 MARDI, RATTRAPAGE, 2A
\begin{feuilleDS}{Seconde A, rattrapage DS numéro 6\\ \small mardi 11 mars 2014}
\Exo{smath-0670}
\Exo{smath-0671}
\Exo{smath-0668}
\Exo{smath-0679}
\Exo{smath-0674}
\end{feuilleDS}

\end{document}
% SECONDE FEUILLE DE SECOURS
\begin{feuilleExo}{Feuille de secours de math numéro 2}
    \begin{multicols}{2}
        \Exo{smath-0648}
\Exo{smath-0678}
\Exo{smath-0679}
\Exo{smath-0677}
\Exo{smath-0680}
\Exo{smath-0682}
\Exo{smath-0683}
\Exo{smath-0684}
\Exo{smath-0685}
    \end{multicols}
\end{feuilleExo}


\end{document}

\newlength{\vertical}
\setlength{\vertical}{1cm}

% This is part of Un soupçon de mathématique sans être agressif pour autant
% Copyright (c) 2014
%   Laurent Claessens
% See the file fdl-1.3.txt for copying conditions.

Soit le programme suivant :

\begin{fmpage}{0.9\linewidth}

    Demander \( x\) et \( y\).

    Si \( y=-2x\) alors :

    \hspace{1cm} Écrire «oui» 

    Sinon :

    \hspace{1cm} Écrire «non» 

\end{fmpage}

Donner quelque valeurs de \( x\) et \( y\) pour lesquelles le programme écrit «oui» ? À quoi sert ce programme ?

Mêmes questions pour ce programme :

\begin{fmpage}{0.9\linewidth}

    Demander \( x\) et \( y\).

    Si \( x=4\) alors :

    \hspace{1cm} Écrire «oui» 

    Sinon :

    \hspace{1cm} Écrire «non» 

\end{fmpage}

\vspace{\vertical}

% This is part of Un soupçon de mathématique sans être agressif pour autant
% Copyright (c) 2014
%   Laurent Claessens
% See the file fdl-1.3.txt for copying conditions.

Soit le programme suivant :

\begin{fmpage}{0.9\linewidth}

    Demander \( x\) et \( y\).

    Si \( y=-2x\) alors :

    \hspace{1cm} Écrire «oui» 

    Sinon :

    \hspace{1cm} Écrire «non» 

\end{fmpage}

Donner quelque valeurs de \( x\) et \( y\) pour lesquelles le programme écrit «oui» ? À quoi sert ce programme ?

Mêmes questions pour ce programme :

\begin{fmpage}{0.9\linewidth}

    Demander \( x\) et \( y\).

    Si \( x=4\) alors :

    \hspace{1cm} Écrire «oui» 

    Sinon :

    \hspace{1cm} Écrire «non» 

\end{fmpage}

\vspace{\vertical}
% This is part of Un soupçon de mathématique sans être agressif pour autant
% Copyright (c) 2014
%   Laurent Claessens
% See the file fdl-1.3.txt for copying conditions.

Soit le programme suivant :

\begin{fmpage}{0.9\linewidth}

    Demander \( x\) et \( y\).

    Si \( y=-2x\) alors :

    \hspace{1cm} Écrire «oui» 

    Sinon :

    \hspace{1cm} Écrire «non» 

\end{fmpage}

Donner quelque valeurs de \( x\) et \( y\) pour lesquelles le programme écrit «oui» ? À quoi sert ce programme ?

Mêmes questions pour ce programme :

\begin{fmpage}{0.9\linewidth}

    Demander \( x\) et \( y\).

    Si \( x=4\) alors :

    \hspace{1cm} Écrire «oui» 

    Sinon :

    \hspace{1cm} Écrire «non» 

\end{fmpage}

\vspace{\vertical}
% This is part of Un soupçon de mathématique sans être agressif pour autant
% Copyright (c) 2014
%   Laurent Claessens
% See the file fdl-1.3.txt for copying conditions.

Soit le programme suivant :

\begin{fmpage}{0.9\linewidth}

    Demander \( x\) et \( y\).

    Si \( y=-2x\) alors :

    \hspace{1cm} Écrire «oui» 

    Sinon :

    \hspace{1cm} Écrire «non» 

\end{fmpage}

Donner quelque valeurs de \( x\) et \( y\) pour lesquelles le programme écrit «oui» ? À quoi sert ce programme ?

Mêmes questions pour ce programme :

\begin{fmpage}{0.9\linewidth}

    Demander \( x\) et \( y\).

    Si \( x=4\) alors :

    \hspace{1cm} Écrire «oui» 

    Sinon :

    \hspace{1cm} Écrire «non» 

\end{fmpage}

\vspace{\vertical}

\end{document}

%DS NUMÉRO 5 MARDI 4 FÉVRIER 2014, 2A
\begin{feuilleDS}{Seconde A, DS numéro 6\\ \small mardi 11 mars 2014}
\Exo{smath-0669}
\Exo{smath-0671}
\Exo{smath-0668}
\Exo{smath-0673}
\Exo{smath-0675}
\end{feuilleDS}

\begin{feuilleDS}{Seconde D, DS numéro 6\\ \small mardi 11 mars 2014}
\Exo{smath-0668}
\Exo{smath-0670}
\Exo{smath-0672}
\Exo{smath-0674}
\Exo{smath-0676}
\end{feuilleDS}
\end{document}

%L'INTERRO DE VECTEURS
% This is part of Un soupçon de mathématique sans être agressif pour autant
% Copyright (c) 2012-2014
%   Laurent Claessens
% See the file fdl-1.3.txt for copying conditions.

\documentclass[a4paper,10pt]{article}
% This is part of Un soupçon de mathématique sans être agressif pour autant
% Copyright (c) 2012-2013
%   Laurent Claessens
% See the file fdl-1.3.txt for copying conditions.


\usepackage{etex}
\usepackage{ifthen}
%\usepackage{pdfsync}       % This package is obsolete : compile with pdflatex -synctex=1 instead.

\usepackage{latexsym}
\usepackage{amsfonts}
\usepackage{amsmath}
\usepackage{amsthm}
\usepackage{amssymb}
\usepackage{bbm}
\usepackage{mathrsfs}           
\usepackage{mathabx}           % Pour \divides

\usepackage{framed}

\usepackage{calc}   % Les dépendances de phystricks si on n'utilise que le pdf.
%\usepackage{pstricks,pst-eucl,pstricks-add,calc,pst-math}   % Les dépendances de phystricks. Peut être qu'il faut ajouter catchfile
\usepackage{graphicx}                   % Pour l'inclusion d'image en pfd.

\newcommand{\EpsOrPdfincludegraphics}[2][]{%
        \ifpdf
            \includegraphics[#1]{#2.png}
        \else
            \includegraphics[#1]{#2.eps}
        \fi
        }

\usepackage{subfigure}

\usepackage{fancyvrb}
\usepackage{stmaryrd}       % Pour le \obslash
\usepackage{xstring}        % Utilisé pour les références vers wikipédia
\usepackage{cases}
\usepackage{lscape}         % pour l'environnement landscape, utilisé dans la correction corr0076.tex
\usepackage{multicol}
\usepackage{import}         % Pour le hack qui sert à inclure GeomAnal

% TODO : n'en utiliser qu'un
\usepackage[normalem]{ulem}		% Pour le barré, commande \sout
\usepackage{soul}		% Pour le barré, commande \st

\usepackage[all]{xy}

\let\second\undefined      % le paquet amthabx définit \second
\let\degree\undefined       % le paquet amthabx définit \degree
\usepackage[cdot,thinqspace,amssymb]{SIunits} 
 % L'option amssymb sert à éviter un conflit avec la commande \square de amssymb. Note qu'elle n'est plus accessible. Si tu en as besoin, faudra RTFM
%ftp://ftp.belnet.be/packages/ctan/macros/latex/contrib/SIunits/SIunits.pdf

\usepackage[nottoc]{tocbibind}

%%%%%%%%%%%%%%%%%%%%%%%%%%
%
%   Trucs mathématiques
%
%%%%%%%%%%%%%%%%%%%%%%%%

% ENSEMBLES DE NOMBRES
\newcommand{\eA}{\mathbbm{A}}
\newcommand{\eC}{\mathbbm{C}}
\newcommand{\eD}{\mathbbm{D}}
\newcommand{\eE}{\mathbbm{E}}
\newcommand{\eF}{\mathbbm{F}}
\newcommand{\eG}{\mathbbm{G}}
\newcommand{\eH}{\mathbbm{H}}
\newcommand{\eK}{\mathbbm{K}}
\newcommand{\eL}{\mathbbm{L}}
\newcommand{\eM}{\mathbbm{M}}
\newcommand{\eN}{\mathbbm{N}}
\newcommand{\eP}{\mathbbm{P}}
\newcommand{\eQ}{\mathbbm{Q}}
\newcommand{\eR}{\mathbbm{R}}
\newcommand{\eZ}{\mathbbm{Z}}

% ENSEMBLES de fonctions
\newcommand{\aL}{\mathcal{L}}       % Les applications linéaires
\newcommand{\aC}{\mathcal{C}}       % Les fonctions C^1, C^2 etc

% AUTRES
\newcommand{\sdS}{\mathcal{S}}      % L'ensemble des subdivisions d'un intervalle.



\newcommand{\mF}{\mathcal{F}}
\newcommand{\mC}{\mathcal{C}}
\newcommand{\mG}{\mathcal{G}}
\newcommand{\mI}{\mathcal{I}}
\newcommand{\mL}{\mathcal{L}}
\newcommand{\mS}{\mathcal{S}}   % Utilisé pour l'espace des fonctions Schwartz
\newcommand{\mZ}{\mathcal{Z}}


\newcommand{\mtu}{\mathbbm{1}}              % La matrice unité
\newcommand{\caract}{\mathbbm{1}}    % Characteristic function of a set

\DeclareMathOperator{\val}{val}     % valuation d'un polynôme


%\newcommand{\efrac}[2]{\frac{ \displaystyle #1 }{\displaystyle #2 }}
%%%%%%%%%%%%%%%%%%%%%%%%%%
%
%   Numérotations en tout genre
%
%%%%%%%%%%%%%%%%%%%%%%%%

\setcounter{tocdepth}{2}        % Profondeur de la table des matières
\setcounter{secnumdepth}{2}     % Profondeur dans le texte

%%%%%%%%%%%%%%%%%%%%%%%%%%
%
%   Les lignes magiques pour le texte en français.
%
%%%%%%%%%%%%%%%%%%%%%%%%

\usepackage[utf8]{inputenc}
\usepackage[T1]{fontenc}

\usepackage{listingsutf8}
\lstset{language=python,basicstyle=\footnotesize,tabsize=3,numbers=left,numberstyle=\tiny,frame=single,commentstyle=\ttfamily\color[rgb]{0,0,0.5},stringstyle=\color[rgb]{0,0.5,0},title=\lstname,inputencoding=utf8/latin1}

\usepackage[fr]{exocorr}
\usepackage{textcomp}
\usepackage{lmodern}
\usepackage[a4paper,margin=2cm]{geometry} 
\usepackage[english,frenchb]{babel}


\usepackage{hyperref}                           %Doit être appelé en dernier.
\hypersetup{
colorlinks=true,
linkcolor=blue,
urlcolor=magenta,     % couleur des url
filecolor=magenta   % couleur des textes qui sont des liens
}

%%%%%%%%%%%%%%%%%%%%%%%%%%
%
%   Les théorèmes et choses attenantes
%
%%%%%%%%%%%%%%%%%%%%%%%%


\newcounter{numtho}
\newcounter{numprob}

\makeatletter
\@addtoreset{numtho}{chapter}
%\@addtoreset{CountExercice}{chapter}
\@addtoreset{chapter}{part}
\makeatother

\newlength{\EnvSpace}
\setlength{\EnvSpace}{9pt}      % C'est la distance que je veux mettre avant et après les théorèmes, remarques, etc.

\newtheoremstyle{MyTheorems}%
        {\EnvSpace}{\EnvSpace}%
        {\itshape}%
        {}%
        {\bfseries}{.}%
        {\newline}%
        {}%
\newtheoremstyle{MyExamples}%
        {\EnvSpace}{\EnvSpace}%
        {}%
        {}%
        {\bfseries}{.}%
        {\newline}%
        {}%
\newtheoremstyle{MyRemarks}%
        {\EnvSpace}{\EnvSpace}%
        {}%
        {}%
        {\bfseries}{.}%
        {\newline}%
        {}%

%\theoremstyle{MyExamples}   %\newtheorem{exemple}[numtho]{Exemple}      % Pour unification, ne plus utiliser
%                            \newtheorem{example}[numtho]{Exemple}
\newcounter{CounterExample}
\renewcommand{\theCounterExample}{\thechapter.\arabic{CounterExample}}

\newenvironment{example}{\vspace{\EnvSpace}\refstepcounter{numtho}\noindent{\bf Exemple \thenumtho}\newline}{\phantom{a}\hfill $\triangle$\vspace{\EnvSpace}}
\newenvironment{Aretenir}{\refstepcounter{numtho}\begin{oframed}\noindent{\bf À retenir \thenumtho}\newline}{\end{oframed}\vspace{\EnvSpace}}
\newenvironment{Enmini}{\begin{oframed}\noindent{\bf Mini résumé}\newline}{\end{oframed}\vspace{\EnvSpace}}
\newenvironment{definition}{\refstepcounter{numtho}\begin{oframed}\noindent{\bf Définition \thenumtho}\newline}{\end{oframed}\vspace{\EnvSpace}}
\newenvironment{propriete}{\refstepcounter{numtho}\begin{oframed}\noindent{\bf Propriété \thenumtho}\newline}{\end{oframed}\vspace{\EnvSpace}}

\theoremstyle{MyRemarks}    \newtheorem{remark}[numtho]{Remarque}

                \newtheorem{amusement}[numtho]{Amusement}
                \newtheorem{erreur}[numtho]{Error}
                \newtheorem{probleme}[numprob]{\fbox{\bf Problèmes et choses à faire}}


\theoremstyle{MyTheorems}
            \newtheorem{lemma}[numtho]{Lemme}
            \newtheorem{corollary}[numtho]{Corollaire}
            \newtheorem{theorem}[numtho]{Théorème}      
            \newtheorem{proposition}[numtho]{Proposition}      

            %\newtheorem{exo}[CountExercice]{Exercice}       % C'est provisoire, pour Chafaï

\renewcommand{\thenumtho}{\thechapter.\arabic{numtho}}
% La numérotation des équations change dans les corrigés
\renewcommand{\theequation}{\thechapter.\arabic{equation}}
\renewcommand{\theCountExercice}{\arabic{CountExercice}}       % Ce compteur est défini dans SystemeCorr.sty
\newcommand{\defe}[2]{\textbf{#1}\index{#2}}

\renewcommand{\labelenumi}{\theenumi}
\renewcommand{\theenumi}{(\arabic{enumi})}


%%%%%%%%%%%%%%%%%%%%%%%%%%
%
%   Les macros qui font des choses
%
%%%%%%%%%%%%%%%%%%%%%%%%

\newcommand{\mA}{\mathcal{A}}
\newcommand{\mO}{\mathcal{O}}
\newcommand{\mR}{\mathcal{R}}
\newcommand{\mT}{\mathcal{T}}
\newcommand{\mU}{\mathcal{U}}

\newcommand{\scal}[2]{ \langle #1,#2\rangle }

\newcommand{\tq}{\text{ tel que }}
\newcommand{\tqs}{\text{ tels que }}
\newcommand{\quext}[1]{ \footnote{\textsf{#1}}  }
\newcommand{\info}[1]{\texttt{#1}}
\newcommand{\vect}[1]{\overrightarrow{#1}}    % Cette macro est codée en dur dans phystricksDefVecteurAXDDGP et dans d'autres

\newcommand{\VarAbs}{\text{Var}_{\text{abs}}}
\newcommand{\VarRel}{\text{Var}_{\text{rel}}}

\newcommand{\normal}{\lhd}
\newcommand{\swS}{\mathscr{S}}          % L'ensemble des fonctions Schwartz

%\newcommand{\defD}{\mathscr{D}}     % Ensemble de définition d'une fonction
\newcommand{\defD}{D}                % Le D avec des croles était impossible à comprendre pour les élèves.

\newcommand{\Borelien}{\mathcal{B}\text{or}}       % Les boréliens
\newcommand{\tribA}{\mathcal{A}}            % Une tribu A
\newcommand{\tribB}{\mathcal{B}}            
\newcommand{\tribF}{\mathcal{F}}            % Une tribu F

\newcommand{\affE}{\mathcal{E}}            % Un espace affine E
\newcommand{\affF}{\mathcal{F}}            
\newcommand{\affG}{\mathcal{G}}            

\newcommand{\statS}{\mathcal{S}}            % Un modèle statistique
\newcommand{\partP}{\mathcal{P}}            % L'ensemble des parties d'un ensemble

\newcommand{\polyP}{\mathcal{P}}            % L'ensemble des polynômes

\newcommand{\dB}{\mathscr{B}}       % la distribution de Bernoulli
\newcommand{\dE}{\mathscr{E}}       % la distribution exponentielle
\newcommand{\dG}{\mathscr{G}}       % la distribution géométrique.
\newcommand{\dM}{\mathscr{M}}       % la distribution multinomiale
\newcommand{\dN}{\mathscr{N}}       % la distribution normale.
\newcommand{\dP}{\mathscr{P}}       % la distribution de Poisson.
\newcommand{\dT}{\mathscr{T}}       % la distribution de Student
\newcommand{\dU}{\mathscr{U}}       % la distribution uniforme

\newcommand{\hL}{\mathscr{L}}       
\newcommand{\cL}{\hL}           % Pour la partie Chafai

\newcommand{\modE}{\mathcal{E}}         % Le E des modules
\newcommand{\modF}{\mathcal{F}}         % Le F des modules
\newcommand{\hH}{\mathscr{H}}           % Le H des espaces de Hilbert

%%%%%%%%%%%%%%%%%%%%%%%%%%
%
%   Bibliographie, index et liste des notations
%
%%%%%%%%%%%%%%%%%%%%%%%%

\usepackage{makeidx}
\usepackage[nottoc]{tocbibind}      % Le paquetage qui fait en sorte que la biblio soit inclue correctement dans la table des matières.
\usepackage[refpage]{nomencl}
\renewcommand{\nomname}{Liste des notations}
%
%   Comment introduire des éléments dans l'index des notations.
%
% La liste des tags à mettre pour bien classer mes notations est :
% T     pour la topologie et théorie des ensembles
%
% La syntaxe est facile, par exemple 
%       $\SL(2,\eR)$\nomenclature[G]{$\SL(2,\eR)$}{Le groupe de matrices deux par deux de déterminant 1.}
%\renewcommand{\nomgroup}[1]{%
%    \ifthenelse{\equal{#1}{A}}{\item[\textbf{Algèbre}]}{}%
%    \ifthenelse{\equal{#1}{G}}{\item[\textbf{Géométrie}]}{}%
%    \ifthenelse{\equal{#1}{R}}{\item[\textbf{Théorie des groupes}]}{}%
%    \ifthenelse{\equal{#1}{P}}{\item[\textbf{Probabilités et statistique}]}{}%
%    \ifthenelse{\equal{#1}{Y}}{\item[\textbf{Analyse}]}{}%
%    \ifthenelse{\equal{#1}{M}}{\item[\textbf{Chaînes de Markov}]}{}%
%}

%%%%%%%%%%%%%%%%%%%%%%%%%%
%
%   DeclareMathOperator
%
%%%%%%%%%%%%%%%%%%%%%%%%

\DeclareMathOperator{\signe}{sgn}
\DeclareMathOperator{\Vol}{Vol}
\DeclareMathOperator{\Int}{Int}     % Intérieur d'un ensemble.
\DeclareMathOperator{\Ind}{Ind}     % l'indice d'un chemin en analyse complexe
\DeclareMathOperator{\Diam}{Diam}   
\DeclareMathOperator{\id}{Id}   
\DeclareMathOperator{\Graph}{Graph} 
\DeclareMathOperator{\pr}{\texttt{proj}}
\DeclareMathOperator{\dom}{dom}

\DeclareMathOperator{\Graphe}{Gr}
\DeclareMathOperator{\Spec}{Spec}   % spectre d'un opérateur
\DeclareMathOperator{\arctg}{arctg}
\DeclareMathOperator{\cotg}{cotg}
\DeclareMathOperator{\cosec}{cosec}
\DeclareMathOperator{\arcsinh}{arcsinh}

\DeclareMathOperator{\GL}{GL}   % le groupe linéaire
\DeclareMathOperator{\PGL}{PGL}   % le groupe projectif
\DeclareMathOperator{\SO}{SO}           
\DeclareMathOperator{\SL}{SL}           
\DeclareMathOperator{\PSL}{PSL}   % Le groupe modulaire SL(2,Z)/Z2
\DeclareMathOperator{\gO}{O}           
\DeclareMathOperator{\SU}{SU}           
\DeclareMathOperator{\gU}{U}           

\DeclareMathOperator{\Reel}{Re}        % La partie réelle d'un nombre complexe

\DeclareMathOperator{\Image}{Image}        % ... avec \Image qui donne l'image d'une fonction ou d'un opérateur.
\DeclareMathOperator{\rang}{rg}   
\DeclareMathOperator{\Kernel}{Ker}
\DeclareMathOperator{\Domaine}{Dom}
\DeclareMathOperator{\Span}{Span}
\DeclareMathOperator{\Hom}{Hom}
\DeclareMathOperator{\End}{End}     % L'ensemble des endomorphismes
\DeclareMathOperator{\tr}{Tr}       % la trace
\DeclareMathOperator{\Majorant}{Maj}
\DeclareMathOperator{\codim}{codim} % pour la codimension.
\DeclareMathOperator{\diam}{diam} % le diamètre d'un ensemble.

\DeclareMathOperator{\Var}{Var}     % Variance d'une variable aléatoire.
\DeclareMathOperator{\Fun}{\texttt{Fun}}     % Ensemble des applications d'un ensemble vers l'autre.
\DeclareMathOperator{\Cov}{Cov}     % la covariance.
\DeclareMathOperator{\gr}{gr}     % le groupe engendré
\DeclareMathOperator{\pgcd}{pgcd}     
\DeclareMathOperator{\ppcm}{ppcm}     
\DeclareMathOperator{\Frob}{Frob}     
\DeclareMathOperator{\Card}{Card}       % Le cardinal d'un ensemble.
\DeclareMathOperator{\Stab}{Stab}       % Le stabilisateur d'un point sous l'action d'un groupe.

\DeclareMathOperator{\Frac}{Frac}       % le corps des fractions d'un anneau
\DeclareMathOperator{\Aff}{Aff}         %  l'espace affine engendré

\newenvironment{subproof}{\begin{description}}{\end{description}}

%%%%%%%%%%%%% TRUCS DE YVIK POUR FAIRE FONCTIONNER CdI1 %%%%%%%%%%%%%%%%%%%%%%
%

%\newcommand{\proofend}{\hspace*{\fill} $\Box$\\}
%\newcommand{\diam}{\hspace*{\fill} $\Diamond$\\}
%\def\s{\smallskip}
%\def\m{\medskip}
%\def\my{\bf}
\newcommand{\eps}{\varepsilon}
\newcommand{\Ker}{\operatorname{Ker}}
\newcommand{\IM}{\operatorname {Im}}
\newcommand{\cat}{\operatorname{cat}}
\newcommand{\crit}{\operatorname{crit}}
\newcommand{\Crit}{\operatorname{Crit}}
\newcommand{\Rest}{\operatorname{Rest}}
\newcommand{\grad}{\operatorname{grad}}
\newcommand{\sgrad}{\operatorname{sgrad}}
\newcommand{\Fix}{\operatorname{Fix}}
\newcommand{\pt}{\operatorname{pt}}
\newcommand{\cl}{\operatorname{cl}}
\newcommand{\B}{\operatorname {B}}
\newcommand{\C}{\operatorname {C}}
%\newcommand{\S}{\operatorname {S}}
\newcommand{\Gr}{\operatorname {Gr\;\!}}
%\def\dim{\operatorname {dim}}
\newcommand{\inj}{\operatorname {inj}}
%\newcommand{\Vol}{\operatorname {Vol}\:\!}
%\newcommand{\Int}{\operatorname {Int}\:\!}
\newcommand{\dist}{\operatorname {dist}}
%\def\inter{\operatorname {int}}
\newcommand{\ext}{\operatorname {ext}}
%\newcommand{\diameter}{\operatorname {diam}\:\!}
\newcommand{\Emb}{\operatorname {Emb}}
\newcommand{\can}{\operatorname {can}}
\newcommand{\euler}{\mbox{\rm e}}
\newcommand{\sii}{\mbox{\rm \scriptsize i}}
\newcommand{\VB}{\mbox{V}_{\!\!B}}   
\newcommand{\VC}{\mbox{V}_{\!\!C}}   
\newcommand{\VS}{\mbox{V}_{\!\!S}}   
\newcommand{\f}{\frac}
\newcommand{\ga}{\alpha}
\newcommand{\gb}{\beta}
%\newcommand{\gg}{\gamma}
\newcommand{\gd}{\delta}
\newcommand{\gve}{\varepsilon}
\newcommand{\gf}{\varphi}
\newcommand{\gk}{\kappa}
\newcommand{\gkk}{\varkappa}
\newcommand{\gl}{\lambda}
\newcommand{\go}{\omega}
\newcommand{\gs}{\sigma}
\newcommand{\gt}{\vartheta}
\newcommand{\gy}{\upsilon}
\newcommand{\gv}{\varrho}
\newcommand{\gz}{\zeta}
\newcommand{\gD}{\Delta}
\newcommand{\gF}{\Phi}
\newcommand{\gG}{\Gamma}
\newcommand{\gL}{\Lambda}
%\newcommand{\gO}{\Omega}
\newcommand{\gS}{\Sigma}

%\long\def\forget#1\forgotten{} %
%\def\end{center}{{\mathfrak C}}
%\def\ea{{\mathfrak A}}

\newcommand{\ca}{{\mathcal A}}
\newcommand{\cb}{{\mathcal B}}
\newcommand{\cc}{{\mathcal C}}
\newcommand{\cd}{{\mathcal D}}
\newcommand{\ce}{{\mathcal E}}
\newcommand{\cf}{{\mathcal F}}
\newcommand{\cg}{{\mathcal G}}
\newcommand{\ch}{{\mathcal H}}
\newcommand{\cj}{{\mathcal J}}
\newcommand{\ck}{{\mathcal K}}
\newcommand{\cn}{{\mathcal N}}
\newcommand{\co}{{\mathcal O}}
\newcommand{\cp}{{\mathcal P}}
\newcommand{\cq}{{\mathcal Q}}
\newcommand{\cs}{{\mathcal S}}
\newcommand{\ct}{{\mathcal T}}
\newcommand{\cu}{{\mathcal U}}
\newcommand{\cv}{{\mathcal V}}
\newcommand{\cw}{{\mathcal W}}
\newcommand{\eb}{{\mathfrak B}}
\newcommand{\ed}{{\mathfrak D}}
\newcommand{\ee}{{\mathfrak E}}
\newcommand{\ef}{{\mathfrak F}}
\newcommand{\eg}{{\mathfrak G}}
\newcommand{\ej}{{\mathfrak J}}
\newcommand{\eh}{{\mathfrak H}}
\newcommand{\en}{{\mathfrak N}}
\newcommand{\eo}{{\mathfrak O}}
\newcommand{\ep}{{\mathfrak P}}
\newcommand{\eq}{{\mathfrak Q}}
\newcommand{\es}{{\mathfrak S}}
\newcommand{\et}{{\mathfrak T}}
\newcommand{\eu}{{\mathfrak U}}
\newcommand{\ev}{{\mathfrak V}}
\newcommand{\ew}{{\mathfrak W}}



%\def\NN{\mathbbm{N}}
%\def\QQ{\mathbbm{Q}}
%\def\RR{\mathbbm{R}}
%\def\SS{\mathbbm{S}}
%\def\11{\mathbbm{1}}
%\def\ZZ{\mathbbm{Z}}
%\def\TT{\mathbbm{T}}
\newcommand{\RR}{\eR}
\newcommand{\DD}{\mathbbm{D}}
\newcommand{\HH}{\mathbbm{H}}
\newcommand{\II}{\mathbbm{I}}
\newcommand{\N}{\mathbbm{N}}
\newcommand{\PP}{\mathbbm{P}}
\newcommand{\Q}{\mathbbm{Q}}
\newcommand{\RRR}{\mathbbm{R}_+}
\newcommand{\Z}{\mathbbm{Z}}
\newcommand{\RP}{{\RR\PP}} 
%\newcommand{\CP}{{\CC\PP}} 
\newcommand{\pp}{\partial}
\newcommand{\ww}{\wedge}
%\newcommand{\dc}{d^\CC}
\newcommand{\sym}{Sp(n;\RR)}
\newcommand{\ha}{\hookrightarrow}
\newcommand{\Ra}{\Rightarrow}
\newcommand{\Lra}{\Leftrightarrow} 

%\def\ni{\noindent}
%\def\b{\bigskip}
%\def\m{\medskip}
%\def\im{\mbox{Im}\,}

\newcommand{\de}{\stackrel{\mbox{\scriptsize{def}}}{=}}
%\newcommand{\id}{\mbox{id}}

%\def\sq{\square}
%\def\tr{\triangle}
%\def\trd{\bigtriangledown}
%\def\proof{\noindent {\it Proof. \;}}


%	La num\'erotation des exercices


\newcounter{exoNico}
\setcounter{exoNico}{1}
\newcommand{\exerNico}{\stepcounter{exoNico}{\bf Exercice }\arabic{exoNico}. }


%++++++++++ACCENTS++++++++++++++++++
\newcommand{\e}{\'{e}}
%\newcommand{\esp}{\'{e }}
%\newcommand{\eg}{\`{e}}
\newcommand{\ac}{\`{a} }
%\newcommand{\meme}{m\^{e}me }
\newcommand{\ou}{o\`{u} }

%+++++++++++NEWCOMMANDS+++++++++++
\newcommand{\dst}{\displaystyle}
\newcommand{\ba}{\begin{array}}
%\newcommand{\ea}{\end{array}}
%++++++++++FORMULAS+++++++++++++
\newcommand{\hs}{\hspace{0.3cm}}
%\newcommand{\eps}{\epsilon}
%\newcommand{\f}{\frac}
\newcommand{\arcth}{{\rm arctanh}}
\newcommand{\arcsh}{{\rm arcsinh}}
\newcommand{\arcch}{{\rm arccosh}}
\newcommand{\csec}{{\rm cosec}}
\newcommand{\cotan}{{\rm cotg}}
\newcommand{\cis}{(\cos+i\sin)( }
%\newcommand{\ra}{\rightarrow}
\newcommand{\lra}{\longrightarrow}
\newcommand{\ceil}{\rm plafond(}
\newcommand{\dfdu}{\frac{\partial f}{\partial u}}
\newcommand{\dfdw}{\frac{\partial f}{\partial w}}
\newcommand{\dfdx}{\frac{\partial f}{\partial x}}
\newcommand{\dfdy}{\frac{\partial f}{\partial y}}
\newcommand{\dudx}{\frac{\partial u}{\partial x}}
\newcommand{\dvdx}{\frac{\partial v}{\partial x}}
\newcommand{\dUdx}{\dfrac{\partial U}{\partial x}}
\newcommand{\dVdx}{\dfrac{\partial V}{\partial x}}
\newcommand{\dhdx}{\frac{\partial h}{\partial x}}
\newcommand{\dhdy}{\frac{\partial h}{\partial y}}
\newcommand{\dgdu}{\frac{\partial g}{\partial u}}
\newcommand{\dgdv}{\frac{\partial g}{\partial v}}
\newcommand{\dgudu}{\frac{\partial g_1}{\partial u}}
\newcommand{\dgudv}{\frac{\partial g_1}{\partial v}}
\newcommand{\dgddu}{\frac{\partial g_2}{\partial u}}
\newcommand{\dgddv}{\frac{\partial g_2}{\partial v}}
\newcommand{\dhdu}{\frac{\partial h}{\partial u}}
\newcommand{\dhdv}{\frac{\partial h}{\partial v}}
\newcommand{\dldu}{\frac{\partial l}{\partial u}}
\newcommand{\dldv}{\frac{\partial l}{\partial v}}
\newcommand{\dgudr}{\frac{\partial g_1}{\partial r}}
\newcommand{\dgudth}{\frac{\partial g_1}{\partial \theta}}
\newcommand{\dgddr}{\frac{\partial g_2}{\partial r}}
\newcommand{\dgddth}{\frac{\partial g_2}{\partial \theta}}
\newcommand{\dfdv}{\frac{\partial f}{\partial v}}
\newcommand{\dfdr}{\frac{\partial f}{\partial r}}

\newcommand{\dfdth}{\frac{\partial f}{\partial \theta}}
\newcommand{\ddfdx}{\frac{\partial^2 f}{\partial x^2}}
\newcommand{\ddfdy}{\frac{\partial^2 f}{\partial y^2}}
\newcommand{\ddfdxy}{\frac{\partial^2 f}{\partial y\partial x}}
\newcommand{\ddfdt}{\frac{\partial^2 f}{\partial^2 t}}

\newcommand{\ud}{\underline}

 % *** Blackboard math symbols ***
 %\newcommand{\N}{\mathbb{N}}
 %\newcommand{\Z}{\mathbb{Z}}
 %\newcommand{\Q}{\mathbb{R}}
 %\newcommand{\K}{\mathbb{K}}
 %\newcommand{\R}{\mathbb{R}}
 %\newcommand{\C}{\mathbb{C}}
 %\newcommand{\F}{\mathbb{F}}
 %\newcommand{\J}{\mathbb{J}}
\newcommand{\Qn}{\mathbb{Q}}

\newcommand{\Rn}{\eR} 
\newcommand{\Nn}{\eN}


\newtheorem{theo}{Th{\'e}or{\`e}me}[section]
\newtheorem{defn}{D{\'e}finition}
\newtheorem{prop}{Proposition}     % redef encore dans Chafaï
%\newtheorem{rem}{Remarque}[section]
\newtheorem{lem}{Lemme}[section]
\newcommand{\R}{\mathbb{R}}
\newcommand{\dem}{\textbf{D{\'e}monstration.}}
\newcommand{\vc}[1]{\boldsymbol{#1}}
\newcommand{\p}{\textrm{P}}
%\newcommand{\e}{\textrm{E}}
\newcommand{\mbt}{arbre binaire markovien}
\newcommand{\mbts}{arbres binaires markoviens}

\newcommand{\ea}{\end{array}}


%%%%%%%%%%%%%%%%%%%%%%%%%%%%%%%%%%%%%%
%
% les petis yeux 
%
%%%%%%%%%%%%%%%%%%%%%%%%%%%%%%%%%%%%%%%%%%%%%

\newcommand{\coolexo}{$\circledast\circledast$}
\newcommand{\boringexo}{$\circleddash\circleddash$}
\newcommand{\minsyndical}{$\odot\odot$}
\newcommand{\mortelexo}{$\obslash\oslash$}


%%%%%%%%%%%%%% FIN TRUCS DE YVIK %%%%%%%%%%%%%%%%%%%%%%

%%%%%%%%%%%%%% TRUCS DE PIERRE %%%%%%%%%%%%%%%%%%%%%%


% Le paquet array est là pour faire fonctionner l'environement arrowcases dans les trucs de Pierre.
\usepackage{array}

%\documentclass[11pt,a4paper,openany]{book}
%\usepackage[ansinew]{inputenc}
%\usepackage{pstricks, pst-node, array, ifpdf, comment, pst-plot}
%\usepackage[marginparwidth=2cm]{geometry}
%\usepackage[dvips,colorlinks]{hyperref}
%\usepackage[frenchb]{entetes}

%\usepackage{bigcenter}
%%%% debut macro %%%%
%%% ----------debut de bigcenter.sty--------------

%%% nouvel environnement bigcenter
%%% pour centrer sur toute la page (sans overfull)
%\makeatletter
%\newskip\@bigflushglue \@bigflushglue = -100pt plus 1fil

%\def\bigcenter{\trivlist \bigcentering\item\relax}
%\def\bigcentering{\let\\\@centercr\rightskip\@bigflushglue%
%\def\endbigcenter{\endtrivlist}

%\leftskip\@bigflushglue
%\parindent\z@\parfillskip\z@skip}
%\makeatother

%%% ----------fin de bigcenter.sty--------------
%%%% fin macro %%%%

%\input{mfpic}

% À régler par l'utilisateur
\newlength{\arrowsep}\setlength{\arrowsep}{3pt}
\newlength{\arrowlength}\setlength{\arrowlength}{1cm}

% Cet environnement est sympa, mais il dépend trop de ps; en tout cas il ne passe pas dans pdflatex
% 15 mars 1012
%\newenvironment{arrowcases}	{%
%			\pnode(\arrowsep,0.5ex){A}%
%			\hspace{\arrowlength}%
%			\begin{array}{>{\displaystyle\pnode(-\arrowsep,0.5ex){B}}l<{\ncline{A}{B}}@{}}
%				}
%			{
%			  \end{array}
%			}

\newenvironment{arrowcases}%
{\begin{cases}}
{\end{cases}}



\makeatletter %% \limite[condition]x x_0
\newcommand*{\limite}[3][\@empty]{\lim_{\substack{#2\rightarrow#3\\#1}}}
\makeatother

% \newenvironment{split+justif}{%
% \begin{split}%
% \let\ampori&
% \def&#1&#2\\{}
% }{%

% \end{split}}%

%\def\ncov{\tilde\nabla} % Nouvelle dérivée covariante
\newcommand*\sev{<} % 

%\newcommand{\hgot}{\mathfrak{h}} % h gothique (ss algebre de Lie)
%\def\var#1{{\mathbf #1}} % \var <-> Une variété
%\def\pardef{\stackrel{def}{=}} % = par définition.
%\newcommand{\bbar#1}{\bar{\bar{#1}}}
%\def\cov{\nabla} % Derivee covariante / connexion
%\newcommand{\gl}{\mathfrak{gl}} % algèbre linéaire
%\def\doubleprime{{\prime\prime}} % Isomorphique 
%\def\scal(#1,#2){\langle #1,#2\rangle}
%\def\agit(#1,#2){\langle #1,#2^\vee\rangle}
%\let\phiori\phi
%\let\phi\varphi
\let\ssi\iff
%\def\iddc{\mathcal I}
\newcommand*{\ideal}[1]{\{#1\}}
\newcommand*{\fleche}[1]{\stackrel{#1}\longrightarrow}

%\newcounter{exercice}
%\setcounter{exercice}{0}
\setcounter{CountExercice}{0}

% \newenvironment{exo}[1][\relax]{%
% \stepcounter{exercice}%
% \par\medskip%
% #1{\textbf{Exercice}~\arabic{exercice}.}\quad}%
% {\par}

% \newenvironment{rep}{\hspace{1em}\par\textbf{Solution
%     proposée.\quad}}{\par\noindent\hrulefill\par}

%\date{}
\newcommand{\Acplx}{A_\cdot}
\newcommand{\Bcplx}{B_\cdot}
\newcommand{\toisom}{\fleche\simeq}
\newcommand{\D}{\partial}
%\newcommand{\cat}[1]{{\bf #1}}
%\newcommand{\donc}{\Rightarrow}
%\newcommand{\im}{\text{im}}
%\newcommand{\coker}{\text{coker}}
\newcommand{\lied}{\mathcal L}
\newcommand*{\nom}[1]{\textsc{#1}}
\makeatletter
\newcommand*{\attention}[1]{\@latex@warning{#1}{!\small\bf #1!}\marginpar{Warning}}
\makeatother
\newcommand*{\inner}{\imath}
\newcommand*{\newexo}{}
\newcommand*{\principe}{}
\newcommand*{\etape}{}
\newcommand*{\preuve}{}
\newcommand*{\exr}{\item}
%\def\prim#1\expandafter\d#2 {\int #1\d#2}

\newcommand*{\crochets}[1]{\Bigl[ #1 \Bigr]}
\newcommand*{\llbrack}[1]{\left\lbrack #1 \right\lbrack}
\newcommand*{\rlbrack}[1]{\left\rbrack #1 \right\lbrack}
\newcommand*{\lrbrack}[1]{\left\lbrack #1 \right\rbrack}
\newcommand*{\rrbrack}[1]{\left\rbrack #1 \right\rbrack}
\newcommand*{\vecteur}[1]{\mathbf{#1}}


%% Maths : Les ensembles
\newcommand*{\ens}[1]{\mathbb{#1}} % Ensemble de nombres
\newcommand*{\var}[1]{\mathbf{#1}} % Variété
\newcommand*{\alg}[1]{\mathcal{#1}} % Algèbre
%\newcommand*{\RR}{\ens R}%
\newcommand*{\TT}{\ens T}% Tore !
%\expandafter\show\csname SS \endcsname
%\renewcommand*{\SS}{\var S}% 
%\newcommand*{\CC}{\ens C}%
\newcommand*{\ZZ}{\ens Z}%
\newcommand*{\QQ}{\ens Q}%
\newcommand*{\NN}{\ens N}%
\newcommand{\schwartz}{\mathcal S} % Espace de Schwartz
\newcommand*{\topologie}{\mathscr{T}}
\newcommand*{\Topologie}{\textcursive{T}}
\newcommand{\LL}{\text{\textup{L}}} %% Espace de Lebesgue droit
\newcommand{\Ll}{\mathcal{L}} %% Lebesgue ronde
\newcommand{\fronde}{\mathcal{F}} %% Transformée de Fourier.
\newcommand{\sigmaalgebre}[1]{\mathcal{#1}} %% Une sigma algèbre...
\DeclareMathOperator{\SymMatrix}{Sym}
\DeclareMathOperator{\ASymMatrix}{ASym}
\newcommand{\Sym}{\SymMatrix}
\newcommand{\ASym}{\ASymMatrix}
\newcommand{\transpose}[1]{{\vphantom{#1}}^{\mathit t}{\/#1}}
\newcommand*{\Sp}{\textup{Sp}}
\newcommand*{\Gl}{\textup{GL}}
%\renewcommand*{\sp}{\textup{sp}}
\newcommand*{\dprime}{{\prime\prime}}
%\show\span
%\newcommand*{\Span}[1]{\mathopen> #1 \mathclose<}

%% Maths : Symboles divers
%\newcommand{\pp}{\text{\textup{~p.p.}}} %% Presque partout
%\PackageWarning{entetes}{Redefining command \d}
%\renewcommand{\d}{\mbox{$\,$\textrm{d}}}
\newcommand{\surj}{\vers}
\newcommand{\isom}{\simeq}
\newcommand*{\Tau}{\alg T}
\newcommand{\cdv}{\mathfrak{X}} % Champs de vecteurs


%% Maths : Constructions

%\let\Exp\exp
%\renewcommand{\exp}[1]{e^{#1}} % On préfère e^{} que exp{}

%\renewcommand{\exp}[1]{e^{#1}} % On préfère e^{} que exp{}
%\renewcommand{\vec}[1]{\mathbf{#1}} % Désigner un vecteur
\newcommand{\set}[1]{\left\{#1\right\}} % Un ensemble { }
\newcommand*{\abs}[1]{\left\vert#1\right\vert} % Valeur absolue.
\newcommand*{\module}[1]{\left\vert#1\right\vert} % Valeur absolue.
\newcommand*{\norme}[1]{\left\Vert#1\right\Vert} % norme
\newcommand*{\ordre}[1]{\left\vert#1\right\vert} % L'ordre d'un élément.
%\def\scal(#1,#2){% Produit scalaire.
%  \PackageWarning{entetes}{Obsolete command \string\scal}%
%  \scalprod{#1}{#2}%
%}
\newcommand*{\scalprod}[2]{\left\langle #1,#2\right\rangle}
\let\dual\ast

\newcommand*{\pardef}{\stackrel{\text{def}}{=}} % Par définition.
\newcommand*{\iffdefn}{\stackrel{\text{def}}{\iff}} % Par définition.
%\newcommand*{\telque}{\mbox{~\entetes@name@telque~}} % tel que, dans un ensemble.
\newcommand*{\Defn}[1]{\emph{#1}} %
\newcommand*{\tensor}{\otimes}
\newcommand*{\pder}[2]{\frac{\partial #1}{\partial #2}}

%{{{ Fraction in-line plus jolie
% \DeclareRobustCommand\sfrac[1]{\@ifnextchar/{\@sfrac{#1}}%
%                                             {\@sfrac{#1}/}}
% \def\@sfrac#1/#2{\leavevmode\kern.1em\raise.5ex
%          \hbox{$\m@th{\fontsize\sf@size\z@
%                            \selectfont#1}$}\kern-.1em
%          /\kern-.15em\lower.25ex
%           \hbox{$\m@th{\fontsize\sf@size\z@
%                             \selectfont#2}$}}
%}}} 

\DeclareRobustCommand{\sfrac}[3][\mathrm]{\hspace{0.1em}%
  \raisebox{0.4ex}{$#1{\scriptstyle
#2}$}\hspace{-0.1em}/\hspace{-0.07em}%
  \mbox{$#1{\scriptstyle #3}$}}



%% Maths : Opérateurs
%\DeclareMathOperator{\tr}{Tr}
%\DeclareMathOperator{\pr}{\texttt{pr}}
\DeclareMathOperator{\supp}{supp}
\DeclareMathOperator{\adh}{adh}
\DeclareMathOperator{\interior}{int}
\DeclareMathOperator{\im}{Im}
\DeclareMathOperator{\Id}{Id}
\DeclareMathOperator{\Aut}{Aut}
\DeclareMathOperator{\Iso}{Iso}
\DeclareMathOperator{\Jac}{Jac} % jacobienne
\DeclareMathOperator{\coker}{coker}
\DeclareMathOperator{\interieur}{int}
\DeclareMathOperator{\Tor}{Tor}
\DeclareMathOperator{\divg}{div}
\DeclareMathOperator{\rot}{rot}
%\DeclareMathOperator{\cosec}{cosec}


%% Pour obtenir le \Sha cyrillique...
% \RequirePackage[OT2,T1]{fontenc}
% \DeclareSymbolFont{cyrletters}{OT2}{wncyr}{m}{n}
% \DeclareMathSymbol{\Sha}{\mathalpha}{cyrletters}{"58}


% \newcounter{@institute}
% \let\authorori\author
% \def\@institute{}\def\@auteurs{}
% %\newcommand{\institute}[2]{\refstepcounter{@institute}\label{#1}\def\@institute{\@institute\small
% %#2}\set@authors}
% \newcommand{\institute}[2]{\refstepcounter{@institute}\label{#1}%
%   \let\maketitleori\maketitle%
%   \renewcommand\maketitle{\footnote{#2}\maketitleori}%
% }%
% \renewcommand{\author}[1]{\def\@auteurs{#1}\set@authors}
% \def\the@institute{${}^{(\roman{@institute})}$}
% \newcommand{\inst}[1]{\ref{#1}}

% \newcommand{\set@authors}{\authorori{\@auteurs}}% \\ \@institute}}


% \newcounter{@institute}
% \newcommand{\institute}[1]{
%   \let\labelori\label
%   \renewcommand{\label}[1]{%
%     \refstepcounter{@institute}\labelori{##1}
%     \begin{tabular}{cc}%format ?!
%     \begin{minipage}[t]
      


%   }

% }%

\newcommand*{\conclusion}{\emph{Conclusion~:~}}
\newcommand{\hint}{\par\emph{Aide~:~}\hspace{1em}}
\newcommand{\rappel}{\par\emph{Rappels~:~}\hspace{1em}}

%\newcounter{enumarray} 
%\newenvironment{enumarray}[1]{% Merci Ulrike Fischer
% \setcounter{enumarray}{0}%
% \begin{array}{% motif
%     >{% Au début de chaque ligne
%       \stepcounter{enumarray}%
%       (\alph{enumarray})%
%       \hspace{2em}
%     }% 
%     #1%
%   }% fin motif
% }{%
% \end{array}%
%}
\newenvironment{displayinline}{% displaystyle + inline.
  $\displaystyle%
}{%
  $%
}

\newcommand{\telque}{\vert\,}
\newcommand{\donc}{\Rightarrow}


\pagestyle{empty}   % Pour éviter les numéros de page.

\begin{document}

\vbox{Numéro 1.
\emph{Toutes les réponses doivent être justifiées par un calcul accompagné d'un raisonnement.}
\begin{enumerate}\item
Soient les points $A(1;-1)$, $ B(-4,10)$ et $ C(-2;2)$. 
    \begin{enumerate}
    \item
    
    Calculer les coordonnées des vecteurs \( \vect{ AB }\) et \( \vect{ AB }+\vect{ BC }\). 

\item
    Donner les coordonnées du point \( X\) tel que \( \vect{ AX }=\vect{ BC }\) (méthode au choix)
    \end{enumerate}
    
    
\item

    Soient les points $M(8;1)$, $K(0;-3)$ et $E(-8;-5)$. Donner les coordonnées du point $L$ tel que $MKEL$ soit un parallélogramme (méthode au choix).
    

\end{enumerate}
}
\vspace{2cm}
\vbox{Numéro 2.
\emph{Toutes les réponses doivent être justifiées par un calcul accompagné d'un raisonnement.}
\begin{enumerate}\item

    Soient les points $L(2;-2)$, $A(3;-3)$ et $B(-1;-1)$. Donner les coordonnées du point $M$ tel que $LABM$ soit un parallélogramme (méthode au choix).
    
\item
Soient les points $A(-7;-1)$, $ B(3,8)$ et $ C(5;1)$. 
    \begin{enumerate}
    \item
    
    Calculer les coordonnées des vecteurs \( \vect{ AB }\) et \( \vect{ AB }+\vect{ BC }\). 

\item
    Donner les coordonnées du point \( X\) tel que \( \vect{ AX }=\vect{ BC }\) (méthode au choix)
    \end{enumerate}
    
    

\end{enumerate}
}
\vspace{2cm}
\vbox{Numéro 3.
\emph{Toutes les réponses doivent être justifiées par un calcul accompagné d'un raisonnement.}
\begin{enumerate}\item
Soient les points $A(-6;-7)$, $ B(-7,-1)$ et $ C(-2;-2)$. 
    \begin{enumerate}
    \item
    
    Calculer les coordonnées des vecteurs \( \vect{ AB }\) et \( \vect{ AB }+\vect{ BC }\). 

\item
    Donner les coordonnées du point \( X\) tel que \( \vect{ AX }=\vect{ BC }\) (méthode au choix)
    \end{enumerate}
    
    
\item

    Soient les points $E(-9;0)$, $L(-1;-3)$ et $F(1;-8)$. Donner les coordonnées du point $K$ tel que $ELFK$ soit un parallélogramme (méthode au choix).
    

\end{enumerate}
}
\vspace{2cm}
\vbox{Numéro 4.
\emph{Toutes les réponses doivent être justifiées par un calcul accompagné d'un raisonnement.}
\begin{enumerate}\item

    Soient les points $A(3;6)$, $F(-9;9)$ et $M(-6;0)$. Donner les coordonnées du point $D$ tel que $AFMD$ soit un parallélogramme (méthode au choix).
    
\item
Soient les points $A(-4;-5)$, $ B(-4,-7)$ et $ C(-7;7)$. 
    \begin{enumerate}
    \item
    
    Calculer les coordonnées des vecteurs \( \vect{ AB }\) et \( \vect{ AB }+\vect{ BC }\). 

\item
    Donner les coordonnées du point \( X\) tel que \( \vect{ AX }=\vect{ BC }\) (méthode au choix)
    \end{enumerate}
    
    

\end{enumerate}
}
\vspace{2cm}
\vbox{Numéro 5.
\emph{Toutes les réponses doivent être justifiées par un calcul accompagné d'un raisonnement.}
\begin{enumerate}\item
Soient les points $A(10;0)$, $ B(8,-1)$ et $ C(-10;3)$. 
    \begin{enumerate}
    \item
    
    Calculer les coordonnées des vecteurs \( \vect{ AB }\) et \( \vect{ AB }+\vect{ BC }\). 

\item
    Donner les coordonnées du point \( X\) tel que \( \vect{ AX }=\vect{ BC }\) (méthode au choix)
    \end{enumerate}
    
    
\item

    Soient les points $B(4;3)$, $D(-10;9)$ et $M(-6;-2)$. Donner les coordonnées du point $K$ tel que $BDMK$ soit un parallélogramme (méthode au choix).
    

\end{enumerate}
}
\vspace{2cm}
\vbox{Numéro 6.
\emph{Toutes les réponses doivent être justifiées par un calcul accompagné d'un raisonnement.}
\begin{enumerate}\item
Soient les points $A(-3;6)$, $ B(4,9)$ et $ C(-5;-10)$. 
    \begin{enumerate}
    \item
    
    Calculer les coordonnées des vecteurs \( \vect{ AB }\) et \( \vect{ AB }+\vect{ BC }\). 

\item
    Donner les coordonnées du point \( X\) tel que \( \vect{ AX }=\vect{ BC }\) (méthode au choix)
    \end{enumerate}
    
    
\item

    Soient les points $A(-6;-5)$, $E(-3;-4)$ et $D(-4;6)$. Donner les coordonnées du point $B$ tel que $AEDB$ soit un parallélogramme (méthode au choix).
    

\end{enumerate}
}
\vspace{2cm}
\vbox{Numéro 7.
\emph{Toutes les réponses doivent être justifiées par un calcul accompagné d'un raisonnement.}
\begin{enumerate}\item
Soient les points $A(-2;-9)$, $ B(-6,-6)$ et $ C(-5;9)$. 
    \begin{enumerate}
    \item
    
    Calculer les coordonnées des vecteurs \( \vect{ AB }\) et \( \vect{ AB }+\vect{ BC }\). 

\item
    Donner les coordonnées du point \( X\) tel que \( \vect{ AX }=\vect{ BC }\) (méthode au choix)
    \end{enumerate}
    
    
\item

    Soient les points $B(-6;-4)$, $D(7;6)$ et $E(-8;-10)$. Donner les coordonnées du point $K$ tel que $BDEK$ soit un parallélogramme (méthode au choix).
    

\end{enumerate}
}
\vspace{2cm}
\vbox{Numéro 8.
\emph{Toutes les réponses doivent être justifiées par un calcul accompagné d'un raisonnement.}
\begin{enumerate}\item

    Soient les points $E(-7;1)$, $K(-2;-7)$ et $B(10;2)$. Donner les coordonnées du point $D$ tel que $EKBD$ soit un parallélogramme (méthode au choix).
    
\item
Soient les points $A(-9;-6)$, $ B(-7,4)$ et $ C(1;-8)$. 
    \begin{enumerate}
    \item
    
    Calculer les coordonnées des vecteurs \( \vect{ AB }\) et \( \vect{ AB }+\vect{ BC }\). 

\item
    Donner les coordonnées du point \( X\) tel que \( \vect{ AX }=\vect{ BC }\) (méthode au choix)
    \end{enumerate}
    
    

\end{enumerate}
}
\vspace{2cm}
\vbox{Numéro 9.
\emph{Toutes les réponses doivent être justifiées par un calcul accompagné d'un raisonnement.}
\begin{enumerate}\item

    Soient les points $M(-6;-8)$, $F(9;8)$ et $B(-7;-7)$. Donner les coordonnées du point $E$ tel que $MFBE$ soit un parallélogramme (méthode au choix).
    
\item
Soient les points $A(9;8)$, $ B(0,-5)$ et $ C(-9;-1)$. 
    \begin{enumerate}
    \item
    
    Calculer les coordonnées des vecteurs \( \vect{ AB }\) et \( \vect{ AB }+\vect{ BC }\). 

\item
    Donner les coordonnées du point \( X\) tel que \( \vect{ AX }=\vect{ BC }\) (méthode au choix)
    \end{enumerate}
    
    

\end{enumerate}
}
\vspace{2cm}
\vbox{Numéro 10.
\emph{Toutes les réponses doivent être justifiées par un calcul accompagné d'un raisonnement.}
\begin{enumerate}\item

    Soient les points $B(9;7)$, $D(3;1)$ et $F(6;-3)$. Donner les coordonnées du point $K$ tel que $BDFK$ soit un parallélogramme (méthode au choix).
    
\item
Soient les points $A(10;-6)$, $ B(1,2)$ et $ C(-8;3)$. 
    \begin{enumerate}
    \item
    
    Calculer les coordonnées des vecteurs \( \vect{ AB }\) et \( \vect{ AB }+\vect{ BC }\). 

\item
    Donner les coordonnées du point \( X\) tel que \( \vect{ AX }=\vect{ BC }\) (méthode au choix)
    \end{enumerate}
    
    

\end{enumerate}
}
\vspace{2cm}
\vbox{Numéro 11.
\emph{Toutes les réponses doivent être justifiées par un calcul accompagné d'un raisonnement.}
\begin{enumerate}\item
Soient les points $A(-6;-2)$, $ B(8,8)$ et $ C(-4;0)$. 
    \begin{enumerate}
    \item
    
    Calculer les coordonnées des vecteurs \( \vect{ AB }\) et \( \vect{ AB }+\vect{ BC }\). 

\item
    Donner les coordonnées du point \( X\) tel que \( \vect{ AX }=\vect{ BC }\) (méthode au choix)
    \end{enumerate}
    
    
\item

    Soient les points $A(-10;-9)$, $F(-2;7)$ et $M(-8;-2)$. Donner les coordonnées du point $B$ tel que $AFMB$ soit un parallélogramme (méthode au choix).
    

\end{enumerate}
}
\vspace{2cm}
\vbox{Numéro 12.
\emph{Toutes les réponses doivent être justifiées par un calcul accompagné d'un raisonnement.}
\begin{enumerate}\item
Soient les points $A(3;9)$, $ B(-8,-9)$ et $ C(2;1)$. 
    \begin{enumerate}
    \item
    
    Calculer les coordonnées des vecteurs \( \vect{ AB }\) et \( \vect{ AB }+\vect{ BC }\). 

\item
    Donner les coordonnées du point \( X\) tel que \( \vect{ AX }=\vect{ BC }\) (méthode au choix)
    \end{enumerate}
    
    
\item

    Soient les points $F(7;-2)$, $K(-7;4)$ et $M(3;6)$. Donner les coordonnées du point $D$ tel que $FKMD$ soit un parallélogramme (méthode au choix).
    

\end{enumerate}
}
\vspace{2cm}
\vbox{Numéro 13.
\emph{Toutes les réponses doivent être justifiées par un calcul accompagné d'un raisonnement.}
\begin{enumerate}\item

    Soient les points $L(10;7)$, $A(-5;0)$ et $E(-5;5)$. Donner les coordonnées du point $D$ tel que $LAED$ soit un parallélogramme (méthode au choix).
    
\item
Soient les points $A(-6;10)$, $ B(8,-7)$ et $ C(-4;-2)$. 
    \begin{enumerate}
    \item
    
    Calculer les coordonnées des vecteurs \( \vect{ AB }\) et \( \vect{ AB }+\vect{ BC }\). 

\item
    Donner les coordonnées du point \( X\) tel que \( \vect{ AX }=\vect{ BC }\) (méthode au choix)
    \end{enumerate}
    
    

\end{enumerate}
}
\vspace{2cm}
\vbox{Numéro 14.
\emph{Toutes les réponses doivent être justifiées par un calcul accompagné d'un raisonnement.}
\begin{enumerate}\item
Soient les points $A(-8;-4)$, $ B(10,6)$ et $ C(-9;8)$. 
    \begin{enumerate}
    \item
    
    Calculer les coordonnées des vecteurs \( \vect{ AB }\) et \( \vect{ AB }+\vect{ BC }\). 

\item
    Donner les coordonnées du point \( X\) tel que \( \vect{ AX }=\vect{ BC }\) (méthode au choix)
    \end{enumerate}
    
    
\item

    Soient les points $B(-5;6)$, $D(10;-8)$ et $A(-3;-1)$. Donner les coordonnées du point $K$ tel que $BDAK$ soit un parallélogramme (méthode au choix).
    

\end{enumerate}
}
\vspace{2cm}
\vbox{Numéro 15.
\emph{Toutes les réponses doivent être justifiées par un calcul accompagné d'un raisonnement.}
\begin{enumerate}\item
Soient les points $A(-9;1)$, $ B(3,-9)$ et $ C(-2;-7)$. 
    \begin{enumerate}
    \item
    
    Calculer les coordonnées des vecteurs \( \vect{ AB }\) et \( \vect{ AB }+\vect{ BC }\). 

\item
    Donner les coordonnées du point \( X\) tel que \( \vect{ AX }=\vect{ BC }\) (méthode au choix)
    \end{enumerate}
    
    
\item

    Soient les points $F(-1;-7)$, $K(-5;-4)$ et $A(0;2)$. Donner les coordonnées du point $L$ tel que $FKAL$ soit un parallélogramme (méthode au choix).
    

\end{enumerate}
}
\vspace{2cm}
\vbox{Numéro 16.
\emph{Toutes les réponses doivent être justifiées par un calcul accompagné d'un raisonnement.}
\begin{enumerate}\item

    Soient les points $L(6;-10)$, $M(-5;10)$ et $E(7;7)$. Donner les coordonnées du point $K$ tel que $LMEK$ soit un parallélogramme (méthode au choix).
    
\item
Soient les points $A(3;5)$, $ B(0,2)$ et $ C(-5;2)$. 
    \begin{enumerate}
    \item
    
    Calculer les coordonnées des vecteurs \( \vect{ AB }\) et \( \vect{ AB }+\vect{ BC }\). 

\item
    Donner les coordonnées du point \( X\) tel que \( \vect{ AX }=\vect{ BC }\) (méthode au choix)
    \end{enumerate}
    
    

\end{enumerate}
}
\vspace{2cm}
\vbox{Numéro 17.
\emph{Toutes les réponses doivent être justifiées par un calcul accompagné d'un raisonnement.}
\begin{enumerate}\item
Soient les points $A(5;-1)$, $ B(2,-8)$ et $ C(10;-3)$. 
    \begin{enumerate}
    \item
    
    Calculer les coordonnées des vecteurs \( \vect{ AB }\) et \( \vect{ AB }+\vect{ BC }\). 

\item
    Donner les coordonnées du point \( X\) tel que \( \vect{ AX }=\vect{ BC }\) (méthode au choix)
    \end{enumerate}
    
    
\item

    Soient les points $M(-6;-3)$, $B(-3;3)$ et $E(7;3)$. Donner les coordonnées du point $F$ tel que $MBEF$ soit un parallélogramme (méthode au choix).
    

\end{enumerate}
}
\vspace{2cm}
\vbox{Numéro 18.
\emph{Toutes les réponses doivent être justifiées par un calcul accompagné d'un raisonnement.}
\begin{enumerate}\item

    Soient les points $E(0;9)$, $L(-2;4)$ et $M(-9;9)$. Donner les coordonnées du point $B$ tel que $ELMB$ soit un parallélogramme (méthode au choix).
    
\item
Soient les points $A(-9;2)$, $ B(4,-7)$ et $ C(-8;5)$. 
    \begin{enumerate}
    \item
    
    Calculer les coordonnées des vecteurs \( \vect{ AB }\) et \( \vect{ AB }+\vect{ BC }\). 

\item
    Donner les coordonnées du point \( X\) tel que \( \vect{ AX }=\vect{ BC }\) (méthode au choix)
    \end{enumerate}
    
    

\end{enumerate}
}
\vspace{2cm}
\vbox{Numéro 19.
\emph{Toutes les réponses doivent être justifiées par un calcul accompagné d'un raisonnement.}
\begin{enumerate}\item

    Soient les points $K(-4;0)$, $A(-2;-2)$ et $F(3;6)$. Donner les coordonnées du point $M$ tel que $KAFM$ soit un parallélogramme (méthode au choix).
    
\item
Soient les points $A(6;6)$, $ B(-8,8)$ et $ C(-3;-5)$. 
    \begin{enumerate}
    \item
    
    Calculer les coordonnées des vecteurs \( \vect{ AB }\) et \( \vect{ AB }+\vect{ BC }\). 

\item
    Donner les coordonnées du point \( X\) tel que \( \vect{ AX }=\vect{ BC }\) (méthode au choix)
    \end{enumerate}
    
    

\end{enumerate}
}
\vspace{2cm}
\vbox{Numéro 20.
\emph{Toutes les réponses doivent être justifiées par un calcul accompagné d'un raisonnement.}
\begin{enumerate}\item

    Soient les points $B(-3;4)$, $M(-1;-6)$ et $L(-6;9)$. Donner les coordonnées du point $K$ tel que $BMLK$ soit un parallélogramme (méthode au choix).
    
\item
Soient les points $A(3;2)$, $ B(-9,-6)$ et $ C(-9;-2)$. 
    \begin{enumerate}
    \item
    
    Calculer les coordonnées des vecteurs \( \vect{ AB }\) et \( \vect{ AB }+\vect{ BC }\). 

\item
    Donner les coordonnées du point \( X\) tel que \( \vect{ AX }=\vect{ BC }\) (méthode au choix)
    \end{enumerate}
    
    

\end{enumerate}
}
\vspace{2cm}
\vbox{Numéro 21.
\emph{Toutes les réponses doivent être justifiées par un calcul accompagné d'un raisonnement.}
\begin{enumerate}\item
Soient les points $A(2;-8)$, $ B(4,-8)$ et $ C(-7;3)$. 
    \begin{enumerate}
    \item
    
    Calculer les coordonnées des vecteurs \( \vect{ AB }\) et \( \vect{ AB }+\vect{ BC }\). 

\item
    Donner les coordonnées du point \( X\) tel que \( \vect{ AX }=\vect{ BC }\) (méthode au choix)
    \end{enumerate}
    
    
\item

    Soient les points $F(9;-3)$, $D(-7;8)$ et $K(-5;-7)$. Donner les coordonnées du point $M$ tel que $FDKM$ soit un parallélogramme (méthode au choix).
    

\end{enumerate}
}
\vspace{2cm}
\vbox{Numéro 22.
\emph{Toutes les réponses doivent être justifiées par un calcul accompagné d'un raisonnement.}
\begin{enumerate}\item
Soient les points $A(5;5)$, $ B(-4,-8)$ et $ C(-1;-2)$. 
    \begin{enumerate}
    \item
    
    Calculer les coordonnées des vecteurs \( \vect{ AB }\) et \( \vect{ AB }+\vect{ BC }\). 

\item
    Donner les coordonnées du point \( X\) tel que \( \vect{ AX }=\vect{ BC }\) (méthode au choix)
    \end{enumerate}
    
    
\item

    Soient les points $M(-1;-7)$, $L(-6;-10)$ et $B(6;6)$. Donner les coordonnées du point $D$ tel que $MLBD$ soit un parallélogramme (méthode au choix).
    

\end{enumerate}
}
\vspace{2cm}
\vbox{Numéro 23.
\emph{Toutes les réponses doivent être justifiées par un calcul accompagné d'un raisonnement.}
\begin{enumerate}\item
Soient les points $A(-5;-4)$, $ B(-1,7)$ et $ C(4;5)$. 
    \begin{enumerate}
    \item
    
    Calculer les coordonnées des vecteurs \( \vect{ AB }\) et \( \vect{ AB }+\vect{ BC }\). 

\item
    Donner les coordonnées du point \( X\) tel que \( \vect{ AX }=\vect{ BC }\) (méthode au choix)
    \end{enumerate}
    
    
\item

    Soient les points $M(7;-5)$, $D(-2;1)$ et $B(-5;-2)$. Donner les coordonnées du point $A$ tel que $MDBA$ soit un parallélogramme (méthode au choix).
    

\end{enumerate}
}
\vspace{2cm}
\vbox{Numéro 24.
\emph{Toutes les réponses doivent être justifiées par un calcul accompagné d'un raisonnement.}
\begin{enumerate}\item

    Soient les points $A(8;-6)$, $L(9;-8)$ et $M(10;9)$. Donner les coordonnées du point $E$ tel que $ALME$ soit un parallélogramme (méthode au choix).
    
\item
Soient les points $A(3;-1)$, $ B(-6,-4)$ et $ C(7;-8)$. 
    \begin{enumerate}
    \item
    
    Calculer les coordonnées des vecteurs \( \vect{ AB }\) et \( \vect{ AB }+\vect{ BC }\). 

\item
    Donner les coordonnées du point \( X\) tel que \( \vect{ AX }=\vect{ BC }\) (méthode au choix)
    \end{enumerate}
    
    

\end{enumerate}
}
\vspace{2cm}
\vbox{Numéro 25.
\emph{Toutes les réponses doivent être justifiées par un calcul accompagné d'un raisonnement.}
\begin{enumerate}\item
Soient les points $A(-4;5)$, $ B(-5,-7)$ et $ C(9;4)$. 
    \begin{enumerate}
    \item
    
    Calculer les coordonnées des vecteurs \( \vect{ AB }\) et \( \vect{ AB }+\vect{ BC }\). 

\item
    Donner les coordonnées du point \( X\) tel que \( \vect{ AX }=\vect{ BC }\) (méthode au choix)
    \end{enumerate}
    
    
\item

    Soient les points $K(-2;7)$, $E(1;-9)$ et $B(7;0)$. Donner les coordonnées du point $F$ tel que $KEBF$ soit un parallélogramme (méthode au choix).
    

\end{enumerate}
}
\vspace{2cm}
\vbox{Numéro 26.
\emph{Toutes les réponses doivent être justifiées par un calcul accompagné d'un raisonnement.}
\begin{enumerate}\item
Soient les points $A(-2;0)$, $ B(2,-8)$ et $ C(-8;-3)$. 
    \begin{enumerate}
    \item
    
    Calculer les coordonnées des vecteurs \( \vect{ AB }\) et \( \vect{ AB }+\vect{ BC }\). 

\item
    Donner les coordonnées du point \( X\) tel que \( \vect{ AX }=\vect{ BC }\) (méthode au choix)
    \end{enumerate}
    
    
\item

    Soient les points $B(7;-8)$, $A(6;-3)$ et $L(3;8)$. Donner les coordonnées du point $M$ tel que $BALM$ soit un parallélogramme (méthode au choix).
    

\end{enumerate}
}
\vspace{2cm}
\vbox{Numéro 27.
\emph{Toutes les réponses doivent être justifiées par un calcul accompagné d'un raisonnement.}
\begin{enumerate}\item

    Soient les points $M(6;9)$, $B(3;1)$ et $F(-10;-1)$. Donner les coordonnées du point $K$ tel que $MBFK$ soit un parallélogramme (méthode au choix).
    
\item
Soient les points $A(-7;-4)$, $ B(8,4)$ et $ C(-3;-8)$. 
    \begin{enumerate}
    \item
    
    Calculer les coordonnées des vecteurs \( \vect{ AB }\) et \( \vect{ AB }+\vect{ BC }\). 

\item
    Donner les coordonnées du point \( X\) tel que \( \vect{ AX }=\vect{ BC }\) (méthode au choix)
    \end{enumerate}
    
    

\end{enumerate}
}
\vspace{2cm}
\vbox{Numéro 28.
\emph{Toutes les réponses doivent être justifiées par un calcul accompagné d'un raisonnement.}
\begin{enumerate}\item

    Soient les points $E(-6;-10)$, $D(-10;9)$ et $L(7;5)$. Donner les coordonnées du point $K$ tel que $EDLK$ soit un parallélogramme (méthode au choix).
    
\item
Soient les points $A(8;-10)$, $ B(-6,0)$ et $ C(1;-1)$. 
    \begin{enumerate}
    \item
    
    Calculer les coordonnées des vecteurs \( \vect{ AB }\) et \( \vect{ AB }+\vect{ BC }\). 

\item
    Donner les coordonnées du point \( X\) tel que \( \vect{ AX }=\vect{ BC }\) (méthode au choix)
    \end{enumerate}
    
    

\end{enumerate}
}
\vspace{2cm}
\vbox{Numéro 29.
\emph{Toutes les réponses doivent être justifiées par un calcul accompagné d'un raisonnement.}
\begin{enumerate}\item
Soient les points $A(-3;9)$, $ B(1,9)$ et $ C(1;10)$. 
    \begin{enumerate}
    \item
    
    Calculer les coordonnées des vecteurs \( \vect{ AB }\) et \( \vect{ AB }+\vect{ BC }\). 

\item
    Donner les coordonnées du point \( X\) tel que \( \vect{ AX }=\vect{ BC }\) (méthode au choix)
    \end{enumerate}
    
    
\item

    Soient les points $A(2;-5)$, $F(-6;7)$ et $B(-10;7)$. Donner les coordonnées du point $E$ tel que $AFBE$ soit un parallélogramme (méthode au choix).
    

\end{enumerate}
}
\vspace{2cm}
\vbox{Numéro 30.
\emph{Toutes les réponses doivent être justifiées par un calcul accompagné d'un raisonnement.}
\begin{enumerate}\item
Soient les points $A(-10;5)$, $ B(-5,7)$ et $ C(1;-4)$. 
    \begin{enumerate}
    \item
    
    Calculer les coordonnées des vecteurs \( \vect{ AB }\) et \( \vect{ AB }+\vect{ BC }\). 

\item
    Donner les coordonnées du point \( X\) tel que \( \vect{ AX }=\vect{ BC }\) (méthode au choix)
    \end{enumerate}
    
    
\item

    Soient les points $A(-9;-1)$, $E(-7;6)$ et $F(4;5)$. Donner les coordonnées du point $B$ tel que $AEFB$ soit un parallélogramme (méthode au choix).
    

\end{enumerate}
}
\vspace{2cm}
\vbox{Numéro 31.
\emph{Toutes les réponses doivent être justifiées par un calcul accompagné d'un raisonnement.}
\begin{enumerate}\item

    Soient les points $B(-3;-9)$, $L(-9;-5)$ et $A(2;10)$. Donner les coordonnées du point $E$ tel que $BLAE$ soit un parallélogramme (méthode au choix).
    
\item
Soient les points $A(8;-10)$, $ B(10,-2)$ et $ C(10;-4)$. 
    \begin{enumerate}
    \item
    
    Calculer les coordonnées des vecteurs \( \vect{ AB }\) et \( \vect{ AB }+\vect{ BC }\). 

\item
    Donner les coordonnées du point \( X\) tel que \( \vect{ AX }=\vect{ BC }\) (méthode au choix)
    \end{enumerate}
    
    

\end{enumerate}
}
\vspace{2cm}
\vbox{Numéro 32.
\emph{Toutes les réponses doivent être justifiées par un calcul accompagné d'un raisonnement.}
\begin{enumerate}\item
Soient les points $A(3;0)$, $ B(2,6)$ et $ C(-6;-4)$. 
    \begin{enumerate}
    \item
    
    Calculer les coordonnées des vecteurs \( \vect{ AB }\) et \( \vect{ AB }+\vect{ BC }\). 

\item
    Donner les coordonnées du point \( X\) tel que \( \vect{ AX }=\vect{ BC }\) (méthode au choix)
    \end{enumerate}
    
    
\item

    Soient les points $D(-7;5)$, $F(4;3)$ et $M(-4;8)$. Donner les coordonnées du point $E$ tel que $DFME$ soit un parallélogramme (méthode au choix).
    

\end{enumerate}
}
\vspace{2cm}
\vbox{Numéro 33.
\emph{Toutes les réponses doivent être justifiées par un calcul accompagné d'un raisonnement.}
\begin{enumerate}\item
Soient les points $A(4;7)$, $ B(-1,4)$ et $ C(4;3)$. 
    \begin{enumerate}
    \item
    
    Calculer les coordonnées des vecteurs \( \vect{ AB }\) et \( \vect{ AB }+\vect{ BC }\). 

\item
    Donner les coordonnées du point \( X\) tel que \( \vect{ AX }=\vect{ BC }\) (méthode au choix)
    \end{enumerate}
    
    
\item

    Soient les points $D(10;-6)$, $M(-8;-3)$ et $B(9;8)$. Donner les coordonnées du point $L$ tel que $DMBL$ soit un parallélogramme (méthode au choix).
    

\end{enumerate}
}
\vspace{2cm}
\vbox{Numéro 34.
\emph{Toutes les réponses doivent être justifiées par un calcul accompagné d'un raisonnement.}
\begin{enumerate}\item

    Soient les points $M(-8;-6)$, $K(4;3)$ et $F(-10;6)$. Donner les coordonnées du point $E$ tel que $MKFE$ soit un parallélogramme (méthode au choix).
    
\item
Soient les points $A(9;-1)$, $ B(-5,-3)$ et $ C(8;-4)$. 
    \begin{enumerate}
    \item
    
    Calculer les coordonnées des vecteurs \( \vect{ AB }\) et \( \vect{ AB }+\vect{ BC }\). 

\item
    Donner les coordonnées du point \( X\) tel que \( \vect{ AX }=\vect{ BC }\) (méthode au choix)
    \end{enumerate}
    
    

\end{enumerate}
}
\vspace{2cm}
\vbox{Numéro 35.
\emph{Toutes les réponses doivent être justifiées par un calcul accompagné d'un raisonnement.}
\begin{enumerate}\item
Soient les points $A(3;1)$, $ B(-5,8)$ et $ C(3;-7)$. 
    \begin{enumerate}
    \item
    
    Calculer les coordonnées des vecteurs \( \vect{ AB }\) et \( \vect{ AB }+\vect{ BC }\). 

\item
    Donner les coordonnées du point \( X\) tel que \( \vect{ AX }=\vect{ BC }\) (méthode au choix)
    \end{enumerate}
    
    
\item

    Soient les points $K(-1;-2)$, $D(2;-4)$ et $F(2;7)$. Donner les coordonnées du point $A$ tel que $KDFA$ soit un parallélogramme (méthode au choix).
    

\end{enumerate}
}
\vspace{2cm}
\vbox{Numéro 36.
\emph{Toutes les réponses doivent être justifiées par un calcul accompagné d'un raisonnement.}
\begin{enumerate}\item
Soient les points $A(-7;6)$, $ B(6,-6)$ et $ C(-10;-7)$. 
    \begin{enumerate}
    \item
    
    Calculer les coordonnées des vecteurs \( \vect{ AB }\) et \( \vect{ AB }+\vect{ BC }\). 

\item
    Donner les coordonnées du point \( X\) tel que \( \vect{ AX }=\vect{ BC }\) (méthode au choix)
    \end{enumerate}
    
    
\item

    Soient les points $A(3;1)$, $D(-6;-8)$ et $L(3;-9)$. Donner les coordonnées du point $B$ tel que $ADLB$ soit un parallélogramme (méthode au choix).
    

\end{enumerate}
}
\vspace{2cm}
\vbox{Numéro 37.
\emph{Toutes les réponses doivent être justifiées par un calcul accompagné d'un raisonnement.}
\begin{enumerate}\item

    Soient les points $K(4;6)$, $F(2;-8)$ et $B(-5;2)$. Donner les coordonnées du point $D$ tel que $KFBD$ soit un parallélogramme (méthode au choix).
    
\item
Soient les points $A(5;-5)$, $ B(10,-5)$ et $ C(4;0)$. 
    \begin{enumerate}
    \item
    
    Calculer les coordonnées des vecteurs \( \vect{ AB }\) et \( \vect{ AB }+\vect{ BC }\). 

\item
    Donner les coordonnées du point \( X\) tel que \( \vect{ AX }=\vect{ BC }\) (méthode au choix)
    \end{enumerate}
    
    

\end{enumerate}
}
\vspace{2cm}
\vbox{Numéro 38.
\emph{Toutes les réponses doivent être justifiées par un calcul accompagné d'un raisonnement.}
\begin{enumerate}\item
Soient les points $A(-9;-1)$, $ B(10,-10)$ et $ C(-4;6)$. 
    \begin{enumerate}
    \item
    
    Calculer les coordonnées des vecteurs \( \vect{ AB }\) et \( \vect{ AB }+\vect{ BC }\). 

\item
    Donner les coordonnées du point \( X\) tel que \( \vect{ AX }=\vect{ BC }\) (méthode au choix)
    \end{enumerate}
    
    
\item

    Soient les points $K(-7;1)$, $D(0;6)$ et $M(-4;-1)$. Donner les coordonnées du point $B$ tel que $KDMB$ soit un parallélogramme (méthode au choix).
    

\end{enumerate}
}
\vspace{2cm}
\vbox{Numéro 39.
\emph{Toutes les réponses doivent être justifiées par un calcul accompagné d'un raisonnement.}
\begin{enumerate}\item
Soient les points $A(-10;9)$, $ B(-6,9)$ et $ C(4;0)$. 
    \begin{enumerate}
    \item
    
    Calculer les coordonnées des vecteurs \( \vect{ AB }\) et \( \vect{ AB }+\vect{ BC }\). 

\item
    Donner les coordonnées du point \( X\) tel que \( \vect{ AX }=\vect{ BC }\) (méthode au choix)
    \end{enumerate}
    
    
\item

    Soient les points $L(-1;-2)$, $M(-4;-7)$ et $E(-6;-6)$. Donner les coordonnées du point $D$ tel que $LMED$ soit un parallélogramme (méthode au choix).
    

\end{enumerate}
}
\vspace{2cm}
\vbox{Numéro 40.
\emph{Toutes les réponses doivent être justifiées par un calcul accompagné d'un raisonnement.}
\begin{enumerate}\item

    Soient les points $K(9;10)$, $F(-7;5)$ et $L(-9;0)$. Donner les coordonnées du point $B$ tel que $KFLB$ soit un parallélogramme (méthode au choix).
    
\item
Soient les points $A(-3;9)$, $ B(-3,-2)$ et $ C(0;0)$. 
    \begin{enumerate}
    \item
    
    Calculer les coordonnées des vecteurs \( \vect{ AB }\) et \( \vect{ AB }+\vect{ BC }\). 

\item
    Donner les coordonnées du point \( X\) tel que \( \vect{ AX }=\vect{ BC }\) (méthode au choix)
    \end{enumerate}
    
    

\end{enumerate}
}
\vspace{2cm}
\vbox{Numéro 41.
\emph{Toutes les réponses doivent être justifiées par un calcul accompagné d'un raisonnement.}
\begin{enumerate}\item
Soient les points $A(-4;5)$, $ B(8,1)$ et $ C(4;-2)$. 
    \begin{enumerate}
    \item
    
    Calculer les coordonnées des vecteurs \( \vect{ AB }\) et \( \vect{ AB }+\vect{ BC }\). 

\item
    Donner les coordonnées du point \( X\) tel que \( \vect{ AX }=\vect{ BC }\) (méthode au choix)
    \end{enumerate}
    
    
\item

    Soient les points $M(10;3)$, $D(2;-4)$ et $L(-4;-4)$. Donner les coordonnées du point $E$ tel que $MDLE$ soit un parallélogramme (méthode au choix).
    

\end{enumerate}
}
\vspace{2cm}
\vbox{Numéro 42.
\emph{Toutes les réponses doivent être justifiées par un calcul accompagné d'un raisonnement.}
\begin{enumerate}\item

    Soient les points $L(7;-5)$, $A(0;10)$ et $D(-8;-8)$. Donner les coordonnées du point $E$ tel que $LADE$ soit un parallélogramme (méthode au choix).
    
\item
Soient les points $A(0;8)$, $ B(-3,5)$ et $ C(4;9)$. 
    \begin{enumerate}
    \item
    
    Calculer les coordonnées des vecteurs \( \vect{ AB }\) et \( \vect{ AB }+\vect{ BC }\). 

\item
    Donner les coordonnées du point \( X\) tel que \( \vect{ AX }=\vect{ BC }\) (méthode au choix)
    \end{enumerate}
    
    

\end{enumerate}
}
\vspace{2cm}
\vbox{Numéro 43.
\emph{Toutes les réponses doivent être justifiées par un calcul accompagné d'un raisonnement.}
\begin{enumerate}\item

    Soient les points $D(1;5)$, $A(-10;-3)$ et $K(-6;-5)$. Donner les coordonnées du point $F$ tel que $DAKF$ soit un parallélogramme (méthode au choix).
    
\item
Soient les points $A(10;8)$, $ B(9,1)$ et $ C(1;-3)$. 
    \begin{enumerate}
    \item
    
    Calculer les coordonnées des vecteurs \( \vect{ AB }\) et \( \vect{ AB }+\vect{ BC }\). 

\item
    Donner les coordonnées du point \( X\) tel que \( \vect{ AX }=\vect{ BC }\) (méthode au choix)
    \end{enumerate}
    
    

\end{enumerate}
}
\vspace{2cm}
\vbox{Numéro 44.
\emph{Toutes les réponses doivent être justifiées par un calcul accompagné d'un raisonnement.}
\begin{enumerate}\item

    Soient les points $A(2;7)$, $M(-9;-7)$ et $K(-5;-5)$. Donner les coordonnées du point $B$ tel que $AMKB$ soit un parallélogramme (méthode au choix).
    
\item
Soient les points $A(-9;-4)$, $ B(-4,2)$ et $ C(6;-1)$. 
    \begin{enumerate}
    \item
    
    Calculer les coordonnées des vecteurs \( \vect{ AB }\) et \( \vect{ AB }+\vect{ BC }\). 

\item
    Donner les coordonnées du point \( X\) tel que \( \vect{ AX }=\vect{ BC }\) (méthode au choix)
    \end{enumerate}
    
    

\end{enumerate}
}
\vspace{2cm}
\vbox{Numéro 45.
\emph{Toutes les réponses doivent être justifiées par un calcul accompagné d'un raisonnement.}
\begin{enumerate}\item
Soient les points $A(-5;0)$, $ B(2,4)$ et $ C(-1;5)$. 
    \begin{enumerate}
    \item
    
    Calculer les coordonnées des vecteurs \( \vect{ AB }\) et \( \vect{ AB }+\vect{ BC }\). 

\item
    Donner les coordonnées du point \( X\) tel que \( \vect{ AX }=\vect{ BC }\) (méthode au choix)
    \end{enumerate}
    
    
\item

    Soient les points $E(-5;-4)$, $D(-2;-5)$ et $F(-9;3)$. Donner les coordonnées du point $B$ tel que $EDFB$ soit un parallélogramme (méthode au choix).
    

\end{enumerate}
}
\vspace{2cm}
\vbox{Numéro 46.
\emph{Toutes les réponses doivent être justifiées par un calcul accompagné d'un raisonnement.}
\begin{enumerate}\item
Soient les points $A(-5;-5)$, $ B(8,-3)$ et $ C(6;-9)$. 
    \begin{enumerate}
    \item
    
    Calculer les coordonnées des vecteurs \( \vect{ AB }\) et \( \vect{ AB }+\vect{ BC }\). 

\item
    Donner les coordonnées du point \( X\) tel que \( \vect{ AX }=\vect{ BC }\) (méthode au choix)
    \end{enumerate}
    
    
\item

    Soient les points $E(-4;-7)$, $A(3;0)$ et $M(-10;-8)$. Donner les coordonnées du point $B$ tel que $EAMB$ soit un parallélogramme (méthode au choix).
    

\end{enumerate}
}
\vspace{2cm}
\vbox{Numéro 47.
\emph{Toutes les réponses doivent être justifiées par un calcul accompagné d'un raisonnement.}
\begin{enumerate}\item

    Soient les points $F(-9;5)$, $M(-8;4)$ et $E(7;6)$. Donner les coordonnées du point $A$ tel que $FMEA$ soit un parallélogramme (méthode au choix).
    
\item
Soient les points $A(9;9)$, $ B(-10,10)$ et $ C(-1;1)$. 
    \begin{enumerate}
    \item
    
    Calculer les coordonnées des vecteurs \( \vect{ AB }\) et \( \vect{ AB }+\vect{ BC }\). 

\item
    Donner les coordonnées du point \( X\) tel que \( \vect{ AX }=\vect{ BC }\) (méthode au choix)
    \end{enumerate}
    
    

\end{enumerate}
}
\vspace{2cm}
\vbox{Numéro 48.
\emph{Toutes les réponses doivent être justifiées par un calcul accompagné d'un raisonnement.}
\begin{enumerate}\item
Soient les points $A(-6;-5)$, $ B(-1,-3)$ et $ C(-1;-6)$. 
    \begin{enumerate}
    \item
    
    Calculer les coordonnées des vecteurs \( \vect{ AB }\) et \( \vect{ AB }+\vect{ BC }\). 

\item
    Donner les coordonnées du point \( X\) tel que \( \vect{ AX }=\vect{ BC }\) (méthode au choix)
    \end{enumerate}
    
    
\item

    Soient les points $D(2;-2)$, $B(-7;-2)$ et $F(-8;10)$. Donner les coordonnées du point $K$ tel que $DBFK$ soit un parallélogramme (méthode au choix).
    

\end{enumerate}
}
\vspace{2cm}
\vbox{Numéro 49.
\emph{Toutes les réponses doivent être justifiées par un calcul accompagné d'un raisonnement.}
\begin{enumerate}\item

    Soient les points $D(9;-7)$, $B(-8;2)$ et $K(9;4)$. Donner les coordonnées du point $A$ tel que $DBKA$ soit un parallélogramme (méthode au choix).
    
\item
Soient les points $A(-1;6)$, $ B(8,2)$ et $ C(-3;1)$. 
    \begin{enumerate}
    \item
    
    Calculer les coordonnées des vecteurs \( \vect{ AB }\) et \( \vect{ AB }+\vect{ BC }\). 

\item
    Donner les coordonnées du point \( X\) tel que \( \vect{ AX }=\vect{ BC }\) (méthode au choix)
    \end{enumerate}
    
    

\end{enumerate}
}
\vspace{2cm}
\vbox{Numéro 50.
\emph{Toutes les réponses doivent être justifiées par un calcul accompagné d'un raisonnement.}
\begin{enumerate}\item
Soient les points $A(-5;-1)$, $ B(9,-8)$ et $ C(-7;-5)$. 
    \begin{enumerate}
    \item
    
    Calculer les coordonnées des vecteurs \( \vect{ AB }\) et \( \vect{ AB }+\vect{ BC }\). 

\item
    Donner les coordonnées du point \( X\) tel que \( \vect{ AX }=\vect{ BC }\) (méthode au choix)
    \end{enumerate}
    
    
\item

    Soient les points $M(-4;-5)$, $E(-6;-3)$ et $B(9;-1)$. Donner les coordonnées du point $A$ tel que $MEBA$ soit un parallélogramme (méthode au choix).
    

\end{enumerate}
}
\vspace{2cm}
\vbox{Numéro 51.
\emph{Toutes les réponses doivent être justifiées par un calcul accompagné d'un raisonnement.}
\begin{enumerate}\item

    Soient les points $M(3;-4)$, $L(-9;8)$ et $K(-6;-6)$. Donner les coordonnées du point $A$ tel que $MLKA$ soit un parallélogramme (méthode au choix).
    
\item
Soient les points $A(-7;3)$, $ B(5,-8)$ et $ C(2;6)$. 
    \begin{enumerate}
    \item
    
    Calculer les coordonnées des vecteurs \( \vect{ AB }\) et \( \vect{ AB }+\vect{ BC }\). 

\item
    Donner les coordonnées du point \( X\) tel que \( \vect{ AX }=\vect{ BC }\) (méthode au choix)
    \end{enumerate}
    
    

\end{enumerate}
}
\vspace{2cm}
\vbox{Numéro 52.
\emph{Toutes les réponses doivent être justifiées par un calcul accompagné d'un raisonnement.}
\begin{enumerate}\item
Soient les points $A(3;-3)$, $ B(4,9)$ et $ C(8;-1)$. 
    \begin{enumerate}
    \item
    
    Calculer les coordonnées des vecteurs \( \vect{ AB }\) et \( \vect{ AB }+\vect{ BC }\). 

\item
    Donner les coordonnées du point \( X\) tel que \( \vect{ AX }=\vect{ BC }\) (méthode au choix)
    \end{enumerate}
    
    
\item

    Soient les points $B(-1;4)$, $L(2;-9)$ et $E(-2;-6)$. Donner les coordonnées du point $F$ tel que $BLEF$ soit un parallélogramme (méthode au choix).
    

\end{enumerate}
}
\vspace{2cm}
\vbox{Numéro 53.
\emph{Toutes les réponses doivent être justifiées par un calcul accompagné d'un raisonnement.}
\begin{enumerate}\item
Soient les points $A(-3;-4)$, $ B(-9,-7)$ et $ C(1;10)$. 
    \begin{enumerate}
    \item
    
    Calculer les coordonnées des vecteurs \( \vect{ AB }\) et \( \vect{ AB }+\vect{ BC }\). 

\item
    Donner les coordonnées du point \( X\) tel que \( \vect{ AX }=\vect{ BC }\) (méthode au choix)
    \end{enumerate}
    
    
\item

    Soient les points $M(9;-8)$, $L(8;-2)$ et $B(4;10)$. Donner les coordonnées du point $F$ tel que $MLBF$ soit un parallélogramme (méthode au choix).
    

\end{enumerate}
}
\vspace{2cm}
\vbox{Numéro 54.
\emph{Toutes les réponses doivent être justifiées par un calcul accompagné d'un raisonnement.}
\begin{enumerate}\item
Soient les points $A(3;-9)$, $ B(-6,0)$ et $ C(-7;-1)$. 
    \begin{enumerate}
    \item
    
    Calculer les coordonnées des vecteurs \( \vect{ AB }\) et \( \vect{ AB }+\vect{ BC }\). 

\item
    Donner les coordonnées du point \( X\) tel que \( \vect{ AX }=\vect{ BC }\) (méthode au choix)
    \end{enumerate}
    
    
\item

    Soient les points $B(-10;-4)$, $F(8;-4)$ et $M(-9;-1)$. Donner les coordonnées du point $A$ tel que $BFMA$ soit un parallélogramme (méthode au choix).
    

\end{enumerate}
}
\vspace{2cm}
\vbox{Numéro 55.
\emph{Toutes les réponses doivent être justifiées par un calcul accompagné d'un raisonnement.}
\begin{enumerate}\item
Soient les points $A(2;4)$, $ B(2,-5)$ et $ C(-1;-6)$. 
    \begin{enumerate}
    \item
    
    Calculer les coordonnées des vecteurs \( \vect{ AB }\) et \( \vect{ AB }+\vect{ BC }\). 

\item
    Donner les coordonnées du point \( X\) tel que \( \vect{ AX }=\vect{ BC }\) (méthode au choix)
    \end{enumerate}
    
    
\item

    Soient les points $L(-7;-8)$, $M(-7;9)$ et $K(5;3)$. Donner les coordonnées du point $A$ tel que $LMKA$ soit un parallélogramme (méthode au choix).
    

\end{enumerate}
}
\vspace{2cm}
\vbox{Numéro 56.
\emph{Toutes les réponses doivent être justifiées par un calcul accompagné d'un raisonnement.}
\begin{enumerate}\item

    Soient les points $A(-2;3)$, $B(2;-2)$ et $E(-9;-8)$. Donner les coordonnées du point $M$ tel que $ABEM$ soit un parallélogramme (méthode au choix).
    
\item
Soient les points $A(-6;-1)$, $ B(9,3)$ et $ C(6;-2)$. 
    \begin{enumerate}
    \item
    
    Calculer les coordonnées des vecteurs \( \vect{ AB }\) et \( \vect{ AB }+\vect{ BC }\). 

\item
    Donner les coordonnées du point \( X\) tel que \( \vect{ AX }=\vect{ BC }\) (méthode au choix)
    \end{enumerate}
    
    

\end{enumerate}
}
\vspace{2cm}
\vbox{Numéro 57.
\emph{Toutes les réponses doivent être justifiées par un calcul accompagné d'un raisonnement.}
\begin{enumerate}\item

    Soient les points $M(5;9)$, $K(-4;-2)$ et $D(9;7)$. Donner les coordonnées du point $F$ tel que $MKDF$ soit un parallélogramme (méthode au choix).
    
\item
Soient les points $A(-3;5)$, $ B(2,4)$ et $ C(5;-6)$. 
    \begin{enumerate}
    \item
    
    Calculer les coordonnées des vecteurs \( \vect{ AB }\) et \( \vect{ AB }+\vect{ BC }\). 

\item
    Donner les coordonnées du point \( X\) tel que \( \vect{ AX }=\vect{ BC }\) (méthode au choix)
    \end{enumerate}
    
    

\end{enumerate}
}
\vspace{2cm}
\vbox{Numéro 58.
\emph{Toutes les réponses doivent être justifiées par un calcul accompagné d'un raisonnement.}
\begin{enumerate}\item

    Soient les points $D(2;-7)$, $F(1;-6)$ et $A(0;-2)$. Donner les coordonnées du point $K$ tel que $DFAK$ soit un parallélogramme (méthode au choix).
    
\item
Soient les points $A(7;1)$, $ B(8,7)$ et $ C(-9;2)$. 
    \begin{enumerate}
    \item
    
    Calculer les coordonnées des vecteurs \( \vect{ AB }\) et \( \vect{ AB }+\vect{ BC }\). 

\item
    Donner les coordonnées du point \( X\) tel que \( \vect{ AX }=\vect{ BC }\) (méthode au choix)
    \end{enumerate}
    
    

\end{enumerate}
}
\vspace{2cm}
\vbox{Numéro 59.
\emph{Toutes les réponses doivent être justifiées par un calcul accompagné d'un raisonnement.}
\begin{enumerate}\item
Soient les points $A(1;9)$, $ B(6,-6)$ et $ C(-3;-7)$. 
    \begin{enumerate}
    \item
    
    Calculer les coordonnées des vecteurs \( \vect{ AB }\) et \( \vect{ AB }+\vect{ BC }\). 

\item
    Donner les coordonnées du point \( X\) tel que \( \vect{ AX }=\vect{ BC }\) (méthode au choix)
    \end{enumerate}
    
    
\item

    Soient les points $M(-3;9)$, $F(-1;0)$ et $L(-7;-6)$. Donner les coordonnées du point $D$ tel que $MFLD$ soit un parallélogramme (méthode au choix).
    

\end{enumerate}
}
\vspace{2cm}
\vbox{Numéro 60.
\emph{Toutes les réponses doivent être justifiées par un calcul accompagné d'un raisonnement.}
\begin{enumerate}\item
Soient les points $A(4;5)$, $ B(-9,3)$ et $ C(-10;-1)$. 
    \begin{enumerate}
    \item
    
    Calculer les coordonnées des vecteurs \( \vect{ AB }\) et \( \vect{ AB }+\vect{ BC }\). 

\item
    Donner les coordonnées du point \( X\) tel que \( \vect{ AX }=\vect{ BC }\) (méthode au choix)
    \end{enumerate}
    
    
\item

    Soient les points $E(1;3)$, $L(-6;3)$ et $F(0;6)$. Donner les coordonnées du point $K$ tel que $ELFK$ soit un parallélogramme (méthode au choix).
    

\end{enumerate}
}
\vspace{2cm}
\vbox{Numéro 61.
\emph{Toutes les réponses doivent être justifiées par un calcul accompagné d'un raisonnement.}
\begin{enumerate}\item

    Soient les points $F(8;1)$, $B(6;7)$ et $K(-6;3)$. Donner les coordonnées du point $D$ tel que $FBKD$ soit un parallélogramme (méthode au choix).
    
\item
Soient les points $A(-1;6)$, $ B(-7,7)$ et $ C(-9;-2)$. 
    \begin{enumerate}
    \item
    
    Calculer les coordonnées des vecteurs \( \vect{ AB }\) et \( \vect{ AB }+\vect{ BC }\). 

\item
    Donner les coordonnées du point \( X\) tel que \( \vect{ AX }=\vect{ BC }\) (méthode au choix)
    \end{enumerate}
    
    

\end{enumerate}
}
\vspace{2cm}
\vbox{Numéro 62.
\emph{Toutes les réponses doivent être justifiées par un calcul accompagné d'un raisonnement.}
\begin{enumerate}\item
Soient les points $A(-7;5)$, $ B(3,-9)$ et $ C(8;-2)$. 
    \begin{enumerate}
    \item
    
    Calculer les coordonnées des vecteurs \( \vect{ AB }\) et \( \vect{ AB }+\vect{ BC }\). 

\item
    Donner les coordonnées du point \( X\) tel que \( \vect{ AX }=\vect{ BC }\) (méthode au choix)
    \end{enumerate}
    
    
\item

    Soient les points $K(-4;3)$, $L(8;-8)$ et $F(-8;3)$. Donner les coordonnées du point $A$ tel que $KLFA$ soit un parallélogramme (méthode au choix).
    

\end{enumerate}
}
\vspace{2cm}
\vbox{Numéro 63.
\emph{Toutes les réponses doivent être justifiées par un calcul accompagné d'un raisonnement.}
\begin{enumerate}\item

    Soient les points $B(-5;9)$, $D(2;6)$ et $A(-6;3)$. Donner les coordonnées du point $K$ tel que $BDAK$ soit un parallélogramme (méthode au choix).
    
\item
Soient les points $A(1;-3)$, $ B(5,2)$ et $ C(7;-4)$. 
    \begin{enumerate}
    \item
    
    Calculer les coordonnées des vecteurs \( \vect{ AB }\) et \( \vect{ AB }+\vect{ BC }\). 

\item
    Donner les coordonnées du point \( X\) tel que \( \vect{ AX }=\vect{ BC }\) (méthode au choix)
    \end{enumerate}
    
    

\end{enumerate}
}
\vspace{2cm}
\vbox{Numéro 64.
\emph{Toutes les réponses doivent être justifiées par un calcul accompagné d'un raisonnement.}
\begin{enumerate}\item

    Soient les points $D(-6;3)$, $K(-1;-1)$ et $L(-8;10)$. Donner les coordonnées du point $B$ tel que $DKLB$ soit un parallélogramme (méthode au choix).
    
\item
Soient les points $A(-4;-4)$, $ B(6,-6)$ et $ C(-10;-4)$. 
    \begin{enumerate}
    \item
    
    Calculer les coordonnées des vecteurs \( \vect{ AB }\) et \( \vect{ AB }+\vect{ BC }\). 

\item
    Donner les coordonnées du point \( X\) tel que \( \vect{ AX }=\vect{ BC }\) (méthode au choix)
    \end{enumerate}
    
    

\end{enumerate}
}
\vspace{2cm}
\vbox{Numéro 65.
\emph{Toutes les réponses doivent être justifiées par un calcul accompagné d'un raisonnement.}
\begin{enumerate}\item

    Soient les points $B(0;10)$, $K(-4;0)$ et $L(-6;10)$. Donner les coordonnées du point $E$ tel que $BKLE$ soit un parallélogramme (méthode au choix).
    
\item
Soient les points $A(-2;-9)$, $ B(-6,-9)$ et $ C(-5;0)$. 
    \begin{enumerate}
    \item
    
    Calculer les coordonnées des vecteurs \( \vect{ AB }\) et \( \vect{ AB }+\vect{ BC }\). 

\item
    Donner les coordonnées du point \( X\) tel que \( \vect{ AX }=\vect{ BC }\) (méthode au choix)
    \end{enumerate}
    
    

\end{enumerate}
}
\vspace{2cm}
\vbox{Numéro 66.
\emph{Toutes les réponses doivent être justifiées par un calcul accompagné d'un raisonnement.}
\begin{enumerate}\item
Soient les points $A(-8;-3)$, $ B(-3,0)$ et $ C(4;5)$. 
    \begin{enumerate}
    \item
    
    Calculer les coordonnées des vecteurs \( \vect{ AB }\) et \( \vect{ AB }+\vect{ BC }\). 

\item
    Donner les coordonnées du point \( X\) tel que \( \vect{ AX }=\vect{ BC }\) (méthode au choix)
    \end{enumerate}
    
    
\item

    Soient les points $F(1;-9)$, $L(4;9)$ et $E(9;9)$. Donner les coordonnées du point $K$ tel que $FLEK$ soit un parallélogramme (méthode au choix).
    

\end{enumerate}
}
\vspace{2cm}
\vbox{Numéro 67.
\emph{Toutes les réponses doivent être justifiées par un calcul accompagné d'un raisonnement.}
\begin{enumerate}\item
Soient les points $A(6;-6)$, $ B(9,-8)$ et $ C(10;-1)$. 
    \begin{enumerate}
    \item
    
    Calculer les coordonnées des vecteurs \( \vect{ AB }\) et \( \vect{ AB }+\vect{ BC }\). 

\item
    Donner les coordonnées du point \( X\) tel que \( \vect{ AX }=\vect{ BC }\) (méthode au choix)
    \end{enumerate}
    
    
\item

    Soient les points $F(0;-1)$, $K(9;2)$ et $M(-4;9)$. Donner les coordonnées du point $A$ tel que $FKMA$ soit un parallélogramme (méthode au choix).
    

\end{enumerate}
}
\vspace{2cm}
\vbox{Numéro 68.
\emph{Toutes les réponses doivent être justifiées par un calcul accompagné d'un raisonnement.}
\begin{enumerate}\item

    Soient les points $E(1;3)$, $K(5;6)$ et $F(-1;10)$. Donner les coordonnées du point $D$ tel que $EKFD$ soit un parallélogramme (méthode au choix).
    
\item
Soient les points $A(4;2)$, $ B(-10,-9)$ et $ C(-7;-1)$. 
    \begin{enumerate}
    \item
    
    Calculer les coordonnées des vecteurs \( \vect{ AB }\) et \( \vect{ AB }+\vect{ BC }\). 

\item
    Donner les coordonnées du point \( X\) tel que \( \vect{ AX }=\vect{ BC }\) (méthode au choix)
    \end{enumerate}
    
    

\end{enumerate}
}
\vspace{2cm}
\vbox{Numéro 69.
\emph{Toutes les réponses doivent être justifiées par un calcul accompagné d'un raisonnement.}
\begin{enumerate}\item

    Soient les points $L(-9;-9)$, $K(6;9)$ et $M(6;-6)$. Donner les coordonnées du point $A$ tel que $LKMA$ soit un parallélogramme (méthode au choix).
    
\item
Soient les points $A(-5;-7)$, $ B(-9,-5)$ et $ C(-4;-3)$. 
    \begin{enumerate}
    \item
    
    Calculer les coordonnées des vecteurs \( \vect{ AB }\) et \( \vect{ AB }+\vect{ BC }\). 

\item
    Donner les coordonnées du point \( X\) tel que \( \vect{ AX }=\vect{ BC }\) (méthode au choix)
    \end{enumerate}
    
    

\end{enumerate}
}
\vspace{2cm}
\vbox{Numéro 70.
\emph{Toutes les réponses doivent être justifiées par un calcul accompagné d'un raisonnement.}
\begin{enumerate}\item

    Soient les points $A(2;9)$, $M(10;-3)$ et $L(5;10)$. Donner les coordonnées du point $D$ tel que $AMLD$ soit un parallélogramme (méthode au choix).
    
\item
Soient les points $A(0;3)$, $ B(5,-4)$ et $ C(3;-4)$. 
    \begin{enumerate}
    \item
    
    Calculer les coordonnées des vecteurs \( \vect{ AB }\) et \( \vect{ AB }+\vect{ BC }\). 

\item
    Donner les coordonnées du point \( X\) tel que \( \vect{ AX }=\vect{ BC }\) (méthode au choix)
    \end{enumerate}
    
    

\end{enumerate}
}
\vspace{2cm}
\vbox{Numéro 71.
\emph{Toutes les réponses doivent être justifiées par un calcul accompagné d'un raisonnement.}
\begin{enumerate}\item
Soient les points $A(8;3)$, $ B(0,-10)$ et $ C(-10;0)$. 
    \begin{enumerate}
    \item
    
    Calculer les coordonnées des vecteurs \( \vect{ AB }\) et \( \vect{ AB }+\vect{ BC }\). 

\item
    Donner les coordonnées du point \( X\) tel que \( \vect{ AX }=\vect{ BC }\) (méthode au choix)
    \end{enumerate}
    
    
\item

    Soient les points $K(7;-2)$, $B(-2;1)$ et $D(6;5)$. Donner les coordonnées du point $E$ tel que $KBDE$ soit un parallélogramme (méthode au choix).
    

\end{enumerate}
}
\vspace{2cm}
\vbox{Numéro 72.
\emph{Toutes les réponses doivent être justifiées par un calcul accompagné d'un raisonnement.}
\begin{enumerate}\item

    Soient les points $K(1;-7)$, $E(6;8)$ et $D(3;-4)$. Donner les coordonnées du point $M$ tel que $KEDM$ soit un parallélogramme (méthode au choix).
    
\item
Soient les points $A(0;-5)$, $ B(-5,4)$ et $ C(6;7)$. 
    \begin{enumerate}
    \item
    
    Calculer les coordonnées des vecteurs \( \vect{ AB }\) et \( \vect{ AB }+\vect{ BC }\). 

\item
    Donner les coordonnées du point \( X\) tel que \( \vect{ AX }=\vect{ BC }\) (méthode au choix)
    \end{enumerate}
    
    

\end{enumerate}
}
\vspace{2cm}
\vbox{Numéro 73.
\emph{Toutes les réponses doivent être justifiées par un calcul accompagné d'un raisonnement.}
\begin{enumerate}\item

    Soient les points $M(1;-3)$, $K(-5;9)$ et $B(10;-2)$. Donner les coordonnées du point $A$ tel que $MKBA$ soit un parallélogramme (méthode au choix).
    
\item
Soient les points $A(1;7)$, $ B(6,7)$ et $ C(2;-7)$. 
    \begin{enumerate}
    \item
    
    Calculer les coordonnées des vecteurs \( \vect{ AB }\) et \( \vect{ AB }+\vect{ BC }\). 

\item
    Donner les coordonnées du point \( X\) tel que \( \vect{ AX }=\vect{ BC }\) (méthode au choix)
    \end{enumerate}
    
    

\end{enumerate}
}
\vspace{2cm}
\vbox{Numéro 74.
\emph{Toutes les réponses doivent être justifiées par un calcul accompagné d'un raisonnement.}
\begin{enumerate}\item
Soient les points $A(-4;-2)$, $ B(10,-3)$ et $ C(3;-9)$. 
    \begin{enumerate}
    \item
    
    Calculer les coordonnées des vecteurs \( \vect{ AB }\) et \( \vect{ AB }+\vect{ BC }\). 

\item
    Donner les coordonnées du point \( X\) tel que \( \vect{ AX }=\vect{ BC }\) (méthode au choix)
    \end{enumerate}
    
    
\item

    Soient les points $F(-10;-3)$, $L(0;3)$ et $K(-2;3)$. Donner les coordonnées du point $M$ tel que $FLKM$ soit un parallélogramme (méthode au choix).
    

\end{enumerate}
}
\vspace{2cm}

\section{correction}
\vbox{Numéro 1.
\emph{Toutes les réponses doivent être justifiées par un calcul accompagné d'un raisonnement.}
\begin{enumerate}\item
Soient les points $A(1;-1)$, $ B(-4,10)$ et $ C(-2;2)$. 
    \begin{enumerate}
    \item
    
    Calculer les coordonnées des vecteurs \( \vect{ AB }\) et \( \vect{ AB }+\vect{ BC }\). 

\item
    Donner les coordonnées du point \( X\) tel que \( \vect{ AX }=\vect{ BC }\) (méthode au choix)
    \end{enumerate}
    
    



    $\vect{ AB }=(-5;11)$

    $\vect{ AB }+\vect{ BC }=(-3;3)$

    $X=(3;-9)$
    \item

    Soient les points $M(8;1)$, $K(0;-3)$ et $E(-8;-5)$. Donner les coordonnées du point $L$ tel que $MKEL$ soit un parallélogramme (méthode au choix).
    

$L=(0;-1)$
\end{enumerate}
}
\vbox{Numéro 2.
\emph{Toutes les réponses doivent être justifiées par un calcul accompagné d'un raisonnement.}
\begin{enumerate}\item

    Soient les points $L(2;-2)$, $A(3;-3)$ et $B(-1;-1)$. Donner les coordonnées du point $M$ tel que $LABM$ soit un parallélogramme (méthode au choix).
    

$M=(-2;0)$\item
Soient les points $A(-7;-1)$, $ B(3,8)$ et $ C(5;1)$. 
    \begin{enumerate}
    \item
    
    Calculer les coordonnées des vecteurs \( \vect{ AB }\) et \( \vect{ AB }+\vect{ BC }\). 

\item
    Donner les coordonnées du point \( X\) tel que \( \vect{ AX }=\vect{ BC }\) (méthode au choix)
    \end{enumerate}
    
    



    $\vect{ AB }=(10;9)$

    $\vect{ AB }+\vect{ BC }=(12;2)$

    $X=(-5;-8)$
    
\end{enumerate}
}
\vbox{Numéro 3.
\emph{Toutes les réponses doivent être justifiées par un calcul accompagné d'un raisonnement.}
\begin{enumerate}\item
Soient les points $A(-6;-7)$, $ B(-7,-1)$ et $ C(-2;-2)$. 
    \begin{enumerate}
    \item
    
    Calculer les coordonnées des vecteurs \( \vect{ AB }\) et \( \vect{ AB }+\vect{ BC }\). 

\item
    Donner les coordonnées du point \( X\) tel que \( \vect{ AX }=\vect{ BC }\) (méthode au choix)
    \end{enumerate}
    
    



    $\vect{ AB }=(-1;6)$

    $\vect{ AB }+\vect{ BC }=(4;5)$

    $X=(-1;-8)$
    \item

    Soient les points $E(-9;0)$, $L(-1;-3)$ et $F(1;-8)$. Donner les coordonnées du point $K$ tel que $ELFK$ soit un parallélogramme (méthode au choix).
    

$K=(-7;-5)$
\end{enumerate}
}
\vbox{Numéro 4.
\emph{Toutes les réponses doivent être justifiées par un calcul accompagné d'un raisonnement.}
\begin{enumerate}\item

    Soient les points $A(3;6)$, $F(-9;9)$ et $M(-6;0)$. Donner les coordonnées du point $D$ tel que $AFMD$ soit un parallélogramme (méthode au choix).
    

$D=(6;-3)$\item
Soient les points $A(-4;-5)$, $ B(-4,-7)$ et $ C(-7;7)$. 
    \begin{enumerate}
    \item
    
    Calculer les coordonnées des vecteurs \( \vect{ AB }\) et \( \vect{ AB }+\vect{ BC }\). 

\item
    Donner les coordonnées du point \( X\) tel que \( \vect{ AX }=\vect{ BC }\) (méthode au choix)
    \end{enumerate}
    
    



    $\vect{ AB }=(0;-2)$

    $\vect{ AB }+\vect{ BC }=(-3;12)$

    $X=(-7;9)$
    
\end{enumerate}
}
\vbox{Numéro 5.
\emph{Toutes les réponses doivent être justifiées par un calcul accompagné d'un raisonnement.}
\begin{enumerate}\item
Soient les points $A(10;0)$, $ B(8,-1)$ et $ C(-10;3)$. 
    \begin{enumerate}
    \item
    
    Calculer les coordonnées des vecteurs \( \vect{ AB }\) et \( \vect{ AB }+\vect{ BC }\). 

\item
    Donner les coordonnées du point \( X\) tel que \( \vect{ AX }=\vect{ BC }\) (méthode au choix)
    \end{enumerate}
    
    



    $\vect{ AB }=(-2;-1)$

    $\vect{ AB }+\vect{ BC }=(-20;3)$

    $X=(-8;4)$
    \item

    Soient les points $B(4;3)$, $D(-10;9)$ et $M(-6;-2)$. Donner les coordonnées du point $K$ tel que $BDMK$ soit un parallélogramme (méthode au choix).
    

$K=(8;-8)$
\end{enumerate}
}
\vbox{Numéro 6.
\emph{Toutes les réponses doivent être justifiées par un calcul accompagné d'un raisonnement.}
\begin{enumerate}\item
Soient les points $A(-3;6)$, $ B(4,9)$ et $ C(-5;-10)$. 
    \begin{enumerate}
    \item
    
    Calculer les coordonnées des vecteurs \( \vect{ AB }\) et \( \vect{ AB }+\vect{ BC }\). 

\item
    Donner les coordonnées du point \( X\) tel que \( \vect{ AX }=\vect{ BC }\) (méthode au choix)
    \end{enumerate}
    
    



    $\vect{ AB }=(7;3)$

    $\vect{ AB }+\vect{ BC }=(-2;-16)$

    $X=(-12;-13)$
    \item

    Soient les points $A(-6;-5)$, $E(-3;-4)$ et $D(-4;6)$. Donner les coordonnées du point $B$ tel que $AEDB$ soit un parallélogramme (méthode au choix).
    

$B=(-7;5)$
\end{enumerate}
}
\vbox{Numéro 7.
\emph{Toutes les réponses doivent être justifiées par un calcul accompagné d'un raisonnement.}
\begin{enumerate}\item
Soient les points $A(-2;-9)$, $ B(-6,-6)$ et $ C(-5;9)$. 
    \begin{enumerate}
    \item
    
    Calculer les coordonnées des vecteurs \( \vect{ AB }\) et \( \vect{ AB }+\vect{ BC }\). 

\item
    Donner les coordonnées du point \( X\) tel que \( \vect{ AX }=\vect{ BC }\) (méthode au choix)
    \end{enumerate}
    
    



    $\vect{ AB }=(-4;3)$

    $\vect{ AB }+\vect{ BC }=(-3;18)$

    $X=(-1;6)$
    \item

    Soient les points $B(-6;-4)$, $D(7;6)$ et $E(-8;-10)$. Donner les coordonnées du point $K$ tel que $BDEK$ soit un parallélogramme (méthode au choix).
    

$K=(-21;-20)$
\end{enumerate}
}
\vbox{Numéro 8.
\emph{Toutes les réponses doivent être justifiées par un calcul accompagné d'un raisonnement.}
\begin{enumerate}\item

    Soient les points $E(-7;1)$, $K(-2;-7)$ et $B(10;2)$. Donner les coordonnées du point $D$ tel que $EKBD$ soit un parallélogramme (méthode au choix).
    

$D=(5;10)$\item
Soient les points $A(-9;-6)$, $ B(-7,4)$ et $ C(1;-8)$. 
    \begin{enumerate}
    \item
    
    Calculer les coordonnées des vecteurs \( \vect{ AB }\) et \( \vect{ AB }+\vect{ BC }\). 

\item
    Donner les coordonnées du point \( X\) tel que \( \vect{ AX }=\vect{ BC }\) (méthode au choix)
    \end{enumerate}
    
    



    $\vect{ AB }=(2;10)$

    $\vect{ AB }+\vect{ BC }=(10;-2)$

    $X=(-1;-18)$
    
\end{enumerate}
}
\vbox{Numéro 9.
\emph{Toutes les réponses doivent être justifiées par un calcul accompagné d'un raisonnement.}
\begin{enumerate}\item

    Soient les points $M(-6;-8)$, $F(9;8)$ et $B(-7;-7)$. Donner les coordonnées du point $E$ tel que $MFBE$ soit un parallélogramme (méthode au choix).
    

$E=(-22;-23)$\item
Soient les points $A(9;8)$, $ B(0,-5)$ et $ C(-9;-1)$. 
    \begin{enumerate}
    \item
    
    Calculer les coordonnées des vecteurs \( \vect{ AB }\) et \( \vect{ AB }+\vect{ BC }\). 

\item
    Donner les coordonnées du point \( X\) tel que \( \vect{ AX }=\vect{ BC }\) (méthode au choix)
    \end{enumerate}
    
    



    $\vect{ AB }=(-9;-13)$

    $\vect{ AB }+\vect{ BC }=(-18;-9)$

    $X=(0;12)$
    
\end{enumerate}
}
\vbox{Numéro 10.
\emph{Toutes les réponses doivent être justifiées par un calcul accompagné d'un raisonnement.}
\begin{enumerate}\item

    Soient les points $B(9;7)$, $D(3;1)$ et $F(6;-3)$. Donner les coordonnées du point $K$ tel que $BDFK$ soit un parallélogramme (méthode au choix).
    

$K=(12;3)$\item
Soient les points $A(10;-6)$, $ B(1,2)$ et $ C(-8;3)$. 
    \begin{enumerate}
    \item
    
    Calculer les coordonnées des vecteurs \( \vect{ AB }\) et \( \vect{ AB }+\vect{ BC }\). 

\item
    Donner les coordonnées du point \( X\) tel que \( \vect{ AX }=\vect{ BC }\) (méthode au choix)
    \end{enumerate}
    
    



    $\vect{ AB }=(-9;8)$

    $\vect{ AB }+\vect{ BC }=(-18;9)$

    $X=(1;-5)$
    
\end{enumerate}
}
\vbox{Numéro 11.
\emph{Toutes les réponses doivent être justifiées par un calcul accompagné d'un raisonnement.}
\begin{enumerate}\item
Soient les points $A(-6;-2)$, $ B(8,8)$ et $ C(-4;0)$. 
    \begin{enumerate}
    \item
    
    Calculer les coordonnées des vecteurs \( \vect{ AB }\) et \( \vect{ AB }+\vect{ BC }\). 

\item
    Donner les coordonnées du point \( X\) tel que \( \vect{ AX }=\vect{ BC }\) (méthode au choix)
    \end{enumerate}
    
    



    $\vect{ AB }=(14;10)$

    $\vect{ AB }+\vect{ BC }=(2;2)$

    $X=(-18;-10)$
    \item

    Soient les points $A(-10;-9)$, $F(-2;7)$ et $M(-8;-2)$. Donner les coordonnées du point $B$ tel que $AFMB$ soit un parallélogramme (méthode au choix).
    

$B=(-16;-18)$
\end{enumerate}
}
\vbox{Numéro 12.
\emph{Toutes les réponses doivent être justifiées par un calcul accompagné d'un raisonnement.}
\begin{enumerate}\item
Soient les points $A(3;9)$, $ B(-8,-9)$ et $ C(2;1)$. 
    \begin{enumerate}
    \item
    
    Calculer les coordonnées des vecteurs \( \vect{ AB }\) et \( \vect{ AB }+\vect{ BC }\). 

\item
    Donner les coordonnées du point \( X\) tel que \( \vect{ AX }=\vect{ BC }\) (méthode au choix)
    \end{enumerate}
    
    



    $\vect{ AB }=(-11;-18)$

    $\vect{ AB }+\vect{ BC }=(-1;-8)$

    $X=(13;19)$
    \item

    Soient les points $F(7;-2)$, $K(-7;4)$ et $M(3;6)$. Donner les coordonnées du point $D$ tel que $FKMD$ soit un parallélogramme (méthode au choix).
    

$D=(17;0)$
\end{enumerate}
}
\vbox{Numéro 13.
\emph{Toutes les réponses doivent être justifiées par un calcul accompagné d'un raisonnement.}
\begin{enumerate}\item

    Soient les points $L(10;7)$, $A(-5;0)$ et $E(-5;5)$. Donner les coordonnées du point $D$ tel que $LAED$ soit un parallélogramme (méthode au choix).
    

$D=(10;12)$\item
Soient les points $A(-6;10)$, $ B(8,-7)$ et $ C(-4;-2)$. 
    \begin{enumerate}
    \item
    
    Calculer les coordonnées des vecteurs \( \vect{ AB }\) et \( \vect{ AB }+\vect{ BC }\). 

\item
    Donner les coordonnées du point \( X\) tel que \( \vect{ AX }=\vect{ BC }\) (méthode au choix)
    \end{enumerate}
    
    



    $\vect{ AB }=(14;-17)$

    $\vect{ AB }+\vect{ BC }=(2;-12)$

    $X=(-18;15)$
    
\end{enumerate}
}
\vbox{Numéro 14.
\emph{Toutes les réponses doivent être justifiées par un calcul accompagné d'un raisonnement.}
\begin{enumerate}\item
Soient les points $A(-8;-4)$, $ B(10,6)$ et $ C(-9;8)$. 
    \begin{enumerate}
    \item
    
    Calculer les coordonnées des vecteurs \( \vect{ AB }\) et \( \vect{ AB }+\vect{ BC }\). 

\item
    Donner les coordonnées du point \( X\) tel que \( \vect{ AX }=\vect{ BC }\) (méthode au choix)
    \end{enumerate}
    
    



    $\vect{ AB }=(18;10)$

    $\vect{ AB }+\vect{ BC }=(-1;12)$

    $X=(-27;-2)$
    \item

    Soient les points $B(-5;6)$, $D(10;-8)$ et $A(-3;-1)$. Donner les coordonnées du point $K$ tel que $BDAK$ soit un parallélogramme (méthode au choix).
    

$K=(-18;13)$
\end{enumerate}
}
\vbox{Numéro 15.
\emph{Toutes les réponses doivent être justifiées par un calcul accompagné d'un raisonnement.}
\begin{enumerate}\item
Soient les points $A(-9;1)$, $ B(3,-9)$ et $ C(-2;-7)$. 
    \begin{enumerate}
    \item
    
    Calculer les coordonnées des vecteurs \( \vect{ AB }\) et \( \vect{ AB }+\vect{ BC }\). 

\item
    Donner les coordonnées du point \( X\) tel que \( \vect{ AX }=\vect{ BC }\) (méthode au choix)
    \end{enumerate}
    
    



    $\vect{ AB }=(12;-10)$

    $\vect{ AB }+\vect{ BC }=(7;-8)$

    $X=(-14;3)$
    \item

    Soient les points $F(-1;-7)$, $K(-5;-4)$ et $A(0;2)$. Donner les coordonnées du point $L$ tel que $FKAL$ soit un parallélogramme (méthode au choix).
    

$L=(4;-1)$
\end{enumerate}
}
\vbox{Numéro 16.
\emph{Toutes les réponses doivent être justifiées par un calcul accompagné d'un raisonnement.}
\begin{enumerate}\item

    Soient les points $L(6;-10)$, $M(-5;10)$ et $E(7;7)$. Donner les coordonnées du point $K$ tel que $LMEK$ soit un parallélogramme (méthode au choix).
    

$K=(18;-13)$\item
Soient les points $A(3;5)$, $ B(0,2)$ et $ C(-5;2)$. 
    \begin{enumerate}
    \item
    
    Calculer les coordonnées des vecteurs \( \vect{ AB }\) et \( \vect{ AB }+\vect{ BC }\). 

\item
    Donner les coordonnées du point \( X\) tel que \( \vect{ AX }=\vect{ BC }\) (méthode au choix)
    \end{enumerate}
    
    



    $\vect{ AB }=(-3;-3)$

    $\vect{ AB }+\vect{ BC }=(-8;-3)$

    $X=(-2;5)$
    
\end{enumerate}
}
\vbox{Numéro 17.
\emph{Toutes les réponses doivent être justifiées par un calcul accompagné d'un raisonnement.}
\begin{enumerate}\item
Soient les points $A(5;-1)$, $ B(2,-8)$ et $ C(10;-3)$. 
    \begin{enumerate}
    \item
    
    Calculer les coordonnées des vecteurs \( \vect{ AB }\) et \( \vect{ AB }+\vect{ BC }\). 

\item
    Donner les coordonnées du point \( X\) tel que \( \vect{ AX }=\vect{ BC }\) (méthode au choix)
    \end{enumerate}
    
    



    $\vect{ AB }=(-3;-7)$

    $\vect{ AB }+\vect{ BC }=(5;-2)$

    $X=(13;4)$
    \item

    Soient les points $M(-6;-3)$, $B(-3;3)$ et $E(7;3)$. Donner les coordonnées du point $F$ tel que $MBEF$ soit un parallélogramme (méthode au choix).
    

$F=(4;-3)$
\end{enumerate}
}
\vbox{Numéro 18.
\emph{Toutes les réponses doivent être justifiées par un calcul accompagné d'un raisonnement.}
\begin{enumerate}\item

    Soient les points $E(0;9)$, $L(-2;4)$ et $M(-9;9)$. Donner les coordonnées du point $B$ tel que $ELMB$ soit un parallélogramme (méthode au choix).
    

$B=(-7;14)$\item
Soient les points $A(-9;2)$, $ B(4,-7)$ et $ C(-8;5)$. 
    \begin{enumerate}
    \item
    
    Calculer les coordonnées des vecteurs \( \vect{ AB }\) et \( \vect{ AB }+\vect{ BC }\). 

\item
    Donner les coordonnées du point \( X\) tel que \( \vect{ AX }=\vect{ BC }\) (méthode au choix)
    \end{enumerate}
    
    



    $\vect{ AB }=(13;-9)$

    $\vect{ AB }+\vect{ BC }=(1;3)$

    $X=(-21;14)$
    
\end{enumerate}
}
\vbox{Numéro 19.
\emph{Toutes les réponses doivent être justifiées par un calcul accompagné d'un raisonnement.}
\begin{enumerate}\item

    Soient les points $K(-4;0)$, $A(-2;-2)$ et $F(3;6)$. Donner les coordonnées du point $M$ tel que $KAFM$ soit un parallélogramme (méthode au choix).
    

$M=(1;8)$\item
Soient les points $A(6;6)$, $ B(-8,8)$ et $ C(-3;-5)$. 
    \begin{enumerate}
    \item
    
    Calculer les coordonnées des vecteurs \( \vect{ AB }\) et \( \vect{ AB }+\vect{ BC }\). 

\item
    Donner les coordonnées du point \( X\) tel que \( \vect{ AX }=\vect{ BC }\) (méthode au choix)
    \end{enumerate}
    
    



    $\vect{ AB }=(-14;2)$

    $\vect{ AB }+\vect{ BC }=(-9;-11)$

    $X=(11;-7)$
    
\end{enumerate}
}
\vbox{Numéro 20.
\emph{Toutes les réponses doivent être justifiées par un calcul accompagné d'un raisonnement.}
\begin{enumerate}\item

    Soient les points $B(-3;4)$, $M(-1;-6)$ et $L(-6;9)$. Donner les coordonnées du point $K$ tel que $BMLK$ soit un parallélogramme (méthode au choix).
    

$K=(-8;19)$\item
Soient les points $A(3;2)$, $ B(-9,-6)$ et $ C(-9;-2)$. 
    \begin{enumerate}
    \item
    
    Calculer les coordonnées des vecteurs \( \vect{ AB }\) et \( \vect{ AB }+\vect{ BC }\). 

\item
    Donner les coordonnées du point \( X\) tel que \( \vect{ AX }=\vect{ BC }\) (méthode au choix)
    \end{enumerate}
    
    



    $\vect{ AB }=(-12;-8)$

    $\vect{ AB }+\vect{ BC }=(-12;-4)$

    $X=(3;6)$
    
\end{enumerate}
}
\vbox{Numéro 21.
\emph{Toutes les réponses doivent être justifiées par un calcul accompagné d'un raisonnement.}
\begin{enumerate}\item
Soient les points $A(2;-8)$, $ B(4,-8)$ et $ C(-7;3)$. 
    \begin{enumerate}
    \item
    
    Calculer les coordonnées des vecteurs \( \vect{ AB }\) et \( \vect{ AB }+\vect{ BC }\). 

\item
    Donner les coordonnées du point \( X\) tel que \( \vect{ AX }=\vect{ BC }\) (méthode au choix)
    \end{enumerate}
    
    



    $\vect{ AB }=(2;0)$

    $\vect{ AB }+\vect{ BC }=(-9;11)$

    $X=(-9;3)$
    \item

    Soient les points $F(9;-3)$, $D(-7;8)$ et $K(-5;-7)$. Donner les coordonnées du point $M$ tel que $FDKM$ soit un parallélogramme (méthode au choix).
    

$M=(11;-18)$
\end{enumerate}
}
\vbox{Numéro 22.
\emph{Toutes les réponses doivent être justifiées par un calcul accompagné d'un raisonnement.}
\begin{enumerate}\item
Soient les points $A(5;5)$, $ B(-4,-8)$ et $ C(-1;-2)$. 
    \begin{enumerate}
    \item
    
    Calculer les coordonnées des vecteurs \( \vect{ AB }\) et \( \vect{ AB }+\vect{ BC }\). 

\item
    Donner les coordonnées du point \( X\) tel que \( \vect{ AX }=\vect{ BC }\) (méthode au choix)
    \end{enumerate}
    
    



    $\vect{ AB }=(-9;-13)$

    $\vect{ AB }+\vect{ BC }=(-6;-7)$

    $X=(8;11)$
    \item

    Soient les points $M(-1;-7)$, $L(-6;-10)$ et $B(6;6)$. Donner les coordonnées du point $D$ tel que $MLBD$ soit un parallélogramme (méthode au choix).
    

$D=(11;9)$
\end{enumerate}
}
\vbox{Numéro 23.
\emph{Toutes les réponses doivent être justifiées par un calcul accompagné d'un raisonnement.}
\begin{enumerate}\item
Soient les points $A(-5;-4)$, $ B(-1,7)$ et $ C(4;5)$. 
    \begin{enumerate}
    \item
    
    Calculer les coordonnées des vecteurs \( \vect{ AB }\) et \( \vect{ AB }+\vect{ BC }\). 

\item
    Donner les coordonnées du point \( X\) tel que \( \vect{ AX }=\vect{ BC }\) (méthode au choix)
    \end{enumerate}
    
    



    $\vect{ AB }=(4;11)$

    $\vect{ AB }+\vect{ BC }=(9;9)$

    $X=(0;-6)$
    \item

    Soient les points $M(7;-5)$, $D(-2;1)$ et $B(-5;-2)$. Donner les coordonnées du point $A$ tel que $MDBA$ soit un parallélogramme (méthode au choix).
    

$A=(4;-8)$
\end{enumerate}
}
\vbox{Numéro 24.
\emph{Toutes les réponses doivent être justifiées par un calcul accompagné d'un raisonnement.}
\begin{enumerate}\item

    Soient les points $A(8;-6)$, $L(9;-8)$ et $M(10;9)$. Donner les coordonnées du point $E$ tel que $ALME$ soit un parallélogramme (méthode au choix).
    

$E=(9;11)$\item
Soient les points $A(3;-1)$, $ B(-6,-4)$ et $ C(7;-8)$. 
    \begin{enumerate}
    \item
    
    Calculer les coordonnées des vecteurs \( \vect{ AB }\) et \( \vect{ AB }+\vect{ BC }\). 

\item
    Donner les coordonnées du point \( X\) tel que \( \vect{ AX }=\vect{ BC }\) (méthode au choix)
    \end{enumerate}
    
    



    $\vect{ AB }=(-9;-3)$

    $\vect{ AB }+\vect{ BC }=(4;-7)$

    $X=(16;-5)$
    
\end{enumerate}
}
\vbox{Numéro 25.
\emph{Toutes les réponses doivent être justifiées par un calcul accompagné d'un raisonnement.}
\begin{enumerate}\item
Soient les points $A(-4;5)$, $ B(-5,-7)$ et $ C(9;4)$. 
    \begin{enumerate}
    \item
    
    Calculer les coordonnées des vecteurs \( \vect{ AB }\) et \( \vect{ AB }+\vect{ BC }\). 

\item
    Donner les coordonnées du point \( X\) tel que \( \vect{ AX }=\vect{ BC }\) (méthode au choix)
    \end{enumerate}
    
    



    $\vect{ AB }=(-1;-12)$

    $\vect{ AB }+\vect{ BC }=(13;-1)$

    $X=(10;16)$
    \item

    Soient les points $K(-2;7)$, $E(1;-9)$ et $B(7;0)$. Donner les coordonnées du point $F$ tel que $KEBF$ soit un parallélogramme (méthode au choix).
    

$F=(4;16)$
\end{enumerate}
}
\vbox{Numéro 26.
\emph{Toutes les réponses doivent être justifiées par un calcul accompagné d'un raisonnement.}
\begin{enumerate}\item
Soient les points $A(-2;0)$, $ B(2,-8)$ et $ C(-8;-3)$. 
    \begin{enumerate}
    \item
    
    Calculer les coordonnées des vecteurs \( \vect{ AB }\) et \( \vect{ AB }+\vect{ BC }\). 

\item
    Donner les coordonnées du point \( X\) tel que \( \vect{ AX }=\vect{ BC }\) (méthode au choix)
    \end{enumerate}
    
    



    $\vect{ AB }=(4;-8)$

    $\vect{ AB }+\vect{ BC }=(-6;-3)$

    $X=(-12;5)$
    \item

    Soient les points $B(7;-8)$, $A(6;-3)$ et $L(3;8)$. Donner les coordonnées du point $M$ tel que $BALM$ soit un parallélogramme (méthode au choix).
    

$M=(4;3)$
\end{enumerate}
}
\vbox{Numéro 27.
\emph{Toutes les réponses doivent être justifiées par un calcul accompagné d'un raisonnement.}
\begin{enumerate}\item

    Soient les points $M(6;9)$, $B(3;1)$ et $F(-10;-1)$. Donner les coordonnées du point $K$ tel que $MBFK$ soit un parallélogramme (méthode au choix).
    

$K=(-7;7)$\item
Soient les points $A(-7;-4)$, $ B(8,4)$ et $ C(-3;-8)$. 
    \begin{enumerate}
    \item
    
    Calculer les coordonnées des vecteurs \( \vect{ AB }\) et \( \vect{ AB }+\vect{ BC }\). 

\item
    Donner les coordonnées du point \( X\) tel que \( \vect{ AX }=\vect{ BC }\) (méthode au choix)
    \end{enumerate}
    
    



    $\vect{ AB }=(15;8)$

    $\vect{ AB }+\vect{ BC }=(4;-4)$

    $X=(-18;-16)$
    
\end{enumerate}
}
\vbox{Numéro 28.
\emph{Toutes les réponses doivent être justifiées par un calcul accompagné d'un raisonnement.}
\begin{enumerate}\item

    Soient les points $E(-6;-10)$, $D(-10;9)$ et $L(7;5)$. Donner les coordonnées du point $K$ tel que $EDLK$ soit un parallélogramme (méthode au choix).
    

$K=(11;-14)$\item
Soient les points $A(8;-10)$, $ B(-6,0)$ et $ C(1;-1)$. 
    \begin{enumerate}
    \item
    
    Calculer les coordonnées des vecteurs \( \vect{ AB }\) et \( \vect{ AB }+\vect{ BC }\). 

\item
    Donner les coordonnées du point \( X\) tel que \( \vect{ AX }=\vect{ BC }\) (méthode au choix)
    \end{enumerate}
    
    



    $\vect{ AB }=(-14;10)$

    $\vect{ AB }+\vect{ BC }=(-7;9)$

    $X=(15;-11)$
    
\end{enumerate}
}
\vbox{Numéro 29.
\emph{Toutes les réponses doivent être justifiées par un calcul accompagné d'un raisonnement.}
\begin{enumerate}\item
Soient les points $A(-3;9)$, $ B(1,9)$ et $ C(1;10)$. 
    \begin{enumerate}
    \item
    
    Calculer les coordonnées des vecteurs \( \vect{ AB }\) et \( \vect{ AB }+\vect{ BC }\). 

\item
    Donner les coordonnées du point \( X\) tel que \( \vect{ AX }=\vect{ BC }\) (méthode au choix)
    \end{enumerate}
    
    



    $\vect{ AB }=(4;0)$

    $\vect{ AB }+\vect{ BC }=(4;1)$

    $X=(-3;10)$
    \item

    Soient les points $A(2;-5)$, $F(-6;7)$ et $B(-10;7)$. Donner les coordonnées du point $E$ tel que $AFBE$ soit un parallélogramme (méthode au choix).
    

$E=(-2;-5)$
\end{enumerate}
}
\vbox{Numéro 30.
\emph{Toutes les réponses doivent être justifiées par un calcul accompagné d'un raisonnement.}
\begin{enumerate}\item
Soient les points $A(-10;5)$, $ B(-5,7)$ et $ C(1;-4)$. 
    \begin{enumerate}
    \item
    
    Calculer les coordonnées des vecteurs \( \vect{ AB }\) et \( \vect{ AB }+\vect{ BC }\). 

\item
    Donner les coordonnées du point \( X\) tel que \( \vect{ AX }=\vect{ BC }\) (méthode au choix)
    \end{enumerate}
    
    



    $\vect{ AB }=(5;2)$

    $\vect{ AB }+\vect{ BC }=(11;-9)$

    $X=(-4;-6)$
    \item

    Soient les points $A(-9;-1)$, $E(-7;6)$ et $F(4;5)$. Donner les coordonnées du point $B$ tel que $AEFB$ soit un parallélogramme (méthode au choix).
    

$B=(2;-2)$
\end{enumerate}
}
\vbox{Numéro 31.
\emph{Toutes les réponses doivent être justifiées par un calcul accompagné d'un raisonnement.}
\begin{enumerate}\item

    Soient les points $B(-3;-9)$, $L(-9;-5)$ et $A(2;10)$. Donner les coordonnées du point $E$ tel que $BLAE$ soit un parallélogramme (méthode au choix).
    

$E=(8;6)$\item
Soient les points $A(8;-10)$, $ B(10,-2)$ et $ C(10;-4)$. 
    \begin{enumerate}
    \item
    
    Calculer les coordonnées des vecteurs \( \vect{ AB }\) et \( \vect{ AB }+\vect{ BC }\). 

\item
    Donner les coordonnées du point \( X\) tel que \( \vect{ AX }=\vect{ BC }\) (méthode au choix)
    \end{enumerate}
    
    



    $\vect{ AB }=(2;8)$

    $\vect{ AB }+\vect{ BC }=(2;6)$

    $X=(8;-12)$
    
\end{enumerate}
}
\vbox{Numéro 32.
\emph{Toutes les réponses doivent être justifiées par un calcul accompagné d'un raisonnement.}
\begin{enumerate}\item
Soient les points $A(3;0)$, $ B(2,6)$ et $ C(-6;-4)$. 
    \begin{enumerate}
    \item
    
    Calculer les coordonnées des vecteurs \( \vect{ AB }\) et \( \vect{ AB }+\vect{ BC }\). 

\item
    Donner les coordonnées du point \( X\) tel que \( \vect{ AX }=\vect{ BC }\) (méthode au choix)
    \end{enumerate}
    
    



    $\vect{ AB }=(-1;6)$

    $\vect{ AB }+\vect{ BC }=(-9;-4)$

    $X=(-5;-10)$
    \item

    Soient les points $D(-7;5)$, $F(4;3)$ et $M(-4;8)$. Donner les coordonnées du point $E$ tel que $DFME$ soit un parallélogramme (méthode au choix).
    

$E=(-15;10)$
\end{enumerate}
}
\vbox{Numéro 33.
\emph{Toutes les réponses doivent être justifiées par un calcul accompagné d'un raisonnement.}
\begin{enumerate}\item
Soient les points $A(4;7)$, $ B(-1,4)$ et $ C(4;3)$. 
    \begin{enumerate}
    \item
    
    Calculer les coordonnées des vecteurs \( \vect{ AB }\) et \( \vect{ AB }+\vect{ BC }\). 

\item
    Donner les coordonnées du point \( X\) tel que \( \vect{ AX }=\vect{ BC }\) (méthode au choix)
    \end{enumerate}
    
    



    $\vect{ AB }=(-5;-3)$

    $\vect{ AB }+\vect{ BC }=(0;-4)$

    $X=(9;6)$
    \item

    Soient les points $D(10;-6)$, $M(-8;-3)$ et $B(9;8)$. Donner les coordonnées du point $L$ tel que $DMBL$ soit un parallélogramme (méthode au choix).
    

$L=(27;5)$
\end{enumerate}
}
\vbox{Numéro 34.
\emph{Toutes les réponses doivent être justifiées par un calcul accompagné d'un raisonnement.}
\begin{enumerate}\item

    Soient les points $M(-8;-6)$, $K(4;3)$ et $F(-10;6)$. Donner les coordonnées du point $E$ tel que $MKFE$ soit un parallélogramme (méthode au choix).
    

$E=(-22;-3)$\item
Soient les points $A(9;-1)$, $ B(-5,-3)$ et $ C(8;-4)$. 
    \begin{enumerate}
    \item
    
    Calculer les coordonnées des vecteurs \( \vect{ AB }\) et \( \vect{ AB }+\vect{ BC }\). 

\item
    Donner les coordonnées du point \( X\) tel que \( \vect{ AX }=\vect{ BC }\) (méthode au choix)
    \end{enumerate}
    
    



    $\vect{ AB }=(-14;-2)$

    $\vect{ AB }+\vect{ BC }=(-1;-3)$

    $X=(22;-2)$
    
\end{enumerate}
}
\vbox{Numéro 35.
\emph{Toutes les réponses doivent être justifiées par un calcul accompagné d'un raisonnement.}
\begin{enumerate}\item
Soient les points $A(3;1)$, $ B(-5,8)$ et $ C(3;-7)$. 
    \begin{enumerate}
    \item
    
    Calculer les coordonnées des vecteurs \( \vect{ AB }\) et \( \vect{ AB }+\vect{ BC }\). 

\item
    Donner les coordonnées du point \( X\) tel que \( \vect{ AX }=\vect{ BC }\) (méthode au choix)
    \end{enumerate}
    
    



    $\vect{ AB }=(-8;7)$

    $\vect{ AB }+\vect{ BC }=(0;-8)$

    $X=(11;-14)$
    \item

    Soient les points $K(-1;-2)$, $D(2;-4)$ et $F(2;7)$. Donner les coordonnées du point $A$ tel que $KDFA$ soit un parallélogramme (méthode au choix).
    

$A=(-1;9)$
\end{enumerate}
}
\vbox{Numéro 36.
\emph{Toutes les réponses doivent être justifiées par un calcul accompagné d'un raisonnement.}
\begin{enumerate}\item
Soient les points $A(-7;6)$, $ B(6,-6)$ et $ C(-10;-7)$. 
    \begin{enumerate}
    \item
    
    Calculer les coordonnées des vecteurs \( \vect{ AB }\) et \( \vect{ AB }+\vect{ BC }\). 

\item
    Donner les coordonnées du point \( X\) tel que \( \vect{ AX }=\vect{ BC }\) (méthode au choix)
    \end{enumerate}
    
    



    $\vect{ AB }=(13;-12)$

    $\vect{ AB }+\vect{ BC }=(-3;-13)$

    $X=(-23;5)$
    \item

    Soient les points $A(3;1)$, $D(-6;-8)$ et $L(3;-9)$. Donner les coordonnées du point $B$ tel que $ADLB$ soit un parallélogramme (méthode au choix).
    

$B=(12;0)$
\end{enumerate}
}
\vbox{Numéro 37.
\emph{Toutes les réponses doivent être justifiées par un calcul accompagné d'un raisonnement.}
\begin{enumerate}\item

    Soient les points $K(4;6)$, $F(2;-8)$ et $B(-5;2)$. Donner les coordonnées du point $D$ tel que $KFBD$ soit un parallélogramme (méthode au choix).
    

$D=(-3;16)$\item
Soient les points $A(5;-5)$, $ B(10,-5)$ et $ C(4;0)$. 
    \begin{enumerate}
    \item
    
    Calculer les coordonnées des vecteurs \( \vect{ AB }\) et \( \vect{ AB }+\vect{ BC }\). 

\item
    Donner les coordonnées du point \( X\) tel que \( \vect{ AX }=\vect{ BC }\) (méthode au choix)
    \end{enumerate}
    
    



    $\vect{ AB }=(5;0)$

    $\vect{ AB }+\vect{ BC }=(-1;5)$

    $X=(-1;0)$
    
\end{enumerate}
}
\vbox{Numéro 38.
\emph{Toutes les réponses doivent être justifiées par un calcul accompagné d'un raisonnement.}
\begin{enumerate}\item
Soient les points $A(-9;-1)$, $ B(10,-10)$ et $ C(-4;6)$. 
    \begin{enumerate}
    \item
    
    Calculer les coordonnées des vecteurs \( \vect{ AB }\) et \( \vect{ AB }+\vect{ BC }\). 

\item
    Donner les coordonnées du point \( X\) tel que \( \vect{ AX }=\vect{ BC }\) (méthode au choix)
    \end{enumerate}
    
    



    $\vect{ AB }=(19;-9)$

    $\vect{ AB }+\vect{ BC }=(5;7)$

    $X=(-23;15)$
    \item

    Soient les points $K(-7;1)$, $D(0;6)$ et $M(-4;-1)$. Donner les coordonnées du point $B$ tel que $KDMB$ soit un parallélogramme (méthode au choix).
    

$B=(-11;-6)$
\end{enumerate}
}
\vbox{Numéro 39.
\emph{Toutes les réponses doivent être justifiées par un calcul accompagné d'un raisonnement.}
\begin{enumerate}\item
Soient les points $A(-10;9)$, $ B(-6,9)$ et $ C(4;0)$. 
    \begin{enumerate}
    \item
    
    Calculer les coordonnées des vecteurs \( \vect{ AB }\) et \( \vect{ AB }+\vect{ BC }\). 

\item
    Donner les coordonnées du point \( X\) tel que \( \vect{ AX }=\vect{ BC }\) (méthode au choix)
    \end{enumerate}
    
    



    $\vect{ AB }=(4;0)$

    $\vect{ AB }+\vect{ BC }=(14;-9)$

    $X=(0;0)$
    \item

    Soient les points $L(-1;-2)$, $M(-4;-7)$ et $E(-6;-6)$. Donner les coordonnées du point $D$ tel que $LMED$ soit un parallélogramme (méthode au choix).
    

$D=(-3;-1)$
\end{enumerate}
}
\vbox{Numéro 40.
\emph{Toutes les réponses doivent être justifiées par un calcul accompagné d'un raisonnement.}
\begin{enumerate}\item

    Soient les points $K(9;10)$, $F(-7;5)$ et $L(-9;0)$. Donner les coordonnées du point $B$ tel que $KFLB$ soit un parallélogramme (méthode au choix).
    

$B=(7;5)$\item
Soient les points $A(-3;9)$, $ B(-3,-2)$ et $ C(0;0)$. 
    \begin{enumerate}
    \item
    
    Calculer les coordonnées des vecteurs \( \vect{ AB }\) et \( \vect{ AB }+\vect{ BC }\). 

\item
    Donner les coordonnées du point \( X\) tel que \( \vect{ AX }=\vect{ BC }\) (méthode au choix)
    \end{enumerate}
    
    



    $\vect{ AB }=(0;-11)$

    $\vect{ AB }+\vect{ BC }=(3;-9)$

    $X=(0;11)$
    
\end{enumerate}
}
\vbox{Numéro 41.
\emph{Toutes les réponses doivent être justifiées par un calcul accompagné d'un raisonnement.}
\begin{enumerate}\item
Soient les points $A(-4;5)$, $ B(8,1)$ et $ C(4;-2)$. 
    \begin{enumerate}
    \item
    
    Calculer les coordonnées des vecteurs \( \vect{ AB }\) et \( \vect{ AB }+\vect{ BC }\). 

\item
    Donner les coordonnées du point \( X\) tel que \( \vect{ AX }=\vect{ BC }\) (méthode au choix)
    \end{enumerate}
    
    



    $\vect{ AB }=(12;-4)$

    $\vect{ AB }+\vect{ BC }=(8;-7)$

    $X=(-8;2)$
    \item

    Soient les points $M(10;3)$, $D(2;-4)$ et $L(-4;-4)$. Donner les coordonnées du point $E$ tel que $MDLE$ soit un parallélogramme (méthode au choix).
    

$E=(4;3)$
\end{enumerate}
}
\vbox{Numéro 42.
\emph{Toutes les réponses doivent être justifiées par un calcul accompagné d'un raisonnement.}
\begin{enumerate}\item

    Soient les points $L(7;-5)$, $A(0;10)$ et $D(-8;-8)$. Donner les coordonnées du point $E$ tel que $LADE$ soit un parallélogramme (méthode au choix).
    

$E=(-1;-23)$\item
Soient les points $A(0;8)$, $ B(-3,5)$ et $ C(4;9)$. 
    \begin{enumerate}
    \item
    
    Calculer les coordonnées des vecteurs \( \vect{ AB }\) et \( \vect{ AB }+\vect{ BC }\). 

\item
    Donner les coordonnées du point \( X\) tel que \( \vect{ AX }=\vect{ BC }\) (méthode au choix)
    \end{enumerate}
    
    



    $\vect{ AB }=(-3;-3)$

    $\vect{ AB }+\vect{ BC }=(4;1)$

    $X=(7;12)$
    
\end{enumerate}
}
\vbox{Numéro 43.
\emph{Toutes les réponses doivent être justifiées par un calcul accompagné d'un raisonnement.}
\begin{enumerate}\item

    Soient les points $D(1;5)$, $A(-10;-3)$ et $K(-6;-5)$. Donner les coordonnées du point $F$ tel que $DAKF$ soit un parallélogramme (méthode au choix).
    

$F=(5;3)$\item
Soient les points $A(10;8)$, $ B(9,1)$ et $ C(1;-3)$. 
    \begin{enumerate}
    \item
    
    Calculer les coordonnées des vecteurs \( \vect{ AB }\) et \( \vect{ AB }+\vect{ BC }\). 

\item
    Donner les coordonnées du point \( X\) tel que \( \vect{ AX }=\vect{ BC }\) (méthode au choix)
    \end{enumerate}
    
    



    $\vect{ AB }=(-1;-7)$

    $\vect{ AB }+\vect{ BC }=(-9;-11)$

    $X=(2;4)$
    
\end{enumerate}
}
\vbox{Numéro 44.
\emph{Toutes les réponses doivent être justifiées par un calcul accompagné d'un raisonnement.}
\begin{enumerate}\item

    Soient les points $A(2;7)$, $M(-9;-7)$ et $K(-5;-5)$. Donner les coordonnées du point $B$ tel que $AMKB$ soit un parallélogramme (méthode au choix).
    

$B=(6;9)$\item
Soient les points $A(-9;-4)$, $ B(-4,2)$ et $ C(6;-1)$. 
    \begin{enumerate}
    \item
    
    Calculer les coordonnées des vecteurs \( \vect{ AB }\) et \( \vect{ AB }+\vect{ BC }\). 

\item
    Donner les coordonnées du point \( X\) tel que \( \vect{ AX }=\vect{ BC }\) (méthode au choix)
    \end{enumerate}
    
    



    $\vect{ AB }=(5;6)$

    $\vect{ AB }+\vect{ BC }=(15;3)$

    $X=(1;-7)$
    
\end{enumerate}
}
\vbox{Numéro 45.
\emph{Toutes les réponses doivent être justifiées par un calcul accompagné d'un raisonnement.}
\begin{enumerate}\item
Soient les points $A(-5;0)$, $ B(2,4)$ et $ C(-1;5)$. 
    \begin{enumerate}
    \item
    
    Calculer les coordonnées des vecteurs \( \vect{ AB }\) et \( \vect{ AB }+\vect{ BC }\). 

\item
    Donner les coordonnées du point \( X\) tel que \( \vect{ AX }=\vect{ BC }\) (méthode au choix)
    \end{enumerate}
    
    



    $\vect{ AB }=(7;4)$

    $\vect{ AB }+\vect{ BC }=(4;5)$

    $X=(-8;1)$
    \item

    Soient les points $E(-5;-4)$, $D(-2;-5)$ et $F(-9;3)$. Donner les coordonnées du point $B$ tel que $EDFB$ soit un parallélogramme (méthode au choix).
    

$B=(-12;4)$
\end{enumerate}
}
\vbox{Numéro 46.
\emph{Toutes les réponses doivent être justifiées par un calcul accompagné d'un raisonnement.}
\begin{enumerate}\item
Soient les points $A(-5;-5)$, $ B(8,-3)$ et $ C(6;-9)$. 
    \begin{enumerate}
    \item
    
    Calculer les coordonnées des vecteurs \( \vect{ AB }\) et \( \vect{ AB }+\vect{ BC }\). 

\item
    Donner les coordonnées du point \( X\) tel que \( \vect{ AX }=\vect{ BC }\) (méthode au choix)
    \end{enumerate}
    
    



    $\vect{ AB }=(13;2)$

    $\vect{ AB }+\vect{ BC }=(11;-4)$

    $X=(-7;-11)$
    \item

    Soient les points $E(-4;-7)$, $A(3;0)$ et $M(-10;-8)$. Donner les coordonnées du point $B$ tel que $EAMB$ soit un parallélogramme (méthode au choix).
    

$B=(-17;-15)$
\end{enumerate}
}
\vbox{Numéro 47.
\emph{Toutes les réponses doivent être justifiées par un calcul accompagné d'un raisonnement.}
\begin{enumerate}\item

    Soient les points $F(-9;5)$, $M(-8;4)$ et $E(7;6)$. Donner les coordonnées du point $A$ tel que $FMEA$ soit un parallélogramme (méthode au choix).
    

$A=(6;7)$\item
Soient les points $A(9;9)$, $ B(-10,10)$ et $ C(-1;1)$. 
    \begin{enumerate}
    \item
    
    Calculer les coordonnées des vecteurs \( \vect{ AB }\) et \( \vect{ AB }+\vect{ BC }\). 

\item
    Donner les coordonnées du point \( X\) tel que \( \vect{ AX }=\vect{ BC }\) (méthode au choix)
    \end{enumerate}
    
    



    $\vect{ AB }=(-19;1)$

    $\vect{ AB }+\vect{ BC }=(-10;-8)$

    $X=(18;0)$
    
\end{enumerate}
}
\vbox{Numéro 48.
\emph{Toutes les réponses doivent être justifiées par un calcul accompagné d'un raisonnement.}
\begin{enumerate}\item
Soient les points $A(-6;-5)$, $ B(-1,-3)$ et $ C(-1;-6)$. 
    \begin{enumerate}
    \item
    
    Calculer les coordonnées des vecteurs \( \vect{ AB }\) et \( \vect{ AB }+\vect{ BC }\). 

\item
    Donner les coordonnées du point \( X\) tel que \( \vect{ AX }=\vect{ BC }\) (méthode au choix)
    \end{enumerate}
    
    



    $\vect{ AB }=(5;2)$

    $\vect{ AB }+\vect{ BC }=(5;-1)$

    $X=(-6;-8)$
    \item

    Soient les points $D(2;-2)$, $B(-7;-2)$ et $F(-8;10)$. Donner les coordonnées du point $K$ tel que $DBFK$ soit un parallélogramme (méthode au choix).
    

$K=(1;10)$
\end{enumerate}
}
\vbox{Numéro 49.
\emph{Toutes les réponses doivent être justifiées par un calcul accompagné d'un raisonnement.}
\begin{enumerate}\item

    Soient les points $D(9;-7)$, $B(-8;2)$ et $K(9;4)$. Donner les coordonnées du point $A$ tel que $DBKA$ soit un parallélogramme (méthode au choix).
    

$A=(26;-5)$\item
Soient les points $A(-1;6)$, $ B(8,2)$ et $ C(-3;1)$. 
    \begin{enumerate}
    \item
    
    Calculer les coordonnées des vecteurs \( \vect{ AB }\) et \( \vect{ AB }+\vect{ BC }\). 

\item
    Donner les coordonnées du point \( X\) tel que \( \vect{ AX }=\vect{ BC }\) (méthode au choix)
    \end{enumerate}
    
    



    $\vect{ AB }=(9;-4)$

    $\vect{ AB }+\vect{ BC }=(-2;-5)$

    $X=(-12;5)$
    
\end{enumerate}
}
\vbox{Numéro 50.
\emph{Toutes les réponses doivent être justifiées par un calcul accompagné d'un raisonnement.}
\begin{enumerate}\item
Soient les points $A(-5;-1)$, $ B(9,-8)$ et $ C(-7;-5)$. 
    \begin{enumerate}
    \item
    
    Calculer les coordonnées des vecteurs \( \vect{ AB }\) et \( \vect{ AB }+\vect{ BC }\). 

\item
    Donner les coordonnées du point \( X\) tel que \( \vect{ AX }=\vect{ BC }\) (méthode au choix)
    \end{enumerate}
    
    



    $\vect{ AB }=(14;-7)$

    $\vect{ AB }+\vect{ BC }=(-2;-4)$

    $X=(-21;2)$
    \item

    Soient les points $M(-4;-5)$, $E(-6;-3)$ et $B(9;-1)$. Donner les coordonnées du point $A$ tel que $MEBA$ soit un parallélogramme (méthode au choix).
    

$A=(11;-3)$
\end{enumerate}
}
\vbox{Numéro 51.
\emph{Toutes les réponses doivent être justifiées par un calcul accompagné d'un raisonnement.}
\begin{enumerate}\item

    Soient les points $M(3;-4)$, $L(-9;8)$ et $K(-6;-6)$. Donner les coordonnées du point $A$ tel que $MLKA$ soit un parallélogramme (méthode au choix).
    

$A=(6;-18)$\item
Soient les points $A(-7;3)$, $ B(5,-8)$ et $ C(2;6)$. 
    \begin{enumerate}
    \item
    
    Calculer les coordonnées des vecteurs \( \vect{ AB }\) et \( \vect{ AB }+\vect{ BC }\). 

\item
    Donner les coordonnées du point \( X\) tel que \( \vect{ AX }=\vect{ BC }\) (méthode au choix)
    \end{enumerate}
    
    



    $\vect{ AB }=(12;-11)$

    $\vect{ AB }+\vect{ BC }=(9;3)$

    $X=(-10;17)$
    
\end{enumerate}
}
\vbox{Numéro 52.
\emph{Toutes les réponses doivent être justifiées par un calcul accompagné d'un raisonnement.}
\begin{enumerate}\item
Soient les points $A(3;-3)$, $ B(4,9)$ et $ C(8;-1)$. 
    \begin{enumerate}
    \item
    
    Calculer les coordonnées des vecteurs \( \vect{ AB }\) et \( \vect{ AB }+\vect{ BC }\). 

\item
    Donner les coordonnées du point \( X\) tel que \( \vect{ AX }=\vect{ BC }\) (méthode au choix)
    \end{enumerate}
    
    



    $\vect{ AB }=(1;12)$

    $\vect{ AB }+\vect{ BC }=(5;2)$

    $X=(7;-13)$
    \item

    Soient les points $B(-1;4)$, $L(2;-9)$ et $E(-2;-6)$. Donner les coordonnées du point $F$ tel que $BLEF$ soit un parallélogramme (méthode au choix).
    

$F=(-5;7)$
\end{enumerate}
}
\vbox{Numéro 53.
\emph{Toutes les réponses doivent être justifiées par un calcul accompagné d'un raisonnement.}
\begin{enumerate}\item
Soient les points $A(-3;-4)$, $ B(-9,-7)$ et $ C(1;10)$. 
    \begin{enumerate}
    \item
    
    Calculer les coordonnées des vecteurs \( \vect{ AB }\) et \( \vect{ AB }+\vect{ BC }\). 

\item
    Donner les coordonnées du point \( X\) tel que \( \vect{ AX }=\vect{ BC }\) (méthode au choix)
    \end{enumerate}
    
    



    $\vect{ AB }=(-6;-3)$

    $\vect{ AB }+\vect{ BC }=(4;14)$

    $X=(7;13)$
    \item

    Soient les points $M(9;-8)$, $L(8;-2)$ et $B(4;10)$. Donner les coordonnées du point $F$ tel que $MLBF$ soit un parallélogramme (méthode au choix).
    

$F=(5;4)$
\end{enumerate}
}
\vbox{Numéro 54.
\emph{Toutes les réponses doivent être justifiées par un calcul accompagné d'un raisonnement.}
\begin{enumerate}\item
Soient les points $A(3;-9)$, $ B(-6,0)$ et $ C(-7;-1)$. 
    \begin{enumerate}
    \item
    
    Calculer les coordonnées des vecteurs \( \vect{ AB }\) et \( \vect{ AB }+\vect{ BC }\). 

\item
    Donner les coordonnées du point \( X\) tel que \( \vect{ AX }=\vect{ BC }\) (méthode au choix)
    \end{enumerate}
    
    



    $\vect{ AB }=(-9;9)$

    $\vect{ AB }+\vect{ BC }=(-10;8)$

    $X=(2;-10)$
    \item

    Soient les points $B(-10;-4)$, $F(8;-4)$ et $M(-9;-1)$. Donner les coordonnées du point $A$ tel que $BFMA$ soit un parallélogramme (méthode au choix).
    

$A=(-27;-1)$
\end{enumerate}
}
\vbox{Numéro 55.
\emph{Toutes les réponses doivent être justifiées par un calcul accompagné d'un raisonnement.}
\begin{enumerate}\item
Soient les points $A(2;4)$, $ B(2,-5)$ et $ C(-1;-6)$. 
    \begin{enumerate}
    \item
    
    Calculer les coordonnées des vecteurs \( \vect{ AB }\) et \( \vect{ AB }+\vect{ BC }\). 

\item
    Donner les coordonnées du point \( X\) tel que \( \vect{ AX }=\vect{ BC }\) (méthode au choix)
    \end{enumerate}
    
    



    $\vect{ AB }=(0;-9)$

    $\vect{ AB }+\vect{ BC }=(-3;-10)$

    $X=(-1;3)$
    \item

    Soient les points $L(-7;-8)$, $M(-7;9)$ et $K(5;3)$. Donner les coordonnées du point $A$ tel que $LMKA$ soit un parallélogramme (méthode au choix).
    

$A=(5;-14)$
\end{enumerate}
}
\vbox{Numéro 56.
\emph{Toutes les réponses doivent être justifiées par un calcul accompagné d'un raisonnement.}
\begin{enumerate}\item

    Soient les points $A(-2;3)$, $B(2;-2)$ et $E(-9;-8)$. Donner les coordonnées du point $M$ tel que $ABEM$ soit un parallélogramme (méthode au choix).
    

$M=(-13;-3)$\item
Soient les points $A(-6;-1)$, $ B(9,3)$ et $ C(6;-2)$. 
    \begin{enumerate}
    \item
    
    Calculer les coordonnées des vecteurs \( \vect{ AB }\) et \( \vect{ AB }+\vect{ BC }\). 

\item
    Donner les coordonnées du point \( X\) tel que \( \vect{ AX }=\vect{ BC }\) (méthode au choix)
    \end{enumerate}
    
    



    $\vect{ AB }=(15;4)$

    $\vect{ AB }+\vect{ BC }=(12;-1)$

    $X=(-9;-6)$
    
\end{enumerate}
}
\vbox{Numéro 57.
\emph{Toutes les réponses doivent être justifiées par un calcul accompagné d'un raisonnement.}
\begin{enumerate}\item

    Soient les points $M(5;9)$, $K(-4;-2)$ et $D(9;7)$. Donner les coordonnées du point $F$ tel que $MKDF$ soit un parallélogramme (méthode au choix).
    

$F=(18;18)$\item
Soient les points $A(-3;5)$, $ B(2,4)$ et $ C(5;-6)$. 
    \begin{enumerate}
    \item
    
    Calculer les coordonnées des vecteurs \( \vect{ AB }\) et \( \vect{ AB }+\vect{ BC }\). 

\item
    Donner les coordonnées du point \( X\) tel que \( \vect{ AX }=\vect{ BC }\) (méthode au choix)
    \end{enumerate}
    
    



    $\vect{ AB }=(5;-1)$

    $\vect{ AB }+\vect{ BC }=(8;-11)$

    $X=(0;-5)$
    
\end{enumerate}
}
\vbox{Numéro 58.
\emph{Toutes les réponses doivent être justifiées par un calcul accompagné d'un raisonnement.}
\begin{enumerate}\item

    Soient les points $D(2;-7)$, $F(1;-6)$ et $A(0;-2)$. Donner les coordonnées du point $K$ tel que $DFAK$ soit un parallélogramme (méthode au choix).
    

$K=(1;-3)$\item
Soient les points $A(7;1)$, $ B(8,7)$ et $ C(-9;2)$. 
    \begin{enumerate}
    \item
    
    Calculer les coordonnées des vecteurs \( \vect{ AB }\) et \( \vect{ AB }+\vect{ BC }\). 

\item
    Donner les coordonnées du point \( X\) tel que \( \vect{ AX }=\vect{ BC }\) (méthode au choix)
    \end{enumerate}
    
    



    $\vect{ AB }=(1;6)$

    $\vect{ AB }+\vect{ BC }=(-16;1)$

    $X=(-10;-4)$
    
\end{enumerate}
}
\vbox{Numéro 59.
\emph{Toutes les réponses doivent être justifiées par un calcul accompagné d'un raisonnement.}
\begin{enumerate}\item
Soient les points $A(1;9)$, $ B(6,-6)$ et $ C(-3;-7)$. 
    \begin{enumerate}
    \item
    
    Calculer les coordonnées des vecteurs \( \vect{ AB }\) et \( \vect{ AB }+\vect{ BC }\). 

\item
    Donner les coordonnées du point \( X\) tel que \( \vect{ AX }=\vect{ BC }\) (méthode au choix)
    \end{enumerate}
    
    



    $\vect{ AB }=(5;-15)$

    $\vect{ AB }+\vect{ BC }=(-4;-16)$

    $X=(-8;8)$
    \item

    Soient les points $M(-3;9)$, $F(-1;0)$ et $L(-7;-6)$. Donner les coordonnées du point $D$ tel que $MFLD$ soit un parallélogramme (méthode au choix).
    

$D=(-9;3)$
\end{enumerate}
}
\vbox{Numéro 60.
\emph{Toutes les réponses doivent être justifiées par un calcul accompagné d'un raisonnement.}
\begin{enumerate}\item
Soient les points $A(4;5)$, $ B(-9,3)$ et $ C(-10;-1)$. 
    \begin{enumerate}
    \item
    
    Calculer les coordonnées des vecteurs \( \vect{ AB }\) et \( \vect{ AB }+\vect{ BC }\). 

\item
    Donner les coordonnées du point \( X\) tel que \( \vect{ AX }=\vect{ BC }\) (méthode au choix)
    \end{enumerate}
    
    



    $\vect{ AB }=(-13;-2)$

    $\vect{ AB }+\vect{ BC }=(-14;-6)$

    $X=(3;1)$
    \item

    Soient les points $E(1;3)$, $L(-6;3)$ et $F(0;6)$. Donner les coordonnées du point $K$ tel que $ELFK$ soit un parallélogramme (méthode au choix).
    

$K=(7;6)$
\end{enumerate}
}
\vbox{Numéro 61.
\emph{Toutes les réponses doivent être justifiées par un calcul accompagné d'un raisonnement.}
\begin{enumerate}\item

    Soient les points $F(8;1)$, $B(6;7)$ et $K(-6;3)$. Donner les coordonnées du point $D$ tel que $FBKD$ soit un parallélogramme (méthode au choix).
    

$D=(-4;-3)$\item
Soient les points $A(-1;6)$, $ B(-7,7)$ et $ C(-9;-2)$. 
    \begin{enumerate}
    \item
    
    Calculer les coordonnées des vecteurs \( \vect{ AB }\) et \( \vect{ AB }+\vect{ BC }\). 

\item
    Donner les coordonnées du point \( X\) tel que \( \vect{ AX }=\vect{ BC }\) (méthode au choix)
    \end{enumerate}
    
    



    $\vect{ AB }=(-6;1)$

    $\vect{ AB }+\vect{ BC }=(-8;-8)$

    $X=(-3;-3)$
    
\end{enumerate}
}
\vbox{Numéro 62.
\emph{Toutes les réponses doivent être justifiées par un calcul accompagné d'un raisonnement.}
\begin{enumerate}\item
Soient les points $A(-7;5)$, $ B(3,-9)$ et $ C(8;-2)$. 
    \begin{enumerate}
    \item
    
    Calculer les coordonnées des vecteurs \( \vect{ AB }\) et \( \vect{ AB }+\vect{ BC }\). 

\item
    Donner les coordonnées du point \( X\) tel que \( \vect{ AX }=\vect{ BC }\) (méthode au choix)
    \end{enumerate}
    
    



    $\vect{ AB }=(10;-14)$

    $\vect{ AB }+\vect{ BC }=(15;-7)$

    $X=(-2;12)$
    \item

    Soient les points $K(-4;3)$, $L(8;-8)$ et $F(-8;3)$. Donner les coordonnées du point $A$ tel que $KLFA$ soit un parallélogramme (méthode au choix).
    

$A=(-20;14)$
\end{enumerate}
}
\vbox{Numéro 63.
\emph{Toutes les réponses doivent être justifiées par un calcul accompagné d'un raisonnement.}
\begin{enumerate}\item

    Soient les points $B(-5;9)$, $D(2;6)$ et $A(-6;3)$. Donner les coordonnées du point $K$ tel que $BDAK$ soit un parallélogramme (méthode au choix).
    

$K=(-13;6)$\item
Soient les points $A(1;-3)$, $ B(5,2)$ et $ C(7;-4)$. 
    \begin{enumerate}
    \item
    
    Calculer les coordonnées des vecteurs \( \vect{ AB }\) et \( \vect{ AB }+\vect{ BC }\). 

\item
    Donner les coordonnées du point \( X\) tel que \( \vect{ AX }=\vect{ BC }\) (méthode au choix)
    \end{enumerate}
    
    



    $\vect{ AB }=(4;5)$

    $\vect{ AB }+\vect{ BC }=(6;-1)$

    $X=(3;-9)$
    
\end{enumerate}
}
\vbox{Numéro 64.
\emph{Toutes les réponses doivent être justifiées par un calcul accompagné d'un raisonnement.}
\begin{enumerate}\item

    Soient les points $D(-6;3)$, $K(-1;-1)$ et $L(-8;10)$. Donner les coordonnées du point $B$ tel que $DKLB$ soit un parallélogramme (méthode au choix).
    

$B=(-13;14)$\item
Soient les points $A(-4;-4)$, $ B(6,-6)$ et $ C(-10;-4)$. 
    \begin{enumerate}
    \item
    
    Calculer les coordonnées des vecteurs \( \vect{ AB }\) et \( \vect{ AB }+\vect{ BC }\). 

\item
    Donner les coordonnées du point \( X\) tel que \( \vect{ AX }=\vect{ BC }\) (méthode au choix)
    \end{enumerate}
    
    



    $\vect{ AB }=(10;-2)$

    $\vect{ AB }+\vect{ BC }=(-6;0)$

    $X=(-20;-2)$
    
\end{enumerate}
}
\vbox{Numéro 65.
\emph{Toutes les réponses doivent être justifiées par un calcul accompagné d'un raisonnement.}
\begin{enumerate}\item

    Soient les points $B(0;10)$, $K(-4;0)$ et $L(-6;10)$. Donner les coordonnées du point $E$ tel que $BKLE$ soit un parallélogramme (méthode au choix).
    

$E=(-2;20)$\item
Soient les points $A(-2;-9)$, $ B(-6,-9)$ et $ C(-5;0)$. 
    \begin{enumerate}
    \item
    
    Calculer les coordonnées des vecteurs \( \vect{ AB }\) et \( \vect{ AB }+\vect{ BC }\). 

\item
    Donner les coordonnées du point \( X\) tel que \( \vect{ AX }=\vect{ BC }\) (méthode au choix)
    \end{enumerate}
    
    



    $\vect{ AB }=(-4;0)$

    $\vect{ AB }+\vect{ BC }=(-3;9)$

    $X=(-1;0)$
    
\end{enumerate}
}
\vbox{Numéro 66.
\emph{Toutes les réponses doivent être justifiées par un calcul accompagné d'un raisonnement.}
\begin{enumerate}\item
Soient les points $A(-8;-3)$, $ B(-3,0)$ et $ C(4;5)$. 
    \begin{enumerate}
    \item
    
    Calculer les coordonnées des vecteurs \( \vect{ AB }\) et \( \vect{ AB }+\vect{ BC }\). 

\item
    Donner les coordonnées du point \( X\) tel que \( \vect{ AX }=\vect{ BC }\) (méthode au choix)
    \end{enumerate}
    
    



    $\vect{ AB }=(5;3)$

    $\vect{ AB }+\vect{ BC }=(12;8)$

    $X=(-1;2)$
    \item

    Soient les points $F(1;-9)$, $L(4;9)$ et $E(9;9)$. Donner les coordonnées du point $K$ tel que $FLEK$ soit un parallélogramme (méthode au choix).
    

$K=(6;-9)$
\end{enumerate}
}
\vbox{Numéro 67.
\emph{Toutes les réponses doivent être justifiées par un calcul accompagné d'un raisonnement.}
\begin{enumerate}\item
Soient les points $A(6;-6)$, $ B(9,-8)$ et $ C(10;-1)$. 
    \begin{enumerate}
    \item
    
    Calculer les coordonnées des vecteurs \( \vect{ AB }\) et \( \vect{ AB }+\vect{ BC }\). 

\item
    Donner les coordonnées du point \( X\) tel que \( \vect{ AX }=\vect{ BC }\) (méthode au choix)
    \end{enumerate}
    
    



    $\vect{ AB }=(3;-2)$

    $\vect{ AB }+\vect{ BC }=(4;5)$

    $X=(7;1)$
    \item

    Soient les points $F(0;-1)$, $K(9;2)$ et $M(-4;9)$. Donner les coordonnées du point $A$ tel que $FKMA$ soit un parallélogramme (méthode au choix).
    

$A=(-13;6)$
\end{enumerate}
}
\vbox{Numéro 68.
\emph{Toutes les réponses doivent être justifiées par un calcul accompagné d'un raisonnement.}
\begin{enumerate}\item

    Soient les points $E(1;3)$, $K(5;6)$ et $F(-1;10)$. Donner les coordonnées du point $D$ tel que $EKFD$ soit un parallélogramme (méthode au choix).
    

$D=(-5;7)$\item
Soient les points $A(4;2)$, $ B(-10,-9)$ et $ C(-7;-1)$. 
    \begin{enumerate}
    \item
    
    Calculer les coordonnées des vecteurs \( \vect{ AB }\) et \( \vect{ AB }+\vect{ BC }\). 

\item
    Donner les coordonnées du point \( X\) tel que \( \vect{ AX }=\vect{ BC }\) (méthode au choix)
    \end{enumerate}
    
    



    $\vect{ AB }=(-14;-11)$

    $\vect{ AB }+\vect{ BC }=(-11;-3)$

    $X=(7;10)$
    
\end{enumerate}
}
\vbox{Numéro 69.
\emph{Toutes les réponses doivent être justifiées par un calcul accompagné d'un raisonnement.}
\begin{enumerate}\item

    Soient les points $L(-9;-9)$, $K(6;9)$ et $M(6;-6)$. Donner les coordonnées du point $A$ tel que $LKMA$ soit un parallélogramme (méthode au choix).
    

$A=(-9;-24)$\item
Soient les points $A(-5;-7)$, $ B(-9,-5)$ et $ C(-4;-3)$. 
    \begin{enumerate}
    \item
    
    Calculer les coordonnées des vecteurs \( \vect{ AB }\) et \( \vect{ AB }+\vect{ BC }\). 

\item
    Donner les coordonnées du point \( X\) tel que \( \vect{ AX }=\vect{ BC }\) (méthode au choix)
    \end{enumerate}
    
    



    $\vect{ AB }=(-4;2)$

    $\vect{ AB }+\vect{ BC }=(1;4)$

    $X=(0;-5)$
    
\end{enumerate}
}
\vbox{Numéro 70.
\emph{Toutes les réponses doivent être justifiées par un calcul accompagné d'un raisonnement.}
\begin{enumerate}\item

    Soient les points $A(2;9)$, $M(10;-3)$ et $L(5;10)$. Donner les coordonnées du point $D$ tel que $AMLD$ soit un parallélogramme (méthode au choix).
    

$D=(-3;22)$\item
Soient les points $A(0;3)$, $ B(5,-4)$ et $ C(3;-4)$. 
    \begin{enumerate}
    \item
    
    Calculer les coordonnées des vecteurs \( \vect{ AB }\) et \( \vect{ AB }+\vect{ BC }\). 

\item
    Donner les coordonnées du point \( X\) tel que \( \vect{ AX }=\vect{ BC }\) (méthode au choix)
    \end{enumerate}
    
    



    $\vect{ AB }=(5;-7)$

    $\vect{ AB }+\vect{ BC }=(3;-7)$

    $X=(-2;3)$
    
\end{enumerate}
}
\vbox{Numéro 71.
\emph{Toutes les réponses doivent être justifiées par un calcul accompagné d'un raisonnement.}
\begin{enumerate}\item
Soient les points $A(8;3)$, $ B(0,-10)$ et $ C(-10;0)$. 
    \begin{enumerate}
    \item
    
    Calculer les coordonnées des vecteurs \( \vect{ AB }\) et \( \vect{ AB }+\vect{ BC }\). 

\item
    Donner les coordonnées du point \( X\) tel que \( \vect{ AX }=\vect{ BC }\) (méthode au choix)
    \end{enumerate}
    
    



    $\vect{ AB }=(-8;-13)$

    $\vect{ AB }+\vect{ BC }=(-18;-3)$

    $X=(-2;13)$
    \item

    Soient les points $K(7;-2)$, $B(-2;1)$ et $D(6;5)$. Donner les coordonnées du point $E$ tel que $KBDE$ soit un parallélogramme (méthode au choix).
    

$E=(15;2)$
\end{enumerate}
}
\vbox{Numéro 72.
\emph{Toutes les réponses doivent être justifiées par un calcul accompagné d'un raisonnement.}
\begin{enumerate}\item

    Soient les points $K(1;-7)$, $E(6;8)$ et $D(3;-4)$. Donner les coordonnées du point $M$ tel que $KEDM$ soit un parallélogramme (méthode au choix).
    

$M=(-2;-19)$\item
Soient les points $A(0;-5)$, $ B(-5,4)$ et $ C(6;7)$. 
    \begin{enumerate}
    \item
    
    Calculer les coordonnées des vecteurs \( \vect{ AB }\) et \( \vect{ AB }+\vect{ BC }\). 

\item
    Donner les coordonnées du point \( X\) tel que \( \vect{ AX }=\vect{ BC }\) (méthode au choix)
    \end{enumerate}
    
    



    $\vect{ AB }=(-5;9)$

    $\vect{ AB }+\vect{ BC }=(6;12)$

    $X=(11;-2)$
    
\end{enumerate}
}
\vbox{Numéro 73.
\emph{Toutes les réponses doivent être justifiées par un calcul accompagné d'un raisonnement.}
\begin{enumerate}\item

    Soient les points $M(1;-3)$, $K(-5;9)$ et $B(10;-2)$. Donner les coordonnées du point $A$ tel que $MKBA$ soit un parallélogramme (méthode au choix).
    

$A=(16;-14)$\item
Soient les points $A(1;7)$, $ B(6,7)$ et $ C(2;-7)$. 
    \begin{enumerate}
    \item
    
    Calculer les coordonnées des vecteurs \( \vect{ AB }\) et \( \vect{ AB }+\vect{ BC }\). 

\item
    Donner les coordonnées du point \( X\) tel que \( \vect{ AX }=\vect{ BC }\) (méthode au choix)
    \end{enumerate}
    
    



    $\vect{ AB }=(5;0)$

    $\vect{ AB }+\vect{ BC }=(1;-14)$

    $X=(-3;-7)$
    
\end{enumerate}
}
\vbox{Numéro 74.
\emph{Toutes les réponses doivent être justifiées par un calcul accompagné d'un raisonnement.}
\begin{enumerate}\item
Soient les points $A(-4;-2)$, $ B(10,-3)$ et $ C(3;-9)$. 
    \begin{enumerate}
    \item
    
    Calculer les coordonnées des vecteurs \( \vect{ AB }\) et \( \vect{ AB }+\vect{ BC }\). 

\item
    Donner les coordonnées du point \( X\) tel que \( \vect{ AX }=\vect{ BC }\) (méthode au choix)
    \end{enumerate}
    
    



    $\vect{ AB }=(14;-1)$

    $\vect{ AB }+\vect{ BC }=(7;-7)$

    $X=(-11;-8)$
    \item

    Soient les points $F(-10;-3)$, $L(0;3)$ et $K(-2;3)$. Donner les coordonnées du point $M$ tel que $FLKM$ soit un parallélogramme (méthode au choix).
    

$M=(-12;-3)$
\end{enumerate}
}


\end{document}



\end{document}


\begin{feuilleExo}{Feuille de secours de math\\Février 2014}
    \begin{multicols}{2}
        \Exo{smath-0637}
        \Exo{smath-0638}
\Exo{smath-0639}
\Exo{smath-0640}
\Exo{smath-0643}
\Exo{smath-0641}
\Exo{smath-0642}
\Exo{smath-0644}
\Exo{smath-0645}
\Exo{smath-0646}
\Exo{smath-0647}
    \end{multicols}
\end{feuilleExo}

\end{document}

%DS NUMÉRO 5 MARDI 4 FÉVRIER 2014, 2A
\begin{feuilleDS}{Seconde A, DS numéro 5\\ \small mardi 4 février 2014}
\Exo{smath-0634}
\Exo{smath-0633}
\Exo{smath-0635}
\Exo{smath-0636}
\end{feuilleDS}

%DS NUMÉRO 5 MARDI 4 FÉVRIER 2014, 2D
\begin{feuilleDS}{Seconde D, DS numéro 5\\ \small mardi 4 février 2014}
\Exo{smath-0630}
    \Exo{smath-0597}
\Exo{smath-0631}
\Exo{smath-0633}
\end{feuilleDS}

\end{document}

\begin{feuilleExo}{Pour aller en salle info}
    \begin{multicols}{2}
%    \Exo{smath-0539}    % info
    \Exo{smath-0540}    % info
    \Exo{smath-0589}    % info
    \Exo{smath-0583}    % info
    \Exo{smath-0581}    % info
    \Exo{smath-0580}    % info
    \Exo{smath-0579}    % info
    \Exo{smath-0586}    % info
    \end{multicols}
\end{feuilleExo}

%FEUILLES D'ALGORITHMIQUE
\begin{feuilleExo}{Algorithmique}
    \begin{multicols}{2}
        \Exo{smath-0526}    % algo
        \Exo{smath-0555}    % algo
        \Exo{smath-0584}
        \Exo{smath-0578}    % algo, DS des autres
        \Exo{smath-0585}
        \Exo{smath-0551}    % algo De DS des autres
        \Exo{smath-0577}
        \Exo{smath-0587}
    \end{multicols}
\end{feuilleExo}


\end{document}

\newlength{\vertical}
\setlength{\vertical}{1cm}

% This is part of Un soupçon de mathématique sans être agressif pour autant
% Copyright (c) 2013
%   Laurent Claessens
% See the file fdl-1.3.txt for copying conditions.

\begin{multicols}{2}
    \begin{enumerate}
        \item
Le TER 894258 a pour horaire :
\begin{description}
    \item[14h56] Besançon-Viotte
    \item[15h06] St-Vit
    \item[15h21] Dole-Ville
    \item[15h30] Auxonne
    \item[15h39] Genlis
    \item[15h51] Dijon-Ville
\end{description}
Dessiner le trajet sur une ligne du temps en indiquant les durées entre les stations.
        \item
            En supposant les mêmes temps de parcours, quelles sont les heures d'arrivées dans les différentes gares du TER partant à 20h23 pour le même trajet ?
        \item
             Les points \( A(1;2)\), \( B(3;3)\), \( C(2;5)\) et \( D(0;4)\) forment un carré. Donner les coordonnées du «même» carré \( A'B'C'D'\) partant de \( A'(4;0)\).
         \item
             Soient les points \( K(0;0)\), \( L(4;1)\) et \( M(3;3)\). Donner les coordonnées du point \( N\) tel que \( KLMN\) soit un parallélogramme.
    \end{enumerate}
\end{multicols}

\vspace{\vertical}

% This is part of Un soupçon de mathématique sans être agressif pour autant
% Copyright (c) 2013
%   Laurent Claessens
% See the file fdl-1.3.txt for copying conditions.

\begin{multicols}{2}
    \begin{enumerate}
        \item
Le TER 894258 a pour horaire :
\begin{description}
    \item[14h56] Besançon-Viotte
    \item[15h06] St-Vit
    \item[15h21] Dole-Ville
    \item[15h30] Auxonne
    \item[15h39] Genlis
    \item[15h51] Dijon-Ville
\end{description}
Dessiner le trajet sur une ligne du temps en indiquant les durées entre les stations.
        \item
            En supposant les mêmes temps de parcours, quelles sont les heures d'arrivées dans les différentes gares du TER partant à 20h23 pour le même trajet ?
        \item
             Les points \( A(1;2)\), \( B(3;3)\), \( C(2;5)\) et \( D(0;4)\) forment un carré. Donner les coordonnées du «même» carré \( A'B'C'D'\) partant de \( A'(4;0)\).
         \item
             Soient les points \( K(0;0)\), \( L(4;1)\) et \( M(3;3)\). Donner les coordonnées du point \( N\) tel que \( KLMN\) soit un parallélogramme.
    \end{enumerate}
\end{multicols}

\vspace{\vertical}

% This is part of Un soupçon de mathématique sans être agressif pour autant
% Copyright (c) 2013
%   Laurent Claessens
% See the file fdl-1.3.txt for copying conditions.

\begin{multicols}{2}
    \begin{enumerate}
        \item
Le TER 894258 a pour horaire :
\begin{description}
    \item[14h56] Besançon-Viotte
    \item[15h06] St-Vit
    \item[15h21] Dole-Ville
    \item[15h30] Auxonne
    \item[15h39] Genlis
    \item[15h51] Dijon-Ville
\end{description}
Dessiner le trajet sur une ligne du temps en indiquant les durées entre les stations.
        \item
            En supposant les mêmes temps de parcours, quelles sont les heures d'arrivées dans les différentes gares du TER partant à 20h23 pour le même trajet ?
        \item
             Les points \( A(1;2)\), \( B(3;3)\), \( C(2;5)\) et \( D(0;4)\) forment un carré. Donner les coordonnées du «même» carré \( A'B'C'D'\) partant de \( A'(4;0)\).
         \item
             Soient les points \( K(0;0)\), \( L(4;1)\) et \( M(3;3)\). Donner les coordonnées du point \( N\) tel que \( KLMN\) soit un parallélogramme.
    \end{enumerate}
\end{multicols}

\vspace{\vertical}

% This is part of Un soupçon de mathématique sans être agressif pour autant
% Copyright (c) 2013
%   Laurent Claessens
% See the file fdl-1.3.txt for copying conditions.

\begin{multicols}{2}
    \begin{enumerate}
        \item
Le TER 894258 a pour horaire :
\begin{description}
    \item[14h56] Besançon-Viotte
    \item[15h06] St-Vit
    \item[15h21] Dole-Ville
    \item[15h30] Auxonne
    \item[15h39] Genlis
    \item[15h51] Dijon-Ville
\end{description}
Dessiner le trajet sur une ligne du temps en indiquant les durées entre les stations.
        \item
            En supposant les mêmes temps de parcours, quelles sont les heures d'arrivées dans les différentes gares du TER partant à 20h23 pour le même trajet ?
        \item
             Les points \( A(1;2)\), \( B(3;3)\), \( C(2;5)\) et \( D(0;4)\) forment un carré. Donner les coordonnées du «même» carré \( A'B'C'D'\) partant de \( A'(4;0)\).
         \item
             Soient les points \( K(0;0)\), \( L(4;1)\) et \( M(3;3)\). Donner les coordonnées du point \( N\) tel que \( KLMN\) soit un parallélogramme.
    \end{enumerate}
\end{multicols}

\vspace{\vertical}

\end{document}

%DS NUMÉRO 4 MARDI 14 JANVIER 2014, 2A
\begin{feuilleDS}{Seconde A, DS numéro 4\\ \small mardi 14 janvier 2014}
\Exo{smath-0602}
    \Exo{smath-0597}
    \Exo{smath-0598}
    \Exo{smath-0599}
\end{feuilleDS}

%DS NUMÉRO 4 MARDI 14 JANVIER 2014, 2D
\begin{feuilleDS}{Seconde D, DS numéro 4\\ \small mardi 14 janvier 2014}
\Exo{smath-0602}
    \Exo{smath-0600}
    \Exo{smath-0601}
    \Exo{smath-0599}
\end{feuilleDS}

\end{document}

% FEUILLE POUR ALLER EN SALLE INFORMATIQUE.
\begin{feuilleExo}{Pour aller en salle info}
    \begin{multicols}{2}
%    \Exo{smath-0539}    % info
    \Exo{smath-0540}    % info
    \Exo{smath-0589}    % info
    \Exo{smath-0583}    % info
    \Exo{smath-0581}    % info
    \Exo{smath-0580}    % info
    \Exo{smath-0579}    % info
    \Exo{smath-0586}    % info
    \end{multicols}
\end{feuilleExo}


%L'INTERRO DE GÉOMÉTRIE DANS LESPACE
\input{interro_geometrie_espace}

% ACTIVITÉ GÉOMÉTRIE DANS L'ESPACE
% This is part of Un soupçon de mathématique sans être agressif pour autant
% Copyright (c) 2013
%   Laurent Claessens
% See the file fdl-1.3.txt for copying conditions.

La classe est divisée en groupes, chacun contenant des perles turquoises et des perles blanches. Les sacs sont identiques mais inconnus. Le but de l'activité est de déterminer ce contenu sans compter toutes les perles. Voici le protocole :
\begin{itemize}
    \item Tirer une perle au hasard dans le sac,
    \item Noter sa couleur.
    \item La remettre dans le sac.
\end{itemize}
Nous parlons de tirage \defe{avec remise}{avec remise!tirage}.

%--------------------------------------------------------------------------------------------------------------------------- 
\subsection*{Après 10 tirages}
%---------------------------------------------------------------------------------------------------------------------------

Reporter les résultats de vos tirages en notant T ou B dans le tableau suivant :

\begin{tabular}[]{|c|c|c|c|c|c|c|c|c|}
    \hline
    &&&&&&&&\\
    \hline
\end{tabular}

Ce tirage forme un \defe{échantillon}{échantillon} du contenu du sac.
\begin{enumerate}
    \item
        Quelle est la fréquence d'apparition de la couleur turquoise ?
     \item
        Reporter les fréquences des autres groupes :
        \begin{tabular}[]{|c|c|c|c|c|c|c|c|c|c|}
        \hline
        &&&&&&&&\\
        \hline
    \end{tabular}
\item
    Que constate-t-on ?
\end{enumerate}

%--------------------------------------------------------------------------------------------------------------------------- 
\subsection*{Après 50 tirages}
%---------------------------------------------------------------------------------------------------------------------------

Effectuer encore \( 40\) tirages.
\begin{enumerate}
    \item
        Calculer la fréquence d'apparition de la couleur turquoise.
    \item
        Compléter le tableau de fréquences d'apparition du turquoise suivant :
        \begin{equation*}
            \begin{array}[]{|c|c|c|c|}
                \hline
                &\text{échantillon de taille 10}&\text{échantillon de taille 40}&\text{Échantillon de taille 50}\\
                  \hline
                  \text{Fréq.}&&&\\ 
                  \hline 
                   \end{array}
               \end{equation*}
               
\end{enumerate}

%--------------------------------------------------------------------------------------------------------------------------- 
\subsection*{Après 100 tirages}
%---------------------------------------------------------------------------------------------------------------------------

Effectuer \( 50\) nouveaux tirages et reporter les résultats :

\begin{equation*}
    \begin{array}[]{|c|c|c|c|}
      \hline
        &\text{Taille 10}&\text{Taille 50}&\text{Taille 100}\\
       \hline
       \text{Fréq.}&&&\\ 
       \hline 
  \end{array}
\end{equation*}
               
Écrire les résultats des autres groupes :
        \begin{tabular}[]{|c|c|c|c|c|c|c|c|c|c|}
        \hline
        &&&&&&&&\\
        \hline
    \end{tabular}

Combien de tirages ont été faits en tout dans la classe ? Quelle est la fréquence observée des boules turquoises ?


\end{document}

% FEUILLE D'AP pour les 2DE -- logique
\begin{feuilleExo}{Feuille d'AP de seconde, logique}
\Exo{smath-0560}
\Exo{smath-0561}
\Exo{smath-0562}
\Exo{smath-0563}
\Exo{smath-0564}    % document d'accompagnement de fonctions.
\Exo{smath-0433}
\Exo{smath-0471}
\end{feuilleExo}

\end{document}

%DS MARDI 10 DÉCEMBRE 2013, 2A
\begin{feuilleDS}{Seconde A, DS numéro 3\\ \small mardi 10 décembre 2013}
\Exo{smath-0570}
\Exo{smath-0571}
\Exo{smath-0572}
\Exo{smath-0574}
\Exo{smath-0575}
\end{feuilleDS}

%DS MARDI 10 DÉCEMBRE 2013, 2D
\begin{feuilleDS}{Seconde D, DS numéro 3\\ \small mardi 10 décembre 2013}

\Exo{smath-0566}

\vspace{1.5cm}

\Exo{smath-0567}
\Exo{smath-0568}
\Exo{smath-0574}
\Exo{smath-0569}

\end{feuilleDS}


\end{document}


%INTERRO STATISTIQUE DESCRIPTIVE
\input{interro_statistique_descriptive_sujet.tex}
\end{document}

% ACTIVITÉ GÉOMÉTRIE DANS L'ESPACE
% This is part of Un soupçon de mathématique sans être agressif pour autant
% Copyright (c) 2013
%   Laurent Claessens
% See the file fdl-1.3.txt for copying conditions.

\begin{wrapfigure}[3]{r}{5.0cm}
   \vspace{-0.5cm}        % à adapter.
   \centering
   \input{Fig_MEzTDZC.pstricks}
\end{wrapfigure}

À l'aide du cube ci-contre :
\begin{enumerate}
    \item
        Quelle est la nature du triangle \( AEF\) ? 
    \item
        Quelle est la nature du quadrilatère \( ABGH\) ?
    \item
        Est-ce que vous pouvez trouver trois points de ce cube formant un triangle équilatéral ?
    \item
        Si ce cube fait \unit{3}{\meter} de côté, quelle est la longueur de la «grande» diagonale \( [AG]\) ?
\end{enumerate}



\vspace{2cm}

% This is part of Un soupçon de mathématique sans être agressif pour autant
% Copyright (c) 2013
%   Laurent Claessens
% See the file fdl-1.3.txt for copying conditions.

\begin{wrapfigure}[3]{r}{5.0cm}
   \vspace{-0.5cm}        % à adapter.
   \centering
   \input{Fig_MEzTDZC.pstricks}
\end{wrapfigure}

À l'aide du cube ci-contre :
\begin{enumerate}
    \item
        Quelle est la nature du triangle \( AEF\) ? 
    \item
        Quelle est la nature du quadrilatère \( ABGH\) ?
    \item
        Est-ce que vous pouvez trouver trois points de ce cube formant un triangle équilatéral ?
    \item
        Si ce cube fait \unit{3}{\meter} de côté, quelle est la longueur de la «grande» diagonale \( [AG]\) ?
\end{enumerate}



\vspace{2cm}

% This is part of Un soupçon de mathématique sans être agressif pour autant
% Copyright (c) 2013
%   Laurent Claessens
% See the file fdl-1.3.txt for copying conditions.

\begin{wrapfigure}[3]{r}{5.0cm}
   \vspace{-0.5cm}        % à adapter.
   \centering
   \input{Fig_MEzTDZC.pstricks}
\end{wrapfigure}

À l'aide du cube ci-contre :
\begin{enumerate}
    \item
        Quelle est la nature du triangle \( AEF\) ? 
    \item
        Quelle est la nature du quadrilatère \( ABGH\) ?
    \item
        Est-ce que vous pouvez trouver trois points de ce cube formant un triangle équilatéral ?
    \item
        Si ce cube fait \unit{3}{\meter} de côté, quelle est la longueur de la «grande» diagonale \( [AG]\) ?
\end{enumerate}




\vspace{2cm}

% This is part of Un soupçon de mathématique sans être agressif pour autant
% Copyright (c) 2013
%   Laurent Claessens
% See the file fdl-1.3.txt for copying conditions.

\begin{wrapfigure}[3]{r}{5.0cm}
   \vspace{-0.5cm}        % à adapter.
   \centering
   \input{Fig_MEzTDZC.pstricks}
\end{wrapfigure}

À l'aide du cube ci-contre :
\begin{enumerate}
    \item
        Quelle est la nature du triangle \( AEF\) ? 
    \item
        Quelle est la nature du quadrilatère \( ABGH\) ?
    \item
        Est-ce que vous pouvez trouver trois points de ce cube formant un triangle équilatéral ?
    \item
        Si ce cube fait \unit{3}{\meter} de côté, quelle est la longueur de la «grande» diagonale \( [AG]\) ?
\end{enumerate}



\vspace{2cm}

% This is part of Un soupçon de mathématique sans être agressif pour autant
% Copyright (c) 2013
%   Laurent Claessens
% See the file fdl-1.3.txt for copying conditions.

\begin{wrapfigure}[3]{r}{5.0cm}
   \vspace{-0.5cm}        % à adapter.
   \centering
   \input{Fig_MEzTDZC.pstricks}
\end{wrapfigure}

À l'aide du cube ci-contre :
\begin{enumerate}
    \item
        Quelle est la nature du triangle \( AEF\) ? 
    \item
        Quelle est la nature du quadrilatère \( ABGH\) ?
    \item
        Est-ce que vous pouvez trouver trois points de ce cube formant un triangle équilatéral ?
    \item
        Si ce cube fait \unit{3}{\meter} de côté, quelle est la longueur de la «grande» diagonale \( [AG]\) ?
\end{enumerate}




\end{document}
% LES FIGURES DE GÉOMÉTRIE DANS L'ESPACE

\renewcommand{\thesection}{\arabic{section}}


%+++++++++++++++++++++++++++++++++++++++++++++++++++++++++++++++++++++++++++++++++++++++++++++++++++++++++++++++++++++++++++ 
\section{Position relatives}
%+++++++++++++++++++++++++++++++++++++++++++++++++++++++++++++++++++++++++++++++++++++++++++++++++++++++++++++++++++++++++++

%--------------------------------------------------------------------------------------------------------------------------- 
\subsection{Positions relatives de deux plans}
%---------------------------------------------------------------------------------------------------------------------------

\begin{definition}
    Deux plans sont \defe{parallèles}{parallèle!deux plans} soit si ils sont confondus, soit si ils n'ont aucun point commun. Si ils n'ont aucun point communs, nous disons qu'ils sont \defe{strictement parallèles}{parallèle!strictement}
\end{definition}

\begin{Aretenir}
    Deux plans non parallèles se coupent en une droite.
\end{Aretenir}

\begin{multicols}{2}

    \begin{center}
        Plans parallèles
    \end{center}
    
\columnbreak

    \begin{center}
\input{Fig_PositionPlansTvKvah.pstricks}
    \end{center}


\end{multicols}

\begin{multicols}{2}
    \begin{center}
        Plans sécants
    \end{center}

    \columnbreak

    \begin{center}
\input{Fig_PositionPlansqSltxa.pstricks}
    \end{center}

\end{multicols}

%---------------------------------------------------------------------------------------------------------------------------
\subsection{Position relative de deux droites}
%---------------------------------------------------------------------------------------------------------------------------

Deux droites peuvent être soit dans un même plan, soit ne pas être dans le même plan. Deux droites contenues dans un même plan sont dires \defe{coplanaires}{coplanaire}.

%///////////////////////////////////////////////////////////////////////////////////////////////////////////////////////////
\subsubsection{Droites coplanaires}
%///////////////////////////////////////////////////////////////////////////////////////////////////////////////////////////

Deux droites coplanaires respectent la géométrie usuelle. Elles peuvent être parallèles ou sécantes.

\vbox{
\begin{multicols}{2}
    \begin{center}
        Droites parallèles dans le plan \( (EBC)\)
    \end{center}

    \columnbreak

    \begin{center}
        \input{Fig_IDqyzXM.pstricks}
    \end{center}
\end{multicols}
}

\vbox{
\begin{multicols}{2}
    \begin{center}
        Droites sécantes dans le plan \( (AEH)\)
    \end{center}

    \columnbreak

    \begin{center}
        \input{Fig_ETfnbsh.pstricks}
    \end{center}
\end{multicols}
}

\begin{Aretenir}
    À propos de droites coplanaires :
    \begin{enumerate}
        \item
            Deux droites sécantes sont toujours coplanaires.
        \item
            Deux droites parallèles sont coplanaires.
    \end{enumerate}
\end{Aretenir}

%///////////////////////////////////////////////////////////////////////////////////////////////////////////////////////////
    \subsubsection{Droites non coplanaires}
%///////////////////////////////////////////////////////////////////////////////////////////////////////////////////////////
    
Deux droites non coplanaires ne peuvent pas être sécantes, ni parallèles.

\vbox{
\begin{multicols}{2}
    \begin{center}
        Il est possible pour deux droites dans l'espace d'être ni sécantes ni parallèles.
    \end{center}

    \columnbreak

    \begin{center}
        \input{Fig_ENQhxmG.pstricks}
    \end{center}
\end{multicols}
}

%---------------------------------------------------------------------------------------------------------------------------
\subsection{Position relative d'une droite et un plan}
%---------------------------------------------------------------------------------------------------------------------------

\begin{definition}
    Une droite est \defe{parallèle}{parallèle!droite et plan} à un plan lorsque soit la droite est contenue dans le plan, soit elle n'a aucun point commun avec le plan.
\end{definition}

%///////////////////////////////////////////////////////////////////////////////////////////////////////////////////////////
\subsubsection{Droite et plan sécants}
%///////////////////////////////////////////////////////////////////////////////////////////////////////////////////////////

\begin{multicols}{2}

    La droite \( (DB)\) intersecte le plan \( (AEF)\).

    \columnbreak
    \begin{center}
\input{Fig_figureBCtCTZo.pstricks}
    \end{center}
\end{multicols}

%///////////////////////////////////////////////////////////////////////////////////////////////////////////////////////////
\subsubsection{Droite et plan parallèles}
%///////////////////////////////////////////////////////////////////////////////////////////////////////////////////////////

\begin{multicols}{2}

    La droite \( (HC)\) et le plan \( (EBF)\) sont parallèles.

    \columnbreak
    \begin{center}
%The result is on figure \ref{LabelFigfigureASkECWS}. % From file figureASkECWS
%\newcommand{\CaptionFigfigureASkECWS}{<+Type your caption here+>}
\input{Fig_figureASkECWS.pstricks}
    \end{center}
\end{multicols}

%///////////////////////////////////////////////////////////////////////////////////////////////////////////////////////////
\subsubsection{Droite contenue dans un plan}
%///////////////////////////////////////////////////////////////////////////////////////////////////////////////////////////

\vbox{
\begin{multicols}{2}

    La droite \( (EB)\) est contenue dans le plan \( (AEF)\).

    \columnbreak
    \begin{center}
\input{Fig_figureCSIQETx.pstricks}
    \end{center}
\end{multicols}
}

\vspace{2cm}
\begin{center}
    Lisez
    
    \url{http://student.ulb.ac.be/~lclaesse/smath.pdf}
\end{center}


\end{document}

% ACTIVITÉ STATISTIQUE DESCRIPTIVE POUR LES 2D


% This is part of Un soupçon de mathématique sans être agressif pour autant
% Copyright (c) 2013
%   Laurent Claessens
% See the file fdl-1.3.txt for copying conditions.

% ATTENTION : les chiffres donnés ici sont repris dans le cours au moment des ECC.

\begin{wrapfigure}[5]{r}{8cm}
   \vspace{-0.5cm}        % à adapter.
   \centering
   \input{Fig_YRQOoPE.pstricks}
\end{wrapfigure}

Le graphique ci-contre illustre le nombre de spam reçus aujourd'hui par les élèves d'une classe.
\begin{enumerate}
    \item
        Combien d'élèves y-a-t-il dans la classe ?
    \item
        Combien d'élèves ont reçu \( 5\) spams ou plus ?
    \item
        En moyenne combien de spam ont reçu les élèves aujourd'hui ?
    \item
        Diviser la classe en 4 groupes suivant le nombre de spams reçus.
\end{enumerate}



\vspace{5cm}

% This is part of Un soupçon de mathématique sans être agressif pour autant
% Copyright (c) 2013
%   Laurent Claessens
% See the file fdl-1.3.txt for copying conditions.

% ATTENTION : les chiffres donnés ici sont repris dans le cours au moment des ECC.

\begin{wrapfigure}[5]{r}{8cm}
   \vspace{-0.5cm}        % à adapter.
   \centering
   \input{Fig_YRQOoPE.pstricks}
\end{wrapfigure}

Le graphique ci-contre illustre le nombre de spam reçus aujourd'hui par les élèves d'une classe.
\begin{enumerate}
    \item
        Combien d'élèves y-a-t-il dans la classe ?
    \item
        Combien d'élèves ont reçu \( 5\) spams ou plus ?
    \item
        En moyenne combien de spam ont reçu les élèves aujourd'hui ?
    \item
        Diviser la classe en 4 groupes suivant le nombre de spams reçus.
\end{enumerate}



\vspace{5cm}

% This is part of Un soupçon de mathématique sans être agressif pour autant
% Copyright (c) 2013
%   Laurent Claessens
% See the file fdl-1.3.txt for copying conditions.

% ATTENTION : les chiffres donnés ici sont repris dans le cours au moment des ECC.

\begin{wrapfigure}[5]{r}{8cm}
   \vspace{-0.5cm}        % à adapter.
   \centering
   \input{Fig_YRQOoPE.pstricks}
\end{wrapfigure}

Le graphique ci-contre illustre le nombre de spam reçus aujourd'hui par les élèves d'une classe.
\begin{enumerate}
    \item
        Combien d'élèves y-a-t-il dans la classe ?
    \item
        Combien d'élèves ont reçu \( 5\) spams ou plus ?
    \item
        En moyenne combien de spam ont reçu les élèves aujourd'hui ?
    \item
        Diviser la classe en 4 groupes suivant le nombre de spams reçus.
\end{enumerate}



\end{document}


%INTERRO GÉOMÉTRIE DANS L'ESPACE
\vbox{1
\emph{Toutes les réponses doivent être justifiées par un calcul accompagné d'un raisonnement.}

\begin{wrapfigure}{r}{5.0cm}
   \vspace{-0.5cm}        % à adapter.
   \centering
   \input{Fig_OKTXHoc.pstricks}
\end{wrapfigure}

    La figure ci-contre est un cube de \unit{2}{\centi\meter}.

    \begin{enumerate}
    \item
    Quelle est la nature du triangle \(EGC\) ? 
    \item
    Quel est son périmètre ?
    \item
    Quelle est son aire ? (pour les rapides)
    \end{enumerate}
    

}
\vspace{2cm}
\vbox{2
\emph{Toutes les réponses doivent être justifiées par un calcul accompagné d'un raisonnement.}

\begin{wrapfigure}{r}{5.0cm}
   \vspace{-0.5cm}        % à adapter.
   \centering
   \input{Fig_OKTXHoc.pstricks}
\end{wrapfigure}

    La figure ci-contre est un cube de \unit{5}{\centi\meter}.

    \begin{enumerate}
    \item
    Quelle est la nature du triangle \(HEB\) ? 
    \item
    Quel est son périmètre ?
    \item
    Quelle est son aire ? (pour les rapides)
    \end{enumerate}
    

}
\vspace{2cm}
\vbox{3
\emph{Toutes les réponses doivent être justifiées par un calcul accompagné d'un raisonnement.}

\begin{wrapfigure}{r}{5.0cm}
   \vspace{-0.5cm}        % à adapter.
   \centering
   \input{Fig_OKTXHoc.pstricks}
\end{wrapfigure}

    La figure ci-contre est un cube de \unit{5}{\centi\meter}.

    \begin{enumerate}
    \item
    Quelle est la nature du triangle \(FBH\) ? 
    \item
    Quel est son périmètre ?
    \item
    Quelle est son aire ? (pour les rapides)
    \end{enumerate}
    

}
\vspace{2cm}
\vbox{4
\emph{Toutes les réponses doivent être justifiées par un calcul accompagné d'un raisonnement.}

\begin{wrapfigure}{r}{5.0cm}
   \vspace{-0.5cm}        % à adapter.
   \centering
   \input{Fig_OKTXHoc.pstricks}
\end{wrapfigure}

    La figure ci-contre est un cube de \unit{6}{\centi\meter}.

    \begin{enumerate}
    \item
    Quelle est la nature du triangle \(BGA\) ? 
    \item
    Quel est son périmètre ?
    \item
    Quelle est son aire ? (pour les rapides)
    \end{enumerate}
    

}
\vspace{2cm}
\vbox{5
\emph{Toutes les réponses doivent être justifiées par un calcul accompagné d'un raisonnement.}

\begin{wrapfigure}{r}{5.0cm}
   \vspace{-0.5cm}        % à adapter.
   \centering
   \input{Fig_OKTXHoc.pstricks}
\end{wrapfigure}

    La figure ci-contre est un cube de \unit{4}{\centi\meter}.

    \begin{enumerate}
    \item
    Quelle est la nature du triangle \(GFC\) ? 
    \item
    Quel est son périmètre ?
    \item
    Quelle est son aire ? (pour les rapides)
    \end{enumerate}
    

}
\vspace{2cm}
\vbox{6
\emph{Toutes les réponses doivent être justifiées par un calcul accompagné d'un raisonnement.}

\begin{wrapfigure}{r}{5.0cm}
   \vspace{-0.5cm}        % à adapter.
   \centering
   \input{Fig_OKTXHoc.pstricks}
\end{wrapfigure}

    La figure ci-contre est un cube de \unit{3}{\centi\meter}.

    \begin{enumerate}
    \item
    Quelle est la nature du triangle \(CAD\) ? 
    \item
    Quel est son périmètre ?
    \item
    Quelle est son aire ? (pour les rapides)
    \end{enumerate}
    

}
\vspace{2cm}
\vbox{7
\emph{Toutes les réponses doivent être justifiées par un calcul accompagné d'un raisonnement.}

\begin{wrapfigure}{r}{5.0cm}
   \vspace{-0.5cm}        % à adapter.
   \centering
   \input{Fig_OKTXHoc.pstricks}
\end{wrapfigure}

    La figure ci-contre est un cube de \unit{4}{\centi\meter}.

    \begin{enumerate}
    \item
    Quelle est la nature du triangle \(EAH\) ? 
    \item
    Quel est son périmètre ?
    \item
    Quelle est son aire ? (pour les rapides)
    \end{enumerate}
    

}
\vspace{2cm}
\vbox{8
\emph{Toutes les réponses doivent être justifiées par un calcul accompagné d'un raisonnement.}

\begin{wrapfigure}{r}{5.0cm}
   \vspace{-0.5cm}        % à adapter.
   \centering
   \input{Fig_OKTXHoc.pstricks}
\end{wrapfigure}

    La figure ci-contre est un cube de \unit{7}{\centi\meter}.

    \begin{enumerate}
    \item
    Quelle est la nature du triangle \(FBD\) ? 
    \item
    Quel est son périmètre ?
    \item
    Quelle est son aire ? (pour les rapides)
    \end{enumerate}
    

}
\vspace{2cm}
\vbox{9
\emph{Toutes les réponses doivent être justifiées par un calcul accompagné d'un raisonnement.}

\begin{wrapfigure}{r}{5.0cm}
   \vspace{-0.5cm}        % à adapter.
   \centering
   \input{Fig_OKTXHoc.pstricks}
\end{wrapfigure}

    La figure ci-contre est un cube de \unit{7}{\centi\meter}.

    \begin{enumerate}
    \item
    Quelle est la nature du triangle \(ACG\) ? 
    \item
    Quel est son périmètre ?
    \item
    Quelle est son aire ? (pour les rapides)
    \end{enumerate}
    

}
\vspace{2cm}
\vbox{10
\emph{Toutes les réponses doivent être justifiées par un calcul accompagné d'un raisonnement.}

\begin{wrapfigure}{r}{5.0cm}
   \vspace{-0.5cm}        % à adapter.
   \centering
   \input{Fig_OKTXHoc.pstricks}
\end{wrapfigure}

    La figure ci-contre est un cube de \unit{4}{\centi\meter}.

    \begin{enumerate}
    \item
    Quelle est la nature du triangle \(ECB\) ? 
    \item
    Quel est son périmètre ?
    \item
    Quelle est son aire ? (pour les rapides)
    \end{enumerate}
    

}
\vspace{2cm}
\vbox{11
\emph{Toutes les réponses doivent être justifiées par un calcul accompagné d'un raisonnement.}

\begin{wrapfigure}{r}{5.0cm}
   \vspace{-0.5cm}        % à adapter.
   \centering
   \input{Fig_OKTXHoc.pstricks}
\end{wrapfigure}

    La figure ci-contre est un cube de \unit{4}{\centi\meter}.

    \begin{enumerate}
    \item
    Quelle est la nature du triangle \(BFD\) ? 
    \item
    Quel est son périmètre ?
    \item
    Quelle est son aire ? (pour les rapides)
    \end{enumerate}
    

}
\vspace{2cm}
\vbox{12
\emph{Toutes les réponses doivent être justifiées par un calcul accompagné d'un raisonnement.}

\begin{wrapfigure}{r}{5.0cm}
   \vspace{-0.5cm}        % à adapter.
   \centering
   \input{Fig_OKTXHoc.pstricks}
\end{wrapfigure}

    La figure ci-contre est un cube de \unit{3}{\centi\meter}.

    \begin{enumerate}
    \item
    Quelle est la nature du triangle \(BEG\) ? 
    \item
    Quel est son périmètre ?
    \item
    Quelle est son aire ? (pour les rapides)
    \end{enumerate}
    

}
\vspace{2cm}
\vbox{13
\emph{Toutes les réponses doivent être justifiées par un calcul accompagné d'un raisonnement.}

\begin{wrapfigure}{r}{5.0cm}
   \vspace{-0.5cm}        % à adapter.
   \centering
   \input{Fig_OKTXHoc.pstricks}
\end{wrapfigure}

    La figure ci-contre est un cube de \unit{4}{\centi\meter}.

    \begin{enumerate}
    \item
    Quelle est la nature du triangle \(BAE\) ? 
    \item
    Quel est son périmètre ?
    \item
    Quelle est son aire ? (pour les rapides)
    \end{enumerate}
    

}
\vspace{2cm}
\vbox{14
\emph{Toutes les réponses doivent être justifiées par un calcul accompagné d'un raisonnement.}

\begin{wrapfigure}{r}{5.0cm}
   \vspace{-0.5cm}        % à adapter.
   \centering
   \input{Fig_OKTXHoc.pstricks}
\end{wrapfigure}

    La figure ci-contre est un cube de \unit{4}{\centi\meter}.

    \begin{enumerate}
    \item
    Quelle est la nature du triangle \(FCE\) ? 
    \item
    Quel est son périmètre ?
    \item
    Quelle est son aire ? (pour les rapides)
    \end{enumerate}
    

}
\vspace{2cm}
\vbox{15
\emph{Toutes les réponses doivent être justifiées par un calcul accompagné d'un raisonnement.}

\begin{wrapfigure}{r}{5.0cm}
   \vspace{-0.5cm}        % à adapter.
   \centering
   \input{Fig_OKTXHoc.pstricks}
\end{wrapfigure}

    La figure ci-contre est un cube de \unit{2}{\centi\meter}.

    \begin{enumerate}
    \item
    Quelle est la nature du triangle \(GBD\) ? 
    \item
    Quel est son périmètre ?
    \item
    Quelle est son aire ? (pour les rapides)
    \end{enumerate}
    

}
\vspace{2cm}
\vbox{16
\emph{Toutes les réponses doivent être justifiées par un calcul accompagné d'un raisonnement.}

\begin{wrapfigure}{r}{5.0cm}
   \vspace{-0.5cm}        % à adapter.
   \centering
   \input{Fig_OKTXHoc.pstricks}
\end{wrapfigure}

    La figure ci-contre est un cube de \unit{5}{\centi\meter}.

    \begin{enumerate}
    \item
    Quelle est la nature du triangle \(EHA\) ? 
    \item
    Quel est son périmètre ?
    \item
    Quelle est son aire ? (pour les rapides)
    \end{enumerate}
    

}
\vspace{2cm}
\vbox{17
\emph{Toutes les réponses doivent être justifiées par un calcul accompagné d'un raisonnement.}

\begin{wrapfigure}{r}{5.0cm}
   \vspace{-0.5cm}        % à adapter.
   \centering
   \input{Fig_OKTXHoc.pstricks}
\end{wrapfigure}

    La figure ci-contre est un cube de \unit{6}{\centi\meter}.

    \begin{enumerate}
    \item
    Quelle est la nature du triangle \(CFD\) ? 
    \item
    Quel est son périmètre ?
    \item
    Quelle est son aire ? (pour les rapides)
    \end{enumerate}
    

}
\vspace{2cm}
\vbox{18
\emph{Toutes les réponses doivent être justifiées par un calcul accompagné d'un raisonnement.}

\begin{wrapfigure}{r}{5.0cm}
   \vspace{-0.5cm}        % à adapter.
   \centering
   \input{Fig_OKTXHoc.pstricks}
\end{wrapfigure}

    La figure ci-contre est un cube de \unit{6}{\centi\meter}.

    \begin{enumerate}
    \item
    Quelle est la nature du triangle \(CFB\) ? 
    \item
    Quel est son périmètre ?
    \item
    Quelle est son aire ? (pour les rapides)
    \end{enumerate}
    

}
\vspace{2cm}
\vbox{19
\emph{Toutes les réponses doivent être justifiées par un calcul accompagné d'un raisonnement.}

\begin{wrapfigure}{r}{5.0cm}
   \vspace{-0.5cm}        % à adapter.
   \centering
   \input{Fig_OKTXHoc.pstricks}
\end{wrapfigure}

    La figure ci-contre est un cube de \unit{7}{\centi\meter}.

    \begin{enumerate}
    \item
    Quelle est la nature du triangle \(DAC\) ? 
    \item
    Quel est son périmètre ?
    \item
    Quelle est son aire ? (pour les rapides)
    \end{enumerate}
    

}
\vspace{2cm}
\vbox{20
\emph{Toutes les réponses doivent être justifiées par un calcul accompagné d'un raisonnement.}

\begin{wrapfigure}{r}{5.0cm}
   \vspace{-0.5cm}        % à adapter.
   \centering
   \input{Fig_OKTXHoc.pstricks}
\end{wrapfigure}

    La figure ci-contre est un cube de \unit{4}{\centi\meter}.

    \begin{enumerate}
    \item
    Quelle est la nature du triangle \(FEC\) ? 
    \item
    Quel est son périmètre ?
    \item
    Quelle est son aire ? (pour les rapides)
    \end{enumerate}
    

}
\vspace{2cm}
\vbox{21
\emph{Toutes les réponses doivent être justifiées par un calcul accompagné d'un raisonnement.}

\begin{wrapfigure}{r}{5.0cm}
   \vspace{-0.5cm}        % à adapter.
   \centering
   \input{Fig_OKTXHoc.pstricks}
\end{wrapfigure}

    La figure ci-contre est un cube de \unit{3}{\centi\meter}.

    \begin{enumerate}
    \item
    Quelle est la nature du triangle \(GCF\) ? 
    \item
    Quel est son périmètre ?
    \item
    Quelle est son aire ? (pour les rapides)
    \end{enumerate}
    

}
\vspace{2cm}
\vbox{22
\emph{Toutes les réponses doivent être justifiées par un calcul accompagné d'un raisonnement.}

\begin{wrapfigure}{r}{5.0cm}
   \vspace{-0.5cm}        % à adapter.
   \centering
   \input{Fig_OKTXHoc.pstricks}
\end{wrapfigure}

    La figure ci-contre est un cube de \unit{3}{\centi\meter}.

    \begin{enumerate}
    \item
    Quelle est la nature du triangle \(ADG\) ? 
    \item
    Quel est son périmètre ?
    \item
    Quelle est son aire ? (pour les rapides)
    \end{enumerate}
    

}
\vspace{2cm}
\vbox{23
\emph{Toutes les réponses doivent être justifiées par un calcul accompagné d'un raisonnement.}

\begin{wrapfigure}{r}{5.0cm}
   \vspace{-0.5cm}        % à adapter.
   \centering
   \input{Fig_OKTXHoc.pstricks}
\end{wrapfigure}

    La figure ci-contre est un cube de \unit{3}{\centi\meter}.

    \begin{enumerate}
    \item
    Quelle est la nature du triangle \(CHF\) ? 
    \item
    Quel est son périmètre ?
    \item
    Quelle est son aire ? (pour les rapides)
    \end{enumerate}
    

}
\vspace{2cm}
\vbox{24
\emph{Toutes les réponses doivent être justifiées par un calcul accompagné d'un raisonnement.}

\begin{wrapfigure}{r}{5.0cm}
   \vspace{-0.5cm}        % à adapter.
   \centering
   \input{Fig_OKTXHoc.pstricks}
\end{wrapfigure}

    La figure ci-contre est un cube de \unit{7}{\centi\meter}.

    \begin{enumerate}
    \item
    Quelle est la nature du triangle \(DBF\) ? 
    \item
    Quel est son périmètre ?
    \item
    Quelle est son aire ? (pour les rapides)
    \end{enumerate}
    

}
\vspace{2cm}
\vbox{25
\emph{Toutes les réponses doivent être justifiées par un calcul accompagné d'un raisonnement.}

\begin{wrapfigure}{r}{5.0cm}
   \vspace{-0.5cm}        % à adapter.
   \centering
   \input{Fig_OKTXHoc.pstricks}
\end{wrapfigure}

    La figure ci-contre est un cube de \unit{6}{\centi\meter}.

    \begin{enumerate}
    \item
    Quelle est la nature du triangle \(HDC\) ? 
    \item
    Quel est son périmètre ?
    \item
    Quelle est son aire ? (pour les rapides)
    \end{enumerate}
    

}
\vspace{2cm}
\vbox{26
\emph{Toutes les réponses doivent être justifiées par un calcul accompagné d'un raisonnement.}

\begin{wrapfigure}{r}{5.0cm}
   \vspace{-0.5cm}        % à adapter.
   \centering
   \input{Fig_OKTXHoc.pstricks}
\end{wrapfigure}

    La figure ci-contre est un cube de \unit{4}{\centi\meter}.

    \begin{enumerate}
    \item
    Quelle est la nature du triangle \(HCA\) ? 
    \item
    Quel est son périmètre ?
    \item
    Quelle est son aire ? (pour les rapides)
    \end{enumerate}
    

}
\vspace{2cm}
\vbox{27
\emph{Toutes les réponses doivent être justifiées par un calcul accompagné d'un raisonnement.}

\begin{wrapfigure}{r}{5.0cm}
   \vspace{-0.5cm}        % à adapter.
   \centering
   \input{Fig_OKTXHoc.pstricks}
\end{wrapfigure}

    La figure ci-contre est un cube de \unit{6}{\centi\meter}.

    \begin{enumerate}
    \item
    Quelle est la nature du triangle \(BEA\) ? 
    \item
    Quel est son périmètre ?
    \item
    Quelle est son aire ? (pour les rapides)
    \end{enumerate}
    

}
\vspace{2cm}
\vbox{28
\emph{Toutes les réponses doivent être justifiées par un calcul accompagné d'un raisonnement.}

\begin{wrapfigure}{r}{5.0cm}
   \vspace{-0.5cm}        % à adapter.
   \centering
   \input{Fig_OKTXHoc.pstricks}
\end{wrapfigure}

    La figure ci-contre est un cube de \unit{2}{\centi\meter}.

    \begin{enumerate}
    \item
    Quelle est la nature du triangle \(EAD\) ? 
    \item
    Quel est son périmètre ?
    \item
    Quelle est son aire ? (pour les rapides)
    \end{enumerate}
    

}
\vspace{2cm}
\vbox{29
\emph{Toutes les réponses doivent être justifiées par un calcul accompagné d'un raisonnement.}

\begin{wrapfigure}{r}{5.0cm}
   \vspace{-0.5cm}        % à adapter.
   \centering
   \input{Fig_OKTXHoc.pstricks}
\end{wrapfigure}

    La figure ci-contre est un cube de \unit{2}{\centi\meter}.

    \begin{enumerate}
    \item
    Quelle est la nature du triangle \(GFA\) ? 
    \item
    Quel est son périmètre ?
    \item
    Quelle est son aire ? (pour les rapides)
    \end{enumerate}
    

}
\vspace{2cm}
\vbox{30
\emph{Toutes les réponses doivent être justifiées par un calcul accompagné d'un raisonnement.}

\begin{wrapfigure}{r}{5.0cm}
   \vspace{-0.5cm}        % à adapter.
   \centering
   \input{Fig_OKTXHoc.pstricks}
\end{wrapfigure}

    La figure ci-contre est un cube de \unit{4}{\centi\meter}.

    \begin{enumerate}
    \item
    Quelle est la nature du triangle \(GDF\) ? 
    \item
    Quel est son périmètre ?
    \item
    Quelle est son aire ? (pour les rapides)
    \end{enumerate}
    

}
\vspace{2cm}
\vbox{31
\emph{Toutes les réponses doivent être justifiées par un calcul accompagné d'un raisonnement.}

\begin{wrapfigure}{r}{5.0cm}
   \vspace{-0.5cm}        % à adapter.
   \centering
   \input{Fig_OKTXHoc.pstricks}
\end{wrapfigure}

    La figure ci-contre est un cube de \unit{4}{\centi\meter}.

    \begin{enumerate}
    \item
    Quelle est la nature du triangle \(BDE\) ? 
    \item
    Quel est son périmètre ?
    \item
    Quelle est son aire ? (pour les rapides)
    \end{enumerate}
    

}
\vspace{2cm}
\vbox{32
\emph{Toutes les réponses doivent être justifiées par un calcul accompagné d'un raisonnement.}

\begin{wrapfigure}{r}{5.0cm}
   \vspace{-0.5cm}        % à adapter.
   \centering
   \input{Fig_OKTXHoc.pstricks}
\end{wrapfigure}

    La figure ci-contre est un cube de \unit{5}{\centi\meter}.

    \begin{enumerate}
    \item
    Quelle est la nature du triangle \(CDB\) ? 
    \item
    Quel est son périmètre ?
    \item
    Quelle est son aire ? (pour les rapides)
    \end{enumerate}
    

}
\vspace{2cm}
\vbox{33
\emph{Toutes les réponses doivent être justifiées par un calcul accompagné d'un raisonnement.}

\begin{wrapfigure}{r}{5.0cm}
   \vspace{-0.5cm}        % à adapter.
   \centering
   \input{Fig_OKTXHoc.pstricks}
\end{wrapfigure}

    La figure ci-contre est un cube de \unit{3}{\centi\meter}.

    \begin{enumerate}
    \item
    Quelle est la nature du triangle \(HBF\) ? 
    \item
    Quel est son périmètre ?
    \item
    Quelle est son aire ? (pour les rapides)
    \end{enumerate}
    

}
\vspace{2cm}
\vbox{34
\emph{Toutes les réponses doivent être justifiées par un calcul accompagné d'un raisonnement.}

\begin{wrapfigure}{r}{5.0cm}
   \vspace{-0.5cm}        % à adapter.
   \centering
   \input{Fig_OKTXHoc.pstricks}
\end{wrapfigure}

    La figure ci-contre est un cube de \unit{7}{\centi\meter}.

    \begin{enumerate}
    \item
    Quelle est la nature du triangle \(BAG\) ? 
    \item
    Quel est son périmètre ?
    \item
    Quelle est son aire ? (pour les rapides)
    \end{enumerate}
    

}
\vspace{2cm}
\vbox{35
\emph{Toutes les réponses doivent être justifiées par un calcul accompagné d'un raisonnement.}

\begin{wrapfigure}{r}{5.0cm}
   \vspace{-0.5cm}        % à adapter.
   \centering
   \input{Fig_OKTXHoc.pstricks}
\end{wrapfigure}

    La figure ci-contre est un cube de \unit{2}{\centi\meter}.

    \begin{enumerate}
    \item
    Quelle est la nature du triangle \(BHE\) ? 
    \item
    Quel est son périmètre ?
    \item
    Quelle est son aire ? (pour les rapides)
    \end{enumerate}
    

}
\vspace{2cm}
\vbox{36
\emph{Toutes les réponses doivent être justifiées par un calcul accompagné d'un raisonnement.}

\begin{wrapfigure}{r}{5.0cm}
   \vspace{-0.5cm}        % à adapter.
   \centering
   \input{Fig_OKTXHoc.pstricks}
\end{wrapfigure}

    La figure ci-contre est un cube de \unit{7}{\centi\meter}.

    \begin{enumerate}
    \item
    Quelle est la nature du triangle \(CEG\) ? 
    \item
    Quel est son périmètre ?
    \item
    Quelle est son aire ? (pour les rapides)
    \end{enumerate}
    

}
\vspace{2cm}
\vbox{37
\emph{Toutes les réponses doivent être justifiées par un calcul accompagné d'un raisonnement.}

\begin{wrapfigure}{r}{5.0cm}
   \vspace{-0.5cm}        % à adapter.
   \centering
   \input{Fig_OKTXHoc.pstricks}
\end{wrapfigure}

    La figure ci-contre est un cube de \unit{3}{\centi\meter}.

    \begin{enumerate}
    \item
    Quelle est la nature du triangle \(ADG\) ? 
    \item
    Quel est son périmètre ?
    \item
    Quelle est son aire ? (pour les rapides)
    \end{enumerate}
    

}
\vspace{2cm}
\vbox{38
\emph{Toutes les réponses doivent être justifiées par un calcul accompagné d'un raisonnement.}

\begin{wrapfigure}{r}{5.0cm}
   \vspace{-0.5cm}        % à adapter.
   \centering
   \input{Fig_OKTXHoc.pstricks}
\end{wrapfigure}

    La figure ci-contre est un cube de \unit{6}{\centi\meter}.

    \begin{enumerate}
    \item
    Quelle est la nature du triangle \(GHD\) ? 
    \item
    Quel est son périmètre ?
    \item
    Quelle est son aire ? (pour les rapides)
    \end{enumerate}
    

}
\vspace{2cm}
\vbox{39
\emph{Toutes les réponses doivent être justifiées par un calcul accompagné d'un raisonnement.}

\begin{wrapfigure}{r}{5.0cm}
   \vspace{-0.5cm}        % à adapter.
   \centering
   \input{Fig_OKTXHoc.pstricks}
\end{wrapfigure}

    La figure ci-contre est un cube de \unit{6}{\centi\meter}.

    \begin{enumerate}
    \item
    Quelle est la nature du triangle \(BCE\) ? 
    \item
    Quel est son périmètre ?
    \item
    Quelle est son aire ? (pour les rapides)
    \end{enumerate}
    

}
\vspace{2cm}

\end{document}


\end{document}



------------------------------------




\end{document}

% ACTIVITÉ STATISTIQUE DESCRIPTIVE


% This is part of Un soupçon de mathématique sans être agressif pour autant
% Copyright (c) 2013
%   Laurent Claessens
% See the file fdl-1.3.txt for copying conditions.

% ATTENTION : les chiffres donnés ici sont repris dans le cours au moment des ECC.

\begin{wrapfigure}[5]{r}{8cm}
   \vspace{-0.5cm}        % à adapter.
   \centering
   \input{Fig_YRQOoPE.pstricks}
\end{wrapfigure}

Le graphique ci-contre illustre le nombre de spam reçus aujourd'hui par les élèves d'une classe.
\begin{enumerate}
    \item
        Combien d'élèves y-a-t-il dans la classe ?
    \item
        Combien d'élèves ont reçu \( 5\) spams ou plus ?
    \item
        En moyenne combien de spam ont reçu les élèves aujourd'hui ?
    \item
        Diviser la classe en 4 groupes suivant le nombre de spams reçus.
\end{enumerate}



\vspace{5cm}

% This is part of Un soupçon de mathématique sans être agressif pour autant
% Copyright (c) 2013
%   Laurent Claessens
% See the file fdl-1.3.txt for copying conditions.

% ATTENTION : les chiffres donnés ici sont repris dans le cours au moment des ECC.

\begin{wrapfigure}[5]{r}{8cm}
   \vspace{-0.5cm}        % à adapter.
   \centering
   \input{Fig_YRQOoPE.pstricks}
\end{wrapfigure}

Le graphique ci-contre illustre le nombre de spam reçus aujourd'hui par les élèves d'une classe.
\begin{enumerate}
    \item
        Combien d'élèves y-a-t-il dans la classe ?
    \item
        Combien d'élèves ont reçu \( 5\) spams ou plus ?
    \item
        En moyenne combien de spam ont reçu les élèves aujourd'hui ?
    \item
        Diviser la classe en 4 groupes suivant le nombre de spams reçus.
\end{enumerate}



\vspace{5cm}

% This is part of Un soupçon de mathématique sans être agressif pour autant
% Copyright (c) 2013
%   Laurent Claessens
% See the file fdl-1.3.txt for copying conditions.

% ATTENTION : les chiffres donnés ici sont repris dans le cours au moment des ECC.

\begin{wrapfigure}[5]{r}{8cm}
   \vspace{-0.5cm}        % à adapter.
   \centering
   \input{Fig_YRQOoPE.pstricks}
\end{wrapfigure}

Le graphique ci-contre illustre le nombre de spam reçus aujourd'hui par les élèves d'une classe.
\begin{enumerate}
    \item
        Combien d'élèves y-a-t-il dans la classe ?
    \item
        Combien d'élèves ont reçu \( 5\) spams ou plus ?
    \item
        En moyenne combien de spam ont reçu les élèves aujourd'hui ?
    \item
        Diviser la classe en 4 groupes suivant le nombre de spams reçus.
\end{enumerate}



\end{document}

% ACTIVITÉ GÉOMÉTRIE DANS L'ESPACE
% This is part of Un soupçon de mathématique sans être agressif pour autant
% Copyright (c) 2013
%   Laurent Claessens
% See the file fdl-1.3.txt for copying conditions.

\begin{wrapfigure}[3]{r}{5.0cm}
   \vspace{-0.5cm}        % à adapter.
   \centering
   \input{Fig_MEzTDZC.pstricks}
\end{wrapfigure}

À l'aide du cube ci-contre :
\begin{enumerate}
    \item
        Quelle est la nature du triangle \( AEF\) ? 
    \item
        Quelle est la nature du quadrilatère \( ABGH\) ?
    \item
        Est-ce que vous pouvez trouver trois points de ce cube formant un triangle équilatéral ?
    \item
        Si ce cube fait \unit{3}{\meter} de côté, quelle est la longueur de la «grande» diagonale \( [AG]\) ?
\end{enumerate}



\vspace{2cm}

% This is part of Un soupçon de mathématique sans être agressif pour autant
% Copyright (c) 2013
%   Laurent Claessens
% See the file fdl-1.3.txt for copying conditions.

\begin{wrapfigure}[3]{r}{5.0cm}
   \vspace{-0.5cm}        % à adapter.
   \centering
   \input{Fig_MEzTDZC.pstricks}
\end{wrapfigure}

À l'aide du cube ci-contre :
\begin{enumerate}
    \item
        Quelle est la nature du triangle \( AEF\) ? 
    \item
        Quelle est la nature du quadrilatère \( ABGH\) ?
    \item
        Est-ce que vous pouvez trouver trois points de ce cube formant un triangle équilatéral ?
    \item
        Si ce cube fait \unit{3}{\meter} de côté, quelle est la longueur de la «grande» diagonale \( [AG]\) ?
\end{enumerate}



\vspace{2cm}

% This is part of Un soupçon de mathématique sans être agressif pour autant
% Copyright (c) 2013
%   Laurent Claessens
% See the file fdl-1.3.txt for copying conditions.

\begin{wrapfigure}[3]{r}{5.0cm}
   \vspace{-0.5cm}        % à adapter.
   \centering
   \input{Fig_MEzTDZC.pstricks}
\end{wrapfigure}

À l'aide du cube ci-contre :
\begin{enumerate}
    \item
        Quelle est la nature du triangle \( AEF\) ? 
    \item
        Quelle est la nature du quadrilatère \( ABGH\) ?
    \item
        Est-ce que vous pouvez trouver trois points de ce cube formant un triangle équilatéral ?
    \item
        Si ce cube fait \unit{3}{\meter} de côté, quelle est la longueur de la «grande» diagonale \( [AG]\) ?
\end{enumerate}




\vspace{2cm}

% This is part of Un soupçon de mathématique sans être agressif pour autant
% Copyright (c) 2013
%   Laurent Claessens
% See the file fdl-1.3.txt for copying conditions.

\begin{wrapfigure}[3]{r}{5.0cm}
   \vspace{-0.5cm}        % à adapter.
   \centering
   \input{Fig_MEzTDZC.pstricks}
\end{wrapfigure}

À l'aide du cube ci-contre :
\begin{enumerate}
    \item
        Quelle est la nature du triangle \( AEF\) ? 
    \item
        Quelle est la nature du quadrilatère \( ABGH\) ?
    \item
        Est-ce que vous pouvez trouver trois points de ce cube formant un triangle équilatéral ?
    \item
        Si ce cube fait \unit{3}{\meter} de côté, quelle est la longueur de la «grande» diagonale \( [AG]\) ?
\end{enumerate}



\vspace{2cm}

% This is part of Un soupçon de mathématique sans être agressif pour autant
% Copyright (c) 2013
%   Laurent Claessens
% See the file fdl-1.3.txt for copying conditions.

\begin{wrapfigure}[3]{r}{5.0cm}
   \vspace{-0.5cm}        % à adapter.
   \centering
   \input{Fig_MEzTDZC.pstricks}
\end{wrapfigure}

À l'aide du cube ci-contre :
\begin{enumerate}
    \item
        Quelle est la nature du triangle \( AEF\) ? 
    \item
        Quelle est la nature du quadrilatère \( ABGH\) ?
    \item
        Est-ce que vous pouvez trouver trois points de ce cube formant un triangle équilatéral ?
    \item
        Si ce cube fait \unit{3}{\meter} de côté, quelle est la longueur de la «grande» diagonale \( [AG]\) ?
\end{enumerate}




\end{document}

%DS MARDI 5 NOVEMBRE 2013, seconde D
\begin{feuilleDS}{Seconde D, DS numéro 2\\ \small mardi 5 novembre 2013}

\Exo{smath-0516}
\Exo{smath-0474}
\Exo{smath-0521}
\Exo{smath-0493}
\Exo{smath-0522}

\end{feuilleDS}


%DS MARDI 12 NOVEMBRE 2013, seconde A
\begin{feuilleDS}{Seconde A, DS numéro 2\\ \small mardi 12 novembre 2013}

\Exo{smath-0515}
\Exo{smath-0517}
\Exo{smath-0518}
\Exo{smath-0519}
\Exo{smath-0520}

\end{feuilleDS}

\end{document}

%DS MARDI 1 OCTOBRE 2013, seconde A

\begin{feuilleDS}{Seconde A, DS numéro 1\\ \small mardi 1 octobre 2013}

\Exo{Seconde-0047}
\Exo{smath-0508}
\Exo{smath-0512}
\Exo{smath-0477}
\Exo{smath-0507}

\end{feuilleDS}

\newpage

\setcounter{CountExercice}{0}

%DS MARDI 1 OCTOBRE 2013, seconde D
\begin{feuilleDS}{Seconde D, DS numéro 1\\ \small mardi 1 octobre 2013}


\Exo{smath-0510}
\Exo{smath-0509}
\Exo{smath-0508}
\Exo{smath-0507}
\Exo{smath-0511}

\end{feuilleDS}

\end{document}



%INTERRO REPERE, DISTANCE, MILIEU et sa correction.
\vbox{1
\emph{Toutes les réponses doivent être justifiées par un calcul accompagné d'un raisonnement.}
\begin{enumerate}\item
Placer les points \( A=(-5;3)\), \( B=(1;9)\), \( C=(-9;8)\) et \( D=(6;-1)\) dans un repère orthonormé. Calculer la longueur du segment \( [AB]\) et les coordonnées du milieu du segment \( [CD]\).
\item
Est-ce que le triangle formé par les points \( A(2;6)\), \( B(6;2)\) et \( C(17;7)\) est isocèle ?

\end{enumerate}
} 
\vspace{2cm}
\vbox{2
\emph{Toutes les réponses doivent être justifiées par un calcul accompagné d'un raisonnement.}
\begin{enumerate}\item
Placer les points \( A=(-5;3)\), \( B=(-6;-10)\), \( C=(10;-7)\) et \( D=(4;1)\) dans un repère orthonormé. Calculer la longueur du segment \( [AB]\) et les coordonnées du milieu du segment \( [CD]\).
\item
Est-ce que le triangle formé par les points \( A(1;-1)\), \( B(-5;5)\) et \( C(2;6)\) est isocèle ?

\end{enumerate}
} 
\vspace{2cm}
\vbox{3
\emph{Toutes les réponses doivent être justifiées par un calcul accompagné d'un raisonnement.}
\begin{enumerate}\item
Placer les points \( A=(0;-9)\), \( B=(6;9)\), \( C=(-1;9)\) et \( D=(8;10)\) dans un repère orthonormé. Calculer la longueur du segment \( [AB]\) et les coordonnées du milieu du segment \( [CD]\).
\item
Est-ce que le triangle formé par les points \( A(-6;-2)\), \( B(-11;3)\) et \( C(1;-5)\) est rectangle ?

\end{enumerate}
} 
\vspace{2cm}
\vbox{4
\emph{Toutes les réponses doivent être justifiées par un calcul accompagné d'un raisonnement.}
\begin{enumerate}\item
Placer les points \( A=(4;-6)\), \( B=(-7;-2)\), \( C=(-8;-8)\) et \( D=(-4;-2)\) dans un repère orthonormé. Calculer la longueur du segment \( [AB]\) et les coordonnées du milieu du segment \( [CD]\).
\item
Est-ce que le triangle formé par les points \( A(-3;-10)\), \( B(-7;-6)\) et \( C(4;-9)\) est isocèle ?

\end{enumerate}
} 
\vspace{2cm}
\vbox{5
\emph{Toutes les réponses doivent être justifiées par un calcul accompagné d'un raisonnement.}
\begin{enumerate}\item
Placer les points \( A=(6;6)\), \( B=(-10;-9)\), \( C=(-4;9)\) et \( D=(-4;2)\) dans un repère orthonormé. Calculer la longueur du segment \( [AB]\) et les coordonnées du milieu du segment \( [CD]\).
\item
Est-ce que le triangle formé par les points \( A(7;10)\), \( B(1;16)\) et \( C(18;17)\) est isocèle ?

\end{enumerate}
} 
\vspace{2cm}
\vbox{6
\emph{Toutes les réponses doivent être justifiées par un calcul accompagné d'un raisonnement.}
\begin{enumerate}\item
Placer les points \( A=(-7;7)\), \( B=(-4;0)\), \( C=(4;-2)\) et \( D=(10;-8)\) dans un repère orthonormé. Calculer la longueur du segment \( [AB]\) et les coordonnées du milieu du segment \( [CD]\).
\item
Est-ce que le triangle formé par les points \( A(-2;10)\), \( B(-1;9)\) et \( C(-4;8)\) est rectangle ?

\end{enumerate}
} 
\vspace{2cm}
\vbox{7
\emph{Toutes les réponses doivent être justifiées par un calcul accompagné d'un raisonnement.}
\begin{enumerate}\item
Placer les points \( A=(8;2)\), \( B=(7;5)\), \( C=(-10;5)\) et \( D=(1;10)\) dans un repère orthonormé. Calculer la longueur du segment \( [AB]\) et les coordonnées du milieu du segment \( [CD]\).
\item
Est-ce que le triangle formé par les points \( A(8;-1)\), \( B(4;3)\) et \( C(10;1)\) est rectangle ?

\end{enumerate}
} 
\vspace{2cm}
\vbox{8
\emph{Toutes les réponses doivent être justifiées par un calcul accompagné d'un raisonnement.}
\begin{enumerate}\item
Placer les points \( A=(-4;-8)\), \( B=(0;8)\), \( C=(-2;1)\) et \( D=(0;-3)\) dans un repère orthonormé. Calculer la longueur du segment \( [AB]\) et les coordonnées du milieu du segment \( [CD]\).
\item
Est-ce que le triangle formé par les points \( A(4;-4)\), \( B(1;-1)\) et \( C(7;-1)\) est rectangle ?

\end{enumerate}
} 
\vspace{2cm}
\vbox{9
\emph{Toutes les réponses doivent être justifiées par un calcul accompagné d'un raisonnement.}
\begin{enumerate}\item
Placer les points \( A=(8;-5)\), \( B=(8;-2)\), \( C=(8;2)\) et \( D=(0;10)\) dans un repère orthonormé. Calculer la longueur du segment \( [AB]\) et les coordonnées du milieu du segment \( [CD]\).
\item
Est-ce que le triangle formé par les points \( A(-6;6)\), \( B(-1;1)\) et \( C(3;5)\) est rectangle ?

\end{enumerate}
} 
\vspace{2cm}
\vbox{10
\emph{Toutes les réponses doivent être justifiées par un calcul accompagné d'un raisonnement.}
\begin{enumerate}\item
Placer les points \( A=(4;-2)\), \( B=(7;9)\), \( C=(-10;-9)\) et \( D=(5;-2)\) dans un repère orthonormé. Calculer la longueur du segment \( [AB]\) et les coordonnées du milieu du segment \( [CD]\).
\item
Est-ce que le triangle formé par les points \( A(-1;-2)\), \( B(-4;1)\) et \( C(1;0)\) est rectangle ?

\end{enumerate}
} 
\vspace{2cm}
\vbox{11
\emph{Toutes les réponses doivent être justifiées par un calcul accompagné d'un raisonnement.}
\begin{enumerate}\item
Placer les points \( A=(5;9)\), \( B=(-7;5)\), \( C=(7;-1)\) et \( D=(9;-6)\) dans un repère orthonormé. Calculer la longueur du segment \( [AB]\) et les coordonnées du milieu du segment \( [CD]\).
\item
Est-ce que le triangle formé par les points \( A(3;9)\), \( B(-3;15)\) et \( C(8;10)\) est isocèle ?

\end{enumerate}
} 
\vspace{2cm}
\vbox{12
\emph{Toutes les réponses doivent être justifiées par un calcul accompagné d'un raisonnement.}
\begin{enumerate}\item
Placer les points \( A=(-10;-5)\), \( B=(-10;1)\), \( C=(-5;5)\) et \( D=(0;-3)\) dans un repère orthonormé. Calculer la longueur du segment \( [AB]\) et les coordonnées du milieu du segment \( [CD]\).
\item
Est-ce que le triangle formé par les points \( A(6;-1)\), \( B(12;-7)\) et \( C(13;0)\) est isocèle ?

\end{enumerate}
} 
\vspace{2cm}
\vbox{13
\emph{Toutes les réponses doivent être justifiées par un calcul accompagné d'un raisonnement.}
\begin{enumerate}\item
Placer les points \( A=(10;6)\), \( B=(-5;-7)\), \( C=(-1;7)\) et \( D=(-8;5)\) dans un repère orthonormé. Calculer la longueur du segment \( [AB]\) et les coordonnées du milieu du segment \( [CD]\).
\item
Est-ce que le triangle formé par les points \( A(-10;-10)\), \( B(-16;-4)\) et \( C(0;-4)\) est isocèle ?

\end{enumerate}
} 
\vspace{2cm}
\vbox{14
\emph{Toutes les réponses doivent être justifiées par un calcul accompagné d'un raisonnement.}
\begin{enumerate}\item
Placer les points \( A=(1;-7)\), \( B=(9;-3)\), \( C=(5;7)\) et \( D=(-6;-5)\) dans un repère orthonormé. Calculer la longueur du segment \( [AB]\) et les coordonnées du milieu du segment \( [CD]\).
\item
Est-ce que le triangle formé par les points \( A(-3;-8)\), \( B(-9;-2)\) et \( C(-9;-8)\) est isocèle ?

\end{enumerate}
} 
\vspace{2cm}
\vbox{15
\emph{Toutes les réponses doivent être justifiées par un calcul accompagné d'un raisonnement.}
\begin{enumerate}\item
Placer les points \( A=(-7;-7)\), \( B=(-1;-9)\), \( C=(10;-8)\) et \( D=(-9;4)\) dans un repère orthonormé. Calculer la longueur du segment \( [AB]\) et les coordonnées du milieu du segment \( [CD]\).
\item
Est-ce que le triangle formé par les points \( A(2;-10)\), \( B(7;-15)\) et \( C(1;-11)\) est rectangle ?

\end{enumerate}
} 
\vspace{2cm}
\vbox{16
\emph{Toutes les réponses doivent être justifiées par un calcul accompagné d'un raisonnement.}
\begin{enumerate}\item
Placer les points \( A=(-3;-10)\), \( B=(-10;-9)\), \( C=(-2;-4)\) et \( D=(-7;3)\) dans un repère orthonormé. Calculer la longueur du segment \( [AB]\) et les coordonnées du milieu du segment \( [CD]\).
\item
Est-ce que le triangle formé par les points \( A(-5;-2)\), \( B(0;-7)\) et \( C(-8;-5)\) est rectangle ?

\end{enumerate}
} 
\vspace{2cm}
\vbox{17
\emph{Toutes les réponses doivent être justifiées par un calcul accompagné d'un raisonnement.}
\begin{enumerate}\item
Placer les points \( A=(3;6)\), \( B=(7;8)\), \( C=(6;9)\) et \( D=(8;-1)\) dans un repère orthonormé. Calculer la longueur du segment \( [AB]\) et les coordonnées du milieu du segment \( [CD]\).
\item
Est-ce que le triangle formé par les points \( A(6;7)\), \( B(8;5)\) et \( C(21;10)\) est isocèle ?

\end{enumerate}
} 
\vspace{2cm}
\vbox{18
\emph{Toutes les réponses doivent être justifiées par un calcul accompagné d'un raisonnement.}
\begin{enumerate}\item
Placer les points \( A=(-1;-10)\), \( B=(4;-4)\), \( C=(-4;-1)\) et \( D=(-5;1)\) dans un repère orthonormé. Calculer la longueur du segment \( [AB]\) et les coordonnées du milieu du segment \( [CD]\).
\item
Est-ce que le triangle formé par les points \( A(-10;-4)\), \( B(-11;-3)\) et \( C(-1;-5)\) est rectangle ?

\end{enumerate}
} 
\vspace{2cm}
\vbox{19
\emph{Toutes les réponses doivent être justifiées par un calcul accompagné d'un raisonnement.}
\begin{enumerate}\item
Placer les points \( A=(-9;0)\), \( B=(-4;-1)\), \( C=(3;-2)\) et \( D=(7;-6)\) dans un repère orthonormé. Calculer la longueur du segment \( [AB]\) et les coordonnées du milieu du segment \( [CD]\).
\item
Est-ce que le triangle formé par les points \( A(10;-6)\), \( B(13;-9)\) et \( C(23;-3)\) est rectangle ?

\end{enumerate}
} 
\vspace{2cm}
\vbox{20
\emph{Toutes les réponses doivent être justifiées par un calcul accompagné d'un raisonnement.}
\begin{enumerate}\item
Placer les points \( A=(-5;0)\), \( B=(6;8)\), \( C=(4;5)\) et \( D=(-4;-6)\) dans un repère orthonormé. Calculer la longueur du segment \( [AB]\) et les coordonnées du milieu du segment \( [CD]\).
\item
Est-ce que le triangle formé par les points \( A(1;-7)\), \( B(5;-11)\) et \( C(5;-7)\) est isocèle ?

\end{enumerate}
} 
\vspace{2cm}
\vbox{21
\emph{Toutes les réponses doivent être justifiées par un calcul accompagné d'un raisonnement.}
\begin{enumerate}\item
Placer les points \( A=(-4;5)\), \( B=(0;-10)\), \( C=(7;-4)\) et \( D=(4;9)\) dans un repère orthonormé. Calculer la longueur du segment \( [AB]\) et les coordonnées du milieu du segment \( [CD]\).
\item
Est-ce que le triangle formé par les points \( A(5;-4)\), \( B(1;0)\) et \( C(17;-2)\) est rectangle ?

\end{enumerate}
} 
\vspace{2cm}
\vbox{22
\emph{Toutes les réponses doivent être justifiées par un calcul accompagné d'un raisonnement.}
\begin{enumerate}\item
Placer les points \( A=(8;8)\), \( B=(-3;-2)\), \( C=(3;2)\) et \( D=(-8;9)\) dans un repère orthonormé. Calculer la longueur du segment \( [AB]\) et les coordonnées du milieu du segment \( [CD]\).
\item
Est-ce que le triangle formé par les points \( A(1;-1)\), \( B(-1;1)\) et \( C(1;1)\) est isocèle ?

\end{enumerate}
} 
\vspace{2cm}
\vbox{23
\emph{Toutes les réponses doivent être justifiées par un calcul accompagné d'un raisonnement.}
\begin{enumerate}\item
Placer les points \( A=(8;10)\), \( B=(-7;1)\), \( C=(-9;5)\) et \( D=(-5;8)\) dans un repère orthonormé. Calculer la longueur du segment \( [AB]\) et les coordonnées du milieu du segment \( [CD]\).
\item
Est-ce que le triangle formé par les points \( A(0;9)\), \( B(-4;13)\) et \( C(7;6)\) est rectangle ?

\end{enumerate}
} 
\vspace{2cm}
\vbox{24
\emph{Toutes les réponses doivent être justifiées par un calcul accompagné d'un raisonnement.}
\begin{enumerate}\item
Placer les points \( A=(4;-2)\), \( B=(8;1)\), \( C=(4;-3)\) et \( D=(-3;6)\) dans un repère orthonormé. Calculer la longueur du segment \( [AB]\) et les coordonnées du milieu du segment \( [CD]\).
\item
Est-ce que le triangle formé par les points \( A(1;4)\), \( B(-4;9)\) et \( C(9;2)\) est rectangle ?

\end{enumerate}
} 
\vspace{2cm}
\vbox{25
\emph{Toutes les réponses doivent être justifiées par un calcul accompagné d'un raisonnement.}
\begin{enumerate}\item
Placer les points \( A=(2;-5)\), \( B=(-9;4)\), \( C=(5;4)\) et \( D=(-7;4)\) dans un repère orthonormé. Calculer la longueur du segment \( [AB]\) et les coordonnées du milieu du segment \( [CD]\).
\item
Est-ce que le triangle formé par les points \( A(-4;3)\), \( B(-2;1)\) et \( C(-3;4)\) est rectangle ?

\end{enumerate}
} 
\vspace{2cm}
\vbox{26
\emph{Toutes les réponses doivent être justifiées par un calcul accompagné d'un raisonnement.}
\begin{enumerate}\item
Placer les points \( A=(1;3)\), \( B=(-4;-8)\), \( C=(-6;9)\) et \( D=(-7;9)\) dans un repère orthonormé. Calculer la longueur du segment \( [AB]\) et les coordonnées du milieu du segment \( [CD]\).
\item
Est-ce que le triangle formé par les points \( A(-5;3)\), \( B(-11;9)\) et \( C(-6;8)\) est isocèle ?

\end{enumerate}
} 
\vspace{2cm}
\vbox{27
\emph{Toutes les réponses doivent être justifiées par un calcul accompagné d'un raisonnement.}
\begin{enumerate}\item
Placer les points \( A=(2;-2)\), \( B=(8;-5)\), \( C=(-1;1)\) et \( D=(10;2)\) dans un repère orthonormé. Calculer la longueur du segment \( [AB]\) et les coordonnées du milieu du segment \( [CD]\).
\item
Est-ce que le triangle formé par les points \( A(1;7)\), \( B(-2;10)\) et \( C(14;10)\) est rectangle ?

\end{enumerate}
} 
\vspace{2cm}
\vbox{28
\emph{Toutes les réponses doivent être justifiées par un calcul accompagné d'un raisonnement.}
\begin{enumerate}\item
Placer les points \( A=(7;5)\), \( B=(1;0)\), \( C=(10;1)\) et \( D=(1;-9)\) dans un repère orthonormé. Calculer la longueur du segment \( [AB]\) et les coordonnées du milieu du segment \( [CD]\).
\item
Est-ce que le triangle formé par les points \( A(-9;-4)\), \( B(-12;-1)\) et \( C(-11;-6)\) est rectangle ?

\end{enumerate}
} 
\vspace{2cm}
\vbox{29
\emph{Toutes les réponses doivent être justifiées par un calcul accompagné d'un raisonnement.}
\begin{enumerate}\item
Placer les points \( A=(-2;-3)\), \( B=(-7;4)\), \( C=(-10;-6)\) et \( D=(-10;-3)\) dans un repère orthonormé. Calculer la longueur du segment \( [AB]\) et les coordonnées du milieu du segment \( [CD]\).
\item
Est-ce que le triangle formé par les points \( A(9;6)\), \( B(10;5)\) et \( C(18;5)\) est rectangle ?

\end{enumerate}
} 
\vspace{2cm}
\vbox{30
\emph{Toutes les réponses doivent être justifiées par un calcul accompagné d'un raisonnement.}
\begin{enumerate}\item
Placer les points \( A=(6;9)\), \( B=(1;9)\), \( C=(3;9)\) et \( D=(1;8)\) dans un repère orthonormé. Calculer la longueur du segment \( [AB]\) et les coordonnées du milieu du segment \( [CD]\).
\item
Est-ce que le triangle formé par les points \( A(10;4)\), \( B(8;6)\) et \( C(19;3)\) est rectangle ?

\end{enumerate}
} 
\vspace{2cm}
\vbox{31
\emph{Toutes les réponses doivent être justifiées par un calcul accompagné d'un raisonnement.}
\begin{enumerate}\item
Placer les points \( A=(-1;7)\), \( B=(3;-9)\), \( C=(7;-1)\) et \( D=(-7;-10)\) dans un repère orthonormé. Calculer la longueur du segment \( [AB]\) et les coordonnées du milieu du segment \( [CD]\).
\item
Est-ce que le triangle formé par les points \( A(-9;3)\), \( B(-5;-1)\) et \( C(-4;4)\) est isocèle ?

\end{enumerate}
} 
\vspace{2cm}
\vbox{32
\emph{Toutes les réponses doivent être justifiées par un calcul accompagné d'un raisonnement.}
\begin{enumerate}\item
Placer les points \( A=(7;8)\), \( B=(-5;3)\), \( C=(1;7)\) et \( D=(-10;-10)\) dans un repère orthonormé. Calculer la longueur du segment \( [AB]\) et les coordonnées du milieu du segment \( [CD]\).
\item
Est-ce que le triangle formé par les points \( A(-4;5)\), \( B(-7;8)\) et \( C(5;4)\) est rectangle ?

\end{enumerate}
} 
\vspace{2cm}
\vbox{33
\emph{Toutes les réponses doivent être justifiées par un calcul accompagné d'un raisonnement.}
\begin{enumerate}\item
Placer les points \( A=(-1;4)\), \( B=(10;8)\), \( C=(4;-8)\) et \( D=(5;7)\) dans un repère orthonormé. Calculer la longueur du segment \( [AB]\) et les coordonnées du milieu du segment \( [CD]\).
\item
Est-ce que le triangle formé par les points \( A(-9;6)\), \( B(-14;11)\) et \( C(-11;4)\) est rectangle ?

\end{enumerate}
} 
\vspace{2cm}
\vbox{34
\emph{Toutes les réponses doivent être justifiées par un calcul accompagné d'un raisonnement.}
\begin{enumerate}\item
Placer les points \( A=(-5;-2)\), \( B=(9;10)\), \( C=(7;0)\) et \( D=(-6;8)\) dans un repère orthonormé. Calculer la longueur du segment \( [AB]\) et les coordonnées du milieu du segment \( [CD]\).
\item
Est-ce que le triangle formé par les points \( A(-6;-7)\), \( B(-10;-3)\) et \( C(3;-4)\) est isocèle ?

\end{enumerate}
} 
\vspace{2cm}
\vbox{35
\emph{Toutes les réponses doivent être justifiées par un calcul accompagné d'un raisonnement.}
\begin{enumerate}\item
Placer les points \( A=(-4;-9)\), \( B=(-6;6)\), \( C=(-6;8)\) et \( D=(8;-8)\) dans un repère orthonormé. Calculer la longueur du segment \( [AB]\) et les coordonnées du milieu du segment \( [CD]\).
\item
Est-ce que le triangle formé par les points \( A(-6;-4)\), \( B(-9;-1)\) et \( C(2;-6)\) est rectangle ?

\end{enumerate}
} 
\vspace{2cm}
\vbox{36
\emph{Toutes les réponses doivent être justifiées par un calcul accompagné d'un raisonnement.}
\begin{enumerate}\item
Placer les points \( A=(1;-4)\), \( B=(9;1)\), \( C=(-8;-1)\) et \( D=(-1;7)\) dans un repère orthonormé. Calculer la longueur du segment \( [AB]\) et les coordonnées du milieu du segment \( [CD]\).
\item
Est-ce que le triangle formé par les points \( A(2;-10)\), \( B(8;-16)\) et \( C(11;-17)\) est isocèle ?

\end{enumerate}
} 
\vspace{2cm}
\vbox{37
\emph{Toutes les réponses doivent être justifiées par un calcul accompagné d'un raisonnement.}
\begin{enumerate}\item
Placer les points \( A=(0;5)\), \( B=(7;4)\), \( C=(-2;-6)\) et \( D=(3;-6)\) dans un repère orthonormé. Calculer la longueur du segment \( [AB]\) et les coordonnées du milieu du segment \( [CD]\).
\item
Est-ce que le triangle formé par les points \( A(9;-3)\), \( B(15;-9)\) et \( C(15;-3)\) est isocèle ?

\end{enumerate}
} 
\vspace{2cm}
\vbox{38
\emph{Toutes les réponses doivent être justifiées par un calcul accompagné d'un raisonnement.}
\begin{enumerate}\item
Placer les points \( A=(1;0)\), \( B=(6;6)\), \( C=(7;3)\) et \( D=(1;8)\) dans un repère orthonormé. Calculer la longueur du segment \( [AB]\) et les coordonnées du milieu du segment \( [CD]\).
\item
Est-ce que le triangle formé par les points \( A(0;-3)\), \( B(-3;0)\) et \( C(-2;-5)\) est rectangle ?

\end{enumerate}
} 
\vspace{2cm}
\vbox{39
\emph{Toutes les réponses doivent être justifiées par un calcul accompagné d'un raisonnement.}
\begin{enumerate}\item
Placer les points \( A=(-4;-4)\), \( B=(0;-3)\), \( C=(-6;1)\) et \( D=(-10;9)\) dans un repère orthonormé. Calculer la longueur du segment \( [AB]\) et les coordonnées du milieu du segment \( [CD]\).
\item
Est-ce que le triangle formé par les points \( A(0;-6)\), \( B(2;-8)\) et \( C(5;-3)\) est isocèle ?

\end{enumerate}
} 
\vspace{2cm}
\vbox{40
\emph{Toutes les réponses doivent être justifiées par un calcul accompagné d'un raisonnement.}
\begin{enumerate}\item
Placer les points \( A=(6;-6)\), \( B=(6;-10)\), \( C=(8;-6)\) et \( D=(2;5)\) dans un repère orthonormé. Calculer la longueur du segment \( [AB]\) et les coordonnées du milieu du segment \( [CD]\).
\item
Est-ce que le triangle formé par les points \( A(10;3)\), \( B(8;5)\) et \( C(7;2)\) est isocèle ?

\end{enumerate}
} 
\vspace{2cm}
\vbox{41
\emph{Toutes les réponses doivent être justifiées par un calcul accompagné d'un raisonnement.}
\begin{enumerate}\item
Placer les points \( A=(0;0)\), \( B=(6;-2)\), \( C=(-4;-3)\) et \( D=(8;7)\) dans un repère orthonormé. Calculer la longueur du segment \( [AB]\) et les coordonnées du milieu du segment \( [CD]\).
\item
Est-ce que le triangle formé par les points \( A(0;-5)\), \( B(1;-6)\) et \( C(9;-6)\) est rectangle ?

\end{enumerate}
} 
\vspace{2cm}
\vbox{42
\emph{Toutes les réponses doivent être justifiées par un calcul accompagné d'un raisonnement.}
\begin{enumerate}\item
Placer les points \( A=(-9;-8)\), \( B=(-5;8)\), \( C=(3;8)\) et \( D=(-4;5)\) dans un repère orthonormé. Calculer la longueur du segment \( [AB]\) et les coordonnées du milieu du segment \( [CD]\).
\item
Est-ce que le triangle formé par les points \( A(2;-4)\), \( B(0;-2)\) et \( C(0;-4)\) est isocèle ?

\end{enumerate}
} 
\vspace{2cm}
\vbox{43
\emph{Toutes les réponses doivent être justifiées par un calcul accompagné d'un raisonnement.}
\begin{enumerate}\item
Placer les points \( A=(-7;-4)\), \( B=(7;-1)\), \( C=(6;-2)\) et \( D=(-9;2)\) dans un repère orthonormé. Calculer la longueur du segment \( [AB]\) et les coordonnées du milieu du segment \( [CD]\).
\item
Est-ce que le triangle formé par les points \( A(1;-6)\), \( B(6;-11)\) et \( C(3;-4)\) est rectangle ?

\end{enumerate}
} 
\vspace{2cm}
\vbox{44
\emph{Toutes les réponses doivent être justifiées par un calcul accompagné d'un raisonnement.}
\begin{enumerate}\item
Placer les points \( A=(6;-3)\), \( B=(-9;-8)\), \( C=(8;-9)\) et \( D=(-6;-7)\) dans un repère orthonormé. Calculer la longueur du segment \( [AB]\) et les coordonnées du milieu du segment \( [CD]\).
\item
Est-ce que le triangle formé par les points \( A(8;1)\), \( B(13;-4)\) et \( C(21;4)\) est rectangle ?

\end{enumerate}
} 
\vspace{2cm}
\vbox{45
\emph{Toutes les réponses doivent être justifiées par un calcul accompagné d'un raisonnement.}
\begin{enumerate}\item
Placer les points \( A=(7;1)\), \( B=(9;5)\), \( C=(-8;6)\) et \( D=(-3;-5)\) dans un repère orthonormé. Calculer la longueur du segment \( [AB]\) et les coordonnées du milieu du segment \( [CD]\).
\item
Est-ce que le triangle formé par les points \( A(-6;-1)\), \( B(-2;-5)\) et \( C(7;2)\) est rectangle ?

\end{enumerate}
} 
\vspace{2cm}
\vbox{46
\emph{Toutes les réponses doivent être justifiées par un calcul accompagné d'un raisonnement.}
\begin{enumerate}\item
Placer les points \( A=(8;9)\), \( B=(5;-9)\), \( C=(-9;-5)\) et \( D=(-1;7)\) dans un repère orthonormé. Calculer la longueur du segment \( [AB]\) et les coordonnées du milieu du segment \( [CD]\).
\item
Est-ce que le triangle formé par les points \( A(-3;8)\), \( B(-5;10)\) et \( C(9;12)\) est isocèle ?

\end{enumerate}
} 
\vspace{2cm}
\vbox{47
\emph{Toutes les réponses doivent être justifiées par un calcul accompagné d'un raisonnement.}
\begin{enumerate}\item
Placer les points \( A=(-8;-2)\), \( B=(10;-8)\), \( C=(0;-6)\) et \( D=(8;4)\) dans un repère orthonormé. Calculer la longueur du segment \( [AB]\) et les coordonnées du milieu du segment \( [CD]\).
\item
Est-ce que le triangle formé par les points \( A(10;-10)\), \( B(12;-12)\) et \( C(9;-11)\) est rectangle ?

\end{enumerate}
} 
\vspace{2cm}
\vbox{48
\emph{Toutes les réponses doivent être justifiées par un calcul accompagné d'un raisonnement.}
\begin{enumerate}\item
Placer les points \( A=(4;-1)\), \( B=(-1;-8)\), \( C=(2;6)\) et \( D=(-5;-7)\) dans un repère orthonormé. Calculer la longueur du segment \( [AB]\) et les coordonnées du milieu du segment \( [CD]\).
\item
Est-ce que le triangle formé par les points \( A(-3;-2)\), \( B(-9;4)\) et \( C(0;-3)\) est isocèle ?

\end{enumerate}
} 
\vspace{2cm}
\vbox{49
\emph{Toutes les réponses doivent être justifiées par un calcul accompagné d'un raisonnement.}
\begin{enumerate}\item
Placer les points \( A=(9;9)\), \( B=(10;-3)\), \( C=(9;0)\) et \( D=(9;-9)\) dans un repère orthonormé. Calculer la longueur du segment \( [AB]\) et les coordonnées du milieu du segment \( [CD]\).
\item
Est-ce que le triangle formé par les points \( A(3;-9)\), \( B(7;-13)\) et \( C(2;-10)\) est rectangle ?

\end{enumerate}
} 
\vspace{2cm}
\vbox{50
\emph{Toutes les réponses doivent être justifiées par un calcul accompagné d'un raisonnement.}
\begin{enumerate}\item
Placer les points \( A=(5;7)\), \( B=(-7;0)\), \( C=(-7;1)\) et \( D=(0;9)\) dans un repère orthonormé. Calculer la longueur du segment \( [AB]\) et les coordonnées du milieu du segment \( [CD]\).
\item
Est-ce que le triangle formé par les points \( A(-6;-3)\), \( B(-11;2)\) et \( C(-4;-1)\) est rectangle ?

\end{enumerate}
} 
\vspace{2cm}
\vbox{51
\emph{Toutes les réponses doivent être justifiées par un calcul accompagné d'un raisonnement.}
\begin{enumerate}\item
Placer les points \( A=(-6;1)\), \( B=(-10;-3)\), \( C=(8;6)\) et \( D=(-8;2)\) dans un repère orthonormé. Calculer la longueur du segment \( [AB]\) et les coordonnées du milieu du segment \( [CD]\).
\item
Est-ce que le triangle formé par les points \( A(-7;7)\), \( B(-6;6)\) et \( C(-4;10)\) est rectangle ?

\end{enumerate}
} 
\vspace{2cm}
\vbox{52
\emph{Toutes les réponses doivent être justifiées par un calcul accompagné d'un raisonnement.}
\begin{enumerate}\item
Placer les points \( A=(3;-2)\), \( B=(5;8)\), \( C=(-3;10)\) et \( D=(4;4)\) dans un repère orthonormé. Calculer la longueur du segment \( [AB]\) et les coordonnées du milieu du segment \( [CD]\).
\item
Est-ce que le triangle formé par les points \( A(-4;8)\), \( B(-8;12)\) et \( C(-5;11)\) est isocèle ?

\end{enumerate}
} 
\vspace{2cm}
\vbox{53
\emph{Toutes les réponses doivent être justifiées par un calcul accompagné d'un raisonnement.}
\begin{enumerate}\item
Placer les points \( A=(10;-2)\), \( B=(-10;-4)\), \( C=(6;1)\) et \( D=(5;-4)\) dans un repère orthonormé. Calculer la longueur du segment \( [AB]\) et les coordonnées du milieu du segment \( [CD]\).
\item
Est-ce que le triangle formé par les points \( A(-7;7)\), \( B(-3;3)\) et \( C(1;1)\) est isocèle ?

\end{enumerate}
} 
\vspace{2cm}
\vbox{54
\emph{Toutes les réponses doivent être justifiées par un calcul accompagné d'un raisonnement.}
\begin{enumerate}\item
Placer les points \( A=(-3;9)\), \( B=(5;-6)\), \( C=(-9;-8)\) et \( D=(-10;-4)\) dans un repère orthonormé. Calculer la longueur du segment \( [AB]\) et les coordonnées du milieu du segment \( [CD]\).
\item
Est-ce que le triangle formé par les points \( A(2;6)\), \( B(1;7)\) et \( C(14;8)\) est rectangle ?

\end{enumerate}
} 
\vspace{2cm}
\vbox{55
\emph{Toutes les réponses doivent être justifiées par un calcul accompagné d'un raisonnement.}
\begin{enumerate}\item
Placer les points \( A=(-9;-9)\), \( B=(-6;-2)\), \( C=(-1;-4)\) et \( D=(7;10)\) dans un repère orthonormé. Calculer la longueur du segment \( [AB]\) et les coordonnées du milieu du segment \( [CD]\).
\item
Est-ce que le triangle formé par les points \( A(1;10)\), \( B(2;9)\) et \( C(8;7)\) est rectangle ?

\end{enumerate}
} 
\vspace{2cm}
\vbox{56
\emph{Toutes les réponses doivent être justifiées par un calcul accompagné d'un raisonnement.}
\begin{enumerate}\item
Placer les points \( A=(6;-6)\), \( B=(9;-8)\), \( C=(9;7)\) et \( D=(-9;4)\) dans un repère orthonormé. Calculer la longueur du segment \( [AB]\) et les coordonnées du milieu du segment \( [CD]\).
\item
Est-ce que le triangle formé par les points \( A(-2;7)\), \( B(2;3)\) et \( C(3;8)\) est isocèle ?

\end{enumerate}
} 
\vspace{2cm}
\vbox{57
\emph{Toutes les réponses doivent être justifiées par un calcul accompagné d'un raisonnement.}
\begin{enumerate}\item
Placer les points \( A=(-9;4)\), \( B=(-2;9)\), \( C=(8;0)\) et \( D=(-10;-2)\) dans un repère orthonormé. Calculer la longueur du segment \( [AB]\) et les coordonnées du milieu du segment \( [CD]\).
\item
Est-ce que le triangle formé par les points \( A(-4;7)\), \( B(-8;11)\) et \( C(3;8)\) est isocèle ?

\end{enumerate}
} 
\vspace{2cm}
\vbox{58
\emph{Toutes les réponses doivent être justifiées par un calcul accompagné d'un raisonnement.}
\begin{enumerate}\item
Placer les points \( A=(-1;2)\), \( B=(-7;-5)\), \( C=(1;1)\) et \( D=(-10;7)\) dans un repère orthonormé. Calculer la longueur du segment \( [AB]\) et les coordonnées du milieu du segment \( [CD]\).
\item
Est-ce que le triangle formé par les points \( A(-4;2)\), \( B(-2;0)\) et \( C(9;5)\) est rectangle ?

\end{enumerate}
} 
\vspace{2cm}
\vbox{59
\emph{Toutes les réponses doivent être justifiées par un calcul accompagné d'un raisonnement.}
\begin{enumerate}\item
Placer les points \( A=(6;3)\), \( B=(7;-7)\), \( C=(10;-1)\) et \( D=(3;0)\) dans un repère orthonormé. Calculer la longueur du segment \( [AB]\) et les coordonnées du milieu du segment \( [CD]\).
\item
Est-ce que le triangle formé par les points \( A(-10;10)\), \( B(-6;6)\) et \( C(6;12)\) est isocèle ?

\end{enumerate}
} 
\vspace{2cm}
\vbox{60
\emph{Toutes les réponses doivent être justifiées par un calcul accompagné d'un raisonnement.}
\begin{enumerate}\item
Placer les points \( A=(3;-7)\), \( B=(-9;7)\), \( C=(-8;-5)\) et \( D=(-7;-5)\) dans un repère orthonormé. Calculer la longueur du segment \( [AB]\) et les coordonnées du milieu du segment \( [CD]\).
\item
Est-ce que le triangle formé par les points \( A(5;7)\), \( B(4;8)\) et \( C(12;4)\) est rectangle ?

\end{enumerate}
} 
\vspace{2cm}
\vbox{61
\emph{Toutes les réponses doivent être justifiées par un calcul accompagné d'un raisonnement.}
\begin{enumerate}\item
Placer les points \( A=(10;-5)\), \( B=(-5;-9)\), \( C=(4;2)\) et \( D=(-4;-3)\) dans un repère orthonormé. Calculer la longueur du segment \( [AB]\) et les coordonnées du milieu du segment \( [CD]\).
\item
Est-ce que le triangle formé par les points \( A(2;3)\), \( B(-2;7)\) et \( C(-1;4)\) est isocèle ?

\end{enumerate}
} 
\vspace{2cm}
\vbox{62
\emph{Toutes les réponses doivent être justifiées par un calcul accompagné d'un raisonnement.}
\begin{enumerate}\item
Placer les points \( A=(-6;1)\), \( B=(-1;-3)\), \( C=(-8;-3)\) et \( D=(8;-6)\) dans un repère orthonormé. Calculer la longueur du segment \( [AB]\) et les coordonnées du milieu du segment \( [CD]\).
\item
Est-ce que le triangle formé par les points \( A(-3;7)\), \( B(-5;9)\) et \( C(10;10)\) est rectangle ?

\end{enumerate}
} 
\vspace{2cm}
\vbox{63
\emph{Toutes les réponses doivent être justifiées par un calcul accompagné d'un raisonnement.}
\begin{enumerate}\item
Placer les points \( A=(-7;-4)\), \( B=(-9;-8)\), \( C=(-5;2)\) et \( D=(-3;-1)\) dans un repère orthonormé. Calculer la longueur du segment \( [AB]\) et les coordonnées du milieu du segment \( [CD]\).
\item
Est-ce que le triangle formé par les points \( A(-1;10)\), \( B(1;8)\) et \( C(9;8)\) est isocèle ?

\end{enumerate}
} 
\vspace{2cm}
\vbox{64
\emph{Toutes les réponses doivent être justifiées par un calcul accompagné d'un raisonnement.}
\begin{enumerate}\item
Placer les points \( A=(-9;-6)\), \( B=(10;-3)\), \( C=(7;-10)\) et \( D=(7;-1)\) dans un repère orthonormé. Calculer la longueur du segment \( [AB]\) et les coordonnées du milieu du segment \( [CD]\).
\item
Est-ce que le triangle formé par les points \( A(1;-4)\), \( B(5;-8)\) et \( C(10;-5)\) est rectangle ?

\end{enumerate}
} 
\vspace{2cm}
\vbox{65
\emph{Toutes les réponses doivent être justifiées par un calcul accompagné d'un raisonnement.}
\begin{enumerate}\item
Placer les points \( A=(-3;9)\), \( B=(7;6)\), \( C=(-4;3)\) et \( D=(1;-1)\) dans un repère orthonormé. Calculer la longueur du segment \( [AB]\) et les coordonnées du milieu du segment \( [CD]\).
\item
Est-ce que le triangle formé par les points \( A(5;-7)\), \( B(3;-5)\) et \( C(13;-7)\) est isocèle ?

\end{enumerate}
} 
\vspace{2cm}
\vbox{66
\emph{Toutes les réponses doivent être justifiées par un calcul accompagné d'un raisonnement.}
\begin{enumerate}\item
Placer les points \( A=(-8;-7)\), \( B=(0;1)\), \( C=(8;2)\) et \( D=(-8;8)\) dans un repère orthonormé. Calculer la longueur du segment \( [AB]\) et les coordonnées du milieu du segment \( [CD]\).
\item
Est-ce que le triangle formé par les points \( A(-1;-7)\), \( B(3;-11)\) et \( C(12;-4)\) est rectangle ?

\end{enumerate}
} 
\vspace{2cm}
\vbox{67
\emph{Toutes les réponses doivent être justifiées par un calcul accompagné d'un raisonnement.}
\begin{enumerate}\item
Placer les points \( A=(3;-4)\), \( B=(5;-2)\), \( C=(-1;9)\) et \( D=(5;-6)\) dans un repère orthonormé. Calculer la longueur du segment \( [AB]\) et les coordonnées du milieu du segment \( [CD]\).
\item
Est-ce que le triangle formé par les points \( A(-7;4)\), \( B(-10;7)\) et \( C(1;2)\) est rectangle ?

\end{enumerate}
} 
\vspace{2cm}
\vbox{68
\emph{Toutes les réponses doivent être justifiées par un calcul accompagné d'un raisonnement.}
\begin{enumerate}\item
Placer les points \( A=(9;2)\), \( B=(-6;6)\), \( C=(-1;-8)\) et \( D=(-7;9)\) dans un repère orthonormé. Calculer la longueur du segment \( [AB]\) et les coordonnées du milieu du segment \( [CD]\).
\item
Est-ce que le triangle formé par les points \( A(5;-6)\), \( B(9;-10)\) et \( C(20;-5)\) est isocèle ?

\end{enumerate}
} 
\vspace{2cm}
\vbox{69
\emph{Toutes les réponses doivent être justifiées par un calcul accompagné d'un raisonnement.}
\begin{enumerate}\item
Placer les points \( A=(-1;2)\), \( B=(0;-4)\), \( C=(-9;-4)\) et \( D=(8;5)\) dans un repère orthonormé. Calculer la longueur du segment \( [AB]\) et les coordonnées du milieu du segment \( [CD]\).
\item
Est-ce que le triangle formé par les points \( A(7;-9)\), \( B(13;-15)\) et \( C(9;-13)\) est isocèle ?

\end{enumerate}
} 
\vspace{2cm}
\vbox{70
\emph{Toutes les réponses doivent être justifiées par un calcul accompagné d'un raisonnement.}
\begin{enumerate}\item
Placer les points \( A=(-1;0)\), \( B=(-1;2)\), \( C=(-2;5)\) et \( D=(9;-8)\) dans un repère orthonormé. Calculer la longueur du segment \( [AB]\) et les coordonnées du milieu du segment \( [CD]\).
\item
Est-ce que le triangle formé par les points \( A(9;-2)\), \( B(6;1)\) et \( C(22;1)\) est rectangle ?

\end{enumerate}
} 
\vspace{2cm}
\vbox{71
\emph{Toutes les réponses doivent être justifiées par un calcul accompagné d'un raisonnement.}
\begin{enumerate}\item
Placer les points \( A=(5;-7)\), \( B=(8;6)\), \( C=(-8;9)\) et \( D=(-5;-6)\) dans un repère orthonormé. Calculer la longueur du segment \( [AB]\) et les coordonnées du milieu du segment \( [CD]\).
\item
Est-ce que le triangle formé par les points \( A(6;-4)\), \( B(9;-7)\) et \( C(17;-3)\) est rectangle ?

\end{enumerate}
} 
\vspace{2cm}
\vbox{72
\emph{Toutes les réponses doivent être justifiées par un calcul accompagné d'un raisonnement.}
\begin{enumerate}\item
Placer les points \( A=(1;-6)\), \( B=(-1;-1)\), \( C=(-4;10)\) et \( D=(-2;-1)\) dans un repère orthonormé. Calculer la longueur du segment \( [AB]\) et les coordonnées du milieu du segment \( [CD]\).
\item
Est-ce que le triangle formé par les points \( A(-5;9)\), \( B(-3;7)\) et \( C(0;12)\) est isocèle ?

\end{enumerate}
} 
\vspace{2cm}
\vbox{73
\emph{Toutes les réponses doivent être justifiées par un calcul accompagné d'un raisonnement.}
\begin{enumerate}\item
Placer les points \( A=(2;3)\), \( B=(0;4)\), \( C=(-6;-8)\) et \( D=(6;-4)\) dans un repère orthonormé. Calculer la longueur du segment \( [AB]\) et les coordonnées du milieu du segment \( [CD]\).
\item
Est-ce que le triangle formé par les points \( A(7;-9)\), \( B(6;-8)\) et \( C(20;-6)\) est rectangle ?

\end{enumerate}
} 
\vspace{2cm}
\vbox{74
\emph{Toutes les réponses doivent être justifiées par un calcul accompagné d'un raisonnement.}
\begin{enumerate}\item
Placer les points \( A=(-1;1)\), \( B=(1;7)\), \( C=(-7;-7)\) et \( D=(8;-2)\) dans un repère orthonormé. Calculer la longueur du segment \( [AB]\) et les coordonnées du milieu du segment \( [CD]\).
\item
Est-ce que le triangle formé par les points \( A(6;-6)\), \( B(10;-10)\) et \( C(20;-6)\) est isocèle ?

\end{enumerate}
} 
\vspace{2cm}
\vbox{75
\emph{Toutes les réponses doivent être justifiées par un calcul accompagné d'un raisonnement.}
\begin{enumerate}\item
Placer les points \( A=(2;-7)\), \( B=(-1;-9)\), \( C=(-4;10)\) et \( D=(4;-4)\) dans un repère orthonormé. Calculer la longueur du segment \( [AB]\) et les coordonnées du milieu du segment \( [CD]\).
\item
Est-ce que le triangle formé par les points \( A(10;0)\), \( B(4;6)\) et \( C(13;-1)\) est isocèle ?

\end{enumerate}
} 
\vspace{2cm}
\vbox{76
\emph{Toutes les réponses doivent être justifiées par un calcul accompagné d'un raisonnement.}
\begin{enumerate}\item
Placer les points \( A=(-2;-8)\), \( B=(-10;-5)\), \( C=(3;-10)\) et \( D=(-8;7)\) dans un repère orthonormé. Calculer la longueur du segment \( [AB]\) et les coordonnées du milieu du segment \( [CD]\).
\item
Est-ce que le triangle formé par les points \( A(6;-5)\), \( B(10;-9)\) et \( C(11;-4)\) est isocèle ?

\end{enumerate}
} 
\vspace{2cm}
\vbox{77
\emph{Toutes les réponses doivent être justifiées par un calcul accompagné d'un raisonnement.}
\begin{enumerate}\item
Placer les points \( A=(9;-1)\), \( B=(-10;-1)\), \( C=(-3;4)\) et \( D=(6;4)\) dans un repère orthonormé. Calculer la longueur du segment \( [AB]\) et les coordonnées du milieu du segment \( [CD]\).
\item
Est-ce que le triangle formé par les points \( A(0;-9)\), \( B(4;-13)\) et \( C(1;-12)\) est isocèle ?

\end{enumerate}
} 
\vspace{2cm}
\vbox{78
\emph{Toutes les réponses doivent être justifiées par un calcul accompagné d'un raisonnement.}
\begin{enumerate}\item
Placer les points \( A=(-7;-3)\), \( B=(7;-5)\), \( C=(3;-6)\) et \( D=(0;-3)\) dans un repère orthonormé. Calculer la longueur du segment \( [AB]\) et les coordonnées du milieu du segment \( [CD]\).
\item
Est-ce que le triangle formé par les points \( A(6;-9)\), \( B(9;-12)\) et \( C(4;-11)\) est rectangle ?

\end{enumerate}
} 
\vspace{2cm}
\vbox{79
\emph{Toutes les réponses doivent être justifiées par un calcul accompagné d'un raisonnement.}
\begin{enumerate}\item
Placer les points \( A=(0;-8)\), \( B=(8;0)\), \( C=(-8;0)\) et \( D=(-7;6)\) dans un repère orthonormé. Calculer la longueur du segment \( [AB]\) et les coordonnées du milieu du segment \( [CD]\).
\item
Est-ce que le triangle formé par les points \( A(10;8)\), \( B(6;12)\) et \( C(17;5)\) est rectangle ?

\end{enumerate}
} 
\vspace{2cm}
\vbox{80
\emph{Toutes les réponses doivent être justifiées par un calcul accompagné d'un raisonnement.}
\begin{enumerate}\item
Placer les points \( A=(-2;2)\), \( B=(-3;-1)\), \( C=(-9;4)\) et \( D=(-1;5)\) dans un repère orthonormé. Calculer la longueur du segment \( [AB]\) et les coordonnées du milieu du segment \( [CD]\).
\item
Est-ce que le triangle formé par les points \( A(9;-10)\), \( B(5;-6)\) et \( C(18;-7)\) est isocèle ?

\end{enumerate}
} 
\vspace{2cm}
\vbox{81
\emph{Toutes les réponses doivent être justifiées par un calcul accompagné d'un raisonnement.}
\begin{enumerate}\item
Placer les points \( A=(10;3)\), \( B=(10;-8)\), \( C=(-2;2)\) et \( D=(-9;7)\) dans un repère orthonormé. Calculer la longueur du segment \( [AB]\) et les coordonnées du milieu du segment \( [CD]\).
\item
Est-ce que le triangle formé par les points \( A(-10;7)\), \( B(-12;9)\) et \( C(-12;5)\) est rectangle ?

\end{enumerate}
} 
\vspace{2cm}
\vbox{82
\emph{Toutes les réponses doivent être justifiées par un calcul accompagné d'un raisonnement.}
\begin{enumerate}\item
Placer les points \( A=(-8;-6)\), \( B=(-9;-8)\), \( C=(-6;8)\) et \( D=(6;0)\) dans un repère orthonormé. Calculer la longueur du segment \( [AB]\) et les coordonnées du milieu du segment \( [CD]\).
\item
Est-ce que le triangle formé par les points \( A(6;-9)\), \( B(3;-6)\) et \( C(7;-8)\) est rectangle ?

\end{enumerate}
} 
\vspace{2cm}
\vbox{83
\emph{Toutes les réponses doivent être justifiées par un calcul accompagné d'un raisonnement.}
\begin{enumerate}\item
Placer les points \( A=(0;-9)\), \( B=(9;-1)\), \( C=(2;8)\) et \( D=(10;-8)\) dans un repère orthonormé. Calculer la longueur du segment \( [AB]\) et les coordonnées du milieu du segment \( [CD]\).
\item
Est-ce que le triangle formé par les points \( A(2;4)\), \( B(4;2)\) et \( C(9;-1)\) est isocèle ?

\end{enumerate}
} 
\vspace{2cm}
\vbox{84
\emph{Toutes les réponses doivent être justifiées par un calcul accompagné d'un raisonnement.}
\begin{enumerate}\item
Placer les points \( A=(1;-4)\), \( B=(7;-5)\), \( C=(-4;-4)\) et \( D=(-8;4)\) dans un repère orthonormé. Calculer la longueur du segment \( [AB]\) et les coordonnées du milieu du segment \( [CD]\).
\item
Est-ce que le triangle formé par les points \( A(-10;3)\), \( B(-4;-3)\) et \( C(-6;1)\) est isocèle ?

\end{enumerate}
} 
\vspace{2cm}


-------------------------

\vbox{1
\emph{Toutes les réponses doivent être justifiées par un calcul accompagné d'un raisonnement.}
\begin{enumerate}\item
Placer les points \( A=(6;-1)\), \( B=(7;2)\), \( C=(1;0)\) et \( D=(-10;10)\) dans un repère orthonormé. Calculer la longueur du segment \( [AB]\) et les coordonnées du milieu du segment \( [CD]\).

 $l^2=10$,$l=sqrt(10)$,$M=(-4.5,5.0)$\item
Est-ce que le triangle formé par les points \( A(8;8)\), \( B(9;7)\) et \( C(17;7)\) est rectangle ?

False
\end{enumerate}
} 
\vspace{2cm}
\vbox{2
\emph{Toutes les réponses doivent être justifiées par un calcul accompagné d'un raisonnement.}
\begin{enumerate}\item
Placer les points \( A=(9;-5)\), \( B=(9;7)\), \( C=(-9;4)\) et \( D=(1;-3)\) dans un repère orthonormé. Calculer la longueur du segment \( [AB]\) et les coordonnées du milieu du segment \( [CD]\).

 $l^2=144$,$l=12$,$M=(-4.0,0.5)$\item
Est-ce que le triangle formé par les points \( A(-8;7)\), \( B(-2;1)\) et \( C(-9;0)\) est isocèle ?

True
\end{enumerate}
} 
\vspace{2cm}
\vbox{3
\emph{Toutes les réponses doivent être justifiées par un calcul accompagné d'un raisonnement.}
\begin{enumerate}\item
Placer les points \( A=(-5;-9)\), \( B=(-3;-1)\), \( C=(8;5)\) et \( D=(-1;-8)\) dans un repère orthonormé. Calculer la longueur du segment \( [AB]\) et les coordonnées du milieu du segment \( [CD]\).

 $l^2=68$,$l=2*sqrt(17)$,$M=(3.5,-1.5)$\item
Est-ce que le triangle formé par les points \( A(-7;-3)\), \( B(-3;-7)\) et \( C(1;-5)\) est rectangle ?

False
\end{enumerate}
} 
\vspace{2cm}
\vbox{4
\emph{Toutes les réponses doivent être justifiées par un calcul accompagné d'un raisonnement.}
\begin{enumerate}\item
Placer les points \( A=(-10;7)\), \( B=(4;3)\), \( C=(0;3)\) et \( D=(10;-1)\) dans un repère orthonormé. Calculer la longueur du segment \( [AB]\) et les coordonnées du milieu du segment \( [CD]\).

 $l^2=212$,$l=2*sqrt(53)$,$M=(5.0,1.0)$\item
Est-ce que le triangle formé par les points \( A(-8;-5)\), \( B(-3;-10)\) et \( C(3;-4)\) est rectangle ?

False
\end{enumerate}
} 
\vspace{2cm}
\vbox{5
\emph{Toutes les réponses doivent être justifiées par un calcul accompagné d'un raisonnement.}
\begin{enumerate}\item
Placer les points \( A=(-4;2)\), \( B=(4;7)\), \( C=(9;8)\) et \( D=(9;0)\) dans un repère orthonormé. Calculer la longueur du segment \( [AB]\) et les coordonnées du milieu du segment \( [CD]\).

 $l^2=89$,$l=sqrt(89)$,$M=(9.0,4.0)$\item
Est-ce que le triangle formé par les points \( A(0;-6)\), \( B(2;-8)\) et \( C(1;-7)\) est isocèle ?

True
\end{enumerate}
} 
\vspace{2cm}
\vbox{6
\emph{Toutes les réponses doivent être justifiées par un calcul accompagné d'un raisonnement.}
\begin{enumerate}\item
Placer les points \( A=(-10;7)\), \( B=(6;-2)\), \( C=(9;-3)\) et \( D=(7;-3)\) dans un repère orthonormé. Calculer la longueur du segment \( [AB]\) et les coordonnées du milieu du segment \( [CD]\).

 $l^2=337$,$l=sqrt(337)$,$M=(8.0,-3.0)$\item
Est-ce que le triangle formé par les points \( A(0;3)\), \( B(4;-1)\) et \( C(13;2)\) est isocèle ?

False
\end{enumerate}
} 
\vspace{2cm}
\vbox{7
\emph{Toutes les réponses doivent être justifiées par un calcul accompagné d'un raisonnement.}
\begin{enumerate}\item
Placer les points \( A=(-4;1)\), \( B=(4;-6)\), \( C=(-9;-3)\) et \( D=(5;-4)\) dans un repère orthonormé. Calculer la longueur du segment \( [AB]\) et les coordonnées du milieu du segment \( [CD]\).

 $l^2=113$,$l=sqrt(113)$,$M=(-2.0,-3.5)$\item
Est-ce que le triangle formé par les points \( A(-3;-10)\), \( B(-7;-6)\) et \( C(10;-7)\) est rectangle ?

False
\end{enumerate}
} 
\vspace{2cm}
\vbox{8
\emph{Toutes les réponses doivent être justifiées par un calcul accompagné d'un raisonnement.}
\begin{enumerate}\item
Placer les points \( A=(9;1)\), \( B=(-6;-8)\), \( C=(-1;-7)\) et \( D=(-5;-1)\) dans un repère orthonormé. Calculer la longueur du segment \( [AB]\) et les coordonnées du milieu du segment \( [CD]\).

 $l^2=306$,$l=3*sqrt(34)$,$M=(-3.0,-4.0)$\item
Est-ce que le triangle formé par les points \( A(1;6)\), \( B(-5;12)\) et \( C(9;10)\) est isocèle ?

False
\end{enumerate}
} 
\vspace{2cm}
\vbox{9
\emph{Toutes les réponses doivent être justifiées par un calcul accompagné d'un raisonnement.}
\begin{enumerate}\item
Placer les points \( A=(-6;-4)\), \( B=(6;10)\), \( C=(0;9)\) et \( D=(-6;8)\) dans un repère orthonormé. Calculer la longueur du segment \( [AB]\) et les coordonnées du milieu du segment \( [CD]\).

 $l^2=340$,$l=2*sqrt(85)$,$M=(-3.0,8.5)$\item
Est-ce que le triangle formé par les points \( A(1;3)\), \( B(3;1)\) et \( C(8;0)\) est rectangle ?

False
\end{enumerate}
} 
\vspace{2cm}
\vbox{10
\emph{Toutes les réponses doivent être justifiées par un calcul accompagné d'un raisonnement.}
\begin{enumerate}\item
Placer les points \( A=(10;-5)\), \( B=(-1;-4)\), \( C=(6;2)\) et \( D=(6;10)\) dans un repère orthonormé. Calculer la longueur du segment \( [AB]\) et les coordonnées du milieu du segment \( [CD]\).

 $l^2=122$,$l=sqrt(122)$,$M=(6.0,6.0)$\item
Est-ce que le triangle formé par les points \( A(-10;-10)\), \( B(-7;-13)\) et \( C(0;-10)\) est rectangle ?

False
\end{enumerate}
} 
\vspace{2cm}
\vbox{11
\emph{Toutes les réponses doivent être justifiées par un calcul accompagné d'un raisonnement.}
\begin{enumerate}\item
Placer les points \( A=(-3;-5)\), \( B=(-6;-2)\), \( C=(-7;-10)\) et \( D=(-8;0)\) dans un repère orthonormé. Calculer la longueur du segment \( [AB]\) et les coordonnées du milieu du segment \( [CD]\).

 $l^2=18$,$l=3*sqrt(2)$,$M=(-7.5,-5.0)$\item
Est-ce que le triangle formé par les points \( A(6;-5)\), \( B(6;-5)\) et \( C(13;-8)\) est rectangle ?

False
\end{enumerate}
} 
\vspace{2cm}
\vbox{12
\emph{Toutes les réponses doivent être justifiées par un calcul accompagné d'un raisonnement.}
\begin{enumerate}\item
Placer les points \( A=(2;-1)\), \( B=(-8;3)\), \( C=(5;-8)\) et \( D=(6;6)\) dans un repère orthonormé. Calculer la longueur du segment \( [AB]\) et les coordonnées du milieu du segment \( [CD]\).

 $l^2=116$,$l=2*sqrt(29)$,$M=(5.5,-1.0)$\item
Est-ce que le triangle formé par les points \( A(-9;3)\), \( B(-7;1)\) et \( C(-11;1)\) est rectangle ?

True
\end{enumerate}
} 
\vspace{2cm}
\vbox{13
\emph{Toutes les réponses doivent être justifiées par un calcul accompagné d'un raisonnement.}
\begin{enumerate}\item
Placer les points \( A=(7;2)\), \( B=(-8;-7)\), \( C=(-6;-2)\) et \( D=(-8;-8)\) dans un repère orthonormé. Calculer la longueur du segment \( [AB]\) et les coordonnées du milieu du segment \( [CD]\).

 $l^2=306$,$l=3*sqrt(34)$,$M=(-7.0,-5.0)$\item
Est-ce que le triangle formé par les points \( A(-6;-1)\), \( B(-8;1)\) et \( C(4;1)\) est isocèle ?

False
\end{enumerate}
} 
\vspace{2cm}
\vbox{14
\emph{Toutes les réponses doivent être justifiées par un calcul accompagné d'un raisonnement.}
\begin{enumerate}\item
Placer les points \( A=(-2;-5)\), \( B=(6;9)\), \( C=(8;-9)\) et \( D=(-10;4)\) dans un repère orthonormé. Calculer la longueur du segment \( [AB]\) et les coordonnées du milieu du segment \( [CD]\).

 $l^2=260$,$l=2*sqrt(65)$,$M=(-1.0,-2.5)$\item
Est-ce que le triangle formé par les points \( A(-10;7)\), \( B(-6;3)\) et \( C(4;7)\) est isocèle ?

False
\end{enumerate}
} 
\vspace{2cm}
\vbox{15
\emph{Toutes les réponses doivent être justifiées par un calcul accompagné d'un raisonnement.}
\begin{enumerate}\item
Placer les points \( A=(8;-4)\), \( B=(-7;-5)\), \( C=(-2;-10)\) et \( D=(3;-9)\) dans un repère orthonormé. Calculer la longueur du segment \( [AB]\) et les coordonnées du milieu du segment \( [CD]\).

 $l^2=226$,$l=sqrt(226)$,$M=(0.5,-9.5)$\item
Est-ce que le triangle formé par les points \( A(9;-10)\), \( B(5;-6)\) et \( C(12;-7)\) est rectangle ?

True
\end{enumerate}
} 
\vspace{2cm}
\vbox{16
\emph{Toutes les réponses doivent être justifiées par un calcul accompagné d'un raisonnement.}
\begin{enumerate}\item
Placer les points \( A=(8;-3)\), \( B=(9;-1)\), \( C=(2;0)\) et \( D=(-3;-7)\) dans un repère orthonormé. Calculer la longueur du segment \( [AB]\) et les coordonnées du milieu du segment \( [CD]\).

 $l^2=5$,$l=sqrt(5)$,$M=(-0.5,-3.5)$\item
Est-ce que le triangle formé par les points \( A(-3;-2)\), \( B(-1;-4)\) et \( C(1;0)\) est isocèle ?

True
\end{enumerate}
} 
\vspace{2cm}
\vbox{17
\emph{Toutes les réponses doivent être justifiées par un calcul accompagné d'un raisonnement.}
\begin{enumerate}\item
Placer les points \( A=(-7;3)\), \( B=(0;-1)\), \( C=(-6;-8)\) et \( D=(5;-4)\) dans un repère orthonormé. Calculer la longueur du segment \( [AB]\) et les coordonnées du milieu du segment \( [CD]\).

 $l^2=65$,$l=sqrt(65)$,$M=(-0.5,-6.0)$\item
Est-ce que le triangle formé par les points \( A(-4;8)\), \( B(0;4)\) et \( C(-4;8)\) est rectangle ?

True
\end{enumerate}
} 
\vspace{2cm}
\vbox{18
\emph{Toutes les réponses doivent être justifiées par un calcul accompagné d'un raisonnement.}
\begin{enumerate}\item
Placer les points \( A=(1;-3)\), \( B=(5;-9)\), \( C=(-2;-5)\) et \( D=(-6;-6)\) dans un repère orthonormé. Calculer la longueur du segment \( [AB]\) et les coordonnées du milieu du segment \( [CD]\).

 $l^2=52$,$l=2*sqrt(13)$,$M=(-4.0,-5.5)$\item
Est-ce que le triangle formé par les points \( A(4;-9)\), \( B(10;-15)\) et \( C(19;-10)\) est isocèle ?

False
\end{enumerate}
} 
\vspace{2cm}
\vbox{19
\emph{Toutes les réponses doivent être justifiées par un calcul accompagné d'un raisonnement.}
\begin{enumerate}\item
Placer les points \( A=(-3;-7)\), \( B=(-6;-5)\), \( C=(4;3)\) et \( D=(3;-10)\) dans un repère orthonormé. Calculer la longueur du segment \( [AB]\) et les coordonnées du milieu du segment \( [CD]\).

 $l^2=13$,$l=sqrt(13)$,$M=(3.5,-3.5)$\item
Est-ce que le triangle formé par les points \( A(8;0)\), \( B(2;6)\) et \( C(19;7)\) est isocèle ?

False
\end{enumerate}
} 
\vspace{2cm}
\vbox{20
\emph{Toutes les réponses doivent être justifiées par un calcul accompagné d'un raisonnement.}
\begin{enumerate}\item
Placer les points \( A=(10;5)\), \( B=(2;0)\), \( C=(-10;-5)\) et \( D=(-1;2)\) dans un repère orthonormé. Calculer la longueur du segment \( [AB]\) et les coordonnées du milieu du segment \( [CD]\).

 $l^2=89$,$l=sqrt(89)$,$M=(-5.5,-1.5)$\item
Est-ce que le triangle formé par les points \( A(2;5)\), \( B(0;7)\) et \( C(-3;2)\) est isocèle ?

True
\end{enumerate}
} 
\vspace{2cm}
\vbox{21
\emph{Toutes les réponses doivent être justifiées par un calcul accompagné d'un raisonnement.}
\begin{enumerate}\item
Placer les points \( A=(4;9)\), \( B=(-5;2)\), \( C=(0;6)\) et \( D=(-6;-5)\) dans un repère orthonormé. Calculer la longueur du segment \( [AB]\) et les coordonnées du milieu du segment \( [CD]\).

 $l^2=130$,$l=sqrt(130)$,$M=(-3.0,0.5)$\item
Est-ce que le triangle formé par les points \( A(-10;6)\), \( B(-15;11)\) et \( C(-3;3)\) est rectangle ?

False
\end{enumerate}
} 
\vspace{2cm}
\vbox{22
\emph{Toutes les réponses doivent être justifiées par un calcul accompagné d'un raisonnement.}
\begin{enumerate}\item
Placer les points \( A=(1;3)\), \( B=(6;0)\), \( C=(-10;-6)\) et \( D=(9;-10)\) dans un repère orthonormé. Calculer la longueur du segment \( [AB]\) et les coordonnées du milieu du segment \( [CD]\).

 $l^2=34$,$l=sqrt(34)$,$M=(-0.5,-8.0)$\item
Est-ce que le triangle formé par les points \( A(4;6)\), \( B(7;3)\) et \( C(3;5)\) est rectangle ?

True
\end{enumerate}
} 
\vspace{2cm}
\vbox{23
\emph{Toutes les réponses doivent être justifiées par un calcul accompagné d'un raisonnement.}
\begin{enumerate}\item
Placer les points \( A=(-6;5)\), \( B=(7;-4)\), \( C=(6;-2)\) et \( D=(7;-10)\) dans un repère orthonormé. Calculer la longueur du segment \( [AB]\) et les coordonnées du milieu du segment \( [CD]\).

 $l^2=250$,$l=5*sqrt(10)$,$M=(6.5,-6.0)$\item
Est-ce que le triangle formé par les points \( A(-7;-1)\), \( B(-12;4)\) et \( C(5;1)\) est rectangle ?

False
\end{enumerate}
} 
\vspace{2cm}
\vbox{24
\emph{Toutes les réponses doivent être justifiées par un calcul accompagné d'un raisonnement.}
\begin{enumerate}\item
Placer les points \( A=(9;-7)\), \( B=(-3;-6)\), \( C=(-10;7)\) et \( D=(-7;9)\) dans un repère orthonormé. Calculer la longueur du segment \( [AB]\) et les coordonnées du milieu du segment \( [CD]\).

 $l^2=145$,$l=sqrt(145)$,$M=(-8.5,8.0)$\item
Est-ce que le triangle formé par les points \( A(10;7)\), \( B(8;9)\) et \( C(10;9)\) est isocèle ?

True
\end{enumerate}
} 
\vspace{2cm}
\vbox{25
\emph{Toutes les réponses doivent être justifiées par un calcul accompagné d'un raisonnement.}
\begin{enumerate}\item
Placer les points \( A=(-4;-9)\), \( B=(7;1)\), \( C=(9;-9)\) et \( D=(1;3)\) dans un repère orthonormé. Calculer la longueur du segment \( [AB]\) et les coordonnées du milieu du segment \( [CD]\).

 $l^2=221$,$l=sqrt(221)$,$M=(5.0,-3.0)$\item
Est-ce que le triangle formé par les points \( A(-2;10)\), \( B(4;4)\) et \( C(4;10)\) est isocèle ?

True
\end{enumerate}
} 
\vspace{2cm}
\vbox{26
\emph{Toutes les réponses doivent être justifiées par un calcul accompagné d'un raisonnement.}
\begin{enumerate}\item
Placer les points \( A=(4;-1)\), \( B=(6;5)\), \( C=(3;10)\) et \( D=(5;8)\) dans un repère orthonormé. Calculer la longueur du segment \( [AB]\) et les coordonnées du milieu du segment \( [CD]\).

 $l^2=40$,$l=2*sqrt(10)$,$M=(4.0,9.0)$\item
Est-ce que le triangle formé par les points \( A(-5;-8)\), \( B(-9;-4)\) et \( C(-1;-10)\) est isocèle ?

False
\end{enumerate}
} 
\vspace{2cm}
\vbox{27
\emph{Toutes les réponses doivent être justifiées par un calcul accompagné d'un raisonnement.}
\begin{enumerate}\item
Placer les points \( A=(-4;5)\), \( B=(10;-10)\), \( C=(-8;6)\) et \( D=(6;-10)\) dans un repère orthonormé. Calculer la longueur du segment \( [AB]\) et les coordonnées du milieu du segment \( [CD]\).

 $l^2=421$,$l=sqrt(421)$,$M=(-1.0,-2.0)$\item
Est-ce que le triangle formé par les points \( A(3;4)\), \( B(-1;8)\) et \( C(8;3)\) est isocèle ?

False
\end{enumerate}
} 
\vspace{2cm}
\vbox{28
\emph{Toutes les réponses doivent être justifiées par un calcul accompagné d'un raisonnement.}
\begin{enumerate}\item
Placer les points \( A=(6;-4)\), \( B=(6;9)\), \( C=(8;9)\) et \( D=(8;-7)\) dans un repère orthonormé. Calculer la longueur du segment \( [AB]\) et les coordonnées du milieu du segment \( [CD]\).

 $l^2=169$,$l=13$,$M=(8.0,1.0)$\item
Est-ce que le triangle formé par les points \( A(-6;10)\), \( B(-4;8)\) et \( C(-6;8)\) est isocèle ?

True
\end{enumerate}
} 
\vspace{2cm}
\vbox{29
\emph{Toutes les réponses doivent être justifiées par un calcul accompagné d'un raisonnement.}
\begin{enumerate}\item
Placer les points \( A=(0;0)\), \( B=(5;10)\), \( C=(9;-10)\) et \( D=(-4;-8)\) dans un repère orthonormé. Calculer la longueur du segment \( [AB]\) et les coordonnées du milieu du segment \( [CD]\).

 $l^2=125$,$l=5*sqrt(5)$,$M=(2.5,-9.0)$\item
Est-ce que le triangle formé par les points \( A(0;-10)\), \( B(0;-10)\) et \( C(-3;-13)\) est isocèle ?

True
\end{enumerate}
} 
\vspace{2cm}
\vbox{30
\emph{Toutes les réponses doivent être justifiées par un calcul accompagné d'un raisonnement.}
\begin{enumerate}\item
Placer les points \( A=(-1;6)\), \( B=(0;10)\), \( C=(0;-8)\) et \( D=(1;-1)\) dans un repère orthonormé. Calculer la longueur du segment \( [AB]\) et les coordonnées du milieu du segment \( [CD]\).

 $l^2=17$,$l=sqrt(17)$,$M=(0.5,-4.5)$\item
Est-ce que le triangle formé par les points \( A(-6;1)\), \( B(-2;-3)\) et \( C(-2;1)\) est isocèle ?

True
\end{enumerate}
} 
\vspace{2cm}
\vbox{31
\emph{Toutes les réponses doivent être justifiées par un calcul accompagné d'un raisonnement.}
\begin{enumerate}\item
Placer les points \( A=(6;-3)\), \( B=(-5;-5)\), \( C=(-3;4)\) et \( D=(8;6)\) dans un repère orthonormé. Calculer la longueur du segment \( [AB]\) et les coordonnées du milieu du segment \( [CD]\).

 $l^2=125$,$l=5*sqrt(5)$,$M=(2.5,5.0)$\item
Est-ce que le triangle formé par les points \( A(-5;-1)\), \( B(-7;1)\) et \( C(0;-4)\) est isocèle ?

False
\end{enumerate}
} 
\vspace{2cm}
\vbox{32
\emph{Toutes les réponses doivent être justifiées par un calcul accompagné d'un raisonnement.}
\begin{enumerate}\item
Placer les points \( A=(-8;4)\), \( B=(1;-9)\), \( C=(-8;2)\) et \( D=(7;5)\) dans un repère orthonormé. Calculer la longueur du segment \( [AB]\) et les coordonnées du milieu du segment \( [CD]\).

 $l^2=250$,$l=5*sqrt(10)$,$M=(-0.5,3.5)$\item
Est-ce que le triangle formé par les points \( A(6;1)\), \( B(11;-4)\) et \( C(19;4)\) est rectangle ?

False
\end{enumerate}
} 
\vspace{2cm}
\vbox{33
\emph{Toutes les réponses doivent être justifiées par un calcul accompagné d'un raisonnement.}
\begin{enumerate}\item
Placer les points \( A=(-10;4)\), \( B=(0;-8)\), \( C=(5;-2)\) et \( D=(-6;-10)\) dans un repère orthonormé. Calculer la longueur du segment \( [AB]\) et les coordonnées du milieu du segment \( [CD]\).

 $l^2=244$,$l=2*sqrt(61)$,$M=(-0.5,-6.0)$\item
Est-ce que le triangle formé par les points \( A(-1;7)\), \( B(-2;8)\) et \( C(0;8)\) est rectangle ?

True
\end{enumerate}
} 
\vspace{2cm}
\vbox{34
\emph{Toutes les réponses doivent être justifiées par un calcul accompagné d'un raisonnement.}
\begin{enumerate}\item
Placer les points \( A=(-2;5)\), \( B=(-4;-4)\), \( C=(1;10)\) et \( D=(-5;-6)\) dans un repère orthonormé. Calculer la longueur du segment \( [AB]\) et les coordonnées du milieu du segment \( [CD]\).

 $l^2=85$,$l=sqrt(85)$,$M=(-2.0,2.0)$\item
Est-ce que le triangle formé par les points \( A(-1;-7)\), \( B(1;-9)\) et \( C(2;-6)\) est isocèle ?

True
\end{enumerate}
} 
\vspace{2cm}
\vbox{35
\emph{Toutes les réponses doivent être justifiées par un calcul accompagné d'un raisonnement.}
\begin{enumerate}\item
Placer les points \( A=(3;-1)\), \( B=(8;-8)\), \( C=(-3;2)\) et \( D=(-6;-2)\) dans un repère orthonormé. Calculer la longueur du segment \( [AB]\) et les coordonnées du milieu du segment \( [CD]\).

 $l^2=74$,$l=sqrt(74)$,$M=(-4.5,0.0)$\item
Est-ce que le triangle formé par les points \( A(5;6)\), \( B(3;8)\) et \( C(15;8)\) est isocèle ?

False
\end{enumerate}
} 
\vspace{2cm}
\vbox{36
\emph{Toutes les réponses doivent être justifiées par un calcul accompagné d'un raisonnement.}
\begin{enumerate}\item
Placer les points \( A=(1;-6)\), \( B=(1;-9)\), \( C=(7;8)\) et \( D=(4;5)\) dans un repère orthonormé. Calculer la longueur du segment \( [AB]\) et les coordonnées du milieu du segment \( [CD]\).

 $l^2=9$,$l=3$,$M=(5.5,6.5)$\item
Est-ce que le triangle formé par les points \( A(-7;10)\), \( B(-3;6)\) et \( C(2;5)\) est isocèle ?

False
\end{enumerate}
} 
\vspace{2cm}
\vbox{37
\emph{Toutes les réponses doivent être justifiées par un calcul accompagné d'un raisonnement.}
\begin{enumerate}\item
Placer les points \( A=(-7;-2)\), \( B=(-5;-10)\), \( C=(-5;10)\) et \( D=(-4;-5)\) dans un repère orthonormé. Calculer la longueur du segment \( [AB]\) et les coordonnées du milieu du segment \( [CD]\).

 $l^2=68$,$l=2*sqrt(17)$,$M=(-4.5,2.5)$\item
Est-ce que le triangle formé par les points \( A(1;-7)\), \( B(-3;-3)\) et \( C(2;-6)\) est rectangle ?

True
\end{enumerate}
} 
\vspace{2cm}
\vbox{38
\emph{Toutes les réponses doivent être justifiées par un calcul accompagné d'un raisonnement.}
\begin{enumerate}\item
Placer les points \( A=(-6;9)\), \( B=(5;-6)\), \( C=(7;0)\) et \( D=(3;7)\) dans un repère orthonormé. Calculer la longueur du segment \( [AB]\) et les coordonnées du milieu du segment \( [CD]\).

 $l^2=346$,$l=sqrt(346)$,$M=(5.0,3.5)$\item
Est-ce que le triangle formé par les points \( A(-1;5)\), \( B(4;0)\) et \( C(12;8)\) est rectangle ?

False
\end{enumerate}
} 
\vspace{2cm}
\vbox{39
\emph{Toutes les réponses doivent être justifiées par un calcul accompagné d'un raisonnement.}
\begin{enumerate}\item
Placer les points \( A=(-2;-5)\), \( B=(6;7)\), \( C=(8;2)\) et \( D=(0;6)\) dans un repère orthonormé. Calculer la longueur du segment \( [AB]\) et les coordonnées du milieu du segment \( [CD]\).

 $l^2=208$,$l=4*sqrt(13)$,$M=(4.0,4.0)$\item
Est-ce que le triangle formé par les points \( A(-9;-1)\), \( B(-15;5)\) et \( C(-5;-1)\) est isocèle ?

False
\end{enumerate}
} 
\vspace{2cm}
\vbox{40
\emph{Toutes les réponses doivent être justifiées par un calcul accompagné d'un raisonnement.}
\begin{enumerate}\item
Placer les points \( A=(0;5)\), \( B=(2;0)\), \( C=(-2;1)\) et \( D=(10;-5)\) dans un repère orthonormé. Calculer la longueur du segment \( [AB]\) et les coordonnées du milieu du segment \( [CD]\).

 $l^2=29$,$l=sqrt(29)$,$M=(4.0,-2.0)$\item
Est-ce que le triangle formé par les points \( A(-9;-5)\), \( B(-11;-3)\) et \( C(-1;-5)\) est isocèle ?

False
\end{enumerate}
} 
\vspace{2cm}
\vbox{41
\emph{Toutes les réponses doivent être justifiées par un calcul accompagné d'un raisonnement.}
\begin{enumerate}\item
Placer les points \( A=(4;-3)\), \( B=(4;4)\), \( C=(9;-5)\) et \( D=(-4;-8)\) dans un repère orthonormé. Calculer la longueur du segment \( [AB]\) et les coordonnées du milieu du segment \( [CD]\).

 $l^2=49$,$l=7$,$M=(2.5,-6.5)$\item
Est-ce que le triangle formé par les points \( A(-2;2)\), \( B(-4;4)\) et \( C(-1;3)\) est rectangle ?

True
\end{enumerate}
} 
\vspace{2cm}
\vbox{42
\emph{Toutes les réponses doivent être justifiées par un calcul accompagné d'un raisonnement.}
\begin{enumerate}\item
Placer les points \( A=(4;-1)\), \( B=(0;-5)\), \( C=(2;-1)\) et \( D=(9;7)\) dans un repère orthonormé. Calculer la longueur du segment \( [AB]\) et les coordonnées du milieu du segment \( [CD]\).

 $l^2=32$,$l=4*sqrt(2)$,$M=(5.5,3.0)$\item
Est-ce que le triangle formé par les points \( A(8;7)\), \( B(10;5)\) et \( C(5;2)\) est isocèle ?

True
\end{enumerate}
} 
\vspace{2cm}
\vbox{43
\emph{Toutes les réponses doivent être justifiées par un calcul accompagné d'un raisonnement.}
\begin{enumerate}\item
Placer les points \( A=(3;8)\), \( B=(-1;6)\), \( C=(2;-7)\) et \( D=(0;-4)\) dans un repère orthonormé. Calculer la longueur du segment \( [AB]\) et les coordonnées du milieu du segment \( [CD]\).

 $l^2=20$,$l=2*sqrt(5)$,$M=(1.0,-5.5)$\item
Est-ce que le triangle formé par les points \( A(7;6)\), \( B(11;2)\) et \( C(7;6)\) est rectangle ?

True
\end{enumerate}
} 
\vspace{2cm}
\vbox{44
\emph{Toutes les réponses doivent être justifiées par un calcul accompagné d'un raisonnement.}
\begin{enumerate}\item
Placer les points \( A=(8;9)\), \( B=(0;10)\), \( C=(-5;10)\) et \( D=(-10;10)\) dans un repère orthonormé. Calculer la longueur du segment \( [AB]\) et les coordonnées du milieu du segment \( [CD]\).

 $l^2=65$,$l=sqrt(65)$,$M=(-7.5,10.0)$\item
Est-ce que le triangle formé par les points \( A(8;-10)\), \( B(4;-6)\) et \( C(19;-5)\) est isocèle ?

False
\end{enumerate}
} 
\vspace{2cm}
\vbox{45
\emph{Toutes les réponses doivent être justifiées par un calcul accompagné d'un raisonnement.}
\begin{enumerate}\item
Placer les points \( A=(-1;-10)\), \( B=(-8;2)\), \( C=(-2;0)\) et \( D=(9;-7)\) dans un repère orthonormé. Calculer la longueur du segment \( [AB]\) et les coordonnées du milieu du segment \( [CD]\).

 $l^2=193$,$l=sqrt(193)$,$M=(3.5,-3.5)$\item
Est-ce que le triangle formé par les points \( A(10;-5)\), \( B(10;-5)\) et \( C(12;-3)\) est rectangle ?

True
\end{enumerate}
} 
\vspace{2cm}
\vbox{46
\emph{Toutes les réponses doivent être justifiées par un calcul accompagné d'un raisonnement.}
\begin{enumerate}\item
Placer les points \( A=(4;0)\), \( B=(0;-10)\), \( C=(3;8)\) et \( D=(-6;-9)\) dans un repère orthonormé. Calculer la longueur du segment \( [AB]\) et les coordonnées du milieu du segment \( [CD]\).

 $l^2=116$,$l=2*sqrt(29)$,$M=(-1.5,-0.5)$\item
Est-ce que le triangle formé par les points \( A(3;9)\), \( B(7;5)\) et \( C(2;8)\) est rectangle ?

True
\end{enumerate}
} 
\vspace{2cm}
\vbox{47
\emph{Toutes les réponses doivent être justifiées par un calcul accompagné d'un raisonnement.}
\begin{enumerate}\item
Placer les points \( A=(2;9)\), \( B=(-6;-2)\), \( C=(0;-6)\) et \( D=(2;-3)\) dans un repère orthonormé. Calculer la longueur du segment \( [AB]\) et les coordonnées du milieu du segment \( [CD]\).

 $l^2=185$,$l=sqrt(185)$,$M=(1.0,-4.5)$\item
Est-ce que le triangle formé par les points \( A(-8;-8)\), \( B(-10;-6)\) et \( C(0;-8)\) est isocèle ?

False
\end{enumerate}
} 
\vspace{2cm}
\vbox{48
\emph{Toutes les réponses doivent être justifiées par un calcul accompagné d'un raisonnement.}
\begin{enumerate}\item
Placer les points \( A=(3;5)\), \( B=(2;-1)\), \( C=(1;9)\) et \( D=(5;8)\) dans un repère orthonormé. Calculer la longueur du segment \( [AB]\) et les coordonnées du milieu du segment \( [CD]\).

 $l^2=37$,$l=sqrt(37)$,$M=(3.0,8.5)$\item
Est-ce que le triangle formé par les points \( A(8;-5)\), \( B(4;-1)\) et \( C(4;-5)\) est isocèle ?

True
\end{enumerate}
} 
\vspace{2cm}
\vbox{49
\emph{Toutes les réponses doivent être justifiées par un calcul accompagné d'un raisonnement.}
\begin{enumerate}\item
Placer les points \( A=(8;6)\), \( B=(-8;-7)\), \( C=(9;-5)\) et \( D=(-5;-2)\) dans un repère orthonormé. Calculer la longueur du segment \( [AB]\) et les coordonnées du milieu du segment \( [CD]\).

 $l^2=425$,$l=5*sqrt(17)$,$M=(2.0,-3.5)$\item
Est-ce que le triangle formé par les points \( A(4;1)\), \( B(0;5)\) et \( C(10;1)\) est isocèle ?

False
\end{enumerate}
} 
\vspace{2cm}
\vbox{50
\emph{Toutes les réponses doivent être justifiées par un calcul accompagné d'un raisonnement.}
\begin{enumerate}\item
Placer les points \( A=(2;-10)\), \( B=(-1;2)\), \( C=(0;1)\) et \( D=(-4;-4)\) dans un repère orthonormé. Calculer la longueur du segment \( [AB]\) et les coordonnées du milieu du segment \( [CD]\).

 $l^2=153$,$l=3*sqrt(17)$,$M=(-2.0,-1.5)$\item
Est-ce que le triangle formé par les points \( A(-9;4)\), \( B(-9;4)\) et \( C(-10;3)\) est isocèle ?

True
\end{enumerate}
} 
\vspace{2cm}
\vbox{51
\emph{Toutes les réponses doivent être justifiées par un calcul accompagné d'un raisonnement.}
\begin{enumerate}\item
Placer les points \( A=(-9;9)\), \( B=(6;-10)\), \( C=(-8;10)\) et \( D=(5;9)\) dans un repère orthonormé. Calculer la longueur du segment \( [AB]\) et les coordonnées du milieu du segment \( [CD]\).

 $l^2=586$,$l=sqrt(586)$,$M=(-1.5,9.5)$\item
Est-ce que le triangle formé par les points \( A(-4;0)\), \( B(-10;6)\) et \( C(-7;3)\) est isocèle ?

True
\end{enumerate}
} 
\vspace{2cm}
\vbox{52
\emph{Toutes les réponses doivent être justifiées par un calcul accompagné d'un raisonnement.}
\begin{enumerate}\item
Placer les points \( A=(-5;0)\), \( B=(8;3)\), \( C=(1;-2)\) et \( D=(-4;5)\) dans un repère orthonormé. Calculer la longueur du segment \( [AB]\) et les coordonnées du milieu du segment \( [CD]\).

 $l^2=178$,$l=sqrt(178)$,$M=(-1.5,1.5)$\item
Est-ce que le triangle formé par les points \( A(7;-6)\), \( B(1;0)\) et \( C(1;-6)\) est isocèle ?

True
\end{enumerate}
} 
\vspace{2cm}
\vbox{53
\emph{Toutes les réponses doivent être justifiées par un calcul accompagné d'un raisonnement.}
\begin{enumerate}\item
Placer les points \( A=(0;8)\), \( B=(7;6)\), \( C=(8;4)\) et \( D=(0;6)\) dans un repère orthonormé. Calculer la longueur du segment \( [AB]\) et les coordonnées du milieu du segment \( [CD]\).

 $l^2=53$,$l=sqrt(53)$,$M=(4.0,5.0)$\item
Est-ce que le triangle formé par les points \( A(10;-8)\), \( B(10;-8)\) et \( C(11;-7)\) est rectangle ?

True
\end{enumerate}
} 
\vspace{2cm}
\vbox{54
\emph{Toutes les réponses doivent être justifiées par un calcul accompagné d'un raisonnement.}
\begin{enumerate}\item
Placer les points \( A=(9;-8)\), \( B=(7;-8)\), \( C=(-5;4)\) et \( D=(-9;-5)\) dans un repère orthonormé. Calculer la longueur du segment \( [AB]\) et les coordonnées du milieu du segment \( [CD]\).

 $l^2=4$,$l=2$,$M=(-7.0,-0.5)$\item
Est-ce que le triangle formé par les points \( A(3;-9)\), \( B(-3;-3)\) et \( C(11;-5)\) est isocèle ?

False
\end{enumerate}
} 
\vspace{2cm}
\vbox{55
\emph{Toutes les réponses doivent être justifiées par un calcul accompagné d'un raisonnement.}
\begin{enumerate}\item
Placer les points \( A=(9;9)\), \( B=(8;9)\), \( C=(9;-4)\) et \( D=(4;-8)\) dans un repère orthonormé. Calculer la longueur du segment \( [AB]\) et les coordonnées du milieu du segment \( [CD]\).

 $l^2=1$,$l=1$,$M=(6.5,-6.0)$\item
Est-ce que le triangle formé par les points \( A(4;5)\), \( B(5;4)\) et \( C(4;5)\) est rectangle ?

True
\end{enumerate}
} 
\vspace{2cm}
\vbox{56
\emph{Toutes les réponses doivent être justifiées par un calcul accompagné d'un raisonnement.}
\begin{enumerate}\item
Placer les points \( A=(2;10)\), \( B=(10;-1)\), \( C=(-9;4)\) et \( D=(1;7)\) dans un repère orthonormé. Calculer la longueur du segment \( [AB]\) et les coordonnées du milieu du segment \( [CD]\).

 $l^2=185$,$l=sqrt(185)$,$M=(-4.0,5.5)$\item
Est-ce que le triangle formé par les points \( A(-7;3)\), \( B(-6;2)\) et \( C(6;6)\) est rectangle ?

False
\end{enumerate}
} 
\vspace{2cm}
\vbox{57
\emph{Toutes les réponses doivent être justifiées par un calcul accompagné d'un raisonnement.}
\begin{enumerate}\item
Placer les points \( A=(-4;-1)\), \( B=(-8;-3)\), \( C=(-3;10)\) et \( D=(-1;-2)\) dans un repère orthonormé. Calculer la longueur du segment \( [AB]\) et les coordonnées du milieu du segment \( [CD]\).

 $l^2=20$,$l=2*sqrt(5)$,$M=(-2.0,4.0)$\item
Est-ce que le triangle formé par les points \( A(-4;8)\), \( B(-6;10)\) et \( C(9;11)\) est rectangle ?

False
\end{enumerate}
} 
\vspace{2cm}
\vbox{58
\emph{Toutes les réponses doivent être justifiées par un calcul accompagné d'un raisonnement.}
\begin{enumerate}\item
Placer les points \( A=(9;-10)\), \( B=(-4;0)\), \( C=(8;-2)\) et \( D=(4;4)\) dans un repère orthonormé. Calculer la longueur du segment \( [AB]\) et les coordonnées du milieu du segment \( [CD]\).

 $l^2=269$,$l=sqrt(269)$,$M=(6.0,1.0)$\item
Est-ce que le triangle formé par les points \( A(-5;7)\), \( B(-3;5)\) et \( C(5;7)\) est rectangle ?

False
\end{enumerate}
} 
\vspace{2cm}
\vbox{59
\emph{Toutes les réponses doivent être justifiées par un calcul accompagné d'un raisonnement.}
\begin{enumerate}\item
Placer les points \( A=(6;-6)\), \( B=(2;2)\), \( C=(4;4)\) et \( D=(1;6)\) dans un repère orthonormé. Calculer la longueur du segment \( [AB]\) et les coordonnées du milieu du segment \( [CD]\).

 $l^2=80$,$l=4*sqrt(5)$,$M=(2.5,5.0)$\item
Est-ce que le triangle formé par les points \( A(2;9)\), \( B(-3;14)\) et \( C(3;10)\) est rectangle ?

True
\end{enumerate}
} 
\vspace{2cm}
\vbox{60
\emph{Toutes les réponses doivent être justifiées par un calcul accompagné d'un raisonnement.}
\begin{enumerate}\item
Placer les points \( A=(-10;-10)\), \( B=(5;-8)\), \( C=(10;7)\) et \( D=(3;-4)\) dans un repère orthonormé. Calculer la longueur du segment \( [AB]\) et les coordonnées du milieu du segment \( [CD]\).

 $l^2=229$,$l=sqrt(229)$,$M=(6.5,1.5)$\item
Est-ce que le triangle formé par les points \( A(3;6)\), \( B(1;8)\) et \( C(0;5)\) est isocèle ?

True
\end{enumerate}
} 
\vspace{2cm}
\vbox{61
\emph{Toutes les réponses doivent être justifiées par un calcul accompagné d'un raisonnement.}
\begin{enumerate}\item
Placer les points \( A=(5;-2)\), \( B=(9;-10)\), \( C=(4;-1)\) et \( D=(-10;-9)\) dans un repère orthonormé. Calculer la longueur du segment \( [AB]\) et les coordonnées du milieu du segment \( [CD]\).

 $l^2=80$,$l=4*sqrt(5)$,$M=(-3.0,-5.0)$\item
Est-ce que le triangle formé par les points \( A(-3;9)\), \( B(1;5)\) et \( C(5;3)\) est isocèle ?

False
\end{enumerate}
} 
\vspace{2cm}
\vbox{62
\emph{Toutes les réponses doivent être justifiées par un calcul accompagné d'un raisonnement.}
\begin{enumerate}\item
Placer les points \( A=(5;3)\), \( B=(-10;9)\), \( C=(2;10)\) et \( D=(-5;-2)\) dans un repère orthonormé. Calculer la longueur du segment \( [AB]\) et les coordonnées du milieu du segment \( [CD]\).

 $l^2=261$,$l=3*sqrt(29)$,$M=(-1.5,4.0)$\item
Est-ce que le triangle formé par les points \( A(-10;-6)\), \( B(-14;-2)\) et \( C(1;-5)\) est rectangle ?

False
\end{enumerate}
} 
\vspace{2cm}
\vbox{63
\emph{Toutes les réponses doivent être justifiées par un calcul accompagné d'un raisonnement.}
\begin{enumerate}\item
Placer les points \( A=(-8;10)\), \( B=(6;3)\), \( C=(-5;-2)\) et \( D=(6;10)\) dans un repère orthonormé. Calculer la longueur du segment \( [AB]\) et les coordonnées du milieu du segment \( [CD]\).

 $l^2=245$,$l=7*sqrt(5)$,$M=(0.5,4.0)$\item
Est-ce que le triangle formé par les points \( A(6;-8)\), \( B(10;-12)\) et \( C(6;-12)\) est isocèle ?

True
\end{enumerate}
} 
\vspace{2cm}
\vbox{64
\emph{Toutes les réponses doivent être justifiées par un calcul accompagné d'un raisonnement.}
\begin{enumerate}\item
Placer les points \( A=(9;2)\), \( B=(-5;0)\), \( C=(4;-8)\) et \( D=(8;-5)\) dans un repère orthonormé. Calculer la longueur du segment \( [AB]\) et les coordonnées du milieu du segment \( [CD]\).

 $l^2=200$,$l=10*sqrt(2)$,$M=(6.0,-6.5)$\item
Est-ce que le triangle formé par les points \( A(-1;4)\), \( B(-1;4)\) et \( C(12;7)\) est isocèle ?

False
\end{enumerate}
} 
\vspace{2cm}
\vbox{65
\emph{Toutes les réponses doivent être justifiées par un calcul accompagné d'un raisonnement.}
\begin{enumerate}\item
Placer les points \( A=(-5;3)\), \( B=(-5;-4)\), \( C=(0;-2)\) et \( D=(3;-5)\) dans un repère orthonormé. Calculer la longueur du segment \( [AB]\) et les coordonnées du milieu du segment \( [CD]\).

 $l^2=49$,$l=7$,$M=(1.5,-3.5)$\item
Est-ce que le triangle formé par les points \( A(3;-2)\), \( B(7;-6)\) et \( C(1;-8)\) est isocèle ?

True
\end{enumerate}
} 
\vspace{2cm}
\vbox{66
\emph{Toutes les réponses doivent être justifiées par un calcul accompagné d'un raisonnement.}
\begin{enumerate}\item
Placer les points \( A=(2;2)\), \( B=(4;-9)\), \( C=(-10;7)\) et \( D=(0;-7)\) dans un repère orthonormé. Calculer la longueur du segment \( [AB]\) et les coordonnées du milieu du segment \( [CD]\).

 $l^2=125$,$l=5*sqrt(5)$,$M=(-5.0,0.0)$\item
Est-ce que le triangle formé par les points \( A(-4;1)\), \( B(-2;-1)\) et \( C(4;-3)\) est isocèle ?

False
\end{enumerate}
} 
\vspace{2cm}
\vbox{67
\emph{Toutes les réponses doivent être justifiées par un calcul accompagné d'un raisonnement.}
\begin{enumerate}\item
Placer les points \( A=(-2;-9)\), \( B=(-6;8)\), \( C=(-9;6)\) et \( D=(4;-8)\) dans un repère orthonormé. Calculer la longueur du segment \( [AB]\) et les coordonnées du milieu du segment \( [CD]\).

 $l^2=305$,$l=sqrt(305)$,$M=(-2.5,-1.0)$\item
Est-ce que le triangle formé par les points \( A(3;-8)\), \( B(7;-12)\) et \( C(4;-11)\) est isocèle ?

True
\end{enumerate}
} 
\vspace{2cm}
\vbox{68
\emph{Toutes les réponses doivent être justifiées par un calcul accompagné d'un raisonnement.}
\begin{enumerate}\item
Placer les points \( A=(3;-8)\), \( B=(-1;6)\), \( C=(-5;4)\) et \( D=(10;-9)\) dans un repère orthonormé. Calculer la longueur du segment \( [AB]\) et les coordonnées du milieu du segment \( [CD]\).

 $l^2=212$,$l=2*sqrt(53)$,$M=(2.5,-2.5)$\item
Est-ce que le triangle formé par les points \( A(2;-1)\), \( B(4;-3)\) et \( C(6;1)\) est isocèle ?

True
\end{enumerate}
} 
\vspace{2cm}
\vbox{69
\emph{Toutes les réponses doivent être justifiées par un calcul accompagné d'un raisonnement.}
\begin{enumerate}\item
Placer les points \( A=(5;1)\), \( B=(6;-2)\), \( C=(-9;10)\) et \( D=(2;2)\) dans un repère orthonormé. Calculer la longueur du segment \( [AB]\) et les coordonnées du milieu du segment \( [CD]\).

 $l^2=10$,$l=sqrt(10)$,$M=(-3.5,6.0)$\item
Est-ce que le triangle formé par les points \( A(-7;-4)\), \( B(-8;-3)\) et \( C(6;-1)\) est rectangle ?

False
\end{enumerate}
} 
\vspace{2cm}
\vbox{70
\emph{Toutes les réponses doivent être justifiées par un calcul accompagné d'un raisonnement.}
\begin{enumerate}\item
Placer les points \( A=(-6;-6)\), \( B=(-6;-5)\), \( C=(10;-3)\) et \( D=(-9;-2)\) dans un repère orthonormé. Calculer la longueur du segment \( [AB]\) et les coordonnées du milieu du segment \( [CD]\).

 $l^2=1$,$l=1$,$M=(0.5,-2.5)$\item
Est-ce que le triangle formé par les points \( A(-6;-5)\), \( B(-6;-5)\) et \( C(7;-2)\) est rectangle ?

False
\end{enumerate}
} 
\vspace{2cm}
\vbox{71
\emph{Toutes les réponses doivent être justifiées par un calcul accompagné d'un raisonnement.}
\begin{enumerate}\item
Placer les points \( A=(-2;-7)\), \( B=(2;-2)\), \( C=(-10;6)\) et \( D=(-7;6)\) dans un repère orthonormé. Calculer la longueur du segment \( [AB]\) et les coordonnées du milieu du segment \( [CD]\).

 $l^2=41$,$l=sqrt(41)$,$M=(-8.5,6.0)$\item
Est-ce que le triangle formé par les points \( A(10;0)\), \( B(8;2)\) et \( C(19;-1)\) est rectangle ?

False
\end{enumerate}
} 
\vspace{2cm}
\vbox{72
\emph{Toutes les réponses doivent être justifiées par un calcul accompagné d'un raisonnement.}
\begin{enumerate}\item
Placer les points \( A=(1;6)\), \( B=(-1;9)\), \( C=(3;5)\) et \( D=(0;8)\) dans un repère orthonormé. Calculer la longueur du segment \( [AB]\) et les coordonnées du milieu du segment \( [CD]\).

 $l^2=13$,$l=sqrt(13)$,$M=(1.5,6.5)$\item
Est-ce que le triangle formé par les points \( A(10;10)\), \( B(14;6)\) et \( C(10;10)\) est rectangle ?

True
\end{enumerate}
} 
\vspace{2cm}
\vbox{73
\emph{Toutes les réponses doivent être justifiées par un calcul accompagné d'un raisonnement.}
\begin{enumerate}\item
Placer les points \( A=(-6;-10)\), \( B=(1;8)\), \( C=(-3;8)\) et \( D=(2;-3)\) dans un repère orthonormé. Calculer la longueur du segment \( [AB]\) et les coordonnées du milieu du segment \( [CD]\).

 $l^2=373$,$l=sqrt(373)$,$M=(-0.5,2.5)$\item
Est-ce que le triangle formé par les points \( A(-7;-6)\), \( B(-13;0)\) et \( C(0;-3)\) est isocèle ?

False
\end{enumerate}
} 
\vspace{2cm}
\vbox{74
\emph{Toutes les réponses doivent être justifiées par un calcul accompagné d'un raisonnement.}
\begin{enumerate}\item
Placer les points \( A=(1;-1)\), \( B=(4;0)\), \( C=(6;-5)\) et \( D=(5;-4)\) dans un repère orthonormé. Calculer la longueur du segment \( [AB]\) et les coordonnées du milieu du segment \( [CD]\).

 $l^2=10$,$l=sqrt(10)$,$M=(5.5,-4.5)$\item
Est-ce que le triangle formé par les points \( A(6;1)\), \( B(4;3)\) et \( C(11;-2)\) est isocèle ?

False
\end{enumerate}
} 
\vspace{2cm}
\vbox{75
\emph{Toutes les réponses doivent être justifiées par un calcul accompagné d'un raisonnement.}
\begin{enumerate}\item
Placer les points \( A=(0;1)\), \( B=(-7;-7)\), \( C=(3;8)\) et \( D=(7;-3)\) dans un repère orthonormé. Calculer la longueur du segment \( [AB]\) et les coordonnées du milieu du segment \( [CD]\).

 $l^2=113$,$l=sqrt(113)$,$M=(5.0,2.5)$\item
Est-ce que le triangle formé par les points \( A(10;5)\), \( B(14;1)\) et \( C(14;5)\) est isocèle ?

True
\end{enumerate}
} 
\vspace{2cm}
\vbox{76
\emph{Toutes les réponses doivent être justifiées par un calcul accompagné d'un raisonnement.}
\begin{enumerate}\item
Placer les points \( A=(10;-4)\), \( B=(-2;-1)\), \( C=(-3;3)\) et \( D=(-3;1)\) dans un repère orthonormé. Calculer la longueur du segment \( [AB]\) et les coordonnées du milieu du segment \( [CD]\).

 $l^2=153$,$l=3*sqrt(17)$,$M=(-3.0,2.0)$\item
Est-ce que le triangle formé par les points \( A(-1;6)\), \( B(-6;11)\) et \( C(-3;4)\) est rectangle ?

True
\end{enumerate}
} 
\vspace{2cm}
\vbox{77
\emph{Toutes les réponses doivent être justifiées par un calcul accompagné d'un raisonnement.}
\begin{enumerate}\item
Placer les points \( A=(-7;-2)\), \( B=(6;10)\), \( C=(0;1)\) et \( D=(1;5)\) dans un repère orthonormé. Calculer la longueur du segment \( [AB]\) et les coordonnées du milieu du segment \( [CD]\).

 $l^2=313$,$l=sqrt(313)$,$M=(0.5,3.0)$\item
Est-ce que le triangle formé par les points \( A(-7;-3)\), \( B(-5;-5)\) et \( C(4;-4)\) est isocèle ?

False
\end{enumerate}
} 
\vspace{2cm}
\vbox{78
\emph{Toutes les réponses doivent être justifiées par un calcul accompagné d'un raisonnement.}
\begin{enumerate}\item
Placer les points \( A=(-7;7)\), \( B=(1;-10)\), \( C=(-7;0)\) et \( D=(-1;-3)\) dans un repère orthonormé. Calculer la longueur du segment \( [AB]\) et les coordonnées du milieu du segment \( [CD]\).

 $l^2=353$,$l=sqrt(353)$,$M=(-4.0,-1.5)$\item
Est-ce que le triangle formé par les points \( A(0;-8)\), \( B(2;-10)\) et \( C(14;-6)\) est isocèle ?

False
\end{enumerate}
} 
\vspace{2cm}
\vbox{79
\emph{Toutes les réponses doivent être justifiées par un calcul accompagné d'un raisonnement.}
\begin{enumerate}\item
Placer les points \( A=(-4;8)\), \( B=(-7;1)\), \( C=(10;8)\) et \( D=(-7;-1)\) dans un repère orthonormé. Calculer la longueur du segment \( [AB]\) et les coordonnées du milieu du segment \( [CD]\).

 $l^2=58$,$l=sqrt(58)$,$M=(1.5,3.5)$\item
Est-ce que le triangle formé par les points \( A(-4;-10)\), \( B(0;-14)\) et \( C(-7;-13)\) est rectangle ?

True
\end{enumerate}
} 
\vspace{2cm}
\vbox{80
\emph{Toutes les réponses doivent être justifiées par un calcul accompagné d'un raisonnement.}
\begin{enumerate}\item
Placer les points \( A=(-10;-10)\), \( B=(-1;2)\), \( C=(-6;-5)\) et \( D=(0;8)\) dans un repère orthonormé. Calculer la longueur du segment \( [AB]\) et les coordonnées du milieu du segment \( [CD]\).

 $l^2=225$,$l=15$,$M=(-3.0,1.5)$\item
Est-ce que le triangle formé par les points \( A(-10;4)\), \( B(-11;5)\) et \( C(2;6)\) est rectangle ?

False
\end{enumerate}
} 
\vspace{2cm}
\vbox{81
\emph{Toutes les réponses doivent être justifiées par un calcul accompagné d'un raisonnement.}
\begin{enumerate}\item
Placer les points \( A=(0;0)\), \( B=(6;9)\), \( C=(-8;-9)\) et \( D=(-8;4)\) dans un repère orthonormé. Calculer la longueur du segment \( [AB]\) et les coordonnées du milieu du segment \( [CD]\).

 $l^2=117$,$l=3*sqrt(13)$,$M=(-8.0,-2.5)$\item
Est-ce que le triangle formé par les points \( A(2;-1)\), \( B(-4;5)\) et \( C(12;5)\) est isocèle ?

False
\end{enumerate}
} 
\vspace{2cm}
\vbox{82
\emph{Toutes les réponses doivent être justifiées par un calcul accompagné d'un raisonnement.}
\begin{enumerate}\item
Placer les points \( A=(-3;-9)\), \( B=(1;7)\), \( C=(5;-1)\) et \( D=(7;9)\) dans un repère orthonormé. Calculer la longueur du segment \( [AB]\) et les coordonnées du milieu du segment \( [CD]\).

 $l^2=272$,$l=4*sqrt(17)$,$M=(6.0,4.0)$\item
Est-ce que le triangle formé par les points \( A(-7;-2)\), \( B(-5;-4)\) et \( C(-8;-3)\) est rectangle ?

True
\end{enumerate}
} 
\vspace{2cm}
\vbox{83
\emph{Toutes les réponses doivent être justifiées par un calcul accompagné d'un raisonnement.}
\begin{enumerate}\item
Placer les points \( A=(4;4)\), \( B=(10;7)\), \( C=(-10;-7)\) et \( D=(-3;-5)\) dans un repère orthonormé. Calculer la longueur du segment \( [AB]\) et les coordonnées du milieu du segment \( [CD]\).

 $l^2=45$,$l=3*sqrt(5)$,$M=(-6.5,-6.0)$\item
Est-ce que le triangle formé par les points \( A(0;8)\), \( B(0;8)\) et \( C(10;8)\) est rectangle ?

False
\end{enumerate}
} 
\vspace{2cm}
\vbox{84
\emph{Toutes les réponses doivent être justifiées par un calcul accompagné d'un raisonnement.}
\begin{enumerate}\item
Placer les points \( A=(-1;5)\), \( B=(-10;-9)\), \( C=(10;9)\) et \( D=(-6;-8)\) dans un repère orthonormé. Calculer la longueur du segment \( [AB]\) et les coordonnées du milieu du segment \( [CD]\).

 $l^2=277$,$l=sqrt(277)$,$M=(2.0,0.5)$\item
Est-ce que le triangle formé par les points \( A(9;-3)\), \( B(6;0)\) et \( C(12;0)\) est rectangle ?

True
\end{enumerate}
} 
\vspace{2cm}


\end{document}

% FONCTION : MODE GRAPHIQUE

%This is part of Un soupçon de mathématique sans être agressif pour autant
% Copyright (c) 2012-2013
%   Laurent Claessens
% See the file fdl-1.3.txt for copying conditions.

    Un berger syldave s'entraine pour le championnat national du lancer de chèvre. L'épreuve consiste à lancer une chèvre vers le haut depuis le bord d'une falaise située au bord d'un lac tranquille. La hauteur de la chèvre en fonction du temps par rapport à la surface du lac tranquille est une fonction \( f\) donnée par le graphique suivant.

    \begin{center}
        \input{Fig_WRXbDCo.pstricks}
    \end{center}
    La dernière partie du graphique correspond à la chèvre que l'on remonte rapidement hors de l'eau.
    À partir du graphique :
    \begin{enumerate}
        \item
            À quelle hauteur se trouve la chèvre au moment du lancer ?
        \item
            Pendant combien de temps la chèvre reste à une hauteur supérieure à celle à laquelle elle a été lancée ?
        \item
            À quel moment la chèvre atteint-elle sa hauteur maximale ? Quelle est cette hauteur ?
        \item
            À quelle hauteur se trouve la chèvre après \( 2.5\) secondes de vol ?
        \item
            Résumer toutes ces informations en dressant le tableau de variation de la fonction \( f\).
    \end{enumerate}

\vspace{1cm}
%This is part of Un soupçon de mathématique sans être agressif pour autant
% Copyright (c) 2012-2013
%   Laurent Claessens
% See the file fdl-1.3.txt for copying conditions.

    Un berger syldave s'entraine pour le championnat national du lancer de chèvre. L'épreuve consiste à lancer une chèvre vers le haut depuis le bord d'une falaise située au bord d'un lac tranquille. La hauteur de la chèvre en fonction du temps par rapport à la surface du lac tranquille est une fonction \( f\) donnée par le graphique suivant.

    \begin{center}
        \input{Fig_WRXbDCo.pstricks}
    \end{center}
    La dernière partie du graphique correspond à la chèvre que l'on remonte rapidement hors de l'eau.
    À partir du graphique :
    \begin{enumerate}
        \item
            À quelle hauteur se trouve la chèvre au moment du lancer ?
        \item
            Pendant combien de temps la chèvre reste à une hauteur supérieure à celle à laquelle elle a été lancée ?
        \item
            À quel moment la chèvre atteint-elle sa hauteur maximale ? Quelle est cette hauteur ?
        \item
            À quelle hauteur se trouve la chèvre après \( 2.5\) secondes de vol ?
        \item
            Résumer toutes ces informations en dressant le tableau de variation de la fonction \( f\).
    \end{enumerate}


\end{document}


\clearpage

% FONCTIONS AFFINES

%This is part of Un soupçon de mathématique sans être agressif pour autant
% Copyright (c) 2012-2013
%   Laurent Claessens
% See the file fdl-1.3.txt for copying conditions.

    Le taxi Besacdanslesac divise son prix en deux parties : $0.2$ euros de frais de prise en charge plus un euro par km parcouru. Le taxi Ledoubstoudoux par contre divise son prix en $1$ euro de frais de prise en charge plus $0.8$ euros par kilomètre parcouru.

    \begin{enumerate}
        \item
            Combien coûte un trajet de \SI{5}{\kilo\meter} avec Besacdanslesac ?
        \item
            Donner une expression algébrique du prix d'une course en fonction du nombre de kilomètres parcourus.
        \item
            Combien de kilomètres peut-t-on effectuer dans Ledoubstoudoux avec \( 10\) euros ?
        \item
            Exprimer les prix en fonction du nombre de kilomètres parcourus sur un graphique (les deux taxis sur le même graphique).
        \item
            À partir de combien de kilomètres parcourus vaut-il mieux prendre Ledoubstoudoux ?
    \end{enumerate}


\vspace{1cm}

%This is part of Un soupçon de mathématique sans être agressif pour autant
% Copyright (c) 2012-2013
%   Laurent Claessens
% See the file fdl-1.3.txt for copying conditions.

    Le taxi Besacdanslesac divise son prix en deux parties : $0.2$ euros de frais de prise en charge plus un euro par km parcouru. Le taxi Ledoubstoudoux par contre divise son prix en $1$ euro de frais de prise en charge plus $0.8$ euros par kilomètre parcouru.

    \begin{enumerate}
        \item
            Combien coûte un trajet de \SI{5}{\kilo\meter} avec Besacdanslesac ?
        \item
            Donner une expression algébrique du prix d'une course en fonction du nombre de kilomètres parcourus.
        \item
            Combien de kilomètres peut-t-on effectuer dans Ledoubstoudoux avec \( 10\) euros ?
        \item
            Exprimer les prix en fonction du nombre de kilomètres parcourus sur un graphique (les deux taxis sur le même graphique).
        \item
            À partir de combien de kilomètres parcourus vaut-il mieux prendre Ledoubstoudoux ?
    \end{enumerate}


\vspace{1cm}
%This is part of Un soupçon de mathématique sans être agressif pour autant
% Copyright (c) 2012-2013
%   Laurent Claessens
% See the file fdl-1.3.txt for copying conditions.

    Le taxi Besacdanslesac divise son prix en deux parties : $0.2$ euros de frais de prise en charge plus un euro par km parcouru. Le taxi Ledoubstoudoux par contre divise son prix en $1$ euro de frais de prise en charge plus $0.8$ euros par kilomètre parcouru.

    \begin{enumerate}
        \item
            Combien coûte un trajet de \SI{5}{\kilo\meter} avec Besacdanslesac ?
        \item
            Donner une expression algébrique du prix d'une course en fonction du nombre de kilomètres parcourus.
        \item
            Combien de kilomètres peut-t-on effectuer dans Ledoubstoudoux avec \( 10\) euros ?
        \item
            Exprimer les prix en fonction du nombre de kilomètres parcourus sur un graphique (les deux taxis sur le même graphique).
        \item
            À partir de combien de kilomètres parcourus vaut-il mieux prendre Ledoubstoudoux ?
    \end{enumerate}


\vspace{1cm}
%This is part of Un soupçon de mathématique sans être agressif pour autant
% Copyright (c) 2012-2013
%   Laurent Claessens
% See the file fdl-1.3.txt for copying conditions.

    Le taxi Besacdanslesac divise son prix en deux parties : $0.2$ euros de frais de prise en charge plus un euro par km parcouru. Le taxi Ledoubstoudoux par contre divise son prix en $1$ euro de frais de prise en charge plus $0.8$ euros par kilomètre parcouru.

    \begin{enumerate}
        \item
            Combien coûte un trajet de \SI{5}{\kilo\meter} avec Besacdanslesac ?
        \item
            Donner une expression algébrique du prix d'une course en fonction du nombre de kilomètres parcourus.
        \item
            Combien de kilomètres peut-t-on effectuer dans Ledoubstoudoux avec \( 10\) euros ?
        \item
            Exprimer les prix en fonction du nombre de kilomètres parcourus sur un graphique (les deux taxis sur le même graphique).
        \item
            À partir de combien de kilomètres parcourus vaut-il mieux prendre Ledoubstoudoux ?
    \end{enumerate}




\end{document}
