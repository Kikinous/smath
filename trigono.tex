% This is part of Un soupçon de mathématique sans être agressif pour autant
% Copyright (c) 2013
%   Laurent Claessens
% See the file fdl-1.3.txt for copying conditions.

%+++++++++++++++++++++++++++++++++++++++++++++++++++++++++++++++++++++++++++++++++++++++++++++++++++++++++++++++++++++++++++ 
\section{Enroulement de la droite numérique sur le cercle}
%+++++++++++++++++++++++++++++++++++++++++++++++++++++++++++++++++++++++++++++++++++++++++++++++++++++++++++++++++++++++++++

\begin{definition}
    Le \defe{cercle trigonométrique}{cercle!trigonométrique} est le cercle de centre \( (0;0)\) et de rayon \( 1\) muni de l'orientation dans le sens direct (le sens inverse des aiguilles d'une montre).
\end{definition}

Nous enroulons la droite réelle sur le cercle, voir la figure \ref{LabelFigYORfWSM}. % From file YORfWSM
\newcommand{\CaptionFigYORfWSM}{Un cercle trigonométrique avec enroulement de la droite rélle.}
\input{Fig_YORfWSM.pstricks}

Étant donné que la circonférence du cercle est \( 2\pi\), le nombre \( 2\pi\) de la droite réelle vient au même endroit que le nombre zéro. Le nombre \( 2\pi+x\) vient alors au même endroit que \( x\) pour tout \( x\).

%+++++++++++++++++++++++++++++++++++++++++++++++++++++++++++++++++++++++++++++++++++++++++++++++++++++++++++++++++++++++++++ 
\section{Cosinus et sinus}
%+++++++++++++++++++++++++++++++++++++++++++++++++++++++++++++++++++++++++++++++++++++++++++++++++++++++++++++++++++++++++++

\begin{definition}
    \begin{multicols}{2}
    Soit le réel \( x\) et le point image \( M\) sur le cercle trigonométrique. Le \defe{cosinus}{cosinus} de \( x\) est l'abscisse du point \( M\) et le \defe{sinus}{sinus} de \(x\) est l'ordonnée du point \( M\).

    \columnbreak

%he result is on figure \ref{LabelFigLOBVHYF}. % From file LOBVHYF
%\newcommand{\CaptionFigLOBVHYF}{<+Type your caption here+>}
    \begin{center}
\input{Fig_LOBVHYF.pstricks}
    \end{center}

    \end{multicols}
\end{definition}


\begin{propriete}
    Vu que \( \big( \cos(x),\sin(x) \big)\) sont le coordonnées de points sur le cercle trigonométrique, nous avons
    \begin{enumerate}
        \item
            \( -1\leq \cos(x)\leq 1\)
        \item
            \( -1\leq \sin(x)\leq 1\)
        \item
            \( \cos^2(x)+\sin^2(x)=1\).
    \end{enumerate}
\end{propriete}

%+++++++++++++++++++++++++++++++++++++++++++++++++++++++++++++++++++++++++++++++++++++++++++++++++++++++++++++++++++++++++++ 
\section{Exercices}
%+++++++++++++++++++++++++++++++++++++++++++++++++++++++++++++++++++++++++++++++++++++++++++++++++++++++++++++++++++++++++++

\Exo{smath-0360}
\Exo{smath-0361}
\Exo{smath-0362}
\Exo{smath-0363}
