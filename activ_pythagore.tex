% This is part of Un soupçon de mathématique sans être agressif pour autant
% Copyright (c) 2014
%   Laurent Claessens
% See the file fdl-1.3.txt for copying conditions.


\begin{wrapfigure}{r}{7.0cm}
   \vspace{-2cm}        % à adapter.
   \centering
   \input{Fig_SAZooPjiyOV.pstricks}
\end{wrapfigure}


Découper sur une feuille dix carrés dont les mesures des côtés sont entières et valent de \SI{1}{\centi\meter} à \SI{10}{\centi\meter}. Indiquer à l'intérieur de chacun d'eux son aire en \si{\centi\meter\squared}. Vous pouvez en découper plusieurs de la même mesure.

Essayer de les assembler de façon à créer un triangle comme sur la figure ci-contre.

Lorsqu'un triangle est formé, aller au tableau écrire
\begin{itemize}
    \item les mesures.
    \item les aires.
    \item la nature du triangle.
\end{itemize}
