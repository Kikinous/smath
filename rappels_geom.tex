% This is part of Un soupçon de mathématique sans être agressif pour autant
% Copyright (c) 2012
%   Laurent Claessens
% See the file fdl-1.3.txt for copying conditions.

%---------------------------------------------------------------------------------------------------------------------------
\subsection{Médianes, hauteurs, médiatrices}
%---------------------------------------------------------------------------------------------------------------------------

Soit un triangle \( ABC\).

%///////////////////////////////////////////////////////////////////////////////////////////////////////////////////////////
\subsubsection{Médianes}
%///////////////////////////////////////////////////////////////////////////////////////////////////////////////////////////

\begin{multicols}{2}
    Une médiane est une droite passant par le centre d'un des côtés et par le sommet opposé. Les trois médianes se coupent en un point nommé \defe{centre}{centre!d'un triangle}. Il est parfois aussi nommé \emph{centre de masse}. De plus le centre de masse du triangle est situé aux deux tiers de la hauteur ne partant du sommet :
    \begin{equation}
        AG=\frac{ 2 }{ 3 }AK.
    \end{equation}

    \columnbreak


%The result is on figure \ref{LabelFigfigureNPQwFTp}. % From file figureNPQwFTp
%\newcommand{\CaptionFigfigureNPQwFTp}{<+Type your caption here+>}
    \begin{center}
\input{Fig_figureNPQwFTp.pstricks}
    \end{center}

\end{multicols}

%///////////////////////////////////////////////////////////////////////////////////////////////////////////////////////////
\subsubsection{Hauteurs}
%///////////////////////////////////////////////////////////////////////////////////////////////////////////////////////////

\begin{multicols}{2}
    Une hauteur est une droite passant par un sommet et coupant perpendiculairement le côté opposé. Les trois hauteurs se coupent en un point appelé l'\defe{orthocentre}{orthocentre} du triangle, nomme \( H\).

    \columnbreak


%The result is on figure \ref{LabelFigfigureITVTofz}. % From file figureITVTofz
%\newcommand{\CaptionFigfigureITVTofz}{<+Type your caption here+>}
\input{Fig_figureITVTofz.pstricks}

\end{multicols}


%///////////////////////////////////////////////////////////////////////////////////////////////////////////////////////////
\subsubsection{Médiatrices}
%///////////////////////////////////////////////////////////////////////////////////////////////////////////////////////////

\begin{multicols}{2}
    Une médiatrice est une droite coupant une côté perpendiculairement à son milieu.

    \columnbreak

%The result is on figure \ref{LabelFigfigureNAKnjxQ}. % From file figureNAKnjxQ
%\newcommand{\CaptionFigfigureNAKnjxQ}{<+Type your caption here+>}
\input{Fig_figureNAKnjxQ.pstricks}

\end{multicols}

%///////////////////////////////////////////////////////////////////////////////////////////////////////////////////////////
\subsubsection{Droite d'Euler}
%///////////////////////////////////////////////////////////////////////////////////////////////////////////////////////////

<++>


%+++++++++++++++++++++++++++++++++++++++++++++++++++++++++++++++++++++++++++++++++++++++++++++++++++++++++++++++++++++++++++
\section{Triangles isométriques}
%+++++++++++++++++++++++++++++++++++++++++++++++++++++++++++++++++++++++++++++++++++++++++++++++++++++++++++++++++++++++++++

Voir \cite{TqcjwY}.

\begin{propriete}       \label{PropRtqqxJ}
    Soient les triangles \( ABC\) et \( MNP\). Si
    \begin{enumerate}
        \item
            \( \hat A=\hat M\)
        \item
            \( AB=MN\)
        \item
            \( AC=MP\)
    \end{enumerate}
    alors les triangles sont isométriques.

    Si
    \begin{enumerate}
        \item
            \( AB=MN\)
        \item
            \( \hat A=\hat M\)
        \item
            \( \hat B=\hat N\)
    \end{enumerate}
    alors les triangles sont isométriques.
\end{propriete}
