% This is part of Un soupçon de mathématique sans être agressif pour autant
% Copyright (c) 2012,2015
%   Laurent Claessens
% See the file fdl-1.3.txt for copying conditions.

\begin{corrige}{Seconde-0071}

    \begin{enumerate}
        \item
            La fonction dessinée est plus grande que zéro entre \( x=-1\) et \( x=2\). L'énoncé demande l'inégalité stricte, donc l'ensemble des solutions est
            \begin{equation}
                \mathopen] -1 , 2 \mathclose[.
            \end{equation}
        \item
            La fonction est toujours strictement positive, donc il n'y a pas de solutions à l'inéquation demandée.
        \item
            La fonction passe au dessus de \( 2\) à deux moments. D'abord entre \( x=-3\) et \( x=-\frac{ 1 }{2}\); et ensuite entre \( x=1\) et \( x=2.5\). L'ensemble des solutions est donc
            \begin{equation}
                \mathopen[ -3 , -\frac{ 1 }{2} \mathclose]\cup\mathopen[ 1 , \frac{ 5 }{2} \mathclose].
            \end{equation}
        \item
            La fonction passe à deux reprises en dessous de \( -1\). D'abord entre \( x=-2.5\) et \( x=-1\); et ensuite entre \( x=2.5\) et \( x=3.5\). L'ensemble des solutions est donc
            \begin{equation}
                \mathopen] -2.5 , -1 \mathclose[\cup\mathopen] 2.5 , 3.5 \mathclose[.
            \end{equation}
    \end{enumerate}

\end{corrige}
