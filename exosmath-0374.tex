% This is part of Un soupçon de mathématique sans être agressif pour autant
% Copyright (c) 2013
%   Laurent Claessens
% See the file fdl-1.3.txt for copying conditions.

\begin{exercice}\label{exosmath-0374}

    Un homme effectue un plongeon d'une hauteur de \unit{180}{\meter}. Nous savons de la physique que la distance (en mètres) parcourue en fonction du temps (en secondes) est :
    \begin{equation}
        d(t)=\frac{ 1 }{2}gt^2
    \end{equation}
    avec \( g\simeq\unit{10}{\meter\per\second\squared}\). Calculer la vitesse du plongeur au moment de pénétrer dans l'eau. 

\corrref{smath-0374}
\end{exercice}
