% This is part of Un soupçon de mathématique sans être agressif pour autant
% Copyright (c) 2013-2014
%   Laurent Claessens
% See the file fdl-1.3.txt for copying conditions.

\begin{exercice}\label{exosmath-0590}

    Donner les coordonnées des points \( A\), \( B\), \( C\) et \( D\) ainsi que celles des vecteurs \( \vect{ AB }\) et \( \vect{ DC }\). Qu'en déduire quant au quadrilatère \( ABCD\) ?

    \begin{center}
   \input{Fig_KYbSnVB.pstricks}
    \end{center}

    Sur le même dessin, placer les vecteurs \( \vect{ u }=\begin{pmatrix}
        0    \\ 
        4    
    \end{pmatrix}\), \( \vect{ v }=\begin{pmatrix}
        -2    \\ 
        -1    
    \end{pmatrix}\) ainsi qu'un point \( X\) tel que \( \vect{ AX }=\begin{pmatrix}
        1    \\ 
        -3    
    \end{pmatrix}\).

    Faire l'exercice 27 de la page 213.

\corrref{smath-0590}
\end{exercice}
