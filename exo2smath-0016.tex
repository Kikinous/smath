% This is part of Un soupçon de mathématique sans être agressif pour autant
% Copyright (c) 2014
%   Laurent Claessens
% See the file fdl-1.3.txt for copying conditions.

\begin{exercice}[\cite{AXYJooDyxYjv}]\label{exo2smath-0016}

-- Les Aanderson vont nous rendre visite ce soir, annonce Monsieur Blum. 

-- Toute la famille, donc Monsieur et Madame Aanderson et leurs trois fils Antoine, Bernard et Claude ? demande Madame Blum craintive.

Monsieur Blum, qui ne rate pas une occasion de provoquer sa femme :

-- Non, pas du tout. Je t’explique. Si le père Aanderson vient, alors il emmène aussi sa femme. Au moins un des deux fils Claude et Bernard vient. Soit Madame Aanderson vient, soit Antoine vient. Soit Antoine et Bernard viennent tous les deux, soit ils ne viennent pas. Et si Claude vient, alors Bernard et Monsieur Aanderson aussi.

Quelles conclusion peut tirer Madame Blum sur qui viendra ce soir ? 

\corrref{2smath-0016}
\end{exercice}
