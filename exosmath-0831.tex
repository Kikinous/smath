% This is part of Un soupçon de mathématique sans être agressif pour autant
% Copyright (c) 2014
%   Laurent Claessens
% See the file fdl-1.3.txt for copying conditions.

\begin{exercice}[\ldots /3]\label{exosmath-0831}

Un nombre divisible en même temps par \( 3\) et par \( 7\) est un multiple de \( 42\). Cet énoncé est faux; le(s)quel(s) des nombres suivants donne un contre-exemple ? Expliquez votre choix.
\begin{multicols}{2}
    \begin{enumerate}
        \item
            \( 21\)
        \item
            \( 42\)
        \item
            \( 126\)
        \item
            \( 63\)
    \end{enumerate}
\end{multicols}

\corrref{smath-0831}
\end{exercice}
