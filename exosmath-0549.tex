% This is part of Un soupçon de mathématique sans être agressif pour autant
% Copyright (c) 2013-2014
%   Laurent Claessens
% See the file fdl-1.3.txt for copying conditions.

\begin{exercice}\label{exosmath-0549}

    Ci-dessous se trouve le tableau de variations d'une fonction \( f\) dont nous notons \( C_f\) la courbe représentative dans un repère du plan. :
    \begin{equation*}
        \begin{array}[]{|c||ccccccc|}
            \hline
            x&-2&&0&&2&&6\\
            \hline\hline
            &5&&&&3&&\\
            f(x)&&\searrow&&\nearrow&&\searrow&\\
            &&&0&&&&-1\\
            \hline
        \end{array}
    \end{equation*}
    Vrai ou faux (justifier) ?
    \begin{enumerate}
        \item
            Le maximum de $f$ est \( 6\).
        \item
            La courbe \( C_f\) passe par le point de coordonnées \( (2;3)\).
        \item
            Le minimum de \( f\) est atteint en \( x=6\).
        \item
            \( f(2.5)<f(2.6)\).
        \item
            \( f(-1)=f(1)\)
        \item
            L'image de \( 5\) par \( f\) est \( -2\).
        \item
            \( f\) est croissante sur \( \mathopen[ 0 ; 3 \mathclose]\).
        \item
            L'équation \( f(x)=2\) admet \( 3\) solutions exactement.
    \end{enumerate}

    Faire à la maison le numéro 20, page 48.

\corrref{smath-0549}
\end{exercice}
