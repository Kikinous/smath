% This is part of Un soupçon de mathématique sans être agressif pour autant
% Copyright (c) 2012
%   Laurent Claessens
% See the file fdl-1.3.txt for copying conditions.

\begin{exercice}\label{exoSeconde-0018}


    \begin{enumerate}
        \item
Sur la figure \ref{LabelFigBpCNVm}, quelles sont (en fonction de \( x\)) la longueur, la largeur et la surface du rectangle ?
\newcommand{\CaptionFigBpCNVm}{figure de l'exercice \ref{exoSeconde-0018}} 
\input{Fig_BpCNVm.pstricks}
        \item
            Factoriser
            \begin{equation}
                2ax+2x   
            \end{equation}
        \item
            Résoudre
            \begin{equation}
                \frac{ 100 }{ x }=2
            \end{equation}
        \item
            Calculer
            \begin{equation}
                \frac{ 7 }{ 8 }\times\frac{ 7 }{ 3 }.
            \end{equation}
        \item
            Mettre au même dénominateur
            \begin{equation}
                \frac{1}{ x+1 }+\frac{1}{ x-1 }.
            \end{equation}
    \end{enumerate}

\corrref{Seconde-0018}
\end{exercice}
