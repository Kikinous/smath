% This is part of Un soupçon de mathématique sans être agressif pour autant
% Copyright (c) 2014
%   Laurent Claessens
% See the file fdl-1.3.txt for copying conditions.

\begin{exercice}\label{exosmath-0982}

    Recopier et compléter les pointillés :
    \begin{enumerate}
        \item
        $2\stackrel{\times x}{\longrightarrow}\ldots\stackrel{-1}{\longrightarrow}\ldots\stackrel{\times 3}{\longrightarrow}\ldots\stackrel{-4}{\longrightarrow}\ldots$
    \item
        \( -x\stackrel{\times 3}{\longrightarrow}\ldots\stackrel{\times 5}{\longrightarrow}\ldots\stackrel{+2}{\longrightarrow}  \ldots\stackrel{\times 4}{\longrightarrow}\ldots\)
    \item
        \( <++>\stackrel{ <++> }{\longrightarrow}<++>\stackrel{<++>}{\longrightarrow}<++>\stackrel{<++>}{\longrightarrow}  <++>\stackrel{<++>}{\longrightarrow}<++>\)
    \item
        \( <++>\stackrel{ <++> }{\longrightarrow}<++>\stackrel{<++>}{\longrightarrow}<++>\stackrel{<++>}{\longrightarrow}  <++>\stackrel{<++>}{\longrightarrow}<++>\)
    \item
        \( <++>\stackrel{ <++> }{\longrightarrow}<++>\stackrel{<++>}{\longrightarrow}<++>\stackrel{<++>}{\longrightarrow}  <++>\stackrel{<++>}{\longrightarrow}<++>\)

    \end{enumerate}
\corrref{smath-0982}
\end{exercice}
