% This is part of Un soupçon de mathématique sans être agressif pour autant
% Copyright (c) 2012-2013
%   Laurent Claessens
% See the file fdl-1.3.txt for copying conditions.

\begin{exercice}\label{exosmath-0006}

    \begin{multicols}{2}
    Soit la fonction \( f\) donnée par le graphique ci-contre.
        \begin{enumerate}
            \item
                Donner l'ensemble de définition de \( f\).
            \item\label{ItemLREjzsg}
                Donner les solutions de \( f(x)=0\).
            \item 
                Tracer la droite \( (AB)\).
            \item\label{ItemODPJesm}
                Si \( g\) est la fonction dont le graphe est la droite \( (AB)\), résoudre \( f(x)=g(x)\).
            \item
                Calculer les coordonnées du milieu du segment \( [AB]\).
        \end{enumerate}


        \columnbreak
%The result is on figure \ref{LabelFigGraphInterfQVfSf}. % From file GraphInterfQVfSf
%\newcommand{\CaptionFigGraphInterfQVfSf}{<+Type your caption here+>}
\input{Fig_GraphInterfQVfSf.pstricks}

    \end{multicols}
    Les réponses aux questions \ref{ItemLREjzsg} et \ref{ItemODPJesm} peuvent être approximatives.
        

\corrref{smath-0006}
\end{exercice}
