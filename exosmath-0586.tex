% This is part of Un soupçon de mathématique sans être agressif pour autant
% Copyright (c) 2013
%   Laurent Claessens
% See the file fdl-1.3.txt for copying conditions.

\begin{exercice}\label{exosmath-0586}

    Nathalie pense qu'en lançant deux dés on obtient plus souvent le huit que le six. Elle a écrit le programme suivant :

    \begin{fmpage}{0.9\linewidth}

    \( d=0\)

    \( n=0\)

    Tant que \( d\) est différent de \( 8\), faire :

    \hspace{0.5cm} \( n=n+1\)

    \hspace{0.5cm} \( a\) est pris au hasard dans \( \{ 1,2,3,4,5,6 \}\)

    \hspace{0.5cm} \( b\) est pris au hasard dans \( \{ 1,2,3,4,5,6 \}\)

    \hspace{0.5cm} \( d\) prend la valeur \( a+b\)

    Écrire \( n\)

\end{fmpage}

\begin{enumerate}
    \item
        Écrire ce programme sur l'ordinateur et le tester.
    \item
        Le modifier pour avoir une idée de la fréquence du \( 8\) et du \( 6\) lorsqu'on lance deux dés.
    \item
        Finalement, est-ce le \( 6\) ou le \( 8\) qui est le plus fréquent ?
\end{enumerate}

\corrref{smath-0586}
\end{exercice}
