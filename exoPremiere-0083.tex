% This is part of Un soupçon de mathématique sans être agressif pour autant
% Copyright (c) 2012
%   Laurent Claessens
% See the file fdl-1.3.txt for copying conditions.

\begin{exercice}\label{exoPremiere-0083}

    Deux courriers électroniques sur trois sont du spam. Notre expérience consiste à tirer $200$ courriers électroniques au hasard dans une base de trente mille.
    \begin{enumerate}
        \item
            Justifier que cette situation est un schéma de Bernoulli.
        \item
            Quelle est la probabilité que \emph{tous} les courriels tirés soient du spam ?
        \item
            Quelle est la probabilité que \emph{aucun} des courriels tirés ne soient du spam ?
        \item
            Quelle est la probabilité d'avoir un seul spam parmi les \( 200\) sélectionnés ?
        \item
            Quelle est la probabilité d'avoir un seul non spam parmi les \( 200\) sélectionnés ?
    \end{enumerate}

\corrref{Premiere-0083}
\end{exercice}
