% This is part of Un soupçon de mathématique sans être agressif pour autant
% Copyright (c) 2012
%   Laurent Claessens
% See the file fdl-1.3.txt for copying conditions.

\begin{exercice}\label{exosmath-0238}

    Un berger Syldave s'entraine pour le championnat national du lancer de chèvre. L'épreuve consiste à lancer une chèvre vers le haut depuis le bord d'une falaise au sommet d'une falaise située au bord d'un lac tranquille. La hauteur de la chèvre en fonction du temps (en secondes) par rapport à la surface du lac tranquille est une fonction \( f\) donnée par le graphique suivant.

    \begin{center}
%The result is on figure \ref{LabelFigfigureXCScSiP}. % From file figureXCScSiP
%\newcommand{\CaptionFigfigureXCScSiP}{<+Type your caption here+>}
\input{Fig_figureXCScSiP.pstricks}
    \end{center}
    À partir du graphique :
    \begin{enumerate}
        \item
            À quelle hauteur se trouve la chèvre au moment du lancer ?
        \item
            Pendant combien de temps la chèvre reste à une hauteur suppérieure à celle à laquelle elle a été lancée ?
        \item
            À quel moment la chèvre atteint-elle sa hauteur maximale ? Quelle est cette hauteur ?
        \item
            Au bout de combien de temps la chèvre touche-t-elle la surface de l'eau ?
        \item
            Résumer toutes ces informations en dressant le tableau de variation de la fonction \( f\).
    \end{enumerate}
    Quelque questions théoriques. Nous considérons la fonction
            \begin{equation}
                \begin{aligned}
                    f\colon \mathopen[ 0 , 3 \mathclose]&\to \eR \\
                    t&\mapsto -5t^2+10t+15. 
                \end{aligned}
            \end{equation}
            
    \begin{enumerate}
        \item
            Vérifier que \( f\) peut s'écrire
            \begin{equation}
                f(t)=20-5(x-1)^2
            \end{equation}
        \item
            Factoriser \( f\) et résoudre l'équation \( f(t)=0\).
        \item
            Interpréter les solutions. Une des deux solutions trouvées est à rejeter en ce qui concerne le problème de berger syldave qui nous intéresse. Pourquoi ?
        \item
            Résoudre l'équation \( f(x)=15\). Que représentent les solutions dans le lancer de chèvre ?
        \item
            Du pied de la falaise, un concurrent mal intentionné veut toucher la chèvre avec un lance-pierre. Il tire verticalement et la hauteur de la pierre en fonction du temps est donnée par la fonction \( g(x)=10x\).
            \begin{enumerate}
                \item
                    Représenter \( g\) sur le même graphique que $f$.
                \item
                    Déterminer par calcul le moment et la hauteur à laquelle la chèvre sera touchée. Montrer sur le graphique le point correspondant.
            \end{enumerate}
    \end{enumerate}

%AFAIRE : citer la page de Guillaume Connan

\corrref{smath-0238}
\end{exercice}
