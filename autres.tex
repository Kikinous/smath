% This is part of Un soupçon de mathématique sans être agressif pour autant
% Copyright (c) 2013
%   Laurent Claessens
% See the file fdl-1.3.txt for copying conditions.

%+++++++++++++++++++++++++++++++++++++++++++++++++++++++++++++++++++++++++++++++++++++++++++++++++++++++++++++++++++++++++++ 
\section{Exercices pour TSTL}
%+++++++++++++++++++++++++++++++++++++++++++++++++++++++++++++++++++++++++++++++++++++++++++++++++++++++++++++++++++++++++++
% Les quatre premiers ont été donnés en colle.
\Exo{smath-0283}
\Exo{smath-0284}
\Exo{smath-0285}
\Exo{smath-0286}
\Exo{smath-0415}
\Exo{smath-0416}

%+++++++++++++++++++++++++++++++++++++++++++++++++++++++++++++++++++++++++++++++++++++++++++++++++++++++++++++++++++++++++++ 
\section{Sujets de dissertations}
%+++++++++++++++++++++++++++++++++++++++++++++++++++++++++++++++++++++++++++++++++++++++++++++++++++++++++++++++++++++++++++

\Exo{smath-0418}
\Exo{smath-0419}
\Exo{smath-0420}    % Celui-ci est un peu nul.

\Exo{smath-0423}
\Exo{smath-0424}
\Exo{smath-0425}
\Exo{smath-0426}
\Exo{smath-0427}
\Exo{smath-0428}
\Exo{smath-0429}
\Exo{smath-0430}


\Exo{smath-0442}
\Exo{smath-0446}
\Exo{smath-0454}
\Exo{smath-0455}
\Exo{smath-0456}
\Exo{smath-0457}
\Exo{smath-0459}

%+++++++++++++++++++++++++++++++++++++++++++++++++++++++++++++++++++++++++++++++++++++++++++++++++++++++++++++++++++++++++++ 
\section{Suites arithmétiques}
%+++++++++++++++++++++++++++++++++++++++++++++++++++++++++++++++++++++++++++++++++++++++++++++++++++++++++++++++++++++++++++

Une suite arithmétique peut également être écrite sous forme «non récurrence». Soit la suite arithmétique
\begin{subequations}
    \begin{numcases}{}
        u_0=2\\
        u_{n+1}=u_n+3.
    \end{numcases}
\end{subequations}
Nous avons \( u_1=2+3\), \( u_2=2+3+3\), \( u_3=2+3+3+3\), etc. Donc \( u_n=2+3n\). Plus généralement pour une suite dont le terme initial est \( u_0\) et la raison vaut \( a\), nous avons
\begin{equation}
    u_n=u_0+na.
\end{equation}
Si par contre le terme initial est \( u_1\), alors nous avons
\begin{equation}
    u_n=u_1+(n-1)a.
\end{equation}
Ce qui est important à retenir est que la raison d'une suite arithmétique est le multiple de \( n\).

%+++++++++++++++++++++++++++++++++++++++++++++++++++++++++++++++++++++++++++++++++++++++++++++++++++++++++++++++++++++++++++ 
\section{Pour la dérivation}
%+++++++++++++++++++++++++++++++++++++++++++++++++++++++++++++++++++++++++++++++++++++++++++++++++++++++++++++++++++++++++++

\Exo{smath-0300}
\Exo{smath-0301}




