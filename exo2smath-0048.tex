% This is part of Un soupçon de mathématique sans être agressif pour autant
% Copyright (c) 2014
%   Laurent Claessens
% See the file fdl-1.3.txt for copying conditions.

\begin{exercice}\label{exo2smath-0048}

Dans chacun des cas suivants, dire quel est le plus grand des deux nombres :
\begin{multicols}{2}
    \begin{enumerate}
        \item
            \( 4\) et \( 4.5\)
        \item
            \( 6\) et \( \opp(6)\)
        \item
            \( 4\) et \( \opp(2)\)
        \item
            \( 4\) et \( \opp(10)\)
        \item
            \( 1\) et \( \opp(3)\)
        \item
            \( 0\) et \( 3\)
        \item
            \( 0\) et \( \opp(3)\)
        \item
            \( \opp(4)\) et \( \opp(4.5)\)
    \end{enumerate}
\end{multicols}

\corrref{2smath-0048}
\end{exercice}
