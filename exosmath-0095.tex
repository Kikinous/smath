% This is part of Un soupçon de mathématique sans être agressif pour autant
% Copyright (c) 2012-2013
%   Laurent Claessens
% See the file fdl-1.3.txt for copying conditions.

\begin{exercice}\label{exosmath-0095}

\begin{wrapfigure}{r}{2.5cm}
   \vspace{-0.5cm}        % à adapter.
   \centering
   \input{Fig_JOKJMTD.pstricks}
\end{wrapfigure}

    Un cornet de glace avec une boule est représenté par un cône de \unit{10}{\centi\meter} de hauteur surmonté d'une demi-sphère de crème glacée de \unit{5}{\centi\meter} de diamètre.

    Le dessin ci-contre représente la situation (très) schématiquement.

    \begin{enumerate}
        \item
            Reporter les mesures de l'énoncé sur le dessin.
        \item
            Donner le volume total du cornet et de la glace en \centi\cubic\meter.
    \end{enumerate}

\corrref{smath-0095}
\end{exercice}
