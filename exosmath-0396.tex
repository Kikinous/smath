% This is part of Un soupçon de mathématique sans être agressif pour autant
% Copyright (c) 2013
%   Laurent Claessens
% See the file fdl-1.3.txt for copying conditions.

\begin{exercice}[\cite{SMPaVRN}]\label{exosmath-0396}

    Nous prélevons \( 250\) pièces produites par une machine, et nous contrôlons la qualité. Nous remarquons alors que \( 10\) des pièces prélevées sont trop lourdes tandis que \( 15\) sont trop légères. Nous admettons que l'échantillon est représentatif de l'ensemble de la production de la machine.

    \begin{enumerate}
        \item
            Si nous prenons une pièce au hasard parmi les pièces produites, quelle est la probabilité que la masse ne soit pas correcte ? (trop lourde ou trop légère)
        \item
            Sachant que la pièce prélevée a une masse inadéquate, quelle est la probabilité qu'elle soit trop lourde ?
    \end{enumerate}
    Pour répondre, ne pas hésiter à dessiner des ensembles sous forme de patates.

\corrref{smath-0396}
\end{exercice}
