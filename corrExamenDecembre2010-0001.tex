% This is part of Exercices de mathématique pour SVT
% Copyright (c) 2011
%   Laurent Claessens et Carlotta Donadello
% See the file fdl-1.3.txt for copying conditions.

\begin{corrige}{ExamenDecembre2010-0001}

	\begin{enumerate}
		\item
			Le graphique est un segment de droite entre les points $(0;0)$ et $(5;5)$.
		\item
			\begin{equation}
				\int_0^5xdx=\left[ \frac{ x^2 }{2} \right]^5_0=\frac{ 25 }{2}-0=\frac{ 25 }{2}.
			\end{equation}
		\item
			%La fonction partie entière est tracée sur la figure \ref{LabelFigPartieEntiere}.
			%\newcommand{\CaptionFigPartieEntiere}{La fonction partie entière.}
			%\input{Fig_PartieEntiere.pstricks}
            %TODO : faire fonctionner cette figure
		\item
			En découpant l'intervalle d'intégration, nous avons le calcul suivant :
			\begin{equation}
				\begin{aligned}[]
					\int_0^5[x]dx&=\int_1^1 0dx+\int_1^2 1dx+\int_2^32dx+\int_3^43dx+\int_4^54dx\\
					&=1+2+3+4=10.
				\end{aligned}
			\end{equation}
			Notez que $[x]=2$ pour $x$ entre $2$ et $3$. C'est pour cela que nous avons
			\begin{equation}
				\int_2^3[x]dx=\int_2^32dx.
			\end{equation}
			
		\item
			%La fonction mantisse est tracée sur la figure \ref{LabelFigMantisse}.
			%\newcommand{\CaptionFigMantisse}{La fonction mantisse. Remarquez que du point de vue de la surface, ce sont des petits triangles.}
			%\input{Fig_Mantisse.pstricks}
            %TODO : faire fonctionner cette figure
		\item
			Nous pouvons utiliser les résultats précédents :
			\begin{equation}
				\int_0^5x-[x]dx=\int_0^5xdx-\int_0^5[x]dx=\frac{ 25 }{ 2 }-10=\frac{ 5 }{2}.
			\end{equation}
			%Une autre façon est de se souvenir que cette intégrale représente la surface sous la courbe (voir figure \ref{LabelFigMantisse}). Cette surface est formée de cinq triangles dont les surfaces valent $\frac{ 1 }{2}$; pour rappel, la surface d'un triangle est donnée par $\frac{ \text{base}\times\text{hauteur} }{2}$. Donc la surface totale est de $\frac{ 5 }{2}$.
            % TODO : remettre cette ligne lorsque la figure sera refaite.

	\end{enumerate}

\end{corrige}
