% This is part of Un soupçon de mathématique sans être agressif pour autant
% Copyright (c) 2012
%   Laurent Claessens
% See the file fdl-1.3.txt for copying conditions.

\begin{corrige}{Seconde-0016}

  
    En nous inspirant des tableaux de la question \ref{Seconde-0037}, nous pouvons proposer la répartition suivante qui laisse le salaire médian $1\,500$ euros :

  \begin{center}
  \begin{tabular}{|c||c|c|c|c|c|c|c|c|c|c|}
    \hline 
    \textbf{Salaire} &900&1\,100&1\,300&1\,500&1\,700&1\,900&2\,100&2\,500&3\,100&4\,500\\
    \hline 
    \textbf{Effectif} &10&10&10&20&5&5&5&2&2&20\\
    \hline 
    \textbf{ECC} &10&20&30&50&55&60&65&67&69&89\\
    \hline
  \end{tabular}
  \end{center}

  (On rappelle qu'avec 89 valeurs, la médiane est la donnée de rang
  45).

  Une repartions moins réaliste mais qui convient également serait la suivante :
  \begin{center}
      \begin{tabular}[h]{|c||c|c|c|}
          \hline
          \textbf{Salaire} & 900&1500&5400\\
          \hline
          \textbf{Effectif}&20&49&20\\
          \hline
      \end{tabular}
  \end{center}

\end{corrige}
