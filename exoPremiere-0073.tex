% This is part of Un soupçon de mathématique sans être agressif pour autant
% Copyright (c) 2013
%   Laurent Claessens
% See the file fdl-1.3.txt for copying conditions.

\begin{exercice}\label{exoPremiere-0073}

    Donner les coordonnées du sommet de la parabole d'équation \( y=2x^2-8x+3\). Donner le tableau de variations de la fonction \( f(x)=2x^2-x+3\).

    Pour rappel, l'abscisse du sommet de la parabole \( y=ax^2+bx+c\) est \(x_0=-b/2a\). {\tiny Au cas où cet exercice arriverait en DS, il n'est pas certain que ce rappel soit encore là.}

\corrref{Premiere-0073}
\end{exercice}
