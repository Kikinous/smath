% This is part of Un soupçon de mathématique sans être agressif pour autant
% Copyright (c) 2013
%   Laurent Claessens
% See the file fdl-1.3.txt for copying conditions.

% Ce fichier contient les choses sur les équations de droites pour les premières stmg.

Ce chapitre a pour objectif de remettre en place les notions utiles à propos des équations de droites vues en seconde. En principe, ceci ne contiendra rien de nouveau pour les élèves de première.

%+++++++++++++++++++++++++++++++++++++++++++++++++++++++++++++++++++++++++++++++++++++++++++++++++++++++++++++++++++++++++++ 
\section{Théorie}
%+++++++++++++++++++++++++++++++++++++++++++++++++++++++++++++++++++++++++++++++++++++++++++++++++++++++++++++++++++++++++++

%--------------------------------------------------------------------------------------------------------------------------- 
\subsection{Définitions}
%---------------------------------------------------------------------------------------------------------------------------

Une droite (non verticale) est le graphe d'une fonction affine et a pour équation
\begin{equation}
    y=mx+p
\end{equation}
pour certains \( m\) et \( p \) réels.
\begin{enumerate}
    \item
        \( m\) est le \defe{coefficient directeur}{coefficient directeur} de la droite.
    \item
        \( p\) est son \defe{ordonnée à l'origine}{ordonnée!à l'origine} de la droite.
    \item
        Plus \( m\) est grand, plus la pente de la droite est forte.
    \item
        Si \( m\) est positif, alors la droite monte; si \( m\) est négatif, alors la droite descend.
\end{enumerate}

%--------------------------------------------------------------------------------------------------------------------------- 
\subsection{Lecture graphique}
%---------------------------------------------------------------------------------------------------------------------------

\begin{minipage}{0.485\textwidth}
Pour trouver l'équation d'une droite en connaissant le graphique, on pose l'équation \( y=mx+p\) et on cherche \( m\) et \( p\).
\begin{enumerate}
    \item
        Trouver \( p\) est facile : c'est l'ordonnée de l'intersection de la droite avec l'axe vertical.
    \item
        Pour trouver \( m\), on prend deux points au hasard sur la droite et on mesure les \( \Delta x\) et \( \Delta y\). Ensuite nous avons
        \begin{equation}
            m=\frac{ \Delta y }{ \Delta x }.
        \end{equation}
\end{enumerate}
\end{minipage}
\hspace{1mm}
\begin{minipage}{0.485\textwidth}
%The result is on figure \ref{LabelFigXSMDwcv}. % From file XSMDwcv
%\newcommand{\CaptionFigXSMDwcv}{<+Type your caption here+>}
\begin{center}
\input{Fig_XSMDwcv.pstricks}
\end{center}
\end{minipage}

% TODO : tous les \columnbreak doivent être remplacés par cette technique de minipage que j'ai trouvée ici:
% http://forum.mathematex.net/latex-f6/mettre-du-texte-a-cote-d-une-figue-geometrique-t13382.html
% par Zobra

Sur le dessin, nous avons \( p=3\) et nous mesurons \( \Delta x=2\), \( \Delta y=1\), donc \( m=\frac{ 1 }{2}\). L'équation de la droite dessinée est donc
\begin{equation}
    y=\frac{ 1 }{2}x+3.
\end{equation}

%--------------------------------------------------------------------------------------------------------------------------- 
\subsection{Calculs}
%---------------------------------------------------------------------------------------------------------------------------

\begin{example}
    Trouver l'équation de la droite passant par le point \( A=(5;7)\) et de coefficient directeur \( m=-3\). 

    L'équation de la droite sera \( y=-3x+p\) avec un \( p\) encore inconnu. Étant donné que le point \( (5;7)\) soit être sur la droite, nous avons l'équation suivant pour \( p\) :
    \begin{equation}
        7=-3\times 5+p,
    \end{equation}
    c'est à dire \( p=7+15=22\). Au final l'équation est
    \begin{equation}
        y=-3x+22.
    \end{equation}
\end{example}


%--------------------------------------------------------------------------------------------------------------------------- 
\subsection{Exercices préliminaires}
%---------------------------------------------------------------------------------------------------------------------------

\Exo{smath-0388}
\Exo{smath-0389}
\Exo{smath-0393}
\Exo{smath-0390}

%--------------------------------------------------------------------------------------------------------------------------- 
\subsection{Exercices pour s'entrainer}
%---------------------------------------------------------------------------------------------------------------------------

\Exo{smath-0395}
\Exo{smath-0354}
\Exo{smath-0405}
