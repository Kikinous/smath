% This is part of Un soupçon de mathématique sans être agressif pour autant
% Copyright (c) 2014
%   Laurent Claessens
% See the file fdl-1.3.txt for copying conditions.

\begin{exercice}[\ldots\ldots/4]\label{exosmath-0903}

    Sur une classe de \( 24\) élèves dont \( 14\) filles, huit élèves ont raté un devoir. Quel(s) diagramme(s) indique(nt) cette proportion ?

\begin{center}
\input{Fig_XYXooNdhQer0.pstricks}
\input{Fig_XYXooNdhQer1.pstricks}
\input{Fig_XYXooNdhQer2.pstricks}
\end{center}
Justifier vos choix.

Quelle est la proportion de garçons dans cette classe ?
\begin{multicols}{4}
    \begin{enumerate}
        \item
            \( \dfrac{ 8 }{ 24 }\)
        \item
            \( \dfrac{ 14 }{ 10 }\)
        \item
            \( \dfrac{ 10 }{ 24 }\).
        \item
            \( \dfrac{ 5 }{ 12 }\)
    \end{enumerate}
\end{multicols}
Indiquer toutes les réponses possibles.


\corrref{smath-0903}
\end{exercice}
