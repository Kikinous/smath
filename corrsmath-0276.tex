% This is part of Un soupçon de mathématique sans être agressif pour autant
% Copyright (c) 2013
%   Laurent Claessens
% See the file fdl-1.3.txt for copying conditions.

\begin{corrige}{smath-0276}

\begin{wrapfigure}{r}{7.0cm}
            \vspace{-2cm}        % à adapter.
                \centering
                    \input{Fig_SWrwUzA.pstricks}
\end{wrapfigure}

La première chose à faire est de bien noter que \( x\neq 0\). Donc si zéro arrive dans les solutions, il faudra l'exclure.

ZCumAwy

Les points qui nous intéressent sur le graphe de la courbe \( y=\frac{1}{ x }\) sont ceux dont les ordonnées sont comprises entre \( -1/10\) et \( 3/4\).

Nous mettons les solutions des deux inéquations sur un seul dessin pour voir quels sont les solutions communes.

\end{corrige}
