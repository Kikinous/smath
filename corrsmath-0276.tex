% This is part of Un soupçon de mathématique sans être agressif pour autant
% Copyright (c) 2013
%   Laurent Claessens
% See the file fdl-1.3.txt for copying conditions.

\begin{corrige}{smath-0276}


    La première chose à faire est de bien noter que \( x\neq 0\). Donc si zéro arrive dans les solutions, il faudra l'exclure. Les points qui nous intéressent sur le graphe de la courbe \( y=\frac{1}{ x }\) sont ceux dont les ordonnées sont comprises entre \( -1/10\) et \( 3/4\). Sur la figure\footnote{Qui n'est pas dessinée à l'échelle; d'ailleurs il est sans espoir de dessiner une hyperbole à l'échelle correctement.} \ref{LabelFigAVFexUi}, nous avons repéré en cyan les ordonnées de \( -1/10\) à $3/4$ et en rouge les points correspondants de l'hyperbole.

\newcommand{\CaptionFigAVFexUi}{Résolution de l'exercice \ref{exosmath-0276}.}
\input{Fig_AVFexUi.pstricks}

Une simple lecture de graphique nous donne donc les solutions :
\begin{equation}
    x\in\mathopen] -\infty , -10 \mathclose]\cup \mathopen[ \frac{ 4 }{ 3 } , \infty [.
\end{equation}
Il se fait que \( x=0\) n'est pas dans cet ensemble; nous ne devons donc rien exclure.

\end{corrige}
