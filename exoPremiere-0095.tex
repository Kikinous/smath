% This is part of Un soupçon de mathématique sans être agressif pour autant
% Copyright (c) 2012
%   Laurent Claessens
% See the file fdl-1.3.txt for copying conditions.

\begin{exercice}\label{exoPremiere-0095}

    Soit \( f(x)=x^2+4x-5\).
    \begin{multicols}{2}
        \begin{enumerate}
            \item
                Parmi les nombres \( -1\), \( -5\), \( 0\), \( 1\), \( 7\), lesquels vérifient \( x^2+4x-5=0\) ?
            \item
                Pour quelles valeurs de \( A\) et \( B\) peut-on écrire \( f(x)=(x-A)(x-B)\) ?
            \item
                En quel point se situe le sommet de la parabole représentative de $f$ ?
        \end{enumerate}
    \end{multicols}

\corrref{Premiere-0095}
\end{exercice}
