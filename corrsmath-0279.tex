% This is part of Un soupçon de mathématique sans être agressif pour autant
% Copyright (c) 2013
%   Laurent Claessens
% See the file fdl-1.3.txt for copying conditions.

\begin{corrige}{smath-0279}

    \begin{enumerate}
        \item
            L'expérience aléatoire consiste à tirer un point au hasard dans la grille de la figure \ref{LabelFigKMJQWwassLabelSubFigKMJQWwa0}. Il y a \( 100\) points possibles et un seul «gagnant». La probabilité de tirer le point \( (3;7)\) est donc de \( \frac{1}{ 100 }\).
        \item
            Sur la seconde figure, nous avons représenté la droite \( y=x+5\) et mis en plus gros les points de cette droites qui pourraient être tirés lors de l'expérience aléatoire. Sur les \( 100\) points possibles, \( 5\) sont sur la droite \( y=x+5\), donc la probabilité de tirer un point sur cette droite est de \( \frac{ 5 }{ 100 }\).
        \item
            Nous avons représenté sur la figure \ref{LabelFigKMJQWwassLabelSubFigKMJQWwa2} la droite passant par \( (0;0)\) et par \( (1;2)\). Et en rouge le point \( (1;2)\) lui-même.
            
            Pour réussir l'expérience qui consiste à ce que le point \( (1;2)\) soit sur la droite \( (OA)\), il faut tirer un des points de cette droite. Il y en a \( 5\), donc la probabilité est de \( \frac{ 5 }{ 100 }\).
    \end{enumerate}

%The result is on figure \ref{LabelFigKMJQWwa}. % From file KMJQWwa
    \newcommand{\CaptionFigKMJQWwa}{Les dessins de l'exercice \ref{exosmath-0279}.}
\input{Fig_KMJQWwa.pstricks}

\end{corrige}
