% This is part of Un soupçon de mathématique sans être agressif pour autant
% Copyright (c) 2012
%   Laurent Claessens
% See the file fdl-1.3.txt for copying conditions.

\begin{exercice}\label{exosmath-0065}

    Compléter les égalités suivantes pour qu'elles soient vraies.
    \begin{multicols}{3}
        \begin{enumerate}
            \item
                \( \vect{ IJ }=\vect{ IB }+\vect{ B\,. }\) 
            \item
                \( \vect{ H\,. }=\vect{ .\,.\vphantom{H} }+\vect{ IJ }\)
            \item
                \( \vect{ AB }+\vect{ BC }+\vect{ CD }=\vect{ .\,.\vphantom{H} }\)
            \item
                \( \vect{ XK }=\vect{ XL }+\vect{ .\,K }\)
            \item
                \( \vect{ RS }=\vect{ R\,. }+\vect{ .\,S }\)
            \item
                \( \vect{ AB }+\vect{ BA }=\vect{ .\vphantom{H} }\)
            \item
                \( \vect{ AB }+\vect{ CA }=\vect{ .\,.\vphantom{H} }\)
            \item
                \( \vect{ PQ }-\vect{ PR }=\vect{ .\,.\vphantom{H} }\)
        \end{enumerate}
    \end{multicols}

\corrref{smath-0065}
\end{exercice}
