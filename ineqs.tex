% This is part of Un soupçon de mathématique sans être agressif pour autant
% Copyright (c) 2012-2013
%   Laurent Claessens
% See the file fdl-1.3.txt for copying conditions.

Avec les paquets, une intéressante est de mettre \( 3x\) et \( a\) dans le premier paquet, et \( 2x\) et \( 4a\) dans le second paquet, comme ça on peut factoriser un \( 5\).

Et pour la fin :
\begin{equation}
    (2a+1)(1-x)+5(x-1),
\end{equation}
parce qu'il y a encore un \( 2\) qui se factorise.

%+++++++++++++++++++++++++++++++++++++++++++++++++++++++++++++++++++++++++++++++++++++++++++++++++++++++++++++++++++++++++++
\section{Exercices}
%+++++++++++++++++++++++++++++++++++++++++++++++++++++++++++++++++++++++++++++++++++++++++++++++++++++++++++++++++++++++++++

\Exo{smath-0341}
\Exo{smath-0324}
\Exo{smath-0343}
\Exo{smath-0340}
\Exo{smath-0325}
\Exo{smath-0319}
\Exo{smath-0408}
\Exo{smath-0320}
\Exo{smath-0322}    % Cet exo est à mettre autre part.
\Exo{smath-0342}
\Exo{smath-0352}
\Exo{smath-0353}
\Exo{smath-0370}
\Exo{smath-0371}
\Exo{smath-0372}
