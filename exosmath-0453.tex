% This is part of Un soupçon de mathématique sans être agressif pour autant
% Copyright (c) 2013
%   Laurent Claessens
% See the file fdl-1.3.txt for copying conditions.

\begin{exercice}\label{exosmath-0453}

    Nous considérons une classe de \( 30\) élèves. Si nous en tirons un au hasard, nous considérons les événements  \( A\) : «l'élève tiré est une fille»  et \( B\) : «l'élève tiré porte des lunettes». Nous supposons que la classe contienne \( 18\) garçons et \( 7\) élèves portant des lunettes. De plus nous savons que \( 3\) filles portent des lunettes.
    \begin{enumerate}
        \item
            Quelles sont les probabilités \( P(A)\) et \( P(B)\) ?
        \item
            Décrire en français les événements \( \bar A\) et \( \bar B\).
        \item
            Décrire en français et donner la probabilité des événements \( A\cup\bar B\).
    \end{enumerate}

\corrref{smath-0453}
\end{exercice}
