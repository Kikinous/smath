% This is part of Un soupçon de mathématique sans être agressif pour autant
% Copyright (c) 2013
%   Laurent Claessens
% See the file fdl-1.3.txt for copying conditions.

\begin{exercice}\label{exosmath-0571}

    Questions type «calcul mental» (mais il faut quand même un peu de justification).
    \begin{enumerate}
        \item
            Simplifier la fraction
            \begin{equation*}
                \frac{ 3x^2+7x }{ 18x }.
            \end{equation*}
        \item
            Donner la valeur exacte de la longueur d'un rectangle dont la largeur est \unit{6}{\centi\meter} et dont l'aire est \unit{23}{\centi\square\meter}.
        \item
            Est-ce que \( 1\) est une solution de l'équation suivante ?
            \begin{equation*}
                5x^2(x-1)=0.
            \end{equation*}
        \item
            Donner un exemple de suite de nombres dont la médiane est plus grande que la moyenne.
        \item 
            Si \( f(x)=\frac{ \displaystyle x+3 }{ \displaystyle -x }\), combien vaut \( f(-4)\) ?
    \end{enumerate}

\corrref{smath-0571}
\end{exercice}
