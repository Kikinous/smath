% This is part of Un soupçon de mathématique sans être agressif pour autant
% Copyright (c) 2014
%   Laurent Claessens
% See the file fdl-1.3.txt for copying conditions.

\begin{corrige}{smath-0889}

        \begin{enumerate}
            \item
                Il s'agit de d'abord effectuer la division (la fraction) : \( \dfrac{ 10 }{ 5 }=2\). Donc \( 4-\dfrac{ 10 }{ 5 }=4-2=2\)
            \item
                \( 5\times 10+10=50+10=60\).
            \item
                Il faut effectuer d'abord la parenthèse : \( 12+88=100\), donc
                \( (12+88)\times 30=100\times 30=3000\).
            \item
                \( \dfrac{ 33\times 44 }{ 44 }=33\) : il y a une simplification directe par \( 44\).
            \item
                \( 60\times a=6\times 10 \times a\)
        \end{enumerate}

\end{corrige}
