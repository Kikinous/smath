% This is part of Un soupçon de mathématique sans être agressif pour autant
% Copyright (c) 2015
%   Laurent Claessens
% See the file fdl-1.3.txt for copying conditions.

\begin{corrige}{2smath-0309}

    L'aire du triangle \( STU\) est donnée par la formule 
    \begin{equation}
        \frac{ \text{base}\times\text{hauteur} }{ 2 }=\frac{ ST\times 4 }{ 2 }=\frac{ \SI{6}{\centi\meter}\times \SI{4}{\centi\meter} }{ 2 }=\SI{12}{\centi\meter\squared}.
    \end{equation}
    

    En ce qui concerne l'aire grisée, c'est la moitié. Pour justifier, on utilise la propriété :

    \begin{quote}
        Une médiane coupe un triangle en deux parties de même aires.
    \end{quote}

    le segment \( [SK]\) est une médiane du triangle \( SIU\), donc l'aire blanche \( SIK\) est égale à l'aire grise \( SKU\). Même chose pour l'autre côté.

    Au final, l'aire grise est la moitié de \( STU\), c'est à dire \( \SI{6}{\centi\meter\squared}\).

\end{corrige}
