% This is part of Un soupçon de mathématique sans être agressif pour autant
% Copyright (c) 2013
%   Laurent Claessens
% See the file fdl-1.3.txt for copying conditions.

La classe est divisée en groupes, chacun contenant des perles turquoises et des perles blanches. Les sacs sont identiques mais inconnus. Le but de l'activité est de déterminer ce contenu sans compter toutes les perles. Voici le protocole :
\begin{itemize}
    \item Tirer une perle au hasard dans le sac,
    \item Noter sa couleur.
    \item La remettre dans le sac.
\end{itemize}
Nous parlons de tirage \defe{avec remise}{avec remise!tirage}.

%--------------------------------------------------------------------------------------------------------------------------- 
\subsection*{Après 10 tirages}
%---------------------------------------------------------------------------------------------------------------------------

Reporter les résultats de vos tirages en notant T ou B dans le tableau suivant :

\begin{tabular}[]{|c|c|c|c|c|c|c|c|c|}
    \hline
    &&&&&&&&\\
    \hline
\end{tabular}

Ce tirage forme un \defe{échantillon}{échantillon} du contenu du sac.
\begin{enumerate}
    \item
        Quelle est la fréquence d'apparition de la couleur turquoise ?
     \item
        Reporter les fréquences des autres groupes :
        \begin{tabular}[]{|c|c|c|c|c|c|c|c|c|c|}
        \hline
        &&&&&&&&\\
        \hline
    \end{tabular}
\item
    Que constate-t-on ?
\end{enumerate}

%--------------------------------------------------------------------------------------------------------------------------- 
\subsection*{Après 50 tirages}
%---------------------------------------------------------------------------------------------------------------------------

Effectuer encore \( 40\) tirages.
\begin{enumerate}
    \item
        Calculer la fréquence d'apparition de la couleur turquoise.
    \item
        Compléter le tableau de fréquences d'apparition du turquoise suivant :
        \begin{equation*}
            \begin{array}[]{|c|c|c|c|}
                \hline
                &\text{échantillon de taille 10}&\text{échantillon de taille 40}&\text{Échantillon de taille 50}\\
                  \hline
                  \text{Fréq.}&&&\\ 
                  \hline 
                   \end{array}
               \end{equation*}
               
\end{enumerate}

%--------------------------------------------------------------------------------------------------------------------------- 
\subsection*{Après 100 tirages}
%---------------------------------------------------------------------------------------------------------------------------

Effectuer \( 50\) nouveaux tirages et reporter les résultats :

\begin{equation*}
    \begin{array}[]{|c|c|c|c|}
      \hline
        &\text{Taille 10}&\text{Taille 50}&\text{Taille 100}\\
       \hline
       \text{Fréq.}&&&\\ 
       \hline 
  \end{array}
\end{equation*}
               
Écrire les résultats des autres groupes :
        \begin{tabular}[]{|c|c|c|c|c|c|c|c|c|c|}
        \hline
        &&&&&&&&\\
        \hline
    \end{tabular}

Combien de tirages ont été faits en tout dans la classe ? Quelle est la fréquence observée des boules turquoises ?
