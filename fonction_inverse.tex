% This is part of Un soupçon de mathématique sans être agressif pour autant
% Copyright (c) 2013
%   Laurent Claessens
% See the file fdl-1.3.txt for copying conditions.

%+++++++++++++++++++++++++++++++++++++++++++++++++++++++++++++++++++++++++++++++++++++++++++++++++++++++++++++++++++++++++++ 
\section{Graphe et tableau de variation}
%+++++++++++++++++++++++++++++++++++++++++++++++++++++++++++++++++++++++++++++++++++++++++++++++++++++++++++++++++++++++++++

\begin{definition}
La \defe{fonction inverse}{fonction!inverse} est la fonction qui à \( x\) fait correspondre \( \frac{1}{ x }\).
\end{definition}

Son graphe et son tableau de variation les suivants : 
\begin{multicols}{2}

\begin{equation*}
    \begin{array}[]{ccccccc}
        -\infty&&&0&&&+\infty\\
        \hline
        0&&&|&+\infty&&\\
        &\searrow&&|&&\searrow&\\
        &&-\infty&|&&&0\\
    \end{array}
\end{equation*}

\columnbreak

%The result is on figure \ref{LabelFigfigureMIdFCNN}. % From file figureMIdFCNN
%\newcommand{\CaptionFigfigureMIdFCNN}{<+Type your caption here+>}
\begin{center}
\input{Fig_figureMIdFCNN.pstricks}
\end{center}
\end{multicols}

L'ensemble de définition de la fonction \( x\mapsto \frac{1}{ x }\) est l'ensemble de tous les réels sauf \( x=0\), c'est à dire \( \eR\setminus\{ 0 \}\).

\begin{definition}
    La courbe représentative de la fonction inverse est nommée \defe{hyperbole}{hyperbole}.
\end{definition}

%+++++++++++++++++++++++++++++++++++++++++++++++++++++++++++++++++++++++++++++++++++++++++++++++++++++++++++++++++++++++++++ 
\section{Inéquations avec la fonction inverse}
%+++++++++++++++++++++++++++++++++++++++++++++++++++++++++++++++++++++++++++++++++++++++++++++++++++++++++++++++++++++++++++

Les inéquations sont toujours un peu délicates parce qu'on ne peut pas multiplier par des nombres dont on ignore la valeur.

\begin{example}
    Il est vrai que
    \begin{equation}
        5<7.
    \end{equation}
    Si nous multiplions les deux côtés de cette inéquation par \( 3\) nous obtenons
    \begin{equation}
        15<21.
    \end{equation}
    Pas de problèmes. Mais si nous multiplions \( 5<7\) par \( -2\) par exemple, nous obtenons
    \begin{equation}
        -10 < -14,
    \end{equation}
    qui est FAUX.
\end{example}
<++>

%+++++++++++++++++++++++++++++++++++++++++++++++++++++++++++++++++++++++++++++++++++++++++++++++++++++++++++++++++++++++++++ 
\section{Exercices}
%+++++++++++++++++++++++++++++++++++++++++++++++++++++++++++++++++++++++++++++++++++++++++++++++++++++++++++++++++++++++++++

% Il faut un peu reclasser ces exercices.

\Exo{smath-0262}
\Exo{smath-0257}
\Exo{smath-0258}

\Exo{smath-0369}
\Exo{smath-0350}
\Exo{smath-0276}
\Exo{smath-0260}
\Exo{smath-0259}
\Exo{smath-0261}
\Exo{smath-0265}
\Exo{smath-0268}
\Exo{smath-0272}
\Exo{smath-0273}
\Exo{smath-0274} 
\Exo{smath-0275}
\Exo{smath-0277}
\Exo{smath-0278}
\Exo{smath-0292}
\Exo{smath-0294}
\Exo{smath-0346}
\Exo{smath-0364}
\Exo{smath-0256}        % Cet exercice n'apporte rien.
\Exo{smath-0406}
