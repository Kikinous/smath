% This is part of Un soupçon de mathématique sans être agressif pour autant
% Copyright (c) 2013
%   Laurent Claessens
% See the file fdl-1.3.txt for copying conditions.

%+++++++++++++++++++++++++++++++++++++++++++++++++++++++++++++++++++++++++++++++++++++++++++++++++++++++++++++++++++++++++++ 
\section{Graphe et tableau de variation}
%+++++++++++++++++++++++++++++++++++++++++++++++++++++++++++++++++++++++++++++++++++++++++++++++++++++++++++++++++++++++++++

La fonction inverse est la fonction définie sur \( \eR\setminus\{ 0 \}\) qui à \( x\) fait correspondre \( \frac{1}{ x }\).

Son graphe et son tableau de variation les suivants : 
\begin{multicols}{2}

\begin{equation*}
    \begin{array}[]{ccccccc}
        -\infty&&&0&&&+\infty\\
        \hline
        0&&&|&+\infty&&\\
        &\searrow&&|&&\searrow&\\
        &&-\infty&|&&&0\\
    \end{array}
\end{equation*}

\columnbreak

%The result is on figure \ref{LabelFigfigureMIdFCNN}. % From file figureMIdFCNN
%\newcommand{\CaptionFigfigureMIdFCNN}{<+Type your caption here+>}
\begin{center}
\input{Fig_figureMIdFCNN.pstricks}
\end{center}
\end{multicols}

L'ensemble de définition de la fonction \( x\mapsto \frac{1}{ x }\) est l'ensemble de tous les réels sauf \( x=0\).

\begin{definition}
    La courbe représentative de la fonction inverse est nommée \defe{hyperbole}{hyperbole}.
\end{definition}

%+++++++++++++++++++++++++++++++++++++++++++++++++++++++++++++++++++++++++++++++++++++++++++++++++++++++++++++++++++++++++++ 
\section{Exerccies}
%+++++++++++++++++++++++++++++++++++++++++++++++++++++++++++++++++++++++++++++++++++++++++++++++++++++++++++++++++++++++++++

\Exo{smath-0256}
\Exo{smath-0262}
\Exo{smath-0257}
\Exo{smath-0258}
\Exo{smath-0260}
\Exo{smath-0259}
\Exo{smath-0261}
