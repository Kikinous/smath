% This is part of Un soupçon de mathématique sans être agressif pour autant
% Copyright (c) 2012
%   Laurent Claessens
% See the file fdl-1.3.txt for copying conditions.

\begin{exercice}\label{exosmath-0188}

    Trois personnes jouent au jeu suivant : ils écrivent leurs noms sur un bout de papier qu'ils plient et qu'ils placent dans un chapeau. Ensuite ils reprennent chacun un des papiers au hasard. Ceux qui tirent leur propre nom gagnent.
    \begin{enumerate}
        \item
            Est-il possible qu'exactement deux des trois personnes tirent leur noms ?
        \item
            De combien de façon différentes les papiers peuvent être tirés ? (exemple : A prend le papier de B, B prend celui de A et C prend le sien) 
        \item
            Calculer la probabilité des événements suivants :
            \begin{enumerate}
                \item
                    «Tout le monde gagne».
                \item
                    «Une seule personne gagne» .
                \item
                    «Personne ne gagne».
            \end{enumerate}
    \end{enumerate}

\corrref{smath-0188}
\end{exercice}
