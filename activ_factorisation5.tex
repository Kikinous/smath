% This is part of Un soupçon de mathématique sans être agressif pour autant
% Copyright (c) 2014
%   Laurent Claessens
% See the file fdl-1.3.txt for copying conditions.

%--------------------------------------------------------------------------------------------------------------------------- 
\subsection*{Activité : Bertrand vend des pots}
%---------------------------------------------------------------------------------------------------------------------------

Bertrand l'artisan vend des pots sur le marché. Chaque pot lui coûte \( 2\)€ de matériel et est revendu \( 7\)€.
\begin{enumerate}
    \item
        Pour savoir quel sera son gain en vendant \( 13\) pots, Bertrand fait l'opération suivante :
        \begin{equation}
            \input{GKJooPMzPdG.calcul}
        \end{equation}
        Calculer cette valeur.
    \item
        Son ami Josef lui fait remarquer qu'il peut plus facilement calculer son bénéfice en calculant d'abord le bénéfice d'un pot et en multipliant ensuite par le nombre de pot.

        Proposer, en suivant cette idée, une expression donnant le bénéfice de Bertrand lorsqu'il vend \( 13\) pots.

        Vérifier qu'elle fonctionne en recalculant le bénéfice réalisé par la vente de \( 13\) pots.
    \item
        Calculer mentalement le bénéfice réalisé lors de la vente de \( 20\) pots.
\end{enumerate}
