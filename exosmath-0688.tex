% This is part of Un soupçon de mathématique sans être agressif pour autant
% Copyright (c) 2014
%   Laurent Claessens
% See the file fdl-1.3.txt for copying conditions.

\begin{exercice}\label{exosmath-0688}

    Existe-t-il un réel \( k\) tel que \( v=k\vect{ u }\) dans les cas suivants ?
    \begin{multicols}{3}
        \begin{enumerate}
            \item
                \( \vect{ u }\begin{pmatrix}
                    -1.5    \\ 
                    3    
                \end{pmatrix}\), \( \vect{ v }\begin{pmatrix}
                    -3    \\ 
                    6    
                \end{pmatrix}\)
            \item
                \( \vect{ u }\begin{pmatrix}
                    4    \\ 
                    5    
                \end{pmatrix}\), \( \vect{ v }=\begin{pmatrix}
                    2    \\ 
                    2.5    
                \end{pmatrix}\)
            \item
                \( \vect{ u }\begin{pmatrix}
                    2.5    \\ 
                       -1 
                   \end{pmatrix}\), \( \vect{ v }\begin{pmatrix}
                       5 \\ 
                       2 
                   \end{pmatrix}\).
        \end{enumerate}
    \end{multicols}

\corrref{smath-0688}
\end{exercice}
