% This is part of Un soupçon de mathématique sans être agressif pour autant
% Copyright (c) 2012-2013
%   Laurent Claessens
% See the file fdl-1.3.txt for copying conditions.

\begin{exercice}\label{exoSeconde-0002}

    \begin{wrapfigure}[30]{r}{10cm}
   \vspace{-1cm}        % à adapter.
   \centering
   \input{Fig_EvZZys.pstricks}
\end{wrapfigure}
    Répondre aux questions suivantes en utilisant le repère orthonormé ci-contre.

    \begin{enumerate}
        \item
            Placer les points \( A=(1;3)\), \( B=(-1;0)\), \( C=(1,\frac{ 3 }{2})\).
        \item
            Donner les coordonnées des points \( X\), \( Y\) et \( Z\).
    \end{enumerate}
    \vspace{1cm}
\corrref{Seconde-0002}

\end{exercice}
