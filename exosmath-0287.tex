% This is part of Un soupçon de mathématique sans être agressif pour autant
% Copyright (c) 2013
%   Laurent Claessens
% See the file fdl-1.3.txt for copying conditions.

\begin{exercice}\label{exosmath-0287}

    Nous tirons un nombre au hasard entre \( 1\) et \( 25\) (y compris). Soient les événements \( A=\)«obtenir un nombre divisible par \( 4\)» et \( B=\)«obtenir un nombre plus grand (ou égal) à \( 13\)».
    \begin{enumerate}
        \item
            Donner l'ensemble des événements élémentaires qui forment l'événement \( A\).
        \item
            Calculer la probabilité de \( A\) et de \( B\). 
        \item
            Calculer \( p(A\cap B)\) et \( p(A\cup B)\).
    \end{enumerate}

\corrref{smath-0287}
\end{exercice}
