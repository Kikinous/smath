% This is part of Un soupçon de mathématique sans être agressif pour autant
% Copyright (c) 2012,2014
%   Laurent Claessens
% See the file fdl-1.3.txt for copying conditions.

\begin{exercice}\label{exoSeconde-0064}

    Soit un verre d'eau parallélépipédique rectangle dont la base est un carré de \unit{4}{\centi\meter}. À quelle hauteur faut-il le remplir pour avoir \unit{64}{\centi\meter\cubed} ?

    Pour rappel, le volume est donné par l'aire de la base multiplié par la hauteur.

\corrref{Seconde-0064}
\end{exercice}
