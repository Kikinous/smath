% This is part of Un soupçon de mathématique sans être agressif pour autant
% Copyright (c) 2012
%   Laurent Claessens
% See the file fdl-1.3.txt for copying conditions.

\begin{exercice}\label{exosmath-0135}

    Soit la fonction \( f(x)=2x-1\) et \( d\) la droite qui la représente. Vrai ou faux ?
    \begin{enumerate}
        \item
            \( f(x)\geq 6\) pour tout \( x\in \eR\).
        \item
            Il existe un \( x\in \eR\) tel que \( f(x)\geq 6\).
        \item
            Il existe un unique réel \( x\) tel que \( f(x)=2\).
        \item
            La fonction \( f\) est croissante.
        \item
            Il existe un nombre \( a\) tel que \( f(a)=a\).
    \end{enumerate}

\corrref{smath-0135}
\end{exercice}
