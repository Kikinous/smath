% This is part of Un soupçon de mathématique sans être agressif pour autant
% Copyright (c) 2014
%   Laurent Claessens
% See the file fdl-1.3.txt for copying conditions.

\begin{corrige}{smath-0971}

    \begin{enumerate}
        \item
            Jean-Luc met \SI{3}{\centi\liter} de sirop et doit donc ajouter \( 3\times 8=24\) centilitres d'eau.
        \item
            Il aura donc \( 24+3=27\) centilitres dans son verre, dont \( 24\) d'eau. La proportion d'eau dans son verre sera donc de
            \begin{equation}
                \frac{ 24 }{ 27 }.
            \end{equation}
            Cela se simplifie en \( \dfrac{ 8 }{ 9 }\).
        \item
            Pour \SI{5}{\centi\liter} de sirop, il faut ajouter \( 5\times 8=40\) centilitres d'eau. Il y aura donc en tout \SI{45}{\centi\liter} dans son verre. La proportion d'eau sera
            \begin{equation}
                \frac{ 40 }{ 45 }.
            \end{equation}
            Cette fraction se simplifie en \( \dfrac{ 8 }{ 9 }\). Et, tiens donc, c'est la même.
    \end{enumerate}

\end{corrige}
