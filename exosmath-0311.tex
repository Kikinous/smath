% This is part of Un soupçon de mathématique sans être agressif pour autant
% Copyright (c) 2013
%   Laurent Claessens
% See the file fdl-1.3.txt for copying conditions.

\begin{exercice}\label{exosmath-0311}

    En solfège le cycle des quintes désigne l'ensemble des notes en sautant de quintes en quintes. Théoriquement, une quinte correspond à une multiplication par \( 1.5\) de la fréquence et un \info{la} correspond à \unit{440}{\hertz}. 
    \begin{enumerate}
        \item
            Calculer les fréquences des notes suivantes dans le cycle des quintes partant du \info{la}.
        \item
            À combien d'octaves du premier \info{la} se trouve le \info{la} suivant dans le cycle ? 
        \item
            Sachant qu'une octave correspond à une multiplication de la fréquence par deux, quelle serait sa fréquence ?
        \item
            En suivant le cycle des quintes, ce \info{la} est après 12 quintes. Quelle devrait alors être sa fréquence ?
    \end{enumerate}
    La différence entre les deux dernières réponses est une mesure de ce que les quintes sont en réalité fausses. Indirectement c'est pour cela que les tonalités ont des «sons» différents.
    
    
\corrref{smath-0311}
\end{exercice}
