% This is part of Un soupçon de mathématique sans être agressif pour autant
% Copyright (c) 2014
%   Laurent Claessens
% See the file fdl-1.3.txt for copying conditions.

\begin{exercice}\label{exosmath-0753}

Nous considérons l'enchaînement d'opérations suivant :
\begin{itemize}
    \item Choisir un entier entre \( 1\) et \( 20\),
    \item ajouter \( 3\),
    \item multiplier par \( 2\),
    \item soustraire \( 6\).
\end{itemize}
\begin{enumerate}
    \item
        Effectuer cet enchaînement pour quelque valeurs de départ au choix. Quel est le lien qui relie le nombre choisit avec le résultat des opérations ?
    \item
        Est-ce que ça fonctionne aussi avec des nombres négatifs ?
    \item
        Donner une expression donnant le résultat lorsque le nombre choisi est $14$.
\end{enumerate}

\corrref{smath-0753}
\end{exercice}
