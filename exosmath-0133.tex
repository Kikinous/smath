% This is part of Un soupçon de mathématique sans être agressif pour autant
% Copyright (c) 2012
%   Laurent Claessens
% See the file fdl-1.3.txt for copying conditions.

\begin{exercice}\label{exosmath-0133}

    Soient les fonction \( f(x)=\frac{ x^2 }{2}\) et \( g(x)=3x-4\).
    \begin{enumerate}
        \item
            Tracer sur un même graphique les courbes représentatives de \( f\) et de \( g\) sur l'intervalle \( \mathopen[ -4 , 6 \mathclose]\).
        \item 
            Résoudre graphiquement l'équation \( f(x)=g(x)\). Indiquer les solutions sur le graphique et écrire les solutions sur la feuille.
        \item
            Démontrer (par un calcul) que
            \begin{equation}
                \frac{ 1 }{2}x^2-3x+4=\frac{ 1 }{2}(x-2)(x-4)
            \end{equation}
            pour tout \( x\in \eR\).
        \item
            Résoudre graphiquement l'inéquation \( f(x)\geq g(x) \). Écrire la solution sous forme d'intervalle et l'indiquer sur le graphique.
    \end{enumerate}

\corrref{smath-0133}
\end{exercice}
