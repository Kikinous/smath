% This is part of Un soupçon de mathématique sans être agressif pour autant
% Copyright (c) 2012-2014
%   Laurent Claessens
% See the file fdl-1.3.txt for copying conditions.

%+++++++++++++++++++++++++++++++++++++++++++++++++++++++++++++++++++++++++++++++++++++++++++++++++++++++++++++++++++++++++++ 
\section*{Introduction}
%+++++++++++++++++++++++++++++++++++++++++++++++++++++++++++++++++++++++++++++++++++++++++++++++++++++++++++++++++++++++++++

% This is part of Un soupçon de mathématique sans être agressif pour autant
% Copyright (c) 2014
%   Laurent Claessens
% See the file fdl-1.3.txt for copying conditions.

Soit le programme suivant :

\begin{fmpage}{0.9\linewidth}

    Demander \( x\) et \( y\).

    Si \( y=-2x\) alors :

    \hspace{1cm} Écrire «oui» 

    Sinon :

    \hspace{1cm} Écrire «non» 

\end{fmpage}

Donner quelque valeurs de \( x\) et \( y\) pour lesquelles le programme écrit «oui» ? À quoi sert ce programme ?

Mêmes questions pour ce programme :

\begin{fmpage}{0.9\linewidth}

    Demander \( x\) et \( y\).

    Si \( x=4\) alors :

    \hspace{1cm} Écrire «oui» 

    Sinon :

    \hspace{1cm} Écrire «non» 

\end{fmpage}


%+++++++++++++++++++++++++++++++++++++++++++++++++++++++++++++++++++++++++++++++++++++++++++++++++++++++++++++++++++++++++++ 
\section{Équation de droite}
%+++++++++++++++++++++++++++++++++++++++++++++++++++++++++++++++++++++++++++++++++++++++++++++++++++++++++++++++++++++++++++

\begin{theorem}
    Toute droite du plan a une équation.
    \begin{enumerate}
        \item
            Si \( d\) est une droite parallèle à l'axe des ordonnées alors elle a une équation \( d:x=c\) pour un certain réel \( c\).
        \item
            Si \( d\) est une droite non parallèle à l'axe des ordonnées, alors elle a une équation \( y=mx+p\) pour certains réels \( m\) et \( p\).
    \end{enumerate}
\end{theorem}

\begin{proof}
    \begin{enumerate}
        \item
            Soit une droite \( d\) parallèle à l'axe des ordonnées. Elle coupe l'axe des abscisses en un point \( C\) de coordonnées \( (c,0)\). Par définition du système de coordonnée, tout point de \( d\) a alors une abscisse égale à \( c\).
        \item
            Nous supposons à présent que la droite \( d\) n'est pas parallèle à l'axe des ordonnées. Cette droite coupe l'axe vertical en un point \( P\) de coordonnées \( (0;p)\) et l'axe \( x=1\) au point \( Q(1;q)\).
            \begin{center}
   \input{Fig_OKeZlpK.pstricks}
            \end{center}
            Le théorème de Thalès nous indique que
            \begin{equation}
                \frac{ MS }{ PS }=\frac{ QR }{ PR },
            \end{equation}
            c'est à dire
            \begin{equation}
                \frac{ y_M-p }{ x_M }=\frac{ q-p }{ 1 },
            \end{equation}
            c'est à dire
            \begin{equation}
                y_M=(q-p)x_M+p.
            \end{equation}
            En posant \( m=q-p\) nous avons bien l'équation demandée :
            \begin{equation}
                y=mx+p.
            \end{equation}
    \end{enumerate}
\end{proof}

\begin{theorem}
    Soit la fonction affine \( f(x)=mx+p\). Alors pour tout nombres distincts \( u\) et \( v\) nous avons
    \begin{equation}
        \frac{ f(u)-f(v) }{ u-v }=m.
    \end{equation}
\end{theorem}

\begin{proof}
    Soient \( u\) et \( v\), deux réels distincts. Nous calculons
    \begin{equation}
        f(u)-f(v)=mu+p-(mv+p)=mu-mv=m(u-v).
    \end{equation}
    En divisant par \( u-v\) on trouve le résultat annoncé.
\end{proof}

%+++++++++++++++++++++++++++++++++++++++++++++++++++++++++++++++++++++++++++++++++++++++++++++++++++++++++++++++++++++++++++ 
\section{Déterminer une équation de droite}
%+++++++++++++++++++++++++++++++++++++++++++++++++++++++++++++++++++++++++++++++++++++++++++++++++++++++++++++++++++++++++++

%TODO : Décommenter cette démonstration.
%
%Soient les points \( A(x_A;y_A)\) et \( B(x_B;y_B)\). Nous voulons déterminer l'équation de la droite
%\begin{equation}
%    y=mx+p
%\end{equation}
%qui passe par ces deux points. Vu que \( A\) est sur cette droite, ses coordonnées vérifient l'équation :
%\begin{equation}
%    y_A=mx_A+p
%\end{equation}
%et pour la même raison :
%\begin{equation}
%    y_B=mx_B+p
%\end{equation}
%En soustrayant ces deux équations :
%\begin{equation}
%    y_A-y_B=  mx_A+p-(mx_B+p)=m(x_A-x_B),
%\end{equation}
%ou encore
%\begin{equation}
%    y_A-y_B=m(x_A-x_B), 
%\end{equation}
%c'est à dire
%\begin{equation}
%    m=\frac{ y_A-y_B }{ x_A-x_B }.
%\end{equation}
%
\begin{Aretenir}
    Le coefficient directeur d'une droite passant par deux points \( A(x_A;y_A)\) et \( B(x_B;y_B)\) est donné par la formule
    \begin{equation}
        m=\frac{ y_B-y_A }{ x_B-x_A }.
    \end{equation}
    L'ordonnée à l'origine se calcule en posant \( y_A=mx_A+p\) et en résolvant pour \( p\).
\end{Aretenir}

\begin{example}
    Trouver l'équation de la droite passant par \( A(-1;-2)\) et \( B(1;4)\).

    D'abord
    \begin{equation}
        m=\frac{ 4-(-2) }{ 1-(-2) }=\frac{ 6 }{ 3 }=3.
    \end{equation}
    Nous cherchons donc une équation du type \( y=3x+p\). Il s'agit donc de trouver \( p\) pour avoir \emph{en même temps}
    \begin{equation}
            -2=3\times (-1)+p
    \end{equation}
    et
    \begin{equation}
        4=3\times 1+p
    \end{equation}
    La solution est \( p=1\). Au final l'équation cherchée est
    \begin{equation}
        y=3x+1.
    \end{equation}

    Vérification : le point \( A(-1;-2)\) est sur la droite d'équation \( y=3x+1\) parce que \( y_A=3x_A+1\); en effet :
    \begin{equation}
        -2=3\times (-1)+1.
    \end{equation}
    Le point \( B(1;4)\) est sur la droite d'équation \( y=3x+1\) parce que \( y_B=3x_B+1\); en effet :
    \begin{equation}
        4=3\times 1+1.
    \end{equation}
\end{example}

%+++++++++++++++++++++++++++++++++++++++++++++++++++++++++++++++++++++++++++++++++++++++++++++++++++++++++++++++++++++++++++ 
\section{Intersection de droites}
%+++++++++++++++++++++++++++++++++++++++++++++++++++++++++++++++++++++++++++++++++++++++++++++++++++++++++++++++++++++++++++

Nous avons déjà vu que les fonctions affines \( f(x)=mx+p\) et \( g(x)=m'x+p'\) ont des droites représentatives parallèles si et seulement si \( m=m'\).

\begin{Aretenir}
    Trouver l'abscisse du point d'intersection des droites \( y=mx+p\) et \( y=m'x+p\) revient à résoudre l'équation
    \begin{equation}
        mx+p=m'x+p.
    \end{equation}
    Pour trouver l'ordonnée il suffit de mettre la réponse dans l'équation d'une des deux droites.

    ATTENTION : si \( m=m'\), les droites sont parallèles et il n'y a pas d'intersection.
\end{Aretenir}

\begin{example}
    Trouver le point d'intersection des droites \( d_1:y=2x-4\) et \( d_2:y=-x+5\). Ce sont les droites représentatives des fonctions affines \( f(x)=2x+4\) et \( g(x)=-x+5\). L'abscisse du point d'intersection est donnée par l'équation \( f(x)=g(x)\), c'est à dire
    \begin{equation}
        2x-4=-x+5,
    \end{equation}
    dont la solution est \( x=3\). 

    Pour trouver l'ordonnée du point d'intersection il faut calculer \( f(3)\) ou \( g(3)\), qui doivent donner le même résultat.
    \begin{equation}
        f(3)=g(3)=2.
    \end{equation}
    Dont le point d'intersection est le point
    \begin{equation}
        I=(3;2).
    \end{equation}
\end{example}

%+++++++++++++++++++++++++++++++++++++++++++++++++++++++++++++++++++++++++++++++++++++++++++++++++++++++++++++++++++++++++++ 
\section{Colinéarité}
%+++++++++++++++++++++++++++++++++++++++++++++++++++++++++++++++++++++++++++++++++++++++++++++++++++++++++++++++++++++++++++

\begin{wrapfigure}{r}{10.cm}
   \vspace{-0.5cm}        % à adapter.
   \centering
   \input{Fig_RVZNtGK.pstricks}
\end{wrapfigure}


Le théorème de Thalès nous donne une bonne façon de savoir si trois points sont alignés. Les points \( A\), \( B\) et \( C\) sont alignés si et seulement si
\begin{equation}
    \frac{ y_C-y_A }{ x_C-x_A }=\frac{ y_B-y_A }{ x_B-x_A }.
\end{equation}
C'est à dire si et seulement si les coefficients directeurs des droites \( (AB)\) et \( (AC)\) sont les mêmes.
