% This is part of Un soupçon de mathématique sans être agressif pour autant
% Copyright (c) 2012-2013
%   Laurent Claessens
% See the file fdl-1.3.txt for copying conditions.

%+++++++++++++++++++++++++++++++++++++++++++++++++++++++++++++++++++++++++++++++++++++++++++++++++++++++++++++++++++++++++++ 
\section{Introduction}
%+++++++++++++++++++++++++++++++++++++++++++++++++++++++++++++++++++++++++++++++++++++++++++++++++++++++++++++++++++++++++++

\begin{Aprojeter}
    Un premier taxi divise son prix en deux paries : $0.2$ euros de frais de prise en charge plus un euro par km parcouru. Un second taxi divise son prix en $1$ euro de frais de prise en charge plus $0.8$ euros par kilomètre parcouru.

    \begin{enumerate}
        \item
            Combien coûte un trajet de \unit{5}{\kilo\meter} avec le premier taxi ?
        \item
            Combien de kilomètres peut-t-on effectuer dans le second taxi avec \( 10\) euros ?
        \item
            Exprimer les prix en fonction du nombre de kilomètres sur un graphique (les deux taxis sur le même graphique).
        \item
            Donner une expression algébrique.
        \item
            À partir de combien de kilomètres parcourus le second taxi est-il avantageux ?
    \end{enumerate}
\end{Aprojeter}

%+++++++++++++++++++++++++++++++++++++++++++++++++++++++++++++++++++++++++++++++++++++++++++++++++++++++++++++++++++++++++++ 
\section{Définitions}
%+++++++++++++++++++++++++++++++++++++++++++++++++++++++++++++++++++++++++++++++++++++++++++++++++++++++++++++++++++++++++++

\begin{definition}
    Une \defe{fonction affine}{affine}\index{fonction!affine} est une fonction définie sur \( \eR\) par
    \begin{equation}
        f(x)=ax+b
    \end{equation}
    où \( a\) et \( b\) sont deux nombres réels fixés.
\end{definition}

Cas particuliers :
\begin{enumerate}
    \item
        Si \( b=0\) alors \( f(x)=ax\) et nous disons que \( f\) est une fonction \defe{linéaire}{fonction!linéaire}. Son graphe passe par l'origine \( (0;0)\).
    \item
        Si \( a=0\) alors \( f(x)=b\) et nous disons que \( f\) est une fonction \defe{constante}{fonction!constante}.
\end{enumerate}

%+++++++++++++++++++++++++++++++++++++++++++++++++++++++++++++++++++++++++++++++++++++++++++++++++++++++++++++++++++++++++++ 
\section{Pour tracer}
%+++++++++++++++++++++++++++++++++++++++++++++++++++++++++++++++++++++++++++++++++++++++++++++++++++++++++++++++++++++++++++

Pour tracer le graphe d'une fonction affine.
\begin{itemize}
    \item
        Vu que le graphe est une droite, il suffit de deux points.
    \item
        Les fonction linéaires passent par l'origine \( (0;0)\).
    \item 
        Pour un tracé à la règle, il est plus précis de prendre deux points relativement éloignés.
    \item
        La droite \( f(x)=ax+b\) monte si \( a>0\) et descend si \( a<0\). La pente est d'autant plus raide que \( a\) est grand.
\end{itemize}

Quelque exemples à la figure \ref{LabelFigGrapheAffinHqXJGx}.
\newcommand{\CaptionFigGrapheAffinHqXJGx}{Des graphes de fonctions linéaires et affines.}
\input{Fig_GrapheAffinHqXJGx.pstricks}

\Exo{smath-0490}
