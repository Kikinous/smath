% This is part of Un soupçon de mathématique sans être agressif pour autant
% Copyright (c) 2012-2013
%   Laurent Claessens
% See the file fdl-1.3.txt for copying conditions.


%+++++++++++++++++++++++++++++++++++++++++++++++++++++++++++++++++++++++++++++++++++++++++++++++++++++++++++++++++++++++++++ 
\section{Introduction}
%+++++++++++++++++++++++++++++++++++++++++++++++++++++++++++++++++++++++++++++++++++++++++++++++++++++++++++++++++++++++++++

Ce chapitre parle d'équations de droites et se voit après le chapitre sur les fonctions affines.
\begin{enumerate}
    \item
        Droite passant par des points donnés.
    \item 
        Parallélisme, intersection.
    \item
        Proportionnalité, coefficient directeur.
    \item
        Lien avec la colinéarité des vecteurs.
    \item
        Dire si trois points sont alignés.
\end{enumerate}

%+++++++++++++++++++++++++++++++++++++++++++++++++++++++++++++++++++++++++++++++++++++++++++++++++++++++++++++++++++++++++++ 
\section{Théorie}
%+++++++++++++++++++++++++++++++++++++++++++++++++++++++++++++++++++++++++++++++++++++++++++++++++++++++++++++++++++++++++++

\begin{theorem}
    Soit la fonction affine \( f(x)=ax+b\). Alors pour tout nombres distincts \( u\) et \( v\) nous avons
    \begin{equation}
        \frac{ f(u)-f(v) }{ u-v }=a.
    \end{equation}
\end{theorem}

\begin{proof}
    Soient \( u\) et \( v\), deux réels distincts. Nous calculons
    \begin{equation}
        f(u)-f(v)=au+b-(av+b)=au-av=a(u-v).
    \end{equation}
    En divisant par \( u-v\) on trouve le résultat annoncé.
\end{proof}

Cela donne une méthode graphique pour déterminer le coefficient directeur d'une droite. Si les points \( A=(x_A;y_A)\) et \( B=(x_B;y_B)\) sont sur la droite \( d\) d'équation \( y=ax+b\), alors
\begin{equation}
    a=\frac{ y_B-y_A }{ x_B-x_A }.
\end{equation}
Notez que cela vient du vecteur directeur
\begin{equation}
    \begin{pmatrix}
        x_B-x_A    \\ 
        y_B-y_A    
    \end{pmatrix}
\end{equation}

%+++++++++++++++++++++++++++++++++++++++++++++++++++++++++++++++++++++++++++++++++++++++++++++++++++++++++++++++++++++++++++ 
\section{Droites parallèles et droites sécantes}
%+++++++++++++++++++++++++++++++++++++++++++++++++++++++++++++++++++++++++++++++++++++++++++++++++++++++++++++++++++++++++++

%--------------------------------------------------------------------------------------------------------------------------- 
\subsection{Critère avec le coefficient directeur}
%---------------------------------------------------------------------------------------------------------------------------

\begin{example}
    Les droites \( y=-3x+4\) et \( y=2x+1\) sont sécantes, comme nous pouvons le voir sur un dessin. Comment trouver les coordonnées du point d'intersection ?

    Le point d'intersection \( (x_0;y_0)\) doit satisfaire \( y_0=-3x_0+4\) et \( y_0=2x_0+1\) en même temps. Donc
    \begin{equation}
        -3x_0+4=2x_0+1,
    \end{equation}
    ce qui donne \( x_0=\frac{ 3 }{ 5 }\). Pour trouver \( y_0\) il suffit de remplacer dans l'une ou l'autre équation :
    \begin{equation}
        y_0=-3x_0+4=-3\frac{ 3 }{ 5 }+4=\frac{ 11 }{ 5 }.
    \end{equation}
    Vérification : dans l'autre on obtient le même résultat :
    \begin{equation}
        y_0=2x_0+1=2\frac{ 3 }{ 5 }+1=\frac{ 11 }{ 5 }.
    \end{equation}
    Donc le point d'intersection des deux droites est le point de coordonnées
    \begin{equation}
        \big( \frac{ 3 }{ 5 };\frac{ 11 }{ 5 } \big).
    \end{equation}
\end{example}

\begin{theorem}
    Dans un repère, les droites d'équations \( y=ax+b\) et \( y=a'x+b'\) sont parallèles si et seulement si elles ont même coefficient directeur, c'est à dire si et seulement si \( a=a'\).

    Les droites sont sécantes si et seulement si \( a\neq a'\).
\end{theorem}

\begin{proof}
    Nous allons seulement démontrer que si \( a\neq a'\), alors les droites s'intersectent. Pour cela nous allons montrer qu'il existe un point \( (x_0,y_0)\) qui se trouve sur le graphe des deux fonctions, c'est à dire qui vérifie
    \begin{subequations}
        \begin{numcases}{}
            y_0=ax_0+b\\
            y_0=a'x_0+b'.
        \end{numcases}
    \end{subequations}
    Le nombre \( x_0\) doit donc satisfaire
    \begin{equation}
        ax_0+b=a'x_0+b',
    \end{equation}
    c'est à dire \( (a-a')x_0=b'-b\) et donc
    \begin{equation}
        x_0=\frac{ b'-b }{ a-a' }.
    \end{equation}
\end{proof}

%--------------------------------------------------------------------------------------------------------------------------- 
\subsection{Et le vecteur directeur}
%---------------------------------------------------------------------------------------------------------------------------

Si \( A\) et \( B\) sont des points du plan, nous avons déjà vu que le vecteur \( \vect{ AB }\) était un critère de parallélisme de la droite \( AB\). Plus précisément, nous avons vu qu'une droite \( (CD)\) était parallèle à \( AB\) si et seulement si \( \vect{ CD }\) est un multiple de \( \vect{ AB }\). Voyons cela de plus près.

Soit la droite \( y=ax+b\). Les points \( A=(0;b)\) et \( B=(1,a+b)\) sont sur cette droite et donc le vecteur directeur est
\begin{equation}
    \vect{ AB }=\begin{pmatrix}
        1    \\ 
        a    
    \end{pmatrix}
\end{equation}
Si les points \( C\) et \( D\) sont tels que \( (CD)\parallel (AB)\), alors
\begin{equation}
    \vect{ CD }=\lambda\begin{pmatrix}
        1    \\ 
        a    
    \end{pmatrix}=\begin{pmatrix}
        \lambda    \\ 
            \lambda a
    \end{pmatrix}.
\end{equation}
Si le point \( C\) a pour coordonnées \( C=(x_C;y_C)\), alors le point \( D\) doit avoir comme coordonnées
\begin{equation}
    D=\big( x_C+\lambda;y_C+\lambda a \big).
\end{equation}
Le coefficient directeur de la droite \( (CD)\) est alors
\begin{equation}
    \frac{ (y_C+\lambda a)-y_C }{ (x_C+\lambda)-x_C }=a.
\end{equation}
Nous avons donc confirmé que le critère du vecteur proportionnel et le critère du coefficient directeur sont en réalité les mêmes.

%+++++++++++++++++++++++++++++++++++++++++++++++++++++++++++++++++++++++++++++++++++++++++++++++++++++++++++++++++++++++++++ 
\section{Colinéarité}
%+++++++++++++++++++++++++++++++++++++++++++++++++++++++++++++++++++++++++++++++++++++++++++++++++++++++++++++++++++++++++++

\begin{theorem}
    Trois points distincts \( A\), \( B\) et \( C\) sont alignés si et seulement si les droites \( (AB)\) et \( (AC)\) ont même coefficient directeur.
\end{theorem}

\begin{proof}
    Si les droites \( (AB)\) et \( (AC)\) ont même coefficient directeur, alors nous pouvons écrire
    \begin{equation}
        (AB)\equiv y=ax+b
    \end{equation}
    et
    \begin{equation}
        (AC)\equiv ax+b'.
    \end{equation}
    Cependant \( A\) est un point commun aux deux droites : \( y_A=ax_A+b=ax_A+b'\) et donc \( b=b'\), ce qui signifie que les droites \( (AB)\) et \( (AC)\) sont les mêmes.
\end{proof}

\begin{example}
    Vérifions que les points \( A=(1;-1)\), \( B=(3;5)\) et \( C=(4;8)\) sont alignés.

    D'abord le coefficient directeur de la droite \( (AB)\) est 
    \begin{equation}
        \frac{ y_B-y_A }{ x_B-x_A }=\frac{ 5-(-1) }{ 3-1 }=3
    \end{equation}
    ensuite le coefficient directeur de la droite \( (AC)\) est
    \begin{equation}
        \frac{ y_C-y_A }{ x_C-x_A }=\frac{ 8+1 }{ 4-1 }=3
    \end{equation}
    Donc les points \( A\), \( B\) et \( C\) sont alignés.
\end{example}

%+++++++++++++++++++++++++++++++++++++++++++++++++++++++++++++++++++++++++++++++++++++++++++++++++++++++++++++++++++++++++++ 
\section{Systèmes et intersections de droites}
%+++++++++++++++++++++++++++++++++++++++++++++++++++++++++++++++++++++++++++++++++++++++++++++++++++++++++++++++++++++++++++

On s'intéresse à des systèmes linéaires de deux équations à deux
inconnues, c'est-à-dire de la forme
    \begin{subequations}
        \begin{numcases}{}
            ax+by=c\\
a'x+b'y=c'
        \end{numcases}
    \end{subequations}
où $a$, $b$, $c$, $a'$, $b'$, $c'$ sont des coefficients réels. Les
solutions sont des couples $(x;y)$.

%+++++++++++++++++++++++++++++++++++++++++++++++++++++++++++++++++++++++++++++++++++++++++++++++++++++++++++++++++++++++++++ 
\section{Méthode par substitution}
%+++++++++++++++++++++++++++++++++++++++++++++++++++++++++++++++++++++++++++++++++++++++++++++++++++++++++++++++++++++++++++


À l'aide de la première équation, on exprime l'une des
  variables en fonction de l'autre, puis on remplace dans la deuxième
  équation par l'expression obtenue.


  \begin{example}
Résoudre par substitution le système
    \begin{subequations}
        \begin{numcases}{}
            x+2y=4\\
-3x+4y=18
        \end{numcases}
    \end{subequations}

  \medskip

  \begin{enumerate}
  \item La première équation permet d'exprimer facilement $x$ en
    fonction de $y$ : \\ $x=4-2y$.

  \item Dans la deuxième équation, on remplace $x$ par $4-2y$. On
    obtient une équation qui ne dépend que de $y$, qu'on résout.
    \begin{gather*}
      -3(4-2y) + 4y = 18 \\
      -12 + 6y + 4 y = 18 \\
      10 y =30 \\
      y = 3
    \end{gather*}

  \item Une fois trouvé $y$, on reprend l'expression de $x$ en
    fonction de $y$, et on remplace $y$ par la valeur trouvée :
    \[
    x = 4-2y = 4-2\times 3 = 4-6=-2
    \]

  \item On donne l'ensemble solution, toujours dans l'ordre $(x;y)$.
    \[
    \boxed{ \mathscr{S} = \left\{ (-2;3) \right\} }
    \]
  \end{enumerate}
  
      
  \end{example}
  


  %+++++++++++++++++++++++++++++++++++++++++++++++++++++++++++++++++++++++++++++++++++++++++++++++++++++++++++++++++++++++++++ 
  \section{Méthode par addition, ou par combinaisons linéaires}
  %+++++++++++++++++++++++++++++++++++++++++++++++++++++++++++++++++++++++++++++++++++++++++++++++++++++++++++++++++++++++++++


On multiplie l'une des équations (ou éventuellement
  les deux) par un facteur, positif ou négatif, de manière à éliminer
  l'une des deux variables lorsqu'on fait la somme des deux équations.


  \begin{example}
Résoudre
\begin{subequations}
    \begin{numcases}{}
        x+2y=4\\
        -3x+4y=18
    \end{numcases}
\end{subequations}


  \begin{enumerate}
  \item On cherche à éliminer $y$. Pour cela, il faut faire en sorte
    que le coefficient devant $y$ dans la première équation soit égal
    à $-4$. On doit donc multiplier la première équation par $-2$.
    
    \underline{\textbf{Attention}} : Il faut multiplier
    \underline{\textbf{tous}} les coefficients de l'équation par $-2$.
    \medskip

  \item Le système est équivalent à :
    \begin{subequations}
        \begin{numcases}{}
            -2x-4y=-8\\
-3x+4y=18
        \end{numcases}
    \end{subequations}

  \item On fait la somme, membre à membre, des deux équations. On
    obtient une équation dans laquelle n'intervient plus que $x$,
    qu'on résout.
    \begin{equation}
        -5x=10
    \end{equation}
    et don
    \begin{equation}
        x=-2
    \end{equation}
    
  \item On reprend l'une des deux équations du système d'origine pour
    exprimer $y$ en fonction de $x$ (choisir si possible une équation
    dans laquelle $y$ intervient avec le coefficient $1$, ou $-1$) et
    calculer $y$.
    \begin{gather*}
      x+2y = 4 \\
      2y  = 4-x \\
      2y = 4-(-2) \\
      2y = 6 \\
      y = 3
    \end{gather*}

  \item On donne l'ensemble solution :
    \[
    \boxed{ \mathscr{S} = \left\{ (-2;3) \right\} }
    \]
  \end{enumerate}

  \end{example}

