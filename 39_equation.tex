% This is part of Un soupçon de mathématique sans être agressif pour autant
% Copyright (c) 2015
%   Laurent Claessens
% See the file fdl-1.3.txt for copying conditions.

Note pour moi : donner l'exercice \ref{exo2smath-0185} en DM en guise d'introduction à ce chapitre.

% This is part of Un soupçon de mathématique sans être agressif pour autant
% Copyright (c) 2015
%   Laurent Claessens
% See the file fdl-1.3.txt for copying conditions.

%--------------------------------------------------------------------------------------------------------------------------- 
\subsection*{Activité : partage inégal}
%---------------------------------------------------------------------------------------------------------------------------

Trois personnes se partagent la somme de \SI{316}{€}. On veut trouver la part de chacune sachant que la seconde a \SI{32}{€} de plus que la première et que la troisième a \SI{15}{€} de plus que la seconde.

Réaliser au choix une des deux questions suivantes :

\begin{enumerate}
    \item
 Soit $x$ la part de la première personne. Mettre ce problème en équation et le résoudre.
\item
Soit $x$ la part de la deuxième personne. Mettre ce problème en équation et le résoudre.
\end{enumerate}

De \cite{NRHooXFvgpp4}

%+++++++++++++++++++++++++++++++++++++++++++++++++++++++++++++++++++++++++++++++++++++++++++++++++++++++++++++++++++++++++++ 
\section{Vocabulaire}
%+++++++++++++++++++++++++++++++++++++++++++++++++++++++++++++++++++++++++++++++++++++++++++++++++++++++++++++++++++++++++++

\begin{definition}[\cite{NRHooXFvgpp4}]
    Une \defe{équation}{équation} est une expression dans laquelle il y a toujours un signe égal et une ou plusieurs inconnues.
\end{definition}

\begin{definition}[\cite{NRHooXFvgpp4}]
    \defe{Résoudre}{résoudre} une équation d'inconnue \( x\), c'est déterminer toutes les valeurs de \( x\) (si elles existent) pour que l'égalité soit vraie. Chacune de ces valeurs est appelée \defe{solution}{solution} de l'équation.
\end{definition}

\begin{example}
    Est-ce que \( 3\) est solution de l'équation \(  x^2-2x-5=3  \) ? Il suffit de remplacer \( x\) par la valeur proposée et vérifier si l'égalité est vérifiée :
    \begin{equation}
        3^2-2\times 3-5=-2
    \end{equation}
    vu que cela n'est pas égal à \( 3\), \( x=3\) n'est pas solution.

    Est-ce que \( x=2\) est une solution ? Oui.
\end{example}

%+++++++++++++++++++++++++++++++++++++++++++++++++++++++++++++++++++++++++++++++++++++++++++++++++++++++++++++++++++++++++++ 
\section{Résolution d'une équation du premier degré}
%+++++++++++++++++++++++++++++++++++++++++++++++++++++++++++++++++++++++++++++++++++++++++++++++++++++++++++++++++++++++++++

Si \( a\), \( b\) et \( c\) sont des nombres (donnés) et si \( x\) est un nombre à trouver pour résoudre
\begin{equation}
    a\times x+b=c,
\end{equation}
on fait :
\begin{enumerate}
    \item
        $ax+b=c$
    \item
        $ax=c-b$
    \item
        $x=\frac{ c-b }{ a }$.
\end{enumerate}

\begin{example}
    Résoudre
    \begin{equation}
        5\times x+9=44.
    \end{equation}
    \begin{enumerate}
        \item
            \( 5x+9=44\)
        \item
            \( 5x=44-9\)
        \item
             \( 5x=35\)
         \item
             \( x=7\).
    \end{enumerate}
\end{example}

%+++++++++++++++++++++++++++++++++++++++++++++++++++++++++++++++++++++++++++++++++++++++++++++++++++++++++++++++++++++++++++ 
\section{Interprétation du résultat}
%+++++++++++++++++++++++++++++++++++++++++++++++++++++++++++++++++++++++++++++++++++++++++++++++++++++++++++++++++++++++++++

En groupe :
% This is part of Un soupçon de mathématique sans être agressif pour autant
% Copyright (c) 2015
%   Laurent Claessens
% See the file fdl-1.3.txt for copying conditions.

%--------------------------------------------------------------------------------------------------------------------------- 
\subsection*{Activité : le double de l'âge}
%---------------------------------------------------------------------------------------------------------------------------

\begin{description}
    \item[Problème 1] Sylvia a sept ans de plus que sa sœur Rose. Dans $10$ ans, Sylvia aura le double de l'âge de Rose. Quel est l'âge de Rose ? Appeler $x$ l'âge de Rose.

    \item[Problème 2] En $2000$, Paul avait $10$ ans et Louis $17$ ans. En quelle année, l'âge de Louis a-t-il été le double de l'âge de Paul ? Appeler $x$ la différence entre cette année et $2000$.
\end{description}

\begin{enumerate}
    \item
        Mettre chacun des deux problèmes en équation.
    \item
        Résoudre ces équations
    \item
        En déduire la solution de chacun des problèmes
\end{enumerate}

De \cite{NRHooXFvgpp4}
