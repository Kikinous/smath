% This is part of Un soupçon de mathématique sans être agressif pour autant
% Copyright (c) 2013
%   Laurent Claessens
% See the file fdl-1.3.txt for copying conditions.

\begin{corrige}{smath-0574}

    Il faut commencer par prouver que \( ABCD\) est un carré. Les longueurs des côtés se calculent facilement; par exemple
    \begin{equation}
        AB^2=4^2+2^2=20,
    \end{equation}
    donc \( AB=\sqrt{20}\). En calculant de même les autres longueurs, on voit que tous les côtés mesurent \( \sqrt{20}\). Notons que cela ne suffit pas pour prouver que le tout soit un carré parce que cela pourrait encore être un losange. 
    
    Nous devons de plus prouver qu'un des angles est droit. Pour cela nous prouvons que le triangle \( ABC\) est rectangle en utilisant la réciproque du théorème de Pythagore. Les carrés des longueurs des côtés de \( ABC\) sont :
    \begin{subequations}
        \begin{align}
            AB^2&=\sqrt{20}\\
            BC^2&=\sqrt{20}\\
            AC^2&=2^2+6^2=40.
        \end{align}
    \end{subequations}
    Nous avons donc bien \( AC^2=AB^2+BC^2\).


    En ce qui concerne la fenêtre, deux réponses sont possibles : le quart de \( [DA]\) du côté de \( D\) et le quart de \( [DA]\) su côté de \( A\). Pour les calculer, nous appliquons le fait que le quart est le milieu du milieu. D'abord le milieu de \( [DA]\) est :
    \begin{equation}
        M=\left( \frac{ 1+3 }{2};\frac{ 5+1 }{2} \right)=(2;3).
    \end{equation}
    Ensuite il faut calculer soit le milieu de \( [DM]\) soit le milieu de \( [AM]\). Le premier est :
    \begin{equation}
        \text{milieu de \( [DM]\)}=\left( \frac{ 2+1 }{2};\frac{ 5+3 }{2} \right)=\left( \frac{ 3 }{2};4 \right)
    \end{equation}
    et
    \begin{equation}
        \text{milieu de \( [AM]\)}=\left( \frac{ 2+3 }{2};\frac{ 3+1 }{2} \right)=\left( \frac{ 5 }{2};2 \right).
    \end{equation}

\end{corrige}
