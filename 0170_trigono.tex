% This is part of Un soupçon de mathématique sans être agressif pour autant
% Copyright (c) 2013-2014
%   Laurent Claessens
% See the file fdl-1.3.txt for copying conditions.

%+++++++++++++++++++++++++++++++++++++++++++++++++++++++++++++++++++++++++++++++++++++++++++++++++++++++++++++++++++++++++++ 
\section{Activité}
%+++++++++++++++++++++++++++++++++++++++++++++++++++++++++++++++++++++++++++++++++++++++++++++++++++++++++++++++++++++++++++

    La SNCF veut enrouler un fil de cuivre de \SI{100}{\meter} de long autour d'une grande bobine de \SI{1}{\meter} de diamètre. Combien de tours seront nécessaires ?

    À la moitié du deuxième tour, nous mettons une marque sur le fil. À quelle distance du début du fil se trouve la marque ?



%+++++++++++++++++++++++++++++++++++++++++++++++++++++++++++++++++++++++++++++++++++++++++++++++++++++++++++++++++++++++++++ 
\section{Enroulement de la droite numérique sur le cercle}
%+++++++++++++++++++++++++++++++++++++++++++++++++++++++++++++++++++++++++++++++++++++++++++++++++++++++++++++++++++++++++++


\begin{definition}
    Le \defe{cercle trigonométrique}{cercle!trigonométrique} est le cercle de centre \( (0;0)\) et de rayon \( 1\) muni de l'orientation dans le sens direct (le sens inverse des aiguilles d'une montre).
\end{definition}

Enroulement de la droite réelle sur le cercle :
\begin{center}
   \input{Fig_YORfWSM.pstricks}
\end{center}

Étant donné que la circonférence du cercle est \( 2\pi\), le nombre \( 2\pi\) de la droite réelle vient au même endroit que le nombre zéro. Le nombre \( 2\pi+x\) vient alors au même endroit que \( x\) pour tout \( x\).

%+++++++++++++++++++++++++++++++++++++++++++++++++++++++++++++++++++++++++++++++++++++++++++++++++++++++++++++++++++++++++++ 
\section{Cosinus et sinus}
%+++++++++++++++++++++++++++++++++++++++++++++++++++++++++++++++++++++++++++++++++++++++++++++++++++++++++++++++++++++++++++

\begin{definition}
    Soit le réel \( x\) et le point image \( M\) sur le cercle trigonométrique. Le \defe{cosinus}{cosinus} de \( x\) est l'abscisse du point \( M\) et le \defe{sinus}{sinus} de \(x\) est l'ordonnée du point \( M\).
\end{definition}

\begin{center}
   \input{Fig_LOBVHYF.pstricks}
\end{center}

\begin{propriete}
    Vu que \( \big( \cos(x),\sin(x) \big)\) sont les coordonnées de points sur le cercle trigonométrique, nous avons
    \begin{enumerate}
        \item
            \( -1\leq \cos(x)\leq 1\)
        \item
            \( -1\leq \sin(x)\leq 1\)
        \item
            \( \cos^2(x)+\sin^2(x)=1\).
    \end{enumerate}
\end{propriete}

\begin{propriete}
    Pour tout réel \( x\), nous avons
    \begin{enumerate}
        \item
            \( \cos(-x)=\cos(x)\)
        \item
        \( \sin(-x)=-\sin(x)\)
    \end{enumerate}
\end{propriete}
Cette propriété est due au fait que les points \( x\) et \( -x\) de l'axe réel viennent se placer sur le cercle trigonométrique sur des points symétriques par rapport à l'axe horizontal.
