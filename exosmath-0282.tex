% This is part of Un soupçon de mathématique sans être agressif pour autant
% Copyright (c) 2013-2014
%   Laurent Claessens
% See the file fdl-1.3.txt for copying conditions.

\begin{exercice}\label{exosmath-0282}

    Un premier sac contient quatre boules portant les numéros \( 0\), \( 1\), \( 4\) et \( 6\). Un second sac contient trois boules numérotées \( 1\), \( 3\) et \( 5\). Nous tirons une boule de chaque sac et considérons la somme des deux nombres obtenus.
    \begin{enumerate}
        \item
            Quelle est la probabilité d'obtenir \( 3\) ? Quelle est le probabilité d'obtenir \( 8\) ?
        \item
            Quel est le nombre le plus haut que l'on puisse obtenir ? Quelle est sa probabilité ?
        \item
            Quels sont les résultats les plus probables ?
    \end{enumerate}

\corrref{smath-0282}
\end{exercice}
