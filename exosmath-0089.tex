% This is part of Un soupçon de mathématique sans être agressif pour autant
% Copyright (c) 2012
%   Laurent Claessens
% See the file fdl-1.3.txt for copying conditions.

\begin{exercice}\label{exosmath-0089}

Une enquête est menée auprès des clients des trains Corail et TGV. Au total l'enquête porte sur \( 2450\) billets vendus dont \( 850\) sur le réseau TGV. Sur le réseau TGV, \( 14\%\) des billets sont de première classe, tandis qu'au total, \( 82\%\) sont de seconde classe.

\begin{enumerate}
    \item
        Remplir le tableau suivant.

\begin{center}
        \begin{tabular}[]{|c||c|c|c|}
            \hline
            &billets Corail&billets TGV& total\\
            \hline\hline
            billet première classe&&&\\
            \hline
            billet seconde classe&&&\\
            \hline
            total&&850&2450\\
            \hline
        \end{tabular}
\end{center}
\item
    Sandrine dit que environ \( 20\%\) des billets Corail vendus sont de première classe. Sophie prétend que cette proportion est en réalité de \( 30\%\). Qui a raison ? Justifier.
\item
    On propose le raisonnement suivant : \( 14\%\) des billets de TGV sont de première classe et \( 20\%\) des billets Corail sont de première classe. Donc, en faisant la somme, \( 34\%\) des billets vendus au total sont de première classe. Qu'en penser ? Vérifier.
\end{enumerate}

\corrref{smath-0089}
\end{exercice}
