% This is part of Un soupçon de mathématique sans être agressif pour autant
% Copyright (c) 2012
%   Laurent Claessens
% See the file fdl-1.3.txt for copying conditions.

\begin{corrige}{Premiere-0018}

    \begin{enumerate}
        \item
            Nous remplaçons \( x\) par \( -1\) :
            \begin{equation}
                f(-1)=\frac{1}{ (-1)^2+4\times (-1) }=\frac{1}{ 1-5 }=\frac{1}{ -4 }.
            \end{equation}
            Rappel : \( (-1)^2=1\). Donc \( f(-1)=-\frac{1}{ 4 }\).

            Pour calculer \( f(0)\), nous remplaçons \( x\) par zéro :
            \begin{equation}
                f(0)=\frac{1}{  0^2+4\times 0 }=\frac{1}{ 0 }.
            \end{equation}
            Et là, c'est le problème : ON NE PEUT PAS DIVISER PAR ZÉRO !! Donc \( f(0)\) n'existe pas. Autrement dit, il n'est pas possible de calculer \( f(0)\). Zéro n'est donc pas dans le domaine de \( f\).

        \item
            Pour le domaine de définition, nous nous souvenons de la règle \ref{ArtJgipNt} : il faut exclure du domaine les valeurs qui annulent le dénominateur. Ici les valeurs de \( x\) à exclure sont les solutions de l'équation
            \begin{equation}    \label{EqUFjiFN}
                x^2+4x=0.
            \end{equation}
            Pour résoudre cette équation, le mieux est de mettre \( x\) en évidence :
            \begin{equation}
                x^2+4x=x(x+4).
            \end{equation}
            Ensuite la règle du produit nul nous dit que \( x=0\) ou \( x=-4\) sont les deux solutions de \eqref{EqUFjiFN}.

            Au final, le domaine de \( f\) est \( \eR\) moins les nombres \( 0\) et \( -4\).
    \end{enumerate}

\end{corrige}
