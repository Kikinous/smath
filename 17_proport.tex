% This is part of Un soupçon de mathématique sans être agressif pour autant
% Copyright (c) 2015
%   Laurent Claessens
% See the file fdl-1.3.txt for copying conditions.

% This is part of Un soupçon de mathématique sans être agressif pour autant
% Copyright (c) 2015
%   Laurent Claessens
% See the file fdl-1.3.txt for copying conditions.

%--------------------------------------------------------------------------------------------------------------------------- 
\subsection*{Activité : qui a dit «proportionnel» ?}
%---------------------------------------------------------------------------------------------------------------------------

Les situations suivantes relèvent-elles d’une situation de proportionnalité ? Pourquoi ?
\begin{enumerate}
    \item

 Saïd achète 2 mètres de corde qui coûte 2,30 € le mètre.

\item
    Daniel a planté huit pieds de tomates dans son potager, et en a récolté \SI{14}{\kilo\gram}. L'an passé, il en avait planté 12 pieds et en avait récolté \SI{18}{\kilo\gram}. L'an prochain, il en plantera 10 pieds et espère en récolter \SI{16}{\kilo\gram}. 

\item
 À 6 ans, Armand chaussait du 30 et à 18 ans, il chausse du 42.
\item

 Abonnement à une revue : \( 6\) mois pour \( 18\)€, un an pour \( 32\)€ et \( 2\) ans pour \( 60\)€.
\item
    Un piéton se promène à allure régulière le long des quais de la Seine et parcourt \SI{3.5}{\kilo\meter} en 1 h 30.
\item

    On peut acheter de l'enduit de lissage par sac de 1 kg, 5 kg et 25 kg. Le mode d’emploi précise qu'il faut \SI{2.5}{\liter} d’eau pour \SI{10}{\kilo\gram}.
\item

 Un commerçant a décidé de faire une journée promotion en baissant tous les prix de $10$\%.
\item

 Un loueur de DVD propose la formule d'abonnement suivante : la carte d'adhésion coûte $10$€ et on paye $2$€ par DVD.

\end{enumerate}

De \cite{NRHooXFvgpp5}

%+++++++++++++++++++++++++++++++++++++++++++++++++++++++++++++++++++++++++++++++++++++++++++++++++++++++++++++++++++++++++++ 
\section{Tableau de proportionnalité}
%+++++++++++++++++++++++++++++++++++++++++++++++++++++++++++++++++++++++++++++++++++++++++++++++++++++++++++++++++++++++++++

\begin{Aretenir}
    Un tableau de nombres décrit une relation de \defe{proportionnalité}{proportionnalité} si un même coefficient (non nul) multiplicateur s’applique dans tout le tableau. On parle alors de \defe{coefficient de proportionnalité}{coefficient!de proportionnalité}.
\end{Aretenir}

\begin{example}
    Un train de marchandises fait la liaison entre Paris et Berlin; voici le relevé de son avancement en fonction du temps :
    \begin{equation*}
        \begin{array}[]{|c|c|c|c|c|}
            \hline
            \text{temps de parcours (heures)}&3&5&6&8\\
            \hline
            \text{distance parcourue (\si{\kilo\meter})}&270&450&540&720\\
            \hline
        \end{array}
    \end{equation*}
    Est-ce une situation de proportionnalité ?

    Le coefficient multiplicateur pour la première colonne est :
    \begin{equation}
        \frac{ 270 }{ 3 }=90.
    \end{equation}
    Vérifions si ce nombre fonctionne pour toutes les colonnes :
    \begin{enumerate}
        \item
            \( 5\times 90=450\)
        \item
            \( 6\times 90=540\)
        \item
            \( 8\times 90=720\).
    \end{enumerate}
    Le tableau proposé est bien un tableau de proportionnalité parce qu'un même coefficient multiplicateur s'applique pour tout le tableau.
\end{example}


Note pour moi : cette activité est très facultative 

% This is part of Un soupçon de mathématique sans être agressif pour autant
% Copyright (c) 2015
%   Laurent Claessens
% See the file fdl-1.3.txt for copying conditions.

%--------------------------------------------------------------------------------------------------------------------------- 
\subsection*{Activité : pétrole et confiture}
%---------------------------------------------------------------------------------------------------------------------------

Voici quelque ingrédients utilisés pour des confitures.
\begin{center}
    \begin{tabular}[]{|c|c|}
        \hline
        Confiture d'abricots& «\SI{500}{\gram} de sucre et \SI{500}{\gram} d'abricots» \\
        \hline
        Confiture de fraises&«\SI{450}{\gram} de sucre et \SI{750}{\gram} de fraises» \\
        \hline
        Confiture de cerises&  «\SI{800}{\gram} de sucre et \SI{2400}{\gram} de cerises» \\ 
        \hline
    \end{tabular}
\end{center}
Est-ce que la quantité de sucre ajoutée est proportionnelle à la quantité de fruits ?


 

%+++++++++++++++++++++++++++++++++++++++++++++++++++++++++++++++++++++++++++++++++++++++++++++++++++++++++++++++++++++++++++ 
\section{Quatrième proportionnelle}
%+++++++++++++++++++++++++++++++++++++++++++++++++++++++++++++++++++++++++++++++++++++++++++++++++++++++++++++++++++++++++++

\begin{definition}
    Dans une situation de proportionnalité, la \defe{quatrième proportionnelle}{quatrième proportionnelle} est le quatrième nombre calculé à partir de trois nombres déjà connus.
\end{definition}

%--------------------------------------------------------------------------------------------------------------------------- 
\subsection{Propriété additive}
%---------------------------------------------------------------------------------------------------------------------------

La pâte à pain coûte \( 12\)€ pour \SI{50}{\kilo\gram}. Remplir le tableau de prix :

\begin{equation*}
    \begin{array}[]{|c||c|c|c|c|c|}
        \hline
        \text{poids (\si{\kilo\gram})}&25&50&75&&200\\
        \hline\hline
        \text{prix (€)}&&12&&30&\\
        \hline
    \end{array}
\end{equation*}

%--------------------------------------------------------------------------------------------------------------------------- 
\subsection{Coefficient multiplicateur}
%---------------------------------------------------------------------------------------------------------------------------

Pour un médicament, il faut deux doses de sirop pour trois dose d'eau. Quelle quantité d'eau pour \SI{4.5}{\centi\liter} de sirop ?
\begin{equation*}
    \begin{array}[]{|c|c|c|}
        \hline
        \text{eau (\si{\centi\liter})}&3&\\
        \hline\hline
        \text{sirop (\si{\centi\liter})}&2&4.5\\
        \hline
    \end{array}
\end{equation*}
Résolution :
\begin{equation}
    4.5\times \frac{ 3 }{ 2 }=\frac{ 4.5\times 3 }{ 2 }=\frac{ 13.5 }{ 2 }=6.75.
\end{equation}

%+++++++++++++++++++++++++++++++++++++++++++++++++++++++++++++++++++++++++++++++++++++++++++++++++++++++++++++++++++++++++++ 
\section{Pourcentage}
%+++++++++++++++++++++++++++++++++++++++++++++++++++++++++++++++++++++++++++++++++++++++++++++++++++++++++++++++++++++++++++

% This is part of Un soupçon de mathématique sans être agressif pour autant
% Copyright (c) 2015
%   Laurent Claessens
% See the file fdl-1.3.txt for copying conditions.

%--------------------------------------------------------------------------------------------------------------------------- 
\subsection*{Activité : des mélanges}
%---------------------------------------------------------------------------------------------------------------------------

Un restaurateur prépare un mélange de jus de fruits : deux litres de jus d'orange pour trois litres de jus de pomme. 
\begin{enumerate}
    \item
        Compléter ce tableau :
        \begin{equation*}
            \begin{array}[]{|c|c|c|}
                \hline
                \text{Volume de jus d'orange}&\SI{2}{\liter}&\ldots\ldots\\
                \hline
                \text{Volume de mélange}&\ldots\ldots&\SI{100}{\liter}\\
                \hline
            \end{array}
        \end{equation*}
    \item
        Exprimer la proportion de jus de pomme dans le mélange sous forme d'une fraction ayant \( 100\) comme dénominateur.
\end{enumerate}


\begin{definition}
    Un \defe{pourcentage}{pourcent} est une fraction ayant \( 100\) comme dénominateur.
\end{definition}

\begin{Aretenir}
    Exprimer une proportion en pourcentage revient à l'exprimer avec une fraction ayant \( 100\) comme dénominateur.
\end{Aretenir}

\begin{example}
    Sur un sac de \( 12\) pommes, \( 3\) sont vertes.

    La proportion de pommes vertes est la fraction \( \dfrac{ 3 }{ 12 }\). Pour trouver le pourcentage de pommes vertes, il faut résoudre
    \begin{equation}
        \frac{ 3 }{ 12 }=\frac{ ? }{ 100 }
    \end{equation}
    ou encore remplir le tableau de proportionnalité
    \begin{equation*}
        \begin{array}[]{|c|c|}
            \hline
            3&\\
            \hline\hline
            12&100\\
            \hline
        \end{array}
    \end{equation*}
\end{example}

\begin{Aretenir}
    Pour calculer \( t\%\) d'un nombre, on le multiplie par \( \dfrac{ t }{ 100 }\).
\end{Aretenir}

\begin{example}
    Si \( 8\%\) des personnes sont gauchères, sur \( 350\) personnes, il y aura
    \begin{equation}
        \frac{ 8 }{ 100 }\times 350=\frac{ 8\times 350 }{ 100 }=28
    \end{equation}
    gauchers (environ).
\end{example}

%+++++++++++++++++++++++++++++++++++++++++++++++++++++++++++++++++++++++++++++++++++++++++++++++++++++++++++++++++++++++++++ 
\section{Types d'écritures}
%+++++++++++++++++++++++++++++++++++++++++++++++++++++++++++++++++++++++++++++++++++++++++++++++++++++++++++++++++++++++++++

Un même nombre peut être écrit de plusieurs façons. Les proportions -- qui sont des nombres -- n'échappent pas à la règle.
\begin{equation*}
    \begin{array}[]{|c|c|c|c|}
        \hline
        \text{situation}&\text{écriture fractionnaire}&\text{numérique}&\text{pourcentage}\\
        \hline\hline
        \text{\( 10\) filles sur une classe de \( 20\) élèves}&\dfrac{ 10 }{ 20 }=\dfrac{ 1 }{2}&0.5&\dfrac{ 50 }{ 100 }=50\%\\
        \hline
        \text{trois pommes vertes parmi \( 12\) pommes}&\dfrac{ 3 }{ 12 }=\dfrac{1}{ 4 }&0.25&\dfrac{ 25 }{ 100 }=25\%\\
        \hline
    \end{array}
\end{equation*}
Il est important de noter dans ces exemples que les différentes colonnes sont des façons différentes d'écrire \emph{le même nombre} :
\begin{equation}
    \frac{ 10 }{ 20 }=\frac{ 1 }{2}=\frac{ 50 }{ 100 }=0.5.
\end{equation}

Lorsque la proportion ne tombe pas sur un nombre décimal, il peut être nécessaire de recourir à une approximation numérique. Par exemple 
\begin{equation}
    \frac{ 3 }{ 7 }=3\div 7\simeq 0.428.
\end{equation}
Nous pouvons ajouter des lignes au tableau :
\begin{equation*}
    \begin{array}[]{|c|c|c|c|}
        \hline
        \text{situation}&\text{écriture fractionnaire}&\text{numéirque}&\text{pourcentage}\\
        \hline\hline
        \text{\( 3\) jours de pluie sur une semaine}&\dfrac{ 3 }{ 7 }&  \simeq 0.428 &\simeq \dfrac{ 42.8 }{ 100 }=42.8\%\\
        \hline
        \text{\( 4\) cahiers verts et \( 5\) cahiers rouges (proportion de verts)}&\dfrac{ 4 }{ 9 }&  \simeq 0.444 &\simeq \dfrac{ 44.4 }{ 100 }=44.4\%\\
        \hline
    \end{array}
\end{equation*}

%+++++++++++++++++++++++++++++++++++++++++++++++++++++++++++++++++++++++++++++++++++++++++++++++++++++++++++++++++++++++++++ 
\section{Notes pour moi-même}
%+++++++++++++++++++++++++++++++++++++++++++++++++++++++++++++++++++++++++++++++++++++++++++++++++++++++++++++++++++++++++++

Rappel pour moi-même en ce qui concerne les exercices \ref{exo2smath-0129} et \ref{exo2smath-0130} : la masse atomique de l'oxygène est \( 16\) (un atome d'oxygène est donc \( 16\) fois plus lourd qu'un d'hydrogène), la France a une population de \( 66\) millions et les USA de \( 320\) millions

