% This is part of Un soupçon de mathématique sans être agressif pour autant
% Copyright (c) 2012
%   Laurent Claessens
% See the file fdl-1.3.txt for copying conditions.

\begin{exercice}\label{exoSeconde-0033}

    Le tableau suivant donne la répartition des entreprises du secteur automobile en fonction de leur chiffre d'affaire (en millions d'euros).

    \begin{center}
    \begin{tabular}{|c||c|c|c|c|c|c|}
        \hline
        chiffre d'affaire&moins de \( 0.25\)&\( \mathopen[ 0.25 ,0.5 [\)&$\mathopen[ 0.5,1  [$&$\mathopen[ 1 , 2.5 [$&$\mathopen[ 2.5,5 ,  [$&$\mathopen[ 5 , 10 [$\\
            \hline\hline
            nombre d'entreprises&\( 137\)&\( 106\)&\( 112\)&$154$&\( 100\)&\( 33\)\\
            \hline
    \end{tabular}
    \end{center}

    Tracer l'histogramme correspondant. Pour rappel, la notation \( \mathopen[ a , b [\) signifie «de \( a\) (compris) à \( b\) (non compris)».

\corrref{Seconde-0033}
\end{exercice}
