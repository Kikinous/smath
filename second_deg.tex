%This is part of Un soupçon de mathématique sans être agressif pour autant
% Copyright (c) 2012-2013
%   Laurent Claessens
% See the file fdl-1.3.txt for copying conditions.


% Ce fichier contient le second degré pour les secondes; pas pour les premières.

%+++++++++++++++++++++++++++++++++++++++++++++++++++++++++++++++++++++++++++++++++++++++++++++++++++++++++++++++++++++++++++ 
\section{Définition}
%+++++++++++++++++++++++++++++++++++++++++++++++++++++++++++++++++++++++++++++++++++++++++++++++++++++++++++++++++++++++++++

\begin{definition}
    Une fonction \defe{polynôme du second degré}{polynôme!second degré} est une fonction de la forme
    \begin{equation}
        f(x)=ax^2+bx+c
    \end{equation}
    avec \( a\neq 0\). L'ensemble de définition des polynômes est \( \eR\).
\end{definition}

%+++++++++++++++++++++++++++++++++++++++++++++++++++++++++++++++++++++++++++++++++++++++++++++++++++++++++++++++++++++++++++ 
\section{Représentation graphique}
%+++++++++++++++++++++++++++++++++++++++++++++++++++++++++++++++++++++++++++++++++++++++++++++++++++++++++++++++++++++++++++

\newcommand{\CaptionFigLSaSLoS}{Quelque paraboles.}
\input{Fig_LSaSLoS.pstricks}

Le graphe d'un polynôme du second degré est une \defe{paraboles}{parabole}. Elles ont toutes la même forme générale : une courbe tournée vers le haut ou vers le bas. La courbe de \( f(x)=ax^2+bx+x\) est tournée vers le haut si \( a>0\) et tournée vers le bas si \( a<0\).
Quelque unes sont dessinées à la figure \ref{LabelFigLSaSLoS}. % From file LSaSLoS

%--------------------------------------------------------------------------------------------------------------------------- 
\subsection{Axe de symétrie}
%---------------------------------------------------------------------------------------------------------------------------

L'axe de symétrie de la courbe représentative de \( f(x)=ax^2+bx+c\) se trouve en résolvant par rapport à \( s\) l'équation
\begin{equation}
    f(s+h)=f(s-h).
\end{equation}

\begin{Aretenir}
    Le sommet de la parabole \( ax^2+bx+c\) se trouve à l'abscisse \( x_0=-\frac{ b }{ 2a }\) et la droite verticale est un axe de symétrie de la parabole.
\end{Aretenir}

%TODO : vérifier si ce clearpage est encore nécessaire.
\clearpage
%--------------------------------------------------------------------------------------------------------------------------- 
\subsection{Tableau de variation}
%---------------------------------------------------------------------------------------------------------------------------

Les tableaux de variations sont
\begin{multicols}{2}

    \begin{center}
        Si \( a>0\)

        \begin{equation*}
            \begin{array}[]{c|ccccc}
                x&&&-\frac{ b }{ 2a }&&\\
                \hline
                &\infty&&&&\infty\\
                f(x)&&\searrow&&\nearrow&\\
                &&&f\left( -\frac{ b }{ 2a } \right)&&\\
            \end{array}
        \end{equation*}
        
    \end{center}

    \columnbreak

    \begin{center}
        Si \( a<0\)


        \begin{equation*}
            \begin{array}[]{c|ccccc}
                x&&&-\frac{ b }{ 2a }&&\\
                \hline
                &&&f\left( -\frac{ b }{ 2a } \right)&&\\
                f(x)&&\nearrow&&\searrow&\\
                &-\infty&&&&-\infty\\
            \end{array}
        \end{equation*}
    \end{center}
\end{multicols}

%+++++++++++++++++++++++++++++++++++++++++++++++++++++++++++++++++++++++++++++++++++++++++++++++++++++++++++++++++++++++++++ 
\section{Étude du second degré}
%+++++++++++++++++++++++++++++++++++++++++++++++++++++++++++++++++++++++++++++++++++++++++++++++++++++++++++++++++++++++++++

En terme de résolution d'équations, il faut savoir que
\begin{equation}
    A^2=B^2
\end{equation}
demande d'avoir \( A=B\) ou \( A=-B\). Rechercher par exemple tous les \( x\) tels que \( x^2=25\).

