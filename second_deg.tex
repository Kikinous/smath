%This is part of Un soupçon de mathématique sans être agressif pour autant
% Copyright (c) 2012-2013
%   Laurent Claessens
% See the file fdl-1.3.txt for copying conditions.

%+++++++++++++++++++++++++++++++++++++++++++++++++++++++++++++++++++++++++++++++++++++++++++++++++++++++++++++++++++++++++++ 
\section{Étude du second degré}
%+++++++++++++++++++++++++++++++++++++++++++++++++++++++++++++++++++++++++++++++++++++++++++++++++++++++++++++++++++++++++++

En terme de résolution d'équations, il faut savoir que
\begin{equation}
    A^2=B^2
\end{equation}
demande d'avoir \( A=B\) ou \( A=-B\). Rechercher par exemple tous les \( x\) tels que \( x^2=25\).

%+++++++++++++++++++++++++++++++++++++++++++++++++++++++++++++++++++++++++++++++++++++++++++++++++++++++++++++++++++++++++++ 
\section{Exercices}
%+++++++++++++++++++++++++++++++++++++++++++++++++++++++++++++++++++++++++++++++++++++++++++++++++++++++++++++++++++++++++++

\Exo{smath-0238}
\Exo{smath-0201}
\Exo{smath-0220}
\Exo{smath-0251}
\Exo{smath-0146}
\Exo{smath-0143}
\Exo{smath-0087}
\Exo{smath-0088}
\Exo{smath-0131}
\Exo{smath-0133}
\Exo{smath-0267}
\Exo{smath-0269}
\Exo{smath-0270}
\Exo{smath-0271}
