% This is part of Un soupçon de mathématique sans être agressif pour autant
% Copyright (c) 2013
%   Laurent Claessens
% See the file fdl-1.3.txt for copying conditions.

\begin{exercice}[\cite{ZNloWAM}]\label{exosmath-0319}

    Soit \( f\) une fonction définie sur \( \mathopen[ -10 ;9 \mathclose]\) et vérifiant les conditions suivantes :
    \begin{itemize}
        \item 
            \( f(0)=1\),
        \item
            l'équation \( f(x)=0\) admet les solutions \( -3\), \( \frac{ 3 }{2}\) et \( 5\),
        \item
            le tableau de variations de \( f\) est
            \begin{equation*}
                \begin{array}[]{|c|ccccccc|}
                    \hline
                    x&-10&&2&&3&&9\\
                    \hline
                    &&&5&&&&3\\
                    f(x)&&\nearrow&&\searrow&&\nearrow&\\
                    &-2&&&&-4&&\\
                    \hline
                \end{array}
            \end{equation*}
            
    \end{itemize}
    Tracer une courbe représentative de la fonction \( f\) en respectant toutes les conditions.

\corrref{smath-0319}
\end{exercice}
