% This is part of Un soupçon de mathématique sans être agressif pour autant
% Copyright (c) 2012-2013
%   Laurent Claessens
% See the file fdl-1.3.txt for copying conditions.

\begin{exercice}\label{exoSeconde-0094}

\begin{wrapfigure}{r}{6.0cm}
   \vspace{-0.5cm}        % à adapter.
   \centering
   \input{Fig_CubeLigneTriangleHFMrVU.pstricks}
\end{wrapfigure}

        La figure ci-contre représente un cube dont les côtés font \unit{5}{\meter}. 
        \begin{enumerate}
            \item
        Est-ce que les droites \( (DB)\) et \( (BF)\) sont perpendiculaires ?
    \item
        Déterminer la longueur de tous les côtés du triangle \( DBF\). Donner une réponse exacte (en laissant les racines et les fractions simplifiées si il y en a).
    \item
        Dessiner en vraie grandeur le triangle \( DBF\) à l'échelle \unit{1}{\centi\meter} égal \unit{1}{\meter}.
        \end{enumerate}

        Attention pour le dessin : il y a moyen (en réfléchissant un peu) de tracer le côté \( [BD]\) de façon exacte, sans avoir recours à une approximation numérique de sa longueur.
        
\corrref{Seconde-0094}
\end{exercice}
