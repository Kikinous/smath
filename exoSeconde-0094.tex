% This is part of Un soupçon de mathématique sans être agressif pour autant
% Copyright (c) 2012
%   Laurent Claessens
% See the file fdl-1.3.txt for copying conditions.

\begin{exercice}\label{exoSeconde-0094}

    \begin{multicols}{2}
        La figure ci-contre représente un cube dont les côtés font \unit{25}{\meter}. 
        \begin{enumerate}
            \item
        Est-ce que les droites \( (DB)\) et \( (BF)\) sont perpendiculaires ?
    \item
        Déterminer la longueur de tous les côtés du triangle \( DBF\).
    \item
        Dessiner le triangle \( DBF\) à l'échelle \unit{1}{\centi\meter} égal \unit{1}{\meter}. Si vous préférez utiliser le carreau comme unité à la place du centimètre, allez-y.
                
        \end{enumerate}

        \columnbreak

        \begin{center}
%The result is on figure \ref{LabelFigCubeLigneTriangleHFMrVU}. % From file CubeLigneTriangleHFMrVU
%\newcommand{\CaptionFigCubeLigneTriangleHFMrVU}{<+Type your caption here+>}
\input{Fig_CubeLigneTriangleHFMrVU.pstricks}
        \end{center}


    \end{multicols}



\corrref{Seconde-0094}
\end{exercice}
