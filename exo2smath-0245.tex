% This is part of Un soupçon de mathématique sans être agressif pour autant
% Copyright (c) 2015
%   Laurent Claessens
% See the file fdl-1.3.txt for copying conditions.

\begin{exercice}\label{exo2smath-0245}

    Vrai ou faux ? (justifier)
    \begin{enumerate}
        \item
            Il est possible de tracer un triangle \( ABC\) rectangle en \( A\) avec \( \cos(\widehat{BCA})=\frac{ 3 }{ 2 }\).
        \item
            Toute pyramide à base triangulaire possède \( 6\) faces.
        \item
            Quelle que soit la valeur du nombre \( x\), l'égalité 
            \begin{equation}
                (x+3)\times (2-x)=4
            \end{equation}
            est correcte.
        \item
            Si un pyramide a pour base un polygone à \( n\) côtés, alors il possède \( n+1\) faces.
    \end{enumerate}

\corrref{2smath-0245}
\end{exercice}
