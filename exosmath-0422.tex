% This is part of Un soupçon de mathématique sans être agressif pour autant
% Copyright (c) 2013
%   Laurent Claessens
% See the file fdl-1.3.txt for copying conditions.

\begin{exercice}[\cite{EUDiHAq}]\label{exosmath-0422}

   \begin{minipage}{8cm}
       À partir du graphe ci-contre :
\begin{enumerate}
    \item
        Déterminer graphiquement les solutions de l'équation \( f(x)=3\).
    \item
        Tracer les tangentes à la parabole aux points d'abscisses \( 3\) et \( -2\).
    \item
        En utilisant votre dessin, donner la valeur approximative de \( f'(3)\).
    \item
        Résoudre l'équation \( f'(x)\leq 0\).
\end{enumerate}
   \end{minipage}
   \begin{minipage}{8cm}
   \input{Fig_VDqUhPV.pstricks}
   \end{minipage}
       
\corrref{smath-0422}
\end{exercice}
