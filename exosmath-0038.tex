% This is part of Un soupçon de mathématique sans être agressif pour autant
% Copyright (c) 2012
%   Laurent Claessens
% See the file fdl-1.3.txt for copying conditions.

\begin{exercice}\label{exosmath-0038}

    Un article a un prix hors taxes (HT) de \( 120\)€ et subit une augmentation de \( 20\%\). La TVA est de \( 19.6\%\). Calculer le prix final en remplissant les tableaux suivants :
    \begin{multicols}{2}
        \begin{tabular}[]{|c||c|}
            \hline
            Prix de départ HT&120\\
            \hline\hline
            TVA (\( 19.6\%\))&\\
            \hline
            PTTC avant augmentation&\\
            \hline
            Augmentation de \( 20\%\)&\\
            \hline
            Prix TTC avec augmentation&\\
            \hline
        \end{tabular}

    \columnbreak

        \begin{tabular}[]{|c||c|}
            \hline
            Prix de départ HT&120\\
            \hline\hline
            Augmentation de \( 20\%\)&\\
            \hline
            Prix HT avant augmentation&\\
            \hline
            TVA (\( 19.6\%\))&\\
            \hline
            Prix TTC avec augmentation&\\
            \hline
        \end{tabular}
    \end{multicols}

    Comparer les résultats du point de vue de l'acheteur puis du point de vue de l'état.

\corrref{smath-0038}
\end{exercice}
