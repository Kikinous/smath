% This is part of Un soupçon de mathématique sans être agressif pour autant
% Copyright (c) 2012
%   Laurent Claessens
% See the file fdl-1.3.txt for copying conditions.

\begin{exercice}\label{exosmath-0038}

    Un article a un prix hors taxes (HT) de \( 120\)€ et subit une augmentation de \( 20\%\). La TVA est de \( 19.6\%\). Calculer le prix final en remplissant les tableaux suivants :
    \begin{multicols}{2}
        \begin{tabular}[]{|c||c|}
            \hline
            Prix de départ HT&120\\
            \hline\hline
            TVA (\( 19.6\%\))&\\
            \hline
            Prix TTC avant augmentation&\\
            \hline
            Augmentation de \( 20\%\)&\\
            \hline
            Prix TTC avec augmentation&\\
            \hline
        \end{tabular}

    \columnbreak

        \begin{tabular}[]{|c||c|}
            \hline
            Prix de départ HT&120\\
            \hline\hline
            Augmentation de \( 20\%\)&\\
            \hline
            Prix HT avant augmentation&\\
            \hline
            TVA (\( 19.6\%\))&\\
            \hline
            Prix TTC avec augmentation&\\
            \hline
        \end{tabular}
    \end{multicols}

    Dans le premier cas, la TVA est appliquée avant l'augmentation du prix et dans le second cas, la TVA est appliquée après. Comparer les résultats du point de vue du client, du commerçant et de l'état.

    Question subsidiaire de recherche personnelle : que dit la loi ? Si un magasin écrit sur sa vitrine «Semaine de promotions : 20\% de réduction», c'est du prix TTC qu'on parle (en tant que clients, vous avez dû le remarquer). Est-ce que le commerçant doit quand même payer la TVA sur le prix HT usuel, ou bien l'état accepte de prendre moins ? (cette question est particulièrement importante pour les semaines de soldes !!)

\corrref{smath-0038}
\end{exercice}
