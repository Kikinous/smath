% This is part of Un soupçon de mathématique sans être agressif pour autant
% Copyright (c) 2015
%   Laurent Claessens
% See the file fdl-1.3.txt for copying conditions.

\begin{exercice}[\cite{NRHooXFvgpp4}]\label{exo2smath-0060}

    Calculer les longueurs demandées.

 %   Les figures sont dans phystricksUCAOooLGwJUe.py

    \begin{multicols}{2}
        \begin{enumerate}
            \item

               \( (BC)\parallel (MN)\).
\begin{center}
   \input{Fig_CSUHooJIJzsW.pstricks}
\end{center}
Calculer \( AN\) et \( AB\).

\item

    \( (ST)\parallel (BA)\)

    \begin{center}
\input{Fig_WRBDooZhkhcW.pstricks}
    \end{center}
    Calculer \( CT\) et \( AB\).

\item
    \( (PM)\parallel (BC)\)

\begin{center}
   \input{Fig_QMJQooQratnO.pstricks}
\end{center}


    Calculer \( AC\) et \( BC\).

\item

\( (ML)\parallel (KJ)\)

\begin{center}
   \input{Fig_BJOSooIdFKWA.pstricks}
\end{center}


Calculer \( IK\), \( MK\) et \( LM\).


        \end{enumerate}
    \end{multicols}
<++>

dsfsf

\corrref{2smath-0060}
\end{exercice}
