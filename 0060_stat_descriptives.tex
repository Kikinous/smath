% This is part of Un soupçon de mathématique sans être agressif pour autant
% Copyright (c) 2012-2013
%   Laurent Claessens, Pauline Klein
% See the file fdl-1.3.txt for copying conditions.


\EpsOrPdfincludegraphics[width=\linewidth]{BO_statistique_descriptive}


\setcounter{section}{-1}
%+++++++++++++++++++++++++++++++++++++++++++++++++++++++++++++++++++++++++++++++++++++++++++++++++++++++++++++++++++++++++++ 
\section{Activité : un peu de spam ?}
%+++++++++++++++++++++++++++++++++++++++++++++++++++++++++++++++++++++++++++++++++++++++++++++++++++++++++++++++++++++++++++

% This is part of Un soupçon de mathématique sans être agressif pour autant
% Copyright (c) 2013
%   Laurent Claessens
% See the file fdl-1.3.txt for copying conditions.

% ATTENTION : les chiffres donnés ici sont repris dans le cours au moment des ECC.

\begin{wrapfigure}[5]{r}{8cm}
   \vspace{-0.5cm}        % à adapter.
   \centering
   \input{Fig_YRQOoPE.pstricks}
\end{wrapfigure}

Le graphique ci-contre illustre le nombre de spam reçus aujourd'hui par les élèves d'une classe.
\begin{enumerate}
    \item
        Combien d'élèves y-a-t-il dans la classe ?
    \item
        Combien d'élèves ont reçu \( 5\) spams ou plus ?
    \item
        En moyenne combien de spam ont reçu les élèves aujourd'hui ?
    \item
        Diviser la classe en 4 groupes suivant le nombre de spams reçus.
\end{enumerate}



%+++++++++++++++++++++++++++++++++++++++++++++++++++++++++++++++++++++++++++++++++++++++++++++++++++++++++++++++++++++++++++ 
\section{Convention sur les quartiles et médiane}
%+++++++++++++++++++++++++++++++++++++++++++++++++++++++++++++++++++++++++++++++++++++++++++++++++++++++++++++++++++++++++++

\begin{definition}
    Nous considérons une liste de \( n\) valeurs triées par ordre croissant (avec éventuellement les répétitions).
    \begin{enumerate}
        \item
            La \defe{médiane}{médiane}, est la valeur qui sépare la liste en deux parties égales. C'est le terme du milieu si l'effectif total est impair et la moyenne des deux termes du milieu si l'effectif total est pair.
      \item 
          Le \defe{premier quartile}{quartile!premier} $Q_1$ est la plus petite valeur de la liste telle qu'au moins un quart des valeurs de la liste soient inférieures ou égales à $Q_1$.
        \item
            Le \defe{troisième quartile}{quartile!troisième} $Q_3$ est la plus petite valeur de la liste telle qu'au moins les trois quarts des valeurs de la liste soient inférieures ou égales à $Q_3$.
  \end{enumerate}
\end{definition}
Les conventions sur les quartiles sont résumées ici :

\begin{center}
   \input{Fig_KDtwIJf.pstricks}
\end{center}

En pratique :
\begin{itemize}
    \item 
  Pour le calcul de $Q_1$, on calcule $\dfrac{n}4$, puis on détermine le premier entier $p$ supérieur ou égal à $\dfrac{n}4$. Cet entier $p$ donne le rang de $Q_1$. 
  \item
  Pour le calcul de $Q_3$, on fait de même en remplaçant $\dfrac{n}4$ par $\dfrac{3n}4$. 
\end{itemize}


\begin{example}
    Cherchons les médianes et quartiles de la liste $5$, $12$, $17$, $12$, $10$, $7$.

    D'abord nous trions par ordre croissant en respectant les répétitions (attention : le \( 12\) vient deux fois) : 
    \begin{equation*}
        \begin{array}[]{|c||c|c|c|c|c|c|}
            \hline
            \text{numéro}&1&2&3&4&5&6\\
            \hline
            \text{valeur}&5&7&10&12&12&17\\
            \hline
        \end{array}
    \end{equation*}
    Il y a \( 6\) valeurs, donc la médiane est la demi-somme des deux du milieu : \( 10\) et \( 12\). La médiane est \( 11\).

    En ce qui concerne les quartiles, il y a \( 6\) valeurs. Donc un quart serait \( 1.5\). Nous mettons le premier quartile sur la deuxième valeur (arrondi par excès), c'est à dire que le premier quartile est \( 7\). Les trois quart serait \( 4.5\) et nous mettons le troisième quartile sur la cinquième valeur, c'est à dire \( 12\).

\end{example}

\section{Représentation graphique d'une série statistique}

%--------------------------------------------------------------------------------------------------------------------------- 
\subsection{Effectifs cumulés croissants}
%---------------------------------------------------------------------------------------------------------------------------

Nous classons les élèves d'une classe d'après la première lettre de leur noms. Le résultat est le tableau
\begin{center}
\begin{tabular}[]{|c||c|c|c|c|c|c|c|c|c|c|c|c|c|}
        \hline
        Lettre&a&b&c&d&h&j&l&m&p&r&s&t&total\\
        \hline\hline
        Effectifs&4&8&9&2&2&1&1&2&2&1&1&2&\\
        \hline
        ECC&&&&&&&&&&&&&\\
        \hline
\end{tabular}
\end{center}

\input{Fig_MAXkaGz.pstricks}

%--------------------------------------------------------------------------------------------------------------------------- 
\subsection{Passage à la fréquence}
%---------------------------------------------------------------------------------------------------------------------------

Nous reprenons le tableau du nombre d'élèves en fonctions de la lettre
\begin{center}
\begin{tabular}[]{|c||c|c|c|c|c|c|c|c|c|c|c|c||c|}
        \hline
        Lettre&a&b&c&d&h&j&l&m&p&r&s&t&total\\
        \hline\hline
        Effectifs&4&8&9&2&2&1&1&2&2&1&1&2&35\\
        \hline
        Fréquences&&&&&&&&&&&&&\\
        \hline
\end{tabular}
\end{center}

La \defe{fréquence}{fréquence} des élèves ayant un nom commençant par «p» est leur proportion dans la classe, à savoir \( \frac{ 2 }{ 35 }\).

\subsection{Histogramme}

Le tableau suivant donne la répartition des entreprises du secteur automobile en fonction de leur chiffre d'affaire (en millions d'euros).

\begin{center}
    \begin{tabular}{|c||c|c|c|c|c|c|}
        \hline
        chiffre d'affaire&moins de \( 0.25\)&\( \mathopen[ 0.25 ,0.5 [\)&$\mathopen[ 0.5;1  [$&$\mathopen[ 1 , 2.5 [$&$\mathopen[ 2.5;5 ,  [$&$\mathopen[ 5 , 10 [$\\
            \hline\hline
            nombre d'entreprises&\( 137\)&\( 106\)&\( 112\)&$154$&\( 100\)&\( 33\)\\
            \hline
    \end{tabular}
\end{center}

\begin{Aretenir}
Pour des données rassemblées en classes, l'\textbf{aire} du rectangle est proportionnelle à l'effectif (ou à la fréquence). 
\end{Aretenir}

Nous en dessinons l'histogramme suivant :
\begin{center}
   \input{Fig_LTenBUj.pstricks}
\end{center}


Consignes pour dessiner un histogramme :
\begin{enumerate}
    \item
        Trouver la plus haute boîte en calculant le rapport \( \frac{ \text{effectif} }{ \text{largeur} }\) pour chaque boîte.
    \item
        Se fixer une échelle pour que la plus haute boîte reste raisonnable : elle ne doit pas faire deux mètres de haut, ni un centimètre. Il faut viser environ \unit{10}{\centi\meter}.
    \item
        Tracer les boîtes.
    \item
        Mettre la graduation \emph{horizontale} en écrivant la légende correspondante.
    \item
        Ne pas mettre de graduation verticale en cas d'histogramme à pas non constant\footnote{C'est à dire ceux dont la largeur n'est pas la même pour toute les boîtes.}.
    \item
        Écrire l'effectif de la boîte au-dessus de la boîte.
    \item
        Éventuellement écrire l'unité de surface «un carreau= \ldots effectifs». Par exemple «un carreau = 50 entreprises», «un carreau = 15 personnes».
\end{enumerate}



\begin{example}
    Si nous avons un effectif de \( 25\) personnes, un quart des effectifs seraient \( 6.25\) personnes, et les trois quarts seraient \( 18.75\) personnes. Donc nous mettons les premier quartile sur la \( 7\Ieme\) personne et le troisième quartile sur la \( 19\Ieme\) personne.

    Le premier quart des effectifs seraient donc les personnes numéro \( 1\), \( 2\), \( 3\), \( 4\), \( 5\), \( 6\) et \( 7\). Et le dernier quart des personnes seront les personnes numéro \( 20\), \( 21\), \( 22\), \( 23\), \( 24\) et \( 25\).
\end{example}
