% This is part of Un soupçon de mathématique sans être agressif pour autant
% Copyright (c) 2012
%   Laurent Claessens
% See the file fdl-1.3.txt for copying conditions.

\begin{exercice}\label{exosmath-0216}

    Brice a lancé \( 50\) fois un dé, tandis que Marie l'a lancé un certain nombre de fois. Leurs résultats sont résumés dans le tableau suivant :
    \begin{center}
        \begin{tabular}[]{|c||c|c|c|c|c|c|}
            \hline
            Chiffre obtenu&1&2&3&4&5&6\\
            \hline\hline
            Brice (effectifs)&9&12&8&7&5&9\\
            \hline
            Marie (fréquence)&$0.24$&$0.12$&???&$0.17$&$0.08$&$0.19$\\
            \hline
        \end{tabular}
    \end{center}
    \begin{enumerate}
        \item
            Interpréter le \( 0.12\) de la ligne de Marie en termes de pourcentage.
            \item
                Quel est le pourcentage de $4$ obtenu par Brice ?
        \item
            Avec quelle fréquence Marie a-t-elle obtenu le chiffre \( 3\) ?
        \item
            Vrai ou faux (justifier)
            \begin{enumerate}
                \item
                    Marie a obtenu moins de \( 3\) que de \( 1\).
                \item
                    Brice et Marie ont obtenu la même proportion de \( 2\).
            \end{enumerate}
        \item
            Calculer la moyenne et l'écart-type de chacune des deux séries.
    \end{enumerate}

    %AFAIRE : citer le site de Guillaume Conan ici.

\corrref{smath-0216}
\end{exercice}
