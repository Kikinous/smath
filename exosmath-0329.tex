% This is part of Un soupçon de mathématique sans être agressif pour autant
% Copyright (c) 2013
%   Laurent Claessens
% See the file fdl-1.3.txt for copying conditions.

\begin{exercice}%[\cite{oklaEg}]
    \label{exosmath-0329}

    Selon une étude fiable, \( 26\%\) des français se déclarent allergiques au pollen. On étudie la fréquence $f$ des personnes allergiques dans un échantillon de taille $400$.
    \begin{enumerate}
        \item
 Au seuil de 95\%, dans quel intervalle $f$ varie-t-elle ?
 \item
    Dans un petit village de $400$ personnes, on relève $120$ personnes allergiques. Peut-on dire que le taux d’allergies dans ce village est anormalement élevé ?
    \end{enumerate}

\corrref{smath-0329}
\end{exercice}
