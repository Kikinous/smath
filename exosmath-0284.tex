% This is part of Un soupçon de mathématique sans être agressif pour autant
% Copyright (c) 2013
%   Laurent Claessens
% See the file fdl-1.3.txt for copying conditions.

\begin{exercice}\label{exosmath-0284}

Chaque année, le PIB augmente de \( 5\%\). Soit \( P_0\) le PIB de l'an \( 2000\), et nous notons \( P_n\) les PIB des années suivantes.
\begin{enumerate}
    \item
        Donner en fonction de \( P_0\) la valeur de \( P_1\), \( P_2\) et \( P_3\).
    \item
        Quel type de suite numérique suit le PIB ? Écrire une formule donnant \( P_n\) en termes de \( n\) et de \( P_0\).
    \item
    Au bout de combien d'années le PIB a-t-il doublé ?
\end{enumerate}

\corrref{smath-0284}
\end{exercice}
