% This is part of Un soupçon de mathématique sans être agressif pour autant
% Copyright (c) 2013
%   Laurent Claessens
% See the file fdl-1.3.txt for copying conditions.

\begin{corrige}{smath-0507}

    Nous laissons au lecteur le soin de faire un beau dessin.
    \begin{enumerate}
        \item
            Le milieu de \( [AC]\) est donné par la formule usuelle :
            \begin{equation}
                M\left( \frac{ 1+5 }{2};\frac{ 1+0 }{2} \right)=(3;\frac{ 1 }{2}).
            \end{equation}
        \item
            Pour trouver le point \( D\) nous savons que \( M\) doit être le milieu de la diagonale \( [BD]\) (parce que dans un parallélogramme, les deux diagonales ont le même milieu). En ce qui concerne les abscisses nous pouvons donc écrire
            \begin{equation}
                x_M=\frac{ x_B+x_D }{2},
            \end{equation}
            et en remplaçant par les nombres connus,
            \begin{equation}
                3=\frac{ 4+x_D }{2},
            \end{equation}
            ce qui donne \( x_D=2\).

            En ce qui concerne les ordonnées,
            \begin{equation}
                y_M=\frac{ y_B+y_D }{2},
            \end{equation}
            et en remplaçant par les nombres connus,
            \begin{equation}
                \frac{ 1 }{2}=\frac{ 2+y_D }{2},
            \end{equation}
            ce qui donne \( y_D=-1\).

            Les coordonnée du point \( D\) sont donc \( D=(2;-1)\).

        \item
            Nous vérifions si ce parallélogramme est un rectangle en vérifiant si l'angle \( \widehat{ABC}\) est droit. Pour cela nous vérifions la réciproque du théorème de Pythagore dans le triangle \( ABC\) dont l'hypoténuse serait \( [AC]\). Les longueurs sont :
            \begin{subequations}
                \begin{align}
                    AB^2=3^2+1^2=10\\
                    AC^2=4^2+1^2=17\\
                    BC^2=1^2+2^2=5.
                \end{align}
            \end{subequations}
            Donc \( AC^2=17\) alors que \( AB^2+BC^2=15\), ce qui signifie que le triangle \( ABC\) n'est pas rectangle, et par conséquent que le parallélogramme \( ABCD\) n'est pas un rectangle.

    \end{enumerate}

\end{corrige}
