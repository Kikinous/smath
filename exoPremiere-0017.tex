% This is part of Un soupçon de mathématique sans être agressif pour autant
% Copyright (c) 2012
%   Laurent Claessens
% See the file fdl-1.3.txt for copying conditions.

\begin{exercice}\label{exoPremiere-0017}

        Une ville comprend \( n\) habitants et augmente sa population de \( 2\%\) durant trois ans de suite. Quelle est la nouvelle population ?

        Solutions possibles :
        \begin{enumerate}
            \item
                \( 3\times 1.02\times n\) habitants;
            \item
                \( (1.02)^3n\) habitants;
            \item
                \( (1.02n)^3\) habitants.
        \end{enumerate}
        Conseil : si vous ne savez pas la solution, calculez les trois nombres proposés pour une population de \( n=10^6\) habitants. Quel est le résultat le plus raisonnable ?

\corrref{Premiere-0017}
\end{exercice}
