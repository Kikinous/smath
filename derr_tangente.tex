% This is part of Un soupçon de mathématique sans être agressif pour autant
% Copyright (c) 2013
%   Laurent Claessens
% See the file fdl-1.3.txt for copying conditions.

Il faut commencer par l'équation d'une droite de coefficient directeur donné passant par un point donné.

%+++++++++++++++++++++++++++++++++++++++++++++++++++++++++++++++++++++++++++++++++++++++++++++++++++++++++++++++++++++++++++ 
\section{Activité TICE : piste de décollage}
%+++++++++++++++++++++++++++++++++++++++++++++++++++++++++++++++++++++++++++++++++++++++++++++++++++++++++++++++++++++++++++

Un avion peu décoller lorsque sa vitesse dépasse les \unit{250}{\kilo\meter\per\hour}. Grâce à la poussée de ses moteurs, un avion de ligne au bout d'un temps \( t\) (en secondes) parcourt la distance (en mètres) de 
\begin{equation}
    d(t)=1.2t^2.
\end{equation}
Quelle longueur de piste faut-il prévoir pour que l'avion décolle avant le bout de la piste ?

Tracer le graphique de la vitesse (instantanée) en fonction de la distance parcourue. Faire attention aux unités.

%+++++++++++++++++++++++++++++++++++++++++++++++++++++++++++++++++++++++++++++++++++++++++++++++++++++++++++++++++++++++++++ 
\section{Tangente}
%+++++++++++++++++++++++++++++++++++++++++++++++++++++++++++++++++++++++++++++++++++++++++++++++++++++++++++++++++++++++++++

\begin{Aretenir}
    Pour trouver la tangente au graphe de la fonction \( f\) au point \( A=\big( a,f(a) \big)\), nous faisons :
    \begin{enumerate}
        \item
            Calculer \( m\), le nombre dérivé de $f$ en \( x=a\).
        \item
            La droite tangente est celle de coefficient angulaire \( m\) et passant par le point \( A=\big( a,f(a) \big)\), c'est à dire
            \begin{equation}
                y=mx+p
            \end{equation}
            où \( p\) doit encore être trouvé.
        \item
            Le \( p\) se trouve en résolvant l'équation
            \begin{equation}
                ma+p=f(a).
            \end{equation}
    \end{enumerate}
    Au final, la tangente a pour équation
    \begin{equation}
        y=m(x-a)+f(a).
    \end{equation}
\end{Aretenir}

Étant donné que la tangente est la droite qui colle le mieux à la courbe, nous avons les interprétations suivantes. Soit \( m\) la dérivée de la fonction \( f\) au point \( x=a\).
\begin{enumerate}
    \item
        Si \( m>0\) alors la fonction est croissante en \( a\).
    \item
        Si \( m<0\) alors la fonction est décroissante en \( a\).
\end{enumerate}
Et sourtout : plus \( m\) est grand, plus la pente de la courbe est forte.

%+++++++++++++++++++++++++++++++++++++++++++++++++++++++++++++++++++++++++++++++++++++++++++++++++++++++++++++++++++++++++++ 
\section{Exercices}
%+++++++++++++++++++++++++++++++++++++++++++++++++++++++++++++++++++++++++++++++++++++++++++++++++++++++++++++++++++++++++++

\Exo{smath-0300}
\Exo{smath-0301}
\Exo{smath-0397}
\Exo{smath-0387}
\Exo{smath-0351}
