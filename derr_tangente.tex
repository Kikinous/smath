% This is part of Un soupçon de mathématique sans être agressif pour autant
% Copyright (c) 2013
%   Laurent Claessens
% See the file fdl-1.3.txt for copying conditions.

Il faut commencer par l'équation d'une droite de coefficient directeur donné passant par un point donné.

%+++++++++++++++++++++++++++++++++++++++++++++++++++++++++++++++++++++++++++++++++++++++++++++++++++++++++++++++++++++++++++ 
\section{Activité TICE : piste de décollage}
%+++++++++++++++++++++++++++++++++++++++++++++++++++++++++++++++++++++++++++++++++++++++++++++++++++++++++++++++++++++++++++

Un avion peu décoller lorsque sa vitesse dépasse les \unit{250}{\kilo\meter\per\hour}. Grâce à la poussée de ses moteurs, un avion de ligne au bout d'un temps \( t\) (en secondes) parcourt la distance (en mètres) de 
\begin{equation}
    d(t)=1.2t^2.
\end{equation}
Quelle longueur de piste faut-il prévoir pour que l'avion décolle avant le bout de la piste ?

Tracer le graphique de la vitesse (instantanée) en fonction de la distance parcourue. Faire attention aux unités.

%+++++++++++++++++++++++++++++++++++++++++++++++++++++++++++++++++++++++++++++++++++++++++++++++++++++++++++++++++++++++++++ 
\section{Exercices}
%+++++++++++++++++++++++++++++++++++++++++++++++++++++++++++++++++++++++++++++++++++++++++++++++++++++++++++++++++++++++++++

\Exo{smath-0300}
\Exo{smath-0301}
\Exo{smath-0351}
