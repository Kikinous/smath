% This is part of Un soupçon de mathématique sans être agressif pour autant
% Copyright (c) 2012
%   Laurent Claessens
% See the file fdl-1.3.txt for copying conditions.

\begin{exercice}\label{exosmath-0076}

    Soit \( I\) le milieu du segment \( [AB]\) et \( K\), un point n'appartenant pas à la droite \( (AB)\).
    \begin{enumerate}
        \item
            Construire les points \( C\) et \( D\) vérifiant \( \vect{ IC }=\vect{ IA }+\vect{ IK }\) et \( \vect{ ID }=\vect{ IB }+\vect{ IK }\).
        \item
            Quelle est la nature du quadrilatère \( AIKC\) ? (justifier en utilisant les propriétés des vecteurs et des parallélogramme).
        \item
            Quelle est la nature du quadrilatère \( IBDK\) ? (justifier en utilisant les propriétés des vecteurs et des parallélogramme).
        \item
            Prouver que \( K\) est le milieu de \( [CD]\).
        \item
            Prouver que \( \vect{ IC }=\vect{ BK }\).
    \end{enumerate}
    Nous nommons \( E\) le symétrique de \( I\) par rapport à \( K\).
    \begin{enumerate}
        \item
            Exprimer la définition de \( E\) par une égalité vectorielle.
        \item
            Prouver que \( \vect{ IC }+\vect{ ID }=\vect{ IE }\).
    \end{enumerate}

\corrref{smath-0076}
\end{exercice}
