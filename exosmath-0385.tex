% This is part of Un soupçon de mathématique sans être agressif pour autant
% Copyright (c) 2013
%   Laurent Claessens
% See the file fdl-1.3.txt for copying conditions.

\begin{exercice}\label{exosmath-0385}

    Rendez-vous dans le fichier \info{TD\_info\_IF\_1stmg1.ods}.
    \begin{enumerate}
        \item
            Dans l'onglet «Vite et mal fait», Max a cherché l'intervalle de fluctuation à \( 95\%\) d'une variable aléatoire binomiale de paramètres \( n=50\) et \( p=0.7\). Max n'a cependant pas laissé beaucoup de détails permettant de comprendre sa démarche. Ajouter des titres aux colonnes et quelque commentaires pour rendre le fichier plus compréhensible.
        \item
            La Terre comporte environ sept milliards d'habitants pour \( 900\) millions d'utilisateurs d'un célèbre(?) réseau social. Si nous prenons un échantillon de \( 30\) humains au hasard, quel est l'intervalle de fluctuation du nombre d'utilisateurs de ce réseau ? Sur une classe française de \( 30\) élèves, \( 29\) l'utilisent. Est-ce que cette classe est «normale» par rapport au reste de l'humanité ?

            Répondre en montrant vos calculs dans l'onglet «FB».
        \item
            Une agence spatiale prévoit de mettre \( 10\) satellites en orbite. Le cahier de charge demande que la fiabilité des lanceurs soit telle qu'au moins \( 9\) lancers sur les \( 10\) soient réussis. Nous demandons donc aux ingénieurs de faire en sorte que la probabilité de succès d'un lancer soit telle que la probabilité d'avoir \( 9\) ou \( 10\) réussites sur \( 10\) lancers soit supérieure ou égale à \( 99\%\).

            Rendez-vous dans l'onglet «fusée». Remplir la colonne \info{P(X>=n)} et trouver la valeur de \info{p} pour que \info{P(X>=9)} soit au moins \( 0.99\).
        \item
            Tracer un diagramme en bâtons montrant la distribution de probabilité d'une loi binomiale de paramètres \( n=200\) et \( p=0.37\). Quelle est la probabilité, pour cette variable aléatoire, d'obtenir entre \( 70\) et \( 103\) succès sur les \( 200\) tentatives ?

            Mettre les réponses dans l'onglet «Bernoulli».

    \end{enumerate}

\corrref{smath-0385}
\end{exercice}
