% This is part of Un soupçon de mathématique sans être agressif pour autant
% Copyright (c) 2014-2015
%   Laurent Claessens
% See the file fdl-1.3.txt for copying conditions.

%--------------------------------------------------------------------------------------------------------------------------- 
\subsection*{Activité : droite graduée}
%---------------------------------------------------------------------------------------------------------------------------

\begin{enumerate}
    \item
        Tracer une droite graduée d'origine \( O\) et y placer les points \( A(3)\), \( B(4)\) et \( D(7)\). 

\begin{center}
\input{Fig_CZFJooUDaKCj.pstricks}
\end{center}

    \item
        Construire ensuite le point \( K\) tel que \( A\) soit le milieu de \( [BK]\). Quelle est l'abscisse du point \( K \) ?
    \item
        Construire ensuite le point \( S\) tel que \( B\) soit le milieu de \( [AS]\). Quelle est l'abscisse du point \( S \) ?
    \item
        Construire ensuite le point \( L\) tel que \( A\) soit le milieu de \( [DL]\). Quelle est l'abscisse du point \( L \) ?
\end{enumerate}
