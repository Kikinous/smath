% This is part of Un soupçon de mathématique sans être agressif pour autant
% Copyright (c) 2012
%   Laurent Claessens
% See the file fdl-1.3.txt for copying conditions.

\begin{corrige}{Seconde-0054}

    \begin{enumerate}
        \item
            Il y a un seul chiffre dans «5», donc \( f(5)=1\).
        \item
            Il y a quatre chiffres dans «5200», donc \( f(5200)=4\).
        \item
            N'importe quel nombre de \( 5\) chiffres fait l'affaire. Par exemple \( 12000\), ou bien \( 99999\).
        \item
            Combien de nombres s'écrivent avec un seul chiffre ? Réponse : \( 10\). Il y a \( 0\), \( 1\), \( 2\), \( 3\), \( 4\), \( 5\), \( 6\), \( 7\), \( 8\) et \( 9\).
        \item
            Combien y a-t-il de nombres à $3$ chiffres ? Le premier nombre à \( 3\) chiffres est \( 100\); le dernier est \( 999\). Ça en fait \( 900\).

            On pourrait croire qu'il y en a seulement \( 899\) parce que \( 999-100=899\). Mais en réalité la différence \( 999-100\) compte le nombre de nombres \emph{entre} \( 100\) et \( 999\), alors qu'ici on veut savoir tous les nombres \emph{y compris} \( 100\) et \( 999\).

            Si vous n'avez pas compris le paragraphe précédent, voici une autre explication. Si le dernier nombre était \( 101\), il y aurait eu deux nombres : \( 100\) et \( 101\).
        \item
            Presque tous les nombres vont bien. Par exemple avec \( n=54\) on a \( f(54+1)=f(55)=2\) et \( f(54)=2\).
        \item
            \( n=99\) répond à la question. En effet \( f(99)=2\) alors que \( f(99+1)=f(100)=3\).
    \end{enumerate}

\end{corrige}
