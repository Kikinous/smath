% This is part of Un soupçon de mathématique sans être agressif pour autant
% Copyright (c) 2015
%   Laurent Claessens
% See the file fdl-1.3.txt for copying conditions.

\begin{exercice}\label{exo2smath-0248}

    \begin{multicols}{2}

Nous avons dessiné la même pyramide en perspective cavalière et sous forme de parton. Les dimensions données sont $RU=\SI{7}{\centi\meter}$, \( US=\SI{5}{\centi\meter}\) et \( TU=\SI{3}{\centi\meter}\). Tous les angles en \( U\) sont droits.


Ajouter au patron un maximum de codage : reporter les longueurs (données ou que vous pouvez déduire), coder les segments de même longueur, les angles droits,\ldots

\columnbreak

 % Ces figures sont de XABSooLSRHxF
\input{Fig_PDNJooMoHgCM.pstricks}\hspace{-1cm}\input{Fig_EIHJooJrGgmN.pstricks}
    \end{multicols}

\corrref{2smath-0248}
\end{exercice}
