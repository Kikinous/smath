% This is part of Un soupçon de mathématique sans être agressif pour autant
% Copyright (c) 2012
%   Laurent Claessens
% See the file fdl-1.3.txt for copying conditions.

\begin{exercice}\label{exoSeconde-0079}

Soit \( \mC\) le cercle de centre \( (0;0)\) et passant par le point \( A=(1;3)\).
\begin{multicols}{2}
    \begin{enumerate}
        \item
            Calculer le rayon du cercle.
        \item
            À quelle distance de \( (0;0)\) se trouve le point \( (3;1)\) ?
        \item
            Est-ce que le point \( (3;1)\) est sur le cercle ? (répondre en utilisant les calculs précédents)
        \item
            Est-ce que le point \( (2;1)\) est sur le cercle ?
        \item
            Donner un point sur le cercle, un point hors du cercle et un point à l'intérieur du cercle (autres que les points \( (3;1)\) et \( (2;1)\)).
    \end{enumerate}
\end{multicols}

\corrref{Seconde-0079}
\end{exercice}
