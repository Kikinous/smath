%This is part of Un soupçon de mathématique sans être agressif pour autant
% Copyright (c) 2013-1014
%   Laurent Claessens
% See the file fdl-1.3.txt for copying conditions.

%+++++++++++++++++++++++++++++++++++++++++++++++++++++++++++++++++++++++++++++++++++++++++++++++++++++++++++++++++++++++++++ 
\section*{Activité}
%+++++++++++++++++++++++++++++++++++++++++++++++++++++++++++++++++++++++++++++++++++++++++++++++++++++++++++++++++++++++++++

%This is part of Un soupçon de mathématique sans être agressif pour autant
% Copyright (c) 2014
%   Laurent Claessens
%See the file fdl-1.3.txt for copying conditions.


    \begin{enumerate}
        \item
    Richard veut parcourir \unit{60}{\kilo\meter} à vélo en deux heures et demie. À quelle vitesse doit-il pédaler ? À quelle vitesse doit-il pédaler pour effectuer ce trajet en deux heures et dix minutes ?

\item
    Écrire la fonction qui à \( x\) fait correspondre la vitesse à laquelle il faut se déplacer pour effectuer \unit{60}{\kilo\meter} en une heure plus \( x\) minutes.
\item
    Retrouver les réponses de la première question en utilisant la fonction trouvée à la seconde question.            
\item
    Toujours en utilisant cette fonction, à quelle vitesse faut-il avancer pour faire le voyage en \( 50\) minutes ? \( 10\) minutes ? \( 1\) minute ?
    \end{enumerate}




%Un muret vertical de deux mètres de haut est placé trois mètres devant la façade d'une maison. Nous voulons placer un projecteur au sol de telle sorte que l'ombre du muret couvre la fenêtre de la chambre des enfants. Le sommet de cette fenêtre se situe à \unit{2.5}{\meter}.
    
%Où placer le projecteur ?

%Voir la figure \ref{LabelFigBIlgjwy}. % From file BIlgjwy
%\newcommand{\CaptionFigBIlgjwy}{La figure de l'exercice \ref{exosmqth-0379}.}
%    \begin{center}
%\input{Fig_BIlgjwy.pstricks}
%    \end{center}



%+++++++++++++++++++++++++++++++++++++++++++++++++++++++++++++++++++++++++++++++++++++++++++++++++++++++++++++++++++++++++++ 
\section{Fonction homographique}
%+++++++++++++++++++++++++++++++++++++++++++++++++++++++++++++++++++++++++++++++++++++++++++++++++++++++++++++++++++++++++++

\begin{definition}
    Une \defe{fonction homographique}{homographique} est une fonction de la forme
    \begin{equation}
        f(x)=\frac{ ax+b }{ cx+d }.
    \end{equation}
Son ensemble de définition est l'ensemble des \( x\in \eR\) tels que \( cx+d\neq 0\).
\end{definition}

%--------------------------------------------------------------------------------------------------------------------------- 
\subsection{Allure de la courbe}
%---------------------------------------------------------------------------------------------------------------------------



Quelque exemples à la figure \ref{LabelFigTQDooEgJgPZ}. % From file TQDooEgJgPZ

\newpage

\newcommand{\CaptionFigTQDooEgJgPZ}{Des fonctions homographiques}
\input{Fig_TQDooEgJgPZ.pstricks}

\clearpage


%+++++++++++++++++++++++++++++++++++++++++++++++++++++++++++++++++++++++++++++++++++++++++++++++++++++++++++++++++++++++++++ 
\section{Fonction inverse}
%+++++++++++++++++++++++++++++++++++++++++++++++++++++++++++++++++++++++++++++++++++++++++++++++++++++++++++++++++++++++++++

\begin{definition}
    La \defe{fonction inverse}{fonction!inverse} est la fonction définie sur \( \eR\setminus\{ 0 \}\) qui à \( x\) fait correspondre \( \frac{1}{ x }\).
\end{definition}

Son graphe et son tableau de variation les suivants : 
\begin{multicols}{2}

\begin{equation*}
    \begin{array}[]{ccccccc}
        -\infty&&&0&&&+\infty\\
        \hline
        0&&&|&+\infty&&\\
        &\searrow&&|&&\searrow&\\
        &&-\infty&|&&&0\\
    \end{array}
\end{equation*}

\columnbreak

%The result is on figure \ref{LabelFigfigureMIdFCNN}. % From file figureMIdFCNN
%\newcommand{\CaptionFigfigureMIdFCNN}{<+Type your caption here+>}
\begin{center}
\input{Fig_figureMIdFCNN.pstricks}
\end{center}
\end{multicols}

L'ensemble de définition de la fonction \( x\mapsto \frac{1}{ x }\) est l'ensemble de tous les réels sauf \( x=0\), c'est à dire \( \eR\setminus\{ 0 \}\).

\begin{definition}
    La courbe représentative de la fonction inverse est nommée \defe{hyperbole}{hyperbole}.
\end{definition}

%+++++++++++++++++++++++++++++++++++++++++++++++++++++++++++++++++++++++++++++++++++++++++++++++++++++++++++++++++++++++++++ 
\section{Inéquations avec la fonction inverse}
%+++++++++++++++++++++++++++++++++++++++++++++++++++++++++++++++++++++++++++++++++++++++++++++++++++++++++++++++++++++++++++

Les inéquations sont toujours un peu délicates parce qu'on ne peut pas multiplier par des nombres dont on ignore la valeur.

\begin{example}
    Il est vrai que
    \begin{equation}
        5<7.
    \end{equation}
    Si nous multiplions les deux côtés de cette inéquation par \( 3\) nous obtenons
    \begin{equation}
        15<21.
    \end{equation}
    Pas de problèmes. Mais si nous multiplions \( 5<7\) par \( -2\) par exemple, nous obtenons
    \begin{equation}
        -10 < -14,
    \end{equation}
    qui est FAUX. D'une certaine manière, les nombres négatifs sont rangés dans le sens inverse des nombres positifs; ce qui est vrai est
    \begin{equation}
        -10 > -14.
    \end{equation}
\end{example}

\begin{Aretenir}
    Si nous multiplions une inéquation par un nombre négatif, nous devons renverser le sens de l'inéquation.
\end{Aretenir}

En particulier, lorsque nous résolvons une inéquation, nous ne pouvons pas multiplier par \( x\) parce que nous ne savons pas a priori la valeur de \( x\). Peut-être qu'il est positif, ou peut-être qu'il est négatif.

\begin{example}
    Si nous devons résoudre
    \begin{equation}
        \frac{1}{ x }>-\frac{ 2 }{ 3 },
    \end{equation}
    nous ne pouvons pas commencer par multiplier par \( x\) et écrire
    \begin{equation}
        1>-\frac{ 2 }{ 3 }x
    \end{equation}
\end{example}



