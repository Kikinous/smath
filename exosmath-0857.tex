% This is part of Un soupçon de mathématique sans être agressif pour autant
% Copyright (c) 2014
%   Laurent Claessens
% See the file fdl-1.3.txt for copying conditions.

\begin{exercice}\label{exosmath-0857}

    Deux classes sur huit au collège de Clerval sont des classes de cinquièmes. Lequel des diagrammes suivants représente cette proportion ?

\begin{center}
   \input{Fig_MBTooHyyNvjooZERO.pstricks}
   \input{Fig_MBTooHyyNvjooTWO.pstricks}
   \input{Fig_MBTooHyyNvjooONE.pstricks}
\end{center}
Quelle est la proportion de classes de niveaux pairs au collège ?
\begin{multicols}{4}
    \begin{enumerate}
        \item
            \( \dfrac{ 4 }{ 8 }\)
        \item
            \( \dfrac{ 1 }{ 3 }\)
        \item
            \( \dfrac{ 2 }{ 6 }\).
        \item
            \( \dfrac{ 1 }{ 2 }\)
    \end{enumerate}
\end{multicols}

\corrref{smath-0857}
\end{exercice}
