% This is part of Un soupçon de mathématique sans être agressif pour autant
% Copyright (c) 2014
%   Laurent Claessens
% See the file fdl-1.3.txt for copying conditions.

\begin{exercice}\label{exosmath-0710}

    Un vendeur de téléphones s'est intéressé aux motifs de mécontentement de ses clients. Plus précisément il a noté les clients qui se plaignait de batterie défectueuses et d'écrans griffés. Sur \( 600\) téléphones retournés en magasin, il a noté les résultats suivants :
    \begin{equation*}
        \begin{array}[]{c|c|c|c}
            &\text{batterie ok}&\text{batterie défectueuse}&\text{total}\\
            \hline
            \text{écran ok}&&150&350\\
            \hline
            \text{écran griffé}&&50&\\
            \hline
            \text{total}&400&&600\\
        \end{array}
    \end{equation*}
    Par exemple le nombre \( 250\) représente le nombre de téléphones retournés avec un écran griffé. Nous notons les événements suivants :
    \begin{itemize}
        \item $E$ : «le téléphone a un écran griffé» 
        \item $B$ : «la batterie du téléphone est défectueuse» 
    \end{itemize}
    \begin{enumerate}
        \item
            Expliquer en français ce que représente le nombre \( 150\) du tableau, et le compléter.
        \item
            Nous tirons un téléphone au hasard parmi les \( 600\). Décrire les événements \( K\), \( L\) et \( M\) à l'aide de \( E\) et \( B\), et en calculer les probabilités.
            \begin{itemize}
                \item \( K\) : «le téléphone a un écran griffé mais la batterie est correcte» 
                \item \( L\) : «le téléphone présente au moins un des deux défauts étudiés» 
                \item \( M\) : «le téléphone a un défaut autre que l'écran ou la batterie» 
            \end{itemize}
    \end{enumerate}

\corrref{smath-0710}
\end{exercice}
