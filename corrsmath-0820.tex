% This is part of Un soupçon de mathématique sans être agressif pour autant
% Copyright (c) 2014
%   Laurent Claessens
% See the file fdl-1.3.txt for copying conditions.

\begin{corrige}{smath-0820}

    \begin{enumerate}
        \item
            Il s'agit d'ajouter les parenthèses pour effectuer la soustraction en premier :
            \begin{equation}
                8\times (5-1)=8\times 4=32.
            \end{equation}
        \item
            Le calcul est
            \begin{equation}
                3\times 7-7\times 2=21-14=7.
            \end{equation}
        \item
            Pour obtenir zéro, il est possible de mettre les parenthèses comme ceci :
            \begin{equation}
                3\times (7-7)\times 2=3\times 0\times 2.
            \end{equation}
    \end{enumerate}

\end{corrige}
