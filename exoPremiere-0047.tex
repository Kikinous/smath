% This is part of Un soupçon de mathématique sans être agressif pour autant
% Copyright (c) 2012
%   Laurent Claessens
% See the file fdl-1.3.txt for copying conditions.

\begin{exercice}\label{exoPremiere-0047}

    Factoriser si possible les polynômes suivants.
    \begin{multicols}{3}
        \begin{enumerate}
            \item
                \( f_1(x)=3x^2-3x-6\)
            \item
                \( f_2(x)=3x^2+\frac{ 3 }{ 2 }x-\frac{ 3 }{2}\)
            \item
                \( f_3(x)=3x^2+7x-1\)
            \item
                \( f_4(x)=-x^2+12x+28\)
            \item
                \( f_5(x)=-3x^2+x-2\)
            \item
                \( f_6(x)=-3x^2+5x-2\)
            \item
                \( f_7(x)=9-x^2\)
            \item
                \( f_8(x)=x^2+2x+1\)
            \item
                \( f_9(x)=-4x^2+5x-3\)
            \item
                \( f_{10}(x)=25-4x^2\)
        \end{enumerate}
    \end{multicols}

\corrref{Premiere-0047}
\end{exercice}
