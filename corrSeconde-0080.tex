% This is part of Un soupçon de mathématique sans être agressif pour autant
% Copyright (c) 2012
%   Laurent Claessens
% See the file fdl-1.3.txt for copying conditions.

\begin{corrige}{Seconde-0080}

    %TODO : faire un dessin.

    \begin{enumerate}
        \item
        \item
        \item
            Le sommet de la tour est à la hauteur \( h\) par rapport au sol qui lui-même est à la distance \( R\) du centre de la Terre. Donc
            \begin{equation}
                OS=OA+AS=R+h.
            \end{equation}
        \item
        \item
            \( R\)
        \item
            Le théorème de Pythagore s'écrit \( SB^2+OB^2=OS^2\) qui donne, en remplaçant par ce qu'on vient de voir :
            \begin{subequations}
                \begin{align}
                    SB^2+R^2=(R+h)^2\\
                    SB=\sqrt{(R+h)^2-R^2}.
                \end{align}
            \end{subequations}
        \item
            En isolant \( SB\) dans l'équation précédente, et en développant \( (R+h)^2=R^2+2Rh+h^2\),
            \begin{equation}
                SB=\sqrt{R^2+h^2+2Rh-R^2}=\sqrt{h^2+2Rh}.
            \end{equation}
        \item
            Le théorème de Pythagore dit que \( SB^2+AS^2=AB^2\) et en remplaçant, \( h^2+2Rh=h^2+AB^2\), donc
            \begin{equation}
                AB=\sqrt{2Rh}.
            \end{equation}
        \item
            Dans le cas de la tour Eiffel, il ne faut pas oublier de convertir les unités : \( h=\unit{0.324}{\kilo\meter}\) et \( R=6300\), donc
            \begin{equation}
                AB\simeq\unit{63.89}{\kilo\meter}.
            \end{equation}
            Donc environ \( 64\) kilomètres.
    \end{enumerate}

\end{corrige}
