% This is part of Un soupçon de mathématique sans être agressif pour autant
% Copyright (c) 2014
%   Laurent Claessens
% See the file fdl-1.3.txt for copying conditions.

% This is part of Un soupçon de mathématique sans être agressif pour autant
% Copyright (c) 2014
%   Laurent Claessens
% See the file fdl-1.3.txt for copying conditions.

%--------------------------------------------------------------------------------------------------------------------------- 
\subsection*{Activité : Napoléon se place}
%---------------------------------------------------------------------------------------------------------------------------

Les Anglais et les Autrichiens ont pris position en deux points \( A\) et \( B\) distants de \( \SI{10}{\kilo\meter}\). Napoléon veut pouvoir être à même de les attaquer tous les deux d'égale manière et décide donc de se positionner en un point $N$ qui serait à égale distance de \( A\) que de \( B\).

Bien entendu il pourrait se placer au milieu du segment \( [AB]\), mais ainsi il se ferait trop facilement attaquer des deux côtés à la fois (pas fou le Corse!). Où peut-il se placer ? Faire un dessin pour l'aider.


%+++++++++++++++++++++++++++++++++++++++++++++++++++++++++++++++++++++++++++++++++++++++++++++++++++++++++++++++++++++++++++ 
\section{Médiatrice}
%+++++++++++++++++++++++++++++++++++++++++++++++++++++++++++++++++++++++++++++++++++++++++++++++++++++++++++++++++++++++++++

\begin{definition}
    La \defe{médiatrice}{médiatrice} d'un segment est la droite perpendiculaire à ce segment en son milieu.
\end{definition}

\begin{center}
   \input{Fig_KSQooHHfEpe.pstricks}
\end{center}

\begin{propriete}
    La médiatrice du segment \( [AB]\) est l'ensemble des points équidistants de \( A\) et \( B\).
\end{propriete}

Pour tracer la médiatrice du segment \( [AB]\), 
\begin{itemize}
    \item tracer un arc de cercle de centre \( A\).
    \item en gardant le même rayon, tracer un arc de cercle de centre \( B\);
    \item la droite passant par les deux intersections est la médiatrice.
\end{itemize}
Attention : il faut choisir le rayon assez grand pour qu'il y ait des intersections.

\begin{center}
   \input{Fig_BKWooKbAHbD.pstricks}
\end{center}

\begin{propriete}
    Dans un triangle, les trois médiatrices sont concourantes.
\end{propriete}

La figure complète et la figure sans les codages, à distribuer.
\begin{center}
   \input{Fig_MRBWooRCSkaB.pstricks}
\end{center}


À ce niveau : exercice. Tracer un triangle au hasard et ses trois médiatrices.

%+++++++++++++++++++++++++++++++++++++++++++++++++++++++++++++++++++++++++++++++++++++++++++++++++++++++++++++++++++++++++++ 
\section{Cercle circonscrit}
%+++++++++++++++++++++++++++++++++++++++++++++++++++++++++++++++++++++++++++++++++++++++++++++++++++++++++++++++++++++++++++

%--------------------------------------------------------------------------------------------------------------------------- 
\subsection*{Activité : chevaliers de la table ronde}
%---------------------------------------------------------------------------------------------------------------------------

% This is part of Un soupçon de mathématique sans être agressif pour autant
% Copyright (c) 2014
%   Laurent Claessens
% See the file fdl-1.3.txt for copying conditions.

%--------------------------------------------------------------------------------------------------------------------------- 
\subsection*{Activité : chevaliers de la table ronde}
%---------------------------------------------------------------------------------------------------------------------------

Les chevaliers de la table ronde se sont disputés; Tristan, Perceval et Lancelot sont allés mettre leurs chaises un peu à l'écart et refusent de bouger. Sur le dessin suivant, l'unité est en mètres. Quel doit être le rayon de la nouvelle table ronde de telle sorte à ce que ces trois chevaliers puissent siéger ?

\begin{center}
   \input{Fig_DTFZooTlciUT.pstricks}
\end{center}


\begin{definition}
    Le \defe{cercle circonscrit}{cercle!circonscrit} à un triangle est le cercle passant par les trois sommets de ce triangle.
\end{definition}

\begin{propriete}
    Le centre du cercle circonscrit à un triangle est le point d'intersection de ses trois médiatrices.
\end{propriete}

\begin{center}
   \input{Fig_QTCQooFtDgwk.pstricks}
\end{center}

\begin{center}
   \input{Fig_ZSAooVHmHWd0.pstricks}
\end{center}

En ce qui concerne la construction.

\begin{center}
   \input{Fig_ZSAooVHmHWd1.pstricks}
\end{center}


\begin{itemize}
    \item Tracer les trois médiatrices,
    \item le centre du cercle circonscrit est le point d'intersections des trois, nommé \( O\)
    \item le rayon du cercle circonscrit est au choix \( OA\), \( OB\) ou \( OC\).
\end{itemize}

Les longueurs \( OA\), \( OB\) et \( OC\) sont égales. En effet,
\begin{enumerate}
    \item
        \( OA=OB\) parce que \( O\) est sur la médiatrice du segment \( [AB]\).
    \item
        \( OA=OC\)parce que \( O\) est sur la médiatrice du segment \( [AC]\).
    \item
        \( OB=OC\)parce que \( O\) est sur la médiatrice du segment \( [BC]\).
\end{enumerate}

%+++++++++++++++++++++++++++++++++++++++++++++++++++++++++++++++++++++++++++++++++++++++++++++++++++++++++++++++++++++++++++ 
\section{Médiane}
%+++++++++++++++++++++++++++++++++++++++++++++++++++++++++++++++++++++++++++++++++++++++++++++++++++++++++++++++++++++++++++

\begin{definition}[\cite{HZJHooXYaaom}]
    Dans un triangle la \defe{médiane}{médiane} issue d'un sommet est la droite passant par ce sommet et par le milieu du côté opposé.
\end{definition}

Sur le dessin suivant, nous voyons la médiane issue du sommet \( C\).
\begin{center}
   \input{Fig_YAFJooWuUmHJ.pstricks}
\end{center}
On dit aussi que \(  (CM) \) est la médiane \defe{relative}{médiane!relativement} au côté \( [AB]\).

\begin{propriete}
    Dans un triangle, les trois médianes sont concourantes.
\end{propriete}

\begin{center}
   \input{Fig_GARYooJCnpFS.pstricks}
\end{center}

%+++++++++++++++++++++++++++++++++++++++++++++++++++++++++++++++++++++++++++++++++++++++++++++++++++++++++++++++++++++++++++ 
\section{Hauteur}
%+++++++++++++++++++++++++++++++++++++++++++++++++++++++++++++++++++++++++++++++++++++++++++++++++++++++++++++++++++++++++++

\begin{definition}
    Dans un triangle, la \defe{hauteur}{hauteur} issue d'un sommet est la droite passant par ce sommet et perpendiculaire au côté opposé.
\end{definition}

Sur le dessin suivant, nous voyons la hauteur issue du sommet \( A\).

\begin{center}
   \input{Fig_ECQDooWEpuCM.pstricks}
\end{center}
On dit aussi la hauteur \defe{relative}{hauteur!relativement} au côté \( (CB)\).

\begin{propriete}
    Dans un triangle, les trois hauteurs sont concourantes.
\end{propriete}

\begin{center}
   \input{Fig_QZPMooIiOQpy.pstricks}
\end{center}
