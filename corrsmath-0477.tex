% This is part of Un soupçon de mathématique sans être agressif pour autant
% Copyright (c) 2013
%   Laurent Claessens
% See the file fdl-1.3.txt for copying conditions.

\begin{corrige}{smath-0477}

    Il suffit d'utiliser la formule du milieu qui n'est autre que la formule de la moyenne :
    \begin{equation}
        \left( \frac{ 1+25 }{2};\frac{ 1+37 }{2} \right)=(13;19)
    \end{equation}
    Les numéros écrits sur le siège du milieu sont donc \( 13\) et \( 19\).

\end{corrige}
