% This is part of Un soupçon de mathématique sans être agressif pour autant
% Copyright (c) 2012
%   Laurent Claessens
% See the file fdl-1.3.txt for copying conditions.

\begin{exercice}\label{exosmath-0046}

    Soit le segment \( [AB]\) de longueur \( 6\), et \( K\) un point situé à sur \( [AB]\) à distance \( x\) de \( A\) (\( x\in\mathopen[  0; 6 \mathclose]\)).
    \begin{multicols}{2}
    \begin{enumerate}
        \item
            Dessiner deux carrés sur les segments \( [AK]\) et \( [KB]\).
        \item
            L'aire totale de la figure est \( A(x)=x^2+(6-x)^2\). Pourquoi ?
        \item
            Démontrer que \( A(x)=2x^2-12x+36\).
        \item
            Démontrer que \( A(x)=2(x-3)^2+18\).
        \item
            Calculer \( A(3)\).
        \item
            Montrer que \( A(x)\geq A(3)\) pour tout \( x\in \mathopen[ 0; 6 \mathclose]\).
    \end{enumerate}
    \end{multicols}

    %TODO : je crois que cet exercice devrait être mis dans un autre chapitre.

\corrref{smath-0046}
\end{exercice}
