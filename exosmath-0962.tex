% This is part of Un soupçon de mathématique sans être agressif pour autant
% Copyright (c) 2014
%   Laurent Claessens
% See the file fdl-1.3.txt for copying conditions.

\begin{exercice}\label{exosmath-0962}

    Le client d'un architecte demande de construire une pièce carré dont l'aire serait \SI{25}{\meter\squared}. L'architecte prévoit donc des plans dont les murs ont une longueur de \SI{5}{\meter}. Pourquoi ?

    Le client change d'avis et veut une pièce plus grande : \SI{30}{\meter\squared}.
    \begin{enumerate}
        \item
            L'architecte comprend qu'il faudra des murs dont la longueur sera entre \( 5\) et \( 6\) mètres. Pourquoi ?
        \item
            Est-ce que la longueur des murs à construire sera plus grande ou plus petite que \SI{5.5}{\meter} ?
        \item
            Dans un tableur, mettez en colonne tous les nombres de \( 5\) à \( 5.5\) avec un intervalle de \( 0.1\). Faites calculer les carrés. Donner un nouvel encadrement, à \( 0.1\) près, de la longueur du mur.
    \end{enumerate}

\corrref{smath-0962}
\end{exercice}
