% This is part of Un soupçon de mathématique sans être agressif pour autant
% Copyright (c) 2014
%   Laurent Claessens
% See the file fdl-1.3.txt for copying conditions.

En mathématique, les énoncés sont souvent écrits sous la forme « {\bf Si} \boxed{condition} {\bf alors} \boxed{conséquence}».

\begin{definition}
    La \defe{réciproque}{réciproque} d'un énoncé s'obtient en inversant la condition et la conséquence.
\end{definition}

\begin{example}
    \begin{description}
        \item[Énoncé] «{\bf Si} le dernier chiffre d'un nombre est «\( 6\)» {\bf alors} il est divisible par \( 2\).» 
        \item[Réciproque] «{\bf Si} un nombre est divisible par \( 2\) {\bf alors} son dernier chiffre est «\( 6\)» 
    \end{description}
    Note : même si un énoncé est vrai, sa réciproque peut être fausse.
\end{example}

\begin{definition}
    Un \defe{contre-exemple}{contre-exemple} à un énoncé est un exemple pour lequel un énoncé est faux.
\end{definition}

\begin{example}
    \begin{description}
        \item[Énoncé] Toutes les fables de La Fontaine parlent d'animaux.
        \item[Contre-exemple] Faux ! «Le Chêne et le roseau» ne parle pas d'animaux.
    \end{description}
\end{example}

\begin{example}
    \begin{description}
        \item[Énoncé] Si un quadrilatère a deux côtés parallèles, alors c'est un parallélogramme.
        \item[Contre-exemple] Faux ! Ceci a deux côtés parallèles mais n'est pas un parallélogramme :
            \begin{center}
               \input{Fig_TJDooMNBjjK.pstricks}
            \end{center}
    \end{description}
\end{example}

