% This is part of Un soupçon de mathématique sans être agressif pour autant
% Copyright (c) 2014
%   Laurent Claessens
% See the file fdl-1.3.txt for copying conditions.

\begin{definition}
    Si une affirmation est écrite sous la forme «  {\bf Si} \texttt{propriété 1} {\bf alors} \texttt{propriété 2} », la \defe{réciproque}{réciproque} en est «  {\bf Si} \texttt{propriété 2} {\bf alors} \texttt{propriété 1} »
\end{definition}
De \cite{XEFooQxuWQb}

\begin{example}
    \begin{description}
        \item[Affirmation] «{\bf Si} le dernier chiffre d'un nombre est «\( 6\)» alors il est divisible par \( 2\).» 
        \item[Réciproque] «{\bf Si} un nombre est divisible par \( 2\) {\bf alors} son dernier chiffre «\( 6\)» 
    \end{description}
    Note : même si une affirmation est vraie, sa réciproque peut être fausse.
\end{example}

\begin{definition}
    Un \defe{contre-exemple}{contre-exemple} est un exemple pour lequel une affirmation est fausse.
\end{definition}

\begin{example}
    \begin{description}
        \item[Affirmation] Toutes les fables de La Fontaine parlent d'animaux.
        \item[Contre-exemple] Faux ! «La Mort et le bûcheron» ne parle pas d'animaux.
    \end{description}
\end{example}

\begin{example}
    \begin{description}
        \item[Affirmation] Si un quadrilatère a deux côtés parallèles, alors c'est un parallélogramme.
        \item[Contre-exemple] Faux ! Ceci a deux côtés parallèles mais n'est pas un parallélogramme :
            \begin{center}
               \input{Fig_TJDooMNBjjK.pstricks}
            \end{center}
    \end{description}
\end{example}


