% This is part of Un soupçon de mathématique sans être agressif pour autant
% Copyright (c) 2012,2014
%   Laurent Claessens
% See the file fdl-1.3.txt for copying conditions.

\begin{exercice}\label{exosmath-0084}

    Dans chacune des configurations suivantes, dire si les droites \( d_1\) et \( d_2\) sont sécantes. Si elles le sont, donner le point d'intersection.
    \begin{enumerate}
        \item
            \( d_1 : y=x-1\) et \( d_2:y=-3x+3\).
        \item
            \( d_1:y= 5x-1\) et \( d_2:y= -2x\).
        \item
            \( d_1:y= 5x-1\) et \( d_2:y= 2+5x\).
    \end{enumerate}

    Faire l'exercice 32 de la page 187.

\corrref{smath-0084}
\end{exercice}
