% This is part of Un soupçon de mathématique sans être agressif pour autant
% Copyright (c) 2013-2014
%   Laurent Claessens
% See the file fdl-1.3.txt for copying conditions.

\begin{exercice}\label{exosmath-0576}

    Une association de riverains se plaint d'un carrefour sur lequel trop de monde oublie son clignotant. Elle affirme que \( 40\%\) des automobilistes tournent sans clignotant. La mairie décide de prendre le problème à bras le corps et place un policier pour compter. Sur \( 500\) voitures observées, le policier note \( 190\) oublis de clignotant.
    \begin{enumerate}
        \item
            Déterminer un intervalle de confiance au seuil \( 95\%\)
        \item
            D'après cet échantillon, peut-on valider les allégations de notre association de riverains ?
        \item
            Combien de voitures faut-il observer pour obtenir un intervalle de confiance d'une amplitude inférieure à \( 0.05\) ?
    \end{enumerate}

\corrref{smath-0576}
\end{exercice}
