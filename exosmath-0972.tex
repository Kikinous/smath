% This is part of Un soupçon de mathématique sans être agressif pour autant
% Copyright (c) 2014
%   Laurent Claessens
% See the file fdl-1.3.txt for copying conditions.

\begin{exercice}[\ldots\ldots/3]\label{exosmath-0972}

    Parmi les fractions suivantes, lesquelles représentent des nombres plus grands que \( 1\) ?
    \begin{multicols}{3}
        \begin{enumerate}
            \item
                \( \dfrac{ 3 }{ 2 }\)
            \item
                \( \dfrac{ 2 }{ 3 }\)
            \item
                \( \dfrac{  12 }{ 10 }\)
        \end{enumerate}
    \end{multicols}
    Pour justifier, rappeler la règle qui détermine si une fraction est plus grande que \( 1\).

\corrref{smath-0972}
\end{exercice}
