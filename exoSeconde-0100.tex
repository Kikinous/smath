% This is part of Un soupçon de mathématique sans être agressif pour autant
% Copyright (c) 2012
%   Laurent Claessens
% See the file fdl-1.3.txt for copying conditions.

\begin{exercice}\label{exoSeconde-0100}

    Soient deux cercles \( C\) et \( C'\) de centre \( O\) et \( O'\) et de rayons différents. Nous supposons qu'ils se coupent en deux points que nous nommons \( A\) et \( B\).
    \begin{enumerate}
        \item
            Dessiner la situation.
        \item
            Prouver que les droites \( (AB)\) et \( (OO')\) sont perpendiculaires.
        \item
            Si \( C\) et \( C'\) ont même rayon, quelle est la nature du quadrilatère \( OAO'B\) ?
    \end{enumerate}

\corrref{Seconde-0100}
\end{exercice}
