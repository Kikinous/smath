% This is part of Un soupçon de mathématique sans être agressif pour autant
% Copyright (c) 2014
%   Laurent Claessens
% See the file fdl-1.3.txt for copying conditions.

\begin{exercice}\label{exosmath-0697}


\begin{wrapfigure}{r}{6.0cm}
   \vspace{-0.5cm}        % à adapter.
   \centering
   \input{Fig_MJZoobAMRb.pstricks}
\end{wrapfigure}

    Sur la figure ci-contre.

    \begin{enumerate}
        \item
            Quelles sont les coordonnées du vecteur \( \vect{ AB }\) ?
        \item
            Construire le vecteur \( \vect{ AB }+\vect{ AC }\).
        \item
            Placer le point \( D\) tel que \( \vect{ AD }=\vect{ AB }+\vect{ AC }\).
        \item
            Placer le point \( E\) de telle sorte que \( \vect{ EB }=\vect{ CA }\).
        \item
            Où se trouve le point \( E\) par rapport à \( D\) ?
    \end{enumerate}

\corrref{smath-0697}
\end{exercice}
