% This is part of Un soupçon de mathématique sans être agressif pour autant
% Copyright (c) 2013
%   Laurent Claessens
% See the file fdl-1.3.txt for copying conditions.

\begin{exercice}\label{exosmath-0403}

    Un directeur de salle de spectacle sait que si il vend le billet d'entrée à $15$ euros, il en vendra environ \( 130\). À chaque baisse de \( 1\) euro du prix de l'entrée, il aura \( 50\) clients supplémentaires.

    \begin{enumerate}
        \item
            Si le billet d'entrée est vendu \( x\) euros, combien de clients payeront ? (nous supposons que \( x\) est entre \( 0\) et \( 15\)).
        \item
            Montrer que si on vend le billet \( x\) euros, la recette sera de \( f(x)=-50x^2+880x\).
        \item
            Donner L'abscisse du sommet de la parabole associée à \( f\).
        \item
            Pour quelle abscisse la dérivée de \( f\) est-elle nulle ?
        \item
            À quel prix le billet doit être vendu pour maximiser les recettes ?
        \item
            Quelles sont les recettes maximales ?
    \end{enumerate}

\corrref{smath-0403}
\end{exercice}
