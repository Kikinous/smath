% This is part of Un soupçon de mathématique sans être agressif pour autant
% Copyright (c) 2012
%   Laurent Claessens
% See the file fdl-1.3.txt for copying conditions.

\begin{exercice}\label{exoPremiere-0074}

    Nous considérons l'épreuve de Bernoulli qui consiste à tirer une carte au hasard dans un paquet de \( 52\) cartes et regarder si c'est un cœur\footnote{Don Diègue était un professionnel de ce tirage dans \href{http://fr.wikisource.org/wiki/Le_Cid}{le Cid}, immortalisé par la demande anxieuse : «Rodrigues, as-tu du cœur ?»}.

    Nous considérons maintenant le schéma de Bernoulli consistant à répéter trois fois cette épreuve, en replaçant la carte dans le paquet après chaque tirage.
    \begin{enumerate}
        \item
            Dessiner l'arbre correspondant.
\item
    Quelle est la probabilité d'obtenir trois fois cœur ?
\item
    Quelle est la probabilité d'obtenir aucun cœur ?
\item
    Quelle est la probabilité d'obtenir exactement deux cœurs ?
            
    \end{enumerate}

\corrref{Premiere-0074}
\end{exercice}
