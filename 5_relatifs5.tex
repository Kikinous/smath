% This is part of Un soupçon de mathématique sans être agressif pour autant
% Copyright (c) 2014
%   Laurent Claessens
% See the file fdl-1.3.txt for copying conditions.

%+++++++++++++++++++++++++++++++++++++++++++++++++++++++++++++++++++++++++++++++++++++++++++++++++++++++++++++++++++++++++++ 
\section{Nombres relatifs}
%+++++++++++++++++++++++++++++++++++++++++++++++++++++++++++++++++++++++++++++++++++++++++++++++++++++++++++++++++++++++++++

% This is part of Un soupçon de mathématique sans être agressif pour autant
% Copyright (c) 2014
%   Laurent Claessens
% See the file fdl-1.3.txt for copying conditions.

%--------------------------------------------------------------------------------------------------------------------------- 
\subsection*{Activité : quelque situations plus petites que zéro}
%---------------------------------------------------------------------------------------------------------------------------

\begin{enumerate}
    \item
        Fantine a une dette de \( 15\) sous vis-à-vis des Thénardier. Elle parvient à leur envoyer \( 12\) sous, mais ensuite elle contracte une nouvelle dette de \( 6\) sous. Quelle est sa nouvelle situation ?
    \item
        Le premier Harry Potter a été écrit en 1997. Il y a combien d'année de cela ? Écrire l'opération effectuée. L'odyssée aurait été écrit environ \( 800\) ans avant notre ère. Il y a combien de temps de cela ? Écrire l'opération effectuée.
    \item
        Un plongeur saute d'un plongeoir situé à \SI{5}{\meter} et touche le fond de la piscine de \SI{4}{\meter} de profondeur. Quelle a été la distance verticale totale parcourue ? Écrire le calcul effectué.

\end{enumerate}



%+++++++++++++++++++++++++++++++++++++++++++++++++++++++++++++++++++++++++++++++++++++++++++++++++++++++++++++++++++++++++++ 
\section{Les opposés}
%+++++++++++++++++++++++++++++++++++++++++++++++++++++++++++++++++++++++++++++++++++++++++++++++++++++++++++++++++++++++++++

Nous connaissons déjà les nombres naturels : \( 1\), \( 2\), \( 3\), etc. Ils servent à compter des objets.
\begin{definition}
    Nous ajoutons les nombres \( \opp(1), \opp(2), \opp(3)\), etc. Ils servent à compter «en-dessous de zéro». On prononcé «opposé».
\end{definition}

\begin{example}
    \begin{itemize}
        \item Les dates avant Jésus,
        \item les dettes,
        \item les profondeurs en-dessous de l'eau ou de la terre.
    \end{itemize}
\end{example}

% This is part of Un soupçon de mathématique sans être agressif pour autant
% Copyright (c) 2014
%   Laurent Claessens
% See the file fdl-1.3.txt for copying conditions.

%--------------------------------------------------------------------------------------------------------------------------- 
\subsection*{Activité : droite graduée}
%---------------------------------------------------------------------------------------------------------------------------

\begin{enumerate}
    \item
        Tracer une droite graduée d'origine \( O\) plaçant points \( A(3)\), \( B(4)\) et \( D(7)\). 

\begin{center}
\input{Fig_CZFJooUDaKCj.pstricks}
\end{center}

    \item
        Construire ensuite le point \( K\) tel que \( A\) soit le milieu de \( [BK]\). Quelle est l'abscisse du point \( K \) ?
    \item
        Construire ensuite le point \( S\) tel que \( B\) soit le milieu de \( [AS]\). Quelle est l'abscisse du point \( S \) ?
    \item
        Construire ensuite le point \( L\) tel que \( A\) soit le milieu de \( [DL]\). Quelle est l'abscisse du point \( L \) ?
        
\end{enumerate}

