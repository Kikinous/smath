% This is part of Un soupçon de mathématique sans être agressif pour autant
% Copyright (c) 2014-2015
%   Laurent Claessens
% See the file fdl-1.3.txt for copying conditions.


% This is part of Un soupçon de mathématique sans être agressif pour autant
% Copyright (c) 2014
%   Laurent Claessens
% See the file fdl-1.3.txt for copying conditions.

%--------------------------------------------------------------------------------------------------------------------------- 
\subsection*{Activité : quelque situations plus petites que zéro}
%---------------------------------------------------------------------------------------------------------------------------

\begin{enumerate}
    \item
        Fantine a une dette de \( 15\) sous vis-à-vis des Thénardier. Elle parvient à leur envoyer \( 12\) sous, mais ensuite elle contracte une nouvelle dette de \( 6\) sous. Quelle est sa nouvelle situation ?
    \item
        Le premier Harry Potter a été écrit en 1997. Il y a combien d'année de cela ? Écrire l'opération effectuée. L'odyssée aurait été écrit environ \( 800\) ans avant notre ère. Il y a combien de temps de cela ? Écrire l'opération effectuée.
    \item
        Un plongeur saute d'un plongeoir situé à \SI{5}{\meter} et touche le fond de la piscine de \SI{4}{\meter} de profondeur. Quelle a été la distance verticale totale parcourue ? Écrire le calcul effectué.

\end{enumerate}



%+++++++++++++++++++++++++++++++++++++++++++++++++++++++++++++++++++++++++++++++++++++++++++++++++++++++++++++++++++++++++++ 
\section{Les opposés}
%+++++++++++++++++++++++++++++++++++++++++++++++++++++++++++++++++++++++++++++++++++++++++++++++++++++++++++++++++++++++++++

Nous connaissons déjà les nombres naturels : \( 1\), \( 2\), \( 3\), etc. Ils servent à compter des objets.
\begin{definition}
    Nous ajoutons les nombres \( -1, -2, -3\), etc. Ils servent à compter «en-dessous de zéro». 

    \begin{enumerate}
        \item
            Le nombre \( -1\) est l'opposé de \( 1\)
        \item
            Le nombre \( -4\) est l'opposé de \( 4\)
        \item
            Le nombre \( -7.3\) est l'opposé de \( 7.3\)
    \end{enumerate}
\end{definition}

\begin{definition}
    L'ensemble de tous les nombres positifs et négatifs est l'ensemble des \defe{nombres relatifs}{nombre!relatifs}.
\end{definition}

\begin{example}
    \begin{itemize}
        \item les dettes,
        \item les profondeurs en-dessous de l'eau ou de la terre.
        \item Les températures.
    \end{itemize}
\end{example}

%+++++++++++++++++++++++++++++++++++++++++++++++++++++++++++++++++++++++++++++++++++++++++++++++++++++++++++++++++++++++++++ 
\section{Droite graduée}
%+++++++++++++++++++++++++++++++++++++++++++++++++++++++++++++++++++++++++++++++++++++++++++++++++++++++++++++++++++++++++++

% This is part of Un soupçon de mathématique sans être agressif pour autant
% Copyright (c) 2014
%   Laurent Claessens
% See the file fdl-1.3.txt for copying conditions.

%--------------------------------------------------------------------------------------------------------------------------- 
\subsection*{Activité : droite graduée}
%---------------------------------------------------------------------------------------------------------------------------

\begin{enumerate}
    \item
        Tracer une droite graduée d'origine \( O\) plaçant points \( A(3)\), \( B(4)\) et \( D(7)\). 

\begin{center}
\input{Fig_CZFJooUDaKCj.pstricks}
\end{center}

    \item
        Construire ensuite le point \( K\) tel que \( A\) soit le milieu de \( [BK]\). Quelle est l'abscisse du point \( K \) ?
    \item
        Construire ensuite le point \( S\) tel que \( B\) soit le milieu de \( [AS]\). Quelle est l'abscisse du point \( S \) ?
    \item
        Construire ensuite le point \( L\) tel que \( A\) soit le milieu de \( [DL]\). Quelle est l'abscisse du point \( L \) ?
        
\end{enumerate}


\vspace{3cm}

Nous étendons la droite graduée pour y ajouter les abscisses «négatives» :
\begin{center}
    \input{Fig_ZMSWooTAPggA.pstricks}
\end{center}

\begin{definition}
    La \defe{distance à zéro}{distance!à zéro} d'un nombre relatif \( x\) est la distance sur la droite graduée entre le point d'abscisse \( x\) et l'origine.
\end{definition}

\begin{example}
    Sur le dessin suivant est représenté la distance à zéro de \( -3\) et \( 4\).
    \begin{center}
        \input{Fig_JSRHooOdlPDT.pstricks}
    \end{center}
    L'abscisse de \( A\) est \( -3\) et celle de \( B\) est \( 4\).
\end{example}

\begin{example}
    \begin{enumerate}
        \item
            La distance à zéro de \( 4\) est \( 4\),
        \item
            La distance à zéro de \( -4\) est \( 4\),
        \item
            La distance à zéro de \( 7.5\) est \( 7.5\),
        \item
            La distance à zéro de \( -7.5 \) est \( 7.5\),
    \end{enumerate}
\end{example}

%+++++++++++++++++++++++++++++++++++++++++++++++++++++++++++++++++++++++++++++++++++++++++++++++++++++++++++++++++++++++++++ 
\section{Comparaison de nombres relatifs}
%+++++++++++++++++++++++++++++++++++++++++++++++++++++++++++++++++++++++++++++++++++++++++++++++++++++++++++++++++++++++++++

% This is part of Un soupçon de mathématique sans être agressif pour autant
% Copyright (c) 2015
%   Laurent Claessens
% See the file fdl-1.3.txt for copying conditions.

%--------------------------------------------------------------------------------------------------------------------------- 
\subsection*{Activité : classer des dates}
%---------------------------------------------------------------------------------------------------------------------------

Placer les événements suivants sur une ligne du temps :
\begin{itemize}
    \item Le premier pas sur la Lune (\( 1969\)).
    \item La conquête des Gaules (\( -51\))
    \item Décès de Catule (\( -54\))
    \item Couronnement de Charlemagne (\( 800\))
    \item Fondation du comté de Bourgogne (\( 986\))
    \item Arrivée d'Aristote à Mytilène (\( -345\))
    \item Disparition des dinosaures (il y a \( 65\) millions d'années)
    \item Formation des Pyrénées (il y a \( 40\) millions d'années)
\end{itemize}


\begin{Aretenir}
    
    Pour comparer des nombres relatifs,

    \begin{itemize}
        \item 
Deux nombres relatifs positifs sont rangés dans l'ordre de leur distance à zéro.  
\item
Un nombre relatif négatif est inférieur à un nombre relatif positif. 
\item
Deux nombres relatifs négatifs sont rangés dans l'ordre inverse de leur distance à zéro.
    \end{itemize}
\end{Aretenir}

\begin{example}
    Comparer les nombres \( -9.9\) et \( -7.7\).
    \begin{itemize}
        \item Ce sont deux nombres négatifs, donc il faut les classer dans l'ordre inverse de leurs distances à zéro.
        \item \( 9.9 > 7.7\),
        \item donc \( -7.7 > -9.9\).
    \end{itemize}
\end{example}

%+++++++++++++++++++++++++++++++++++++++++++++++++++++++++++++++++++++++++++++++++++++++++++++++++++++++++++++++++++++++++++ 
\section{Dans le plan}
%+++++++++++++++++++++++++++++++++++++++++++++++++++++++++++++++++++++++++++++++++++++++++++++++++++++++++++++++++++++++++++

% This is part of Un soupçon de mathématique sans être agressif pour autant
% Copyright (c) 2015
%   Laurent Claessens
% See the file fdl-1.3.txt for copying conditions.

%--------------------------------------------------------------------------------------------------------------------------- 
\subsection*{Activité : placer dans un repère}
%---------------------------------------------------------------------------------------------------------------------------

<++>



\begin{definition}
    Deux droites graduées perpendiculaires de même origine est un \defe{repère orthogonal}{repère!orthogonal}. 
\end{definition}

\begin{Aretenir}
    Pour repérer un point dans le plan, on donne deux nombre : l'abscisse et l'ordonnée.
\end{Aretenir}

\begin{example}
    \begin{center}
        \input{Fig_QBDMooGskfKN.pstricks}
    \end{center}
    Sur ce dessin,
    \begin{itemize}
        \item les coordonnées du point \( A\) sont \(  (3;2)  \)
        \item les coordonnées du point \( B\) sont \(  (-5;3)  \)
        \item les coordonnées du point \( C\) sont \(  (1;-4)  \)
        \item les coordonnées du point \( H\) sont \(  (-3.5;0)  \)
    \end{itemize}
\end{example}

\begin{definition}
    \begin{itemize}
        \item 
            L'axe horizontal est appelé l'\defe{axe des abscisses}{axe!abscisse}.
        \item
            L'axe vertical est appelé l'\defe{axe des ordonnées}{axe!ordonnées}.
    \end{itemize}
\end{definition}


