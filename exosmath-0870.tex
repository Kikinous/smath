% This is part of Un soupçon de mathématique sans être agressif pour autant
% Copyright (c) 2014
%   Laurent Claessens
% See the file fdl-1.3.txt for copying conditions.

\begin{exercice}[\cite{NRHooXFvgpp5}]\label{exosmath-0870}

Effectuer les opérations suivantes :
\begin{multicols}{3}
    \begin{enumerate}
        \item
            \( \dfrac{ 3 }{ 4 }+\dfrac{ 6 }{ 4 }\)
        \item
            \( \dfrac{  1  }{ 2 }+\dfrac{  1  }{ 4 }\) 
        \item
            \( \dfrac{  5  }{ 6 }+\dfrac{  5  }{ 12 }\) 
        \item
            \( \dfrac{  13  }{ 14 }+\dfrac{  5  }{ 7 }\) 
        \item
            \( \dfrac{  6  }{ 7 }+\dfrac{  2  }{ 35 }\) 
        \item
            \( \dfrac{  11  }{ 81 }+\dfrac{  1  }{ 9 }\) 
    \end{enumerate}
\end{multicols}

\corrref{smath-0870}
\end{exercice}
