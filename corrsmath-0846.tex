% This is part of Un soupçon de mathématique sans être agressif pour autant
% Copyright (c) 2014
%   Laurent Claessens
% See the file fdl-1.3.txt for copying conditions.

\begin{corrige}{smath-0846}

    Le passage des degrés Fahrenheit aux degrés Celsius revient à appliquer le programme de calcul
    \begin{itemize}
        \item prendre la température en Fahrenheit,
        \item soustraire \( 32\),
        \item diviser par \( 1.8\).
    \end{itemize}
    \begin{enumerate}
        \item
            Pour savoir à quelle température en Celsius correspond \( 100 \) °F, on applique le programme :
            \begin{itemize}
                \item d'abord \( 100-32=68\),
                \item ensuite \( \dfrac{ 68 }{ 1.8 }\simeq 37.7\)
            \end{itemize}
            Pour être complet et réellement répondre à la question «donner une expression qui permet de calculer», il faut écrire
            \begin{equation}
                (100-32)\div 1.8
            \end{equation}
            ou
            \begin{equation}
                \frac{ 100-32 }{ 1.8 }
            \end{equation}
        \item 
            Il s'agit maintenant d'effectuer la conversion contraire : nous savons la température en Celsius et nous voulons savoir la température correspondante en Fahrenheit. Nous devons donc appliquer le programme à l'envers en partant de \( 20\). Autrement dit nous devons compléter les cases dans l'enchaînement
            \begin{equation}
                \boxed{\phantom{301}}\stackrel{-32}{\longrightarrow}\boxed{\phantom{301}}\stackrel{\div 1.8}{\longrightarrow}\boxed{20}.
            \end{equation}
            Cela revient à faire
            \begin{equation}
                \boxed{\phantom{301}}\stackrel{+32}{\longleftarrow}\boxed{\phantom{301}}\stackrel{\times 1.8}{\longleftarrow}\boxed{20}.
            \end{equation}
            La réponse est :
            \begin{equation}
                \boxed{68}\stackrel{+32}{\longleftarrow}\boxed{36}\stackrel{\times 1.8}{\longleftarrow}\boxed{20}.
            \end{equation}
            Donc \( 20\) degrés Fahrenheit correspondent à \( 68\) degrés Celsius.
        \item
            Il faut convertir \( 212\)°F en Celsius en suivant l'enchaînement
            \begin{equation}
                \boxed{212}\stackrel{-32}{\longrightarrow}\boxed{\phantom{301}}\stackrel{\div 1.8}{\longrightarrow}\boxed{ \phantom{301} }.
            \end{equation}
            C'est à dire :
            \begin{equation}
                \boxed{212}\stackrel{-32}{\longrightarrow}\boxed{180}\stackrel{\div 1.8}{\longrightarrow}\boxed{100}.
            \end{equation}
            Adèle doit donc porter de l'eau à ébullition. Elle est donc en train de faire des pâtes ou de cuire un œuf dur \ldots
    \end{enumerate}

\end{corrige}
