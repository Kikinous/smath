% This is part of Un soupçon de mathématique sans être agressif pour autant
% Copyright (c) 2013
%   Laurent Claessens
% See the file fdl-1.3.txt for copying conditions.

\begin{exercice}\label{exosmath-0518}

    Dessiner le graphe d'une fonction \( f\) définie sur l'intervalle \( \mathopen[ -3 ;10 \mathclose]\)  satisfaisant aux caractéristiques suivantes :
    \begin{enumerate}
        \item
            \( f\) est décroissante sur \( \mathopen[ 0; 2 \mathclose]\),
        \item
            \( f(0)=0\),
        \item
            \( f(-3)=6\).
        \item
            Le maximum de la fonction \( f\) sur l'intervalle \( \mathopen[ 2 , 10 \mathclose]\) est atteint pour \( x=10\).
    \end{enumerate}
    

\corrref{smath-0518}
\end{exercice}
