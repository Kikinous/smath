% This is part of Un soupçon de mathématique sans être agressif pour autant
% Copyright (c) 2015
%   Laurent Claessens
% See the file fdl-1.3.txt for copying conditions.

\begin{exercice}\label{exo2smath-0127}

    Le collège électoral italien devant élire le nouveau président de la république compte \( 1009\) électeurs. Lors du quatrième tour (le 31 janvier 2015), un dépouillement partiel des \( 660\) premiers bulletins donne \( 430\) voix en faveur de Sergio Mattarella.
    \begin{enumerate}
        \item
            Quel résultat final peut attendre Mattarella ?
        \item
            Au final il obtient \( 665\) voix et devient ainsi le nouveau président italien. A-t-il cependant obtenu la majorité des deux tiers.
    \end{enumerate}

\corrref{2smath-0127}
\end{exercice}
