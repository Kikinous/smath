% This is part of Un soupçon de mathématique sans être agressif pour autant
% Copyright (c) 2013
%   Laurent Claessens
% See the file fdl-1.3.txt for copying conditions.

\begin{definition}\cite{IYfYrIl}
    Une suite est \defe{géométrique}{géométrique!suite} quand on multiplie toujours par le même nombre pour passer d'un terme au suivant.

Une suite géométrique est donc définie par :
\begin{enumerate}
    \item
la donnée de son premier terme $u_0$
\item
une relation de récurrence de la forme : $u_{n+1} = q\times u_n$.
\end{enumerate}
Le facteur q qui permet de passer d'un terme au suivant s'appelle la raison de la suite $(u_n)$.
\end{definition}

\begin{example}
    Une colonie de fourmis arrive dans une foret dans laquelle les fourmis trouvent une nourriture abondante, mais pas de prédateurs naturels. Le nombre de fourmis va donc doubler chaque année. La première année, elles sont un million.

    La seconde année, elles seront deux millions, la troisième année, quatre millions, etc.

    Combien de fourmis avons-nous au bout de dix ans ?

\end{example}

\begin{Aretenir}
    Sens de variation. Une suite géométrique de raison \( q\) est :
    \begin{enumerate}
        \item
            croissante si \( q>1\),
        \item
            décroissante si \( q<1\),
        \item
            constante si \( q=1\).
    \end{enumerate}
\end{Aretenir}

%+++++++++++++++++++++++++++++++++++++++++++++++++++++++++++++++++++++++++++++++++++++++++++++++++++++++++++++++++++++++++++ 
\section{Exercices}
%+++++++++++++++++++++++++++++++++++++++++++++++++++++++++++++++++++++++++++++++++++++++++++++++++++++++++++++++++++++++++++

\Exo{smath-0306}
\Exo{smath-0307}
\Exo{smath-0263}
\Exo{smath-0308}
\Exo{smath-0302}
\Exo{smath-0305}
\Exo{smath-0310}
\Exo{smath-0311}

%+++++++++++++++++++++++++++++++++++++++++++++++++++++++++++++++++++++++++++++++++++++++++++++++++++++++++++++++++++++++++++ 
\section{À l'ordinateur}
%+++++++++++++++++++++++++++++++++++++++++++++++++++++++++++++++++++++++++++++++++++++++++++++++++++++++++++++++++++++++++++

\Exo{smath-0304}        % celui-ci est à faire à l'ordinateur.
