% This is part of Un soupçon de mathématique sans être agressif pour autant
% Copyright (c) 2013
%   Laurent Claessens
% See the file fdl-1.3.txt for copying conditions.

\begin{exercice}\label{exosmath-0355}

Nous considérons un dé truqué à \( 6\) faces. Nous savons que les faces \( 1\), \( 2\), \( 3\), \( 4\) et \( 5\) ont la même probabilité, mais que la face \( 6\) a une probabilité \( 1/3\).

\begin{enumerate}
    \item
        Donner la probabilité d'obtenir un nombre différent de \( 6\).       
    \item
        En déduire la probabilité d'obtenir \( 1\). Compléter le tableau
        \begin{equation*}
            \begin{array}[]{|c||c|c|c|c|c|c|c|}
                \hline
                \text{Issue}&1&2&3&4&5&6&\text{total}\\
                \hline\hline
                \text{Probabilité}&&&&&&\frac{1}{ 3 }&1\\
                \hline
            \end{array}
        \end{equation*}
    \item
        Calculer la probabilité de l'événement «obtenir un nombre pair».
\end{enumerate}


% AFAIRE : ceci provient d'un corrigé d'un DS du jeudi 31 mai 2012 de M.Lenzen. Il faut le retrouver et le citer.

\corrref{smath-0355}
\end{exercice}
