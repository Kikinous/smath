% This is part of Un soupçon de mathématique sans être agressif pour autant
% Copyright (c) 2014
%   Laurent Claessens
% See the file fdl-1.3.txt for copying conditions.

\begin{corrige}{smath-0869}

    Le premier frère a payé un quart du jeu (\( 9\) sur \( 36\)). Le second en a payé un tiers. Le troisième aura payé 
    \begin{equation}
        36-(9+12)=15
    \end{equation}
    euros, c'est à dire la fraction \( \dfrac{ 15 }{ 36 }\) du jeu.

    Le partage du temps de jeu se fera donc de la façon suivante :
    \begin{enumerate}
        \item
           Pour le premier : un quart des douze heures, c'est à dire trois heures.
       \item
           Pour le second : un tiers des douze heures, c'est à dire quatre heures.
       \item
           Le troisième prend ce qui reste : \( 12-7=5\) heures.
    \end{enumerate}
    Notons que le résultat du troisième se confirme en calculant la fraction \( \dfrac{ 15 }{ 36 }\) des douze heures :
    \begin{equation}
        \frac{ 15 }{ 36 }\times 12=\frac{ 15\times 12 }{ 36 }=\frac{ 180 }{ 36 }=5.
    \end{equation}
    Notons aussi que la fraction \( \dfrac{ 15 }{ 36 }\) se simplifie en \( \dfrac{ 5 }{ 12 }\), ce qui rend certains calculs plus simples.

\end{corrige}
