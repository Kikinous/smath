% This is part of Un soupçon de mathématique sans être agressif pour autant
% Copyright (c) 2012
%   Laurent Claessens
% See the file fdl-1.3.txt for copying conditions.

\begin{exercice}\label{exoSeconde-0046}

    \begin{enumerate}
        \item
            Pour quelle valeur de \( x\) avons-nous \( 3x+4=10\) ?
        \item
            Pour quelles valeurs de \( x\) avons-nous \( x^2-1=8\) ?
    \end{enumerate}
    Les valeurs de \( x\) telles que \( 3x+4=10\) sont les \defe{antécédents}{antécédent} de \( 10\) par la fonction \( f(x)=3x+4\). Les valeurs de \( x\) telles que \( x^2-1=8\) sont les antécédents de \( 8\) par la fonction \( f(x)=x^2-1\).

    Il peut y en avoir plusieurs.

\corrref{Seconde-0046}
\end{exercice}
