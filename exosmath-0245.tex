% This is part of Un soupçon de mathématique sans être agressif pour autant
% Copyright (c) 2013
%   Laurent Claessens
% See the file fdl-1.3.txt for copying conditions.

\begin{exercice}[\cite{EUDiHAq}]\label{exosmath-0245}

Le chef du rayon électroménager d’un grand magasin dispose du tableau ci-dessous qui donne, en milliers
d’euros, les chiffres d’affaires mensuels de son rayon pour les années $2003$, $2004$ et $2005$.


\begin{center}
\begin{tabular}[c]{|l|c|c|c|}
    \hline
Mois& Année 2003& Année 2004& Année 2005\\
\hline\hline
Janvier& 330& 380& 415\\
\hline
Février& 450& 485& 510\\
\hline
Mars& 645& 660& 700\\
\hline
Avril& 795& 810& 845\\
\hline
Mai& 975& 960& 990\\
\hline
Juin& 1125& 1125& 1150\\
\hline
Juillet& 330& 360& 390\\
\hline
Août& 285& 270& 385\\
\hline
Septembre& 420& 470& 525\\
\hline
Octobre& 540& 615& 670\\
\hline
Novembre& 1155& 1125& 1090\\
\hline
Décembre& 1740& 1610& \\
\hline\hline
Total& 8790 & & \\
 \hline\hline
Moyenne&&739&753\\
\hline
Écart type&&383&319\\
\hline
\end{tabular}
\end{center}


\begin{enumerate}
    \item
        Compléter le tableau.
            \item
 Quelle proportion le chiffre d’affaires de décembre 2005 représente-t-il par rapport au chiffre d’affaires total de l’année 2005 ?
 \item
 On suppose que cette proportion reste inchangée en 2006. On pense que le chiffre d’affaires de 2006 sera de 10 millions d’euros.  Quel sera, sous ces hypothèses, le chiffre d’affaires de décembre 2006 ?
      \item
Quel commentaire vous suggère l’évolution, sur les années 2003, 2004 et 2005, des chiffres d’affaires moyens et celle des écarts types des séries statistiques de ces trois années ?
                  
        
\end{enumerate}

\corrref{smath-0245}
\end{exercice}
