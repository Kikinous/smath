% This is part of Un soupçon de mathématique sans être agressif pour autant
% Copyright (c) 2014
%   Laurent Claessens
% See the file fdl-1.3.txt for copying conditions.

\begin{exercice}[\ldots\ldots/3]\label{exosmath-0889}

    Remplir les pointillés (\( a\) représente un nombre entier quelconque) :
    \begin{multicols}{2}
        \begin{enumerate}
            \item
                \( 4-\dfrac{ 10 }{ 5 }=\ldots\)
            \item
                \( 5\times 10+10=\ldots\).
            \item
                \( (12+88)\times 30=\ldots\)
            \item
                \( \dfrac{ 33\times 44 }{ 44 }=\ldots\)
            \item
                \( 60\times a=6\times \ldots\times a\)
        \end{enumerate}
    \end{multicols}

\corrref{smath-0889}
\end{exercice}
