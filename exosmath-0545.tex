% This is part of Un soupçon de mathématique sans être agressif pour autant
% Copyright (c) 2013
%   Laurent Claessens
% See the file fdl-1.3.txt for copying conditions.

\begin{exercice}\label{exosmath-0545}

\begin{wrapfigure}{r}{5.0cm}
   \vspace{-0.5cm}        % à adapter.
   \centering
   \input{Fig_PYYEasw.pstricks}
\end{wrapfigure}

    Une façon compliquée de savoir ses tables de multiplication. Pour trouver graphiquement le produit \( 7\times 5\), il faut dessiner le graphe de la fonction \( f\colon x\mapsto x^2\), prendre le point \( A\) qui est d'abscisse \( -7\) sur le graphe de \( f\) et le point \( B\) d'abscisse \( 5\) sur le graphe de \( f\). L'ordonnée à l'origine de la droite \( (AB)\) est le produit \( 7\times 5\). Pourquoi ?

    Est-ce que ce truc fonctionne pour calculer n'importe quel produit ? Pour information, le graphe de la fonction \( x\mapsto x^2\) est ci-contre.

\corrref{smath-0545}
\end{exercice}
