% This is part of Un soupçon de mathématique sans être agressif pour autant
% Copyright (c) 2013
%   Laurent Claessens
% See the file fdl-1.3.txt for copying conditions.

\begin{exercice}\label{exosmath-0277}

    En utilisant les propriétés de la fonction inverse, comparer les nombres suivants. Expliquer la démarche.
    \begin{enumerate}
        \item
            \( \frac{1}{ \sqrt{10}+3 }\) et \( \frac{1}{ \sqrt{10}-4 }\).
        \item
            \( \frac{1}{ x^2+3 }\) et \( \frac{1}{ x^2+4 }\).
    \end{enumerate}
    Discutez éventuellement en fonction du signe ou de la valeur de \( x\).

\corrref{smath-0277}
\end{exercice}
