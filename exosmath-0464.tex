% This is part of Un soupçon de mathématique sans être agressif pour autant
% Copyright (c) 2013
%   Laurent Claessens
% See the file fdl-1.3.txt for copying conditions.

\begin{exercice}\label{exosmath-0464}

    La SNCF veut enrouler un fil de cuivre de \unit{100}{\meter} de long autour d'une grande bobine de \unit{1}{\meter} de diamètre. Combien de tours seront nécessaires ?

    À la moitié du deuxième tour, nous mettons une marque sur le fil. À quelle distance du début du fil se trouve la marque ?

\corrref{smath-0464}
\end{exercice}
