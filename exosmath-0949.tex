% This is part of Un soupçon de mathématique sans être agressif pour autant
% Copyright (c) 2014
%   Laurent Claessens
% See the file fdl-1.3.txt for copying conditions.

\begin{exercice}[\cite{NRHooXFvgpp4}]\label{exosmath-0949}

Ce tableau indique la taille de Raphaël en fonction de son âge.
\begin{equation*}
    \begin{array}[]{|c|c|c|c|c|}
        \hline
        \text{Âge (en années)}&2&5&10&12\\
        \hline
        \text{Taille (en \si{\centi\meter})}&80&100&125&150\\
        \hline
    \end{array}
\end{equation*}
 \begin{enumerate}
     \item
Est-ce une situation de proportionnalité ?
\item
 Représenter graphiquement l'évolution de la taille de Raphaël en fonction de son âge. Peux-t-on répondre à la première question sans faire de calculs ?  Justifier.
 \end{enumerate}

\corrref{smath-0949}
\end{exercice}
