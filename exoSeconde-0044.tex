% This is part of Un soupçon de mathématique sans être agressif pour autant
% Copyright (c) 2012,2014
%   Laurent Claessens
% See the file fdl-1.3.txt for copying conditions.

\begin{exercice}\label{exoSeconde-0044}

    Considérons la situation de la figure \ref{LabelFigSurfaceTriangletcNPPE}. Le point $M$ est mobile sur le segment \( AB\) et nous avons \( \| BC \|=\unit{2}{\centi\meter}\), \( \| AB \|=\unit{3}{\centi\meter}\).

    Si nous notons par \( x\) la distance entre \( A\) et \( M\), nous voulons calculer l'aire du triangle \( AMQ\). Pour cela, commencer par écrire le théorème de Thalès pour les droites \( MQ\) et \( BC\) qui intersectent l'angle \( A\). Exprimer alors la hauteur du triangle en fonction de \( x\). Conclure.

    \newcommand{\CaptionFigSurfaceTriangletcNPPE}{La figure de l'exercice \ref{exoSeconde-0044}.}
\input{Fig_SurfaceTriangletcNPPE.pstricks}

\corrref{Seconde-0044}
\end{exercice}
