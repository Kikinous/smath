% This is part of Un soupçon de mathématique sans être agressif pour autant
% Copyright (c) 2014
%   Laurent Claessens
% See the file fdl-1.3.txt for copying conditions.

\begin{exercice}\label{exosmath-0824}

    Un charpentier doit couper des poutres de bonne longueur pour créer un triangle isocèle. La poutre transversale horizontale fait \unit{8}{\meter} et l'inclinaison du toit est de \unit{40}{\degree}. 
    \begin{enumerate}
        \item
            Dessiner un schéma à l'échelle (par exemple \unit{1}{\centi\meter} sur la feuille représente \unit{1}{\meter} dans la réalité). Préciser l'échelle choisie.
        \item
            En déduire la longueur des poutres inclinées.
    \end{enumerate}
    
\corrref{smath-0824}
\end{exercice}
