% This is part of Un soupçon de mathématique sans être agressif pour autant
% Copyright (c) 2014
%   Laurent Claessens
% See the file fdl-1.3.txt for copying conditions.

\begin{exercice}\label{exosmath-0998}

    Dans chacune des situations suivantes, faire un dessin et calculer en justifiant les longueurs demandées.
    \begin{enumerate}
        \item
            Dans le triangle \( ABC\), \( J\) est le milieu de \( [BC]\), le point \( I\) est sur le côté \( [AC]\) et la droite \( (IJ)\) est parallèle à \( (AB)\). De plus \( AB=\SI{5}{\centi\meter}\) et \( IC=\SI{4}{\centi\meter}\). Calculer les longueurs de \( [IJ]\) et \( [AI]\).
        \item
            Dans le triangle \( KLM\), les points \( I\) et \( J\) sont les milieux de \( [KM]\) et \( [ML]\). La longueur \( IJ\) est de \( 6\) centimètres. Calculer la longueur de \( [KL]\).
        \item
RNTRooAXhubs
    \end{enumerate}
    <++>

\corrref{smath-0998}
\end{exercice}
