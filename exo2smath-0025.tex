% This is part of Un soupçon de mathématique sans être agressif pour autant
% Copyright (c) 2014
%   Laurent Claessens
% See the file fdl-1.3.txt for copying conditions.

\begin{exercice}[\cite{NRHooXFvgpp4}]\label{exo2smath-0025}

    Quatre amis font un voyage en trois jours. Le premier jour, ils parcourent $40$\% du trajet total ; le deuxième jour, un quart et le dernier jour, \( \dfrac{ 7 }{ 20 }\) du trajet total. 

    \begin{enumerate}
        \item
 Quel jour ont-ils parcouru la plus grande distance ?
\item
Peut-on calculer la distance parcourue chaque jour ?
    \end{enumerate}

\corrref{2smath-0025}
\end{exercice}
