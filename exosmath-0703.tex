% This is part of Un soupçon de mathématique sans être agressif pour autant
% Copyright (c) 2014
%   Laurent Claessens
% See the file fdl-1.3.txt for copying conditions.

\begin{exercice}[\ldots/3]\label{exosmath-0703}

    Le programme suivant demande deux entrées à l'utilisateur et vérifie si elles sont identiques (par exemple pour s'assurer qu'il n'y ait pas de fautes de frappes lors de la création d'un mot de passe) :

    \begin{fmpage}{0.9\linewidth}

        demander \( a\)

        demander \( b\)

        Si \( a=b\) :

        ~~ Écrire «Correct» 

        Sinon :

        ~~ «Saisie incorrecte» 

    \end{fmpage}

    En utilisant une boucle «tant que», modifier ce programme pour que l'ordinateur redemande les deux saisies mot de passe jusqu'à ce qu'elles soient identiques. À chaque saisie incorrecte, l'ordinateur devra écrite «incorrect, recommencez», et lorsque la saisie sera bonne, il devra écrire «Correct».

\corrref{smath-0703}
\end{exercice}
