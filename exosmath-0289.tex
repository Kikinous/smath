% This is part of Un soupçon de mathématique sans être agressif pour autant
% Copyright (c) 2012-2013
%   Laurent Claessens
% See the file fdl-1.3.txt for copying conditions.

\begin{exercice}\label{exosmath-0289}

    Chacune des questions possède une, deux ou trois bonnes réponses. Lesquelles ?
    \begin{multicols}{2}
    \begin{enumerate}

        \item
        Dans une classe de \( 30\) élèves, \( 21\) ont \( 17\) ans, \( 20\%\) ont \( 18\) ans et le reste a \( 16\) ans.  Alors
        \begin{enumerate}
            \item

         \( 3\%\) des élèves ont \( 16\) ans
     \item
         \( 75\%\) des élèves ont \( 17\) ans.
     \item
         \( 3\) élèves ont \( 16\) ans
                
        \end{enumerate}

    \item
        L'ancien prix d'un objet était \( Q\). Le nouveau prix est \( 0.75\times Q\). Qu'a fait le prix ?
        \begin{enumerate}
            \item
                Augmenté de \( 75\%\).
            \item
                Diminué de \( 75\%\).
            \item
                Diminué de \( 25\%\).
            \item
                Augmenté de \( 25\%\).
        \end{enumerate}

    \item
        Le prix d'une matière première a baissé de \( 5\%\) et vaut maintenant \( 1000\) euros la tonne. Quel était (en arrondi) le prix original ?
        \begin{enumerate}
            \item
                \( 1050\) euros.
            \item
                \( 1053\) euros.
            \item
                \( 950\) euros.
            \item
                \( 1047\) euros.
        \end{enumerate}

    \end{enumerate}
    \end{multicols}
       
\corrref{smath-0289}
\end{exercice}
