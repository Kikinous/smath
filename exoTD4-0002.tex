% This is part of Exercices de mathématique pour SVT
% Copyright (c) 2010
%   Laurent Claessens et Carlotta Donadello
% See the file fdl-1.3.txt for copying conditions.

\begin{exercice}\label{exoTD4-0002}

	Les fonctions trigonométriques réciproques.

	\begin{enumerate}
		\item
			Montrer que la fonction sinus est une bijection croissante de $\mathopen[ -\frac{ \pi }{2} , \frac{ \pi }{2} \mathclose]$ dans $\mathopen[ -1 , 1 \mathclose]$.
		\item
			On note $\arcsin$ (se prononce «arc sinus») sa bijection réciproque : $\arcsin\colon \mathopen[ -1 , 1 \mathclose]\to \mathopen[ -\frac{ \pi }{2} , \frac{ \pi }{2} \mathclose]$. Montrer que $\arcsin$ est dérivable sur $\mathopen] -1 , 1 \mathclose[$ avec
			\begin{equation}
				\big( \arcsin(x) \big)'=\frac{1}{ \sqrt{1-x^2} }.
			\end{equation}
			Conseil : utiliser la formule $\cos^2(t)+\sin^2(t)=1$ pour tout $t\in\eR$.
		\item
			Montrer que la fonction cosinus est une bijection décroissante de $\mathopen[ 0 , \pi \mathclose]$ dans $\mathopen[ -1 , 1 \mathclose]$. On note $\arccos$ (prononcer «arc cosinus») sa bijection réciproque
			\begin{equation}
				\arccos\colon \mathopen[ -1 , 1 \mathclose]\to \mathopen[ 0 , \pi \mathclose].
			\end{equation}
			Montrer que $\arccos$ est dérivable sur $\mathopen] -1 , 1 \mathclose[$ avec
			\begin{equation}
				\big( \arccos(x) \big)'=-\frac{ 1 }{ \sqrt{1-x^2} }.
			\end{equation}
		\item
			Montrer que la fonction $\tan$ est une bijection croissante de $\mathopen] -\pi/2 , \pi/2 \mathclose[$ dans $\eR$. On note $\arctan$ (prononcer «arc tangente») sa bijection réciproque. Montrer que $\arctan$ est dérivable sur $\eR$ et que
			\begin{equation}
				\big( \arctan(x) \big)'=\frac{1}{ 1+x^2 }.
			\end{equation}
	\end{enumerate}

\corrref{TD4-0002}
\end{exercice}
