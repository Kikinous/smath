% This is part of Un soupçon de mathématique sans être agressif pour autant
% Copyright (c) 2013
%   Laurent Claessens
% See the file fdl-1.3.txt for copying conditions.

\begin{exercice}\label{exosmath-0569}

\begin{wrapfigure}{r}{7.0cm}
   \vspace{-2.5cm}        % à adapter.
   \centering
   \input{Fig_JLUTBlD.pstricks}
\end{wrapfigure}

    Soient les fonctions affines \( f(x)=3x-2\) et \( g(x)=-x+3\). 
    \begin{enumerate}
        \item
            Le graphe ci-contre représente une de ces deux fonctions; laquelle ?
        \item
            Tracer l'autre sur le même graphe.
        \item
            Pour quelle valeur de \( x\) a-t-on \( f(x)=g(x)\) ? 
    \end{enumerate}

\corrref{smath-0569}
\end{exercice}
