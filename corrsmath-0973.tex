% This is part of Un soupçon de mathématique sans être agressif pour autant
% Copyright (c) 2014
%   Laurent Claessens
% See the file fdl-1.3.txt for copying conditions.

\begin{corrige}{smath-0973}

    \begin{enumerate}
        \item
            Ils ont coupé le gâteau en six, donc chacune des parts vaut un sixième du gâteau.
            Louis en aura mangé deux parts, soit deux sixièmes : \( \dfrac{ 2 }{ 6 }\) du gâteau; Capucine a pris une part, c'est à dire un sixième : \( \dfrac{ 1 }{ 6 }\).
        \item
            Ensemble ils auront mangé \( \dfrac{ 3 }{ 6 }\), c'est à dire la moitié du gâteau.
        \item
            Si le gâteau pesait \SI{300}{\gram}, il reste \SI{150}{\gram} parce que la moitié a été mangée.
    \end{enumerate}

\end{corrige}
