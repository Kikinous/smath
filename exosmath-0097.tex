% This is part of Un soupçon de mathématique sans être agressif pour autant
% Copyright (c) 2012
%   Laurent Claessens
% See the file fdl-1.3.txt for copying conditions.

\begin{exercice}\label{exosmath-0097}

    Dans une classe d'EPS de \( 25\) élèves on demande qui aime le lancer du poids et qui aime la course. Le résultat est que \( 17\) élèves aiment le lancer du poids, et \( 8\) élèves aiment la course mais pas le lancer du poids. En tout, \( 10\) élèves aiment la course.

    Attention : parmi les \( 17\) qui aiment le lancer du poids, certains aiment aussi la course.

    Remplir le tableau d'effectifs suivant.

    \begin{center}
    \begin{tabular}[]{|c||c|c||c|}
        \hline
        &aime le lancer&n'aime pas le lancer&total\\
        \hline\hline
        aime la course&&8&10\\
        \hline
        n'aime pas la course&&&\\
        \hline
        total&17&&25\\
        \hline
    \end{tabular}
    \end{center}

\corrref{smath-0097}
\end{exercice}
