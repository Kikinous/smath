% This is part of Un soupçon de mathématique sans être agressif pour autant
% Copyright (c) 2014
%   Laurent Claessens
% See the file fdl-1.3.txt for copying conditions.

\begin{exercice}\label{exosmath-0654}

%\begin{wrapfigure}{r}{4.0cm}
%   \vspace{-0.5cm}        % à adapter.
%   \centering
%   \input{Fig_OLuvnaY.pstricks}
%\end{wrapfigure}


    Nous considérons un carré \( ABCD\) de côté \unit{5}{\centi\meter}. Le point \( M\) est sur \( [BC]\) et \( N\) sur \( [CD]\) et sont tels que \( BM=DN\). Le point \( P\) est placé de telle façon à ce que \( MCNP\) soit un carré. Les points \( M\) et \( N\) sont mobiles et nous notons \( x\) la distance \( BM\).

    \begin{center}
   \input{Fig_OLuvnaY.pstricks}
    \end{center}

    \begin{enumerate}
        \item
            Exprimer en fonction de \( x\) les aires du carré \( MCNP\) et des triangles \( ABM\) et \( ADN\).
        \item
            En déduire que l'aire de la zone grisée est donnée par \( A(x)=-x^2+5x\) pour \( x\in\mathopen[ 0 ; 5 \mathclose]\).
        \item
            Montrer que \( A(x)=-(x-2.5)^2+6.25\).
        \item
            Le but maintenant est de déterminer où placer \( M\) de telle sorte que l'aire de la zone grisée soit supérieure ou égale à \unit{5.25}{\centi\meter\squared}
            \begin{enumerate}
                \item
                    Montrer que pour tout réel \( x\), \\ \( A(x)-5.25=(x-3.5)(1.5-x)\).
                \item
                    Dresser le tableau de signes de\\ \( f(x)=(x-3.5)(1.5-x)\)
                \item
                    En déduire l'ensemble des solutions de l'inéquation \( A(x)\geq 5.25\).
                \item
                    Conclure.
            \end{enumerate}
    \end{enumerate}

\corrref{smath-0654}
\end{exercice}
