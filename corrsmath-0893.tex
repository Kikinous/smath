% This is part of Un soupçon de mathématique sans être agressif pour autant
% Copyright (c) 2014
%   Laurent Claessens
% See the file fdl-1.3.txt for copying conditions.

\begin{corrige}{smath-0893}

    Le gâteau contient en tout \( 300+150+100+100\) grammes d'ingrédients, dont \( 100\) de chocolat. La fraction de chocolat dans le gâteau est donc de
    \begin{equation}
        \frac{ 100 }{ 300+150+100+100 }=\frac{ 100 }{ 650 }.
    \end{equation}
    Cette fraction peut être simplifiée. D'abord par \( 10\) :
    \begin{equation}
        \frac{ 100 }{ 650 }=\frac{ 100\div 10 }{ 650\div 10 }=\frac{ 10 }{ 65 },
    \end{equation}
    et ensuite par \( 5\):
    \begin{equation}
        \frac{ 10 }{ 65 }=\frac{ 10\div 5 }{ 65\div 5 }=\frac{ 2 }{ 13 }.
    \end{equation}
    Au final la fraction de chocolat dans le gâteau est de \( \dfrac{ 2 }{ 13 }\).

    C'est à dire que sur \( 13\) grammes de gâteau, deux sont du chocolat.

\end{corrige}
