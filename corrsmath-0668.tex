% This is part of Un soupçon de mathématique sans être agressif pour autant
% Copyright (c) 2014
%   Laurent Claessens
% See the file fdl-1.3.txt for copying conditions.

\begin{corrige}{smath-0668}

\begin{wrapfigure}{r}{12.cm}
   \vspace{-0.5cm}        % à adapter.
   \centering
   \input{Fig_OZXooCVkgQ.pstricks}
\end{wrapfigure}


    Les effectifs des classes sont comme suit :
    \begin{equation*}
        \begin{array}[]{|c||c|}
            \hline
            \text{Classe}&\text{Effectif}\\
            \hline\hline
            \mathopen[ 0 ; 5 [&2\\
            \hline
            \mathopen[ 5 ; 8 [&7\\
            \hline
            \mathopen[ 8; 10 [&10\\
            \hline
            \mathopen[ 10;12 [&12\\
            \hline
            \mathopen[ 12 ; 15 [&1\\
            \hline
            \mathopen[ 15 ; 20 \mathclose]&1\\
            \hline
        \end{array}
    \end{equation*}
    Attention : la notation \( \mathopen[ 0; 5 [\) dit que la classe contient toutes les notes de \( 0\) à \( 5\), le \( 5\) n'étant pas compris.

        En ce qui concerne l'histogramme, la première classe a une largeur de \( 5\) et un effectif de \( 2\), donc la hauteur devra être \( \frac{ 2 }{ 5 }\). La classe \( \mathopen[ 8; 10 [\) a une largeur de \( 2\) et un effectif de \( 10\), donc une hauteur de \( \frac{ 10 }{ 2 }=5\).

            Il est particulièrement visible que les deux dernières classes, bien que toutes deux d'effectifs \( 1\) n'ont pas la même hauteur.


\end{corrige}
