% This is part of Un soupçon de mathématique sans être agressif pour autant
% Copyright (c) 2014
%   Laurent Claessens
% See the file fdl-1.3.txt for copying conditions.

\begin{center}
    \small
Ce chapitre recense les exercices des feuilles distribuées en classe, avec certaines modifications, ajouts et suppressions. Ce sont donc plutôt les exercices qui seront sur les feuilles l'année prochaine.
\end{center}

%+++++++++++++++++++++++++++++++++++++++++++++++++++++++++++++++++++++++++++++++++++++++++++++++++++++++++++++++++++++++++++ 
\section{Règles de calcul}
%+++++++++++++++++++++++++++++++++++++++++++++++++++++++++++++++++++++++++++++++++++++++++++++++++++++++++++++++++++++++++++

\Exo{smath-0728}
\Exo{smath-0736}
\Exo{smath-0729}
\Exo{smath-0732}
\Exo{smath-0731}
\Exo{smath-0730}
\Exo{smath-0733}
\Exo{smath-0734}

\Exo{smath-0735}
\Exo{smath-0737}
\Exo{smath-0738}

\Exo{smath-0740}
\Exo{smath-0739}
\Exo{smath-0741}
\Exo{smath-0742}

\Exo{smath-0758}
\Exo{smath-0762}
\Exo{smath-0761}


%+++++++++++++++++++++++++++++++++++++++++++++++++++++++++++++++++++++++++++++++++++++++++++++++++++++++++++++++++++++++++++ 
\section{Triangles}
%+++++++++++++++++++++++++++++++++++++++++++++++++++++++++++++++++++++++++++++++++++++++++++++++++++++++++++++++++++++++++++

\Exo{smath-0768}
\Exo{smath-0769}
\Exo{smath-0770}
\Exo{smath-0773}
\Exo{smath-0771}
\Exo{smath-0774}
\Exo{smath-0775}
\Exo{smath-0776}
\Exo{smath-0777}
\Exo{smath-0779}

%+++++++++++++++++++++++++++++++++++++++++++++++++++++++++++++++++++++++++++++++++++++++++++++++++++++++++++++++++++++++++++ 
\section{Méthodologie mathématique}
%+++++++++++++++++++++++++++++++++++++++++++++++++++++++++++++++++++++++++++++++++++++++++++++++++++++++++++++++++++++++++++

\Exo{smath-0787}
\Exo{smath-0785}
\Exo{smath-0791}
\Exo{smath-0782}

\Exo{smath-0783}
\Exo{smath-0811}
\Exo{smath-0789}
\Exo{smath-0790}
\Exo{smath-0812}

%+++++++++++++++++++++++++++++++++++++++++++++++++++++++++++++++++++++++++++++++++++++++++++++++++++++++++++++++++++++++++++ 
\section{Écriture fractionnaire}
%+++++++++++++++++++++++++++++++++++++++++++++++++++++++++++++++++++++++++++++++++++++++++++++++++++++++++++++++++++++++++++

% Il faut des exercices de calcul direct genre 10/5, 100/25, 4-10/5 ou 3/4
% et de simplifications directes, qui se font en un seul coup et d'une seule façon : 21/14, 15/3, 2/4, etc.

\Exo{smath-0874}
\Exo{smath-0793}
\Exo{smath-0795}
\Exo{smath-0792}
\Exo{smath-0798}
\Exo{smath-0799}
\Exo{smath-0800}
\Exo{smath-0803}
\Exo{smath-0802}    % Celui-ci est en activité; il faut changer (activ_melange.tex)
\Exo{smath-0796}
\Exo{smath-0908}
\Exo{smath-0818}
\Exo{smath-0794}
\Exo{smath-0817}
\Exo{smath-0819}

%+++++++++++++++++++++++++++++++++++++++++++++++++++++++++++++++++++++++++++++++++++++++++++++++++++++++++++++++++++++++++++ 
\section{Opérations sur les fractions}
%+++++++++++++++++++++++++++++++++++++++++++++++++++++++++++++++++++++++++++++++++++++++++++++++++++++++++++++++++++++++++++

\Exo{smath-0870}
\Exo{smath-0871}
\Exo{smath-0844}
\Exo{smath-0838}
\Exo{smath-0867}
\Exo{smath-0866}
\Exo{smath-0872}
\Exo{smath-0869}
\Exo{smath-0875}
\Exo{smath-0876}
\Exo{smath-0877}
\Exo{smath-0878}
\Exo{smath-0879}
\Exo{smath-0881}
\Exo{smath-0884}

%+++++++++++++++++++++++++++++++++++++++++++++++++++++++++++++++++++++++++++++++++++++++++++++++++++++++++++++++++++++++++++ 
\section{Droites remarquables dans un triangle}
%+++++++++++++++++++++++++++++++++++++++++++++++++++++++++++++++++++++++++++++++++++++++++++++++++++++++++++++++++++++++++++3

%--------------------------------------------------------------------------------------------------------------------------- 
%\subsection{Médiatrice}
%---------------------------------------------------------------------------------------------------------------------------
\Exo{smath-0960}
\Exo{smath-0917}
\Exo{smath-0925}
%--------------------------------------------------------------------------------------------------------------------------- 
%\subsection{Médiane}
%---------------------------------------------------------------------------------------------------------------------------
\Exo{smath-0929}
\Exo{smath-0959}
%--------------------------------------------------------------------------------------------------------------------------- 
%\subsection{Hauteur}
%---------------------------------------------------------------------------------------------------------------------------
\Exo{smath-0930}
\Exo{smath-0928}
\Exo{smath-0927}
%--------------------------------------------------------------------------------------------------------------------------- 
%\subsection{Tous les trois}
%---------------------------------------------------------------------------------------------------------------------------
\Exo{smath-0932}
\Exo{smath-0918}
\Exo{smath-0922}
\Exo{smath-0920}
%--------------------------------------------------------------------------------------------------------------------------- 
%\subsection{Plus}
%---------------------------------------------------------------------------------------------------------------------------
\Exo{smath-0919}
\Exo{smath-0926}
\Exo{smath-0931}


%+++++++++++++++++++++++++++++++++++++++++++++++++++++++++++++++++++++++++++++++++++++++++++++++++++++++++++++++++++++++++++ 
\section{Expressions littérales}
%+++++++++++++++++++++++++++++++++++++++++++++++++++++++++++++++++++++++++++++++++++++++++++++++++++++++++++++++++++++++++++

% Pas de résolution d'équations dans ce chapitre
\Exo{smath-0935}
\Exo{smath-0979}
\Exo{smath-0967}
\Exo{smath-0968}
\Exo{smath-0934}
\Exo{smath-0938}
\Exo{smath-0939}

\Exo{smath-0984}
\Exo{smath-0983}
\Exo{smath-0980}

\Exo{smath-0981}

% Plus
\Exo{smath-0936}
\Exo{smath-0940}

%+++++++++++++++++++++++++++++++++++++++++++++++++++++++++++++++++++++++++++++++++++++++++++++++++++++++++++++++++++++++++++ 
\section{Symétrie centrale}
%+++++++++++++++++++++++++++++++++++++++++++++++++++++++++++++++++++++++++++++++++++++++++++++++++++++++++++++++++++++++++++

\Exo{smath-0988}
\Exo{smath-0987}
\Exo{smath-0991}
\Exo{smath-0989}
\Exo{smath-0990}
\Exo{smath-0992}
\Exo{2smath-0076}

%+++++++++++++++++++++++++++++++++++++++++++++++++++++++++++++++++++++++++++++++++++++++++++++++++++++++++++++++++++++++++++ 
\section{Nombres relatifs}
%+++++++++++++++++++++++++++++++++++++++++++++++++++++++++++++++++++++++++++++++++++++++++++++++++++++++++++++++++++++++++++

\Exo{2smath-0043}
\Exo{2smath-0045}
\Exo{2smath-0044}
\Exo{2smath-0046}
\Exo{2smath-0049}
\Exo{2smath-0059}
\Exo{2smath-0053}
\Exo{2smath-0054}
\Exo{2smath-0055}
\Exo{2smath-0047}
\Exo{2smath-0048}
\Exo{2smath-0050}
\Exo{2smath-0051}
\Exo{2smath-0052}

%+++++++++++++++++++++++++++++++++++++++++++++++++++++++++++++++++++++++++++++++++++++++++++++++++++++++++++++++++++++++++++ 
\section{Angles et parallélisme}
%+++++++++++++++++++++++++++++++++++++++++++++++++++++++++++++++++++++++++++++++++++++++++++++++++++++++++++++++++++++++++++

\Exo{2smath-0062}    % il faut plus d'exercices comme ceux-ci
\Exo{2smath-0067}
\Exo{2smath-0098}
\Exo{2smath-0068}
\Exo{2smath-0069}
\Exo{2smath-0070}
\Exo{2smath-0071}
