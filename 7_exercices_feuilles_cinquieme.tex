% This is part of Un soupçon de mathématique sans être agressif pour autant
% Copyright (c) 2014
%   Laurent Claessens
% See the file fdl-1.3.txt for copying conditions.

\begin{center}
    \small
Ce chapitre recense les exercices des feuilles distribuées en classe, avec certaines modifications, ajouts et suppressions. Ce sont donc plutôt les exercices qui seront sur les feuilles l'année prochaine.
\end{center}

%+++++++++++++++++++++++++++++++++++++++++++++++++++++++++++++++++++++++++++++++++++++++++++++++++++++++++++++++++++++++++++ 
\section{Règles de calcul}
%+++++++++++++++++++++++++++++++++++++++++++++++++++++++++++++++++++++++++++++++++++++++++++++++++++++++++++++++++++++++++++

\Exo{smath-0728}
\Exo{smath-0736}
\Exo{smath-0729}
\Exo{smath-0732}
\Exo{smath-0731}
\Exo{smath-0730}
\Exo{smath-0733}
\Exo{smath-0734}

\Exo{smath-0735}
\Exo{smath-0737}
\Exo{smath-0738}

\Exo{smath-0740}
\Exo{smath-0739}
\Exo{smath-0741}
\Exo{smath-0742}

\Exo{smath-0758}
\Exo{smath-0762}
\Exo{smath-0761}


%+++++++++++++++++++++++++++++++++++++++++++++++++++++++++++++++++++++++++++++++++++++++++++++++++++++++++++++++++++++++++++ 
\section{Triangles}
%+++++++++++++++++++++++++++++++++++++++++++++++++++++++++++++++++++++++++++++++++++++++++++++++++++++++++++++++++++++++++++

\Exo{smath-0768}
\Exo{smath-0769}
\Exo{smath-0770}
\Exo{smath-0773}
\Exo{smath-0771}
\Exo{smath-0774}
\Exo{smath-0775}
\Exo{smath-0776}
\Exo{smath-0777}
\Exo{smath-0779}

%+++++++++++++++++++++++++++++++++++++++++++++++++++++++++++++++++++++++++++++++++++++++++++++++++++++++++++++++++++++++++++ 
\section{Méthodologie mathématique}
%+++++++++++++++++++++++++++++++++++++++++++++++++++++++++++++++++++++++++++++++++++++++++++++++++++++++++++++++++++++++++++

\Exo{smath-0787}
\Exo{smath-0785}
\Exo{smath-0791}
\Exo{smath-0782}

\Exo{smath-0783}
\Exo{smath-0811}
\Exo{smath-0789}
\Exo{smath-0790}
\Exo{smath-0812}

%+++++++++++++++++++++++++++++++++++++++++++++++++++++++++++++++++++++++++++++++++++++++++++++++++++++++++++++++++++++++++++ 
\section{Écriture fractionnaire}
%+++++++++++++++++++++++++++++++++++++++++++++++++++++++++++++++++++++++++++++++++++++++++++++++++++++++++++++++++++++++++++

% Il faut des exercices de calcul direct genre 10/5, 100/25, 4-10/5 ou 3/4
% et de simplifications directes, qui se font en un seul coup et d'une seule façon : 21/14, 15/3, 2/4, etc.

\Exo{smath-0874}
\Exo{smath-0793}
\Exo{smath-0795}
\Exo{smath-0792}
\Exo{smath-0798}
\Exo{smath-0799}
\Exo{smath-0800}
\Exo{smath-0803}
\Exo{smath-0802}    % Celui-ci est en activité; il faut changer (activ_melange.tex)
\Exo{smath-0796}
\Exo{smath-0908}
\Exo{smath-0818}
\Exo{smath-0794}
\Exo{smath-0817}
\Exo{smath-0819}

%+++++++++++++++++++++++++++++++++++++++++++++++++++++++++++++++++++++++++++++++++++++++++++++++++++++++++++++++++++++++++++ 
%\section{Opérations sur les fractions}
%+++++++++++++++++++++++++++++++++++++++++++++++++++++++++++++++++++++++++++++++++++++++++++++++++++++++++++++++++++++++++++

% Ceci est ici pour la version 'simple' pour les 5A.
% C'est à enlever après compilation et envoi. Surtout que les 5B ne l'ont pas encore fait et que des modifications sont à faire.

%\Exo{smath-0872}
%\Exo{smath-0844}
%\Exo{smath-0870}
%\Exo{smath-0871}
%\Exo{smath-0838}
%\Exo{smath-0867}
%\Exo{smath-0866}
%\Exo{smath-0869}
%\Exo{smath-0875}
%\Exo{smath-0876}
%\Exo{smath-0877}
%\Exo{smath-0878}
%\Exo{smath-0879}
%\Exo{smath-0881}
%\Exo{smath-0884}
