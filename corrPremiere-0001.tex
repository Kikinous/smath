% This is part of Un soupçon de mathématique sans être agressif pour autant
% Copyright (c) 2012
%   Laurent Claessens
% See the file fdl-1.3.txt for copying conditions.

\begin{corrige}{Premiere-0001}

    La population totale est \( E=7\times 10^9\), et la sous-population des «ayant un comte» est de \( 900\times 10^6\). La proportion est donc de
    \begin{equation}
        \frac{ 900\times 10^6 }{ 7\times 10^9 }=\frac{ 9 }{ 70 }\simeq 0.12.
    \end{equation}
    Facebook concerne donc \( 12\%\) de la population mondiale.

    Si \( a\) est la proportion des lettrés dans une population de \( 7\times 10^9\) humains, la sous-population des lettrés est de \( 7a\times 10^9\) personnes. La proportion demandé est donc
    \begin{equation}
        \frac{ 900\times 10^6 }{ 7a\times 10^9 }\simeq \frac{ 0.12 }{ a }.
    \end{equation}

\end{corrige}
