% This is part of Un soupçon de mathématique sans être agressif pour autant
% Copyright (c) 2012
%   Laurent Claessens
% See the file fdl-1.3.txt for copying conditions.

\begin{exercice}\label{exosmath-0155}

    Compléter le tableau suivant. Il y a souvent beaucoup de possibilités. Chaque ligne concerne une droite : les points \( A\) et \( B\) sont sur la droite et \( \vect{ v }\) est un vecteur directeur de la droite.

    \begin{equation*}
        \begin{array}[]{|c|c|c|c|}
            \hline
            A&B&\text{Équation}&\text{Vecteur directeur}\\
            \hline\hline
            (1,2)&(4,4)&&\\
            \hline
            (-1,0)&&&\begin{pmatrix}
                1    \\ 
                3    
            \end{pmatrix}\\
            \hline
            &&y=-x+3&\\
            \hline
            (0,0)&&&\begin{pmatrix}
                -1    \\ 
                1    
            \end{pmatrix}\\
            \hline
            &(7,128)&&\begin{pmatrix}
                -1    \\ 
                -1    
            \end{pmatrix}\\
            \hline
            (-1;-1)&(0,0)&&\\
            \hline
            &&y=-x&\\
            \hline
            (1,1)&&&\begin{pmatrix}
                1    \\ 
                4    
            \end{pmatrix}\\
            \hline
            &&y=x&\\
            \hline
            (-1,7)&(7,1)&&\\
            \hline
            (1,1)&&&\begin{pmatrix}
                3    \\ 
                0    
            \end{pmatrix}\\
            \hline
        \end{array}
    \end{equation*}

\corrref{smath-0155}
\end{exercice}
