% This is part of Un soupçon de mathématique sans être agressif pour autant
% Copyright (c) 2013
%   Laurent Claessens
% See the file fdl-1.3.txt for copying conditions.

\begin{exercice}\label{exosmath-0533}

    La gendarmerie a effectué un contrôle de vitesse sur une route nationale à deux voies. La limitation est de \unit{90}{\kilo\meter\per\hour}. Voici les résultats groupés en classes :
    \begin{equation*}
        \begin{array}[]{|c||c|c|c|c|c|}
            \hline
            \text{Vitesse (en \kilo\meter\per\hour)}&\mathopen] 60 , 70 \mathclose]&\mathopen] 70 , 80 \mathclose]&\mathopen] 90 , 90 \mathclose]&\mathopen] 90 , 100 \mathclose]&\mathopen] 100 , 120 \mathclose]\\
              \hline\hline
              \text{Nombre de véhicules}&1&6&20&19&4\\ 
              \hline 
               \end{array}
    \end{equation*}
    \begin{enumerate}
        \item
            Dans quelle classe va une voiture roulant à exactement \unit{90}{\kilo\meter\per\hour} ?
        \item
            Quel est le pourcentage de véhicule roulant trop vite ?
        \item
            Dessiner un histogramme représentant la situation.
    \end{enumerate}

\corrref{smath-0533}
\end{exercice}
