% This is part of Un soupçon de mathématique sans être agressif pour autant
% Copyright (c) 2012-2013
%   Laurent Claessens
% See the file fdl-1.3.txt for copying conditions.


%+++++++++++++++++++++++++++++++++++++++++++++++++++++++++++++++++++++++++++++++++++++++++++++++++++++++++++++++++++++++++++ 
\section{Diagramme en boites}
%+++++++++++++++++++++++++++++++++++++++++++++++++++++++++++++++++++++++++++++++++++++++++++++++++++++++++++++++++++++++++++

Nous avons déjà vu les notions de médianes et quartiles plus tôt dans l'année.

\begin{example}
    Une association de consommateurs étudie le temps de vie (en années) d'un millier de machines à laver vendues. Les résultats bruts sont :
    \begin{equation*}
        \begin{array}[]{|c||c|c|c|c|c|c|c|c|c|c|c|c|c|c|c|}
            \hline
            \text{temps de vie}&1&2&3&4&5&6&7&8&9&10&11&12&13&14&15\\
            \hline\hline
            \text{Effectifs}&30&50&120&200&250&80&100&50&10&0&30&25&20&10&25\\
            \hline
        \end{array}
    \end{equation*}
    Les valeurs statistiques calculées sont :
    \begin{multicols}{2}
        \begin{itemize}
            \item
                Premier quartile : $4$.
            \item
                Médiane : \( 5\).
            \item
                Troisième quartile : \( 7\).
            \item
                Valeur minimale : \( 1\).
            \item 
                Valeur maximale : \( 15\).
        \end{itemize}
    \end{multicols}

    Le diagramme en boites est

%The result is on figure \ref{LabelFigfigureSNpNWPt}. % From file figureSNpNWPt
%\newcommand{\CaptionFigfigureSNpNWPt}{<+Type your caption here+>}
\begin{center}
\input{Fig_figureSNpNWPt.pstricks}
\end{center}

    Quelques informations lisibles sur ce graphiques.
    \begin{enumerate}
        \item
            Lorsqu'on achète une machine à laver, il y a de bonnes chances qu'elle arrive à tenir \( 4\) ans : \( 75\%\) des machines sont encore là.
        \item
            Entre \( 4\) et \( 7\) ans, c'est hécatombe. La moitié des \( 1000\) des machines tombe en pane durant ces trois ans. Il y a donc deux fois plus de machines qui tombent en panne durant ces 3 ans que durant les trois premières années.
        \item
            Si votre machine passe le cap des \( 7\) ans, elle peut aller loin.
    \end{enumerate}

    \begin{remark}
        \begin{enumerate}
            \item
                Le diagramme en boite, comme toute représentation graphique, ne donne pas toute l'information. Il est par exemple possible que sur les \( 1000\) machines, une seule soit tombée en panne après un an, et que $249$ soient tombées en panne au début de la quatrième année.
            \item
                Il est également possible qu'après la huitième année, \( 999\) machines soient en panne, et qu'une seule ait encore tenu jusqu'à la quinzième année.
        \end{enumerate}
    \end{remark}
\end{example}


