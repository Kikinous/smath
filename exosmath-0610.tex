% This is part of Un soupçon de mathématique sans être agressif pour autant
% Copyright (c) 2014
%   Laurent Claessens
% See the file fdl-1.3.txt for copying conditions.

\begin{exercice}\label{exosmath-0610}

    Soit \( d_1\) la droite d'équation \( y=2x-1\).
    \begin{enumerate}
        \item
            Déterminer l'équation de la droite \( d_2\) parallèle à \( d_1\) et passant par \( A(-2;1)\).
        \item
            Soient les points \( C(-1;2)\) et \( E(1;6)\). Est-ce que la droite \( (CE)\) est parallèle à \( d_1\) ?
        \item
            Le point \( B(1;3)\) est-il sur la droite \( (CE)\) ?
        \item
            Les droite \( (CE)\) et \( d_2\) sont-elles sécantes ?
    \end{enumerate}

\corrref{smath-0610}
\end{exercice}
