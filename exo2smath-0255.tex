% This is part of Un soupçon de mathématique sans être agressif pour autant
% Copyright (c) 2015
%   Laurent Claessens
% See the file fdl-1.3.txt for copying conditions.

\begin{exercice}\label{exo2smath-0255}

    Le but de cet exercice est de montrer que les formules de distribution et de factorisation fonctionnent \emph{vraiment} pour toutes les valeurs «des lettres».

    \begin{enumerate}
        \item
            Sur papier : développer et réduire : \( 3\times (x+3)-x\).
        \item
            Avec un tableur, faire calculer les valeurs de \( 3(x+3)-x\) pour une centaine de valeurs de \( x\) entre \( -10\) et \( 10\). Qu'est-ce que l'on remarque ?
        \item
            Quel est le problème avec ce tableau ?

            \includegraphics[width=5cm]\{faux_tableur.pdf} 

    \end{enumerate}
    

<+Exo2smath-0255+>

\corrref{2smath-0255}
\end{exercice}
