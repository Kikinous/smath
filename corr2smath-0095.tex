% This is part of Un soupçon de mathématique sans être agressif pour autant
% Copyright (c) 2015
%   Laurent Claessens
% See the file fdl-1.3.txt for copying conditions.

\begin{corrige}{2smath-0095}

    \begin{enumerate}
        \item
            Il s'agit ici de bien faire la différence entre les élèves qui partent en vacance (deux sur cinq), et ceux qui partent en vacances \emph{à l'étranger}. Ceux qui partent en vacances à l'étranger sont la moitiés de ceux qui partent en vacances, donc il sont un sur cinq (la moitié de \( 2/5\)).
        \item
            Vu que deux élèves sur cinq partent en vacances, cela fait
            \begin{equation}
                \frac{ 2 }{ 5 }\times 530=212 
            \end{equation}
            élèves. Les élèves qui restent sont donc \( 530-212=318\).
    \end{enumerate}
\end{corrige}
