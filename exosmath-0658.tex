% This is part of Un soupçon de mathématique sans être agressif pour autant
% Copyright (c) 2014-2015
%   Laurent Claessens
% See the file fdl-1.3.txt for copying conditions.

\begin{exercice}\label{exosmath-0658}

    Un oiseau plonge dans l'eau depuis une branche. La trajectoire de l'oiseau est donnée par une fonction \( h(x)\) donnant sa hauteur au dessus du niveau de l'eau. Il s'agit d'une parabole donnée par \( h(x)=x^2-6x+5\) pour \( x\in\mathopen[ 0 ; 6 \mathclose]\).

    \begin{enumerate}
        \item
            Quelle est la hauteur de la branche ?
        \item
            Montrer que \( h(x)=(x-3)^2-4\)
        \item
            Donner le tableau de variations de \( h\).
        \item
            Jusqu'à quelle profondeur plonge l'oiseau ?
        \item
            Quelle équation faut-il résoudre pour trouver à quel endroit l'oiseau entre et sort de l'eau ?
        \item
            Résoudre cette équation.
        \item
            Donner un tableau de signe de \( h\).
        \item
            Dessiner l'allure générale de la trajectoire de l'oiseau.
    \end{enumerate}

\corrref{smath-0658}
\end{exercice}
