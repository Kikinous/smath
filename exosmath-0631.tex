% This is part of Un soupçon de mathématique sans être agressif pour autant
% Copyright (c) 2014
%   Laurent Claessens
% See the file fdl-1.3.txt for copying conditions.

\begin{exercice}\label{exosmath-0631}

    Soit une fonction \( g\) définie sur \( [-5;6]\) telle que
    \begin{enumerate}
        \item
            l'image de \( -5\) est zéro,
        \item
            \( g\) est croissante sur \( [-5;0]\) et décroissante sur \( [3;6]\).
        \item
            \( g(0)=3\) et \( g(6)=5\)
        \item
            Le maximum de \( g\) est atteint pour \( x=2\).
    \end{enumerate}
    À partir de ces données, répondre aux questions suivantes :
    \begin{enumerate}
        \item
            Comparer \( g(-1)\) et \( g(5)\).
        \item
            Dessiner une représentation graphique possible de \( g\).
        \item
            Dresser le tableau de variations de la fonction que vous avez dessinée.
        \item
            Dresser le tableau de signes de la fonction telle que vous l'avez dessinée.
    \end{enumerate}

\corrref{smath-0631}
\end{exercice}
