% This is part of Un soupçon de mathématique sans être agressif pour autant
% Copyright (c) 2012
%   Laurent Claessens
% See the file fdl-1.3.txt for copying conditions.

\begin{exercice}\label{exoPremiere-0039}

    \begin{enumerate}
        \item
            Écrire une fonction qui prend une liste en arguent et en retourne le dernier élément. Tester cette fonction pour les listes \info{A=[1,5,-1]} et \info{A=[6,5,4]} et \info{A=range(1,5)}.

            Formellement, l'objet \info{range(1,5)} n'est pas une liste, mais en pratique, ça y ressemble beaucoup et votre fonction devrait retourner \( 4\), comme si \info{range(1,5)} était la liste \info{[1,2,3,4]}.
        \item
            Écrire une fonction qui prend une liste en argument et retourne le plus grand élément. Pour ce faire, commencez par trier la liste et utilisez la fonction précédente pour extraire le dernier élément.
    \end{enumerate}
    En pratique, sachez que pour savoir le plus grand élément d'une liste, python dispose de la fonction \info{max}. Pour la suite, utilisez cette dernière, et non cette que vous aurez programmé dans cet exercice.

\corrref{Premiere-0039}
\end{exercice}
