% This is part of Un soupçon de mathématique sans être agressif pour autant
% Copyright (c) 2012-2013
%   Laurent Claessens
% See the file fdl-1.3.txt for copying conditions.

\begin{exercice}\label{exoSeconde-0091}

\begin{wrapfigure}{r}{6.0cm}
            \vspace{-0.5cm}        % à adapter.
                \centering
                    \input{Fig_FaussePerspectivewAwxAJ.pstricks}
                \end{wrapfigure}


    Sur la figure ci-contre, \( ABCD\) est un carré tandis que \( J\), \( K\) et \( L\) sont trois points dans l'espace. Les propositions suivantes sont-elles vraies, fausses, ou indécidables sur le dessin ?

    \begin{enumerate}
        \item
            Le point \( K\) est sur le segment \( [AB]\).
        \item
            Les points \( D\), \( L\) et \( C\) sont alignés.
        \item
            Les points \( A\), \( L\) et \( C\) sont alignées.
        \item
            La droite \( (KL)\) est parallèle à la droite \( (AD)\).
        \item
            La droite \( (BJ)\) est parallèle à la droite \( (DC)\).
        \item
            La droite \( (AL)\) intersecte la droite \( (DC)\).
    \end{enumerate}

\corrref{Seconde-0091}
\end{exercice}
