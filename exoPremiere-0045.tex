% This is part of Un soupçon de mathématique sans être agressif pour autant
% Copyright (c) 2012
%   Laurent Claessens
% See the file fdl-1.3.txt for copying conditions.

\begin{exercice}\label{exoPremiere-0045}

    Soit la fonction du second degré \( f(x)=(m-2)x^2+5x+7-m\) avec \( m\) étant un paramètre différent de \( 2\).
    \begin{enumerate}
        \item
            Prouver que \( -1\) est une racine de \( f\) pour tout \( m\).
        \item
            Trouver la seconde racine sans utiliser la technique du discriminant.
        \item
            Déterminer la valeur de \( m\) pour que cette seconde racine soit égale à \( 10\).
    \end{enumerate}
    %Source : \cite{nhcezt}.
    % Remettre ce \cite
\corrref{Premiere-0045}
\end{exercice}
