% This is part of Un soupçon de mathématique sans être agressif pour autant
% Copyright (c) 2012
%   Laurent Claessens
% See the file fdl-1.3.txt for copying conditions.

\begin{corrige}{Seconde-0017}

    \begin{enumerate}
        \item
            Vrai. D'ailleurs il est possible qu'un peu plus de la moitié des hommes ait \( 38\) ans ou moins.
        \item
            Vrai. D'ailleurs il est possible qu'un peu plus de la moitié des femme ait \( 41\) ans ou plus.
        \item
            Faux. Moins de la moitié des hommes ont plus que \( 41\) ans, et en même temps au maximum la moitié des femmes ont plus de \( 41\) ans.
        \item
            Vrai. Au moins la moitié des femmes ont \( 41\) ans ou plus alors qu'au maximum la moitié des hommes y sont\footnote{Pour que exactement la moitié des hommes y soient il faudrait une distribution des âges en pratique complètement impossible, mais mathématiquement pas exclue.}.
    \end{enumerate}

\end{corrige}
