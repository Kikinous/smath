% This is part of Un soupçon de mathématique sans être agressif pour autant
% Copyright (c) 2014
%   Laurent Claessens
% See the file fdl-1.3.txt for copying conditions.

\begin{exercice}\label{exosmath-0851}

    Le triangle \( IJK\) est isocèle en \( J\); la longueur \( IK\) est \SI{5}{\centi\meter} et l'angle \( \widehat{JIK}\) mesure \SI{35}{\degree}. Le dessiner en vraie grandeur.

\corrref{smath-0851}
\end{exercice}
