% This is part of Un soupçon de mathématique sans être agressif pour autant
% Copyright (c) 2015
%   Laurent Claessens
% See the file fdl-1.3.txt for copying conditions.

\begin{exercice}[\cite{PMUPooZAPegi}]\label{exo2smath-0237}

    Voici les éphémérides d'une éclipse.
    \begin{center}
    \begin{tabular}[]{|c|c|}
        \hline
        Entrée dans l'ombre&6h58\\
        \hline
        Commencement de la totalité de l'éclipse&8h06\\
        \hline
        Maximum de l'éclipse&8h45\\
        \hline
        Fin de la totalité de l'éclipse&9h24\\
        \hline
        Sortie de l'ombre de la Terre&10h33\\
        \hline
    \end{tabular}
    \end{center}
    Pendant combien de temps l'éclipse est-elle totale ?

\corrref{2smath-0237}
\end{exercice}
