% This is part of Un soupçon de mathématique sans être agressif pour autant
% Copyright (c) 2013
%   Laurent Claessens
% See the file fdl-1.3.txt for copying conditions.

\begin{exercice}\label{exosmath-0487}

Dans un repère orthonormé nous considérons les points \( K=(-1;3)\), \( L=(4;2)\), \( M=(2;0)\) et \( N=(-3;1)\).
\begin{enumerate}
    \item
        Calculer les milieux des segments \( [KM]\) et \( [NL]\).
    \item
        Que pouvons-nous en déduire sur la nature du quadrilatère \( KLMN\) ?
\end{enumerate}

\corrref{smath-0487}
\end{exercice}
