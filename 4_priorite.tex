% This is part of Un soupçon de mathématique sans être agressif pour autant
% Copyright (c) 2014
%   Laurent Claessens
% See the file fdl-1.3.txt for copying conditions.

%+++++++++++++++++++++++++++++++++++++++++++++++++++++++++++++++++++++++++++++++++++++++++++++++++++++++++++++++++++++++++++ 
\section{Priorité des opérations}
%+++++++++++++++++++++++++++++++++++++++++++++++++++++++++++++++++++++++++++++++++++++++++++++++++++++++++++++++++++++++++++

% This is part of Un soupçon de mathématique sans être agressif pour autant
% Copyright (c) 2014
%   Laurent Claessens
% See the file fdl-1.3.txt for copying conditions.

%--------------------------------------------------------------------------------------------------------------------------- 
\subsection*{Sans multiplications}
%---------------------------------------------------------------------------------------------------------------------------

\begin{enumerate}
    \item
Calculer \( A=12+5-4\), \( B=12-4+5\) et \( C=5-4+12\).
\item
    Qu'observe-t-on ?
\end{enumerate}

%--------------------------------------------------------------------------------------------------------------------------- 
\subsection*{Avec multiplications}
%---------------------------------------------------------------------------------------------------------------------------

\begin{enumerate}
    \item
        Calculer mentalement \( D=4\times 5+2\), \( E=5\times 4+2\) et \( F=2\times 4+5\).
    \item
        Recalculer ces expressions en les tapant à la calculatrice exactement comme elles sont écrites.
    \item
        Qu'observe-t-on ?
\end{enumerate}

%--------------------------------------------------------------------------------------------------------------------------- 
\subsection*{Avec des parenthèses}
%---------------------------------------------------------------------------------------------------------------------------

\begin{enumerate}
    \item
        Calculer \( G=(7+3)\times 3\), \( H=4\times (30-21)\) et \( K=(3\times 4)\times (7-2)\).
    \item
        Dans quel ordre faut-il faire les calculs ?
\end{enumerate}


%--------------------------------------------------------------------------------------------------------------------------- 
\subsection{Additions et soustractions}
%---------------------------------------------------------------------------------------------------------------------------

\begin{Aretenir}
    Un enchaînement sans parenthèses d'additions et de soustractions s'effectue de gauche à droite.
\end{Aretenir}

\begin{example}
    \begin{subequations}
        \begin{align}
            A&=12+5-6\\
            &=17-6\\
            &=11.
        \end{align}
    \end{subequations}
\end{example}

%--------------------------------------------------------------------------------------------------------------------------- 
\subsection{Priorité de la multiplication}
%---------------------------------------------------------------------------------------------------------------------------

\begin{Aretenir}
    Dans un enchaînement sans parenthèses, les multiplications et les divisions sont à effectuer en priorité.
\end{Aretenir}

Moyen mnémotechnique : la multiplication passe avant parce que c'est elle qui fait foi.

\begin{example}
    \begin{equation}
        5\times 8+7=40+7=47
    \end{equation}
    et
    \begin{equation}
        2+6\times 3=2+18=20
    \end{equation}
\end{example}

\begin{remark}
    Si il y a des multiplication et des divisions mélangées, il faut calculer de gauche à droite.
\end{remark}

\begin{example}
    \begin{equation}
        12\times 2\div 3=24\div 3=8.
    \end{equation}
\end{example}

\begin{remark}
    Les fractions sont des quotients avec des parenthèses sous-entendues.
\end{remark}

\begin{example}
    \begin{equation}
        \frac{ 13+7 }{ 4+1 }=(13+7)\div (4+1)=20\div 5=4.
    \end{equation}
\end{example}

%--------------------------------------------------------------------------------------------------------------------------- 
\subsection{Avec des parenthèses}
%---------------------------------------------------------------------------------------------------------------------------

\begin{Aretenir}
    Lorsqu'une expression contient des parenthèses, elles sont à effectuer en priorité.
\end{Aretenir}

\begin{example}
    \begin{equation}
        5\times (8+13)=5\times 21=105
    \end{equation}
    et
    \begin{equation}
         (8-2)\times 15=6\times 15=90.
    \end{equation}
\end{example}

%+++++++++++++++++++++++++++++++++++++++++++++++++++++++++++++++++++++++++++++++++++++++++++++++++++++++++++++++++++++++++++ 
\section{Factorisation}
%+++++++++++++++++++++++++++++++++++++++++++++++++++++++++++++++++++++++++++++++++++++++++++++++++++++++++++++++++++++++++++

% This is part of Un soupçon de mathématique sans être agressif pour autant
% Copyright (c) 2014
%   Laurent Claessens
% See the file fdl-1.3.txt for copying conditions.

%--------------------------------------------------------------------------------------------------------------------------- 
\subsection*{Activité : Bertrand vend des pots}
%---------------------------------------------------------------------------------------------------------------------------

Bertrand l'artisan vend des pots sur le marché. Chaque pot lui coûte \( 2\)€ de matériel et est revendu \( 7\)€.
\begin{enumerate}
    \item
        Pour savoir quel sera son gain en vendant \( 13\) pots, Bertrand fait l'opération suivante :
        \begin{equation}
            \input{GKJooPMzPdG.calcul}
        \end{equation}
        Calculer cette valeur.
    \item
        Son ami Josef lui fait remarquer qu'il peut plus facilement calculer son bénéfice en calculant d'abord le bénéfice d'un pot et en multipliant ensuite par le nombre de pot.

        Proposer, en suivant cette idée, une expression donnant le bénéfice de Bertrand lorsqu'il vend \( 13\) pots.

        Vérifier qu'elle fonctionne en recalculant le bénéfice réalisé par la vente de \( 13\) pots.
    \item
        Calculer mentalement le bénéfice réalisé lors de la vente de \( 20\) pots.
\end{enumerate}


%--------------------------------------------------------------------------------------------------------------------------- 
\subsection{Calcul malin}
%---------------------------------------------------------------------------------------------------------------------------

Pour effectuer mentalement
\begin{equation}
    102\times 53,
\end{equation}
on fait
\begin{itemize}
    \item \( 100\times 53=5300\)
    \item \( 2\times 53=106\)
    \item \( 5300+106=5406\).
\end{itemize}
Autrement dit nous faisons
\begin{equation}
    (100+2)\times 53=100\times 53+2\times 53.
\end{equation}

%--------------------------------------------------------------------------------------------------------------------------- 
\subsection{factorisation}
%---------------------------------------------------------------------------------------------------------------------------

Si \( a,x,y\) sont de nombres quelconques, alors
\begin{subequations}
    \begin{align}
        a\times x+a\times y&=a\times (x+y)\\
        a\times x-a\times y&=a\times (x-y)
    \end{align}
\end{subequations}

\begin{example}
    Pour calculer le bénéfice de vente de \( 13\) pots nous avons fait
    \begin{equation}
        13\times 7-13\times 2=13\times (7-2).
    \end{equation}
    Ce passage s'appelle \defe{factoriser}{factoriser}.
\end{example}

\begin{example}
    Lorsque nous faisons
    \begin{equation}
        98\times 53=(100-2)\times 53=100\times 53-2\times 53,
    \end{equation}
    c'est un développement.
\end{example}
