% This is part of Un soupçon de mathématique sans être agressif pour autant
% Copyright (c) 2012-2014
%   Laurent Claessens
% See the file fdl-1.3.txt for copying conditions.

% This is part of Un soupçon de mathématique sans être agressif pour autant
% Copyright (c) 2014
%   Laurent Claessens
% See the file fdl-1.3.txt for copying conditions.

%--------------------------------------------------------------------------------------------------------------------------- 
\subsection*{Sans multiplications}
%---------------------------------------------------------------------------------------------------------------------------

\begin{enumerate}
    \item
Calculer \( A=12+5-4\), \( B=12-4+5\) et \( C=5-4+12\).
\item
    Qu'observe-t-on ?
\end{enumerate}

%--------------------------------------------------------------------------------------------------------------------------- 
\subsection*{Avec multiplications}
%---------------------------------------------------------------------------------------------------------------------------

\begin{enumerate}
    \item
        Calculer mentalement \( D=4\times 5+2\), \( E=5\times 4+2\) et \( F=2\times 4+5\).
    \item
        Recalculer ces expressions en les tapant à la calculatrice exactement comme elles sont écrites.
    \item
        Qu'observe-t-on ?
\end{enumerate}

%--------------------------------------------------------------------------------------------------------------------------- 
\subsection*{Avec des parenthèses}
%---------------------------------------------------------------------------------------------------------------------------

\begin{enumerate}
    \item
        Calculer \( G=(7+3)\times 3\), \( H=4\times (30-21)\) et \( K=(3\times 4)\times (7-2)\).
    \item
        Dans quel ordre faut-il faire les calculs ?
\end{enumerate}


%+++++++++++++++++++++++++++++++++++++++++++++++++++++++++++++++++++++++++++++++++++++++++++++++++++++++++++++++++++++++++++ 
\section{Priorité des opérations}
%+++++++++++++++++++++++++++++++++++++++++++++++++++++++++++++++++++++++++++++++++++++++++++++++++++++++++++++++++++++++++++

%--------------------------------------------------------------------------------------------------------------------------- 
\subsection{Additions et soustractions}
%---------------------------------------------------------------------------------------------------------------------------

\begin{Aretenir}
    Un enchaînement sans parenthèses d'additions et de soustractions s'effectue dans l'ordre au choix.
\end{Aretenir}

Attention cependant :
\begin{equation}
    A=12+5-6=17-6=11,
\end{equation}
mais nous ne pouvons pas faire les opérations dans l'ordre \( 5-6+12\) parce que \( 5-6\) est une opération «impossible».

%--------------------------------------------------------------------------------------------------------------------------- 
\subsection{Priorité de la multiplication}
%---------------------------------------------------------------------------------------------------------------------------

\begin{Aretenir}
    Dans un enchaînement sans parenthèses d'additions et de multiplications, les multiplications sont à effectuer d'abord.

    Attention : les divisions sont prioritaires comme les multiplications.
\end{Aretenir}

Moyen mnémotechnique : la multiplication passe avant parce que c'est elle qui fait foi.

%--------------------------------------------------------------------------------------------------------------------------- 
\subsection{Avec des parenthèses}
%---------------------------------------------------------------------------------------------------------------------------

\begin{Aretenir}
    Lorsqu'une expression contient des parenthèses, elles sont à faire en priorité.
\end{Aretenir}

%+++++++++++++++++++++++++++++++++++++++++++++++++++++++++++++++++++++++++++++++++++++++++++++++++++++++++++++++++++++++++++ 
\section{Nombres relatifs}
%+++++++++++++++++++++++++++++++++++++++++++++++++++++++++++++++++++++++++++++++++++++++++++++++++++++++++++++++++++++++++++

L'opération
\begin{equation}
    A=25+7-17=32-17=15
\end{equation}
est faite plus simplement \( 7-17+25=-10+25=15\).

