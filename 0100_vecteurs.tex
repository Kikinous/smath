% This is part of Un soupçon de mathématique sans être agressif pour autant
% Copyright (c) 2012-2014
%   Laurent Claessens
% See the file fdl-1.3.txt for copying conditions.

% Ceci sont les vecteurs sans la colinéarité sans le produit par un nombre et sans alignement de trois points.

%+++++++++++++++++++++++++++++++++++++++++++++++++++++++++++++++++++++++++++++++++++++++++++++++++++++++++++++++++++++++++++ 
\section*{Introduction}
%+++++++++++++++++++++++++++++++++++++++++++++++++++++++++++++++++++++++++++++++++++++++++++++++++++++++++++++++++++++++++++

% This is part of Un soupçon de mathématique sans être agressif pour autant
% Copyright (c) 2013
%   Laurent Claessens
% See the file fdl-1.3.txt for copying conditions.

\begin{multicols}{2}
    \begin{enumerate}
        \item
Le TER 894258 a pour horaire :
\begin{description}
    \item[14h56] Besançon-Viotte
    \item[15h06] St-Vit
    \item[15h21] Dole-Ville
    \item[15h30] Auxonne
    \item[15h39] Genlis
    \item[15h51] Dijon-Ville
\end{description}
Dessiner le trajet sur une ligne du temps en indiquant les durées entre les stations.
        \item
            En supposant les mêmes temps de parcours, quelles sont les heures d'arrivées dans les différentes gares du TER partant à 20h23 pour le même trajet ?
        \item
             Les points \( A(1;2)\), \( B(3;3)\), \( C(2;5)\) et \( D(0;4)\) forment un carré. Donner les coordonnées du «même» carré \( A'B'C'D'\) partant de \( A'(4;0)\).
         \item
             Soient les points \( K(0;0)\), \( L(4;1)\) et \( M(3;3)\). Donner les coordonnées du point \( N\) tel que \( KLMN\) soit un parallélogramme.
    \end{enumerate}
\end{multicols}


\clearpage

Éléments de réponse :
\begin{enumerate}
    \item
        Pour calculer l'intervalle de temps entre deux heures données, il faut seulement faire la différence entre les deux.
        \begin{center}
           \input{Fig_EDEYRhQ.pstricks}
        \end{center}
    \item
        Ce serait 20h23, 20h33, 20h48, 20h57,21h06,21h18. Il suffit de garder les mêmes «intervalles».
    \item
        Voici le dessin
        \begin{center}
           \input{Fig_WPrirwB.pstricks}
        \end{center}
\end{enumerate}
La conclusion est que nous pouvons faire des translations en comptant les distances «horizontales» et «verticales».

%+++++++++++++++++++++++++++++++++++++++++++++++++++++++++++++++++++++++++++++++++++++++++++++++++++++++++++++++++++++++++++
\section{Translation}
%+++++++++++++++++++++++++++++++++++++++++++++++++++++++++++++++++++++++++++++++++++++++++++++++++++++++++++++++++++++++++++

\begin{definition}  
    La \defe{translation}{translation} \( t_{\vect{ AB }}\) est la transformation du plan qui à un point \( C\) fait correspondre l'unique point \( D\) tel que les segments \( [AD]\) et \( [BC]\) aient même milieu.

Nous notons \( \vect{ AB }\) le vecteur associé à cette translation.
\end{definition}

\begin{center}
   \input{Fig_DefVecteurAXDDGP.pstricks}
\end{center}

Étant donné qu'un quadrilatère dont les diagonales se coupent en leur milieu est un parallélogramme, nous avons immédiatement le règle suivante :
\begin{Aretenir}
    Le quadrilatère \( ABDC\) est un parallélogramme si et seulement si \( D=t_{\vect{ AB }}(C)\). C'est à dire si \( D\) est donné à partir de \( C\) par la translation de vecteur \( \vect{ AB }\).

Attention : il s'agit bien de \( ABDC\) et non de \( ABCD\).
\end{Aretenir}

%Le dessin à côté de la définition \ref{DefAAJEuS}, aplati, donne immédiatement aussi
%\begin{equation}
%    t_{\vect{ AB }}(A)=B.
%\end{equation}

    Le \defe{vecteur}{vecteur} \vect{ AB } est le déplacement qui permet d'aller de $A$ à \( B\).

\begin{definition}
    Nous disons que \( \vect{ AB }=\vect{ CD }\) si et seulement si \( t_{\vect{ AB }}(C)=D\), c'est à dire si \( ABDC\) est un parallélogramme.
\end{definition}

%+++++++++++++++++++++++++++++++++++++++++++++++++++++++++++++++++++++++++++++++++++++++++++++++++++++++++++++++++++++++++++ 
\section{Somme de vecteurs}
%+++++++++++++++++++++++++++++++++++++++++++++++++++++++++++++++++++++++++++++++++++++++++++++++++++++++++++++++++++++++++++

\begin{definition}
    Le vecteur \defe{somme}{somme!de vecteur}\index{vecteur!somme} \( \vect{ AB }+\vect{ CD }\) est le vecteur qui correspond à la translation composée de \( t_{\vect{ AB }}\) par \( t_{\vect{ CD }}\)
\end{definition}

\begin{Aretenir}
    \begin{multicols}{2}
    Nous avons la \defe{relation de Chasles}{Chasles} qui permet de mettre des vecteurs «bout à bout» :
    \begin{equation}
        \vect{ AB }+\vect{ BC }=\vect{ AC }.
    \end{equation}

    \columnbreak

%Une illustration de la relation de Chasles est donnée à la figure \ref{LabelFigChaslesGTRtKR}. % From file ChaslesGTRtKR
%\newcommand{\CaptionFigChaslesGTRtKR}{La somme $ \vect{ AB }+\vect{ BC }$ est le vecteur $ \vect{ AC }$.}
\input{Fig_ChaslesGTRtKR.pstricks}

    \end{multicols}
\end{Aretenir}

%+++++++++++++++++++++++++++++++++++++++++++++++++++++++++++++++++++++++++++++++++++++++++++++++++++++++++++++++++++++++++++ 
\section{Coordonnées d'un vecteur dans un repère}
%+++++++++++++++++++++++++++++++++++++++++++++++++++++++++++++++++++++++++++++++++++++++++++++++++++++++++++++++++++++++++++

\begin{multicols}{2}
    \begin{definition}
        Les coordonnées du vecteur \( \vect{ u }\) dans un repère d'origine \( O\) sont les coordonnées du point \( M\) tel que \( \vect{ u }=\vect{ OM }\).
    \end{definition}

    \columnbreak

    %The result is on figure \ref{LabelFigfigureNNgEEzx}. % From file figureNNgEEzx
    %\newcommand{\CaptionFigfigureNNgEEzx}{<+Type your caption here+>}
    \begin{center}
\input{Fig_figureNNgEEzx.pstricks}
    \end{center}
\end{multicols}
Dans le cas ci-dessus, nous avons \( \vect{ AB }=\vect{ OM }\) et les coordonnées de \( M\) sont \( M=(-2;1)\). Nous notons
\begin{equation}
    \vect{ AB }=\begin{pmatrix}
        -2    \\ 
        1    
    \end{pmatrix}.
\end{equation}

\begin{Aretenir}
    \begin{multicols}{2}
    Si \( A=(x_A;x_B)\) et \( B=(x_B;y_B)\) alors
    \begin{equation*}
        \vect{ AB }=\begin{pmatrix}
            x_B-x_A    \\ 
            y_B-y_A    
        \end{pmatrix}.
    \end{equation*}

    \columnbreak

   %The result is on figure \ref{LabelFigfigureLEOvqez}. % From file figureLEOvqez
%\newcommand{\CaptionFigfigureLEOvqez}{<+Type your caption here+>}
    \begin{center}
\input{Fig_figureLEOvqez.pstricks}
    \end{center}

    \end{multicols}
   % TODO : lire le TODO qu'il y a dans le fichier phystricksfigureLEOvqez 
\end{Aretenir}

\begin{Aretenir}
    \begin{enumerate}
        \item
    Le vecteur \( \vect{ AB }\) est \defe{nul}{vecteur!nul} si les points \( A\) et \( B\) sont confondus. On le note alors \( \vect{ 0 }\).

        \item
    Nous définissons aussi \( -\vect{ AB }=\vect{ BA }\).
    \end{enumerate}
    
\end{Aretenir}
%--------------------------------------------------------------------------------------------------------------------------- 
\section{Somme et coordonnées}
%---------------------------------------------------------------------------------------------------------------------------

\begin{propriete}
    Si dans un repère \( \vect{ u }=\begin{pmatrix}
        x    \\ 
        y    
    \end{pmatrix}\) et \( \vect{ v }=\begin{pmatrix}
        x'    \\ 
        y'    
    \end{pmatrix}\), alors 
    \begin{equation}
        \vect{ u }+\vect{ v }=\begin{pmatrix}
            x+x'    \\ 
            y+y'    
        \end{pmatrix}.
    \end{equation}
    
\end{propriete}

\begin{proof}
    Soient \( A\) et \( B\) des points tels que \( \vect{ u }=\vect{ AB }\). Vu que les vecteurs peuvent être placés n'importe où, nous pouvons placer \( \vect{ v }\) au point \( B\) et dire \( \vect{ v }=\vect{ BC }\) pour un certain point \( C\). Par les relations de Chasles,
    \begin{equation}    \label{EqASjyYXs}
        \vect{ u }+\vect{ v }=\vect{ AC }=\begin{pmatrix}
            x_C-x_A    \\ 
            y_C-y_A    
        \end{pmatrix}.
    \end{equation}
    
    D'autre part étant donné que \( \vect{ u }=\vect{ AB }\), nous avons
    \begin{subequations}
        \begin{align}
            x=x_B-x_A\\
            y=y_B-y_A;
        \end{align}
    \end{subequations}
    et vu que \( \vect{ v }=\vect{ BC }\)
    \begin{subequations}
        \begin{align}
            x'=x_C-x_B\\
            y'=y_C-y_B
        \end{align}
    \end{subequations}
    Du coup nous avons
    \begin{subequations}
        \begin{align}
            x+x'=(x_B-x_A)+(x_C-x_B)=x_C-x_A\\
            y+y'=(y_B-y_A)+(y_C-y_B)=y_C-y_A
        \end{align}
    \end{subequations}
    En comparant avec \eqref{EqASjyYXs}, nous avons bien
    \begin{equation}
        \vect{ u }+\vect{ v }=\vect{ AC }=\begin{pmatrix}
            x_C-x_A    \\ 
            y_C-y_A    
        \end{pmatrix}=\begin{pmatrix}
            x+x'    \\ 
            y+y'    
        \end{pmatrix},
    \end{equation}
    ce qu'il fallait démontrer.
\end{proof}
