% This is part of Un soupçon de mathématique sans être agressif pour autant
% Copyright (c) 2012
%   Laurent Claessens
% See the file fdl-1.3.txt for copying conditions.

\begin{exercice}\label{exosmath-0147}

    Le tableau ci-dessous présente la série de notes obtenues par les
élèves d'une classe de 2\up{nde}  à un devoir.

\begin{center}
  \begin{tabular}{|c|*{10}{c|}}
    \hline
    \textbf{Note sur 20} & \ 5&\ 6&\ 8&\ 9&11&12&13&15&18&19 \\
    \hline
    \textbf{Effectif} & \ 1&\ 2&\ 6&\ 2&1&4&2&3&1&1 \\
    \hline
  \end{tabular}
\end{center}
\medskip

\begin{enumerate}
\item Représenter cette série par un diagramme en bâtons.
\item Quel est l'effectif de la classe ?
\item Calculer la note moyenne de ce devoir. En donner la valeur
  arrondie au dixième de point près.
\item Quel pourcentage, arrondi à 1 \% près, représente l'ensemble des
  élèves ayant obtenu une note inférieure ou égale à 8 ?
\item Déterminer la note médiane de cette série.
\item Déterminer les quartiles de cette série.
\item Représenter le diagramme en boites de cette série, et
  interpréter par une ou deux phrases les résultats obtenus.
\end{enumerate}



\corrref{smath-0147}
\end{exercice}
