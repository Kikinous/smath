% This is part of Un soupçon de mathématique sans être agressif pour autant
% Copyright (c) 2015
%   Laurent Claessens
% See the file fdl-1.3.txt for copying conditions.

\begin{corrige}{2smath-0312}

    Le nombre total de lettres dans le texte est :
    \begin{equation}
        2326+3982+28+21262=27598.
    \end{equation}
    \begin{enumerate}
        \item
            Il y a \( 3982\) «e» parmi \( 27598\) lettres, donc la proportion est :
            \begin{equation}
                \frac{ 3982 }{ 27598 }\simeq 0.14428\simeq 14.42\%.
            \end{equation}
        \item
            Il y a \( 28\) «w» parmi \( 27598\) lettres, donc la proportion est :
            \begin{equation}
                \frac{ 28 }{ 27598 }\simeq 0.00101\simeq 0.1\%.
            \end{equation}
        \item
            Sur les \( 1000\) lettres ajoutées, \( 800\) sont autres que des «e», c'est à dire une proportion de
            \begin{equation}
                \frac{ 800 }{ 1000 }=\frac{ 4 }{ 5 }=0.8.
            \end{equation}
            <++>
    \end{enumerate}
    <++>

\end{corrige}
