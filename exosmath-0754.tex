% This is part of Un soupçon de mathématique sans être agressif pour autant
% Copyright (c) 2014
%   Laurent Claessens
% See the file fdl-1.3.txt for copying conditions.

\begin{exercice}[\cite{NRHooXFvgpp}]\label{exosmath-0754}

    Pour chacune des égalités suivantes, remplacer le symbole \( \diamondsuit\) par l'opération qui convient :
    \begin{multicols}{2}
        \begin{enumerate}
            \item
                \( (-3)\diamondsuit (-2)=-5\)
            \item
                \( (-3)\diamondsuit(-2)=6\)
            \item
                \( (-2)\diamondsuit (-2)=4\)
            \item
                \( (-2)\diamondsuit (-2)=-4\)
        \end{enumerate}
    \end{multicols}
    Et une dernière : \( (-5)\diamondsuit (+4)=(-12)\diamondsuit (+8)\).

\corrref{smath-0754}
\end{exercice}
