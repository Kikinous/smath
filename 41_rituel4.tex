% This is part of Un soupçon de mathématique sans être agressif pour autant
% Copyright (c) 2015
%   Laurent Claessens
% See the file fdl-1.3.txt for copying conditions.


\setcounter{exorituel}{0}

\begin{rituel}
    
    Quelle est la meilleure approximation de la longueur du segment \( [SA]\) ?

\begin{center}
    \large
    \input{Fig_MXXQooRUNiwK.pstricks}
\end{center}

\( AH=5\) et \( HS=6\)


\begin{enumerate}
    \item
        \( 11.3\)
    \item
        \( 7.81\)
    \item
        \( 5.34\)
\end{enumerate}
\end{rituel}

\begin{rituel}
    
\begin{enumerate}
    \item
        Développer et réduire l'expression
        \begin{equation}
            3(x+2)-x.
        \end{equation}
    \item
        Pour quelle valeur de \( x\), cela vaut \( 52\) ?
\end{enumerate}

\end{rituel}

\begin{rituel}
    Déterminer les longueurs approximatives des trois côtés :

\begin{center}
   \input{Fig_MXGJooGTXbPw.pstricks}
\end{center} 

Vous pouvez utiliser les approximations
\begin{enumerate}
    \item \( \cos(\SI{30}{\degree})\simeq 0.866\)
    \item \( \cos(\SI{60}{\degree})=0.5\)
\end{enumerate}

\end{rituel}

% This is part of Un soupçon de mathématique sans être agressif pour autant
% Copyright (c) 2015
%   Laurent Claessens
% See the file fdl-1.3.txt for copying conditions.

\begin{rituel}
    Parmi les expressions suivantes, dire lesquelles sont égales à
    \begin{equation}
        \frac{ 5\times 7 }{ 3\times 11 }.
    \end{equation}
    \begin{multicols}{2}
    \begin{enumerate}
        \item
            \begin{equation}
                5\times \frac{ 7 }{ 3\times 11 }
            \end{equation}
        \item
            \begin{equation}
                3\times \frac{ 5\times 7 }{ 11 }
            \end{equation}
        \item
            \begin{equation}
                \frac{ 5\times 7 }{ 11 }\div 3
            \end{equation}
        \item
            \begin{equation}
                \frac{ 5 }{ 33 }\times 7
            \end{equation}
    \end{enumerate}
    \end{multicols}
\end{rituel}

% This is part of Un soupçon de mathématique sans être agressif pour autant
% Copyright (c) 2015
%   Laurent Claessens
% See the file fdl-1.3.txt for copying conditions.


\begin{rituel}
    <+rituel 0011+>
\end{rituel}

