% This is part of Un soupçon de mathématique sans être agressif pour autant
% Copyright (c) 2014
%   Laurent Claessens
% See the file fdl-1.3.txt for copying conditions.

Une ligne de chemin de fer doit être construite entre des villes \( A\) et \( B\). La ville \( B\) est à \unit{60}{\kilo\meter} au nord et \unit{20}{\kilo\meter} à l'ouest de la ville \( A\). Une route droite relie déjà ces deux villes et le cahier de charge indique que le chemin de fer doit être parallèle à la route, tout en passant par la gare la plus proche de \( A\), à savoir une petite gare située à \unit{1}{\kilo\meter} à l'est de \( A\).

\begin{enumerate}
    \item
        Faire un dessin de la situation en choisissant un système d'axe pratique à utiliser.
    \item
        Un vieux château situé à \unit{15}{\kilo\meter} au sud et \unit{6}{\kilo\meter} à l'est de \( B\) est un monument classé. Est-ce que cela pose un problème en ce qui concerne le tracé du chemin de fer ?
\end{enumerate}
