% This is part of Un soupçon de mathématique sans être agressif pour autant
% Copyright (c) 2012
%   Laurent Claessens
% See the file fdl-1.3.txt for copying conditions.

\begin{exercice}[\cite{ZNloWAM}]\label{exosmath-0239}

    Donner les équations des droites \( d_i\) données par les renseignements suivants.
    \begin{enumerate}
        \item
            La droite \( d_1\) passe par l'origine et a pour coefficient directeur \( -2\).
        \item
            La droite \( d_2\) passe par les points \( A=(2;3)\) et \( B=(-1;-3)\).
        \item
            La droite \( d_3\) passe par le point \( A\) et est parallèle à l'axe des ordonnées.
        \item
            La droite \( d_4\) passe par le point \( B\) et est parallèle à l'axe des abscisses.
        \item
            La droite \( d_5\) est parallèle à \( d_2\) et passe par \( A\).
    \end{enumerate}

\corrref{smath-0239}
\end{exercice}
