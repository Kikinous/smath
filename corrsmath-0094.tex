% This is part of Un soupçon de mathématique sans être agressif pour autant
% Copyright (c) 2013
%   Laurent Claessens
% See the file fdl-1.3.txt for copying conditions.

\begin{corrige}{smath-0094}


\begin{wrapfigure}{r}{3.5cm}
   \vspace{-0.5cm}        % à adapter.
   \centering
   \input{Fig_LZDNqtV.pstricks}
\end{wrapfigure}

La figure explicative est ci-contre.

    \begin{enumerate}
        \item
            Nous coupons la sphère par un plan situé à une hauteur de \unit{56}{\centi\meter}, comme montré sur la figure ci-contre. La longueur \( OC\) est \unit{56}{\centi\meter} et la longueur \( OI\) est le rayon de la sphère, c'est à dire \unit{65}{\centi\meter}. En appliquant Pythagore dans le triangle \( OCI\) (rectangle en \( C\)) nous trouvons \( CI=\unit{33}{\centi\meter}\).
        \item
            Le rayon du cylindre est, comme nous venons de le voir, \unit{33}{\centi\meter} et sa hauteur est de \unit{112}{\centi\meter}. Donc son volume est de \( V=\pi R^2h=\pi\times 1089\times 112=\)\unit{121968}{\centi\cubic\meter}. Cela fait environ \unit{1.22}{\deci\cubic\meter}.
           
    \end{enumerate}

\end{corrige}
