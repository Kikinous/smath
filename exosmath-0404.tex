% This is part of Un soupçon de mathématique sans être agressif pour autant
% Copyright (c) 2013
%   Laurent Claessens
% See the file fdl-1.3.txt for copying conditions.

\begin{exercice}\label{exosmath-0404}

    Lors d'un saut en chute libre (avant que le parachute ne s'ouvre), la distance parcourue en fonction du temps est donnée par la formule
    \begin{equation}
        d(t)=\frac{ gt^2 }{2}
    \end{equation}
    avec \( g\simeq 9.81\). Ici la distance est donnée en mètres et le temps en secondes.
    \begin{enumerate}
        \item
            Quelle est la distance parcourue en vingt secondes ?
        \item
            Quelle est la vitesse instantanée à ce moment ?
    \end{enumerate}

\corrref{smath-0404}
\end{exercice}
