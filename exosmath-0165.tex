% This is part of Un soupçon de mathématique sans être agressif pour autant
% Copyright (c) 2012
%   Laurent Claessens
% See the file fdl-1.3.txt for copying conditions.

\begin{exercice}\label{exosmath-0165}

    \begin{multicols}{2}

        \begin{enumerate}
            \item
    Les points du graphique ci-contre représentent une suite arithmétique. Quelle est sa raison ? Quel est son terme initial ?
        \item
        Combien vaut \( u_4\) ?
    \item
        Mettre les points \( u_5\) et \( u_6\) sur le même graphique.
    \item
        Dessiner sur le même graphique la suite \( u_n=\frac{ 3 }{2}n-1\).

        \end{enumerate}

    \columnbreak

%The result is on figure \ref{LabelFigfigureQUtOjcm}. % From file figureQUtOjcm
%\newcommand{\CaptionFigfigureQUtOjcm}{<+Type your caption here+>}

    \begin{center}
\input{Fig_figureQUtOjcm.pstricks}
    \end{center}

    \end{multicols}
    
\corrref{smath-0165}
\end{exercice}
