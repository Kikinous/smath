% This is part of Un soupçon de mathématique sans être agressif pour autant
% Copyright (c) 2012
%   Laurent Claessens
% See the file fdl-1.3.txt for copying conditions.

\begin{exercice}\label{exoSeconde-0067}

    Un magasin voudrait vendre une valise à \( 90€\), mais en même temps il voudrait pouvoir afficher «-70\%». Nous voudrions savoir à quel prix «de départ» il doit vendre la valise.
    \begin{enumerate}
        \item
            Nous nommons \( x\) le prix «de départ» de la valise.
        \item
            Écrire le prix de la valise à «-70\%» en fonction de \( x\). Soit \( p(x)\) cette fonction.
        \item
            Résoudre \( p(x)=50\).
    \end{enumerate}
    Est-ce que le résultat trouvé est \( 90\) plus \( 70\%\) de \( 90\) ? Est-ce logique ?

\corrref{Seconde-0067}
\end{exercice}
