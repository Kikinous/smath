% This is part of Un soupçon de mathématique sans être agressif pour autant
% Copyright (c) 2015
%   Laurent Claessens
% See the file fdl-1.3.txt for copying conditions.

\begin{exercice}[\cite{NRHooXFvgpp4}]\label{exo2smath-0184}

    Réaliser les patrons des solides suivants.
            

    \begin{enumerate}

    \item
    En sachant que les trois angles en \( M\) sont droits et les mesures \( ML=\SI{5}{\centi\meter}\), \( MK=\SI{4}{\centi\meter}\) et \( MS=\SI{6}{\centi\meter}\). 

\item\label{ItemTDFMooPuUJaDii}

    Le point \( I\) est le centre du cube.
    \end{enumerate}

   \input{Fig_LOPLooZJmTuk.pstricks}   
   \hspace{-1cm}
   \input{Fig_NFCHooAFFYPx.pstricks}


\corrref{2smath-0184}
\end{exercice}
