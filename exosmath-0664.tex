% This is part of Un soupçon de mathématique sans être agressif pour autant
% Copyright (c) 2014
%   Laurent Claessens
% See the file fdl-1.3.txt for copying conditions.

\begin{exercice}\label{exosmath-0664}

    Soient les points \( A(1;2)\), \( B(-2;4)\) et \( C(-1;1)\).
    \begin{enumerate}
        \item
            Montrer que les points \( A\), \( B\) et \( C\) ne sont pas alignés.
        \item
            Déterminer par le calcul les coordonnées du point \( E\) tel que \( -2\vect{ EA }+3\vect{ EB }=\vect{ AC }\).
        \item
            Soit \( F(x,x+1)\). Déterminer pour quelle valeur de \( x\) les droites \( (AB)\) et \( (CF)\) sont parallèles.
    \end{enumerate}
    <++>

\corrref{smath-0664}
\end{exercice}
