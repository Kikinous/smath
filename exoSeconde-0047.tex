% This is part of Un soupçon de mathématique sans être agressif pour autant
% Copyright (c) 2012-2013
%   Laurent Claessens
% See the file fdl-1.3.txt for copying conditions.

\begin{exercice}\label{exoSeconde-0047}

    À partir du graphe ci-dessous, donner graphiquement les solutions de \( f(x)=0\) et de  \( f(x)=1\). Combien en y a-t-il ?
%\newcommand{\CaptionFigLectureGraphnrkEEM}{Dessiner les soutions de \( f(x)=1\).}
%\input{Fig_LectureGraphnrkEEM.pstricks}
    \begin{center}
   \input{Fig_LectureGraphnrkEEM.pstricks}
    \end{center}

    Étudier les variations de cette fonction. Pour quelles valeurs de l'abscisse la fonction est-elle décroissante ?

\corrref{Seconde-0047}
\end{exercice}
