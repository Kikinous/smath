% This is part of Un soupçon de mathématique sans être agressif pour autant
% Copyright (c) 2012
%   Laurent Claessens
% See the file fdl-1.3.txt for copying conditions.

\begin{exercice}\label{exoSeconde-0047}

    À partir de la figure \ref{LabelFigLectureGraphnrkEEM}, donner graphiquement les solutions de \( f(x)=0\) et de  \( f(x)=1\). Combien en y a-t-il ?
\newcommand{\CaptionFigLectureGraphnrkEEM}{Dessiner les soutions de \( f(x)=1\).}
\input{Fig_LectureGraphnrkEEM.pstricks}

Ne pas donner une valeur exacte, mais une valeur approchée en lisant sur une règle.

\corrref{Seconde-0047}
\end{exercice}
