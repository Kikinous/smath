% This is part of Un soupçon de mathématique sans être agressif pour autant
% Copyright (c) 2013
%   Laurent Claessens
% See the file fdl-1.3.txt for copying conditions.

\begin{corrige}{smath-0216}

    \begin{enumerate}
        \item
            La somme de toutes les fréquences doit faire \( 1\), donc celle qui manque est \( 0.2\).


            \begin{center}
        \begin{tabular}[]{|c||c|c|c|c|c|c|}
            \hline
            Chiffre obtenu&1&2&3&4&5&6\\
            \hline\hline
            Brice (effectifs)&9&12&8&7&5&9\\
            \hline
            Marie (fréquence)&$0.24$&$0.12$&$0.2$&$0.17$&$0.08$&$0.19$\\
            \hline
        \end{tabular}
            \end{center}


        \item
            Marie a obtenu \( 12\%\) de \( 2\).
        \item
            \( 7\) fois sur \( 50\) lancers signifie \( 14\%\).
        \item
            \( 0.2\) 
        \item
            \begin{itemize}
                \item Vrai : le \( 3\) est obtenu avec une fréquence \( 0.2\) tandis que le \( 1\) est obtenu avec une fréquence \( 0.24\).
                \item Faux : Brice a obtenu \( 24\%\) tandis que Marie a obtenu \( 12\%\).
            \end{itemize}
        \item
            En ce qui concerne Brice, la moyenne est :
            \begin{equation}
                m=\frac{ 9\times 1+12\times 2+\ldots +9\times 6 }{ 50 }=3.28,
            \end{equation}
            et l'écart-type s'obtient à la calculatrice en insérant les valeurs et les effectifs :
            \begin{equation}
                \sigma\simeq 1.744.
            \end{equation}
            Pour Marie, vu qu'on ne connaît pas les effectifs, mais seulement les fréquences, la moyenne se calcule comme ceci :
            \begin{equation}
                0.24\times 1+0.12\times 2+\ldots +0.19\times 6=3.3.
            \end{equation}
            En ce qui concerne son écart-type, il faudrait des effectifs au lieu des fréquences. Une chose de bien avec l'écart-type est qu'on peut inventer les effectifs qu'on veut pourvu qu'on respecte les fréquences. Nous pouvons par exemple prendre  
            \begin{center}
        \begin{tabular}[]{|c||c|c|c|c|c|c|}
            \hline
            Chiffre obtenu&1&2&3&4&5&6\\
            \hline\hline
            Marie (effectifs)&$24$&$12$&$20$&$17$&$8$&$19$\\
            \hline
        \end{tabular}
            \end{center}
            En utilisant ces chiffres, nous obtenons l'écart-type de Marie : \( \sigma\simeq 1.786\).
    \end{enumerate}

\end{corrige}
