% This is part of Un soupçon de mathématique sans être agressif pour autant
% Copyright (c) 2012
%   Laurent Claessens
% See the file fdl-1.3.txt for copying conditions.

\begin{exercice}\label{exoPremiere-0094}


\begin{wrapfigure}{r}{4.0cm}
   \vspace{-0.5cm}        % à adapter.
   \centering
   \input{Fig_RouletteACaVAA.pstricks}
\end{wrapfigure}
   Nous considérons un jeu dans lequel il faut faire tourner une roulette qui est peinte à \( 75\%\) en rouge et \( 25\%\) en bleu.  Le joueur gagne si la flèche s'arrête sur la zone bleue et perd si elle s'arrête sur la zone rouge. À chaque réussite, le joueur gagne \( 1\) euro, et il a le droit de jouer dix fois.

   \begin{enumerate}
       \item
           Quel est le gain maximum ?
       \item
           Donner les paramètres de la loi binomiale correspondant au gain.
       \item
           Quelle est l'espérance de son gain ?
   \end{enumerate}

\corrref{Premiere-0094}
\end{exercice}
