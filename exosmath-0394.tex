% This is part of Un soupçon de mathématique sans être agressif pour autant
% Copyright (c) 2013
%   Laurent Claessens
% See the file fdl-1.3.txt for copying conditions.

\begin{exercice}\label{exosmath-0394}

    Quel est le coefficient directeur de la droite passant par les points \( A=(2,5)\) et \( B=(4,-1)\) ?
    \begin{multicols}{2}
        \begin{enumerate}
            \item
                \( -3\)
            \item
                \( 2\)
            \item
                \( -1/3\)
            \item
                \( 3\)
        \end{enumerate}
    \end{multicols}

    % Note : la figure de l'exercice suivant (smath-0229) rentre un peu dans celui-ci, mais s'y intègre bien. Si la mise en page de
    % cet exercice change, il faudra bouger la figure.

\corrref{smath-0394}
\end{exercice}
