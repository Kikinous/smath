% This is part of Un soupçon de mathématique sans être agressif pour autant
% Copyright (c) 2014
%   Laurent Claessens
% See the file fdl-1.3.txt for copying conditions.

\begin{exercice}\label{exosmath-0900}

    Le but de cet exercice est de trouver, étant donné un segment \( [AB]\), tous les points \( C\) du plans tels que le triangle \( ABC\) soit rectangle en \( C\).
    \begin{enumerate}
        \item
            Lancer géogebra.
        \item
            Construire un segment \( [AB]\).
        \item
            Créer un point \( K\) hors de \( (AB)\) et tracer la droite \( (AK)\).
        \item
            Comment trouver un point \( C\) sur \( (AK)\) tel que \( \widehat{ACB}=\SI{90}{\degree}\) ?
    \end{enumerate}

    Lorsque cela est fait, écrire le titre «Triangle rectangle» dans le cahier d'exercice. Exprimer les résultats des recherches sous forme d'un dessin et de quelque phrases. 

    Émettre une conjecture à propos de l'ensemble des points \( C\) tels que \( ABC\) soit rectangle en \( C\).

\corrref{smath-0900}
\end{exercice}
