% This is part of Un soupçon de mathématique sans être agressif pour autant
% Copyright (c) 2014
%   Laurent Claessens
% See the file fdl-1.3.txt for copying conditions.


% THÉORÈME DE HAGA POUR L'AP
\begin{feuilleExo}{Théorème de Haga}

%\begin{wrapfigure}{r}{10.cm}
    \begin{center}
\EpsOrPdfincludegraphics[width=10cm]{Haga}
    \end{center}
%\end{wrapfigure}

\paragraph{Un peu de Pythagore}


Nous avons appelé \( a\) la longueur du segment \( [AC]\). Pourquoi \( BC=1-a\) ? Quelle est la valeur de \( a\) ?

\paragraph{Question d'angle}

Notons \( \hat B_1\) et \( \hat B_2\) les deux angles situés au point \( B\) (\( \hat B_1\) est celui du triangle \( ABC\)).
\begin{enumerate}
    \item
        Combien vaut \( \hat B_1+\hat B_2\) ?
    \item
        Combien vaut \( \hat B_1+\hat C\) ?
    \item
        En déduire que \( \hat B_2=\hat C\).
\end{enumerate}
Pourquoi \( \hat B_1=\hat E\) ?

\paragraph{Triangles semblables}

Les triangles \( ABC\) et \( BDE\) sont donc deux triangles ayant les mêmes angles; ils sont donc semblables et donc «proportionnels». De la même façon que pour Thalès, nous pouvons écrire les égalités «grand divisé par petit» :
\begin{equation}
    \frac{ DE }{ BD }=\frac{ AB }{ AC }.
\end{equation}
Reporter les valeurs connues et déduire la valeur de \( x=DE\).

\paragraph{Conclusion}

Sachant la valeur de \( x\), conclure que nous avons bien obtenu un pliage coupant en trois le côté du carré.


\end{feuilleExo}
