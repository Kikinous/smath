% This is part of Un soupçon de mathématique sans être agressif pour autant
% Copyright (c) 2012
%   Laurent Claessens
% See the file fdl-1.3.txt for copying conditions.

\begin{exercice}\label{exoPremiere-0090}

    Une urne contient \( 4\) boules rouges et deux boules noires. Nous tirons (avec remise) trois boules.
    \begin{enumerate}
        \item
            Quelle est la probabilité d'avoir trois noires ?
        \item
            Quelle est la probabilité d'en avoir au moins une rouge ?
        \item
            Quelle est la probabilité d'avoir trois rouges ?
        \item 
            Quelle est la probabilité d'avoir exactement deux rouges ?
    \end{enumerate}

\corrref{Premiere-0090}
\end{exercice}
