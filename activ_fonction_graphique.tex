%This is part of Un soupçon de mathématique sans être agressif pour autant
% Copyright (c) 2012-2013
%   Laurent Claessens
% See the file fdl-1.3.txt for copying conditions.

    Un berger syldave s'entraine pour le championnat national du lancer de chèvre. L'épreuve consiste à lancer une chèvre vers le haut depuis le bord d'une falaise située au bord d'un lac tranquille. La hauteur de la chèvre en fonction du temps par rapport à la surface du lac tranquille est une fonction \( f\) donnée par le graphique suivant.

    \begin{center}
        \input{Fig_WRXbDCo.pstricks}
    \end{center}
    La dernière partie du graphique correspond à la chèvre que l'on remonte rapidement hors de l'eau.
    À partir du graphique :
    \begin{enumerate}
        \item
            À quelle hauteur se trouve la chèvre au moment du lancer ?
        \item
            Pendant combien de temps la chèvre reste à une hauteur supérieure à celle à laquelle elle a été lancée ?
        \item
            À quel moment la chèvre atteint-elle sa hauteur maximale ? Quelle est cette hauteur ?
        \item
            À quelle hauteur se trouve la chèvre après \( 2.5\) secondes de vol ?
        \item
            Résumer toutes ces informations en dressant le tableau de variation de la fonction \( f\).
    \end{enumerate}
