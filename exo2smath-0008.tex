% This is part of Un soupçon de mathématique sans être agressif pour autant
% Copyright (c) 2014
%   Laurent Claessens
% See the file fdl-1.3.txt for copying conditions.

\begin{exercice}\label{exo2smath-0008}

    La figure suivante représente un carré d'une longueur de \SI{10}{\meter}.
\begin{center}
\input{Fig_OBLUooWYQnuB.pstricks}
\end{center}

    \begin{enumerate}
        \item
            Compléter les mesures demandées (éventuellement en fonction de \( \ell\)).
        \item
            Calculer l'aire de la surface grisée.
        \item
            Quelle est l'aire de la surface blanche lorsque \( \ell=\SI{3}{\meter}\) ?
    \end{enumerate}


\corrref{2smath-0008}
\end{exercice}
