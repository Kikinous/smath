% This is part of Un soupçon de mathématique sans être agressif pour autant
% Copyright (c) 2012
%   Laurent Claessens
% See the file fdl-1.3.txt for copying conditions.

\begin{corrige}{Seconde-0078}

    Première chose à dire : il ne suffit pas de montrer que les quatre côtés ont même longueur, parce que les losanges ont quatre côtés de même longueur.

    Une méthode directe sans difficultés majeures serait de 
    \begin{enumerate}
        \item
            Calculer les quatre longueurs et remarquer qu'elles sont les même
        \item
            Montrer que les angles sont droits en utilisant le théorème de Pythagore dans les triangle \( DAC\), \( DAB\), \( ACB\) et \( DCB\).
    \end{enumerate}
    Cette méthode a au moins le mérite de ne pas se poser trop de questions, mais elle entraine une foule de calculs inutiles.

    Une méthode plus rapide est d'utiliser les propriétés connues des quadrilatères.
    \begin{enumerate}
        \item
            D'abord on montre que \( ABCD\) est un parallélogramme en montrant que les diagonales se coupent en leur milieu.
        \item
            Ensuite on montre qu'il y a un angle droit (n'importe lequel). Dans ce cas nous aurons un rectangle parce qu'un parallélogramme contenant un angle droit est un rectangle.
        \item
            Enfin nous prouvons que deux côtés consécutifs ont même longueur.
        \item
            Pour conclure, un rectangle dont deux côtés consécutifs ont même longueur est un carré.
    \end{enumerate}
    Nous allons donner le détail de cette méthode.
    \begin{enumerate}
        \item
            Le milieu de la diagonale \( [AC]\) est le point
            \begin{equation}
                \left( \frac{ 3+2 }{2};\frac{ 5+2 }{2} \right)=\left( \frac{ 5 }{2},\frac{ 7 }{2} \right);
            \end{equation}
            la milieu de la diagonale \( [DB]\) est le point
            \begin{equation}
                \left( \frac{ 4+1 }{2};\frac{ 3+4 }{2} \right)=\left( \frac{ 5 }{2};\frac{ 7 }{2} \right).
            \end{equation}
            Donc c'est gagné pour dire qu'ils se coupent en leur milieu.
        \item
            Montrons que l'angle \( D\) est droit en utilisant (la réciproque de) Pythagore dans le triangle \( ACD\). Nous devons calculer les trois longueurs :
            \begin{subequations}
                \begin{align}
                    DA^2&=(1-3)^2+(4-5)^2=4+1=5     \label{subeqIxHkrm}\\
                    DC^2&=(1-2)^2+(4-2)^2=1+4=5     \label{subeqLnDRXe}\\
                    AC^2&=(2-3)^2+(2-5)^2=1+9=10.
                \end{align}
            \end{subequations}
            Nous avons donc bien la relation \( AC^2=DA^2+DC^2\) qui montre que le triangle \( ACD\) est rectangle en \( D\). Nous sommes dont en présence d'un parallélogramme contenant un angle droit, c'est à dire un rectangle.

        \item
            Comme déjà calculé aux équations \eqref{subeqIxHkrm} et \eqref{subeqLnDRXe}, nous avons \( AD=DC=\sqrt{5}\). Nous avons donc deux côtés consécutifs de même longueur. Le rectangle est donc un carré.
    \end{enumerate}

\end{corrige}
