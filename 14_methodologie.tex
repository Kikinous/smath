% This is part of Un soupçon de mathématique sans être agressif pour autant
% Copyright (c) 2014
%   Laurent Claessens
% See the file fdl-1.3.txt for copying conditions.

% This is part of Un soupçon de mathématique sans être agressif pour autant
% Copyright (c) 2014
%   Laurent Claessens
% See the file fdl-1.3.txt for copying conditions.

%--------------------------------------------------------------------------------------------------------------------------- 
\subsection*{Mesurer sur un dessin}
%---------------------------------------------------------------------------------------------------------------------------

Rechercher dans votre cahier d'exercices l'activité «mesure astronomique». Quelle distance avez-vous trouvé entre la Terre et la comète ? Mettons tous les résultats en commun et comparons.

Est-ce que nous pouvons nous fier à un dessin ?

%--------------------------------------------------------------------------------------------------------------------------- 
\subsection*{Recherche d'exemples}
%---------------------------------------------------------------------------------------------------------------------------

Si \( x\) est un nombre positif, est-il vrai que \( x\times x\) est plus grand que \( x\) ? (exemple : si \( x=5\) alors \( x\times x=25\), et c'est effectivement plus grand que \( 5\))


%+++++++++++++++++++++++++++++++++++++++++++++++++++++++++++++++++++++++++++++++++++++++++++++++++++++++++++++++++++++++++++ 
\section{Conjecture}
%+++++++++++++++++++++++++++++++++++++++++++++++++++++++++++++++++++++++++++++++++++++++++++++++++++++++++++++++++++++++++++

\begin{Aretenir}
    Un dessin ne constitue pas une preuve. Il peut cependant donner une idée du résultat.
\end{Aretenir}

Dans le cas de l'activité «distance astronomique», les dessins et les mesures faites sur les dessins permettent de {\bf conjecturer} que la distance entre la comète et la Terre est d'environ \( 500\) millions de kilomètres.

Dans le cas de la recherche d'exemples, nous avons pu conjecturer que \( x\times x >x\). Mais c'est faux. Par exemple avec \( x=0.1\) nous avons \( x\times x=0.1\times 0.1=0.01\).

\begin{Aretenir}
    Une conjecture, même appuyée par beaucoup d'exemples, peut être fausse.
\end{Aretenir}

%+++++++++++++++++++++++++++++++++++++++++++++++++++++++++++++++++++++++++++++++++++++++++++++++++++++++++++++++++++++++++++ 
\section{Réciproque}
%+++++++++++++++++++++++++++++++++++++++++++++++++++++++++++++++++++++++++++++++++++++++++++++++++++++++++++++++++++++++++++

En mathématique, les énoncés sont souvent écrits sous la forme « {\bf Si} \boxed{condition} {\bf alors} \boxed{conséquence}».

\begin{definition}
    La \defe{réciproque}{réciproque} d'un énoncé s'obtient en inversant la condition et la conséquence.
\end{definition}

\begin{example}
    \begin{description}
        \item[Énoncé] «{\bf Si} le dernier chiffre d'un nombre est «\( 6\)» {\bf alors} il est divisible par \( 2\).» 
        \item[Réciproque] «{\bf Si} un nombre est divisible par \( 2\) {\bf alors} son dernier chiffre est «\( 6\)» 
    \end{description}
    Note : même si un énoncé est vrai, sa réciproque peut être fausse.
\end{example}
