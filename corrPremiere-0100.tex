% This is part of Un soupçon de mathématique sans être agressif pour autant
% Copyright (c) 2012
%   Laurent Claessens
% See the file fdl-1.3.txt for copying conditions.

\begin{corrige}{Premiere-0100}

La probabilité d'obtenir au moins un six en lançant \( 4\) fois le dé est «l'opposé» de la probabilité de n'obtenir aucun six en lançant quatre fois le dé :
\begin{equation}
    P(X\geq 1)=1-P(X=0)
\end{equation}
où \( X\) est une variable aléatoire suivant une loi binomiale de paramètres \( p=1/6\) et \( n=4\). Avec cela \( P(X=0)=\left( \frac{ 5 }{ 6 } \right)^4\) et
\begin{equation}
    P(X\geq 1)\approx 0.51.
\end{equation}

Par ailleurs, la probabilité d'obtenir au moins deux fois le \( 6\) en \( 8\) lancers est
\begin{equation}
    1-P(X=0)-P(X=1)
\end{equation}
où \( X\) est une binomiale de paramètres \( n=8\), \( p=1/6\). Le calcul est rapide :
\begin{equation}
    1-\left( \frac{ 5 }{ 6 } \right)^8-8\times \left( \frac{ 5 }{ 6 } \right)^7\left( \frac{1}{ 6 } \right)\approx 0.39.
\end{equation}

La première est donc plus grande.

\end{corrige}
