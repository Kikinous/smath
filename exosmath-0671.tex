% This is part of Un soupçon de mathématique sans être agressif pour autant
% Copyright (c) 2014
%   Laurent Claessens
% See the file fdl-1.3.txt for copying conditions.

\begin{exercice}\label{exosmath-0671}

    À propos de fonctions affines.
    \begin{enumerate}
        \item
            Donner une fonction affine \( f\) dont le graphe passe par les points \( (0;0)\) et \( (6;3)\).
        \item
            Est-ce que les points \( A(20;10)\), \( B(2;4)\) et \( C(0;5)\) sont sur le graphe de la fonction \( f\) ?
        \item
            Tracer dans un repère le graphe des fonctions \( g(x)=2x+1\) et \( h(x)=3x-4\).
        \item
            Résoudre l'équation \( g(x)=h(x)\).
        \item
            En déduire les coordonnées du point d'intersection entre les droites représentatives de \( h\) et de \( g\) ?
    \end{enumerate}

\corrref{smath-0671}
\end{exercice}
