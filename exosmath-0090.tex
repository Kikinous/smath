% This is part of Un soupçon de mathématique sans être agressif pour autant
% Copyright (c) 2012
%   Laurent Claessens
% See the file fdl-1.3.txt for copying conditions.

\begin{exercice}\label{exosmath-0090}

    Deux hôpitaux \( A\) et \( B\) testent indépendamment deux médicaments nommés \( X\) et \( Y\) traitant la même maladie.
    
    Le premier hôpital traite \( 500\) patients avec le médicament \( X\) et \( 500\) avec le médicament \( Y\). Les résultats sont que \( 90\%\) de patients traités avec le médicament \( A\) guérissent alors que \( 92\%\) des patients traités par le médicament \( Y\) guérissent.

    Le second hôpital a des résultats moins encourageants. Sur \( 300\) malades ayant reçus le traitement \( X\), \( 54\%\) ont guéri et sur \( 700\) ayant reçu le traitement \( Y\), \( 55\%\) ont guéri.

    Nous observons que le taux de réussite du médicament \( X\) est meilleur dans les deux hôpitaux. Attention : surprise !

    \begin{enumerate}
        \item
            Combien de personnes au total ont reçu le traitement $X$ ? Parmi elles combien sont guéries ? Quelle est alors la proportion de guérison dans parmi les personnes ayant reçu le médicament \( Y\) ?
        \item
            Mêmes questions pour le traitement \( Y\).
        \item
            Conclusion : quel médicament a l'air d'être le plus prometteur ?
    \end{enumerate}

\corrref{smath-0090}
\end{exercice}
