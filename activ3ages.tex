% This is part of Un soupçon de mathématique sans être agressif pour autant
% Copyright (c) 2015
%   Laurent Claessens
% See the file fdl-1.3.txt for copying conditions.

%--------------------------------------------------------------------------------------------------------------------------- 
\subsection*{Activité : le double de l'âge}
%---------------------------------------------------------------------------------------------------------------------------

\begin{description}
    \item[Problème 1] Sylvia a sept ans de plus que sa sœur Rose. Dans $10$ ans, Sylvia aura le double de l'âge de Rose. Quel est l'âge de Rose ? Appeler $x$ l'âge de Rose.

    \item[Problème 2] En $2000$, Paul avait $10$ ans et Louis $17$ ans. En quelle année, l'âge de Louis a-t-il été le double de l'âge de Paul ? Appeler $x$ la différence entre cette année et $2000$.
\end{description}

\begin{enumerate}
    \item
        Mettre chacun des deux problèmes en équation.
    \item
        Résoudre ces équations
    \item
        En déduire la solution de chacun des problèmes
\end{enumerate}
