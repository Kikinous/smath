% This is part of Un soupçon de mathématique sans être agressif pour autant
% Copyright (c) 2012
%   Laurent Claessens
% See the file fdl-1.3.txt for copying conditions.

\begin{corrige}{Premiere-0042}

    Tout développer consiste à faire la (double) distribution
    \begin{equation}
        (x-8)(-2x-2)=-2x^2-2x+16x+16
    \end{equation}
    et le produit remarquable
    \begin{equation}
        (x+4)^2=x^2+8x+16.
    \end{equation}
    Nous avons donc l'équation
    \begin{equation}
        -2x^2-2x+16x+16=x^2+8x+16.
    \end{equation}
    En mettant tout dans le membre de gauche nous trouvons l'équation
    \begin{equation}
        -3x^2+6x=0.
    \end{equation}

    La résolution de cette équation demande de mettre \( x\) en évidence :
    \begin{equation}
        x(-3x+6)=0.
    \end{equation}
    En appliquant la règle du produit nul, nous savons que soit \( x=0\) (c'est une première solution de l'équation de départ), soit
    \begin{equation}
        -3x+6=0,
    \end{equation}
    et dans ce cas c'est \( x=2\).

    Les deux solutions d l'équation de départ sont \( x=0\) et \( x=2\).
    


\end{corrige}
