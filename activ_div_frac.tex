% This is part of Un soupçon de mathématique sans être agressif pour autant
% Copyright (c) 2014-2015
%   Laurent Claessens
% See the file fdl-1.3.txt for copying conditions.

%--------------------------------------------------------------------------------------------------------------------------- 
\subsection*{Activité : division par une fraction}
%---------------------------------------------------------------------------------------------------------------------------

\begin{enumerate}
    \item
        Il faut partager \( 45\) œufs en chocolat en \( 5\) personnes. Combien d'œufs par personne ? Écrire le calcul effectué sous forme de fraction.
    \item
        Nous avons \( 45\) œufs en chocolat avec lesquels il faut remplir des sachets. Chaque sachet peut contenir \( 5\) œufs. Combien de sachets faut-il ? Écrire le calcul effectué sous forme de fraction.
    \item
        Nous avons \SI{15}{\liter} d'eau pour remplir des bouteilles de $\SI{\dfrac{ 1 }{ 2 }}{\liter}$. Combien de bouteilles faudra-t-il ?
    \item
        Combien de fois \( 5\) rentre-t-il dans \( 15\) ?
    \item
        Combien de fois \( 1/6\) rentre-t-il dans \( 2/3\) ?

\begin{center}
   \input{Fig_MXHCooEnhHYeZ.pstricks}
\end{center}
\begin{center}
   \input{Fig_MXHCooEnhHYeO.pstricks}
\end{center}



    \item
        Combien peut-on faire de huitièmes de pizzas à partir de trois quarts de pizza ? Pour vous aider, voici un dessin de trois quart de pizzas et un dessin d'une pizza coupée en \( 8\) :
        \begin{center}
           \input{Fig_PJUIooZuEnPkZ.pstricks}
           \input{Fig_PJUIooZuEnPkO.pstricks}
        \end{center}
    \item
        Calculer
        \begin{equation}
            \dfrac{ 1 }{ 2 }\div\frac{1}{ 8 }.
        \end{equation}
\end{enumerate}
