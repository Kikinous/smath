% This is part of Un soupçon de mathématique sans être agressif pour autant
% Copyright (c) 2014
%   Laurent Claessens
% See the file fdl-1.3.txt for copying conditions.

% This is part of Un soupçon de mathématique sans être agressif pour autant
% Copyright (c) 2014
%   Laurent Claessens
% See the file fdl-1.3.txt for copying conditions.


\begin{wrapfigure}{r}{7.0cm}
   \vspace{-2cm}        % à adapter.
   \centering
   \input{Fig_SAZooPjiyOV.pstricks}
\end{wrapfigure}


Découper sur une feuille dix carrés dont les mesures des côtés sont entières et valent de \SI{1}{\centi\meter} à \SI{10}{\centi\meter}. Indiquer à l'intérieur de chacun d'eux son aire en \si{\centi\meter\squared}. Vous pouvez en découper plusieurs de la même mesure.

Essayer de les assembler de façon à créer un triangle comme sur la figure ci-contre.

Lorsqu'un triangle est formé, aller au tableau écrire
\begin{itemize}
    \item les mesures.
    \item les aires.
    \item la nature du triangle.
\end{itemize}

De \cite{NRHooXFvgpp4}

\begin{definition}
    Dans un triangle rectangle, l\defe{hypoténuse}{hypoténuse} est le côté opposé à l'angle droit.
\end{definition}
\begin{remark}
    L'hypoténuse est le côté le plus long du triangle rectangle.
\end{remark}

JTQooDUZpht

\begin{theorem}[de Pythagore]
    Si un triangle est rectangle, alors le carré de la longueur de son hypoténuse est égal à la somme des carrés des longueurs des deux autres longueurs.
\end{theorem}
<++>

