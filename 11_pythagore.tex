% This is part of Un soupçon de mathématique sans être agressif pour autant
% Copyright (c) 2014
%   Laurent Claessens
% See the file fdl-1.3.txt for copying conditions.

% This is part of Un soupçon de mathématique sans être agressif pour autant
% Copyright (c) 2014
%   Laurent Claessens
% See the file fdl-1.3.txt for copying conditions.


\begin{wrapfigure}{r}{7.0cm}
   \vspace{-2cm}        % à adapter.
   \centering
   \input{Fig_SAZooPjiyOV.pstricks}
\end{wrapfigure}


Découper sur une feuille dix carrés dont les mesures des côtés sont entières et valent de \SI{1}{\centi\meter} à \SI{10}{\centi\meter}. Indiquer à l'intérieur de chacun d'eux son aire en \si{\centi\meter\squared}. Vous pouvez en découper plusieurs de la même mesure.

Essayer de les assembler de façon à créer un triangle comme sur la figure ci-contre.

Lorsqu'un triangle est formé, aller au tableau écrire
\begin{itemize}
    \item les mesures.
    \item les aires.
    \item la nature du triangle.
\end{itemize}

De \cite{NRHooXFvgpp4}

\begin{definition}
    Dans un triangle rectangle, l'\defe{hypoténuse}{hypoténuse} est le côté opposé à l'angle droit.
\end{definition}
\begin{remark}
    L'hypoténuse est le côté le plus long du triangle rectangle.
\end{remark}

\begin{example}
    Sur ce triangle, c'est le côté \( BC\).
    \begin{center}
        \input{Fig_JTQooDUZpht.pstricks}
    \end{center}
\end{example}

\begin{theorem}[de Pythagore]
    Un triangle est rectangle si et seulement si le carré de la longueur de son plus long côté est égal à la somme des carrés des deux autres longueurs.
\end{theorem}

\begin{remark}
    L'expression «si et seulement si» signifie que l'énoncé et sa réciproque sont vrais.
    \begin{enumerate}
        \item
            {\bf Si} un triangle est rectangle, {\bf alors} le carré de la plus grande longueur est la somme des deux autres carrés.
        \item
            {\bf Si} le carré de la plus grande longueur est égale à la somme des deux autres carrés, {\bf alors} le triangle est rectangle.
    \end{enumerate}
\end{remark}

\begin{proof}

    Nous considérons un triangle rectangle de côtés \( a\), \( b\) et \( c\) :
\begin{center}
   \input{Fig_YDAooMNHhCN2.pstricks}
\end{center}

Dans ce triangle, nous remarquons que \( \alpha+\beta=\SI{90}{\degree}\).

Ensuite nous disposons quatre de ces triangles pour constituer un carré de cette façon-ci :
\begin{center}
   \input{Fig_YDAooMNHhCN0.pstricks}
\end{center}


\begin{enumerate}
    \item
        Le quadrilatère \( MNOP\) est un losange parce qu'il a quatre côtés de même longueurs.
    \item
        Par conséquent \( \widehat{PMN}=\SI{90}{\degree}\) et \( MNOP\) est un carré.
    \item
        L'aire de \( MNOP\) vaut \( c^2\).
\end{enumerate}

Maintenant nous complétons les triangles \( PDO\) et \( OCN\) pour en faire des rectangles.

\begin{center}
   \input{Fig_YDAooMNHhCN1.pstricks}
\end{center}

\begin{enumerate}
    \item
        Le carré \( RBNL\) a pour aire \( a^2\).
    \item
        Le carré \( ARKP\) a pour aire \( b^2\).
    \item
        La somme des deux correspond au carré \( ABCD\) auquel nous avons enlevé quatre triangles, c'est à dire à \( MNOP\).
    \item
        Nous avons donc l'égalité
        \begin{equation}
            \text{aire}(RBNL)+\text{aire}(ARKP)=\text{aire}(MNOP),
        \end{equation}
        qui signifie \( a^2+b^2=c^2\).
\end{enumerate}

\end{proof}

\begin{example}
    Un triangle a pour côtés \( AB=9\), \( BC=12\) et \( AC=15\). Est-il rectangle ?
    \begin{itemize}
        \item Son plus long côté est \( AC=15\). Le carré de cette longueur est \( 225\).
        \item La somme des carrés des deux autres côtés est : \( 9^2+12^2=81+144=225\).
    \end{itemize}
    Par le théorème de Pythagore, l'égalité \( AC^2=AB^2+BC^2\) implique que \( ABC\) est rectangle en \( B\).
\end{example}

\begin{example}
    Quelle est la longueur de l'hypoténuse du triangle rectangle suivant ?
    \begin{center}
        \input{Fig_NARooFiHuAy.pstricks}
    \end{center}
    Vu que ce triangle est rectangle, le théorème de Pythagore implique que
    \begin{equation}
        AB^2+BC^2=AC^2.
    \end{equation}
    Nous connaissons \( AB=8\) et \( BC=6\). L'égalité de Pythagore est :
    \begin{equation}
        8^2+6^2=AC^2.
    \end{equation}
    Donc \( AC^2=64+36=100\). Vu que \( AC^2=100\), nous déduisons \( AC=10\).
\end{example}
