% This is part of Un soupçon de mathématique sans être agressif pour autant
% Copyright (c) 2014
%   Laurent Claessens
% See the file fdl-1.3.txt for copying conditions.

% This is part of Un soupçon de mathématique sans être agressif pour autant
% Copyright (c) 2014
%   Laurent Claessens
% See the file fdl-1.3.txt for copying conditions.


\begin{wrapfigure}{r}{7.0cm}
   \vspace{-2cm}        % à adapter.
   \centering
   \input{Fig_SAZooPjiyOV.pstricks}
\end{wrapfigure}


Découper sur une feuille dix carrés dont les mesures des côtés sont entières et valent de \SI{1}{\centi\meter} à \SI{10}{\centi\meter}. Indiquer à l'intérieur de chacun d'eux son aire en \si{\centi\meter\squared}. Vous pouvez en découper plusieurs de la même mesure.

Essayer de les assembler de façon à créer un triangle comme sur la figure ci-contre.

Lorsqu'un triangle est formé, aller au tableau écrire
\begin{itemize}
    \item les mesures.
    \item les aires.
    \item la nature du triangle.
\end{itemize}

De \cite{NRHooXFvgpp4}

\begin{definition}
    Dans un triangle rectangle, l\defe{hypoténuse}{hypoténuse} est le côté opposé à l'angle droit.
\end{definition}
\begin{remark}
    L'hypoténuse est le côté le plus long du triangle rectangle.
\end{remark}

\begin{example}
    Sur ce triangle, c'est le côté \( BC\).
    \begin{center}
        \input{Fig_JTQooDUZpht.pstricks}
    \end{center}
\end{example}

\begin{theorem}[de Pythagore]
    Un triangle est rectangle si et seulement si le carré de la longueur de son plus long côté est égal à la somme des carrés des deux autres longueurs.
\end{theorem}

\begin{remark}
    L'expression «si et seulement si» signifie que l'énoncé et sa réciproque sont vrais.
    \begin{enumerate}
        \item
            {\bf Si} un triangle est rectangle, {\bf alors} le carré de la plus grande longueur est la somme des deux autres carrés.
        \item
            {\bf Si} le carré de la plus grande longueur est égale à la somme des deux autres carrés, {\bf alors} le triangle est rectangle.
    \end{enumerate}
\end{remark}

\begin{example}
    Un triangle dont les côtés sont \( AB=9\), \( BC=12\) et \( AC=15\) est rectangle parce que \( 9^2+12^2=15^2\). Il est rectangle en \( B\).
\end{example}

\begin{example}
    Quelle est la longueur de l'hypoténuse du triangle rectangle suivant ?
    \begin{center}
        \input{Fig_NARooFiHuAy.pstricks}
    \end{center}
    La somme des carrés des deux côtés adjacents à l'angle droit est :
    \begin{equation}
        8^2+6^2=64+36=100.
    \end{equation}
    Donc \( AC^2=100\), c'est à dire \( AC=10\).
\end{example}
