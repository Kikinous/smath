% This is part of Un soupçon de mathématique sans être agressif pour autant
% Copyright (c) 2012
%   Laurent Claessens
% See the file fdl-1.3.txt for copying conditions.

%+++++++++++++++++++++++++++++++++++++++++++++++++++++++++++++++++++++++++++++++++++++++++++++++++++++++++++++++++++++++++++
\section{Introduction}
%+++++++++++++++++++++++++++++++++++++++++++++++++++++++++++++++++++++++++++++++++++++++++++++++++++++++++++++++++++++++++++

La figure \ref{LabelFigDesSections} montre un cube. Êtes-vous capables de donner la nature des surfaces coloriées ?
\newcommand{\CaptionFigDesSections}{Exercice de vision dans l'espace.}
\input{Fig_DesSections.pstricks}
Le triangle vert est isocèle et rectangle parce que deux de ses côtés sont des arrêtes du cube. Le triangle rouge est plus troublant, mais il est équilatéral : ses trois côtés sont des diagonales des faces du cube. Notez que \emph{sur le dessin}, les trois côtés ont des longueur différentes.

%+++++++++++++++++++++++++++++++++++++++++++++++++++++++++++++++++++++++++++++++++++++++++++++++++++++++++++++++++++++++++++
\section{Les règles de la perspective cavalière}
%+++++++++++++++++++++++++++++++++++++++++++++++++++++++++++++++++++++++++++++++++++++++++++++++++++++++++++++++++++++++++++

Lorsque nous dessinons en trois dimension, nous prenons les conventions suivantes qui définissent la \defe{\wikipedia{fr}{Perspective_cavalière}{perspective cavalière}}{perspective!cavalière}.

\begin{Aretenir}
    D'abord nous choisissons
    \begin{enumerate}
        \item
            Nous choisissons un \defe{angle de fuite}{angle!de fuite} \( \alpha\) qui sera entre \unit{30}{\degree} et \( \unit{45}{\degree}\) avec l'horizontale\footnote{Sur une feuille à carreaux, le plus simple est de prendre \unit{45}{\degree}.}.
        \item
            Un \defe{coefficient de réduction}{coefficient de réduction} que nous noterons \( k\) et qui sera compris entre \( 0\) et \( 1\).
    \end{enumerate}

    Ensuite nous prenons les correspondances suivantes entre la réalité et le dessin :
    \begin{center}
        \begin{tabular}{|p{7.5cm}|p{7.5cm}|}
            \hline
            {\bf dans la réalité}&{\bf sur le dessin}\\
            \hline\hline
            Segment caché  & Segment pointillé\\
            \hline
            Segment parallèle à la feuille de dessin & Segment représenté en vraie grandeur\\
            \hline
            Segment perpendiculaire à la feuille de dessin & Segment faisant un angle \( \alpha\) avec l'horizontale.\\
            \hline
            Une arrête de longueur \( l\) perpendiculaire à la feuille de dessin & Une arrête de longueur \( k\times l\) faisant un angle \( \alpha\) avec l'horizontale.\\
            \hline
        \end{tabular}
        % Note : c'est ce genre de tableaux qui ne fonctionnent pas avec le paquet pdfsync.
        % Il faut oublier pdfsync et compiler avec 'pdflatex -synctex=1'
    \end{center}
\end{Aretenir}

\begin{propriete}
    La perspective cavalière respecte les conditions suivantes.
    \begin{enumerate}
        \item
             Deux segments parallèles dans la réalité sont représentés par deux segments parallèles sur le dessin.
         \item
             Trois points alignés dans la réalité sont représentés par trois points alignés sur le dessin.
         \item
             Si \( M\) est le milieu du segment \( [AB]\) dans la réalité, alors \( M\) est le milieu du segment \( [AB]\) dans la réalité.
         \item
             La perspective cavalière respecte les proportions. C'est à dire que si le segment \( [AB]\) est \( p\) fois plus grand que le segment \( [CD]\) dans la réalité, alors il sera \( p\) fois plus grand sur le dessin.
    \end{enumerate}
\end{propriete}

Le cube de la figure \ref{LabelFigCubeLFZuiW} est dessiné avec \( \alpha=\unit{45}{\degree}\) et \( k=0.5\). Notez que les côtés parallèles restent parallèles.
\newcommand{\CaptionFigCubeLFZuiW}{Les segments perpendiculaires à la feuille sont de longueur moitié des autres.}
\input{Fig_CubeLFZuiW.pstricks}

La perspective cavalière n'est pas parfaite; il est aisé de créer des illusions d'optique comme celle de la figure \ref{LabelFigIllusionNHwEtp}. % From file IllusionNHwEtp
\newcommand{\CaptionFigIllusionNHwEtp}{Une petite illusion d'optique facile.}
\input{Fig_IllusionNHwEtp.pstricks}

%+++++++++++++++++++++++++++++++++++++++++++++++++++++++++++++++++++++++++++++++++++++++++++++++++++++++++++++++++++++++++++
\section{Dessiner en vraie grandeur}
%+++++++++++++++++++++++++++++++++++++++++++++++++++++++++++++++++++++++++++++++++++++++++++++++++++++++++++++++++++++++++++

La perspective cavalière n'est pas parfaite; aucune perspective n'est parfaite. Étant donné que les surfaces sont déformées, nous voudrions être capables de dessiner certaines surfaces sans déformations. C'est ce que nous appelons \emph{dessiner en vraie grandeur}.


\begin{Aretenir}
    Lorsqu'on demande de dessiner une surface en vraie grandeur, voici les mouvements «de base» que vous êtes autorisés à effectuer \emph{dans un plan}.
    \begin{enumerate}
        \item
            Tracer des segments de longueur \emph{entières} !
        \item
            Tracer des cercles dont le centre et un point sont donnés.
        \item
            Tracer les droites dont deux points sont donnés.
        \item
            Considérer les intersections des droites et cercles.
    \end{enumerate}
    À partir ce ces mouvements, il est possible de tracer de nombreuses choses, dont par exemple
    \begin{enumerate}
        \item
            Le milieu du segment \( [AB]\) lorsque les points \( A\) et \( B\) sont donnés.
        \item
            Si la droite \( d\) et le point \( I\) plan sont donnés, la perpendiculaire à \( d\) passant par \( I\).
        \item
            Si les points \( d\) du plan et le point \( I\) sont donnés, la parallèle à \( d\) passant par \( I\).
    \end{enumerate}
\end{Aretenir}

\begin{proof}
    Nous allons donner les constructions.
    \begin{enumerate}
        \item
            Soient \( A\) et \( B\) données. À construire : la médiatrice du segment \( [AB]\). Avec cela, nous connaîtrons le milieu. Le technique est simple : il suffit de tracer deux cercles de même rayons \( r\) centrés en \( A\) et en \( B\). Les points d'intersection \( I\) et \( J\) donnent la médiatrice. En effet, par construction \( I\) est équidistant de \( A\) et \( B\) parce que \( I\) est à la fois sur le cercle de rayon \( r\) autour de \( A\) et sur le cercle de rayon \( r\) autour de \( B\). Or nous savons que les points équidistants de \( A\) et \( B\) sont sur la médiatrice du segment \( [AB]\).

            Pour la même raison, le point \( J\) est également sur la médiatrice. Cela fait que \( (IJ)\) coupe \( [AB]\) en son milieu.

Cette construction est illustrée sur la figure \ref{LabelFigIsoceleVdviOE}. % From file IsoceleVdviOE
\newcommand{\CaptionFigIsoceleVdviOE}{Le triangle \( AJB\) est isocèle.}
\input{Fig_IsoceleVdviOE.pstricks}

        \item

            Soient donnés la droite \( d\) et le point \( I\). Nous construisons un cercle centré en \( I\) et nous nommons \( A\) et $B$ les points d'intersection entre ce cercle et la droite \( A\). Par construction le triangle \( AIB\) est isocèle. Donc \( I\) est sur la médiatrice du segment \( [AB]\). La construction précédente permet de construire cette médiatrice qui sera alors perpendiculaire à \( d\) passant par \( I\).

Un dessin de la situation est proposé à la figure \ref{LabelFigPerpSegqrbMBZ}. % From file PerpSegqrbMBZ
\newcommand{\CaptionFigPerpSegqrbMBZ}{Le triangle \( AIB\) est isocèle et la droite rouge est la médiatrice de \( [AB]\).}
\input{Fig_PerpSegqrbMBZ.pstricks}

    \item
        
        Si la droite \( d\) et le point \( I\) sont donnés, pour construire la parallèle à \( d\) passant par \( I\), il suffit de construire la perpendiculaire \( d'\) à \( d\) passant par \( I\) et ensuite la perpendiculaire à \( d'\) passant par \( I\).

    \end{enumerate}
\end{proof}

%+++++++++++++++++++++++++++++++++++++++++++++++++++++++++++++++++++++++++++++++++++++++++++++++++++++++++++++++++++++++++++
\section{Droites et plans dans l'espace}
%+++++++++++++++++++++++++++++++++++++++++++++++++++++++++++++++++++++++++++++++++++++++++++++++++++++++++++++++++++++++++++

%---------------------------------------------------------------------------------------------------------------------------
\subsection{Deux droites}
%---------------------------------------------------------------------------------------------------------------------------

Deux droites peuvent être soit dans un même plan, soit ne pas être dans le même plan.
\begin{Aretenir}
    Si deux droites non confondues sont contenues dans un même plan, alors soit elles sont parallèles, soit elles sont sécantes. Dans ce dernier cas, elles se coupent en \emph{un seul} point.

    Deux droites sécantes sont toujours coplanaires.
\end{Aretenir}
Des exemples sont donnés à al figure \ref{LabelFigPositionsDroitesbnYIsH}. % From file PositionsDroitesbnYIsH
\newcommand{\CaptionFigPositionsDroitesbnYIsH}{Droites coplanaires ou non.}
\input{Fig_PositionsDroitesbnYIsH.pstricks}

%See also the subfigure \ref{LabelFigPositionsDroitesbnYIsHssLabelSubFigPositionsDroitesbnYIsH0}
%See also the subfigure \ref{LabelFigPositionsDroitesbnYIsHssLabelSubFigPositionsDroitesbnYIsH1}
%See also the subfigure \ref{LabelFigPositionsDroitesbnYIsHssLabelSubFigPositionsDroitesbnYIsH2}
Notez en particulier la figure \ref{LabelFigPositionsDroitesbnYIsHssLabelSubFigPositionsDroitesbnYIsH3} sur laquelle les droites \( (FH)\) et \( (AC)\) ne sont pas sécantes !



%+++++++++++++++++++++++++++++++++++++++++++++++++++++++++++++++++++++++++++++++++++++++++++++++++++++++++++++++++++++++++++
\section{Exercices sur la géométrie dans l'espace}
%+++++++++++++++++++++++++++++++++++++++++++++++++++++++++++++++++++++++++++++++++++++++++++++++++++++++++++++++++++++++++++

\Exo{Seconde-0088}
\Exo{Seconde-0087}
\Exo{Seconde-0090}
\Exo{Seconde-0091}
\Exo{Seconde-0092}
\Exo{Seconde-0094}
\Exo{Seconde-0093}
\Exo{Seconde-0089}
\Exo{Seconde-0095}
\Exo{Seconde-0096}
\Exo{Seconde-0097}
\Exo{smath-0009}
