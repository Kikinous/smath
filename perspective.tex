% This is part of Un soupçon de mathématique sans être agressif pour autant
% Copyright (c) 2012
%   Laurent Claessens
% See the file fdl-1.3.txt for copying conditions.

%+++++++++++++++++++++++++++++++++++++++++++++++++++++++++++++++++++++++++++++++++++++++++++++++++++++++++++++++++++++++++++
\section{Les règles de la perspective cavalière}
%+++++++++++++++++++++++++++++++++++++++++++++++++++++++++++++++++++++++++++++++++++++++++++++++++++++++++++++++++++++++++++



Lorsque nous dessinons en trois dimension, nous prenons les conventions suivantes qui définissent la \defe{perspective cavalière}{perspective!cavalière}.

\begin{Aretenir}
    D'abord nous choisissons
    \begin{enumerate}
        \item
            Nous choisissons un \defe{angle de fuite}{angle!de fuite} \( \alpha\) qui sera entre \unit{30}{\degree} et \( \unit{45}{\degree}\) avec l'horizontale\footnote{Sur une feuille à carreaux, le plus simple est de prendre \unit{45}{\degree}.}.
        \item
            Un \defe{coefficient de réduction}{coefficient de réduction} que nous noterons \( k\) et qui sera compris entre \( 0\) et \( 1\).
    \end{enumerate}

    Ensuite nous prenons les correspondances suivantes entre la réalité et le dessin :
    \begin{center}
        \begin{tabular}{|p{2cm}|p{2cm}|}
            \hline
            {\bf dans la réalité}&{\bf sur le dessin}\\
            \hline\hline
            Segment caché  & Segment pointillé\\
            \hline
            Segment parallèle à la feuille de dessin & segment représenté en vraie grandeur\\
            \hline
            Segment perpendiculaire à la feuille de dessin & segment faisant un angle \( \alpha\) avec l'horizontale.\\
            \hline
            Une arrête de longueur \( l\) perpendiculaire à la feuille de dessin & Une arrête de longueur \( k\times l\) faisant un angle \( \alpha\) avec l'horizontale.\\
            \hline
        \end{tabular}
    \end{center}
\end{Aretenir}

Et quoi encore ?

\begin{center}
     \begin{tabular}{ | l | l | l | p{5cm} |}
     \hline
     Day & Min Temp & Max Temp & Summary \\ \hline
     Monday & 11C & 22C & A clear day with lots of sunshine.  
     However, the strong breeze will bring down the temperatures. \\ \hline
     Tuesday & 9C & 19C & Cloudy with rain, across many northern regions. Clear spells
     across most of Scotland and Northern Ireland,
     but rain reaching the far northwest. \\ \hline
     Wednesday & 10C & 21C & Rain will still linger for the morning.
     Conditions will improve by early afternoon and continue
     throughout the evening. \\
     \hline
     \end{tabular}
 \end{center}

\begin{center}
     \begin{tabularx}{\linewidth}{ | l | l | l | X |}
     \hline
     Day & Min Temp & Max Temp & Summary \\ \hline
     Monday & 11C & 22C & A clear day with lots of sunshine.
     However, the strong breeze will bring down the temperatures. \\ \hline
     Tuesday & 9C & 19C & Cloudy with rain, across many northern regions. Clear spells
     across most of Scotland and Northern Ireland,
     but rain reaching the far northwest. \\ \hline
     Wednesday & 10C & 21C & Rain will still linger for the morning.
     Conditions will improve by early afternoon and continue
     throughout the evening. \\
     \hline
     \end{tabularx}
 \end{center}

\begin{propriete}
    La perspective cavalière respecte les conditions suivantes.
    \begin{enumerate}
        \item
             Deux segments parallèles dans la réalité sont représentés par deux segments parallèles sur le dessin.
         \item
             Trois points alignés dans la réalité sont représentés par trois points alignés sur le dessin.
         \item
             Si \( M\) est le milieu du segment \( [AB]\) dans la réalité, alors \( M\) est le milieu du segment \( [AB]\) dans la réalité.
         \item
             La perspective cavalière respecte les proportions. C'est à dire que si le segment \( [AB]\) est \( p\) fois plus grand que le segment \( [CD]\) dans la réalité, alors il sera \( p\) fois plus grand sur le dessin.
    \end{enumerate}
\end{propriete}
