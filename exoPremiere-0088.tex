% This is part of Un soupçon de mathématique sans être agressif pour autant
% Copyright (c) 2012
%   Laurent Claessens
% See the file fdl-1.3.txt for copying conditions.

\begin{exercice}\label{exoPremiere-0088}

Un commerçant a oublié sa calculatrice et n'est donc pas capable de rendre correctement la monnaie à ses clients. On estime qu'il se trompe trois fois sur quatre. Heureusement, ses clients ne sont pas non plus capables de calculer et ne se rendent compte qu'il y a une erreur que une fois sur quatre.

En bref, trois clients sur \( 16\) se plaignent.
\begin{enumerate}
    \item
        Est-ce que vous comprenez d'où vient le \( 3/16\) ?
    \item
        Le commerçant sert \( 50\) clients; quelle est la probabilité qu'il y ait au moins une plainte ? Indice : calculer la probabilité qu'il n'y en ait aucune.
    \item
        Sur les \( 50\) clients, quelle est la probabilité qu'au moins la moitié se plaignent ?
    \item
        Quelle est la probabilité que moins de \( 5\) clients ne se plaignent ?
\end{enumerate}

\corrref{Premiere-0088}
\end{exercice}
