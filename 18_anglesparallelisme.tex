% This is part of Un soupçon de mathématique sans être agressif pour autant
% Copyright (c) 2014-2015
%   Laurent Claessens
% See the file fdl-1.3.txt for copying conditions.

%+++++++++++++++++++++++++++++++++++++++++++++++++++++++++++++++++++++++++++++++++++++++++++++++++++++++++++++++++++++++++++ 
\section{Sécantes et parallèles}
%+++++++++++++++++++++++++++++++++++++++++++++++++++++++++++++++++++++++++++++++++++++++++++++++++++++++++++++++++++++++++++

% This is part of Un soupçon de mathématique sans être agressif pour autant
% Copyright (c) 2015
%   Laurent Claessens
% See the file fdl-1.3.txt for copying conditions.

%--------------------------------------------------------------------------------------------------------------------------- 
\subsection*{Activité : graphisme d'un N}
%---------------------------------------------------------------------------------------------------------------------------

Un ébéniste voudrait dessiner un «X» et un «N» de la forme suivante :

\begin{center}
\input{Fig_MLZLooYDRsFl.pstricks}\quad
   \input{Fig_EZTHooBttIaQ.pstricks}
\end{center}

Si il voudrait que l'angle indiqué sur le «X»  soit de \SI{30}{\degree} et celui sur le «N»  soit de \SI{45}{\degree}, quelles sont les mesures des autres angles ?


\begin{definition}
    Des angles \defe{opposés par le sommet}{angle!opposé par le sommet} sont deux angles qui ont un sommet commun et qui ont leurs côtés dans le prolongement l'un de l'autre.
\end{definition}

\begin{propriete}
    Deux angles opposés par le sommet ont même mesure.
    \begin{center}
        \input{Fig_ZRDYooCERmeJ.pstricks}
    \end{center}
\end{propriete}

\begin{proof}
    Sur le dessin suivant, \( a\) et \(b\) sont deux angles opposés par le sommet et \( c\) est un troisième angle :

\begin{center}
    \input{Fig_PQYKooJWKpVZ.pstricks}
\end{center}

Nous avons
\begin{equation}
    a+c=180
\end{equation}
et
\begin{equation}
    b+c=180,
\end{equation}
donc \( a=b\).

\end{proof}

%+++++++++++++++++++++++++++++++++++++++++++++++++++++++++++++++++++++++++++++++++++++++++++++++++++++++++++++++++++++++++++ 
\section{Angles correspondants et alternes-internes}
%+++++++++++++++++++++++++++++++++++++++++++++++++++++++++++++++++++++++++++++++++++++++++++++++++++++++++++++++++++++++++++

% This is part of Un soupçon de mathématique sans être agressif pour autant
% Copyright (c) 2015
%   Laurent Claessens
% See the file fdl-1.3.txt for copying conditions.


    Une tarte est découpée selon des lignes parallèles horizontales et diagonales.

\begin{center}
   \input{Fig_ANRUooVCOTOb.pstricks}
\end{center}

Quelle est la forme des morceaux ainsi obtenus (à part ceux du bord) ? Est-ce qui'ils peuvent se superposer ? Indiquer les angles égaux.





\begin{definition}
    Lorsque deux droites parallèles sont coupées par une même droite, les angles suivants sont les \defe{angles correspondants}{angles!correspondants} :

\begin{center}
   \input{Fig_RYNYooMFTMNR0.pstricks}
   \hfill et\hfill
   \input{Fig_RYNYooMFTMNR2.pstricks}
\end{center}

et les angles suivants sont dits \defe{alternes-internes}{angles!alternes-internes} :

\begin{center}
   \input{Fig_RYNYooMFTMNR1.pstricks}
   \hfill et\hfill
   \input{Fig_RYNYooMFTMNR3.pstricks}
\end{center}

\end{definition}

\begin{propriete}
    Lorsque deux droites parallèles sont coupées par une même droite,
    \begin{enumerate}
        \item
            les angles correspondants sont égaux,
        \item
            les angles alternes-internes sont égaux.
    \end{enumerate}
\end{propriete}

%+++++++++++++++++++++++++++++++++++++++++++++++++++++++++++++++++++++++++++++++++++++++++++++++++++++++++++++++++++++++++++ 
\section{Angles internes d'un triangle}
%+++++++++++++++++++++++++++++++++++++++++++++++++++++++++++++++++++++++++++++++++++++++++++++++++++++++++++++++++++++++++++

% This is part of Un soupçon de mathématique sans être agressif pour autant
% Copyright (c) 2015
%   Laurent Claessens
% See the file fdl-1.3.txt for copying conditions.

%--------------------------------------------------------------------------------------------------------------------------- 
\subsection*{Activité : angles correspondants et triangle}
%---------------------------------------------------------------------------------------------------------------------------

Sur la figure suivante nous avons tracé un triangle \( ABC\) et la droite parallèle à \( (AB)\) passant par \( C\). Parmi les trois angles que l'on voit au point \( C\), lequel fait \SI{34}{\degree} ? Est-il possible de déterminer la mesure des deux autres ?

% ATTENTION : ces deux figures sont reprise dans une autre activité (Activité : angles correspondants et triangle). Ne pas les changer sans les doubler.

\begin{center}
   \input{Fig_UZOQooTSAQcl.pstricks}
\end{center}

Sur le dessin suivant, indiquer quels sont les angles égaux à \( a\) et à \( b\). 
\begin{center}
    \input{Fig_QZABooEsqWaq.pstricks}
\end{center}
Si \( a\) et \( b\) étaient connus, comment feriez-vous pour calculer l'angle \( c\) ?


\begin{theorem}[Angles internes d'un triangle]
    La somme des angles internes d'un triangle vaut \SI{180}{\degree}.
\end{theorem}

\begin{example}
    Dans la situation suivante, nous pouvons calculer l'angle \( \hat B\).

    \begin{center}
        \input{Fig_GQVWooMiTJnC.pstricks}
    \end{center}
    
    En effet les deux angles déjà connus sont \( \hat A=\SI{110}{\degree}\) et \( \hat C=\SI{40}{\degree}\). Leur somme vaut
    \begin{equation}
        \hat A+\hat C=110+40=150.
    \end{equation}
    Pour avoir un total de \SI{180}{\degree}, il faut donc avoir \( \hat B=180-150=\SI{30}{\degree}\).
\end{example}

%+++++++++++++++++++++++++++++++++++++++++++++++++++++++++++++++++++++++++++++++++++++++++++++++++++++++++++++++++++++++++++ 
\section{Triangle isocèle et équilatéral}
%+++++++++++++++++++++++++++++++++++++++++++++++++++++++++++++++++++++++++++++++++++++++++++++++++++++++++++++++++++++++++++

\begin{propriete}
    Deux cas particuliers :
    \begin{enumerate}
        \item
            Les deux angles de la base d'un triangle isocèle sont égaux.
        \item
            Les angles d'un triangle équilatéral valent tous \SI{60}{\degree}.
    \end{enumerate}
\end{propriete}

\begin{example}
    
    Un triangle isocèle :
    \begin{center}
        \input{Fig_OGWJooWdCsvT.pstricks}
    \end{center}
     et un équilatéral :
     \begin{center}
        \input{Fig_CRKLooLhpFTy.pstricks}
     \end{center}
\end{example}
