% This is part of Un soupçon de mathématique sans être agressif pour autant
% Copyright (c) 2015
%   Laurent Claessens
% See the file fdl-1.3.txt for copying conditions.

\begin{exercice}\label{exo2smath-0182}

    Nous considérons un cône de sommet \( S\) dont la base a un rayon de \SI{7}{\centi\meter} et dont la hauteur mesure \SI{10}{\centi\meter}. Nous nommons \( C\) le centre de la base. Pour les questions suivantes, vous pouvez répondre par des approximations numériques.
    \begin{enumerate}
        \item
            Soit \( P\) un point sur la circonférence de la base. Quelle est la distance \( SP\) ?
        \item
            Soit \( C'\) sur le segment \( [CS]\) tel que \( CC'=3\). Nous nommons \( R\) le point d'intersection entre la droite \( (SP)\) et le plan parallèle au plan de la base, passant par \( C'\). Quelle est la longueur \( RP\) ?
    \end{enumerate}

\corrref{2smath-0182}
\end{exercice}
