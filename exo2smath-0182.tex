% This is part of Un soupçon de mathématique sans être agressif pour autant
% Copyright (c) 2015
%   Laurent Claessens
% See the file fdl-1.3.txt for copying conditions.

\begin{exercice}\label{exo2smath-0182}

    Nous considérons un cône de sommet \( S\) dont la base a un rayon de \SI{7}{\centi\meter} et dont la hauteur mesure \SI{10}{\centi\meter}. Pour les questions suivantes, vous pouvez répondre par des approximations.
    \begin{enumerate}
        \item
            Soit \( P\) un point sur la circonférence de la base. Quelle est la distance \( SP\) ?
        \item
            Soit \( R\) un point sur le segment \( [SP]\) situé à la hauteur \SI{3}{\centi\meter}. Quelle est la longueur \( RP\) ?
    \end{enumerate}
    <++>



\corrref{2smath-0182}
\end{exercice}
