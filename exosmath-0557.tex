% This is part of Un soupçon de mathématique sans être agressif pour autant
% Copyright (c) 2013
%   Laurent Claessens
% See the file fdl-1.3.txt for copying conditions.

\begin{exercice}\label{exosmath-0557}

    Un luthier construit des clarinettes qu'il vend à \( 600\)€.  Ses frais de fabrication (pièce et main d'œuvre) lui reviennent à
    \begin{equation*}
        c(x)=0.5x^2-105x+71500
    \end{equation*}
    euros pour fabriquer \( x\) instruments. Lorsqu'il fait ses comptes, ce commerçant considère la fonction
    \begin{equation*}
        f(x)=600x-(0.5x^2-105x+71500).
    \end{equation*}
    \begin{enumerate}
        \item
            Pourquoi la fonction \( f\) est-elle intéressante ? Qu'est-ce qu'elle représente.
        \item
            Résoudre l'équation \( f(x)=0\). Que représentent ces nombres pour notre luthier ?
        \item
            Résoudre l'inéquation \( f(x)\geq 133500\). Interpréter le résultat.
    \end{enumerate}

\corrref{smath-0557}
\end{exercice}
