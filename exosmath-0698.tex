% This is part of Un soupçon de mathématique sans être agressif pour autant
% Copyright (c) 2014
%   Laurent Claessens
% See the file fdl-1.3.txt for copying conditions.

\begin{exercice}\label{exosmath-0698}

\begin{wrapfigure}{r}{5.0cm}
   \vspace{-0.5cm}        % à adapter.
   \centering
   \input{Fig_MTZooSGviQ.pstricks}
\end{wrapfigure}

    Dans le parallélépipède suivant, \( AD=2\), \( CD=3\) et \( AE=4\). Répondre aux questions suivantes.

\begin{enumerate}
    \item
        Calculer la longueur de \(  [ED]\) et \( [EC]\).
    \item
        Quelle est l'aire du triangle \( BGF\) ?
    \item
        Est-ce que la droite \( (EF)\) intersecte la droite \( (AC)\) ?
    \item
        Placer un point \( I\) au milieu du segment \( [HG]\). Est-ce que la droite \( (DI)\) intersecte la droite \( (GC)\) ?
\end{enumerate}


\corrref{smath-0698}
\end{exercice}
