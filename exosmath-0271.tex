% This is part of Un soupçon de mathématique sans être agressif pour autant
% Copyright (c) 2013
%   Laurent Claessens
% See the file fdl-1.3.txt for copying conditions.

\begin{exercice}\label{exosmath-0271}

\begin{wrapfigure}{r}{7.0cm}
   \vspace{-3.5cm}        % à adapter.
   \centering
   \input{Fig_figureZEKOYck.pstricks}
\end{wrapfigure}

Associer à chacune des fonctions 
\begin{enumerate}
    \item
        \( f(x)=\frac{ x^2 }{2}+2x+1\).
\item
\( g(x)=\frac{ 1 }{2}x^2+x-\frac{ 1 }{2}\)
\item
\( h(x)=-\frac{ 1 }{2}x^2+2x\)
\end{enumerate}
son graphe représenté ci-contre.

\corrref{smath-0271}
\end{exercice}
