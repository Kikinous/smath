% This is part of Un soupçon de mathématique sans être agressif pour autant
% Copyright (c) 2012
%   Laurent Claessens
% See the file fdl-1.3.txt for copying conditions.

\begin{exercice}\label{exoPremiere-0086}

    Un joueur de jeu de rôles doit lancer un dé trois fois. À chaque test, il réussi si il fait \( 1\), \( 2\), \( 3\) ou \( 4\) et échoue sinon. Nous désignons par \( X\) la variable aléatoire correspondante au nombre de réussites.
    \begin{enumerate}
        \item
            Identifier les paramètres du schéma de Bernoulli sous-jacent à cette situation.
        \item
            Dessiner un arbre.
        \item
            Compléter le tableau
            \begin{equation*}
                \begin{array}[]{|c||c|c|c|c|c|c|}
                    \hline
                    k&0&1&2&3&4&5\\
                    \hline\hline
                    P(X=k)&&&&&&\\
                    \hline
                \end{array}
            \end{equation*}
    \end{enumerate}
    Donner les réponses sous forme de fraction et d'approximations numériques.

\corrref{Premiere-0086}
\end{exercice}
