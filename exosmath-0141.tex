% This is part of Un soupçon de mathématique sans être agressif pour autant
% Copyright (c) 2012
%   Laurent Claessens
% See the file fdl-1.3.txt for copying conditions.

\begin{exercice}\label{exosmath-0141}

    Nous voudrions comparer les nombres \( A=\sqrt{2}+\sqrt{7}\) et \( B=\sqrt{9+2\sqrt{14}}\) sans faire appel à la calculatrice.
    \begin{enumerate}
        \item
            Calculer \( A^2\) et \( B^2\) (penser aux identités remarquables).
        \item
            Que peut-on déduire comme comparaison entre \( A^2\) et \( B^2\) ?
        \item
            Déduire une comparaison entre \( A\) et \( B\) en expliquant les étapes du raisonnement.
    \end{enumerate}

\corrref{smath-0141}
\end{exercice}
