% This is part of Un soupçon de mathématique sans être agressif pour autant
% Copyright (c) 2015
%   Laurent Claessens
% See the file fdl-1.3.txt for copying conditions.

\begin{exercice}[\ldots\ldots/5]\label{exo2smath-0327}


    \begin{multicols}{2}

    Sur ce dessin \( ABCD\) est un rectangle, l'angle \( \widehat{AED}\) mesure \SI{70}{\degree}, et les points \( D\), \( E\), \( K\) sont alignés.
    \begin{enumerate}
        \item
            Quelle est la mesure de l'angle \( \widehat{AEK}\) ?
        \item
            Déterminer tous les angles du triangle \( AED\).
        \item
            Déterminer tous les angles du triangle \( AEK\).
        \item
            En déduire la mesure de l'angle \( \widehat{DEF}\).
    \end{enumerate}

\columnbreak

\begin{center}
\input{Fig_SMXRooCnrlNw.pstricks}
\end{center}

    \end{multicols}

\corrref{2smath-0327}
\end{exercice}
