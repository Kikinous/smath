% This is part of Un soupçon de mathématique sans être agressif pour autant
% Copyright (c) 2013
%   Laurent Claessens
% See the file fdl-1.3.txt for copying conditions.

\begin{exercice}\label{exosmath-0494}

\begin{wrapfigure}{r}{4.0cm}
   \vspace{-0.5cm}        % à adapter.
   \centering
   \input{Fig_GRRhrJe.pstricks}
\end{wrapfigure}

À propos des droites \( d_1\), \( d_2\), \( d_3\) et \( d_4\) ci-contre.
\begin{enumerate}
    \item
        Les classer par ordre croissant de coefficient directeur. 
    \item
        Laquelle a un coefficient directeur positif et une ordonnée à l'origine négative ?
\end{enumerate}

\corrref{smath-0494}
\end{exercice}
