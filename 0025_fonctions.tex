%This is part of Un soupçon de mathématique sans être agressif pour autant
% Copyright (c) 2012-2014
%   Laurent Claessens
% See the file fdl-1.3.txt for copying conditions.

Dans ce chapitre :
\begin{enumerate}
    \item
        Résoudre graphiquement des inéquations : \( f(x)\leq k\) ou \( f(x)\geq g(x)\).
    \item
        Lien tableau de variations, tableau de valeurs et dessin.
    \item
        Comparer des images depuis un tableau de variation.
    \item
        Donner des fonctions sous forme de programmes ou algos.
\end{enumerate}


\begin{definition}
    L'\defe{ensemble de définition}{ensemble de définition} d'une fonction est l'ensemble de tous les \( x\) pour lesquels \( f(x)\) est définie.
\end{definition}

\begin{example}
    \begin{enumerate}
        \item
            \( f(x)=x^2\)
        \item
            \( g(x)=\frac{1}{ x }\)
        \item
            \( h(x)=\sqrt{x}\)
        \item
            \( k(x)\) donnée par un graphique.
    \end{enumerate}
\end{example}

%+++++++++++++++++++++++++++++++++++++++++++++++++++++++++++++++++++++++++++++++++++++++++++++++++++++++++++++++++++++++++++ 
\section{Différentes manières de parler d'une fonction}
%+++++++++++++++++++++++++++++++++++++++++++++++++++++++++++++++++++++++++++++++++++++++++++++++++++++++++++++++++++++++++++

\begin{definition}
    L'\defe{ensemble de définition}{ensemble!de définition} d'une fonction est l'ensemble de tous les \( x\) tels que \( f(x)\) est définie.
\end{definition}

\begin{example}
    \begin{enumerate}
        \item
            \( f(x)=x^2\).
        \item
            \( f(x)=\frac{1}{ x }\).
    \end{enumerate}
\end{example}

%--------------------------------------------------------------------------------------------------------------------------- 
\subsection{Tableau de valeurs}
%---------------------------------------------------------------------------------------------------------------------------

Le tableau de valeurs consiste à donner quelque valeurs connues de la fonction.

\begin{example}
    Soit une fonction \( f\) définie sur \( \mathopen[ -5 ; 10 \mathclose]\) et dont nous connaissons le tableau de valeurs suivant :
    \begin{equation}
        \begin{array}[h]{|c||c|c|c|c|c|c|}
            \hline
            x&-5&-1&1&3&9&10\\
            \hline
            f(x)&0&2&-3&7&2&5\\
            \hline
        \end{array}
    \end{equation}
    Nous pouvons dire que
    \begin{enumerate}
        \item
            L'image de \( 1\) par \( f\) est \( -3\).
        \item
            Les nombres \( -1\) et \( 9\) sont tous deux des antécédents de \( 2\).
    \end{enumerate}
    Nous ne pouvons pas dire 
    \begin{enumerate}
        \item
            L'image de \( 2\).
        \item
            Le nombre \( 3\) a d'autres antécédents que \( 7\).
        \item
            Si la fonction est croissante sur \( \mathopen[ -5 ;0 \mathclose]\).
    \end{enumerate}


    En réalité n'importe quelle courbe qui passe par les points suivants peut être \( f\) :
    \begin{center}
   \input{Fig_WYeESAN.pstricks}
    \end{center}

\end{example}

%--------------------------------------------------------------------------------------------------------------------------- 
\subsection{Tableau de signe}
%---------------------------------------------------------------------------------------------------------------------------

Le tableau de signe ne donne que le signe de la fonction. Prenons une fonction \( f\) définie sur \( \mathopen[ -4 , 10 \mathclose]\) dont nous savons le tableau de signe :
\begin{equation*}
    \begin{array}[]{c|ccccccccc}
        x&&-1&&2&&3&&7&\\
        \hline
        f(x)&+&0&-&0&-&0&+&0&-\\
    \end{array}
\end{equation*}
Nous savons que
\begin{enumerate}
    \item
        \( 0\) a pour antécédents \( -1\), \( 2\), \( 3\), \( 7\).
    \item
        \( f(1)<0\).
\end{enumerate}
Nous ne savons pas si 
\begin{enumerate}
    \item
        \( f\) est croissante sur \( \mathopen[ 0 ; 2 \mathclose]\).
    \item
        Quelle est l'image de \( 5\).
\end{enumerate}

Une forme possible de \( f\) est donnée ci-dessous :
\begin{center}
\input{Fig_EIxhcRb.pstricks}
\end{center}

Et de nombreuses autres sont possibles.

%--------------------------------------------------------------------------------------------------------------------------- 
\subsection{Tableau de variations}
%---------------------------------------------------------------------------------------------------------------------------

Le tableau de variations dit pour chaque abscisse si la fonction est croissante ou décroissante ainsi que certaines valeurs.

\begin{example}
    \begin{equation*}
        \begin{array}[]{c|ccccccc}
            x&4&&-1&&3&&6\\
            \hline
            &5&&&&3&&\\
            f(x)&&\searrow&&\nearrow&&\searrow&\\
            &&&-2&&&&-1\\
        \end{array}
    \end{equation*}

Ce que le tableau dit :
\begin{enumerate}
    \item
        \( f(-1)=-2\).
    \item
        Entre \( x=3\) et \( x=6\), la fonction est décroissante.
    \item
        \( f(-2)\geq f(-1)\).
    \item
        \( f(5)\) est entre \( 3\) et \( 6\).
\end{enumerate}
Ce que le tableau ne dit pas :
\begin{enumerate}
    \item
        La valeur de \( f(1)\).
    \item
        Si \( f(4)\) est plus grand ou plus petit que \( f(-3)\).
\end{enumerate}

\end{example}
