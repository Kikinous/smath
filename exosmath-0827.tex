% This is part of Un soupçon de mathématique sans être agressif pour autant
% Copyright (c) 2014
%   Laurent Claessens
% See the file fdl-1.3.txt for copying conditions.

\begin{exercice}\label{exosmath-0827}

Nous considérons l'enchaînement d'opérations suivant :
\begin{itemize}
    \item Choisir un nombre entier.
    \item ajouter \( 7\),
    \item multiplier par \( 3\),
    \item soustraire \( 21\).
\end{itemize}
\begin{enumerate}
    \item
        Donner le résultat de cet enchaînement lorsque le nombre choisi est \( 3\), \( 10\) et \( 12\).
    \item
        Quel semble être le lien entre le nombre choisi et le résultat ? Démontrer.
    \item
        Quel nombre faut-il choisir au départ pour obtenir \( 21\) ?
    \item
        Combien obtient-t-on en choisissant \( 7001\) ?
\end{enumerate}

\corrref{smath-0827}
\end{exercice}
