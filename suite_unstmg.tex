% This is part of Un soupçon de mathématique sans être agressif pour autant
% Copyright (c) 2012
%   Laurent Claessens
% See the file fdl-1.3.txt for copying conditions.

%+++++++++++++++++++++++++++++++++++++++++++++++++++++++++++++++++++++++++++++++++++++++++++++++++++++++++++++++++++++++++++ 
\section{Suites numériques}
%+++++++++++++++++++++++++++++++++++++++++++++++++++++++++++++++++++++++++++++++++++++++++++++++++++++++++++++++++++++++++++

\begin{example}
    Un randonneur marche à \unit{4}{\kilo\meter\per\hour}. Il regarde et note la distance parcourue toutes les demi-heures : \( u_1=2\), \( u_2=4\), \( u_3=6\), etc.
\end{example}

\begin{example}
    Un taxi fait payer \( 3\) euros la prise en charge puis deux euro par kilomètres entamés. La suite donnant le prix à payer en fonction du nombre de kilomètres entamés est
    \begin{equation}
        u_n=3+2n.
    \end{equation}
\end{example}

\begin{definition}
    Une \defe{suite numérique}{suite numérique} est une liste de nombres réels numérotés par les nombres entiers soit à partir de zéro soit à partir de \( 1\).

    La suite est \defe{croissante}{croissante!suite} si pour tout \( n\) on a \( u_n<u_{n+1}\) et \defe{décroissante}{décroissante!suite} si pour tout \( n\) on a \( u_n>u_{n+1}\).
\end{definition}

%+++++++++++++++++++++++++++++++++++++++++++++++++++++++++++++++++++++++++++++++++++++++++++++++++++++++++++++++++++++++++++ 
\section{Suites définies par récurrence}
%+++++++++++++++++++++++++++++++++++++++++++++++++++++++++++++++++++++++++++++++++++++++++++++++++++++++++++++++++++++++++++

\begin{example}
    Chaque mois un locataire reçoit \( 1700\) euros de salaire, en paye \( 700\) de loyer, \( 400\) de nourriture et \( 200\) de frais divers. Au premier janvier, il avait \( 5200\) euros sur son compte. Donner la suite qui donne l'argent qu'il lui reste en fonction du nombre de mois écoulés.

    Première étape : on commence à \( u_0=5200\). Chaque mois il gagne \( 1700\) et perd \( 1300\), donc il met \( 400\) euros sur son compte, donc
    \begin{equation}
        u_{n+1}=u_n+400.
    \end{equation}
\end{example}

%+++++++++++++++++++++++++++++++++++++++++++++++++++++++++++++++++++++++++++++++++++++++++++++++++++++++++++++++++++++++++++ 
\section{Suites arithmétiques}
%+++++++++++++++++++++++++++++++++++++++++++++++++++++++++++++++++++++++++++++++++++++++++++++++++++++++++++++++++++++++++++

\begin{definition}
    Soit \( a\in \eR\). Une suite \( (u_n)\) est une \defe{suite arithmétique}{suite!arithmétique} de raison \( a\) si pour tout entier \( n\) nous avons \( u_{n+1}=u_n+a\).
\end{definition}

\begin{example}
    Une chaîne de supermarché a déjà \( 200\) points de vente. Chaque année elle en ouvre \( 6\) de plus.
\end{example}

Placés sur un graphique, les éléments d'une suite arithmétique se mettent sur une droite.

%+++++++++++++++++++++++++++++++++++++++++++++++++++++++++++++++++++++++++++++++++++++++++++++++++++++++++++++++++++++++++++ 
\section{Exercices}
%+++++++++++++++++++++++++++++++++++++++++++++++++++++++++++++++++++++++++++++++++++++++++++++++++++++++++++++++++++++++++++

%--------------------------------------------------------------------------------------------------------------------------- 
\subsection{Suites numériques}
%---------------------------------------------------------------------------------------------------------------------------

\Exo{smath-0158}
\Exo{smath-0160}

%--------------------------------------------------------------------------------------------------------------------------- 
\subsection{Suite définie par récurrence}
%---------------------------------------------------------------------------------------------------------------------------

\Exo{smath-0161}
\Exo{smath-0162}
\Exo{smath-0159}

%--------------------------------------------------------------------------------------------------------------------------- 
\subsection{Suites arithmétiques}
%---------------------------------------------------------------------------------------------------------------------------

\Exo{smath-0163}
\Exo{smath-0164}
\Exo{smath-0165}
\Exo{smath-0166}
