% This is part of Un soupçon de mathématique sans être agressif pour autant
% Copyright (c) 2012
%   Laurent Claessens
% See the file fdl-1.3.txt for copying conditions.

\begin{exercice}\label{exosmath-0172}

Écrire une fonction qui prend un nombre \( x\) en argument et qui retourne \( x\) si \( x>0\) et \( -x\) sinon. Sous forme mathématique, la fonction est donnée par
\begin{equation}
    f(x)=\begin{cases}
        x    &   \text{si \( x>0\)}\\
        0    &    \text{sinon}.
    \end{cases}
\end{equation}
Tester cette fonction sur quelque nombres positifs et négatifs.

\corrref{smath-0172}
\end{exercice}
