% This is part of Un soupçon de mathématique sans être agressif pour autant
% Copyright (c) 2014-2015
%   Laurent Claessens
% See the file fdl-1.3.txt for copying conditions.

\begin{corrige}{smath-0600}

    \begin{enumerate}
        \item
            Réponse courte : \( (GHE)\) est le plan de la face arrière; donc \( I\) est dedans et \( J\) n'y est pas.

            Réponse longue : d'abord la droite \( (FG)\) est dans le plan \( (GHE)\) parce que c'est la parallèle à \( (EH)\) passant par \( G\). Donc \( F\in (GHE)\). Ensuite vu que \( E\) et \( F\) sont dans le plan \( (GHE)\), la droite \( (EF)\) y est aussi. En particulier le point \( I\).

            Le point \( J\) n'est pas dans le plan \( (GHE)\) parce que la droite \( (BF)\) intersecte ce plan en un seul point (et ce point est \( F\)).

        \item
            Ils sont sécants : comme cela se voit déjà dans leurs noms, le point \( A\) est dans l'intersection :
            \begin{equation}
                A\in (ABF)\cap(HGA).
            \end{equation}
            Vu que ces deux plans sont sécants et non confondus, l'intersection est une droite, et pour la caractériser il nous faut un second point.

            Le plan \( (ABF)\) est celui de la face du dessus. Le plan \( (HGA)\) est celui qui coupe le cube en deux parties égales de façon diagonale; il contient entre autres le rectangle \( ABGH\). Le second point d'intersection qu'il nous faut est \( B\).

            Donc
            \begin{equation}
                (ABF)\cap(HGA)=(AB);
            \end{equation}
            c'est la droite horizontale située à l'avant du cube (en haut).

        \item   \label{ItemECveWvo}
            Le plus simple est de dessiner le cube vu d'au dessus :
            \begin{center}
                \input{Fig_RHfkPKj.pstricks}
            \end{center}
            Vu l'énoncé nous avons \( IF=FJ=\unit{3.5}{\kilo\meter}\). Une simple application du théorème de Pythagore nous indique alors que
            \begin{equation}
                IJ^2=2\times (3.5)^2=24.5
            \end{equation}
            et donc la longueur demandée est
            \begin{equation}
                IJ=\sqrt{24.5}\si{\kilo\meter}.
            \end{equation}
            
    \end{enumerate}

\end{corrige}
