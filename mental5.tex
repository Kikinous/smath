% This is part of Un soupçon de mathématique sans être agressif pour autant
% Copyright (c) 2014
%   Laurent Claessens
% See the file fdl-1.3.txt for copying conditions.

%%%%%%%%%%%%%%%%%%%%%%%%%%%%%%%%%%%%%%%%%%%%%%%%%%%%%%%%%%%%
\begin{MentalActivity}

    \begin{mental}
        Vrai ou faux ?
        \begin{enumerate}
            \item
                Si un stylo coûte \( 1.5\) euros, alors trois stylos coûtent \( 4.5\) euros.
            \item
                Le quadruple d'un nombre \( x\) s'écrit \( 4\times x\).
            \item
                Si j'ai \( x\) pièces de \( 1\) centime, alors j'ai \( x-100\) euros.
        \end{enumerate}
    \end{mental}

    \begin{mental}
        Quelle est la circonférence d'une cercle de rayon \( \unit{2}{\centi\meter}\) ? (rappel : la formule est \( 2\times\pi\times R\)) ?
    \end{mental}

    \begin{mental}
        \begin{enumerate}
            \item
                L'aire de ce rectangle est de \(\unit{20}{\centi\meter\squared}\).
        \begin{center}
           \input{Fig_LCUooNGZJFk.pstricks}
        \end{center}
        Que vaut sa hauteur \( h\) ?
            \item
                L'aire de ce rectangle est de \(\unit{20}{\centi\meter\squared}\).
                \begin{center}
                    \input{Fig_KXXooKBoqAY.pstricks}
                \end{center}
                Que vaut sa hauteur \( h\) en fonction de sa longueur \( l\) ?
        \end{enumerate}
    \end{mental}
    
    \begin{mental}
        Calculs 
        \begin{enumerate}
            \item
                Combien vaut \( 7\times a\) si \( a=9\) ?
            \item
                Combien vaut \( 2\times b+5\) si \( b=3\) ?
            \item
                Combien vaut \( 7\times a+b\) si \( a=2\) et \( b=6\) ?
        \end{enumerate}
    \end{mental}

\end{MentalActivity}

%%%%%%%%%%%%%%%%%%%%%%%%%%%%%%%%%%%%%%%%%%%%%%%%%%%%%%%%%%

% Les exercices suivants contienent des vrai/faux sur le second degré. À mettre dans les prochains calculs mental.
% M'est avis qu'il faut les ajouter aussi à ``autres exercices de seconde''.
%\Exo{smath-0652}    % Haag; c'est mon vrai ou faux en vrac à compléter.
%\Exo{smath-0254}
