% This is part of Un soupçon de mathématique sans être agressif pour autant
% Copyright (c) 2012
%   Laurent Claessens
% See the file fdl-1.3.txt for copying conditions.

\begin{exercice}\label{exosmath-0195}

Un premier sac contient trois boules numérotées \( 0\), \( 2\) et \( 4\). Un second sac contient trois boules numérotées \( 0\), \( 1\) et \( 3\). Nous tirons une boule de chaque sac et considérons la somme des deux nombres obtenus.
\begin{enumerate}
    \item
        Quelle est la probabilité que la somme soit égale à \( 0\) ?
    \item
        Quelle est la probabilité que la somme soit égale à \( 5\) ?
    \item
        Quelle est la probabilité que la somme soit un nombre pair ?
\end{enumerate}

\corrref{smath-0195}
\end{exercice}
