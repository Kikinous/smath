% This is part of Un soupçon de mathématique sans être agressif pour autant
% Copyright (c) 2012
%   Laurent Claessens
% See the file fdl-1.3.txt for copying conditions.

\begin{exercice}\label{exosmath-0195}

Un premier sac contient trois boules numérotées \( 0\), \( 2\) et \( 4\). Un second sac contient trois boules numérotées \( 0\), \( 1\) et \( 3\). Nous tirons une boule de chaque sac et considérons la somme des deux nombres obtenus. Compléter le tableau :
\begin{center}
    \begin{equation*}
        \begin{array}[]{|c||c|c|c|c|c|c|c|c|c|}
            \hline
            x_i&0&1&2&3&4&5&6&7&\text{total}\\
              \hline\hline
              p_i&&&&&&&&&\\
              \hline 
               \end{array}
           \end{equation*}
\end{center}

\corrref{smath-0195}
\end{exercice}
