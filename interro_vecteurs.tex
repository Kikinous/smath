% This is part of Un soupçon de mathématique sans être agressif pour autant
% Copyright (c) 2012-2014
%   Laurent Claessens
% See the file fdl-1.3.txt for copying conditions.

\documentclass[a4paper,10pt]{article}
% This is part of Un soupçon de mathématique sans être agressif pour autant
% Copyright (c) 2012-2013
%   Laurent Claessens
% See the file fdl-1.3.txt for copying conditions.


\usepackage{etex}
\usepackage{ifthen}
%\usepackage{pdfsync}       % This package is obsolete : compile with pdflatex -synctex=1 instead.

\usepackage{latexsym}
\usepackage{amsfonts}
\usepackage{amsmath}
\usepackage{amsthm}
\usepackage{amssymb}
\usepackage{bbm}
\usepackage{mathrsfs}           
\usepackage{mathabx}           % Pour \divides

\usepackage{framed}

\usepackage{calc}   % Les dépendances de phystricks si on n'utilise que le pdf.
%\usepackage{pstricks,pst-eucl,pstricks-add,calc,pst-math}   % Les dépendances de phystricks. Peut être qu'il faut ajouter catchfile
\usepackage{graphicx}                   % Pour l'inclusion d'image en pfd.

\newcommand{\EpsOrPdfincludegraphics}[2][]{%
        \ifpdf
            \includegraphics[#1]{#2.png}
        \else
            \includegraphics[#1]{#2.eps}
        \fi
        }

\usepackage{subfigure}

\usepackage{fancyvrb}
\usepackage{stmaryrd}       % Pour le \obslash
\usepackage{xstring}        % Utilisé pour les références vers wikipédia
\usepackage{cases}
\usepackage{lscape}         % pour l'environnement landscape, utilisé dans la correction corr0076.tex
\usepackage{multicol}
\usepackage{import}         % Pour le hack qui sert à inclure GeomAnal

% TODO : n'en utiliser qu'un
\usepackage[normalem]{ulem}		% Pour le barré, commande \sout
\usepackage{soul}		% Pour le barré, commande \st

\usepackage[all]{xy}

\let\second\undefined      % le paquet amthabx définit \second
\let\degree\undefined       % le paquet amthabx définit \degree
\usepackage[cdot,thinqspace,amssymb]{SIunits} 
 % L'option amssymb sert à éviter un conflit avec la commande \square de amssymb. Note qu'elle n'est plus accessible. Si tu en as besoin, faudra RTFM
%ftp://ftp.belnet.be/packages/ctan/macros/latex/contrib/SIunits/SIunits.pdf

\usepackage[nottoc]{tocbibind}

%%%%%%%%%%%%%%%%%%%%%%%%%%
%
%   Trucs mathématiques
%
%%%%%%%%%%%%%%%%%%%%%%%%

% ENSEMBLES DE NOMBRES
\newcommand{\eA}{\mathbbm{A}}
\newcommand{\eC}{\mathbbm{C}}
\newcommand{\eD}{\mathbbm{D}}
\newcommand{\eE}{\mathbbm{E}}
\newcommand{\eF}{\mathbbm{F}}
\newcommand{\eG}{\mathbbm{G}}
\newcommand{\eH}{\mathbbm{H}}
\newcommand{\eK}{\mathbbm{K}}
\newcommand{\eL}{\mathbbm{L}}
\newcommand{\eM}{\mathbbm{M}}
\newcommand{\eN}{\mathbbm{N}}
\newcommand{\eP}{\mathbbm{P}}
\newcommand{\eQ}{\mathbbm{Q}}
\newcommand{\eR}{\mathbbm{R}}
\newcommand{\eZ}{\mathbbm{Z}}

% ENSEMBLES de fonctions
\newcommand{\aL}{\mathcal{L}}       % Les applications linéaires
\newcommand{\aC}{\mathcal{C}}       % Les fonctions C^1, C^2 etc

% AUTRES
\newcommand{\sdS}{\mathcal{S}}      % L'ensemble des subdivisions d'un intervalle.



\newcommand{\mF}{\mathcal{F}}
\newcommand{\mC}{\mathcal{C}}
\newcommand{\mG}{\mathcal{G}}
\newcommand{\mI}{\mathcal{I}}
\newcommand{\mL}{\mathcal{L}}
\newcommand{\mS}{\mathcal{S}}   % Utilisé pour l'espace des fonctions Schwartz
\newcommand{\mZ}{\mathcal{Z}}


\newcommand{\mtu}{\mathbbm{1}}              % La matrice unité
\newcommand{\caract}{\mathbbm{1}}    % Characteristic function of a set

\DeclareMathOperator{\val}{val}     % valuation d'un polynôme


%\newcommand{\efrac}[2]{\frac{ \displaystyle #1 }{\displaystyle #2 }}
%%%%%%%%%%%%%%%%%%%%%%%%%%
%
%   Numérotations en tout genre
%
%%%%%%%%%%%%%%%%%%%%%%%%

\setcounter{tocdepth}{2}        % Profondeur de la table des matières
\setcounter{secnumdepth}{2}     % Profondeur dans le texte

%%%%%%%%%%%%%%%%%%%%%%%%%%
%
%   Les lignes magiques pour le texte en français.
%
%%%%%%%%%%%%%%%%%%%%%%%%

\usepackage[utf8]{inputenc}
\usepackage[T1]{fontenc}

\usepackage{listingsutf8}
\lstset{language=python,basicstyle=\footnotesize,tabsize=3,numbers=left,numberstyle=\tiny,frame=single,commentstyle=\ttfamily\color[rgb]{0,0,0.5},stringstyle=\color[rgb]{0,0.5,0},title=\lstname,inputencoding=utf8/latin1}

\usepackage[fr]{exocorr}
\usepackage{textcomp}
\usepackage{lmodern}
\usepackage[a4paper,margin=2cm]{geometry} 
\usepackage[english,frenchb]{babel}


\usepackage{hyperref}                           %Doit être appelé en dernier.
\hypersetup{
colorlinks=true,
linkcolor=blue,
urlcolor=magenta,     % couleur des url
filecolor=magenta   % couleur des textes qui sont des liens
}

%%%%%%%%%%%%%%%%%%%%%%%%%%
%
%   Les théorèmes et choses attenantes
%
%%%%%%%%%%%%%%%%%%%%%%%%


\newcounter{numtho}
\newcounter{numprob}

\makeatletter
\@addtoreset{numtho}{chapter}
%\@addtoreset{CountExercice}{chapter}
\@addtoreset{chapter}{part}
\makeatother

\newlength{\EnvSpace}
\setlength{\EnvSpace}{9pt}      % C'est la distance que je veux mettre avant et après les théorèmes, remarques, etc.

\newtheoremstyle{MyTheorems}%
        {\EnvSpace}{\EnvSpace}%
        {\itshape}%
        {}%
        {\bfseries}{.}%
        {\newline}%
        {}%
\newtheoremstyle{MyExamples}%
        {\EnvSpace}{\EnvSpace}%
        {}%
        {}%
        {\bfseries}{.}%
        {\newline}%
        {}%
\newtheoremstyle{MyRemarks}%
        {\EnvSpace}{\EnvSpace}%
        {}%
        {}%
        {\bfseries}{.}%
        {\newline}%
        {}%

%\theoremstyle{MyExamples}   %\newtheorem{exemple}[numtho]{Exemple}      % Pour unification, ne plus utiliser
%                            \newtheorem{example}[numtho]{Exemple}
\newcounter{CounterExample}
\renewcommand{\theCounterExample}{\thechapter.\arabic{CounterExample}}

\newenvironment{example}{\vspace{\EnvSpace}\refstepcounter{numtho}\noindent{\bf Exemple \thenumtho}\newline}{\phantom{a}\hfill $\triangle$\vspace{\EnvSpace}}
\newenvironment{Aretenir}{\refstepcounter{numtho}\begin{oframed}\noindent{\bf À retenir \thenumtho}\newline}{\end{oframed}\vspace{\EnvSpace}}
\newenvironment{Enmini}{\begin{oframed}\noindent{\bf Mini résumé}\newline}{\end{oframed}\vspace{\EnvSpace}}
\newenvironment{definition}{\refstepcounter{numtho}\begin{oframed}\noindent{\bf Définition \thenumtho}\newline}{\end{oframed}\vspace{\EnvSpace}}
\newenvironment{propriete}{\refstepcounter{numtho}\begin{oframed}\noindent{\bf Propriété \thenumtho}\newline}{\end{oframed}\vspace{\EnvSpace}}

\theoremstyle{MyRemarks}    \newtheorem{remark}[numtho]{Remarque}

                \newtheorem{amusement}[numtho]{Amusement}
                \newtheorem{erreur}[numtho]{Error}
                \newtheorem{probleme}[numprob]{\fbox{\bf Problèmes et choses à faire}}


\theoremstyle{MyTheorems}
            \newtheorem{lemma}[numtho]{Lemme}
            \newtheorem{corollary}[numtho]{Corollaire}
            \newtheorem{theorem}[numtho]{Théorème}      
            \newtheorem{proposition}[numtho]{Proposition}      

            %\newtheorem{exo}[CountExercice]{Exercice}       % C'est provisoire, pour Chafaï

\renewcommand{\thenumtho}{\thechapter.\arabic{numtho}}
% La numérotation des équations change dans les corrigés
\renewcommand{\theequation}{\thechapter.\arabic{equation}}
\renewcommand{\theCountExercice}{\arabic{CountExercice}}       % Ce compteur est défini dans SystemeCorr.sty
\newcommand{\defe}[2]{\textbf{#1}\index{#2}}

\renewcommand{\labelenumi}{\theenumi}
\renewcommand{\theenumi}{(\arabic{enumi})}


%%%%%%%%%%%%%%%%%%%%%%%%%%
%
%   Les macros qui font des choses
%
%%%%%%%%%%%%%%%%%%%%%%%%

\newcommand{\mA}{\mathcal{A}}
\newcommand{\mO}{\mathcal{O}}
\newcommand{\mR}{\mathcal{R}}
\newcommand{\mT}{\mathcal{T}}
\newcommand{\mU}{\mathcal{U}}

\newcommand{\scal}[2]{ \langle #1,#2\rangle }

\newcommand{\tq}{\text{ tel que }}
\newcommand{\tqs}{\text{ tels que }}
\newcommand{\quext}[1]{ \footnote{\textsf{#1}}  }
\newcommand{\info}[1]{\texttt{#1}}
\newcommand{\vect}[1]{\overrightarrow{#1}}    % Cette macro est codée en dur dans phystricksDefVecteurAXDDGP et dans d'autres

\newcommand{\VarAbs}{\text{Var}_{\text{abs}}}
\newcommand{\VarRel}{\text{Var}_{\text{rel}}}

\newcommand{\normal}{\lhd}
\newcommand{\swS}{\mathscr{S}}          % L'ensemble des fonctions Schwartz

%\newcommand{\defD}{\mathscr{D}}     % Ensemble de définition d'une fonction
\newcommand{\defD}{D}                % Le D avec des croles était impossible à comprendre pour les élèves.

\newcommand{\Borelien}{\mathcal{B}\text{or}}       % Les boréliens
\newcommand{\tribA}{\mathcal{A}}            % Une tribu A
\newcommand{\tribB}{\mathcal{B}}            
\newcommand{\tribF}{\mathcal{F}}            % Une tribu F

\newcommand{\affE}{\mathcal{E}}            % Un espace affine E
\newcommand{\affF}{\mathcal{F}}            
\newcommand{\affG}{\mathcal{G}}            

\newcommand{\statS}{\mathcal{S}}            % Un modèle statistique
\newcommand{\partP}{\mathcal{P}}            % L'ensemble des parties d'un ensemble

\newcommand{\polyP}{\mathcal{P}}            % L'ensemble des polynômes

\newcommand{\dB}{\mathscr{B}}       % la distribution de Bernoulli
\newcommand{\dE}{\mathscr{E}}       % la distribution exponentielle
\newcommand{\dG}{\mathscr{G}}       % la distribution géométrique.
\newcommand{\dM}{\mathscr{M}}       % la distribution multinomiale
\newcommand{\dN}{\mathscr{N}}       % la distribution normale.
\newcommand{\dP}{\mathscr{P}}       % la distribution de Poisson.
\newcommand{\dT}{\mathscr{T}}       % la distribution de Student
\newcommand{\dU}{\mathscr{U}}       % la distribution uniforme

\newcommand{\hL}{\mathscr{L}}       
\newcommand{\cL}{\hL}           % Pour la partie Chafai

\newcommand{\modE}{\mathcal{E}}         % Le E des modules
\newcommand{\modF}{\mathcal{F}}         % Le F des modules
\newcommand{\hH}{\mathscr{H}}           % Le H des espaces de Hilbert

%%%%%%%%%%%%%%%%%%%%%%%%%%
%
%   Bibliographie, index et liste des notations
%
%%%%%%%%%%%%%%%%%%%%%%%%

\usepackage{makeidx}
\usepackage[nottoc]{tocbibind}      % Le paquetage qui fait en sorte que la biblio soit inclue correctement dans la table des matières.
\usepackage[refpage]{nomencl}
\renewcommand{\nomname}{Liste des notations}
%
%   Comment introduire des éléments dans l'index des notations.
%
% La liste des tags à mettre pour bien classer mes notations est :
% T     pour la topologie et théorie des ensembles
%
% La syntaxe est facile, par exemple 
%       $\SL(2,\eR)$\nomenclature[G]{$\SL(2,\eR)$}{Le groupe de matrices deux par deux de déterminant 1.}
%\renewcommand{\nomgroup}[1]{%
%    \ifthenelse{\equal{#1}{A}}{\item[\textbf{Algèbre}]}{}%
%    \ifthenelse{\equal{#1}{G}}{\item[\textbf{Géométrie}]}{}%
%    \ifthenelse{\equal{#1}{R}}{\item[\textbf{Théorie des groupes}]}{}%
%    \ifthenelse{\equal{#1}{P}}{\item[\textbf{Probabilités et statistique}]}{}%
%    \ifthenelse{\equal{#1}{Y}}{\item[\textbf{Analyse}]}{}%
%    \ifthenelse{\equal{#1}{M}}{\item[\textbf{Chaînes de Markov}]}{}%
%}

%%%%%%%%%%%%%%%%%%%%%%%%%%
%
%   DeclareMathOperator
%
%%%%%%%%%%%%%%%%%%%%%%%%

\DeclareMathOperator{\signe}{sgn}
\DeclareMathOperator{\Vol}{Vol}
\DeclareMathOperator{\Int}{Int}     % Intérieur d'un ensemble.
\DeclareMathOperator{\Ind}{Ind}     % l'indice d'un chemin en analyse complexe
\DeclareMathOperator{\Diam}{Diam}   
\DeclareMathOperator{\id}{Id}   
\DeclareMathOperator{\Graph}{Graph} 
\DeclareMathOperator{\pr}{\texttt{proj}}
\DeclareMathOperator{\dom}{dom}

\DeclareMathOperator{\Graphe}{Gr}
\DeclareMathOperator{\Spec}{Spec}   % spectre d'un opérateur
\DeclareMathOperator{\arctg}{arctg}
\DeclareMathOperator{\cotg}{cotg}
\DeclareMathOperator{\cosec}{cosec}
\DeclareMathOperator{\arcsinh}{arcsinh}

\DeclareMathOperator{\GL}{GL}   % le groupe linéaire
\DeclareMathOperator{\PGL}{PGL}   % le groupe projectif
\DeclareMathOperator{\SO}{SO}           
\DeclareMathOperator{\SL}{SL}           
\DeclareMathOperator{\PSL}{PSL}   % Le groupe modulaire SL(2,Z)/Z2
\DeclareMathOperator{\gO}{O}           
\DeclareMathOperator{\SU}{SU}           
\DeclareMathOperator{\gU}{U}           

\DeclareMathOperator{\Reel}{Re}        % La partie réelle d'un nombre complexe

\DeclareMathOperator{\Image}{Image}        % ... avec \Image qui donne l'image d'une fonction ou d'un opérateur.
\DeclareMathOperator{\rang}{rg}   
\DeclareMathOperator{\Kernel}{Ker}
\DeclareMathOperator{\Domaine}{Dom}
\DeclareMathOperator{\Span}{Span}
\DeclareMathOperator{\Hom}{Hom}
\DeclareMathOperator{\End}{End}     % L'ensemble des endomorphismes
\DeclareMathOperator{\tr}{Tr}       % la trace
\DeclareMathOperator{\Majorant}{Maj}
\DeclareMathOperator{\codim}{codim} % pour la codimension.
\DeclareMathOperator{\diam}{diam} % le diamètre d'un ensemble.

\DeclareMathOperator{\Var}{Var}     % Variance d'une variable aléatoire.
\DeclareMathOperator{\Fun}{\texttt{Fun}}     % Ensemble des applications d'un ensemble vers l'autre.
\DeclareMathOperator{\Cov}{Cov}     % la covariance.
\DeclareMathOperator{\gr}{gr}     % le groupe engendré
\DeclareMathOperator{\pgcd}{pgcd}     
\DeclareMathOperator{\ppcm}{ppcm}     
\DeclareMathOperator{\Frob}{Frob}     
\DeclareMathOperator{\Card}{Card}       % Le cardinal d'un ensemble.
\DeclareMathOperator{\Stab}{Stab}       % Le stabilisateur d'un point sous l'action d'un groupe.

\DeclareMathOperator{\Frac}{Frac}       % le corps des fractions d'un anneau
\DeclareMathOperator{\Aff}{Aff}         %  l'espace affine engendré

\newenvironment{subproof}{\begin{description}}{\end{description}}

%%%%%%%%%%%%% TRUCS DE YVIK POUR FAIRE FONCTIONNER CdI1 %%%%%%%%%%%%%%%%%%%%%%
%

%\newcommand{\proofend}{\hspace*{\fill} $\Box$\\}
%\newcommand{\diam}{\hspace*{\fill} $\Diamond$\\}
%\def\s{\smallskip}
%\def\m{\medskip}
%\def\my{\bf}
\newcommand{\eps}{\varepsilon}
\newcommand{\Ker}{\operatorname{Ker}}
\newcommand{\IM}{\operatorname {Im}}
\newcommand{\cat}{\operatorname{cat}}
\newcommand{\crit}{\operatorname{crit}}
\newcommand{\Crit}{\operatorname{Crit}}
\newcommand{\Rest}{\operatorname{Rest}}
\newcommand{\grad}{\operatorname{grad}}
\newcommand{\sgrad}{\operatorname{sgrad}}
\newcommand{\Fix}{\operatorname{Fix}}
\newcommand{\pt}{\operatorname{pt}}
\newcommand{\cl}{\operatorname{cl}}
\newcommand{\B}{\operatorname {B}}
\newcommand{\C}{\operatorname {C}}
%\newcommand{\S}{\operatorname {S}}
\newcommand{\Gr}{\operatorname {Gr\;\!}}
%\def\dim{\operatorname {dim}}
\newcommand{\inj}{\operatorname {inj}}
%\newcommand{\Vol}{\operatorname {Vol}\:\!}
%\newcommand{\Int}{\operatorname {Int}\:\!}
\newcommand{\dist}{\operatorname {dist}}
%\def\inter{\operatorname {int}}
\newcommand{\ext}{\operatorname {ext}}
%\newcommand{\diameter}{\operatorname {diam}\:\!}
\newcommand{\Emb}{\operatorname {Emb}}
\newcommand{\can}{\operatorname {can}}
\newcommand{\euler}{\mbox{\rm e}}
\newcommand{\sii}{\mbox{\rm \scriptsize i}}
\newcommand{\VB}{\mbox{V}_{\!\!B}}   
\newcommand{\VC}{\mbox{V}_{\!\!C}}   
\newcommand{\VS}{\mbox{V}_{\!\!S}}   
\newcommand{\f}{\frac}
\newcommand{\ga}{\alpha}
\newcommand{\gb}{\beta}
%\newcommand{\gg}{\gamma}
\newcommand{\gd}{\delta}
\newcommand{\gve}{\varepsilon}
\newcommand{\gf}{\varphi}
\newcommand{\gk}{\kappa}
\newcommand{\gkk}{\varkappa}
\newcommand{\gl}{\lambda}
\newcommand{\go}{\omega}
\newcommand{\gs}{\sigma}
\newcommand{\gt}{\vartheta}
\newcommand{\gy}{\upsilon}
\newcommand{\gv}{\varrho}
\newcommand{\gz}{\zeta}
\newcommand{\gD}{\Delta}
\newcommand{\gF}{\Phi}
\newcommand{\gG}{\Gamma}
\newcommand{\gL}{\Lambda}
%\newcommand{\gO}{\Omega}
\newcommand{\gS}{\Sigma}

%\long\def\forget#1\forgotten{} %
%\def\end{center}{{\mathfrak C}}
%\def\ea{{\mathfrak A}}

\newcommand{\ca}{{\mathcal A}}
\newcommand{\cb}{{\mathcal B}}
\newcommand{\cc}{{\mathcal C}}
\newcommand{\cd}{{\mathcal D}}
\newcommand{\ce}{{\mathcal E}}
\newcommand{\cf}{{\mathcal F}}
\newcommand{\cg}{{\mathcal G}}
\newcommand{\ch}{{\mathcal H}}
\newcommand{\cj}{{\mathcal J}}
\newcommand{\ck}{{\mathcal K}}
\newcommand{\cn}{{\mathcal N}}
\newcommand{\co}{{\mathcal O}}
\newcommand{\cp}{{\mathcal P}}
\newcommand{\cq}{{\mathcal Q}}
\newcommand{\cs}{{\mathcal S}}
\newcommand{\ct}{{\mathcal T}}
\newcommand{\cu}{{\mathcal U}}
\newcommand{\cv}{{\mathcal V}}
\newcommand{\cw}{{\mathcal W}}
\newcommand{\eb}{{\mathfrak B}}
\newcommand{\ed}{{\mathfrak D}}
\newcommand{\ee}{{\mathfrak E}}
\newcommand{\ef}{{\mathfrak F}}
\newcommand{\eg}{{\mathfrak G}}
\newcommand{\ej}{{\mathfrak J}}
\newcommand{\eh}{{\mathfrak H}}
\newcommand{\en}{{\mathfrak N}}
\newcommand{\eo}{{\mathfrak O}}
\newcommand{\ep}{{\mathfrak P}}
\newcommand{\eq}{{\mathfrak Q}}
\newcommand{\es}{{\mathfrak S}}
\newcommand{\et}{{\mathfrak T}}
\newcommand{\eu}{{\mathfrak U}}
\newcommand{\ev}{{\mathfrak V}}
\newcommand{\ew}{{\mathfrak W}}



%\def\NN{\mathbbm{N}}
%\def\QQ{\mathbbm{Q}}
%\def\RR{\mathbbm{R}}
%\def\SS{\mathbbm{S}}
%\def\11{\mathbbm{1}}
%\def\ZZ{\mathbbm{Z}}
%\def\TT{\mathbbm{T}}
\newcommand{\RR}{\eR}
\newcommand{\DD}{\mathbbm{D}}
\newcommand{\HH}{\mathbbm{H}}
\newcommand{\II}{\mathbbm{I}}
\newcommand{\N}{\mathbbm{N}}
\newcommand{\PP}{\mathbbm{P}}
\newcommand{\Q}{\mathbbm{Q}}
\newcommand{\RRR}{\mathbbm{R}_+}
\newcommand{\Z}{\mathbbm{Z}}
\newcommand{\RP}{{\RR\PP}} 
%\newcommand{\CP}{{\CC\PP}} 
\newcommand{\pp}{\partial}
\newcommand{\ww}{\wedge}
%\newcommand{\dc}{d^\CC}
\newcommand{\sym}{Sp(n;\RR)}
\newcommand{\ha}{\hookrightarrow}
\newcommand{\Ra}{\Rightarrow}
\newcommand{\Lra}{\Leftrightarrow} 

%\def\ni{\noindent}
%\def\b{\bigskip}
%\def\m{\medskip}
%\def\im{\mbox{Im}\,}

\newcommand{\de}{\stackrel{\mbox{\scriptsize{def}}}{=}}
%\newcommand{\id}{\mbox{id}}

%\def\sq{\square}
%\def\tr{\triangle}
%\def\trd{\bigtriangledown}
%\def\proof{\noindent {\it Proof. \;}}


%	La num\'erotation des exercices


\newcounter{exoNico}
\setcounter{exoNico}{1}
\newcommand{\exerNico}{\stepcounter{exoNico}{\bf Exercice }\arabic{exoNico}. }


%++++++++++ACCENTS++++++++++++++++++
\newcommand{\e}{\'{e}}
%\newcommand{\esp}{\'{e }}
%\newcommand{\eg}{\`{e}}
\newcommand{\ac}{\`{a} }
%\newcommand{\meme}{m\^{e}me }
\newcommand{\ou}{o\`{u} }

%+++++++++++NEWCOMMANDS+++++++++++
\newcommand{\dst}{\displaystyle}
\newcommand{\ba}{\begin{array}}
%\newcommand{\ea}{\end{array}}
%++++++++++FORMULAS+++++++++++++
\newcommand{\hs}{\hspace{0.3cm}}
%\newcommand{\eps}{\epsilon}
%\newcommand{\f}{\frac}
\newcommand{\arcth}{{\rm arctanh}}
\newcommand{\arcsh}{{\rm arcsinh}}
\newcommand{\arcch}{{\rm arccosh}}
\newcommand{\csec}{{\rm cosec}}
\newcommand{\cotan}{{\rm cotg}}
\newcommand{\cis}{(\cos+i\sin)( }
%\newcommand{\ra}{\rightarrow}
\newcommand{\lra}{\longrightarrow}
\newcommand{\ceil}{\rm plafond(}
\newcommand{\dfdu}{\frac{\partial f}{\partial u}}
\newcommand{\dfdw}{\frac{\partial f}{\partial w}}
\newcommand{\dfdx}{\frac{\partial f}{\partial x}}
\newcommand{\dfdy}{\frac{\partial f}{\partial y}}
\newcommand{\dudx}{\frac{\partial u}{\partial x}}
\newcommand{\dvdx}{\frac{\partial v}{\partial x}}
\newcommand{\dUdx}{\dfrac{\partial U}{\partial x}}
\newcommand{\dVdx}{\dfrac{\partial V}{\partial x}}
\newcommand{\dhdx}{\frac{\partial h}{\partial x}}
\newcommand{\dhdy}{\frac{\partial h}{\partial y}}
\newcommand{\dgdu}{\frac{\partial g}{\partial u}}
\newcommand{\dgdv}{\frac{\partial g}{\partial v}}
\newcommand{\dgudu}{\frac{\partial g_1}{\partial u}}
\newcommand{\dgudv}{\frac{\partial g_1}{\partial v}}
\newcommand{\dgddu}{\frac{\partial g_2}{\partial u}}
\newcommand{\dgddv}{\frac{\partial g_2}{\partial v}}
\newcommand{\dhdu}{\frac{\partial h}{\partial u}}
\newcommand{\dhdv}{\frac{\partial h}{\partial v}}
\newcommand{\dldu}{\frac{\partial l}{\partial u}}
\newcommand{\dldv}{\frac{\partial l}{\partial v}}
\newcommand{\dgudr}{\frac{\partial g_1}{\partial r}}
\newcommand{\dgudth}{\frac{\partial g_1}{\partial \theta}}
\newcommand{\dgddr}{\frac{\partial g_2}{\partial r}}
\newcommand{\dgddth}{\frac{\partial g_2}{\partial \theta}}
\newcommand{\dfdv}{\frac{\partial f}{\partial v}}
\newcommand{\dfdr}{\frac{\partial f}{\partial r}}

\newcommand{\dfdth}{\frac{\partial f}{\partial \theta}}
\newcommand{\ddfdx}{\frac{\partial^2 f}{\partial x^2}}
\newcommand{\ddfdy}{\frac{\partial^2 f}{\partial y^2}}
\newcommand{\ddfdxy}{\frac{\partial^2 f}{\partial y\partial x}}
\newcommand{\ddfdt}{\frac{\partial^2 f}{\partial^2 t}}

\newcommand{\ud}{\underline}

 % *** Blackboard math symbols ***
 %\newcommand{\N}{\mathbb{N}}
 %\newcommand{\Z}{\mathbb{Z}}
 %\newcommand{\Q}{\mathbb{R}}
 %\newcommand{\K}{\mathbb{K}}
 %\newcommand{\R}{\mathbb{R}}
 %\newcommand{\C}{\mathbb{C}}
 %\newcommand{\F}{\mathbb{F}}
 %\newcommand{\J}{\mathbb{J}}
\newcommand{\Qn}{\mathbb{Q}}

\newcommand{\Rn}{\eR} 
\newcommand{\Nn}{\eN}


\newtheorem{theo}{Th{\'e}or{\`e}me}[section]
\newtheorem{defn}{D{\'e}finition}
\newtheorem{prop}{Proposition}     % redef encore dans Chafaï
%\newtheorem{rem}{Remarque}[section]
\newtheorem{lem}{Lemme}[section]
\newcommand{\R}{\mathbb{R}}
\newcommand{\dem}{\textbf{D{\'e}monstration.}}
\newcommand{\vc}[1]{\boldsymbol{#1}}
\newcommand{\p}{\textrm{P}}
%\newcommand{\e}{\textrm{E}}
\newcommand{\mbt}{arbre binaire markovien}
\newcommand{\mbts}{arbres binaires markoviens}

\newcommand{\ea}{\end{array}}


%%%%%%%%%%%%%%%%%%%%%%%%%%%%%%%%%%%%%%
%
% les petis yeux 
%
%%%%%%%%%%%%%%%%%%%%%%%%%%%%%%%%%%%%%%%%%%%%%

\newcommand{\coolexo}{$\circledast\circledast$}
\newcommand{\boringexo}{$\circleddash\circleddash$}
\newcommand{\minsyndical}{$\odot\odot$}
\newcommand{\mortelexo}{$\obslash\oslash$}


%%%%%%%%%%%%%% FIN TRUCS DE YVIK %%%%%%%%%%%%%%%%%%%%%%

%%%%%%%%%%%%%% TRUCS DE PIERRE %%%%%%%%%%%%%%%%%%%%%%


% Le paquet array est là pour faire fonctionner l'environement arrowcases dans les trucs de Pierre.
\usepackage{array}

%\documentclass[11pt,a4paper,openany]{book}
%\usepackage[ansinew]{inputenc}
%\usepackage{pstricks, pst-node, array, ifpdf, comment, pst-plot}
%\usepackage[marginparwidth=2cm]{geometry}
%\usepackage[dvips,colorlinks]{hyperref}
%\usepackage[frenchb]{entetes}

%\usepackage{bigcenter}
%%%% debut macro %%%%
%%% ----------debut de bigcenter.sty--------------

%%% nouvel environnement bigcenter
%%% pour centrer sur toute la page (sans overfull)
%\makeatletter
%\newskip\@bigflushglue \@bigflushglue = -100pt plus 1fil

%\def\bigcenter{\trivlist \bigcentering\item\relax}
%\def\bigcentering{\let\\\@centercr\rightskip\@bigflushglue%
%\def\endbigcenter{\endtrivlist}

%\leftskip\@bigflushglue
%\parindent\z@\parfillskip\z@skip}
%\makeatother

%%% ----------fin de bigcenter.sty--------------
%%%% fin macro %%%%

%\input{mfpic}

% À régler par l'utilisateur
\newlength{\arrowsep}\setlength{\arrowsep}{3pt}
\newlength{\arrowlength}\setlength{\arrowlength}{1cm}

% Cet environnement est sympa, mais il dépend trop de ps; en tout cas il ne passe pas dans pdflatex
% 15 mars 1012
%\newenvironment{arrowcases}	{%
%			\pnode(\arrowsep,0.5ex){A}%
%			\hspace{\arrowlength}%
%			\begin{array}{>{\displaystyle\pnode(-\arrowsep,0.5ex){B}}l<{\ncline{A}{B}}@{}}
%				}
%			{
%			  \end{array}
%			}

\newenvironment{arrowcases}%
{\begin{cases}}
{\end{cases}}



\makeatletter %% \limite[condition]x x_0
\newcommand*{\limite}[3][\@empty]{\lim_{\substack{#2\rightarrow#3\\#1}}}
\makeatother

% \newenvironment{split+justif}{%
% \begin{split}%
% \let\ampori&
% \def&#1&#2\\{}
% }{%

% \end{split}}%

%\def\ncov{\tilde\nabla} % Nouvelle dérivée covariante
\newcommand*\sev{<} % 

%\newcommand{\hgot}{\mathfrak{h}} % h gothique (ss algebre de Lie)
%\def\var#1{{\mathbf #1}} % \var <-> Une variété
%\def\pardef{\stackrel{def}{=}} % = par définition.
%\newcommand{\bbar#1}{\bar{\bar{#1}}}
%\def\cov{\nabla} % Derivee covariante / connexion
%\newcommand{\gl}{\mathfrak{gl}} % algèbre linéaire
%\def\doubleprime{{\prime\prime}} % Isomorphique 
%\def\scal(#1,#2){\langle #1,#2\rangle}
%\def\agit(#1,#2){\langle #1,#2^\vee\rangle}
%\let\phiori\phi
%\let\phi\varphi
\let\ssi\iff
%\def\iddc{\mathcal I}
\newcommand*{\ideal}[1]{\{#1\}}
\newcommand*{\fleche}[1]{\stackrel{#1}\longrightarrow}

%\newcounter{exercice}
%\setcounter{exercice}{0}
\setcounter{CountExercice}{0}

% \newenvironment{exo}[1][\relax]{%
% \stepcounter{exercice}%
% \par\medskip%
% #1{\textbf{Exercice}~\arabic{exercice}.}\quad}%
% {\par}

% \newenvironment{rep}{\hspace{1em}\par\textbf{Solution
%     proposée.\quad}}{\par\noindent\hrulefill\par}

%\date{}
\newcommand{\Acplx}{A_\cdot}
\newcommand{\Bcplx}{B_\cdot}
\newcommand{\toisom}{\fleche\simeq}
\newcommand{\D}{\partial}
%\newcommand{\cat}[1]{{\bf #1}}
%\newcommand{\donc}{\Rightarrow}
%\newcommand{\im}{\text{im}}
%\newcommand{\coker}{\text{coker}}
\newcommand{\lied}{\mathcal L}
\newcommand*{\nom}[1]{\textsc{#1}}
\makeatletter
\newcommand*{\attention}[1]{\@latex@warning{#1}{!\small\bf #1!}\marginpar{Warning}}
\makeatother
\newcommand*{\inner}{\imath}
\newcommand*{\newexo}{}
\newcommand*{\principe}{}
\newcommand*{\etape}{}
\newcommand*{\preuve}{}
\newcommand*{\exr}{\item}
%\def\prim#1\expandafter\d#2 {\int #1\d#2}

\newcommand*{\crochets}[1]{\Bigl[ #1 \Bigr]}
\newcommand*{\llbrack}[1]{\left\lbrack #1 \right\lbrack}
\newcommand*{\rlbrack}[1]{\left\rbrack #1 \right\lbrack}
\newcommand*{\lrbrack}[1]{\left\lbrack #1 \right\rbrack}
\newcommand*{\rrbrack}[1]{\left\rbrack #1 \right\rbrack}
\newcommand*{\vecteur}[1]{\mathbf{#1}}


%% Maths : Les ensembles
\newcommand*{\ens}[1]{\mathbb{#1}} % Ensemble de nombres
\newcommand*{\var}[1]{\mathbf{#1}} % Variété
\newcommand*{\alg}[1]{\mathcal{#1}} % Algèbre
%\newcommand*{\RR}{\ens R}%
\newcommand*{\TT}{\ens T}% Tore !
%\expandafter\show\csname SS \endcsname
%\renewcommand*{\SS}{\var S}% 
%\newcommand*{\CC}{\ens C}%
\newcommand*{\ZZ}{\ens Z}%
\newcommand*{\QQ}{\ens Q}%
\newcommand*{\NN}{\ens N}%
\newcommand{\schwartz}{\mathcal S} % Espace de Schwartz
\newcommand*{\topologie}{\mathscr{T}}
\newcommand*{\Topologie}{\textcursive{T}}
\newcommand{\LL}{\text{\textup{L}}} %% Espace de Lebesgue droit
\newcommand{\Ll}{\mathcal{L}} %% Lebesgue ronde
\newcommand{\fronde}{\mathcal{F}} %% Transformée de Fourier.
\newcommand{\sigmaalgebre}[1]{\mathcal{#1}} %% Une sigma algèbre...
\DeclareMathOperator{\SymMatrix}{Sym}
\DeclareMathOperator{\ASymMatrix}{ASym}
\newcommand{\Sym}{\SymMatrix}
\newcommand{\ASym}{\ASymMatrix}
\newcommand{\transpose}[1]{{\vphantom{#1}}^{\mathit t}{\/#1}}
\newcommand*{\Sp}{\textup{Sp}}
\newcommand*{\Gl}{\textup{GL}}
%\renewcommand*{\sp}{\textup{sp}}
\newcommand*{\dprime}{{\prime\prime}}
%\show\span
%\newcommand*{\Span}[1]{\mathopen> #1 \mathclose<}

%% Maths : Symboles divers
%\newcommand{\pp}{\text{\textup{~p.p.}}} %% Presque partout
%\PackageWarning{entetes}{Redefining command \d}
%\renewcommand{\d}{\mbox{$\,$\textrm{d}}}
\newcommand{\surj}{\vers}
\newcommand{\isom}{\simeq}
\newcommand*{\Tau}{\alg T}
\newcommand{\cdv}{\mathfrak{X}} % Champs de vecteurs


%% Maths : Constructions

%\let\Exp\exp
%\renewcommand{\exp}[1]{e^{#1}} % On préfère e^{} que exp{}

%\renewcommand{\exp}[1]{e^{#1}} % On préfère e^{} que exp{}
%\renewcommand{\vec}[1]{\mathbf{#1}} % Désigner un vecteur
\newcommand{\set}[1]{\left\{#1\right\}} % Un ensemble { }
\newcommand*{\abs}[1]{\left\vert#1\right\vert} % Valeur absolue.
\newcommand*{\module}[1]{\left\vert#1\right\vert} % Valeur absolue.
\newcommand*{\norme}[1]{\left\Vert#1\right\Vert} % norme
\newcommand*{\ordre}[1]{\left\vert#1\right\vert} % L'ordre d'un élément.
%\def\scal(#1,#2){% Produit scalaire.
%  \PackageWarning{entetes}{Obsolete command \string\scal}%
%  \scalprod{#1}{#2}%
%}
\newcommand*{\scalprod}[2]{\left\langle #1,#2\right\rangle}
\let\dual\ast

\newcommand*{\pardef}{\stackrel{\text{def}}{=}} % Par définition.
\newcommand*{\iffdefn}{\stackrel{\text{def}}{\iff}} % Par définition.
%\newcommand*{\telque}{\mbox{~\entetes@name@telque~}} % tel que, dans un ensemble.
\newcommand*{\Defn}[1]{\emph{#1}} %
\newcommand*{\tensor}{\otimes}
\newcommand*{\pder}[2]{\frac{\partial #1}{\partial #2}}

%{{{ Fraction in-line plus jolie
% \DeclareRobustCommand\sfrac[1]{\@ifnextchar/{\@sfrac{#1}}%
%                                             {\@sfrac{#1}/}}
% \def\@sfrac#1/#2{\leavevmode\kern.1em\raise.5ex
%          \hbox{$\m@th{\fontsize\sf@size\z@
%                            \selectfont#1}$}\kern-.1em
%          /\kern-.15em\lower.25ex
%           \hbox{$\m@th{\fontsize\sf@size\z@
%                             \selectfont#2}$}}
%}}} 

\DeclareRobustCommand{\sfrac}[3][\mathrm]{\hspace{0.1em}%
  \raisebox{0.4ex}{$#1{\scriptstyle
#2}$}\hspace{-0.1em}/\hspace{-0.07em}%
  \mbox{$#1{\scriptstyle #3}$}}



%% Maths : Opérateurs
%\DeclareMathOperator{\tr}{Tr}
%\DeclareMathOperator{\pr}{\texttt{pr}}
\DeclareMathOperator{\supp}{supp}
\DeclareMathOperator{\adh}{adh}
\DeclareMathOperator{\interior}{int}
\DeclareMathOperator{\im}{Im}
\DeclareMathOperator{\Id}{Id}
\DeclareMathOperator{\Aut}{Aut}
\DeclareMathOperator{\Iso}{Iso}
\DeclareMathOperator{\Jac}{Jac} % jacobienne
\DeclareMathOperator{\coker}{coker}
\DeclareMathOperator{\interieur}{int}
\DeclareMathOperator{\Tor}{Tor}
\DeclareMathOperator{\divg}{div}
\DeclareMathOperator{\rot}{rot}
%\DeclareMathOperator{\cosec}{cosec}


%% Pour obtenir le \Sha cyrillique...
% \RequirePackage[OT2,T1]{fontenc}
% \DeclareSymbolFont{cyrletters}{OT2}{wncyr}{m}{n}
% \DeclareMathSymbol{\Sha}{\mathalpha}{cyrletters}{"58}


% \newcounter{@institute}
% \let\authorori\author
% \def\@institute{}\def\@auteurs{}
% %\newcommand{\institute}[2]{\refstepcounter{@institute}\label{#1}\def\@institute{\@institute\small
% %#2}\set@authors}
% \newcommand{\institute}[2]{\refstepcounter{@institute}\label{#1}%
%   \let\maketitleori\maketitle%
%   \renewcommand\maketitle{\footnote{#2}\maketitleori}%
% }%
% \renewcommand{\author}[1]{\def\@auteurs{#1}\set@authors}
% \def\the@institute{${}^{(\roman{@institute})}$}
% \newcommand{\inst}[1]{\ref{#1}}

% \newcommand{\set@authors}{\authorori{\@auteurs}}% \\ \@institute}}


% \newcounter{@institute}
% \newcommand{\institute}[1]{
%   \let\labelori\label
%   \renewcommand{\label}[1]{%
%     \refstepcounter{@institute}\labelori{##1}
%     \begin{tabular}{cc}%format ?!
%     \begin{minipage}[t]
      


%   }

% }%

\newcommand*{\conclusion}{\emph{Conclusion~:~}}
\newcommand{\hint}{\par\emph{Aide~:~}\hspace{1em}}
\newcommand{\rappel}{\par\emph{Rappels~:~}\hspace{1em}}

%\newcounter{enumarray} 
%\newenvironment{enumarray}[1]{% Merci Ulrike Fischer
% \setcounter{enumarray}{0}%
% \begin{array}{% motif
%     >{% Au début de chaque ligne
%       \stepcounter{enumarray}%
%       (\alph{enumarray})%
%       \hspace{2em}
%     }% 
%     #1%
%   }% fin motif
% }{%
% \end{array}%
%}
\newenvironment{displayinline}{% displaystyle + inline.
  $\displaystyle%
}{%
  $%
}

\newcommand{\telque}{\vert\,}
\newcommand{\donc}{\Rightarrow}


\pagestyle{empty}   % Pour éviter les numéros de page.

\begin{document}

\vbox{Numéro 1.
\emph{Toutes les réponses doivent être justifiées par un calcul accompagné d'un raisonnement.}
\begin{enumerate}\item
Soient les points $A(4;-7)$, $ B(5,10)$ et $ C(1;-6)$. 
    \begin{enumerate}
    \item
    
    Calculer les coordonnées des vecteurs \( \vect{ AB }\) et \( \vect{ AB }+\vect{ BC }\). 

\item
    Donner les coordonnées du point \( X\) tel que \( \vect{ AX }=\vect{ BC }\) (méthode au choix)
    \end{enumerate}
    
    
\item

    Soient les points $D(1;-3)$, $L(4;0)$ et $B(-4;7)$. Donner les coordonnées du point $K$ tel que $DLBK$ soit un parallélogramme (méthode au choix).
    

\end{enumerate}
}
\vspace{2cm}
\vbox{Numéro 2.
\emph{Toutes les réponses doivent être justifiées par un calcul accompagné d'un raisonnement.}
\begin{enumerate}\item
Soient les points $A(8;-1)$, $ B(4,-7)$ et $ C(8;-5)$. 
    \begin{enumerate}
    \item
    
    Calculer les coordonnées des vecteurs \( \vect{ AB }\) et \( \vect{ AB }+\vect{ BC }\). 

\item
    Donner les coordonnées du point \( X\) tel que \( \vect{ AX }=\vect{ BC }\) (méthode au choix)
    \end{enumerate}
    
    
\item

    Soient les points $E(-10;2)$, $B(8;-5)$ et $D(3;10)$. Donner les coordonnées du point $F$ tel que $EBDF$ soit un parallélogramme (méthode au choix).
    

\end{enumerate}
}
\vspace{2cm}
\vbox{Numéro 3.
\emph{Toutes les réponses doivent être justifiées par un calcul accompagné d'un raisonnement.}
\begin{enumerate}\item

    Soient les points $F(9;0)$, $L(6;3)$ et $E(1;6)$. Donner les coordonnées du point $M$ tel que $FLEM$ soit un parallélogramme (méthode au choix).
    
\item
Soient les points $A(1;-5)$, $ B(7,-6)$ et $ C(1;-4)$. 
    \begin{enumerate}
    \item
    
    Calculer les coordonnées des vecteurs \( \vect{ AB }\) et \( \vect{ AB }+\vect{ BC }\). 

\item
    Donner les coordonnées du point \( X\) tel que \( \vect{ AX }=\vect{ BC }\) (méthode au choix)
    \end{enumerate}
    
    

\end{enumerate}
}
\vspace{2cm}
\vbox{Numéro 4.
\emph{Toutes les réponses doivent être justifiées par un calcul accompagné d'un raisonnement.}
\begin{enumerate}\item
Soient les points $A(-8;9)$, $ B(-3,-7)$ et $ C(-9;-8)$. 
    \begin{enumerate}
    \item
    
    Calculer les coordonnées des vecteurs \( \vect{ AB }\) et \( \vect{ AB }+\vect{ BC }\). 

\item
    Donner les coordonnées du point \( X\) tel que \( \vect{ AX }=\vect{ BC }\) (méthode au choix)
    \end{enumerate}
    
    
\item

    Soient les points $L(-7;7)$, $A(7;-8)$ et $F(-3;-8)$. Donner les coordonnées du point $D$ tel que $LAFD$ soit un parallélogramme (méthode au choix).
    

\end{enumerate}
}
\vspace{2cm}
\vbox{Numéro 5.
\emph{Toutes les réponses doivent être justifiées par un calcul accompagné d'un raisonnement.}
\begin{enumerate}\item

    Soient les points $K(0;8)$, $A(0;-7)$ et $D(3;4)$. Donner les coordonnées du point $B$ tel que $KADB$ soit un parallélogramme (méthode au choix).
    
\item
Soient les points $A(-4;3)$, $ B(8,-6)$ et $ C(-5;-7)$. 
    \begin{enumerate}
    \item
    
    Calculer les coordonnées des vecteurs \( \vect{ AB }\) et \( \vect{ AB }+\vect{ BC }\). 

\item
    Donner les coordonnées du point \( X\) tel que \( \vect{ AX }=\vect{ BC }\) (méthode au choix)
    \end{enumerate}
    
    

\end{enumerate}
}
\vspace{2cm}
\vbox{Numéro 6.
\emph{Toutes les réponses doivent être justifiées par un calcul accompagné d'un raisonnement.}
\begin{enumerate}\item

    Soient les points $F(10;-5)$, $M(-7;-5)$ et $B(5;-7)$. Donner les coordonnées du point $D$ tel que $FMBD$ soit un parallélogramme (méthode au choix).
    
\item
Soient les points $A(1;-5)$, $ B(9,10)$ et $ C(-6;3)$. 
    \begin{enumerate}
    \item
    
    Calculer les coordonnées des vecteurs \( \vect{ AB }\) et \( \vect{ AB }+\vect{ BC }\). 

\item
    Donner les coordonnées du point \( X\) tel que \( \vect{ AX }=\vect{ BC }\) (méthode au choix)
    \end{enumerate}
    
    

\end{enumerate}
}
\vspace{2cm}
\vbox{Numéro 7.
\emph{Toutes les réponses doivent être justifiées par un calcul accompagné d'un raisonnement.}
\begin{enumerate}\item
Soient les points $A(-9;-2)$, $ B(1,10)$ et $ C(1;4)$. 
    \begin{enumerate}
    \item
    
    Calculer les coordonnées des vecteurs \( \vect{ AB }\) et \( \vect{ AB }+\vect{ BC }\). 

\item
    Donner les coordonnées du point \( X\) tel que \( \vect{ AX }=\vect{ BC }\) (méthode au choix)
    \end{enumerate}
    
    
\item

    Soient les points $M(3;7)$, $B(-6;-5)$ et $A(1;10)$. Donner les coordonnées du point $L$ tel que $MBAL$ soit un parallélogramme (méthode au choix).
    

\end{enumerate}
}
\vspace{2cm}
\vbox{Numéro 8.
\emph{Toutes les réponses doivent être justifiées par un calcul accompagné d'un raisonnement.}
\begin{enumerate}\item

    Soient les points $A(-10;-7)$, $F(-5;0)$ et $D(3;4)$. Donner les coordonnées du point $E$ tel que $AFDE$ soit un parallélogramme (méthode au choix).
    
\item
Soient les points $A(8;2)$, $ B(-5,-1)$ et $ C(4;-9)$. 
    \begin{enumerate}
    \item
    
    Calculer les coordonnées des vecteurs \( \vect{ AB }\) et \( \vect{ AB }+\vect{ BC }\). 

\item
    Donner les coordonnées du point \( X\) tel que \( \vect{ AX }=\vect{ BC }\) (méthode au choix)
    \end{enumerate}
    
    

\end{enumerate}
}
\vspace{2cm}
\vbox{Numéro 9.
\emph{Toutes les réponses doivent être justifiées par un calcul accompagné d'un raisonnement.}
\begin{enumerate}\item

    Soient les points $B(10;0)$, $K(6;-7)$ et $D(10;-8)$. Donner les coordonnées du point $L$ tel que $BKDL$ soit un parallélogramme (méthode au choix).
    
\item
Soient les points $A(-1;4)$, $ B(-7,2)$ et $ C(5;5)$. 
    \begin{enumerate}
    \item
    
    Calculer les coordonnées des vecteurs \( \vect{ AB }\) et \( \vect{ AB }+\vect{ BC }\). 

\item
    Donner les coordonnées du point \( X\) tel que \( \vect{ AX }=\vect{ BC }\) (méthode au choix)
    \end{enumerate}
    
    

\end{enumerate}
}
\vspace{2cm}
\vbox{Numéro 10.
\emph{Toutes les réponses doivent être justifiées par un calcul accompagné d'un raisonnement.}
\begin{enumerate}\item
Soient les points $A(-2;-1)$, $ B(-10,-1)$ et $ C(-7;-7)$. 
    \begin{enumerate}
    \item
    
    Calculer les coordonnées des vecteurs \( \vect{ AB }\) et \( \vect{ AB }+\vect{ BC }\). 

\item
    Donner les coordonnées du point \( X\) tel que \( \vect{ AX }=\vect{ BC }\) (méthode au choix)
    \end{enumerate}
    
    
\item

    Soient les points $A(2;4)$, $F(-9;-9)$ et $B(-3;3)$. Donner les coordonnées du point $L$ tel que $AFBL$ soit un parallélogramme (méthode au choix).
    

\end{enumerate}
}
\vspace{2cm}
\vbox{Numéro 11.
\emph{Toutes les réponses doivent être justifiées par un calcul accompagné d'un raisonnement.}
\begin{enumerate}\item

    Soient les points $F(-7;6)$, $D(-3;-5)$ et $M(-4;10)$. Donner les coordonnées du point $L$ tel que $FDML$ soit un parallélogramme (méthode au choix).
    
\item
Soient les points $A(2;-3)$, $ B(0,-6)$ et $ C(3;-2)$. 
    \begin{enumerate}
    \item
    
    Calculer les coordonnées des vecteurs \( \vect{ AB }\) et \( \vect{ AB }+\vect{ BC }\). 

\item
    Donner les coordonnées du point \( X\) tel que \( \vect{ AX }=\vect{ BC }\) (méthode au choix)
    \end{enumerate}
    
    

\end{enumerate}
}
\vspace{2cm}
\vbox{Numéro 12.
\emph{Toutes les réponses doivent être justifiées par un calcul accompagné d'un raisonnement.}
\begin{enumerate}\item

    Soient les points $F(9;3)$, $L(10;4)$ et $A(-5;-6)$. Donner les coordonnées du point $D$ tel que $FLAD$ soit un parallélogramme (méthode au choix).
    
\item
Soient les points $A(10;2)$, $ B(7,-9)$ et $ C(-1;-8)$. 
    \begin{enumerate}
    \item
    
    Calculer les coordonnées des vecteurs \( \vect{ AB }\) et \( \vect{ AB }+\vect{ BC }\). 

\item
    Donner les coordonnées du point \( X\) tel que \( \vect{ AX }=\vect{ BC }\) (méthode au choix)
    \end{enumerate}
    
    

\end{enumerate}
}
\vspace{2cm}
\vbox{Numéro 13.
\emph{Toutes les réponses doivent être justifiées par un calcul accompagné d'un raisonnement.}
\begin{enumerate}\item
Soient les points $A(-1;-10)$, $ B(7,4)$ et $ C(-5;-10)$. 
    \begin{enumerate}
    \item
    
    Calculer les coordonnées des vecteurs \( \vect{ AB }\) et \( \vect{ AB }+\vect{ BC }\). 

\item
    Donner les coordonnées du point \( X\) tel que \( \vect{ AX }=\vect{ BC }\) (méthode au choix)
    \end{enumerate}
    
    
\item

    Soient les points $E(-10;-5)$, $L(8;-3)$ et $D(3;2)$. Donner les coordonnées du point $A$ tel que $ELDA$ soit un parallélogramme (méthode au choix).
    

\end{enumerate}
}
\vspace{2cm}
\vbox{Numéro 14.
\emph{Toutes les réponses doivent être justifiées par un calcul accompagné d'un raisonnement.}
\begin{enumerate}\item
Soient les points $A(5;-10)$, $ B(-7,-10)$ et $ C(9;6)$. 
    \begin{enumerate}
    \item
    
    Calculer les coordonnées des vecteurs \( \vect{ AB }\) et \( \vect{ AB }+\vect{ BC }\). 

\item
    Donner les coordonnées du point \( X\) tel que \( \vect{ AX }=\vect{ BC }\) (méthode au choix)
    \end{enumerate}
    
    
\item

    Soient les points $F(-9;8)$, $K(-8;-3)$ et $L(-2;-2)$. Donner les coordonnées du point $A$ tel que $FKLA$ soit un parallélogramme (méthode au choix).
    

\end{enumerate}
}
\vspace{2cm}
\vbox{Numéro 15.
\emph{Toutes les réponses doivent être justifiées par un calcul accompagné d'un raisonnement.}
\begin{enumerate}\item

    Soient les points $D(7;-6)$, $L(3;-7)$ et $K(0;-3)$. Donner les coordonnées du point $A$ tel que $DLKA$ soit un parallélogramme (méthode au choix).
    
\item
Soient les points $A(7;-9)$, $ B(-6,-2)$ et $ C(-7;10)$. 
    \begin{enumerate}
    \item
    
    Calculer les coordonnées des vecteurs \( \vect{ AB }\) et \( \vect{ AB }+\vect{ BC }\). 

\item
    Donner les coordonnées du point \( X\) tel que \( \vect{ AX }=\vect{ BC }\) (méthode au choix)
    \end{enumerate}
    
    

\end{enumerate}
}
\vspace{2cm}
\vbox{Numéro 16.
\emph{Toutes les réponses doivent être justifiées par un calcul accompagné d'un raisonnement.}
\begin{enumerate}\item

    Soient les points $F(-5;8)$, $M(6;5)$ et $D(5;10)$. Donner les coordonnées du point $K$ tel que $FMDK$ soit un parallélogramme (méthode au choix).
    
\item
Soient les points $A(9;-8)$, $ B(9,0)$ et $ C(0;-3)$. 
    \begin{enumerate}
    \item
    
    Calculer les coordonnées des vecteurs \( \vect{ AB }\) et \( \vect{ AB }+\vect{ BC }\). 

\item
    Donner les coordonnées du point \( X\) tel que \( \vect{ AX }=\vect{ BC }\) (méthode au choix)
    \end{enumerate}
    
    

\end{enumerate}
}
\vspace{2cm}
\vbox{Numéro 17.
\emph{Toutes les réponses doivent être justifiées par un calcul accompagné d'un raisonnement.}
\begin{enumerate}\item
Soient les points $A(-4;4)$, $ B(-6,-8)$ et $ C(10;6)$. 
    \begin{enumerate}
    \item
    
    Calculer les coordonnées des vecteurs \( \vect{ AB }\) et \( \vect{ AB }+\vect{ BC }\). 

\item
    Donner les coordonnées du point \( X\) tel que \( \vect{ AX }=\vect{ BC }\) (méthode au choix)
    \end{enumerate}
    
    
\item

    Soient les points $K(-7;-10)$, $M(-7;5)$ et $E(9;2)$. Donner les coordonnées du point $D$ tel que $KMED$ soit un parallélogramme (méthode au choix).
    

\end{enumerate}
}
\vspace{2cm}
\vbox{Numéro 18.
\emph{Toutes les réponses doivent être justifiées par un calcul accompagné d'un raisonnement.}
\begin{enumerate}\item

    Soient les points $L(4;-7)$, $F(5;0)$ et $A(2;8)$. Donner les coordonnées du point $M$ tel que $LFAM$ soit un parallélogramme (méthode au choix).
    
\item
Soient les points $A(-10;-3)$, $ B(10,-2)$ et $ C(7;6)$. 
    \begin{enumerate}
    \item
    
    Calculer les coordonnées des vecteurs \( \vect{ AB }\) et \( \vect{ AB }+\vect{ BC }\). 

\item
    Donner les coordonnées du point \( X\) tel que \( \vect{ AX }=\vect{ BC }\) (méthode au choix)
    \end{enumerate}
    
    

\end{enumerate}
}
\vspace{2cm}
\vbox{Numéro 19.
\emph{Toutes les réponses doivent être justifiées par un calcul accompagné d'un raisonnement.}
\begin{enumerate}\item

    Soient les points $B(1;-9)$, $E(8;0)$ et $L(10;-4)$. Donner les coordonnées du point $M$ tel que $BELM$ soit un parallélogramme (méthode au choix).
    
\item
Soient les points $A(2;-9)$, $ B(5,5)$ et $ C(4;-10)$. 
    \begin{enumerate}
    \item
    
    Calculer les coordonnées des vecteurs \( \vect{ AB }\) et \( \vect{ AB }+\vect{ BC }\). 

\item
    Donner les coordonnées du point \( X\) tel que \( \vect{ AX }=\vect{ BC }\) (méthode au choix)
    \end{enumerate}
    
    

\end{enumerate}
}
\vspace{2cm}
\vbox{Numéro 20.
\emph{Toutes les réponses doivent être justifiées par un calcul accompagné d'un raisonnement.}
\begin{enumerate}\item
Soient les points $A(9;3)$, $ B(8,-2)$ et $ C(-6;-3)$. 
    \begin{enumerate}
    \item
    
    Calculer les coordonnées des vecteurs \( \vect{ AB }\) et \( \vect{ AB }+\vect{ BC }\). 

\item
    Donner les coordonnées du point \( X\) tel que \( \vect{ AX }=\vect{ BC }\) (méthode au choix)
    \end{enumerate}
    
    
\item

    Soient les points $L(0;-6)$, $F(-6;9)$ et $E(7;-5)$. Donner les coordonnées du point $D$ tel que $LFED$ soit un parallélogramme (méthode au choix).
    

\end{enumerate}
}
\vspace{2cm}
\vbox{Numéro 21.
\emph{Toutes les réponses doivent être justifiées par un calcul accompagné d'un raisonnement.}
\begin{enumerate}\item
Soient les points $A(9;-4)$, $ B(-8,9)$ et $ C(-10;9)$. 
    \begin{enumerate}
    \item
    
    Calculer les coordonnées des vecteurs \( \vect{ AB }\) et \( \vect{ AB }+\vect{ BC }\). 

\item
    Donner les coordonnées du point \( X\) tel que \( \vect{ AX }=\vect{ BC }\) (méthode au choix)
    \end{enumerate}
    
    
\item

    Soient les points $D(-9;1)$, $E(-6;-3)$ et $A(-9;-3)$. Donner les coordonnées du point $L$ tel que $DEAL$ soit un parallélogramme (méthode au choix).
    

\end{enumerate}
}
\vspace{2cm}
\vbox{Numéro 22.
\emph{Toutes les réponses doivent être justifiées par un calcul accompagné d'un raisonnement.}
\begin{enumerate}\item
Soient les points $A(-4;-3)$, $ B(-8,-9)$ et $ C(9;5)$. 
    \begin{enumerate}
    \item
    
    Calculer les coordonnées des vecteurs \( \vect{ AB }\) et \( \vect{ AB }+\vect{ BC }\). 

\item
    Donner les coordonnées du point \( X\) tel que \( \vect{ AX }=\vect{ BC }\) (méthode au choix)
    \end{enumerate}
    
    
\item

    Soient les points $F(1;-10)$, $B(-8;-10)$ et $L(-3;3)$. Donner les coordonnées du point $M$ tel que $FBLM$ soit un parallélogramme (méthode au choix).
    

\end{enumerate}
}
\vspace{2cm}
\vbox{Numéro 23.
\emph{Toutes les réponses doivent être justifiées par un calcul accompagné d'un raisonnement.}
\begin{enumerate}\item

    Soient les points $A(8;-8)$, $B(-6;5)$ et $L(5;7)$. Donner les coordonnées du point $E$ tel que $ABLE$ soit un parallélogramme (méthode au choix).
    
\item
Soient les points $A(1;3)$, $ B(-4,5)$ et $ C(10;-4)$. 
    \begin{enumerate}
    \item
    
    Calculer les coordonnées des vecteurs \( \vect{ AB }\) et \( \vect{ AB }+\vect{ BC }\). 

\item
    Donner les coordonnées du point \( X\) tel que \( \vect{ AX }=\vect{ BC }\) (méthode au choix)
    \end{enumerate}
    
    

\end{enumerate}
}
\vspace{2cm}
\vbox{Numéro 24.
\emph{Toutes les réponses doivent être justifiées par un calcul accompagné d'un raisonnement.}
\begin{enumerate}\item

    Soient les points $F(-5;3)$, $K(1;3)$ et $A(7;6)$. Donner les coordonnées du point $L$ tel que $FKAL$ soit un parallélogramme (méthode au choix).
    
\item
Soient les points $A(2;0)$, $ B(-2,5)$ et $ C(-6;-7)$. 
    \begin{enumerate}
    \item
    
    Calculer les coordonnées des vecteurs \( \vect{ AB }\) et \( \vect{ AB }+\vect{ BC }\). 

\item
    Donner les coordonnées du point \( X\) tel que \( \vect{ AX }=\vect{ BC }\) (méthode au choix)
    \end{enumerate}
    
    

\end{enumerate}
}
\vspace{2cm}
\vbox{Numéro 25.
\emph{Toutes les réponses doivent être justifiées par un calcul accompagné d'un raisonnement.}
\begin{enumerate}\item
Soient les points $A(-8;-8)$, $ B(0,1)$ et $ C(9;3)$. 
    \begin{enumerate}
    \item
    
    Calculer les coordonnées des vecteurs \( \vect{ AB }\) et \( \vect{ AB }+\vect{ BC }\). 

\item
    Donner les coordonnées du point \( X\) tel que \( \vect{ AX }=\vect{ BC }\) (méthode au choix)
    \end{enumerate}
    
    
\item

    Soient les points $L(-7;4)$, $M(9;-1)$ et $F(-3;-8)$. Donner les coordonnées du point $A$ tel que $LMFA$ soit un parallélogramme (méthode au choix).
    

\end{enumerate}
}
\vspace{2cm}
\vbox{Numéro 26.
\emph{Toutes les réponses doivent être justifiées par un calcul accompagné d'un raisonnement.}
\begin{enumerate}\item
Soient les points $A(0;2)$, $ B(-10,-2)$ et $ C(2;-3)$. 
    \begin{enumerate}
    \item
    
    Calculer les coordonnées des vecteurs \( \vect{ AB }\) et \( \vect{ AB }+\vect{ BC }\). 

\item
    Donner les coordonnées du point \( X\) tel que \( \vect{ AX }=\vect{ BC }\) (méthode au choix)
    \end{enumerate}
    
    
\item

    Soient les points $A(-9;-9)$, $K(-2;9)$ et $E(-7;1)$. Donner les coordonnées du point $B$ tel que $AKEB$ soit un parallélogramme (méthode au choix).
    

\end{enumerate}
}
\vspace{2cm}
\vbox{Numéro 27.
\emph{Toutes les réponses doivent être justifiées par un calcul accompagné d'un raisonnement.}
\begin{enumerate}\item
Soient les points $A(6;7)$, $ B(9,7)$ et $ C(9;5)$. 
    \begin{enumerate}
    \item
    
    Calculer les coordonnées des vecteurs \( \vect{ AB }\) et \( \vect{ AB }+\vect{ BC }\). 

\item
    Donner les coordonnées du point \( X\) tel que \( \vect{ AX }=\vect{ BC }\) (méthode au choix)
    \end{enumerate}
    
    
\item

    Soient les points $A(-7;-6)$, $D(2;-2)$ et $M(10;-4)$. Donner les coordonnées du point $K$ tel que $ADMK$ soit un parallélogramme (méthode au choix).
    

\end{enumerate}
}
\vspace{2cm}
\vbox{Numéro 28.
\emph{Toutes les réponses doivent être justifiées par un calcul accompagné d'un raisonnement.}
\begin{enumerate}\item

    Soient les points $F(-4;2)$, $E(2;-1)$ et $B(2;-10)$. Donner les coordonnées du point $L$ tel que $FEBL$ soit un parallélogramme (méthode au choix).
    
\item
Soient les points $A(-5;2)$, $ B(-6,8)$ et $ C(-7;-1)$. 
    \begin{enumerate}
    \item
    
    Calculer les coordonnées des vecteurs \( \vect{ AB }\) et \( \vect{ AB }+\vect{ BC }\). 

\item
    Donner les coordonnées du point \( X\) tel que \( \vect{ AX }=\vect{ BC }\) (méthode au choix)
    \end{enumerate}
    
    

\end{enumerate}
}
\vspace{2cm}
\vbox{Numéro 29.
\emph{Toutes les réponses doivent être justifiées par un calcul accompagné d'un raisonnement.}
\begin{enumerate}\item

    Soient les points $L(-4;-4)$, $A(-7;-6)$ et $K(9;-7)$. Donner les coordonnées du point $D$ tel que $LAKD$ soit un parallélogramme (méthode au choix).
    
\item
Soient les points $A(8;-2)$, $ B(9,5)$ et $ C(9;6)$. 
    \begin{enumerate}
    \item
    
    Calculer les coordonnées des vecteurs \( \vect{ AB }\) et \( \vect{ AB }+\vect{ BC }\). 

\item
    Donner les coordonnées du point \( X\) tel que \( \vect{ AX }=\vect{ BC }\) (méthode au choix)
    \end{enumerate}
    
    

\end{enumerate}
}
\vspace{2cm}
\vbox{Numéro 30.
\emph{Toutes les réponses doivent être justifiées par un calcul accompagné d'un raisonnement.}
\begin{enumerate}\item
Soient les points $A(4;-9)$, $ B(10,4)$ et $ C(10;-8)$. 
    \begin{enumerate}
    \item
    
    Calculer les coordonnées des vecteurs \( \vect{ AB }\) et \( \vect{ AB }+\vect{ BC }\). 

\item
    Donner les coordonnées du point \( X\) tel que \( \vect{ AX }=\vect{ BC }\) (méthode au choix)
    \end{enumerate}
    
    
\item

    Soient les points $K(-5;4)$, $A(-4;-5)$ et $E(5;4)$. Donner les coordonnées du point $B$ tel que $KAEB$ soit un parallélogramme (méthode au choix).
    

\end{enumerate}
}
\vspace{2cm}
\vbox{Numéro 31.
\emph{Toutes les réponses doivent être justifiées par un calcul accompagné d'un raisonnement.}
\begin{enumerate}\item
Soient les points $A(-4;1)$, $ B(2,-9)$ et $ C(-3;-2)$. 
    \begin{enumerate}
    \item
    
    Calculer les coordonnées des vecteurs \( \vect{ AB }\) et \( \vect{ AB }+\vect{ BC }\). 

\item
    Donner les coordonnées du point \( X\) tel que \( \vect{ AX }=\vect{ BC }\) (méthode au choix)
    \end{enumerate}
    
    
\item

    Soient les points $B(-1;3)$, $K(-3;-9)$ et $A(-10;-6)$. Donner les coordonnées du point $D$ tel que $BKAD$ soit un parallélogramme (méthode au choix).
    

\end{enumerate}
}
\vspace{2cm}
\vbox{Numéro 32.
\emph{Toutes les réponses doivent être justifiées par un calcul accompagné d'un raisonnement.}
\begin{enumerate}\item

    Soient les points $A(-8;8)$, $L(9;3)$ et $D(3;-6)$. Donner les coordonnées du point $M$ tel que $ALDM$ soit un parallélogramme (méthode au choix).
    
\item
Soient les points $A(9;0)$, $ B(-1,4)$ et $ C(6;-4)$. 
    \begin{enumerate}
    \item
    
    Calculer les coordonnées des vecteurs \( \vect{ AB }\) et \( \vect{ AB }+\vect{ BC }\). 

\item
    Donner les coordonnées du point \( X\) tel que \( \vect{ AX }=\vect{ BC }\) (méthode au choix)
    \end{enumerate}
    
    

\end{enumerate}
}
\vspace{2cm}
\vbox{Numéro 33.
\emph{Toutes les réponses doivent être justifiées par un calcul accompagné d'un raisonnement.}
\begin{enumerate}\item
Soient les points $A(1;-9)$, $ B(3,6)$ et $ C(8;6)$. 
    \begin{enumerate}
    \item
    
    Calculer les coordonnées des vecteurs \( \vect{ AB }\) et \( \vect{ AB }+\vect{ BC }\). 

\item
    Donner les coordonnées du point \( X\) tel que \( \vect{ AX }=\vect{ BC }\) (méthode au choix)
    \end{enumerate}
    
    
\item

    Soient les points $A(6;-5)$, $D(7;-7)$ et $B(-2;-7)$. Donner les coordonnées du point $M$ tel que $ADBM$ soit un parallélogramme (méthode au choix).
    

\end{enumerate}
}
\vspace{2cm}
\vbox{Numéro 34.
\emph{Toutes les réponses doivent être justifiées par un calcul accompagné d'un raisonnement.}
\begin{enumerate}\item
Soient les points $A(-1;1)$, $ B(0,-7)$ et $ C(5;-1)$. 
    \begin{enumerate}
    \item
    
    Calculer les coordonnées des vecteurs \( \vect{ AB }\) et \( \vect{ AB }+\vect{ BC }\). 

\item
    Donner les coordonnées du point \( X\) tel que \( \vect{ AX }=\vect{ BC }\) (méthode au choix)
    \end{enumerate}
    
    
\item

    Soient les points $F(-1;3)$, $M(2;8)$ et $E(0;-3)$. Donner les coordonnées du point $K$ tel que $FMEK$ soit un parallélogramme (méthode au choix).
    

\end{enumerate}
}
\vspace{2cm}
\vbox{Numéro 35.
\emph{Toutes les réponses doivent être justifiées par un calcul accompagné d'un raisonnement.}
\begin{enumerate}\item

    Soient les points $F(9;-4)$, $L(5;-5)$ et $E(-3;-9)$. Donner les coordonnées du point $D$ tel que $FLED$ soit un parallélogramme (méthode au choix).
    
\item
Soient les points $A(-6;2)$, $ B(-5,-3)$ et $ C(-6;-1)$. 
    \begin{enumerate}
    \item
    
    Calculer les coordonnées des vecteurs \( \vect{ AB }\) et \( \vect{ AB }+\vect{ BC }\). 

\item
    Donner les coordonnées du point \( X\) tel que \( \vect{ AX }=\vect{ BC }\) (méthode au choix)
    \end{enumerate}
    
    

\end{enumerate}
}
\vspace{2cm}
\vbox{Numéro 36.
\emph{Toutes les réponses doivent être justifiées par un calcul accompagné d'un raisonnement.}
\begin{enumerate}\item
Soient les points $A(3;-5)$, $ B(7,-10)$ et $ C(-9;6)$. 
    \begin{enumerate}
    \item
    
    Calculer les coordonnées des vecteurs \( \vect{ AB }\) et \( \vect{ AB }+\vect{ BC }\). 

\item
    Donner les coordonnées du point \( X\) tel que \( \vect{ AX }=\vect{ BC }\) (méthode au choix)
    \end{enumerate}
    
    
\item

    Soient les points $M(10;-8)$, $F(-5;-4)$ et $L(0;6)$. Donner les coordonnées du point $K$ tel que $MFLK$ soit un parallélogramme (méthode au choix).
    

\end{enumerate}
}
\vspace{2cm}
\vbox{Numéro 37.
\emph{Toutes les réponses doivent être justifiées par un calcul accompagné d'un raisonnement.}
\begin{enumerate}\item

    Soient les points $F(10;-3)$, $B(4;-7)$ et $M(10;1)$. Donner les coordonnées du point $L$ tel que $FBML$ soit un parallélogramme (méthode au choix).
    
\item
Soient les points $A(-1;9)$, $ B(1,6)$ et $ C(0;-1)$. 
    \begin{enumerate}
    \item
    
    Calculer les coordonnées des vecteurs \( \vect{ AB }\) et \( \vect{ AB }+\vect{ BC }\). 

\item
    Donner les coordonnées du point \( X\) tel que \( \vect{ AX }=\vect{ BC }\) (méthode au choix)
    \end{enumerate}
    
    

\end{enumerate}
}
\vspace{2cm}
\vbox{Numéro 38.
\emph{Toutes les réponses doivent être justifiées par un calcul accompagné d'un raisonnement.}
\begin{enumerate}\item
Soient les points $A(-1;4)$, $ B(1,-4)$ et $ C(4;-6)$. 
    \begin{enumerate}
    \item
    
    Calculer les coordonnées des vecteurs \( \vect{ AB }\) et \( \vect{ AB }+\vect{ BC }\). 

\item
    Donner les coordonnées du point \( X\) tel que \( \vect{ AX }=\vect{ BC }\) (méthode au choix)
    \end{enumerate}
    
    
\item

    Soient les points $K(-5;6)$, $L(-6;-2)$ et $E(-5;-1)$. Donner les coordonnées du point $M$ tel que $KLEM$ soit un parallélogramme (méthode au choix).
    

\end{enumerate}
}
\vspace{2cm}
\vbox{Numéro 39.
\emph{Toutes les réponses doivent être justifiées par un calcul accompagné d'un raisonnement.}
\begin{enumerate}\item

    Soient les points $E(-4;9)$, $L(-5;-2)$ et $B(-6;10)$. Donner les coordonnées du point $D$ tel que $ELBD$ soit un parallélogramme (méthode au choix).
    
\item
Soient les points $A(6;-8)$, $ B(10,10)$ et $ C(1;1)$. 
    \begin{enumerate}
    \item
    
    Calculer les coordonnées des vecteurs \( \vect{ AB }\) et \( \vect{ AB }+\vect{ BC }\). 

\item
    Donner les coordonnées du point \( X\) tel que \( \vect{ AX }=\vect{ BC }\) (méthode au choix)
    \end{enumerate}
    
    

\end{enumerate}
}
\vspace{2cm}
\vbox{Numéro 40.
\emph{Toutes les réponses doivent être justifiées par un calcul accompagné d'un raisonnement.}
\begin{enumerate}\item
Soient les points $A(-7;4)$, $ B(4,-10)$ et $ C(-2;-1)$. 
    \begin{enumerate}
    \item
    
    Calculer les coordonnées des vecteurs \( \vect{ AB }\) et \( \vect{ AB }+\vect{ BC }\). 

\item
    Donner les coordonnées du point \( X\) tel que \( \vect{ AX }=\vect{ BC }\) (méthode au choix)
    \end{enumerate}
    
    
\item

    Soient les points $F(8;9)$, $K(-9;-7)$ et $D(-8;7)$. Donner les coordonnées du point $B$ tel que $FKDB$ soit un parallélogramme (méthode au choix).
    

\end{enumerate}
}
\vspace{2cm}
\vbox{Numéro 41.
\emph{Toutes les réponses doivent être justifiées par un calcul accompagné d'un raisonnement.}
\begin{enumerate}\item

    Soient les points $A(-10;9)$, $F(4;-6)$ et $E(-10;0)$. Donner les coordonnées du point $M$ tel que $AFEM$ soit un parallélogramme (méthode au choix).
    
\item
Soient les points $A(-1;1)$, $ B(6,-8)$ et $ C(-4;-4)$. 
    \begin{enumerate}
    \item
    
    Calculer les coordonnées des vecteurs \( \vect{ AB }\) et \( \vect{ AB }+\vect{ BC }\). 

\item
    Donner les coordonnées du point \( X\) tel que \( \vect{ AX }=\vect{ BC }\) (méthode au choix)
    \end{enumerate}
    
    

\end{enumerate}
}
\vspace{2cm}
\vbox{Numéro 42.
\emph{Toutes les réponses doivent être justifiées par un calcul accompagné d'un raisonnement.}
\begin{enumerate}\item

    Soient les points $M(-8;5)$, $E(-5;10)$ et $L(-6;1)$. Donner les coordonnées du point $B$ tel que $MELB$ soit un parallélogramme (méthode au choix).
    
\item
Soient les points $A(0;-5)$, $ B(-6,8)$ et $ C(-3;-1)$. 
    \begin{enumerate}
    \item
    
    Calculer les coordonnées des vecteurs \( \vect{ AB }\) et \( \vect{ AB }+\vect{ BC }\). 

\item
    Donner les coordonnées du point \( X\) tel que \( \vect{ AX }=\vect{ BC }\) (méthode au choix)
    \end{enumerate}
    
    

\end{enumerate}
}
\vspace{2cm}
\vbox{Numéro 43.
\emph{Toutes les réponses doivent être justifiées par un calcul accompagné d'un raisonnement.}
\begin{enumerate}\item

    Soient les points $E(-2;-6)$, $D(-3;6)$ et $A(1;7)$. Donner les coordonnées du point $B$ tel que $EDAB$ soit un parallélogramme (méthode au choix).
    
\item
Soient les points $A(-1;-5)$, $ B(0,-3)$ et $ C(-4;0)$. 
    \begin{enumerate}
    \item
    
    Calculer les coordonnées des vecteurs \( \vect{ AB }\) et \( \vect{ AB }+\vect{ BC }\). 

\item
    Donner les coordonnées du point \( X\) tel que \( \vect{ AX }=\vect{ BC }\) (méthode au choix)
    \end{enumerate}
    
    

\end{enumerate}
}
\vspace{2cm}
\vbox{Numéro 44.
\emph{Toutes les réponses doivent être justifiées par un calcul accompagné d'un raisonnement.}
\begin{enumerate}\item
Soient les points $A(9;5)$, $ B(-9,-8)$ et $ C(6;-7)$. 
    \begin{enumerate}
    \item
    
    Calculer les coordonnées des vecteurs \( \vect{ AB }\) et \( \vect{ AB }+\vect{ BC }\). 

\item
    Donner les coordonnées du point \( X\) tel que \( \vect{ AX }=\vect{ BC }\) (méthode au choix)
    \end{enumerate}
    
    
\item

    Soient les points $L(7;9)$, $A(-7;9)$ et $K(10;-8)$. Donner les coordonnées du point $E$ tel que $LAKE$ soit un parallélogramme (méthode au choix).
    

\end{enumerate}
}
\vspace{2cm}
\vbox{Numéro 45.
\emph{Toutes les réponses doivent être justifiées par un calcul accompagné d'un raisonnement.}
\begin{enumerate}\item
Soient les points $A(2;-10)$, $ B(-8,7)$ et $ C(-7;5)$. 
    \begin{enumerate}
    \item
    
    Calculer les coordonnées des vecteurs \( \vect{ AB }\) et \( \vect{ AB }+\vect{ BC }\). 

\item
    Donner les coordonnées du point \( X\) tel que \( \vect{ AX }=\vect{ BC }\) (méthode au choix)
    \end{enumerate}
    
    
\item

    Soient les points $L(-5;-5)$, $F(-10;3)$ et $K(-10;9)$. Donner les coordonnées du point $B$ tel que $LFKB$ soit un parallélogramme (méthode au choix).
    

\end{enumerate}
}
\vspace{2cm}
\vbox{Numéro 46.
\emph{Toutes les réponses doivent être justifiées par un calcul accompagné d'un raisonnement.}
\begin{enumerate}\item

    Soient les points $M(10;3)$, $B(-3;1)$ et $K(-10;-5)$. Donner les coordonnées du point $E$ tel que $MBKE$ soit un parallélogramme (méthode au choix).
    
\item
Soient les points $A(4;-6)$, $ B(-2,4)$ et $ C(8;-10)$. 
    \begin{enumerate}
    \item
    
    Calculer les coordonnées des vecteurs \( \vect{ AB }\) et \( \vect{ AB }+\vect{ BC }\). 

\item
    Donner les coordonnées du point \( X\) tel que \( \vect{ AX }=\vect{ BC }\) (méthode au choix)
    \end{enumerate}
    
    

\end{enumerate}
}
\vspace{2cm}
\vbox{Numéro 47.
\emph{Toutes les réponses doivent être justifiées par un calcul accompagné d'un raisonnement.}
\begin{enumerate}\item

    Soient les points $L(-4;6)$, $F(4;9)$ et $M(-10;-3)$. Donner les coordonnées du point $D$ tel que $LFMD$ soit un parallélogramme (méthode au choix).
    
\item
Soient les points $A(-6;-1)$, $ B(10,-2)$ et $ C(-8;-1)$. 
    \begin{enumerate}
    \item
    
    Calculer les coordonnées des vecteurs \( \vect{ AB }\) et \( \vect{ AB }+\vect{ BC }\). 

\item
    Donner les coordonnées du point \( X\) tel que \( \vect{ AX }=\vect{ BC }\) (méthode au choix)
    \end{enumerate}
    
    

\end{enumerate}
}
\vspace{2cm}
\vbox{Numéro 48.
\emph{Toutes les réponses doivent être justifiées par un calcul accompagné d'un raisonnement.}
\begin{enumerate}\item

    Soient les points $D(-9;9)$, $F(7;3)$ et $B(6;10)$. Donner les coordonnées du point $E$ tel que $DFBE$ soit un parallélogramme (méthode au choix).
    
\item
Soient les points $A(4;-1)$, $ B(-10,7)$ et $ C(10;5)$. 
    \begin{enumerate}
    \item
    
    Calculer les coordonnées des vecteurs \( \vect{ AB }\) et \( \vect{ AB }+\vect{ BC }\). 

\item
    Donner les coordonnées du point \( X\) tel que \( \vect{ AX }=\vect{ BC }\) (méthode au choix)
    \end{enumerate}
    
    

\end{enumerate}
}
\vspace{2cm}
\vbox{Numéro 49.
\emph{Toutes les réponses doivent être justifiées par un calcul accompagné d'un raisonnement.}
\begin{enumerate}\item

    Soient les points $D(-4;-9)$, $L(1;6)$ et $K(-4;-4)$. Donner les coordonnées du point $E$ tel que $DLKE$ soit un parallélogramme (méthode au choix).
    
\item
Soient les points $A(6;-2)$, $ B(2,-6)$ et $ C(-6;-1)$. 
    \begin{enumerate}
    \item
    
    Calculer les coordonnées des vecteurs \( \vect{ AB }\) et \( \vect{ AB }+\vect{ BC }\). 

\item
    Donner les coordonnées du point \( X\) tel que \( \vect{ AX }=\vect{ BC }\) (méthode au choix)
    \end{enumerate}
    
    

\end{enumerate}
}
\vspace{2cm}
\vbox{Numéro 50.
\emph{Toutes les réponses doivent être justifiées par un calcul accompagné d'un raisonnement.}
\begin{enumerate}\item

    Soient les points $F(-10;2)$, $B(5;3)$ et $L(-8;-10)$. Donner les coordonnées du point $E$ tel que $FBLE$ soit un parallélogramme (méthode au choix).
    
\item
Soient les points $A(10;-8)$, $ B(3,-2)$ et $ C(0;-8)$. 
    \begin{enumerate}
    \item
    
    Calculer les coordonnées des vecteurs \( \vect{ AB }\) et \( \vect{ AB }+\vect{ BC }\). 

\item
    Donner les coordonnées du point \( X\) tel que \( \vect{ AX }=\vect{ BC }\) (méthode au choix)
    \end{enumerate}
    
    

\end{enumerate}
}
\vspace{2cm}
\vbox{Numéro 51.
\emph{Toutes les réponses doivent être justifiées par un calcul accompagné d'un raisonnement.}
\begin{enumerate}\item

    Soient les points $E(-6;10)$, $K(7;-1)$ et $A(1;-2)$. Donner les coordonnées du point $L$ tel que $EKAL$ soit un parallélogramme (méthode au choix).
    
\item
Soient les points $A(-7;1)$, $ B(6,5)$ et $ C(2;4)$. 
    \begin{enumerate}
    \item
    
    Calculer les coordonnées des vecteurs \( \vect{ AB }\) et \( \vect{ AB }+\vect{ BC }\). 

\item
    Donner les coordonnées du point \( X\) tel que \( \vect{ AX }=\vect{ BC }\) (méthode au choix)
    \end{enumerate}
    
    

\end{enumerate}
}
\vspace{2cm}
\vbox{Numéro 52.
\emph{Toutes les réponses doivent être justifiées par un calcul accompagné d'un raisonnement.}
\begin{enumerate}\item
Soient les points $A(6;-5)$, $ B(-4,6)$ et $ C(-4;-6)$. 
    \begin{enumerate}
    \item
    
    Calculer les coordonnées des vecteurs \( \vect{ AB }\) et \( \vect{ AB }+\vect{ BC }\). 

\item
    Donner les coordonnées du point \( X\) tel que \( \vect{ AX }=\vect{ BC }\) (méthode au choix)
    \end{enumerate}
    
    
\item

    Soient les points $M(-8;-3)$, $K(9;-10)$ et $D(-4;-6)$. Donner les coordonnées du point $B$ tel que $MKDB$ soit un parallélogramme (méthode au choix).
    

\end{enumerate}
}
\vspace{2cm}
\vbox{Numéro 53.
\emph{Toutes les réponses doivent être justifiées par un calcul accompagné d'un raisonnement.}
\begin{enumerate}\item

    Soient les points $D(8;9)$, $A(-2;2)$ et $B(1;-8)$. Donner les coordonnées du point $M$ tel que $DABM$ soit un parallélogramme (méthode au choix).
    
\item
Soient les points $A(-5;6)$, $ B(-2,0)$ et $ C(4;10)$. 
    \begin{enumerate}
    \item
    
    Calculer les coordonnées des vecteurs \( \vect{ AB }\) et \( \vect{ AB }+\vect{ BC }\). 

\item
    Donner les coordonnées du point \( X\) tel que \( \vect{ AX }=\vect{ BC }\) (méthode au choix)
    \end{enumerate}
    
    

\end{enumerate}
}
\vspace{2cm}
\vbox{Numéro 54.
\emph{Toutes les réponses doivent être justifiées par un calcul accompagné d'un raisonnement.}
\begin{enumerate}\item

    Soient les points $A(4;-6)$, $B(-5;3)$ et $K(-3;-8)$. Donner les coordonnées du point $D$ tel que $ABKD$ soit un parallélogramme (méthode au choix).
    
\item
Soient les points $A(1;3)$, $ B(0,6)$ et $ C(3;-4)$. 
    \begin{enumerate}
    \item
    
    Calculer les coordonnées des vecteurs \( \vect{ AB }\) et \( \vect{ AB }+\vect{ BC }\). 

\item
    Donner les coordonnées du point \( X\) tel que \( \vect{ AX }=\vect{ BC }\) (méthode au choix)
    \end{enumerate}
    
    

\end{enumerate}
}
\vspace{2cm}
\vbox{Numéro 55.
\emph{Toutes les réponses doivent être justifiées par un calcul accompagné d'un raisonnement.}
\begin{enumerate}\item
Soient les points $A(-7;7)$, $ B(-2,-8)$ et $ C(-10;-7)$. 
    \begin{enumerate}
    \item
    
    Calculer les coordonnées des vecteurs \( \vect{ AB }\) et \( \vect{ AB }+\vect{ BC }\). 

\item
    Donner les coordonnées du point \( X\) tel que \( \vect{ AX }=\vect{ BC }\) (méthode au choix)
    \end{enumerate}
    
    
\item

    Soient les points $M(-8;1)$, $B(-5;9)$ et $L(3;9)$. Donner les coordonnées du point $E$ tel que $MBLE$ soit un parallélogramme (méthode au choix).
    

\end{enumerate}
}
\vspace{2cm}
\vbox{Numéro 56.
\emph{Toutes les réponses doivent être justifiées par un calcul accompagné d'un raisonnement.}
\begin{enumerate}\item
Soient les points $A(-1;0)$, $ B(-6,8)$ et $ C(4;-10)$. 
    \begin{enumerate}
    \item
    
    Calculer les coordonnées des vecteurs \( \vect{ AB }\) et \( \vect{ AB }+\vect{ BC }\). 

\item
    Donner les coordonnées du point \( X\) tel que \( \vect{ AX }=\vect{ BC }\) (méthode au choix)
    \end{enumerate}
    
    
\item

    Soient les points $D(1;-7)$, $E(6;2)$ et $B(-4;-9)$. Donner les coordonnées du point $F$ tel que $DEBF$ soit un parallélogramme (méthode au choix).
    

\end{enumerate}
}
\vspace{2cm}
\vbox{Numéro 57.
\emph{Toutes les réponses doivent être justifiées par un calcul accompagné d'un raisonnement.}
\begin{enumerate}\item

    Soient les points $E(-2;-4)$, $B(-6;-10)$ et $A(-6;-7)$. Donner les coordonnées du point $L$ tel que $EBAL$ soit un parallélogramme (méthode au choix).
    
\item
Soient les points $A(-8;4)$, $ B(8,-7)$ et $ C(8;6)$. 
    \begin{enumerate}
    \item
    
    Calculer les coordonnées des vecteurs \( \vect{ AB }\) et \( \vect{ AB }+\vect{ BC }\). 

\item
    Donner les coordonnées du point \( X\) tel que \( \vect{ AX }=\vect{ BC }\) (méthode au choix)
    \end{enumerate}
    
    

\end{enumerate}
}
\vspace{2cm}
\vbox{Numéro 58.
\emph{Toutes les réponses doivent être justifiées par un calcul accompagné d'un raisonnement.}
\begin{enumerate}\item

    Soient les points $L(1;-4)$, $D(0;8)$ et $K(7;-9)$. Donner les coordonnées du point $B$ tel que $LDKB$ soit un parallélogramme (méthode au choix).
    
\item
Soient les points $A(2;-6)$, $ B(-3,3)$ et $ C(-8;8)$. 
    \begin{enumerate}
    \item
    
    Calculer les coordonnées des vecteurs \( \vect{ AB }\) et \( \vect{ AB }+\vect{ BC }\). 

\item
    Donner les coordonnées du point \( X\) tel que \( \vect{ AX }=\vect{ BC }\) (méthode au choix)
    \end{enumerate}
    
    

\end{enumerate}
}
\vspace{2cm}
\vbox{Numéro 59.
\emph{Toutes les réponses doivent être justifiées par un calcul accompagné d'un raisonnement.}
\begin{enumerate}\item
Soient les points $A(2;6)$, $ B(-8,-3)$ et $ C(3;-9)$. 
    \begin{enumerate}
    \item
    
    Calculer les coordonnées des vecteurs \( \vect{ AB }\) et \( \vect{ AB }+\vect{ BC }\). 

\item
    Donner les coordonnées du point \( X\) tel que \( \vect{ AX }=\vect{ BC }\) (méthode au choix)
    \end{enumerate}
    
    
\item

    Soient les points $A(7;0)$, $D(7;6)$ et $L(-8;-1)$. Donner les coordonnées du point $F$ tel que $ADLF$ soit un parallélogramme (méthode au choix).
    

\end{enumerate}
}
\vspace{2cm}
\vbox{Numéro 60.
\emph{Toutes les réponses doivent être justifiées par un calcul accompagné d'un raisonnement.}
\begin{enumerate}\item
Soient les points $A(-2;8)$, $ B(1,10)$ et $ C(2;-7)$. 
    \begin{enumerate}
    \item
    
    Calculer les coordonnées des vecteurs \( \vect{ AB }\) et \( \vect{ AB }+\vect{ BC }\). 

\item
    Donner les coordonnées du point \( X\) tel que \( \vect{ AX }=\vect{ BC }\) (méthode au choix)
    \end{enumerate}
    
    
\item

    Soient les points $K(2;-5)$, $L(-7;9)$ et $E(6;-10)$. Donner les coordonnées du point $B$ tel que $KLEB$ soit un parallélogramme (méthode au choix).
    

\end{enumerate}
}
\vspace{2cm}
\vbox{Numéro 61.
\emph{Toutes les réponses doivent être justifiées par un calcul accompagné d'un raisonnement.}
\begin{enumerate}\item

    Soient les points $F(9;-5)$, $A(9;-8)$ et $D(-2;4)$. Donner les coordonnées du point $M$ tel que $FADM$ soit un parallélogramme (méthode au choix).
    
\item
Soient les points $A(-6;1)$, $ B(-1,1)$ et $ C(6;-2)$. 
    \begin{enumerate}
    \item
    
    Calculer les coordonnées des vecteurs \( \vect{ AB }\) et \( \vect{ AB }+\vect{ BC }\). 

\item
    Donner les coordonnées du point \( X\) tel que \( \vect{ AX }=\vect{ BC }\) (méthode au choix)
    \end{enumerate}
    
    

\end{enumerate}
}
\vspace{2cm}
\vbox{Numéro 62.
\emph{Toutes les réponses doivent être justifiées par un calcul accompagné d'un raisonnement.}
\begin{enumerate}\item

    Soient les points $E(0;-1)$, $L(1;-5)$ et $F(-9;5)$. Donner les coordonnées du point $K$ tel que $ELFK$ soit un parallélogramme (méthode au choix).
    
\item
Soient les points $A(3;-6)$, $ B(0,-10)$ et $ C(-8;-2)$. 
    \begin{enumerate}
    \item
    
    Calculer les coordonnées des vecteurs \( \vect{ AB }\) et \( \vect{ AB }+\vect{ BC }\). 

\item
    Donner les coordonnées du point \( X\) tel que \( \vect{ AX }=\vect{ BC }\) (méthode au choix)
    \end{enumerate}
    
    

\end{enumerate}
}
\vspace{2cm}
\vbox{Numéro 63.
\emph{Toutes les réponses doivent être justifiées par un calcul accompagné d'un raisonnement.}
\begin{enumerate}\item
Soient les points $A(5;-6)$, $ B(-9,7)$ et $ C(-10;-4)$. 
    \begin{enumerate}
    \item
    
    Calculer les coordonnées des vecteurs \( \vect{ AB }\) et \( \vect{ AB }+\vect{ BC }\). 

\item
    Donner les coordonnées du point \( X\) tel que \( \vect{ AX }=\vect{ BC }\) (méthode au choix)
    \end{enumerate}
    
    
\item

    Soient les points $M(5;-3)$, $L(-6;-3)$ et $B(-2;-6)$. Donner les coordonnées du point $E$ tel que $MLBE$ soit un parallélogramme (méthode au choix).
    

\end{enumerate}
}
\vspace{2cm}
\vbox{Numéro 64.
\emph{Toutes les réponses doivent être justifiées par un calcul accompagné d'un raisonnement.}
\begin{enumerate}\item
Soient les points $A(-7;1)$, $ B(-6,2)$ et $ C(-7;-5)$. 
    \begin{enumerate}
    \item
    
    Calculer les coordonnées des vecteurs \( \vect{ AB }\) et \( \vect{ AB }+\vect{ BC }\). 

\item
    Donner les coordonnées du point \( X\) tel que \( \vect{ AX }=\vect{ BC }\) (méthode au choix)
    \end{enumerate}
    
    
\item

    Soient les points $L(5;5)$, $M(7;10)$ et $E(8;-2)$. Donner les coordonnées du point $F$ tel que $LMEF$ soit un parallélogramme (méthode au choix).
    

\end{enumerate}
}
\vspace{2cm}
\vbox{Numéro 65.
\emph{Toutes les réponses doivent être justifiées par un calcul accompagné d'un raisonnement.}
\begin{enumerate}\item

    Soient les points $M(-3;-9)$, $D(-2;0)$ et $A(5;1)$. Donner les coordonnées du point $F$ tel que $MDAF$ soit un parallélogramme (méthode au choix).
    
\item
Soient les points $A(-4;9)$, $ B(0,0)$ et $ C(-6;-7)$. 
    \begin{enumerate}
    \item
    
    Calculer les coordonnées des vecteurs \( \vect{ AB }\) et \( \vect{ AB }+\vect{ BC }\). 

\item
    Donner les coordonnées du point \( X\) tel que \( \vect{ AX }=\vect{ BC }\) (méthode au choix)
    \end{enumerate}
    
    

\end{enumerate}
}
\vspace{2cm}
\vbox{Numéro 66.
\emph{Toutes les réponses doivent être justifiées par un calcul accompagné d'un raisonnement.}
\begin{enumerate}\item
Soient les points $A(2;-10)$, $ B(1,-1)$ et $ C(6;6)$. 
    \begin{enumerate}
    \item
    
    Calculer les coordonnées des vecteurs \( \vect{ AB }\) et \( \vect{ AB }+\vect{ BC }\). 

\item
    Donner les coordonnées du point \( X\) tel que \( \vect{ AX }=\vect{ BC }\) (méthode au choix)
    \end{enumerate}
    
    
\item

    Soient les points $K(-7;-1)$, $B(6;-2)$ et $M(-5;2)$. Donner les coordonnées du point $E$ tel que $KBME$ soit un parallélogramme (méthode au choix).
    

\end{enumerate}
}
\vspace{2cm}
\vbox{Numéro 67.
\emph{Toutes les réponses doivent être justifiées par un calcul accompagné d'un raisonnement.}
\begin{enumerate}\item

    Soient les points $L(-10;10)$, $F(0;1)$ et $E(0;6)$. Donner les coordonnées du point $K$ tel que $LFEK$ soit un parallélogramme (méthode au choix).
    
\item
Soient les points $A(-5;10)$, $ B(7,-5)$ et $ C(7;5)$. 
    \begin{enumerate}
    \item
    
    Calculer les coordonnées des vecteurs \( \vect{ AB }\) et \( \vect{ AB }+\vect{ BC }\). 

\item
    Donner les coordonnées du point \( X\) tel que \( \vect{ AX }=\vect{ BC }\) (méthode au choix)
    \end{enumerate}
    
    

\end{enumerate}
}
\vspace{2cm}
\vbox{Numéro 68.
\emph{Toutes les réponses doivent être justifiées par un calcul accompagné d'un raisonnement.}
\begin{enumerate}\item

    Soient les points $D(-7;-2)$, $F(-6;2)$ et $K(9;-8)$. Donner les coordonnées du point $E$ tel que $DFKE$ soit un parallélogramme (méthode au choix).
    
\item
Soient les points $A(7;-6)$, $ B(6,-2)$ et $ C(-9;0)$. 
    \begin{enumerate}
    \item
    
    Calculer les coordonnées des vecteurs \( \vect{ AB }\) et \( \vect{ AB }+\vect{ BC }\). 

\item
    Donner les coordonnées du point \( X\) tel que \( \vect{ AX }=\vect{ BC }\) (méthode au choix)
    \end{enumerate}
    
    

\end{enumerate}
}
\vspace{2cm}
\vbox{Numéro 69.
\emph{Toutes les réponses doivent être justifiées par un calcul accompagné d'un raisonnement.}
\begin{enumerate}\item
Soient les points $A(-1;-9)$, $ B(5,-8)$ et $ C(4;8)$. 
    \begin{enumerate}
    \item
    
    Calculer les coordonnées des vecteurs \( \vect{ AB }\) et \( \vect{ AB }+\vect{ BC }\). 

\item
    Donner les coordonnées du point \( X\) tel que \( \vect{ AX }=\vect{ BC }\) (méthode au choix)
    \end{enumerate}
    
    
\item

    Soient les points $E(-3;3)$, $F(-4;5)$ et $D(-3;10)$. Donner les coordonnées du point $A$ tel que $EFDA$ soit un parallélogramme (méthode au choix).
    

\end{enumerate}
}
\vspace{2cm}
\vbox{Numéro 70.
\emph{Toutes les réponses doivent être justifiées par un calcul accompagné d'un raisonnement.}
\begin{enumerate}\item

    Soient les points $K(-2;-6)$, $F(8;9)$ et $B(5;10)$. Donner les coordonnées du point $M$ tel que $KFBM$ soit un parallélogramme (méthode au choix).
    
\item
Soient les points $A(5;8)$, $ B(-10,-6)$ et $ C(2;6)$. 
    \begin{enumerate}
    \item
    
    Calculer les coordonnées des vecteurs \( \vect{ AB }\) et \( \vect{ AB }+\vect{ BC }\). 

\item
    Donner les coordonnées du point \( X\) tel que \( \vect{ AX }=\vect{ BC }\) (méthode au choix)
    \end{enumerate}
    
    

\end{enumerate}
}
\vspace{2cm}
\vbox{Numéro 71.
\emph{Toutes les réponses doivent être justifiées par un calcul accompagné d'un raisonnement.}
\begin{enumerate}\item
Soient les points $A(-10;8)$, $ B(-9,10)$ et $ C(-6;-1)$. 
    \begin{enumerate}
    \item
    
    Calculer les coordonnées des vecteurs \( \vect{ AB }\) et \( \vect{ AB }+\vect{ BC }\). 

\item
    Donner les coordonnées du point \( X\) tel que \( \vect{ AX }=\vect{ BC }\) (méthode au choix)
    \end{enumerate}
    
    
\item

    Soient les points $K(9;1)$, $L(-2;-8)$ et $F(-6;-7)$. Donner les coordonnées du point $A$ tel que $KLFA$ soit un parallélogramme (méthode au choix).
    

\end{enumerate}
}
\vspace{2cm}
\vbox{Numéro 72.
\emph{Toutes les réponses doivent être justifiées par un calcul accompagné d'un raisonnement.}
\begin{enumerate}\item

    Soient les points $M(8;10)$, $E(-9;-6)$ et $L(-4;2)$. Donner les coordonnées du point $F$ tel que $MELF$ soit un parallélogramme (méthode au choix).
    
\item
Soient les points $A(5;10)$, $ B(-7,-10)$ et $ C(1;8)$. 
    \begin{enumerate}
    \item
    
    Calculer les coordonnées des vecteurs \( \vect{ AB }\) et \( \vect{ AB }+\vect{ BC }\). 

\item
    Donner les coordonnées du point \( X\) tel que \( \vect{ AX }=\vect{ BC }\) (méthode au choix)
    \end{enumerate}
    
    

\end{enumerate}
}
\vspace{2cm}
\vbox{Numéro 73.
\emph{Toutes les réponses doivent être justifiées par un calcul accompagné d'un raisonnement.}
\begin{enumerate}\item
Soient les points $A(1;2)$, $ B(-4,9)$ et $ C(-10;4)$. 
    \begin{enumerate}
    \item
    
    Calculer les coordonnées des vecteurs \( \vect{ AB }\) et \( \vect{ AB }+\vect{ BC }\). 

\item
    Donner les coordonnées du point \( X\) tel que \( \vect{ AX }=\vect{ BC }\) (méthode au choix)
    \end{enumerate}
    
    
\item

    Soient les points $A(0;7)$, $K(-5;-1)$ et $B(-3;6)$. Donner les coordonnées du point $F$ tel que $AKBF$ soit un parallélogramme (méthode au choix).
    

\end{enumerate}
}
\vspace{2cm}
\vbox{Numéro 74.
\emph{Toutes les réponses doivent être justifiées par un calcul accompagné d'un raisonnement.}
\begin{enumerate}\item

    Soient les points $M(-1;-6)$, $F(-7;8)$ et $A(10;-5)$. Donner les coordonnées du point $E$ tel que $MFAE$ soit un parallélogramme (méthode au choix).
    
\item
Soient les points $A(7;-3)$, $ B(-9,2)$ et $ C(-3;9)$. 
    \begin{enumerate}
    \item
    
    Calculer les coordonnées des vecteurs \( \vect{ AB }\) et \( \vect{ AB }+\vect{ BC }\). 

\item
    Donner les coordonnées du point \( X\) tel que \( \vect{ AX }=\vect{ BC }\) (méthode au choix)
    \end{enumerate}
    
    

\end{enumerate}
}
\vspace{2cm}

\section{correction}
\vbox{Numéro 1.
\emph{Toutes les réponses doivent être justifiées par un calcul accompagné d'un raisonnement.}
\begin{enumerate}\item
Soient les points $A(4;-7)$, $ B(5,10)$ et $ C(1;-6)$. 
    \begin{enumerate}
    \item
    
    Calculer les coordonnées des vecteurs \( \vect{ AB }\) et \( \vect{ AB }+\vect{ BC }\). 

\item
    Donner les coordonnées du point \( X\) tel que \( \vect{ AX }=\vect{ BC }\) (méthode au choix)
    \end{enumerate}
    
    



    $\vect{ AB }=(1;17)$

    $\vect{ AB }+\vect{ BC }=(-3;1)$

    $X=(0;-23)$
    \item

    Soient les points $D(1;-3)$, $L(4;0)$ et $B(-4;7)$. Donner les coordonnées du point $K$ tel que $DLBK$ soit un parallélogramme (méthode au choix).
    

$K=(-7;4)$
\end{enumerate}
}
\vbox{Numéro 2.
\emph{Toutes les réponses doivent être justifiées par un calcul accompagné d'un raisonnement.}
\begin{enumerate}\item
Soient les points $A(8;-1)$, $ B(4,-7)$ et $ C(8;-5)$. 
    \begin{enumerate}
    \item
    
    Calculer les coordonnées des vecteurs \( \vect{ AB }\) et \( \vect{ AB }+\vect{ BC }\). 

\item
    Donner les coordonnées du point \( X\) tel que \( \vect{ AX }=\vect{ BC }\) (méthode au choix)
    \end{enumerate}
    
    



    $\vect{ AB }=(-4;-6)$

    $\vect{ AB }+\vect{ BC }=(0;-4)$

    $X=(12;1)$
    \item

    Soient les points $E(-10;2)$, $B(8;-5)$ et $D(3;10)$. Donner les coordonnées du point $F$ tel que $EBDF$ soit un parallélogramme (méthode au choix).
    

$F=(-15;17)$
\end{enumerate}
}
\vbox{Numéro 3.
\emph{Toutes les réponses doivent être justifiées par un calcul accompagné d'un raisonnement.}
\begin{enumerate}\item

    Soient les points $F(9;0)$, $L(6;3)$ et $E(1;6)$. Donner les coordonnées du point $M$ tel que $FLEM$ soit un parallélogramme (méthode au choix).
    

$M=(4;3)$\item
Soient les points $A(1;-5)$, $ B(7,-6)$ et $ C(1;-4)$. 
    \begin{enumerate}
    \item
    
    Calculer les coordonnées des vecteurs \( \vect{ AB }\) et \( \vect{ AB }+\vect{ BC }\). 

\item
    Donner les coordonnées du point \( X\) tel que \( \vect{ AX }=\vect{ BC }\) (méthode au choix)
    \end{enumerate}
    
    



    $\vect{ AB }=(6;-1)$

    $\vect{ AB }+\vect{ BC }=(0;1)$

    $X=(-5;-3)$
    
\end{enumerate}
}
\vbox{Numéro 4.
\emph{Toutes les réponses doivent être justifiées par un calcul accompagné d'un raisonnement.}
\begin{enumerate}\item
Soient les points $A(-8;9)$, $ B(-3,-7)$ et $ C(-9;-8)$. 
    \begin{enumerate}
    \item
    
    Calculer les coordonnées des vecteurs \( \vect{ AB }\) et \( \vect{ AB }+\vect{ BC }\). 

\item
    Donner les coordonnées du point \( X\) tel que \( \vect{ AX }=\vect{ BC }\) (méthode au choix)
    \end{enumerate}
    
    



    $\vect{ AB }=(5;-16)$

    $\vect{ AB }+\vect{ BC }=(-1;-17)$

    $X=(-14;8)$
    \item

    Soient les points $L(-7;7)$, $A(7;-8)$ et $F(-3;-8)$. Donner les coordonnées du point $D$ tel que $LAFD$ soit un parallélogramme (méthode au choix).
    

$D=(-17;7)$
\end{enumerate}
}
\vbox{Numéro 5.
\emph{Toutes les réponses doivent être justifiées par un calcul accompagné d'un raisonnement.}
\begin{enumerate}\item

    Soient les points $K(0;8)$, $A(0;-7)$ et $D(3;4)$. Donner les coordonnées du point $B$ tel que $KADB$ soit un parallélogramme (méthode au choix).
    

$B=(3;19)$\item
Soient les points $A(-4;3)$, $ B(8,-6)$ et $ C(-5;-7)$. 
    \begin{enumerate}
    \item
    
    Calculer les coordonnées des vecteurs \( \vect{ AB }\) et \( \vect{ AB }+\vect{ BC }\). 

\item
    Donner les coordonnées du point \( X\) tel que \( \vect{ AX }=\vect{ BC }\) (méthode au choix)
    \end{enumerate}
    
    



    $\vect{ AB }=(12;-9)$

    $\vect{ AB }+\vect{ BC }=(-1;-10)$

    $X=(-17;2)$
    
\end{enumerate}
}
\vbox{Numéro 6.
\emph{Toutes les réponses doivent être justifiées par un calcul accompagné d'un raisonnement.}
\begin{enumerate}\item

    Soient les points $F(10;-5)$, $M(-7;-5)$ et $B(5;-7)$. Donner les coordonnées du point $D$ tel que $FMBD$ soit un parallélogramme (méthode au choix).
    

$D=(22;-7)$\item
Soient les points $A(1;-5)$, $ B(9,10)$ et $ C(-6;3)$. 
    \begin{enumerate}
    \item
    
    Calculer les coordonnées des vecteurs \( \vect{ AB }\) et \( \vect{ AB }+\vect{ BC }\). 

\item
    Donner les coordonnées du point \( X\) tel que \( \vect{ AX }=\vect{ BC }\) (méthode au choix)
    \end{enumerate}
    
    



    $\vect{ AB }=(8;15)$

    $\vect{ AB }+\vect{ BC }=(-7;8)$

    $X=(-14;-12)$
    
\end{enumerate}
}
\vbox{Numéro 7.
\emph{Toutes les réponses doivent être justifiées par un calcul accompagné d'un raisonnement.}
\begin{enumerate}\item
Soient les points $A(-9;-2)$, $ B(1,10)$ et $ C(1;4)$. 
    \begin{enumerate}
    \item
    
    Calculer les coordonnées des vecteurs \( \vect{ AB }\) et \( \vect{ AB }+\vect{ BC }\). 

\item
    Donner les coordonnées du point \( X\) tel que \( \vect{ AX }=\vect{ BC }\) (méthode au choix)
    \end{enumerate}
    
    



    $\vect{ AB }=(10;12)$

    $\vect{ AB }+\vect{ BC }=(10;6)$

    $X=(-9;-8)$
    \item

    Soient les points $M(3;7)$, $B(-6;-5)$ et $A(1;10)$. Donner les coordonnées du point $L$ tel que $MBAL$ soit un parallélogramme (méthode au choix).
    

$L=(10;22)$
\end{enumerate}
}
\vbox{Numéro 8.
\emph{Toutes les réponses doivent être justifiées par un calcul accompagné d'un raisonnement.}
\begin{enumerate}\item

    Soient les points $A(-10;-7)$, $F(-5;0)$ et $D(3;4)$. Donner les coordonnées du point $E$ tel que $AFDE$ soit un parallélogramme (méthode au choix).
    

$E=(-2;-3)$\item
Soient les points $A(8;2)$, $ B(-5,-1)$ et $ C(4;-9)$. 
    \begin{enumerate}
    \item
    
    Calculer les coordonnées des vecteurs \( \vect{ AB }\) et \( \vect{ AB }+\vect{ BC }\). 

\item
    Donner les coordonnées du point \( X\) tel que \( \vect{ AX }=\vect{ BC }\) (méthode au choix)
    \end{enumerate}
    
    



    $\vect{ AB }=(-13;-3)$

    $\vect{ AB }+\vect{ BC }=(-4;-11)$

    $X=(17;-6)$
    
\end{enumerate}
}
\vbox{Numéro 9.
\emph{Toutes les réponses doivent être justifiées par un calcul accompagné d'un raisonnement.}
\begin{enumerate}\item

    Soient les points $B(10;0)$, $K(6;-7)$ et $D(10;-8)$. Donner les coordonnées du point $L$ tel que $BKDL$ soit un parallélogramme (méthode au choix).
    

$L=(14;-1)$\item
Soient les points $A(-1;4)$, $ B(-7,2)$ et $ C(5;5)$. 
    \begin{enumerate}
    \item
    
    Calculer les coordonnées des vecteurs \( \vect{ AB }\) et \( \vect{ AB }+\vect{ BC }\). 

\item
    Donner les coordonnées du point \( X\) tel que \( \vect{ AX }=\vect{ BC }\) (méthode au choix)
    \end{enumerate}
    
    



    $\vect{ AB }=(-6;-2)$

    $\vect{ AB }+\vect{ BC }=(6;1)$

    $X=(11;7)$
    
\end{enumerate}
}
\vbox{Numéro 10.
\emph{Toutes les réponses doivent être justifiées par un calcul accompagné d'un raisonnement.}
\begin{enumerate}\item
Soient les points $A(-2;-1)$, $ B(-10,-1)$ et $ C(-7;-7)$. 
    \begin{enumerate}
    \item
    
    Calculer les coordonnées des vecteurs \( \vect{ AB }\) et \( \vect{ AB }+\vect{ BC }\). 

\item
    Donner les coordonnées du point \( X\) tel que \( \vect{ AX }=\vect{ BC }\) (méthode au choix)
    \end{enumerate}
    
    



    $\vect{ AB }=(-8;0)$

    $\vect{ AB }+\vect{ BC }=(-5;-6)$

    $X=(1;-7)$
    \item

    Soient les points $A(2;4)$, $F(-9;-9)$ et $B(-3;3)$. Donner les coordonnées du point $L$ tel que $AFBL$ soit un parallélogramme (méthode au choix).
    

$L=(8;16)$
\end{enumerate}
}
\vbox{Numéro 11.
\emph{Toutes les réponses doivent être justifiées par un calcul accompagné d'un raisonnement.}
\begin{enumerate}\item

    Soient les points $F(-7;6)$, $D(-3;-5)$ et $M(-4;10)$. Donner les coordonnées du point $L$ tel que $FDML$ soit un parallélogramme (méthode au choix).
    

$L=(-8;21)$\item
Soient les points $A(2;-3)$, $ B(0,-6)$ et $ C(3;-2)$. 
    \begin{enumerate}
    \item
    
    Calculer les coordonnées des vecteurs \( \vect{ AB }\) et \( \vect{ AB }+\vect{ BC }\). 

\item
    Donner les coordonnées du point \( X\) tel que \( \vect{ AX }=\vect{ BC }\) (méthode au choix)
    \end{enumerate}
    
    



    $\vect{ AB }=(-2;-3)$

    $\vect{ AB }+\vect{ BC }=(1;1)$

    $X=(5;1)$
    
\end{enumerate}
}
\vbox{Numéro 12.
\emph{Toutes les réponses doivent être justifiées par un calcul accompagné d'un raisonnement.}
\begin{enumerate}\item

    Soient les points $F(9;3)$, $L(10;4)$ et $A(-5;-6)$. Donner les coordonnées du point $D$ tel que $FLAD$ soit un parallélogramme (méthode au choix).
    

$D=(-6;-7)$\item
Soient les points $A(10;2)$, $ B(7,-9)$ et $ C(-1;-8)$. 
    \begin{enumerate}
    \item
    
    Calculer les coordonnées des vecteurs \( \vect{ AB }\) et \( \vect{ AB }+\vect{ BC }\). 

\item
    Donner les coordonnées du point \( X\) tel que \( \vect{ AX }=\vect{ BC }\) (méthode au choix)
    \end{enumerate}
    
    



    $\vect{ AB }=(-3;-11)$

    $\vect{ AB }+\vect{ BC }=(-11;-10)$

    $X=(2;3)$
    
\end{enumerate}
}
\vbox{Numéro 13.
\emph{Toutes les réponses doivent être justifiées par un calcul accompagné d'un raisonnement.}
\begin{enumerate}\item
Soient les points $A(-1;-10)$, $ B(7,4)$ et $ C(-5;-10)$. 
    \begin{enumerate}
    \item
    
    Calculer les coordonnées des vecteurs \( \vect{ AB }\) et \( \vect{ AB }+\vect{ BC }\). 

\item
    Donner les coordonnées du point \( X\) tel que \( \vect{ AX }=\vect{ BC }\) (méthode au choix)
    \end{enumerate}
    
    



    $\vect{ AB }=(8;14)$

    $\vect{ AB }+\vect{ BC }=(-4;0)$

    $X=(-13;-24)$
    \item

    Soient les points $E(-10;-5)$, $L(8;-3)$ et $D(3;2)$. Donner les coordonnées du point $A$ tel que $ELDA$ soit un parallélogramme (méthode au choix).
    

$A=(-15;0)$
\end{enumerate}
}
\vbox{Numéro 14.
\emph{Toutes les réponses doivent être justifiées par un calcul accompagné d'un raisonnement.}
\begin{enumerate}\item
Soient les points $A(5;-10)$, $ B(-7,-10)$ et $ C(9;6)$. 
    \begin{enumerate}
    \item
    
    Calculer les coordonnées des vecteurs \( \vect{ AB }\) et \( \vect{ AB }+\vect{ BC }\). 

\item
    Donner les coordonnées du point \( X\) tel que \( \vect{ AX }=\vect{ BC }\) (méthode au choix)
    \end{enumerate}
    
    



    $\vect{ AB }=(-12;0)$

    $\vect{ AB }+\vect{ BC }=(4;16)$

    $X=(21;6)$
    \item

    Soient les points $F(-9;8)$, $K(-8;-3)$ et $L(-2;-2)$. Donner les coordonnées du point $A$ tel que $FKLA$ soit un parallélogramme (méthode au choix).
    

$A=(-3;9)$
\end{enumerate}
}
\vbox{Numéro 15.
\emph{Toutes les réponses doivent être justifiées par un calcul accompagné d'un raisonnement.}
\begin{enumerate}\item

    Soient les points $D(7;-6)$, $L(3;-7)$ et $K(0;-3)$. Donner les coordonnées du point $A$ tel que $DLKA$ soit un parallélogramme (méthode au choix).
    

$A=(4;-2)$\item
Soient les points $A(7;-9)$, $ B(-6,-2)$ et $ C(-7;10)$. 
    \begin{enumerate}
    \item
    
    Calculer les coordonnées des vecteurs \( \vect{ AB }\) et \( \vect{ AB }+\vect{ BC }\). 

\item
    Donner les coordonnées du point \( X\) tel que \( \vect{ AX }=\vect{ BC }\) (méthode au choix)
    \end{enumerate}
    
    



    $\vect{ AB }=(-13;7)$

    $\vect{ AB }+\vect{ BC }=(-14;19)$

    $X=(6;3)$
    
\end{enumerate}
}
\vbox{Numéro 16.
\emph{Toutes les réponses doivent être justifiées par un calcul accompagné d'un raisonnement.}
\begin{enumerate}\item

    Soient les points $F(-5;8)$, $M(6;5)$ et $D(5;10)$. Donner les coordonnées du point $K$ tel que $FMDK$ soit un parallélogramme (méthode au choix).
    

$K=(-6;13)$\item
Soient les points $A(9;-8)$, $ B(9,0)$ et $ C(0;-3)$. 
    \begin{enumerate}
    \item
    
    Calculer les coordonnées des vecteurs \( \vect{ AB }\) et \( \vect{ AB }+\vect{ BC }\). 

\item
    Donner les coordonnées du point \( X\) tel que \( \vect{ AX }=\vect{ BC }\) (méthode au choix)
    \end{enumerate}
    
    



    $\vect{ AB }=(0;8)$

    $\vect{ AB }+\vect{ BC }=(-9;5)$

    $X=(0;-11)$
    
\end{enumerate}
}
\vbox{Numéro 17.
\emph{Toutes les réponses doivent être justifiées par un calcul accompagné d'un raisonnement.}
\begin{enumerate}\item
Soient les points $A(-4;4)$, $ B(-6,-8)$ et $ C(10;6)$. 
    \begin{enumerate}
    \item
    
    Calculer les coordonnées des vecteurs \( \vect{ AB }\) et \( \vect{ AB }+\vect{ BC }\). 

\item
    Donner les coordonnées du point \( X\) tel que \( \vect{ AX }=\vect{ BC }\) (méthode au choix)
    \end{enumerate}
    
    



    $\vect{ AB }=(-2;-12)$

    $\vect{ AB }+\vect{ BC }=(14;2)$

    $X=(12;18)$
    \item

    Soient les points $K(-7;-10)$, $M(-7;5)$ et $E(9;2)$. Donner les coordonnées du point $D$ tel que $KMED$ soit un parallélogramme (méthode au choix).
    

$D=(9;-13)$
\end{enumerate}
}
\vbox{Numéro 18.
\emph{Toutes les réponses doivent être justifiées par un calcul accompagné d'un raisonnement.}
\begin{enumerate}\item

    Soient les points $L(4;-7)$, $F(5;0)$ et $A(2;8)$. Donner les coordonnées du point $M$ tel que $LFAM$ soit un parallélogramme (méthode au choix).
    

$M=(1;1)$\item
Soient les points $A(-10;-3)$, $ B(10,-2)$ et $ C(7;6)$. 
    \begin{enumerate}
    \item
    
    Calculer les coordonnées des vecteurs \( \vect{ AB }\) et \( \vect{ AB }+\vect{ BC }\). 

\item
    Donner les coordonnées du point \( X\) tel que \( \vect{ AX }=\vect{ BC }\) (méthode au choix)
    \end{enumerate}
    
    



    $\vect{ AB }=(20;1)$

    $\vect{ AB }+\vect{ BC }=(17;9)$

    $X=(-13;5)$
    
\end{enumerate}
}
\vbox{Numéro 19.
\emph{Toutes les réponses doivent être justifiées par un calcul accompagné d'un raisonnement.}
\begin{enumerate}\item

    Soient les points $B(1;-9)$, $E(8;0)$ et $L(10;-4)$. Donner les coordonnées du point $M$ tel que $BELM$ soit un parallélogramme (méthode au choix).
    

$M=(3;-13)$\item
Soient les points $A(2;-9)$, $ B(5,5)$ et $ C(4;-10)$. 
    \begin{enumerate}
    \item
    
    Calculer les coordonnées des vecteurs \( \vect{ AB }\) et \( \vect{ AB }+\vect{ BC }\). 

\item
    Donner les coordonnées du point \( X\) tel que \( \vect{ AX }=\vect{ BC }\) (méthode au choix)
    \end{enumerate}
    
    



    $\vect{ AB }=(3;14)$

    $\vect{ AB }+\vect{ BC }=(2;-1)$

    $X=(1;-24)$
    
\end{enumerate}
}
\vbox{Numéro 20.
\emph{Toutes les réponses doivent être justifiées par un calcul accompagné d'un raisonnement.}
\begin{enumerate}\item
Soient les points $A(9;3)$, $ B(8,-2)$ et $ C(-6;-3)$. 
    \begin{enumerate}
    \item
    
    Calculer les coordonnées des vecteurs \( \vect{ AB }\) et \( \vect{ AB }+\vect{ BC }\). 

\item
    Donner les coordonnées du point \( X\) tel que \( \vect{ AX }=\vect{ BC }\) (méthode au choix)
    \end{enumerate}
    
    



    $\vect{ AB }=(-1;-5)$

    $\vect{ AB }+\vect{ BC }=(-15;-6)$

    $X=(-5;2)$
    \item

    Soient les points $L(0;-6)$, $F(-6;9)$ et $E(7;-5)$. Donner les coordonnées du point $D$ tel que $LFED$ soit un parallélogramme (méthode au choix).
    

$D=(13;-20)$
\end{enumerate}
}
\vbox{Numéro 21.
\emph{Toutes les réponses doivent être justifiées par un calcul accompagné d'un raisonnement.}
\begin{enumerate}\item
Soient les points $A(9;-4)$, $ B(-8,9)$ et $ C(-10;9)$. 
    \begin{enumerate}
    \item
    
    Calculer les coordonnées des vecteurs \( \vect{ AB }\) et \( \vect{ AB }+\vect{ BC }\). 

\item
    Donner les coordonnées du point \( X\) tel que \( \vect{ AX }=\vect{ BC }\) (méthode au choix)
    \end{enumerate}
    
    



    $\vect{ AB }=(-17;13)$

    $\vect{ AB }+\vect{ BC }=(-19;13)$

    $X=(7;-4)$
    \item

    Soient les points $D(-9;1)$, $E(-6;-3)$ et $A(-9;-3)$. Donner les coordonnées du point $L$ tel que $DEAL$ soit un parallélogramme (méthode au choix).
    

$L=(-12;1)$
\end{enumerate}
}
\vbox{Numéro 22.
\emph{Toutes les réponses doivent être justifiées par un calcul accompagné d'un raisonnement.}
\begin{enumerate}\item
Soient les points $A(-4;-3)$, $ B(-8,-9)$ et $ C(9;5)$. 
    \begin{enumerate}
    \item
    
    Calculer les coordonnées des vecteurs \( \vect{ AB }\) et \( \vect{ AB }+\vect{ BC }\). 

\item
    Donner les coordonnées du point \( X\) tel que \( \vect{ AX }=\vect{ BC }\) (méthode au choix)
    \end{enumerate}
    
    



    $\vect{ AB }=(-4;-6)$

    $\vect{ AB }+\vect{ BC }=(13;8)$

    $X=(13;11)$
    \item

    Soient les points $F(1;-10)$, $B(-8;-10)$ et $L(-3;3)$. Donner les coordonnées du point $M$ tel que $FBLM$ soit un parallélogramme (méthode au choix).
    

$M=(6;3)$
\end{enumerate}
}
\vbox{Numéro 23.
\emph{Toutes les réponses doivent être justifiées par un calcul accompagné d'un raisonnement.}
\begin{enumerate}\item

    Soient les points $A(8;-8)$, $B(-6;5)$ et $L(5;7)$. Donner les coordonnées du point $E$ tel que $ABLE$ soit un parallélogramme (méthode au choix).
    

$E=(19;-6)$\item
Soient les points $A(1;3)$, $ B(-4,5)$ et $ C(10;-4)$. 
    \begin{enumerate}
    \item
    
    Calculer les coordonnées des vecteurs \( \vect{ AB }\) et \( \vect{ AB }+\vect{ BC }\). 

\item
    Donner les coordonnées du point \( X\) tel que \( \vect{ AX }=\vect{ BC }\) (méthode au choix)
    \end{enumerate}
    
    



    $\vect{ AB }=(-5;2)$

    $\vect{ AB }+\vect{ BC }=(9;-7)$

    $X=(15;-6)$
    
\end{enumerate}
}
\vbox{Numéro 24.
\emph{Toutes les réponses doivent être justifiées par un calcul accompagné d'un raisonnement.}
\begin{enumerate}\item

    Soient les points $F(-5;3)$, $K(1;3)$ et $A(7;6)$. Donner les coordonnées du point $L$ tel que $FKAL$ soit un parallélogramme (méthode au choix).
    

$L=(1;6)$\item
Soient les points $A(2;0)$, $ B(-2,5)$ et $ C(-6;-7)$. 
    \begin{enumerate}
    \item
    
    Calculer les coordonnées des vecteurs \( \vect{ AB }\) et \( \vect{ AB }+\vect{ BC }\). 

\item
    Donner les coordonnées du point \( X\) tel que \( \vect{ AX }=\vect{ BC }\) (méthode au choix)
    \end{enumerate}
    
    



    $\vect{ AB }=(-4;5)$

    $\vect{ AB }+\vect{ BC }=(-8;-7)$

    $X=(-2;-12)$
    
\end{enumerate}
}
\vbox{Numéro 25.
\emph{Toutes les réponses doivent être justifiées par un calcul accompagné d'un raisonnement.}
\begin{enumerate}\item
Soient les points $A(-8;-8)$, $ B(0,1)$ et $ C(9;3)$. 
    \begin{enumerate}
    \item
    
    Calculer les coordonnées des vecteurs \( \vect{ AB }\) et \( \vect{ AB }+\vect{ BC }\). 

\item
    Donner les coordonnées du point \( X\) tel que \( \vect{ AX }=\vect{ BC }\) (méthode au choix)
    \end{enumerate}
    
    



    $\vect{ AB }=(8;9)$

    $\vect{ AB }+\vect{ BC }=(17;11)$

    $X=(1;-6)$
    \item

    Soient les points $L(-7;4)$, $M(9;-1)$ et $F(-3;-8)$. Donner les coordonnées du point $A$ tel que $LMFA$ soit un parallélogramme (méthode au choix).
    

$A=(-19;-3)$
\end{enumerate}
}
\vbox{Numéro 26.
\emph{Toutes les réponses doivent être justifiées par un calcul accompagné d'un raisonnement.}
\begin{enumerate}\item
Soient les points $A(0;2)$, $ B(-10,-2)$ et $ C(2;-3)$. 
    \begin{enumerate}
    \item
    
    Calculer les coordonnées des vecteurs \( \vect{ AB }\) et \( \vect{ AB }+\vect{ BC }\). 

\item
    Donner les coordonnées du point \( X\) tel que \( \vect{ AX }=\vect{ BC }\) (méthode au choix)
    \end{enumerate}
    
    



    $\vect{ AB }=(-10;-4)$

    $\vect{ AB }+\vect{ BC }=(2;-5)$

    $X=(12;1)$
    \item

    Soient les points $A(-9;-9)$, $K(-2;9)$ et $E(-7;1)$. Donner les coordonnées du point $B$ tel que $AKEB$ soit un parallélogramme (méthode au choix).
    

$B=(-14;-17)$
\end{enumerate}
}
\vbox{Numéro 27.
\emph{Toutes les réponses doivent être justifiées par un calcul accompagné d'un raisonnement.}
\begin{enumerate}\item
Soient les points $A(6;7)$, $ B(9,7)$ et $ C(9;5)$. 
    \begin{enumerate}
    \item
    
    Calculer les coordonnées des vecteurs \( \vect{ AB }\) et \( \vect{ AB }+\vect{ BC }\). 

\item
    Donner les coordonnées du point \( X\) tel que \( \vect{ AX }=\vect{ BC }\) (méthode au choix)
    \end{enumerate}
    
    



    $\vect{ AB }=(3;0)$

    $\vect{ AB }+\vect{ BC }=(3;-2)$

    $X=(6;5)$
    \item

    Soient les points $A(-7;-6)$, $D(2;-2)$ et $M(10;-4)$. Donner les coordonnées du point $K$ tel que $ADMK$ soit un parallélogramme (méthode au choix).
    

$K=(1;-8)$
\end{enumerate}
}
\vbox{Numéro 28.
\emph{Toutes les réponses doivent être justifiées par un calcul accompagné d'un raisonnement.}
\begin{enumerate}\item

    Soient les points $F(-4;2)$, $E(2;-1)$ et $B(2;-10)$. Donner les coordonnées du point $L$ tel que $FEBL$ soit un parallélogramme (méthode au choix).
    

$L=(-4;-7)$\item
Soient les points $A(-5;2)$, $ B(-6,8)$ et $ C(-7;-1)$. 
    \begin{enumerate}
    \item
    
    Calculer les coordonnées des vecteurs \( \vect{ AB }\) et \( \vect{ AB }+\vect{ BC }\). 

\item
    Donner les coordonnées du point \( X\) tel que \( \vect{ AX }=\vect{ BC }\) (méthode au choix)
    \end{enumerate}
    
    



    $\vect{ AB }=(-1;6)$

    $\vect{ AB }+\vect{ BC }=(-2;-3)$

    $X=(-6;-7)$
    
\end{enumerate}
}
\vbox{Numéro 29.
\emph{Toutes les réponses doivent être justifiées par un calcul accompagné d'un raisonnement.}
\begin{enumerate}\item

    Soient les points $L(-4;-4)$, $A(-7;-6)$ et $K(9;-7)$. Donner les coordonnées du point $D$ tel que $LAKD$ soit un parallélogramme (méthode au choix).
    

$D=(12;-5)$\item
Soient les points $A(8;-2)$, $ B(9,5)$ et $ C(9;6)$. 
    \begin{enumerate}
    \item
    
    Calculer les coordonnées des vecteurs \( \vect{ AB }\) et \( \vect{ AB }+\vect{ BC }\). 

\item
    Donner les coordonnées du point \( X\) tel que \( \vect{ AX }=\vect{ BC }\) (méthode au choix)
    \end{enumerate}
    
    



    $\vect{ AB }=(1;7)$

    $\vect{ AB }+\vect{ BC }=(1;8)$

    $X=(8;-1)$
    
\end{enumerate}
}
\vbox{Numéro 30.
\emph{Toutes les réponses doivent être justifiées par un calcul accompagné d'un raisonnement.}
\begin{enumerate}\item
Soient les points $A(4;-9)$, $ B(10,4)$ et $ C(10;-8)$. 
    \begin{enumerate}
    \item
    
    Calculer les coordonnées des vecteurs \( \vect{ AB }\) et \( \vect{ AB }+\vect{ BC }\). 

\item
    Donner les coordonnées du point \( X\) tel que \( \vect{ AX }=\vect{ BC }\) (méthode au choix)
    \end{enumerate}
    
    



    $\vect{ AB }=(6;13)$

    $\vect{ AB }+\vect{ BC }=(6;1)$

    $X=(4;-21)$
    \item

    Soient les points $K(-5;4)$, $A(-4;-5)$ et $E(5;4)$. Donner les coordonnées du point $B$ tel que $KAEB$ soit un parallélogramme (méthode au choix).
    

$B=(4;13)$
\end{enumerate}
}
\vbox{Numéro 31.
\emph{Toutes les réponses doivent être justifiées par un calcul accompagné d'un raisonnement.}
\begin{enumerate}\item
Soient les points $A(-4;1)$, $ B(2,-9)$ et $ C(-3;-2)$. 
    \begin{enumerate}
    \item
    
    Calculer les coordonnées des vecteurs \( \vect{ AB }\) et \( \vect{ AB }+\vect{ BC }\). 

\item
    Donner les coordonnées du point \( X\) tel que \( \vect{ AX }=\vect{ BC }\) (méthode au choix)
    \end{enumerate}
    
    



    $\vect{ AB }=(6;-10)$

    $\vect{ AB }+\vect{ BC }=(1;-3)$

    $X=(-9;8)$
    \item

    Soient les points $B(-1;3)$, $K(-3;-9)$ et $A(-10;-6)$. Donner les coordonnées du point $D$ tel que $BKAD$ soit un parallélogramme (méthode au choix).
    

$D=(-8;6)$
\end{enumerate}
}
\vbox{Numéro 32.
\emph{Toutes les réponses doivent être justifiées par un calcul accompagné d'un raisonnement.}
\begin{enumerate}\item

    Soient les points $A(-8;8)$, $L(9;3)$ et $D(3;-6)$. Donner les coordonnées du point $M$ tel que $ALDM$ soit un parallélogramme (méthode au choix).
    

$M=(-14;-1)$\item
Soient les points $A(9;0)$, $ B(-1,4)$ et $ C(6;-4)$. 
    \begin{enumerate}
    \item
    
    Calculer les coordonnées des vecteurs \( \vect{ AB }\) et \( \vect{ AB }+\vect{ BC }\). 

\item
    Donner les coordonnées du point \( X\) tel que \( \vect{ AX }=\vect{ BC }\) (méthode au choix)
    \end{enumerate}
    
    



    $\vect{ AB }=(-10;4)$

    $\vect{ AB }+\vect{ BC }=(-3;-4)$

    $X=(16;-8)$
    
\end{enumerate}
}
\vbox{Numéro 33.
\emph{Toutes les réponses doivent être justifiées par un calcul accompagné d'un raisonnement.}
\begin{enumerate}\item
Soient les points $A(1;-9)$, $ B(3,6)$ et $ C(8;6)$. 
    \begin{enumerate}
    \item
    
    Calculer les coordonnées des vecteurs \( \vect{ AB }\) et \( \vect{ AB }+\vect{ BC }\). 

\item
    Donner les coordonnées du point \( X\) tel que \( \vect{ AX }=\vect{ BC }\) (méthode au choix)
    \end{enumerate}
    
    



    $\vect{ AB }=(2;15)$

    $\vect{ AB }+\vect{ BC }=(7;15)$

    $X=(6;-9)$
    \item

    Soient les points $A(6;-5)$, $D(7;-7)$ et $B(-2;-7)$. Donner les coordonnées du point $M$ tel que $ADBM$ soit un parallélogramme (méthode au choix).
    

$M=(-3;-5)$
\end{enumerate}
}
\vbox{Numéro 34.
\emph{Toutes les réponses doivent être justifiées par un calcul accompagné d'un raisonnement.}
\begin{enumerate}\item
Soient les points $A(-1;1)$, $ B(0,-7)$ et $ C(5;-1)$. 
    \begin{enumerate}
    \item
    
    Calculer les coordonnées des vecteurs \( \vect{ AB }\) et \( \vect{ AB }+\vect{ BC }\). 

\item
    Donner les coordonnées du point \( X\) tel que \( \vect{ AX }=\vect{ BC }\) (méthode au choix)
    \end{enumerate}
    
    



    $\vect{ AB }=(1;-8)$

    $\vect{ AB }+\vect{ BC }=(6;-2)$

    $X=(4;7)$
    \item

    Soient les points $F(-1;3)$, $M(2;8)$ et $E(0;-3)$. Donner les coordonnées du point $K$ tel que $FMEK$ soit un parallélogramme (méthode au choix).
    

$K=(-3;-8)$
\end{enumerate}
}
\vbox{Numéro 35.
\emph{Toutes les réponses doivent être justifiées par un calcul accompagné d'un raisonnement.}
\begin{enumerate}\item

    Soient les points $F(9;-4)$, $L(5;-5)$ et $E(-3;-9)$. Donner les coordonnées du point $D$ tel que $FLED$ soit un parallélogramme (méthode au choix).
    

$D=(1;-8)$\item
Soient les points $A(-6;2)$, $ B(-5,-3)$ et $ C(-6;-1)$. 
    \begin{enumerate}
    \item
    
    Calculer les coordonnées des vecteurs \( \vect{ AB }\) et \( \vect{ AB }+\vect{ BC }\). 

\item
    Donner les coordonnées du point \( X\) tel que \( \vect{ AX }=\vect{ BC }\) (méthode au choix)
    \end{enumerate}
    
    



    $\vect{ AB }=(1;-5)$

    $\vect{ AB }+\vect{ BC }=(0;-3)$

    $X=(-7;4)$
    
\end{enumerate}
}
\vbox{Numéro 36.
\emph{Toutes les réponses doivent être justifiées par un calcul accompagné d'un raisonnement.}
\begin{enumerate}\item
Soient les points $A(3;-5)$, $ B(7,-10)$ et $ C(-9;6)$. 
    \begin{enumerate}
    \item
    
    Calculer les coordonnées des vecteurs \( \vect{ AB }\) et \( \vect{ AB }+\vect{ BC }\). 

\item
    Donner les coordonnées du point \( X\) tel que \( \vect{ AX }=\vect{ BC }\) (méthode au choix)
    \end{enumerate}
    
    



    $\vect{ AB }=(4;-5)$

    $\vect{ AB }+\vect{ BC }=(-12;11)$

    $X=(-13;11)$
    \item

    Soient les points $M(10;-8)$, $F(-5;-4)$ et $L(0;6)$. Donner les coordonnées du point $K$ tel que $MFLK$ soit un parallélogramme (méthode au choix).
    

$K=(15;2)$
\end{enumerate}
}
\vbox{Numéro 37.
\emph{Toutes les réponses doivent être justifiées par un calcul accompagné d'un raisonnement.}
\begin{enumerate}\item

    Soient les points $F(10;-3)$, $B(4;-7)$ et $M(10;1)$. Donner les coordonnées du point $L$ tel que $FBML$ soit un parallélogramme (méthode au choix).
    

$L=(16;5)$\item
Soient les points $A(-1;9)$, $ B(1,6)$ et $ C(0;-1)$. 
    \begin{enumerate}
    \item
    
    Calculer les coordonnées des vecteurs \( \vect{ AB }\) et \( \vect{ AB }+\vect{ BC }\). 

\item
    Donner les coordonnées du point \( X\) tel que \( \vect{ AX }=\vect{ BC }\) (méthode au choix)
    \end{enumerate}
    
    



    $\vect{ AB }=(2;-3)$

    $\vect{ AB }+\vect{ BC }=(1;-10)$

    $X=(-2;2)$
    
\end{enumerate}
}
\vbox{Numéro 38.
\emph{Toutes les réponses doivent être justifiées par un calcul accompagné d'un raisonnement.}
\begin{enumerate}\item
Soient les points $A(-1;4)$, $ B(1,-4)$ et $ C(4;-6)$. 
    \begin{enumerate}
    \item
    
    Calculer les coordonnées des vecteurs \( \vect{ AB }\) et \( \vect{ AB }+\vect{ BC }\). 

\item
    Donner les coordonnées du point \( X\) tel que \( \vect{ AX }=\vect{ BC }\) (méthode au choix)
    \end{enumerate}
    
    



    $\vect{ AB }=(2;-8)$

    $\vect{ AB }+\vect{ BC }=(5;-10)$

    $X=(2;2)$
    \item

    Soient les points $K(-5;6)$, $L(-6;-2)$ et $E(-5;-1)$. Donner les coordonnées du point $M$ tel que $KLEM$ soit un parallélogramme (méthode au choix).
    

$M=(-4;7)$
\end{enumerate}
}
\vbox{Numéro 39.
\emph{Toutes les réponses doivent être justifiées par un calcul accompagné d'un raisonnement.}
\begin{enumerate}\item

    Soient les points $E(-4;9)$, $L(-5;-2)$ et $B(-6;10)$. Donner les coordonnées du point $D$ tel que $ELBD$ soit un parallélogramme (méthode au choix).
    

$D=(-5;21)$\item
Soient les points $A(6;-8)$, $ B(10,10)$ et $ C(1;1)$. 
    \begin{enumerate}
    \item
    
    Calculer les coordonnées des vecteurs \( \vect{ AB }\) et \( \vect{ AB }+\vect{ BC }\). 

\item
    Donner les coordonnées du point \( X\) tel que \( \vect{ AX }=\vect{ BC }\) (méthode au choix)
    \end{enumerate}
    
    



    $\vect{ AB }=(4;18)$

    $\vect{ AB }+\vect{ BC }=(-5;9)$

    $X=(-3;-17)$
    
\end{enumerate}
}
\vbox{Numéro 40.
\emph{Toutes les réponses doivent être justifiées par un calcul accompagné d'un raisonnement.}
\begin{enumerate}\item
Soient les points $A(-7;4)$, $ B(4,-10)$ et $ C(-2;-1)$. 
    \begin{enumerate}
    \item
    
    Calculer les coordonnées des vecteurs \( \vect{ AB }\) et \( \vect{ AB }+\vect{ BC }\). 

\item
    Donner les coordonnées du point \( X\) tel que \( \vect{ AX }=\vect{ BC }\) (méthode au choix)
    \end{enumerate}
    
    



    $\vect{ AB }=(11;-14)$

    $\vect{ AB }+\vect{ BC }=(5;-5)$

    $X=(-13;13)$
    \item

    Soient les points $F(8;9)$, $K(-9;-7)$ et $D(-8;7)$. Donner les coordonnées du point $B$ tel que $FKDB$ soit un parallélogramme (méthode au choix).
    

$B=(9;23)$
\end{enumerate}
}
\vbox{Numéro 41.
\emph{Toutes les réponses doivent être justifiées par un calcul accompagné d'un raisonnement.}
\begin{enumerate}\item

    Soient les points $A(-10;9)$, $F(4;-6)$ et $E(-10;0)$. Donner les coordonnées du point $M$ tel que $AFEM$ soit un parallélogramme (méthode au choix).
    

$M=(-24;15)$\item
Soient les points $A(-1;1)$, $ B(6,-8)$ et $ C(-4;-4)$. 
    \begin{enumerate}
    \item
    
    Calculer les coordonnées des vecteurs \( \vect{ AB }\) et \( \vect{ AB }+\vect{ BC }\). 

\item
    Donner les coordonnées du point \( X\) tel que \( \vect{ AX }=\vect{ BC }\) (méthode au choix)
    \end{enumerate}
    
    



    $\vect{ AB }=(7;-9)$

    $\vect{ AB }+\vect{ BC }=(-3;-5)$

    $X=(-11;5)$
    
\end{enumerate}
}
\vbox{Numéro 42.
\emph{Toutes les réponses doivent être justifiées par un calcul accompagné d'un raisonnement.}
\begin{enumerate}\item

    Soient les points $M(-8;5)$, $E(-5;10)$ et $L(-6;1)$. Donner les coordonnées du point $B$ tel que $MELB$ soit un parallélogramme (méthode au choix).
    

$B=(-9;-4)$\item
Soient les points $A(0;-5)$, $ B(-6,8)$ et $ C(-3;-1)$. 
    \begin{enumerate}
    \item
    
    Calculer les coordonnées des vecteurs \( \vect{ AB }\) et \( \vect{ AB }+\vect{ BC }\). 

\item
    Donner les coordonnées du point \( X\) tel que \( \vect{ AX }=\vect{ BC }\) (méthode au choix)
    \end{enumerate}
    
    



    $\vect{ AB }=(-6;13)$

    $\vect{ AB }+\vect{ BC }=(-3;4)$

    $X=(3;-14)$
    
\end{enumerate}
}
\vbox{Numéro 43.
\emph{Toutes les réponses doivent être justifiées par un calcul accompagné d'un raisonnement.}
\begin{enumerate}\item

    Soient les points $E(-2;-6)$, $D(-3;6)$ et $A(1;7)$. Donner les coordonnées du point $B$ tel que $EDAB$ soit un parallélogramme (méthode au choix).
    

$B=(2;-5)$\item
Soient les points $A(-1;-5)$, $ B(0,-3)$ et $ C(-4;0)$. 
    \begin{enumerate}
    \item
    
    Calculer les coordonnées des vecteurs \( \vect{ AB }\) et \( \vect{ AB }+\vect{ BC }\). 

\item
    Donner les coordonnées du point \( X\) tel que \( \vect{ AX }=\vect{ BC }\) (méthode au choix)
    \end{enumerate}
    
    



    $\vect{ AB }=(1;2)$

    $\vect{ AB }+\vect{ BC }=(-3;5)$

    $X=(-5;-2)$
    
\end{enumerate}
}
\vbox{Numéro 44.
\emph{Toutes les réponses doivent être justifiées par un calcul accompagné d'un raisonnement.}
\begin{enumerate}\item
Soient les points $A(9;5)$, $ B(-9,-8)$ et $ C(6;-7)$. 
    \begin{enumerate}
    \item
    
    Calculer les coordonnées des vecteurs \( \vect{ AB }\) et \( \vect{ AB }+\vect{ BC }\). 

\item
    Donner les coordonnées du point \( X\) tel que \( \vect{ AX }=\vect{ BC }\) (méthode au choix)
    \end{enumerate}
    
    



    $\vect{ AB }=(-18;-13)$

    $\vect{ AB }+\vect{ BC }=(-3;-12)$

    $X=(24;6)$
    \item

    Soient les points $L(7;9)$, $A(-7;9)$ et $K(10;-8)$. Donner les coordonnées du point $E$ tel que $LAKE$ soit un parallélogramme (méthode au choix).
    

$E=(24;-8)$
\end{enumerate}
}
\vbox{Numéro 45.
\emph{Toutes les réponses doivent être justifiées par un calcul accompagné d'un raisonnement.}
\begin{enumerate}\item
Soient les points $A(2;-10)$, $ B(-8,7)$ et $ C(-7;5)$. 
    \begin{enumerate}
    \item
    
    Calculer les coordonnées des vecteurs \( \vect{ AB }\) et \( \vect{ AB }+\vect{ BC }\). 

\item
    Donner les coordonnées du point \( X\) tel que \( \vect{ AX }=\vect{ BC }\) (méthode au choix)
    \end{enumerate}
    
    



    $\vect{ AB }=(-10;17)$

    $\vect{ AB }+\vect{ BC }=(-9;15)$

    $X=(3;-12)$
    \item

    Soient les points $L(-5;-5)$, $F(-10;3)$ et $K(-10;9)$. Donner les coordonnées du point $B$ tel que $LFKB$ soit un parallélogramme (méthode au choix).
    

$B=(-5;1)$
\end{enumerate}
}
\vbox{Numéro 46.
\emph{Toutes les réponses doivent être justifiées par un calcul accompagné d'un raisonnement.}
\begin{enumerate}\item

    Soient les points $M(10;3)$, $B(-3;1)$ et $K(-10;-5)$. Donner les coordonnées du point $E$ tel que $MBKE$ soit un parallélogramme (méthode au choix).
    

$E=(3;-3)$\item
Soient les points $A(4;-6)$, $ B(-2,4)$ et $ C(8;-10)$. 
    \begin{enumerate}
    \item
    
    Calculer les coordonnées des vecteurs \( \vect{ AB }\) et \( \vect{ AB }+\vect{ BC }\). 

\item
    Donner les coordonnées du point \( X\) tel que \( \vect{ AX }=\vect{ BC }\) (méthode au choix)
    \end{enumerate}
    
    



    $\vect{ AB }=(-6;10)$

    $\vect{ AB }+\vect{ BC }=(4;-4)$

    $X=(14;-20)$
    
\end{enumerate}
}
\vbox{Numéro 47.
\emph{Toutes les réponses doivent être justifiées par un calcul accompagné d'un raisonnement.}
\begin{enumerate}\item

    Soient les points $L(-4;6)$, $F(4;9)$ et $M(-10;-3)$. Donner les coordonnées du point $D$ tel que $LFMD$ soit un parallélogramme (méthode au choix).
    

$D=(-18;-6)$\item
Soient les points $A(-6;-1)$, $ B(10,-2)$ et $ C(-8;-1)$. 
    \begin{enumerate}
    \item
    
    Calculer les coordonnées des vecteurs \( \vect{ AB }\) et \( \vect{ AB }+\vect{ BC }\). 

\item
    Donner les coordonnées du point \( X\) tel que \( \vect{ AX }=\vect{ BC }\) (méthode au choix)
    \end{enumerate}
    
    



    $\vect{ AB }=(16;-1)$

    $\vect{ AB }+\vect{ BC }=(-2;0)$

    $X=(-24;0)$
    
\end{enumerate}
}
\vbox{Numéro 48.
\emph{Toutes les réponses doivent être justifiées par un calcul accompagné d'un raisonnement.}
\begin{enumerate}\item

    Soient les points $D(-9;9)$, $F(7;3)$ et $B(6;10)$. Donner les coordonnées du point $E$ tel que $DFBE$ soit un parallélogramme (méthode au choix).
    

$E=(-10;16)$\item
Soient les points $A(4;-1)$, $ B(-10,7)$ et $ C(10;5)$. 
    \begin{enumerate}
    \item
    
    Calculer les coordonnées des vecteurs \( \vect{ AB }\) et \( \vect{ AB }+\vect{ BC }\). 

\item
    Donner les coordonnées du point \( X\) tel que \( \vect{ AX }=\vect{ BC }\) (méthode au choix)
    \end{enumerate}
    
    



    $\vect{ AB }=(-14;8)$

    $\vect{ AB }+\vect{ BC }=(6;6)$

    $X=(24;-3)$
    
\end{enumerate}
}
\vbox{Numéro 49.
\emph{Toutes les réponses doivent être justifiées par un calcul accompagné d'un raisonnement.}
\begin{enumerate}\item

    Soient les points $D(-4;-9)$, $L(1;6)$ et $K(-4;-4)$. Donner les coordonnées du point $E$ tel que $DLKE$ soit un parallélogramme (méthode au choix).
    

$E=(-9;-19)$\item
Soient les points $A(6;-2)$, $ B(2,-6)$ et $ C(-6;-1)$. 
    \begin{enumerate}
    \item
    
    Calculer les coordonnées des vecteurs \( \vect{ AB }\) et \( \vect{ AB }+\vect{ BC }\). 

\item
    Donner les coordonnées du point \( X\) tel que \( \vect{ AX }=\vect{ BC }\) (méthode au choix)
    \end{enumerate}
    
    



    $\vect{ AB }=(-4;-4)$

    $\vect{ AB }+\vect{ BC }=(-12;1)$

    $X=(-2;3)$
    
\end{enumerate}
}
\vbox{Numéro 50.
\emph{Toutes les réponses doivent être justifiées par un calcul accompagné d'un raisonnement.}
\begin{enumerate}\item

    Soient les points $F(-10;2)$, $B(5;3)$ et $L(-8;-10)$. Donner les coordonnées du point $E$ tel que $FBLE$ soit un parallélogramme (méthode au choix).
    

$E=(-23;-11)$\item
Soient les points $A(10;-8)$, $ B(3,-2)$ et $ C(0;-8)$. 
    \begin{enumerate}
    \item
    
    Calculer les coordonnées des vecteurs \( \vect{ AB }\) et \( \vect{ AB }+\vect{ BC }\). 

\item
    Donner les coordonnées du point \( X\) tel que \( \vect{ AX }=\vect{ BC }\) (méthode au choix)
    \end{enumerate}
    
    



    $\vect{ AB }=(-7;6)$

    $\vect{ AB }+\vect{ BC }=(-10;0)$

    $X=(7;-14)$
    
\end{enumerate}
}
\vbox{Numéro 51.
\emph{Toutes les réponses doivent être justifiées par un calcul accompagné d'un raisonnement.}
\begin{enumerate}\item

    Soient les points $E(-6;10)$, $K(7;-1)$ et $A(1;-2)$. Donner les coordonnées du point $L$ tel que $EKAL$ soit un parallélogramme (méthode au choix).
    

$L=(-12;9)$\item
Soient les points $A(-7;1)$, $ B(6,5)$ et $ C(2;4)$. 
    \begin{enumerate}
    \item
    
    Calculer les coordonnées des vecteurs \( \vect{ AB }\) et \( \vect{ AB }+\vect{ BC }\). 

\item
    Donner les coordonnées du point \( X\) tel que \( \vect{ AX }=\vect{ BC }\) (méthode au choix)
    \end{enumerate}
    
    



    $\vect{ AB }=(13;4)$

    $\vect{ AB }+\vect{ BC }=(9;3)$

    $X=(-11;0)$
    
\end{enumerate}
}
\vbox{Numéro 52.
\emph{Toutes les réponses doivent être justifiées par un calcul accompagné d'un raisonnement.}
\begin{enumerate}\item
Soient les points $A(6;-5)$, $ B(-4,6)$ et $ C(-4;-6)$. 
    \begin{enumerate}
    \item
    
    Calculer les coordonnées des vecteurs \( \vect{ AB }\) et \( \vect{ AB }+\vect{ BC }\). 

\item
    Donner les coordonnées du point \( X\) tel que \( \vect{ AX }=\vect{ BC }\) (méthode au choix)
    \end{enumerate}
    
    



    $\vect{ AB }=(-10;11)$

    $\vect{ AB }+\vect{ BC }=(-10;-1)$

    $X=(6;-17)$
    \item

    Soient les points $M(-8;-3)$, $K(9;-10)$ et $D(-4;-6)$. Donner les coordonnées du point $B$ tel que $MKDB$ soit un parallélogramme (méthode au choix).
    

$B=(-21;1)$
\end{enumerate}
}
\vbox{Numéro 53.
\emph{Toutes les réponses doivent être justifiées par un calcul accompagné d'un raisonnement.}
\begin{enumerate}\item

    Soient les points $D(8;9)$, $A(-2;2)$ et $B(1;-8)$. Donner les coordonnées du point $M$ tel que $DABM$ soit un parallélogramme (méthode au choix).
    

$M=(11;-1)$\item
Soient les points $A(-5;6)$, $ B(-2,0)$ et $ C(4;10)$. 
    \begin{enumerate}
    \item
    
    Calculer les coordonnées des vecteurs \( \vect{ AB }\) et \( \vect{ AB }+\vect{ BC }\). 

\item
    Donner les coordonnées du point \( X\) tel que \( \vect{ AX }=\vect{ BC }\) (méthode au choix)
    \end{enumerate}
    
    



    $\vect{ AB }=(3;-6)$

    $\vect{ AB }+\vect{ BC }=(9;4)$

    $X=(1;16)$
    
\end{enumerate}
}
\vbox{Numéro 54.
\emph{Toutes les réponses doivent être justifiées par un calcul accompagné d'un raisonnement.}
\begin{enumerate}\item

    Soient les points $A(4;-6)$, $B(-5;3)$ et $K(-3;-8)$. Donner les coordonnées du point $D$ tel que $ABKD$ soit un parallélogramme (méthode au choix).
    

$D=(6;-17)$\item
Soient les points $A(1;3)$, $ B(0,6)$ et $ C(3;-4)$. 
    \begin{enumerate}
    \item
    
    Calculer les coordonnées des vecteurs \( \vect{ AB }\) et \( \vect{ AB }+\vect{ BC }\). 

\item
    Donner les coordonnées du point \( X\) tel que \( \vect{ AX }=\vect{ BC }\) (méthode au choix)
    \end{enumerate}
    
    



    $\vect{ AB }=(-1;3)$

    $\vect{ AB }+\vect{ BC }=(2;-7)$

    $X=(4;-7)$
    
\end{enumerate}
}
\vbox{Numéro 55.
\emph{Toutes les réponses doivent être justifiées par un calcul accompagné d'un raisonnement.}
\begin{enumerate}\item
Soient les points $A(-7;7)$, $ B(-2,-8)$ et $ C(-10;-7)$. 
    \begin{enumerate}
    \item
    
    Calculer les coordonnées des vecteurs \( \vect{ AB }\) et \( \vect{ AB }+\vect{ BC }\). 

\item
    Donner les coordonnées du point \( X\) tel que \( \vect{ AX }=\vect{ BC }\) (méthode au choix)
    \end{enumerate}
    
    



    $\vect{ AB }=(5;-15)$

    $\vect{ AB }+\vect{ BC }=(-3;-14)$

    $X=(-15;8)$
    \item

    Soient les points $M(-8;1)$, $B(-5;9)$ et $L(3;9)$. Donner les coordonnées du point $E$ tel que $MBLE$ soit un parallélogramme (méthode au choix).
    

$E=(0;1)$
\end{enumerate}
}
\vbox{Numéro 56.
\emph{Toutes les réponses doivent être justifiées par un calcul accompagné d'un raisonnement.}
\begin{enumerate}\item
Soient les points $A(-1;0)$, $ B(-6,8)$ et $ C(4;-10)$. 
    \begin{enumerate}
    \item
    
    Calculer les coordonnées des vecteurs \( \vect{ AB }\) et \( \vect{ AB }+\vect{ BC }\). 

\item
    Donner les coordonnées du point \( X\) tel que \( \vect{ AX }=\vect{ BC }\) (méthode au choix)
    \end{enumerate}
    
    



    $\vect{ AB }=(-5;8)$

    $\vect{ AB }+\vect{ BC }=(5;-10)$

    $X=(9;-18)$
    \item

    Soient les points $D(1;-7)$, $E(6;2)$ et $B(-4;-9)$. Donner les coordonnées du point $F$ tel que $DEBF$ soit un parallélogramme (méthode au choix).
    

$F=(-9;-18)$
\end{enumerate}
}
\vbox{Numéro 57.
\emph{Toutes les réponses doivent être justifiées par un calcul accompagné d'un raisonnement.}
\begin{enumerate}\item

    Soient les points $E(-2;-4)$, $B(-6;-10)$ et $A(-6;-7)$. Donner les coordonnées du point $L$ tel que $EBAL$ soit un parallélogramme (méthode au choix).
    

$L=(-2;-1)$\item
Soient les points $A(-8;4)$, $ B(8,-7)$ et $ C(8;6)$. 
    \begin{enumerate}
    \item
    
    Calculer les coordonnées des vecteurs \( \vect{ AB }\) et \( \vect{ AB }+\vect{ BC }\). 

\item
    Donner les coordonnées du point \( X\) tel que \( \vect{ AX }=\vect{ BC }\) (méthode au choix)
    \end{enumerate}
    
    



    $\vect{ AB }=(16;-11)$

    $\vect{ AB }+\vect{ BC }=(16;2)$

    $X=(-8;17)$
    
\end{enumerate}
}
\vbox{Numéro 58.
\emph{Toutes les réponses doivent être justifiées par un calcul accompagné d'un raisonnement.}
\begin{enumerate}\item

    Soient les points $L(1;-4)$, $D(0;8)$ et $K(7;-9)$. Donner les coordonnées du point $B$ tel que $LDKB$ soit un parallélogramme (méthode au choix).
    

$B=(8;-21)$\item
Soient les points $A(2;-6)$, $ B(-3,3)$ et $ C(-8;8)$. 
    \begin{enumerate}
    \item
    
    Calculer les coordonnées des vecteurs \( \vect{ AB }\) et \( \vect{ AB }+\vect{ BC }\). 

\item
    Donner les coordonnées du point \( X\) tel que \( \vect{ AX }=\vect{ BC }\) (méthode au choix)
    \end{enumerate}
    
    



    $\vect{ AB }=(-5;9)$

    $\vect{ AB }+\vect{ BC }=(-10;14)$

    $X=(-3;-1)$
    
\end{enumerate}
}
\vbox{Numéro 59.
\emph{Toutes les réponses doivent être justifiées par un calcul accompagné d'un raisonnement.}
\begin{enumerate}\item
Soient les points $A(2;6)$, $ B(-8,-3)$ et $ C(3;-9)$. 
    \begin{enumerate}
    \item
    
    Calculer les coordonnées des vecteurs \( \vect{ AB }\) et \( \vect{ AB }+\vect{ BC }\). 

\item
    Donner les coordonnées du point \( X\) tel que \( \vect{ AX }=\vect{ BC }\) (méthode au choix)
    \end{enumerate}
    
    



    $\vect{ AB }=(-10;-9)$

    $\vect{ AB }+\vect{ BC }=(1;-15)$

    $X=(13;0)$
    \item

    Soient les points $A(7;0)$, $D(7;6)$ et $L(-8;-1)$. Donner les coordonnées du point $F$ tel que $ADLF$ soit un parallélogramme (méthode au choix).
    

$F=(-8;-7)$
\end{enumerate}
}
\vbox{Numéro 60.
\emph{Toutes les réponses doivent être justifiées par un calcul accompagné d'un raisonnement.}
\begin{enumerate}\item
Soient les points $A(-2;8)$, $ B(1,10)$ et $ C(2;-7)$. 
    \begin{enumerate}
    \item
    
    Calculer les coordonnées des vecteurs \( \vect{ AB }\) et \( \vect{ AB }+\vect{ BC }\). 

\item
    Donner les coordonnées du point \( X\) tel que \( \vect{ AX }=\vect{ BC }\) (méthode au choix)
    \end{enumerate}
    
    



    $\vect{ AB }=(3;2)$

    $\vect{ AB }+\vect{ BC }=(4;-15)$

    $X=(-1;-9)$
    \item

    Soient les points $K(2;-5)$, $L(-7;9)$ et $E(6;-10)$. Donner les coordonnées du point $B$ tel que $KLEB$ soit un parallélogramme (méthode au choix).
    

$B=(15;-24)$
\end{enumerate}
}
\vbox{Numéro 61.
\emph{Toutes les réponses doivent être justifiées par un calcul accompagné d'un raisonnement.}
\begin{enumerate}\item

    Soient les points $F(9;-5)$, $A(9;-8)$ et $D(-2;4)$. Donner les coordonnées du point $M$ tel que $FADM$ soit un parallélogramme (méthode au choix).
    

$M=(-2;7)$\item
Soient les points $A(-6;1)$, $ B(-1,1)$ et $ C(6;-2)$. 
    \begin{enumerate}
    \item
    
    Calculer les coordonnées des vecteurs \( \vect{ AB }\) et \( \vect{ AB }+\vect{ BC }\). 

\item
    Donner les coordonnées du point \( X\) tel que \( \vect{ AX }=\vect{ BC }\) (méthode au choix)
    \end{enumerate}
    
    



    $\vect{ AB }=(5;0)$

    $\vect{ AB }+\vect{ BC }=(12;-3)$

    $X=(1;-2)$
    
\end{enumerate}
}
\vbox{Numéro 62.
\emph{Toutes les réponses doivent être justifiées par un calcul accompagné d'un raisonnement.}
\begin{enumerate}\item

    Soient les points $E(0;-1)$, $L(1;-5)$ et $F(-9;5)$. Donner les coordonnées du point $K$ tel que $ELFK$ soit un parallélogramme (méthode au choix).
    

$K=(-10;9)$\item
Soient les points $A(3;-6)$, $ B(0,-10)$ et $ C(-8;-2)$. 
    \begin{enumerate}
    \item
    
    Calculer les coordonnées des vecteurs \( \vect{ AB }\) et \( \vect{ AB }+\vect{ BC }\). 

\item
    Donner les coordonnées du point \( X\) tel que \( \vect{ AX }=\vect{ BC }\) (méthode au choix)
    \end{enumerate}
    
    



    $\vect{ AB }=(-3;-4)$

    $\vect{ AB }+\vect{ BC }=(-11;4)$

    $X=(-5;2)$
    
\end{enumerate}
}
\vbox{Numéro 63.
\emph{Toutes les réponses doivent être justifiées par un calcul accompagné d'un raisonnement.}
\begin{enumerate}\item
Soient les points $A(5;-6)$, $ B(-9,7)$ et $ C(-10;-4)$. 
    \begin{enumerate}
    \item
    
    Calculer les coordonnées des vecteurs \( \vect{ AB }\) et \( \vect{ AB }+\vect{ BC }\). 

\item
    Donner les coordonnées du point \( X\) tel que \( \vect{ AX }=\vect{ BC }\) (méthode au choix)
    \end{enumerate}
    
    



    $\vect{ AB }=(-14;13)$

    $\vect{ AB }+\vect{ BC }=(-15;2)$

    $X=(4;-17)$
    \item

    Soient les points $M(5;-3)$, $L(-6;-3)$ et $B(-2;-6)$. Donner les coordonnées du point $E$ tel que $MLBE$ soit un parallélogramme (méthode au choix).
    

$E=(9;-6)$
\end{enumerate}
}
\vbox{Numéro 64.
\emph{Toutes les réponses doivent être justifiées par un calcul accompagné d'un raisonnement.}
\begin{enumerate}\item
Soient les points $A(-7;1)$, $ B(-6,2)$ et $ C(-7;-5)$. 
    \begin{enumerate}
    \item
    
    Calculer les coordonnées des vecteurs \( \vect{ AB }\) et \( \vect{ AB }+\vect{ BC }\). 

\item
    Donner les coordonnées du point \( X\) tel que \( \vect{ AX }=\vect{ BC }\) (méthode au choix)
    \end{enumerate}
    
    



    $\vect{ AB }=(1;1)$

    $\vect{ AB }+\vect{ BC }=(0;-6)$

    $X=(-8;-6)$
    \item

    Soient les points $L(5;5)$, $M(7;10)$ et $E(8;-2)$. Donner les coordonnées du point $F$ tel que $LMEF$ soit un parallélogramme (méthode au choix).
    

$F=(6;-7)$
\end{enumerate}
}
\vbox{Numéro 65.
\emph{Toutes les réponses doivent être justifiées par un calcul accompagné d'un raisonnement.}
\begin{enumerate}\item

    Soient les points $M(-3;-9)$, $D(-2;0)$ et $A(5;1)$. Donner les coordonnées du point $F$ tel que $MDAF$ soit un parallélogramme (méthode au choix).
    

$F=(4;-8)$\item
Soient les points $A(-4;9)$, $ B(0,0)$ et $ C(-6;-7)$. 
    \begin{enumerate}
    \item
    
    Calculer les coordonnées des vecteurs \( \vect{ AB }\) et \( \vect{ AB }+\vect{ BC }\). 

\item
    Donner les coordonnées du point \( X\) tel que \( \vect{ AX }=\vect{ BC }\) (méthode au choix)
    \end{enumerate}
    
    



    $\vect{ AB }=(4;-9)$

    $\vect{ AB }+\vect{ BC }=(-2;-16)$

    $X=(-10;2)$
    
\end{enumerate}
}
\vbox{Numéro 66.
\emph{Toutes les réponses doivent être justifiées par un calcul accompagné d'un raisonnement.}
\begin{enumerate}\item
Soient les points $A(2;-10)$, $ B(1,-1)$ et $ C(6;6)$. 
    \begin{enumerate}
    \item
    
    Calculer les coordonnées des vecteurs \( \vect{ AB }\) et \( \vect{ AB }+\vect{ BC }\). 

\item
    Donner les coordonnées du point \( X\) tel que \( \vect{ AX }=\vect{ BC }\) (méthode au choix)
    \end{enumerate}
    
    



    $\vect{ AB }=(-1;9)$

    $\vect{ AB }+\vect{ BC }=(4;16)$

    $X=(7;-3)$
    \item

    Soient les points $K(-7;-1)$, $B(6;-2)$ et $M(-5;2)$. Donner les coordonnées du point $E$ tel que $KBME$ soit un parallélogramme (méthode au choix).
    

$E=(-18;3)$
\end{enumerate}
}
\vbox{Numéro 67.
\emph{Toutes les réponses doivent être justifiées par un calcul accompagné d'un raisonnement.}
\begin{enumerate}\item

    Soient les points $L(-10;10)$, $F(0;1)$ et $E(0;6)$. Donner les coordonnées du point $K$ tel que $LFEK$ soit un parallélogramme (méthode au choix).
    

$K=(-10;15)$\item
Soient les points $A(-5;10)$, $ B(7,-5)$ et $ C(7;5)$. 
    \begin{enumerate}
    \item
    
    Calculer les coordonnées des vecteurs \( \vect{ AB }\) et \( \vect{ AB }+\vect{ BC }\). 

\item
    Donner les coordonnées du point \( X\) tel que \( \vect{ AX }=\vect{ BC }\) (méthode au choix)
    \end{enumerate}
    
    



    $\vect{ AB }=(12;-15)$

    $\vect{ AB }+\vect{ BC }=(12;-5)$

    $X=(-5;20)$
    
\end{enumerate}
}
\vbox{Numéro 68.
\emph{Toutes les réponses doivent être justifiées par un calcul accompagné d'un raisonnement.}
\begin{enumerate}\item

    Soient les points $D(-7;-2)$, $F(-6;2)$ et $K(9;-8)$. Donner les coordonnées du point $E$ tel que $DFKE$ soit un parallélogramme (méthode au choix).
    

$E=(8;-12)$\item
Soient les points $A(7;-6)$, $ B(6,-2)$ et $ C(-9;0)$. 
    \begin{enumerate}
    \item
    
    Calculer les coordonnées des vecteurs \( \vect{ AB }\) et \( \vect{ AB }+\vect{ BC }\). 

\item
    Donner les coordonnées du point \( X\) tel que \( \vect{ AX }=\vect{ BC }\) (méthode au choix)
    \end{enumerate}
    
    



    $\vect{ AB }=(-1;4)$

    $\vect{ AB }+\vect{ BC }=(-16;6)$

    $X=(-8;-4)$
    
\end{enumerate}
}
\vbox{Numéro 69.
\emph{Toutes les réponses doivent être justifiées par un calcul accompagné d'un raisonnement.}
\begin{enumerate}\item
Soient les points $A(-1;-9)$, $ B(5,-8)$ et $ C(4;8)$. 
    \begin{enumerate}
    \item
    
    Calculer les coordonnées des vecteurs \( \vect{ AB }\) et \( \vect{ AB }+\vect{ BC }\). 

\item
    Donner les coordonnées du point \( X\) tel que \( \vect{ AX }=\vect{ BC }\) (méthode au choix)
    \end{enumerate}
    
    



    $\vect{ AB }=(6;1)$

    $\vect{ AB }+\vect{ BC }=(5;17)$

    $X=(-2;7)$
    \item

    Soient les points $E(-3;3)$, $F(-4;5)$ et $D(-3;10)$. Donner les coordonnées du point $A$ tel que $EFDA$ soit un parallélogramme (méthode au choix).
    

$A=(-2;8)$
\end{enumerate}
}
\vbox{Numéro 70.
\emph{Toutes les réponses doivent être justifiées par un calcul accompagné d'un raisonnement.}
\begin{enumerate}\item

    Soient les points $K(-2;-6)$, $F(8;9)$ et $B(5;10)$. Donner les coordonnées du point $M$ tel que $KFBM$ soit un parallélogramme (méthode au choix).
    

$M=(-5;-5)$\item
Soient les points $A(5;8)$, $ B(-10,-6)$ et $ C(2;6)$. 
    \begin{enumerate}
    \item
    
    Calculer les coordonnées des vecteurs \( \vect{ AB }\) et \( \vect{ AB }+\vect{ BC }\). 

\item
    Donner les coordonnées du point \( X\) tel que \( \vect{ AX }=\vect{ BC }\) (méthode au choix)
    \end{enumerate}
    
    



    $\vect{ AB }=(-15;-14)$

    $\vect{ AB }+\vect{ BC }=(-3;-2)$

    $X=(17;20)$
    
\end{enumerate}
}
\vbox{Numéro 71.
\emph{Toutes les réponses doivent être justifiées par un calcul accompagné d'un raisonnement.}
\begin{enumerate}\item
Soient les points $A(-10;8)$, $ B(-9,10)$ et $ C(-6;-1)$. 
    \begin{enumerate}
    \item
    
    Calculer les coordonnées des vecteurs \( \vect{ AB }\) et \( \vect{ AB }+\vect{ BC }\). 

\item
    Donner les coordonnées du point \( X\) tel que \( \vect{ AX }=\vect{ BC }\) (méthode au choix)
    \end{enumerate}
    
    



    $\vect{ AB }=(1;2)$

    $\vect{ AB }+\vect{ BC }=(4;-9)$

    $X=(-7;-3)$
    \item

    Soient les points $K(9;1)$, $L(-2;-8)$ et $F(-6;-7)$. Donner les coordonnées du point $A$ tel que $KLFA$ soit un parallélogramme (méthode au choix).
    

$A=(5;2)$
\end{enumerate}
}
\vbox{Numéro 72.
\emph{Toutes les réponses doivent être justifiées par un calcul accompagné d'un raisonnement.}
\begin{enumerate}\item

    Soient les points $M(8;10)$, $E(-9;-6)$ et $L(-4;2)$. Donner les coordonnées du point $F$ tel que $MELF$ soit un parallélogramme (méthode au choix).
    

$F=(13;18)$\item
Soient les points $A(5;10)$, $ B(-7,-10)$ et $ C(1;8)$. 
    \begin{enumerate}
    \item
    
    Calculer les coordonnées des vecteurs \( \vect{ AB }\) et \( \vect{ AB }+\vect{ BC }\). 

\item
    Donner les coordonnées du point \( X\) tel que \( \vect{ AX }=\vect{ BC }\) (méthode au choix)
    \end{enumerate}
    
    



    $\vect{ AB }=(-12;-20)$

    $\vect{ AB }+\vect{ BC }=(-4;-2)$

    $X=(13;28)$
    
\end{enumerate}
}
\vbox{Numéro 73.
\emph{Toutes les réponses doivent être justifiées par un calcul accompagné d'un raisonnement.}
\begin{enumerate}\item
Soient les points $A(1;2)$, $ B(-4,9)$ et $ C(-10;4)$. 
    \begin{enumerate}
    \item
    
    Calculer les coordonnées des vecteurs \( \vect{ AB }\) et \( \vect{ AB }+\vect{ BC }\). 

\item
    Donner les coordonnées du point \( X\) tel que \( \vect{ AX }=\vect{ BC }\) (méthode au choix)
    \end{enumerate}
    
    



    $\vect{ AB }=(-5;7)$

    $\vect{ AB }+\vect{ BC }=(-11;2)$

    $X=(-5;-3)$
    \item

    Soient les points $A(0;7)$, $K(-5;-1)$ et $B(-3;6)$. Donner les coordonnées du point $F$ tel que $AKBF$ soit un parallélogramme (méthode au choix).
    

$F=(2;14)$
\end{enumerate}
}
\vbox{Numéro 74.
\emph{Toutes les réponses doivent être justifiées par un calcul accompagné d'un raisonnement.}
\begin{enumerate}\item

    Soient les points $M(-1;-6)$, $F(-7;8)$ et $A(10;-5)$. Donner les coordonnées du point $E$ tel que $MFAE$ soit un parallélogramme (méthode au choix).
    

$E=(16;-19)$\item
Soient les points $A(7;-3)$, $ B(-9,2)$ et $ C(-3;9)$. 
    \begin{enumerate}
    \item
    
    Calculer les coordonnées des vecteurs \( \vect{ AB }\) et \( \vect{ AB }+\vect{ BC }\). 

\item
    Donner les coordonnées du point \( X\) tel que \( \vect{ AX }=\vect{ BC }\) (méthode au choix)
    \end{enumerate}
    
    



    $\vect{ AB }=(-16;5)$

    $\vect{ AB }+\vect{ BC }=(-10;12)$

    $X=(13;4)$
    
\end{enumerate}
}


\end{document}

