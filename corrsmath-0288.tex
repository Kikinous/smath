% This is part of Un soupçon de mathématique sans être agressif pour autant
% Copyright (c) 2013
%   Laurent Claessens
% See the file fdl-1.3.txt for copying conditions.

\begin{corrige}{smath-0288}

    L'abscisse du sommet est donnée par la formule \( x_0=-\frac{ b }{ 2a }\). Ici, l'abscisse du sommet est \( x_0=-\frac{ 1 }{2}\). Les coordonnées du sommet sont alors
    \begin{equation}
        S=\left( -\frac{ 1 }{2};-\frac{ 25 }{ 4 } \right).
    \end{equation}
    
    Les racines sont données par les formules à connaître :
    \begin{equation}
        \frac{ -1\pm\sqrt{1-4\times (-6)} }{2}=\frac{ -1\pm\sqrt{25} }{ 2 }.
    \end{equation}
    Les racines sont
    \begin{equation}
        \begin{aligned}[]
            x_1&=2\\
            x_2&=-3.
        \end{aligned}
    \end{equation}
    
    Le tableau de signe est le suivant :
    \begin{equation*}
        \begin{array}[]{|c||c|c|c|c|c|}
            \hline
             x&&-3&&2&\\
              \hline\hline
              f(x)&+&0&-&0&+\\ 
              \hline 
               \end{array}
           \end{equation*}
           parce que le signe du coefficient de \( x^2\) est positif.

\end{corrige}
