% This is part of Un soupçon de mathématique sans être agressif pour autant
% Copyright (c) 2014
%   Laurent Claessens
% See the file fdl-1.3.txt for copying conditions.

\begin{exercice}\label{exosmath-0699}

    Questions de type «calcul mental». Justifier les réponses.
    \begin{multicols}{2}
    \begin{enumerate}
        \item
            Résoudre $6x+2=14-4x$.
        \item
            Simplifier
            \begin{equation}
                \frac{ 21x+14b }{ 7x };
            \end{equation}
            dire par combien vous avez simplifié.
        \item
            Quelle est l'ordonnée du point d'abscisse \( 12\) sur la droite \( y=5x-2\) ?
        \item
            Développer \( (x+4)^2\).
        \item
            Donner l'équation de la droite parallèle à \( y=5x+2\) passant par le point \( A(1;1)\).
    \end{enumerate}
    \end{multicols}

\corrref{smath-0699}
\end{exercice}
