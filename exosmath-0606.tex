% This is part of Un soupçon de mathématique sans être agressif pour autant
% Copyright (c) 2014
%   Laurent Claessens
% See the file fdl-1.3.txt for copying conditions.

\begin{exercice}\label{exosmath-0606}

\begin{wrapfigure}{r}{5.0cm}
   \vspace{-1.5cm}        % à adapter.
   \centering
   \input{Fig_XMjsBcU.pstricks}
\end{wrapfigure}

À partir du dessin ci-contre :
\begin{enumerate}
    \item
        Déterminer l'équation de la droite \( d_1\).
    \item
        Tracer une droite \( d_2\) parallèle à \( d_1\) et donner son équation.
    \item
        Déterminer une équation de la droite \( d_3\) passant par \( A(-10;-6)\) et \( B(14,6)\).
    \item
        Calculer les coordonnées de l'intersection \( d_1\cap d_3\).
\end{enumerate}


\corrref{smath-0606}
\end{exercice}
