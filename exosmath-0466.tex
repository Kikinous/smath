% This is part of Un soupçon de mathématique sans être agressif pour autant
% Copyright (c) 2013
%   Laurent Claessens
% See the file fdl-1.3.txt for copying conditions.

\begin{exercice}\label{exosmath-0466}

    Soit \( \vect{ v }\) les vecteur de coordonnées \( \vect{v }=\begin{pmatrix}
        -5    \\ 
        15    
    \end{pmatrix}\) et le point \( K=(100;-34)\). Les affirmations suivantes sont-elles vraies ou fausses ?
    \begin{enumerate}
        \item
            \( \vect{ v }\) est colinéaire au vecteur \( \begin{pmatrix}
                1    \\ 
                3    
            \end{pmatrix}\).
        \item
            Si \( A=(95;-19)\) alors \( \vect{ v }=\vect{ KA }\).
    \end{enumerate}

\corrref{smath-0466}
\end{exercice}
