% This is part of Un soupçon de mathématique sans être agressif pour autant
% Copyright (c) 2015
%   Laurent Claessens
% See the file fdl-1.3.txt for copying conditions.

%--------------------------------------------------------------------------------------------------------------------------- 
\subsection*{Activité : qui a dit «proportionnel» ?}
%---------------------------------------------------------------------------------------------------------------------------

Les situations suivantes relèvent-elles d’une situation de proportionnalité ? Pourquoi ?
\begin{enumerate}
    \item

 Saïd achète 2 mètres de corde qui coûte 2,30 € le mètre.

\item
    Daniel a planté dans son potager 8 pieds de tomates et en a récolté \SI{14}{\kilo\gram}. L'an passé, il en avait planté 12 pieds et en avait récolté \SI{18}{\kilo\gram}. L'an prochain, il en plantera 10 pieds et espère en récolter \SI{16}{\kilo\gram}. 

\item
 À 6 ans, Armand chaussait du 30 et à 18 ans, il chausse du 42.
\item

 Abonnement à une revue : \( 6\) mois pour \( 18\)€, un an pour \( 32\)€ et \( 2\) ans pour \( 60\)€.
\item
    Un piéton se promène à allure régulière le long des quais de la Seine et parcourt \SI{3.5}{\kilo\meter} en 1 h 30.
\item

    On peut acheter de l'enduit de lissage par sac de 1 kg, 5 kg et 25 kg. Le mode d’emploi précise qu'il faut \SI{2.5}{\liter} d’eau pour \SI{10}{\kilo\gram}.
\item

 Un commerçant a décidé de faire une journée promotion en baissant tous les prix de $10$\%.
\item

 Un loueur de DVD propose la formule d'abonnement suivante : la carte d'adhésion coûte $10$€ et on paye $2$€ par DVD.

\end{enumerate}
