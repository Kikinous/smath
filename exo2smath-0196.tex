% This is part of Un soupçon de mathématique sans être agressif pour autant
% Copyright (c) 2015
%   Laurent Claessens
% See the file fdl-1.3.txt for copying conditions.

\begin{exercice}\label{exo2smath-0196}

\begin{center}                                                             
    \input{Fig_MWFBooZHANAt.pstricks}           
\end{center}

\begin{enumerate}
    \item
        Par quelle expression peut-on remplacer le point d'interrogation ?
        \begin{enumerate}
            \item
                \( 10-x\)
            \item
                \( x-3\)
            \item
                \( 6\)
        \end{enumerate}
    \item
        Exprimer en fonction de \( x\) l'aire blanche et l'aire hachurée.
\end{enumerate}

\corrref{2smath-0196}
\end{exercice}
