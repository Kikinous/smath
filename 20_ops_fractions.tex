% This is part of Un soupçon de mathématique sans être agressif pour autant
% Copyright (c) 2014
%   Laurent Claessens
% See the file fdl-1.3.txt for copying conditions.

% This is part of Un soupçon de mathématique sans être agressif pour autant
% Copyright (c) 2014
%   Laurent Claessens
% See the file fdl-1.3.txt for copying conditions.

%+++++++++++++++++++++++++++++++++++++++++++++++++++++++++++++++++++++++++++++++++++++++++++++++++++++++++++++++++++++++++++ 
\section*{Activité : secteurs d'émissions (1)}
%+++++++++++++++++++++++++++++++++++++++++++++++++++++++++++++++++++++++++++++++++++++++++++++++++++++++++++++++++++++++++++

Sur cette planète, un cinquième des émissions de dioxyde de carbone sont dus aux processus industriels, un autre cinquième aux transports et un dixième aux bâtiments. Quelle fraction du total des émissions de \( CO_2\) est due à ces trois activités ?

Colorier l'égalité suivante :
\begin{center}
    \input{Fig_GLVooNqioTo0.pstricks}+\input{Fig_GLVooNqioTo1.pstricks}+\input{Fig_GLVooNqioTo2.pstricks}=\input{Fig_GLVooNqioTo3.pstricks}.
\end{center}

Les centrales énergétiques sont responsables d'un tiers des émissions de dioxyde de carbone. Quel est le total de ces quatre activités ?

\noindent {\scriptsize Pour plus d'informations, voir\ \url{http://savoirsenmultimedia.ens.fr/uploads/videos//diffusion/2012_02_09_jancovici.mp4}}


%+++++++++++++++++++++++++++++++++++++++++++++++++++++++++++++++++++++++++++++++++++++++++++++++++++++++++++++++++++++++++++ 
\section{Addition et soustraction}
%+++++++++++++++++++++++++++++++++++++++++++++++++++++++++++++++++++++++++++++++++++++++++++++++++++++++++++++++++++++++++++

\begin{Aretenir}
    Pour additionner (ou soustraire) des nombres en écriture fractionnaire :
    \begin{itemize}
        \item 
            on écrit les nombres avec le même dénominateur ;
        \item 
            on additionne (ou on soustrait) les numérateurs et on garde le dénominateur commun.
    \end{itemize}
\end{Aretenir}

\begin{example}
    À calculer : 
    \begin{equation}
        A=\frac{ 7 }{ 3 }+\frac{ 6 }{ 12 }.
    \end{equation}
    Nous procédons comme suit.
    \begin{enumerate}
        \item
            Le dénominateur commun est \( 12\) parce que \( 12\) est un multiple à la fois de \( 3\) et de \( 12\).
            \begin{equation}
                \frac{ 7\times 4 }{ 3\times 4 }+\frac{ 6 }{ 12 }
            \end{equation}
        \item
            La somme à calculer devient :
            \begin{equation}
                A=\frac{ 28 }{ 12 }+\frac{ 6 }{ 12 }
            \end{equation}
        \item
            Nous sommons les numérateurs :
            \begin{equation}
                A=\frac{ 28+6 }{ 12 }=\frac{ 34 }{ 12 }
            \end{equation}
        \item
            Nous simplifions le résultat :
            \begin{equation}
                A=\frac{ 17 }{ 6 }.
            \end{equation}
    \end{enumerate}
\end{example}

\begin{example}
    Parfois le dénominateur commun n'est aucun des deux dénominateurs :
    \begin{equation}
        B=\frac{5}{ 6 }-\frac{ 3 }{ 14 }.
    \end{equation}
    Le plus petit dénominateur commun est \( 42\) : c'est le plus petit nombre à être multiple en même temps de \( 6\) et \( 14\). En choisissant cela,
    \begin{equation}
        B=\frac{ 5\times 7 }{ 6\times 7 }-\frac{ 3\times 3 }{ 14\times 3 }=\frac{ 35 }{ 42 }-\frac{ 9 }{ 42 }=\frac{ 26 }{ 42 }=\frac{ 13 }{ 21 }.
    \end{equation}
    
    Dans ce cas, il est peut-être plus simple de choisir le produit \( 14\times 6\) comme dénominateur commun (ça marche toujours) :
    \begin{equation}
        B=\frac{ 5\times 14 }{ 6\times 14 }-\frac{ 3\times 6 }{ 14\times 6 }=\frac{ 70 }{ 84 }-\frac{ 18 }{ 84 }=\frac{ 52 }{ 84 }=\frac{ 13 }{ 21 }.
    \end{equation}
    Prendre le produit des dénominateurs fonctionne toujours, mais mène à des calculs sur de plus grands nombres.    
\end{example}


%+++++++++++++++++++++++++++++++++++++++++++++++++++++++++++++++++++++++++++++++++++++++++++++++++++++++++++++++++++++++++++ 
\section{Produit}
%+++++++++++++++++++++++++++++++++++++++++++++++++++++++++++++++++++++++++++++++++++++++++++++++++++++++++++++++++++++++++++

% This is part of Un soupçon de mathématique sans être agressif pour autant
% Copyright (c) 2014
%   Laurent Claessens
% See the file fdl-1.3.txt for copying conditions.

%--------------------------------------------------------------------------------------------------------------------------- 
\subsection*{Activité : fraction d'un grand rectangle}
%---------------------------------------------------------------------------------------------------------------------------

Quel est le périmètre du rectangle grisé ? Quelle est son aire ?
\begin{center}
   \input{Fig_FDOooRCCWGn.pstricks}
\end{center}

De \cite{NRHooXFvgpp5}

\begin{Aretenir}
Pour multiplier des nombres en écriture fractionnaire, on multiplie les numérateurs entre eux et les dénominateurs entre eux.
\end{Aretenir}

\begin{example}
    Calculer : \( A=\dfrac{ 8 }{ 7 }\times \dfrac{ 5 }{ 3 }\). On fait :
    \begin{subequations}
        \begin{align}
            A&=\frac{ 8 }{ 7 }\times \frac{ 5 }{ 3 }\\
            &=\frac{ 8\times 5 }{ 7\times 3 }\\
            &=\frac{ 40 }{ 21 }.
        \end{align}
    \end{subequations}
\end{example}

\begin{Aretenir}
    La multiplication de fraction correspond à ce qu'en français on nomme «telle partie {\bf de}».
\end{Aretenir}

\begin{example}
    Une canette contient \SI{33}{\centi\liter}. Un quart de la canette contient
    \begin{equation}
        \frac{1}{ 4 }\times 33 =\SI{8.25}{\centi\liter}.
    \end{equation}
\end{example}

\begin{example}
    Manger les trois quart d'un paquet d'un demi kilo de pâtes revient à manger
    \begin{equation}
        \SI{\frac{ 3 }{ 4 }\times \frac{1}{ 2 }}{\kilo\gram}=\SI{\frac{ 3 }{ 8 }}{\kilo\gram}
    \end{equation}
    de pâtes.
\end{example}
