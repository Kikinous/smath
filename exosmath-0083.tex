% This is part of Un soupçon de mathématique sans être agressif pour autant
% Copyright (c) 2012-2013
%   Laurent Claessens
% See the file fdl-1.3.txt for copying conditions.

\begin{exercice}\label{exosmath-0083}

    Dans chacune des configurations suivantes, déterminer par le calcul si les droites \( (AB)\) et \( (CD)\) sont parallèles.
    \begin{enumerate}
        \item
            \( A=(3,7)\), \( B=(0,1)\), \( C=(10;13)\), \( D=(-6;-19)\)
        \item
            \( A=(1;5)\), \( B=(4;-10)\), \( C=(0;-1)\), \( D=(\frac{ 1 }{2},-\frac{ 1 }{2})\).
        \item
            \( A=(3;8)\), \( B=(3;-1)\), \( C=(-4;1)\), \( D=(-4;10)\)
    \end{enumerate}
    Conseil : (au moins) deux méthodes sont possibles.
    \begin{itemize}
        \item Calculer le coefficient directeur de la droite (déplacement vertical divisé par le déplacement horizontal)
        \item Voir si le vecteur \( \vect{ AB } \) est colinéaire au vecteur \( \vect{ CD }\).
    \end{itemize}
    À vous de choisir laquelle vous préférez.

\corrref{smath-0083}
\end{exercice}
