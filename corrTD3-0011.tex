% This is part of Exercices de mathématique pour SVT
% Copyright (C) 2010
%   Laurent Claessens et Carlotta Donadello
% See the file fdl-1.3.txt for copying conditions.

\begin{corrige}{TD3-0011}

	\begin{enumerate}
		\item
			Supposons que la formule soit vraie pour tous les termes jusqu'à $k+1$, et prouvons qu'elle est encore vraie pour le terme numéro $k+2$. Pour simplifier la notation, nous notons
			\begin{equation}
				\begin{aligned}[]
					A&=\frac{ 1+\sqrt{5} }{2}	&B&=\frac{ 1-\sqrt{5} }{2}.
				\end{aligned}
			\end{equation}
			En utilisant la définition de $u_{k+2}$ et en supposant que $u_{k+1}=\frac{1}{ \sqrt{5} }(A^{k+1}-B^{k+1})$ et $u_k=\frac{1}{ \sqrt{5} }(A^k-B^k)$, nous avons
			\begin{equation}		\label{EqFiboUnMOcheAB}
				\begin{aligned}[]
					u_{k+2}&=u_{k+1}+u_k\\
					&=\frac{1}{ \sqrt{5} }(A^{k+1}+A^k-B^{k+1}-B^{k})\\
					&=\frac{1}{ \sqrt{5} }\big( A^k(A+1)-B^k(B+1) \big).
				\end{aligned}
			\end{equation}
			Nous voudrions obtenir
			\begin{equation}
				u_{k+2}=\frac{1}{ \sqrt{5} }(A^{k+2}-B^{k+2}).
			\end{equation}
			En comparant avec l'expression \eqref{EqFiboUnMOcheAB}, nous devons prouver que $A+1=A^2$ et $B+1=B^2$. Cela est vrai parce que
			\begin{equation}
				A+1=\frac{ 1+\sqrt{5} }{ 2 }=\frac{ 3+\sqrt{5} }{2},
			\end{equation}
			mais
			\begin{equation}
				A^2=\left( \frac{ 1+\sqrt{5} }{2} \right)^2=\frac{ 1+5+2\sqrt{5} }{ 4 }=\frac{ 3+\sqrt{5} }{2}.
			\end{equation}
			Il en va de même pour $B$ : $B+1=\frac{ 3-\sqrt{5} }{2}$, et
			\begin{equation}
				B^2=\frac{ 1+5-2\sqrt{5} }{ 4 }=\frac{ 6-2\sqrt{5} }{ 4 }=\frac{ 3-\sqrt{5} }{ 2 }.
			\end{equation}
		\item
			Faisons un petit schéma. Les $\heartsuit$ désignent les couples en âge de se reproduire tandis que les $\clubsuit$ désignent les couples qui viennent de naître.
			\begin{equation}
			\xymatrix{%
			&	&	&				&				&\clubsuit\ar[d]		&	&			&	\\
			&	&	&				&				&\heartsuit\ar[dll]\ar[drr]	&	&			&	\\
			&	&	&\heartsuit\ar[dll]\ar[dr]	&				&				&	&\clubsuit\ar[d]	&	\\
			&\heartsuit\ar[dl]\ar[dr]	&	&	&\clubsuit\ar[d]	&	&	&\heartsuit\ar[dl]\ar[dr]	&\\
			\heartsuit&	&\clubsuit&	&\heartsuit&	&	\heartsuit&&\clubsuit
			   }
			\end{equation}
			   À chaque nouvelle étape, les $\heartsuit$ restent, mais chaque $\heartsuit$ donne naissance à un $\clubsuit$, tandis que les $\clubsuit$ deviennent des $\heartsuit$. Donc à l'étape $n$, le nombre de $\heartsuit$ est égal au nombre de $\heartsuit$ de l'étape $n-1$ plus le nombre de $\clubsuit$ de l'étape $n-1$. Mais le nombre de $\clubsuit$ de l'étape $n-1$ est égal au nombre de $\heartsuit$ de l'étape $n-2$. Par conséquent $u_n=u_{n-1}+u_{n-2}$.
	\end{enumerate}

\end{corrige}
