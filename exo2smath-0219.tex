% This is part of Un soupçon de mathématique sans être agressif pour autant
% Copyright (c) 2015
%   Laurent Claessens
% See the file fdl-1.3.txt for copying conditions.

\begin{exercice}[\cite{NRHooXFvgpp5}]\label{exo2smath-0219}

\begin{center}
   \input{Fig_PYSIooVNiiuo.pstricks}
\end{center}

{\bf Horizontalement}

a : Opposé de $8$ $\diamond$ Positif et négatif à la fois.

b : $-13+215-7-6$

c : Opposé de \( -5\) $\diamond$ \( -(-6-6)\)

d : $-0.5+1.5$ $\diamond$ Opposé de l'opposé de \( 6\).

{\bf Verticalement}

1 : Entier relatif compris entre $-15.6$ et $-14,9$.

2 : $(-3+7)-(4-88)$ $\diamond$ $(-4)-(-5)$

3 : $52+34-(35-41) -(8-7)$.

4 : $(-3)-(-3)$ $\diamond$ $2$ dizaines et $6$ unités.

\corrref{2smath-0219}
\end{exercice}
