% This is part of Un soupçon de mathématique sans être agressif pour autant
% Copyright (c) 2013-2014
%   Laurent Claessens
% See the file fdl-1.3.txt for copying conditions.

\begin{exercice}\label{exosmath-0594}

\begin{wrapfigure}{r}{5.0cm}
   \vspace{-0.5cm}        % à adapter.
   \centering
   \input{Fig_XDdastq.pstricks}
\end{wrapfigure}

Donner la ou les bonnes possibilités.

        \begin{enumerate}
            \item
                L'image de \( E\) par la translation de vecteur \( \vect{ JP }\) est :
                \begin{multicols}{3}
                    \begin{enumerate}
                        \item $H$\item $P$\item \( L\)\item \( P\).
                    \end{enumerate}
                \end{multicols}
            \item
                Le vecteur \( \vect{ AE }\) est égal au(x) vecteur(s)
                \begin{multicols}{3}
                    \begin{enumerate}
                        \item \( \vect{ GH }\)\item \( \vect{ QM }\)\item \( \vect{ HM }\)\item \( \vect{ AI }\).
                    \end{enumerate}
                \end{multicols}
        \end{enumerate}

Faire l'exercice 11 de la page 211.

\corrref{smath-0594}
\end{exercice}
