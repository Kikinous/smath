% This is part of Un soupçon de mathématique sans être agressif pour autant
% Copyright (c) 2013
%   Laurent Claessens
% See the file fdl-1.3.txt for copying conditions.

\begin{exercice}\label{exosmath-0375}

    L'économiste britannique \wikipedia{fr}{Malthus}{Malthus} pense que sans freins, les populations augmentent en suivant une suite géométrique alors que les ressources n'augmentent que suivant une suite arithmétique. D'où le concept de \wikipedia{fr}{Catastrophe_Malthusienne}{catastrophe malthusienne}.

    Vers \( 1800\), la population anglaise est estimée à \( 8\) millions d'habitants alors que l'agriculture ne pouvait nourrir plus de \( 10\) millions de personnes. En supposant que la population augmente de \( 2\%\) par an et que l'amélioration des techniques agricoles permettent de nourrir \( 500\,000\) personnes supplémentaires par an, calculer en quelle année la situation anglaise deviendrait critique.

    \vspace{1cm}

    Idée pour aller plus loin. Trouver des courbes de population pour la France entre 1100 et 1400; étudier en même temps l'évolution des techniques agricoles. Est-ce que la \wikipedia{fr}{Crise\_de\_la\_fin\_du\_Moyen\_Âge}{famine} de 1314 suivie de la guerre de cent ans (1337) accompagnée de la peste noire (1347) n'est pas une catastrophe malthusienne ? Pour faire court, en 1400 la population européenne est la moitié de celle de 1300.
\corrref{smath-0375}
\end{exercice}
