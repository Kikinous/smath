% This is part of Un soupçon de mathématique sans être agressif pour autant
% Copyright (c) 2014
%   Laurent Claessens
% See the file fdl-1.3.txt for copying conditions.

\begin{exercice}\label{exo2smath-0019}

    Une étude portant sur certaine exploitations agricoles de betteraves a noté la production en tonnes et la superficie d'un certain nombre d'exploitations. Voici les résultats :
    \begin{equation*}
        \begin{array}[]{|c|c|c|c|c|c|}
            \hline
            \text{superficie (dizaine d'hectare)}&5      &   7.7    &  12    &  23     &   50.8\\
              \hline\hline
              \text{Production (millier de tonne)}&4.85  &  7.469&   11.640   &   22.31 &   49.276\\ 
              \hline 
               \end{array}
    \end{equation*}
    (sources : invention personnelle et \href{ http://fr.wikipedia.org/wiki/Rendement_agricole }{ Wikipédia })
    \begin{enumerate}
        \item
            Représenter ces données sous forme d'un graphique.
        \item
            Est-ce une situation de proportionnalité ? Justification par le graphique ou par le calcul, au choix.
        \item
            Quelle superficie d'exploitation faudrait-il pour produire cent mille tonnes de betteraves ?
    \end{enumerate}

\corrref{2smath-0019}
\end{exercice}
