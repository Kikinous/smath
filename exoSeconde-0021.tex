% This is part of Un soupçon de mathématique sans être agressif pour autant
% Copyright (c) 2012
%   Laurent Claessens
% See the file fdl-1.3.txt for copying conditions.

\begin{exercice}\label{exoSeconde-0021}

    On se place dans un repère orthonormé\cite{BAIljy}.
    \begin{enumerate}
        \item
            Dessiner les points \( A=(-3,-4)\), \( B=(-1,6)\), \( C=(3,2)\) et \( D=(1,-8)\).
        \item
            Montrer que \( ABCD\) est un parallélogramme.
        \item
            Nous nommons \( E\) le point tel que \( C\) soit le milieu du segment \( [EB]\). Montrer par le calcul que les coordonnées de \( E\) sont \( (7,-2)\).
        \item
            Quelle est la nature du quadrilatère \( ACED\) ? Justifier.
    \end{enumerate}

\corrref{Seconde-0021}
\end{exercice}
