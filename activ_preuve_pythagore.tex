% This is part of Un soupçon de mathématique sans être agressif pour autant
% Copyright (c) 2014
%   Laurent Claessens
% See the file fdl-1.3.txt for copying conditions.

%--------------------------------------------------------------------------------------------------------------------------- 
\subsection*{Activité : démonstration du théorème de Pythagore}
%---------------------------------------------------------------------------------------------------------------------------

Nous considérons un triangle rectangle de côtés \( a\), \( b\) et \( c\) :
\begin{center}
   \input{Fig_YDAooMNHhCNT.pstricks}
\end{center}

\begin{enumerate}
    \item
        Quelle égalité voulons-nous montrer, en termes de \( a\), \( b\) et \( c\) ?
    \item
        \( \alpha+\beta=\ldots\)
\end{enumerate}

Ensuite nous disposons quatre de ces triangles pour constituer un carré de cette façon-ci :
\begin{center}
   \input{Fig_YDAooMNHhCNTh.pstricks}
\end{center}

\begin{enumerate}
    \item
        Pourquoi peut-on affirmer que le quadrilatère \( MNOP\) est-il un losange ?
    \item
        Quelle est la mesure de l'angle $\widehat{PMN}$ ?
    \item
        Quelle est la nature exacte du quadrilatère \( MNOP\) ?
    \item
        Exprimer son aire en fonction de \( c\).
\end{enumerate}

Maintenant nous replaçons les quatre triangles rectangles (identiques) de façon un peu différente :

\begin{center}
   \input{Fig_YDAooMNHhCNO.pstricks}
\end{center}

\begin{enumerate}
    \item
        Quelle est l'aire du carré \( RBNL\) ?
    \item
        Quelle est l'aire du carré \( ARKP\) ?
    \item
        Comparer les aires remplies sur les deux dessins.
    \item
        Conclure.
\end{enumerate}

