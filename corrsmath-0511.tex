% This is part of Un soupçon de mathématique sans être agressif pour autant
% Copyright (c) 2013
%   Laurent Claessens
% See the file fdl-1.3.txt for copying conditions.

\begin{corrige}{smath-0511}

    \begin{enumerate}
        \item
            \( f(0)=0\), \( f(1)=2\), \( f(2)=4\) et \( f(3)=6\).
        \item
            \( g(0)=4\), \( g(1)=3\), \( g(2)=2\) et \( g(3)=1\).
        \item
            Les ordonnées à l'origine sont faciles : ce sont \( f(0)\) et \( g(0)\). Cherchons pour \( f\). D'abord son coefficient directeur est \( f(0)=0\). Ensuite il faut trouver \( m\) pour que la droite \( mx\) soit celle représentée. Il est vite vu que c'est
            \begin{equation}
                f(x)=2x.
            \end{equation}
            En ce qui concerne \( g\) nous avons déjà \( 4\) comme ordonnée à l'origine, ensuite nous cherchons \( m\) tel que la droite \( g\) représentée soit la droite \( mx+4\). C'est
            \begin{equation}
                g(x)=-x+4.
            \end{equation}
        \item
            La fonction \( h\) donnée est une fonction affine; sa représentation graphique sera donc une droite. Il suffit alors d'en trouver deux points. Le plus facile est le point d'abscisse \( 0\) : \( h(0)=2\), donc le point \( (0;2)\) est sur le graphe.

            Ensuite on peut calculer \( f(1)=6\), ce qui donne le point \( (1;6)\). Il faut donc tracer la droite passant par les points \( (0;2)\) et \(   (1;6)\).

    \end{enumerate}

\end{corrige}
