% This is part of Un soupçon de mathématique sans être agressif pour autant
% Copyright (c) 2015
%   Laurent Claessens
% See the file fdl-1.3.txt for copying conditions.

\begin{exercice}[\cite{NRHooXFvgpp5}]\label{exo2smath-0122}

%--------------------------------------------------------------------------------------------------------------------------- 
\subsection*{Activité : encore des confitures}
%---------------------------------------------------------------------------------------------------------------------------

Voici quelque ingrédients utilisés pour des confitures.
\begin{center}
    \begin{tabular}[]{|c|c|}
        \hline
        Confiture d'abricots& «\SI{500}{\gram} de sucre et \SI{500}{\gram} d'abricots» \\
        \hline
        Confiture de fraises&«\SI{450}{\gram} de sucre et \SI{750}{\gram} de fraises» \\
        \hline
        Confiture de cerises&  «\SI{800}{\gram} de sucre et \SI{2400}{\gram} de cerises» \\ 
        \hline
    \end{tabular}
\end{center}
Est-ce que la quantité de sucre ajoutée est proportionnelle à la quantité de fruits ?


\corrref{2smath-0122}
\end{exercice}
