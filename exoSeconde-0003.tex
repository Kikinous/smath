% This is part of Un soupçon de mathématique sans être agressif pour autant
% Copyright (c) 2012
%   Laurent Claessens
% See the file fdl-1.3.txt for copying conditions.

\begin{exercice}\label{exoSeconde-0003}

    Les points \( (1;1)\), \( (3;1)\) et \( (2;2)\) forment-ils un triangle isocèle ? Justifier la réponse en utilisant des propriétés (ou la définition) du triangle isocèle.

\corrref{Seconde-0003}
\end{exercice}
