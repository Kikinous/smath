% This is part of Un soupçon de mathématique sans être agressif pour autant
% Copyright (c) 2014
%   Laurent Claessens
% See the file fdl-1.3.txt for copying conditions.

\begin{corrige}{2smath-0009}

    \begin{enumerate}
        \item
            \( 24\) divisé par combien vaut \( 8\) ? Réponse : \( 3\). Donc \( \dfrac{ 24 }{ 3 }=8\).
        \item
            Il s'agit de la formule de distributivité : \( 5\times (18+27)=5\times 18+5\times 27\).
        \item
            Il s'agit d'utiliser la formule de factorisation :
            \begin{equation}
                12\times 34+12\times 66=12\times (34+66)=12\times 100=1200.
            \end{equation}
        \item
            On peut raisonner de la façon suivante : \( 4\times \ldots+2=30\), donc \( 4\times \ldots=28\). La réponse est \( 28\div 4=7\). Donc
            \begin{equation}
                4\times 7+2=30.
            \end{equation}
    \end{enumerate}

\end{corrige}
