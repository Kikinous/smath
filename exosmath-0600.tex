% This is part of Un soupçon de mathématique sans être agressif pour autant
% Copyright (c) 2014
%   Laurent Claessens
% See the file fdl-1.3.txt for copying conditions.

\begin{exercice}\label{exosmath-0600}

\begin{wrapfigure}{r}{5.0cm}
   \vspace{-0.5cm}        % à adapter.
   \centering
   \input{Fig_RZwXiJD.pstricks}
\end{wrapfigure}

La figure ci-contre est un cube. Les points \( I\) et \( J\) sont les milieux de \( [EF]\) et \( [BF]\).
\begin{enumerate}
    \item
        Pourquoi peut-on dire que \( I\) est dans le plan \( (GHE)\) ? Et pourquoi peut-on dire que \( J\) n'est pas dans le plan \( (GHE)\) ?
    \item
        Est-ce que les plans \( (ABF) \) et \( (HGA)\) sont sécants ? Si oui, déterminer l'intersection \( (ABF)\cap (HGA)\).
    \item
        Si la longueur du côté du cube est de \unit{7}{\kilo\meter}, quelle est la longueur du segment \( [IJ]\) ?
\end{enumerate}

\corrref{smath-0600}
\end{exercice}
