% This is part of Un soupçon de mathématique sans être agressif pour autant
% Copyright (c) 2012
%   Laurent Claessens
% See the file fdl-1.3.txt for copying conditions.

\begin{exercice}\label{exosmath-0018}

    \begin{multicols}{2}
    \begin{enumerate}
        \item
            Calculer
            \begin{equation}
                \frac{ 1 }{2}-\frac{1}{ 4 }.
            \end{equation}
        \item
            Mettre au même dénominateur
            \begin{equation}
                \frac{1}{ x }+\frac{ 5 }{ 4 }.
            \end{equation}
        \item
            Résoudre 
            \begin{equation}
                5x=4
            \end{equation}
            en laissant le résultat sous forme de fractions.
        \item
            Simplifier
            \begin{equation}
                \frac{ 5x }{ x^2 }
            \end{equation}


        \item
            Résoudre
            \begin{equation}
                \frac{ x+2 }{ 3 }=4.
            \end{equation}
        \item
            Calculer
            \begin{equation}
                \frac{ 15 }{ 7 }\times \frac{ 7 }{ 5 }.
            \end{equation}
        \item
            Mettre au même dénominateur
            \begin{equation}
                \frac{1}{ x+2 }+\frac{1}{ x-1 }
            \end{equation}
        \item
            Simplifier 
            \begin{equation}
                \frac{ a+2a }{ a }.
            \end{equation}


        \item
            Résoudre
            \begin{equation}
                10x+5=5.
            \end{equation}
        \item
            Calculer
            \begin{equation}
                \frac{ 3 }{ 4 }+\frac{ 5 }{ 8 }.
            \end{equation}
        \item
            Mettre au même dénominateur
            \begin{equation}
                \frac{ 1 }{ 2 }-\frac{ a }{ x }
            \end{equation}
        \item
            Effectuer
            \begin{equation}
                x^2(x^{-3}+2).
            \end{equation}

        \item
            Calculer
            \begin{equation}
                \frac{ x }{ 2 }\times\frac{ 4 }{ x^2 }.
            \end{equation}
        \item
            Factoriser
            \begin{equation}
                x^2-9
            \end{equation}
        \item
            Résoudre
            \begin{equation}
                \frac{ 3 }{ 4 }x+5=10
            \end{equation}
        \item
            Effectuer
            \begin{equation}
                -2(x+3-x^3).
            \end{equation}

        \item
            Résoudre \( 3x+4=6\)
        \item
            Calculer
            \begin{equation}
                \frac{1}{ 3 }+\frac{1}{ 7 }.
            \end{equation}
        \item
            Quelle est la surface d'un carré de côté de \unit{12}{\centi\meter\squared}.
        \item
            Simplifier
            \begin{equation}
                \frac{ 2x+4 }{ 2 }.
            \end{equation}


        \item
Sur la figure \ref{LabelFigBpCNVm}, quelles sont (en fonction de \( x\)) la longueur, la largeur et la surface du rectangle ?

        \item
            Factoriser
            \begin{equation}
                2ax+2x   
            \end{equation}
        \item
            Résoudre
            \begin{equation}
                \frac{ 100 }{ x }=2
            \end{equation}
        \item
            Calculer
            \begin{equation}
                \frac{ 7 }{ 8 }\times\frac{ 7 }{ 3 }.
            \end{equation}
        \item
            Mettre au même dénominateur
            \begin{equation}
                \frac{1}{ x+1 }+\frac{1}{ x-1 }.
            \end{equation}

        \item
            Résoudre
            \begin{equation}
                \frac{ 31 }{ 27 }x=1.
            \end{equation}
        \item
            Quelles sont les coordonnées du milieu du segment délimité par les points \( A=(1;2)\) et \( B=(-3;8)\) ?
        \item
            Simplifier
            \begin{equation}
                \frac{ x^2-9 }{ x+3 }.
            \end{equation}
        \item
            Résoudre
            \begin{equation}
                3x+1=1.
            \end{equation}

        \item
            Mettre \( 2\) en évidence dans l'expression \( 2x+4\).
        \item
            Simplifier
            \begin{equation}
                \frac{ 3x+6 }{ 3 }.
            \end{equation}
        \item
            Résoudre
            \begin{equation}
                \frac{ 6x }{ 7 }=3.
            \end{equation}
        \item
            Développer \( x^2(1+x)\).

            \item
                Simplifier
                \begin{equation*}
                    \frac{ 6+12a }{ 3 }.
                \end{equation*}
            \item
                Résoudre
                \begin{equation*}
                    \frac{ 2x+1 }{ 3 }=7
                \end{equation*}

            \item
                Mettre \( x\) en évidence dans \( ax^2-x\).
            \item
                Mettre au même dénominateur et effectuer la somme
                \begin{equation*}
                    \frac{ a }{ 4 }+\frac{ 2 }{ x }.
                \end{equation*}

            \item
                 Développer \( (1+x)^2\)
             \item
                 Développer \( (1-a)^2\)
             \item
                 Factoriser \( 9-a^2\)
             \item
                 Factoriser \( (x-1)^2-49\)
             \item
                 Factoriser \( a^2+8a+16\)
             \item
                 Résoudre \( x^2+6x+9=0\)

         \end{enumerate}
             \end{multicols}
\corrref{smath-0018}
\end{exercice}

\newcommand{\CaptionFigBpCNVm}{figure de l'exercice \ref{exosmath-0018}} 
\input{Fig_BpCNVm.pstricks}
