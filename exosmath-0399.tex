% This is part of Un soupçon de mathématique sans être agressif pour autant
% Copyright (c) 2013
%   Laurent Claessens
% See the file fdl-1.3.txt for copying conditions.

\begin{exercice}[\cite{SMPaVRN}]\label{exosmath-0399}

Monsieur R. possède d'une ébénisterie d'armoires de luxe. Plus la demande est grande, plus il doit payer des heures supplémentaires à ses employés (ou pire : il doit engager des travailleurs temporaires) et plus le prix unitaire de ses armoires est élevé. Monsieur R. détermine que le prix unitaire de production de ses armoires est donné par la fonction
\begin{equation*}
    p(x)=x+72
\end{equation*}
où \( x\) est la quantité d'armoires demandée en une semaine. Autrement dit, si il a un seul client, fabriquer l'armoire de ce client coûtera \( 73\) euros à monsieur R. alors que si il a \( 10\) clients, chacune des \( 10\) armoires lui coûtera \( 82\) euros.

De plus chaque semaine monsieur R. a des frais fixes de \( 3952\) euros. Chaque armoire est vendue \( 200\) euros.

\begin{enumerate}
    \item
        Montrer que le bénéfice de la semaine est de
        \begin{equation*}
            B(x)=-x^2+128x-3952
        \end{equation*}
        si il a \( x\) commandes.
    \item
        Dresser une tableau de variation de \( B\).
    \item
        Quel bénéfice maximum peut espérer monsieur R. ? Combien d'armoires doit-il vendre pour l'attendre ?
    \item
        Montrer que \( B(x)=(x-52)(76-x)\).
    \item
        Combien d'armoires doit-il construire pour être rentable ?
\end{enumerate}

\corrref{smath-0399}
\end{exercice}
