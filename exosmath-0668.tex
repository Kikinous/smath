% This is part of Un soupçon de mathématique sans être agressif pour autant
% Copyright (c) 2014
%   Laurent Claessens
% See the file fdl-1.3.txt for copying conditions.

\begin{exercice}[\ldots/4]\label{exosmath-0668}

    Les notes à un devoir d'une classe de \( 33\) élèves sont :
    \begin{equation*}
        \begin{array}[]{|c|c|c|c|c|c|c|c|c|c|c|}
            \hline
            3&4.5    &5&5&5&5.5&6&7&7.5   &8&8.5\\
            \hline
            8.5&9&9&9&9.5&9.5&9.&9.75    &10&10&10\\
            \hline
            10.5&10.5&10.75&11&11&11.5&11.5&11.75&11.75&14&15.5\\
            \hline
        \end{array}
    \end{equation*}
    \begin{enumerate}
        \item
            Nous souhaitons regrouper ces note dans les classes \( \mathopen[ 0 ;5 [\), \( \mathopen[ 5 ;8 [\), \( \mathopen[ 8 ; 10 [\), \( \mathopen[ 10; 12 [\), \( \mathopen[ 12 ; 15 [\) et \( \mathopen[ 15 , 20 \mathclose]\). Quels sont les effectifs de ces classes ?
        \item
            Tracer un histogramme de ces données avec ces classes. Prendre pour échelle \( 1\) carré \( =\) un élève.
    \end{enumerate}

\corrref{smath-0668}
\end{exercice}
