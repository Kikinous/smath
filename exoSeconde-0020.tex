% This is part of Un soupçon de mathématique sans être agressif pour autant
% Copyright (c) 2012
%   Laurent Claessens
% See the file fdl-1.3.txt for copying conditions.

\begin{exercice}\label{exoSeconde-0020}

    Soient\cite{BAIljy} les points \( A,B,C\) dont les coordonnées dans un repère orthonormé sont \( A=(-1,1)\), \( B=(1,2)\) et \( C=(3,-2)\).
    \begin{enumerate}
        \item
            Calculer les longueurs des côtés du triangle \( ABC\).
        \item
            Calculer l'aire du triangle.
        \item
            Donner le rayon et le centre du cercle circonscrit au triangle \( ABC\).
    \end{enumerate}

\corrref{Seconde-0020}
\end{exercice}
