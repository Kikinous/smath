% This is part of Un soupçon de mathématique sans être agressif pour autant
% Copyright (c) 2012-2014
%   Laurent Claessens
% See the file fdl-1.3.txt for copying conditions.

%+++++++++++++++++++++++++++++++++++++++++++++++++++++++++++++++++++++++++++++++++++++++++++++++++++++++++++++++++++++++++++ 
\section{Statistique descriptive}
%+++++++++++++++++++++++++++++++++++++++++++++++++++++++++++++++++++++++++++++++++++++++++++++++++++++++++++++++++++++++++++


\Exo{Seconde-0033}  % mis en exemple dans le cours.
\Exo{Seconde-0015}
\Exo{Seconde-0074}

\begin{multicols}{2}
\Exo{smath-0012}
\Exo{smath-0011}



% Moyenne

% Médiane, quartiles
\Exo{Seconde-0029}
\Exo{Seconde-0027}
\Exo{smath-0147}
\Exo{Seconde-0032}
\Exo{Seconde-0016}
\Exo{Seconde-0017}

% Histogrammes
\Exo{Seconde-0036}
\Exo{Seconde-0038}

\Exo{Seconde-0040}


% Problèmes en plusieurs coups
\Exo{Seconde-0039}
\Exo{smath-0250}
\Exo{smath-0315}
\end{multicols}
\Exo{smath-0531}

%+++++++++++++++++++++++++++++++++++++++++++++++++++++++++++++++++++++++++++++++++++++++++++++++++++++++++++++++++++++++++++ 
\section{Repères}
%+++++++++++++++++++++++++++++++++++++++++++++++++++++++++++++++++++++++++++++++++++++++++++++++++++++++++++++++++++++++++++

%\Exo{Seconde-0002}         % Celui-ci est enlevé parce qu'il est inséré au cours.
\Exo{smath-0479}
\Exo{smath-0021}
\Exo{smath-0028}
\Exo{Seconde-0100}
\Exo{Seconde-0085}
\Exo{Seconde-0083}
\Exo{Seconde-0003}
\Exo{Seconde-0084}
\Exo{Seconde-0099}
\Exo{Seconde-0013}
\Exo{Seconde-0021}
\Exo{Seconde-0079}
\Exo{Seconde-0080}
\Exo{Seconde-0078}
\Exo{smath-0029}
\Exo{Seconde-0081}
\Exo{Seconde-0077}
\Exo{smath-0027}
\Exo{Seconde-0082}
\Exo{Seconde-0001}
\Exo{Seconde-0062}
\Exo{smath-0026}

%+++++++++++++++++++++++++++++++++++++++++++++++++++++++++++++++++++++++++++++++++++++++++++++++++++++++++++++++++++++++++++ 
\section{Fonctions affines}
%+++++++++++++++++++++++++++++++++++++++++++++++++++++++++++++++++++++++++++++++++++++++++++++++++++++++++++++++++++++++++++


%TODO : écrire un exercice avec un rebond sur une table de billard.
%+++++++++++++++++++++++++++++++++++++++++++++++++++++++++++++++++++++++++++++++++++++++++++++++++++++++++++++++++++++++++++ 
\section{Fonctions et résolutions}
%+++++++++++++++++++++++++++++++++++++++++++++++++++++++++++++++++++++++++++++++++++++++++++++++++++++++++++++++++++++++++++

%---------------------------------------------------------------------------------------------------------------------------
\subsection{Image, antécédent}
%---------------------------------------------------------------------------------------------------------------------------

\Exo{Seconde-0042}
\Exo{Seconde-0048}
\Exo{Seconde-0054}
\Exo{Seconde-0051}
\Exo{Seconde-0050}

%---------------------------------------------------------------------------------------------------------------------------
\subsection{Exemples de fonctions}
%---------------------------------------------------------------------------------------------------------------------------

\Exo{Seconde-0058}
\Exo{Seconde-0057}
\Exo{Seconde-0063}
\Exo{Seconde-0061}
\Exo{Seconde-0067}
\Exo{smath-0295}

\Exo{Seconde-0071}
\Exo{smath-0015}
\Exo{smath-0210}
\Exo{smath-0004}
\Exo{smath-0005}
\Exo{smath-0006}
\Exo{Seconde-0046}

\Exo{smath-0186}    % Cet exercice est le même que le smath-0111, mais vu différemment.
\Exo{smath-0078}
\Exo{smath-0013}
\Exo{Seconde-0068}
\Exo{Seconde-0064}
\Exo{Seconde-0044}
\Exo{Seconde-0066}
\Exo{Seconde-0076}
\Exo{Seconde-0059}
\Exo{Seconde-0060}
%---------------------------------------------------------------------------------------------------------------------------
\subsection{Plus avancés}
%---------------------------------------------------------------------------------------------------------------------------

\Exo{Premiere-0018}

%+++++++++++++++++++++++++++++++++++++++++++++++++++++++++++++++++++++++++++++++++++++++++++++++++++++++++++++++++++++++++++ 
\section{Comparaison de séries statistiques}
%+++++++++++++++++++++++++++++++++++++++++++++++++++++++++++++++++++++++++++++++++++++++++++++++++++++++++++++++++++++++++++

\Exo{smath-0378}
\Exo{smath-0243}
\Exo{smath-0244}
\Exo{smath-0245}
\Exo{smath-0246}
\Exo{smath-0247}
\Exo{smath-0248}
\Exo{smath-0249}
\Exo{smath-0216}
\Exo{smath-0392}

%+++++++++++++++++++++++++++++++++++++++++++++++++++++++++++++++++++++++++++++++++++++++++++++++++++++++++++++++++++++++++++ 
\section{Probabilités}
%+++++++++++++++++++++++++++++++++++++++++++++++++++++++++++++++++++++++++++++++++++++++++++++++++++++++++++++++++++++++++++

%TODO : il faut des exercices sur les complémentaires.

\Exo{smath-0195}
\Exo{smath-0189}
\Exo{smath-0215}

\Exo{smath-0355}   % Plus dur
\Exo{smath-0190}
\Exo{smath-0197}
\Exo{smath-0191}
\Exo{smath-0194}

%+++++++++++++++++++++++++++++++++++++++++++++++++++++++++++++++++++++++++++++++++++++++++++++++++++++++++++++++++++++++++++ 
\section{Intervalles de confiance et de fluctuation}
%+++++++++++++++++++++++++++++++++++++++++++++++++++++++++++++++++++++++++++++++++++++++++++++++++++++++++++++++++++++++++++

\Exo{smath-0380}
\Exo{smath-0386}
\Exo{smath-0333}

%+++++++++++++++++++++++++++++++++++++++++++++++++++++++++++++++++++++++++++++++++++++++++++++++++++++++++++++++++++++++++++
\section{Exercices sur la géométrie dans l'espace}
%+++++++++++++++++++++++++++++++++++++++++++++++++++++++++++++++++++++++++++++++++++++++++++++++++++++++++++++++++++++++++++

\Exo{Seconde-0091}
\Exo{Seconde-0088}
\Exo{Seconde-0089}
\Exo{Seconde-0095}
\Exo{Seconde-0094}
\Exo{smath-0095}
\Exo{Seconde-0092}
\Exo{smath-0093}
\Exo{Seconde-0090}

%--------------------------------------------------------------------------------------------------------------------------- 
\subsection{Questions de cours}
%---------------------------------------------------------------------------------------------------------------------------

\Exo{smath-0184}


%+++++++++++++++++++++++++++++++++++++++++++++++++++++++++++++++++++++++++++++++++++++++++++++++++++++++++++++++++++++++++++ 
\section{Trigonométrie}
%+++++++++++++++++++++++++++++++++++++++++++++++++++++++++++++++++++++++++++++++++++++++++++++++++++++++++++++++++++++++++++

\Exo{smath-0468}  % Celui-ci est l'automatique et très long
\Exo{smath-0464}

%+++++++++++++++++++++++++++++++++++++++++++++++++++++++++++++++++++++++++++++++++++++++++++++++++++++++++++++++++++++++++++ 
\section{Vecteurs}
%+++++++++++++++++++++++++++++++++++++++++++++++++++++++++++++++++++++++++++++++++++++++++++++++++++++++++++++++++++++++++++

\Exo{smath-0465}
\Exo{smath-0053}
\Exo{smath-0196}
\Exo{smath-0065}
\Exo{smath-0297}    % Ceci est le même que le smath0221, mais en plus court pour une interrogation.
\Exo{smath-0068}
\Exo{Seconde-0093}
\Exo{smath-0072}
\Exo{smath-0052}
\Exo{smath-0075}
\Exo{smath-0432}    % devoir commun Jules Haag.
\Exo{smath-0112}
\Exo{smath-0107}
\Exo{smath-0199}
\Exo{smath-0074}
\Exo{smath-0071}
\Exo{smath-0055}
\Exo{smath-0105}
\Exo{Seconde-0098}
\Exo{smath-0221}
\Exo{smath-0059}
\Exo{smath-0104}
\Exo{smath-0066}
\Exo{smath-0070}
\Exo{smath-0213}
\Exo{smath-0076}

%--------------------------------------------------------------------------------------------------------------------------- 
\subsection{Colinéarité}
%---------------------------------------------------------------------------------------------------------------------------

\Exo{smath-0299}
\Exo{smath-0108}
\Exo{smath-0061}    % Ici il y a un lien avec les droites.
\Exo{smath-0109}    % Ici il y a un lien avec les droites.
\Exo{smath-0060}
\Exo{smath-0054}
\Exo{smath-0331}
\Exo{smath-0466}


%+++++++++++++++++++++++++++++++++++++++++++++++++++++++++++++++++++++++++++++++++++++++++++++++++++++++++++++++++++++++++++ 
\section{Second degré}
%+++++++++++++++++++++++++++++++++++++++++++++++++++++++++++++++++++++++++++++++++++++++++++++++++++++++++++++++++++++++++++

\Exo{smath-0632}    % Haag (mais très adapté à ne pas utiliser de connaissances sur le second degré)
\Exo{smath-0238}
\Exo{smath-0201}
\Exo{smath-0220}
\Exo{smath-0251}
\Exo{smath-0146}
\Exo{smath-0143}
\Exo{smath-0087}
\Exo{smath-0088}
\Exo{smath-0131}
\Exo{smath-0133}
\Exo{smath-0267}
\Exo{smath-0269}
\Exo{smath-0270}
\Exo{smath-0271}
\Exo{smath-0050}
\Exo{smath-0051}
\Exo{smath-0391}

\Exo{smath-0652}
\Exo{smath-0254}
\Exo{smath-0174}
\Exo{smath-0179}
\Exo{smath-0178}
\Exo{smath-0266}


%+++++++++++++++++++++++++++++++++++++++++++++++++++++++++++++++++++++++++++++++++++++++++++++++++++++++++++++++++++++++++++ 
\section{Systèmes}
%+++++++++++++++++++++++++++++++++++++++++++++++++++++++++++++++++++++++++++++++++++++++++++++++++++++++++++++++++++++++++++

\Exo{smath-0323}
\Exo{smath-0327}

\Exo{smath-0226}
\Exo{smath-0227}
\Exo{smath-0228}
\Exo{smath-0240}
\Exo{smath-0156}
\Exo{smath-0157}
\Exo{smath-0185}
\Exo{smath-0000}

\Exo{smath-0339}
\Exo{smath-0411}    % Celui-ci est le même que le 0339, mais en plus détaillé pour un DM, et sans la grande photo.

%---------------------------------------------------------------------------------------------------------------------------
\subsection{Factorisation}
%---------------------------------------------------------------------------------------------------------------------------

\Exo{smath-0047}
\Exo{smath-0048}
\Exo{Premiere-0020}
\Exo{Premiere-0021}
\Exo{smath-0145}
\Exo{smath-0317}
\Exo{smath-0318}

\Exo{smath-0462}


%\Exo{Seconde-0018}
%\Exo{Seconde-0041}
%\Exo{Premiere-0064}
%\Exo{Seconde-0065}
%\Exo{smath-0014}
%\Exo{smath-0017}

%TODO : vider ces fichiers et les remettre en disponibilité parce qu'ils sont tous dans exosmath-0018.tex
% (18 octobre 2012)

%+++++++++++++++++++++++++++++++++++++++++++++++++++++++++++++++++++++++++++++++++++++++++++++++++++++++++++++++++++++++++++ 
\section{Fonction inverse}
%+++++++++++++++++++++++++++++++++++++++++++++++++++++++++++++++++++++++++++++++++++++++++++++++++++++++++++++++++++++++++++

% Il faut un peu reclasser ces exercices.

\Exo{smath-0262} 
\Exo{smath-0257}
\Exo{smath-0258}

\Exo{smath-0369}
\Exo{smath-0350}
\Exo{smath-0276}
\Exo{smath-0260}
\Exo{smath-0259}
\Exo{smath-0261}
\Exo{smath-0265}
\Exo{smath-0268}
\Exo{smath-0272}
\Exo{smath-0273}
\Exo{smath-0274} 
\Exo{smath-0275}
\Exo{smath-0277}
\Exo{smath-0278}
\Exo{smath-0292}
\Exo{smath-0294}
\Exo{smath-0346}
\Exo{smath-0364}
\Exo{smath-0256}        % Cet exercice n'apporte rien.
\Exo{smath-0406}


%+++++++++++++++++++++++++++++++++++++++++++++++++++++++++++++++++++++++++++++++++++++++++++++++++++++++++++++++++++++++++++ 
\section{Techniques de calcul}
%+++++++++++++++++++++++++++++++++++++++++++++++++++++++++++++++++++++++++++++++++++++++++++++++++++++++++++++++++++++++++++

%---------------------------------------------------------------------------------------------------------------------------
\subsection{Choses basiques}
%---------------------------------------------------------------------------------------------------------------------------

\Exo{smath-0412}
\Exo{smath-0018}
\Exo{Premiere-0019}
\Exo{Premiere-0022}
\Exo{smath-0205}
\Exo{smath-0206}
\Exo{smath-0207}
\Exo{smath-0208}

%--------------------------------------------------------------------------------------------------------------------------- 
\subsection{Calcul algébrique}
%---------------------------------------------------------------------------------------------------------------------------

\Exo{smath-0204}
\Exo{smath-0222}
\Exo{smath-0223}
\Exo{smath-0224}
\Exo{smath-0225}


%+++++++++++++++++++++++++++++++++++++++++++++++++++++++++++++++++++++++++++++++++++++++++++++++++++++++++++++++++++++++++++ 
\section{Équations de droites}
%+++++++++++++++++++++++++++++++++++++++++++++++++++++++++++++++++++++++++++++++++++++++++++++++++++++++++++++++++++++++++++

\Exo{smath-0154}
\Exo{smath-0330}        % Cet exo est le même que le smath-155, mais en plus court pour rentrer dans un devoir.
\Exo{smath-0344}
\Exo{smath-0128}
\Exo{smath-0409}    % presque le même que le smath-0149, mais plus court pour un DS.
\Exo{smath-0069}
\Exo{smath-0394}
\Exo{smath-0229}    % Le même que smath-0413, mais en plus guidé pour un DS.
\Exo{smath-0150}    
\Exo{smath-0155}
\Exo{smath-0151}
\Exo{smath-0086}
\Exo{smath-0138}
\Exo{smath-0233}
\Exo{smath-0153}
\Exo{Seconde-0019}
\Exo{smath-0573}


\Exo{smath-0083}
\Exo{smath-0129}

\Exo{smath-0235}
\Exo{smath-0236}
\Exo{smath-0237}
\Exo{smath-0239}
\Exo{smath-0345}

%--------------------------------------------------------------------------------------------------------------------------- 
\subsection{Tableaux de variation et de signe}
%---------------------------------------------------------------------------------------------------------------------------

\Exo{smath-0132}
\Exo{smath-0321}

%--------------------------------------------------------------------------------------------------------------------------- 
\subsection{Problèmes}
%---------------------------------------------------------------------------------------------------------------------------

\Exo{smath-0241}
\Exo{smath-0242}

%+++++++++++++++++++++++++++++++++++++++++++++++++++++++++++++++++++++++++++++++++++++++++++++++++++++++++++++++++++++++++++ 
\section{Révision de seconde}
%+++++++++++++++++++++++++++++++++++++++++++++++++++++++++++++++++++++++++++++++++++++++++++++++++++++++++++++++++++++++++++

\Exo{smath-0448}
\Exo{smath-0449}
\Exo{smath-0452}
\Exo{smath-0453}
\Exo{smath-0440}
\Exo{smath-0472}
\Exo{smath-0473}

%+++++++++++++++++++++++++++++++++++++++++++++++++++++++++++++++++++++++++++++++++++++++++++++++++++++++++++++++++++++++++++ 
\section{Algorithmique}
%+++++++++++++++++++++++++++++++++++++++++++++++++++++++++++++++++++++++++++++++++++++++++++++++++++++++++++++++++++++++++++

\Exo{smath-0696}
