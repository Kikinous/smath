% This is part of Un soupçon de mathématique sans être agressif pour autant
% Copyright (c) 2015
%   Laurent Claessens
% See the file fdl-1.3.txt for copying conditions.

\begin{corrige}{2smath-0239}

    La méthode la plus simple consiste à dire qu'il y a une symétrie axiale le long de l'axe horizontal passant par \( E\). Par conséquent les deux angles demandés sont égaux. Étant donné que la somme des trois angles en \( E\) est égale à \SI{360}{\degree} et que l'un des trois est déjà égal à \SI{70}{\degree}, la somme des deux demandés vaut
    \begin{equation}
        360-70=290.
    \end{equation}
    Chacun des deux angles mesure donc \( 290\div 2=145\). La réponse est donc \SI{145}{\degree}.

    Il existe une méthode plus compliquée qui consiste à commencer par déterminer les angles du triangle \( AED\). Étant donné qu'il est isocèle, les angles en \( A\) et \( D\) sont égaux et valent
    \begin{equation}
        \frac{ 180-70 }{ 2 }=\SI{55}{\degree}.
    \end{equation}
    
    L'angle \( \widehat{EAB}\) mesure alors \( 90-55=35\).

    L'astuce est alors de prolonger le segment \( [DE]\) jusqu'à couper le bord supérieur du drapeau au point \( K\).

    Sur le dessin suivant, l'angle en \( K\) est altrne-intere à la droite \( (DK)\) qui coupe les deux droites parallèles \( (AB)\) et \( (DC)\).
    \begin{center}
        \small
\input{Fig_LLBQooAjaorQooONE.pstricks}
    \end{center}
    L'angle noté \SI{110}{\degree} en \( E\) est obtenu en utilisant la somme des angles internes au triangle \( AKE\).


\end{corrige}
