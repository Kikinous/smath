% This is part of Un soupçon de mathématique sans être agressif pour autant
% Copyright (c) 2014
%   Laurent Claessens
% See the file fdl-1.3.txt for copying conditions.

\begin{exercice}\label{exosmath-0782}

    Exprimer les phrases suivantes sous la forme «si \ldots alors \ldots» 
    \begin{enumerate}
        \item
            Lorsqu'il pleut, il y a des nuages.
        \item
            Un parallélogramme a deux diagonales de même longueurs.
    \end{enumerate}
    En s'appuyant sur les affirmations précédentes, répondre si possible aux questions suivantes.
    \begin{enumerate}
        \item
            Le \( 5\) avril, il y a eu des nuages. Est-ce qu'il a plu ?
        \item
            Les segments \( [AC]\) et \( [BD]\) sont de même longueur. Est-ce que le quadrilatère \( ABCD\) est un parallélogramme ?
    \end{enumerate}

\corrref{smath-0782}
\end{exercice}
