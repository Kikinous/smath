% This is part of Un soupçon de mathématique sans être agressif pour autant
% Copyright (c) 2013
%   Laurent Claessens
% See the file fdl-1.3.txt for copying conditions.

\begin{exercice}\label{exosmath-0297}

    \begin{multicols}{2}
        Sur la figure ci-contre, \( MNBP\) est un rectangle, \( C\) est le milieu de \( [MN]\) et $A$ est le milieu de \( [BP]\). 

    \columnbreak

%The result is on figure \ref{LabelFigfigureFGgTGJA}. % From file figureFGgTGJA
%\newcommand{\CaptionFigfigureFGgTGJA}{<+Type your caption here+>}

    \begin{center}
\input{Fig_figureFGgTGJA.pstricks}
    \end{center}

    \end{multicols}
    Compléter les égalités suivantes de telles façon à ce qu'elles soient correctes.
    \begin{multicols}{3}
        \begin{enumerate}
            \item
                \( \vect{ NB }+\vect{ CA }-\vect{ CP }=\vect{ ?B }\)
            \item
                \( \vect{ MC }+\vect{ AB }=\vect{ P? }\)
            \item
                \( \vect{ AC }+\vect{ AB }=\vect{ ?? }\)
        \end{enumerate}
    \end{multicols}

\corrref{smath-0297}
\end{exercice}
