% This is part of Un soupçon de mathématique sans être agressif pour autant
% Copyright (c) 2014
%   Laurent Claessens
% See the file fdl-1.3.txt for copying conditions.

\begin{corrige}{smath-0633}

    \begin{enumerate}
        \item
            Vu que \( M\) varie entre \( C\) et \( B\), la distance \( CM\) varie entre \( 0\) et \( 4\). L'intervalle de variation de \( x\) est \( \mathopen[ 0 ; 4 \mathclose]\).
        \item
            Les mesures utiles et non données directement dans l'énoncé sont \( NB=6-x\) et \( BM=4-x\).
        \item
            Le triangle \( DAN\) est rectangle en \( A\); son aire est donnée par
            \begin{equation}
                f(x)=\frac{ AN\times DA }{ 2 }=\frac{ 4x }{ 2 }=2x.
            \end{equation}
        \item
            L'aire du rectangle dont les côtés sont \( 6-x\) et \( (4-x)\) est
            \begin{equation}
                g(x)=(6-x)(4-x)=x^2-10x+24.
            \end{equation}
            L'aire du triangle \( DCM\) (rectangle en \( C\)) est
            \begin{equation}
                h(x)=\frac{ DC\times CM}{ 2 }=\frac{ 6\times x }{2}=3x.
            \end{equation}
        \item
            Pour calculer l'aire de la zone grisé, il suffit de calculer celle du grand rectangle et de soustraire les aires que nous venons de calculer :
            \begin{subequations}
                \begin{align}
                A(x)&=6\times 4-f(x)-g(x)-h(x)\\
                &=24-2x-(x^2-10x+24)-3x\\
                &=24-2x-x^2+10x-24-3x\\
                &=-x^2+5x.
                \end{align}
            \end{subequations}
        \item
            Le raisonnement de Bob est de résoudre \( A(x)=0\). Il est vrai que \( x=0\) et \( x=5\) sont des solutions (pour le vérifier, remplacer \( x=5\) et \( x=0\) dans l'expression de \( A(x)\)).

            Lorsque \( x=0\), la zone grise est effectivement complètement plate, par contre la solution \( x=5\) n'est pas correcte parce que \( x\) n'a le droite de varier que dans l'intervalle \( \mathopen[ 0; 4 \mathclose]\). Avec \( x=5\) le point \( M\) est au-dehors de \( [BC]\).

    \end{enumerate}

\end{corrige}
