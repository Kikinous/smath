% This is part of Un soupçon de mathématique sans être agressif pour autant
% Copyright (c) 2014
%   Laurent Claessens
% See the file fdl-1.3.txt for copying conditions.

\begin{corrige}{smath-0601}

    \begin{enumerate}
        \item
    Règle du produit nul : soit \( x+3=0\), soit \( x-15=0\). Les deux solutions sont donc \( x=-3\) et \( x=15\).
\item
    Il suffit de remplacer \( x\) par \( 7\) :
    \begin{equation}
        f(7)=-2\times 7+12=-14+12=-2.
    \end{equation}
\item
    Le seul facteur commun à \( 6a\), \( a^2\) et \( 3a\) est \( a\) et nous pouvons simplifier par \( a\). Par exemple en mettant en facteur :
    \begin{equation}
        \frac{ 6a+a^2 }{ 3a }=\frac{ a(6+a) }{ 3a }=\frac{ 6+a }{ 3 }.
    \end{equation}
\item
    Il suffit d'appliquer Pythagore :
    \begin{equation}
        AB^2=6^2+9^2=36+81=117.
    \end{equation}
    Donc \( AB=\sqrt{117}=3\sqrt{13}\).
\item
    L'aire étant le produit des deux côtés, si nous appelons \( x\) le côté cherché,
    \begin{equation}
        7x=2
    \end{equation}
    donc \( x=\frac{ 2 }{ 7 }\) est la longueur cherchée.
            
    \end{enumerate}

\end{corrige}
