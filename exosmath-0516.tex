% This is part of Un soupçon de mathématique sans être agressif pour autant
% Copyright (c) 2013
%   Laurent Claessens
% See the file fdl-1.3.txt for copying conditions.

\begin{exercice}\label{exosmath-0516}

\begin{wrapfigure}{r}{10cm}
   \vspace{-1cm}        % à adapter.
   \centering
   \input{Fig_PXsdjSu.pstricks}
\end{wrapfigure}

    Répondre aux questions à partir du graphe ci-contre.
    \begin{enumerate}
        \item
            Sur quel intervalle la fonction \( f\) est-elle décroissante ?
        \item
            Donner les antécédents de \( 2\)
        \item
            Que vaut \( f(0)\) ?
        \item
            Résoudre l'équation \( f(x)=1\).
        \item
            Quel est le maximum de \( f\) sur l'intervalle \( \mathopen[ -1 ; 2 \mathclose]\) ?
    \end{enumerate}

\corrref{smath-0516}
\end{exercice}
