% This is part of Un soupçon de mathématique sans être agressif pour autant
% Copyright (c) 2012
%   Laurent Claessens
% See the file fdl-1.3.txt for copying conditions.

\begin{corrige}{Seconde-0045}

    %TODO : faire un dessin de la situation.
    \begin{enumerate}
        \item
            Le point \( J\) est l'intersection de deux \defe{médianes}{médiane} du triangle, et donc l'intersection des trois. Ce point d'intersection est le \defe{centre de gravité}{centre de gravité (triangle)}\cite{VhTiRd} du triangle.
        \item
            Les trois médianes d'un triangle sont concourantes. La droite \( AJ\) est donc la troisième médiane du triangle \( ADC\) parce que cette troisième médiane est une droite passant par \( A\) et par \( J\). Donc \( AJ\) coupe le côté opposé en son milieu.
    \end{enumerate}

\end{corrige}
