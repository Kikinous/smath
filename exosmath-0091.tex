% This is part of Un soupçon de mathématique sans être agressif pour autant
% Copyright (c) 2012
%   Laurent Claessens
% See the file fdl-1.3.txt for copying conditions.

\begin{exercice}\label{exosmath-0091}

    \begin{enumerate}
        \item
            Rappeler la formule qui donne l'abscisse du sommet de la courbe représentative du polynôme du second degré \( f(x)=ax^2+bx+c\).
        \item
            Donner les coordonnées le sommet de la parabole \( f(x)=x^2-3x+5\). (Donner seulement l'abscisse du sommet n'est pas suffisant, mais rapporte donne déjà es points)
    \end{enumerate}

\corrref{smath-0091}
\end{exercice}
