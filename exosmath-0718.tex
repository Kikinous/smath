% This is part of Un soupçon de mathématique sans être agressif pour autant
% Copyright (c) 2014
%   Laurent Claessens
% See the file fdl-1.3.txt for copying conditions.

\begin{exercice}\label{exosmath-0718}

    Questions de type «calcul mental» 
    \begin{multicols}{2}
    \begin{enumerate}
        \item
            Quel est l'ensemble de définition de la fonction définie par \( f(x)=\frac{ x-7 }{ -3x+10 }\) ?
        \item
            Donner l'équation d'une droite passant par le point \( A(2;15)\).
        \item
            Soit la suite de nombres \( 1\), \( 7\), \( 8\), \( 8\), \( 10\), \( 15\), \( 15\), \( 20\). Est-il possible d'augmenter la moyenne en enlevant un des nombres ?
        \item
            Résoudre l'équation \( -7x-4=x+5\).
        \item
            Donner les coordonnées du vecteur \( \vect{ v }+\vect{ w }\) si \( \vect{ v }=\begin{pmatrix}
                3    \\ 
                3    
            \end{pmatrix}\) et \( \vect{ w }=\begin{pmatrix}
                -3    \\ 
                10    
            \end{pmatrix}\).
    \end{enumerate}
    \end{multicols}


\corrref{smath-0718}
\end{exercice}
