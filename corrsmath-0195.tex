% This is part of Un soupçon de mathématique sans être agressif pour autant
% Copyright (c) 2013
%   Laurent Claessens
% See the file fdl-1.3.txt for copying conditions.

\begin{corrige}{smath-0195}

Nous dessinons le tableau des possibilités de tirage :
\begin{equation*}
    \begin{array}[]{|c||c|c|c|}
        \hline
        &0&1&3\\
        \hline\hline
        0&0&1&3\\
        \hline
        2&2&3&5\\
        \hline
        4&4&5&7\\
        \hline
    \end{array}
\end{equation*}
Les nombres \( 0\), \( 1\), \( 2\), \( 3\), \( 4\) et \( 7\) arrivent une fois et ont donc une probabilité \( 1/9\) d'arriver. Par contre le nombre \( 5\) arrive deux fois et a une probabilité \( 2/9\) d'arriver et le nombre \( 6\) n'arrive pas, donc il est impossible à obtenir. En fin de compte, le tableau est :

\begin{center}
    \begin{equation*}
        \begin{array}[]{|c||c|c|c|c|c|c|c|c|c|}
            \hline
            x_i&0&1&2&3&4&5&6&7&\text{total}\\
              \hline\hline
              p_i&1/7&1/7&1/7&1/7&1/7&2/7&0&1/7&1\\
              \hline 
               \end{array}
           \end{equation*}
\end{center}


\end{corrige}
