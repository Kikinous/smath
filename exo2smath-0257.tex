% This is part of Un soupçon de mathématique sans être agressif pour autant
% Copyright (c) 2015
%   Laurent Claessens
% See the file fdl-1.3.txt for copying conditions.

\begin{exercice}\label{exo2smath-0257}

    Bob doit envoyer son code de carte bancaire à Alice (un nombre entre \( 0\) et \( 9999\)). Pour conserver le secret, Alice lui demande de partir de son code et d'effectuer les calculs suivants :
    \begin{itemize}
        \item Diviser par \( 12345\)
        \item prendre le cosinus du résultat
        \item multiplier le résultat par le code secret
        \item additionner le double du code secret.
    \end{itemize}
    Bob fait le calcul et envoie à Alice le nombre \( 5083.906586\). Comment va faire Alice pour trouver le code secret de Bob ?

\corrref{2smath-0257}
\end{exercice}
