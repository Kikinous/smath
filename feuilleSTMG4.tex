% This is part of Un soupçon de mathématique sans être agressif pour autant
% Copyright (c) 2012
%   Laurent Claessens
% See the file fdl-1.3.txt for copying conditions.

\documentclass[a4paper,12pt]{article}


\usepackage{ifthen}
\usepackage{calc}
\usepackage{pdfsync}

\usepackage{latexsym}
\usepackage{amsfonts}
\usepackage{amsmath}
\usepackage{amsthm}
\usepackage{amssymb}
\usepackage{bbm}
\usepackage{mathrsfs}           
\usepackage{mathabx}           % Pour \divides


\let\second\undefined      % le paquet amthabx définit \second
\let\degree\undefined       % le paquet amthabx définit \degree
\usepackage[cdot,thinqspace,amssymb]{SIunits} 

\usepackage{tabularx}

\usepackage[fr]{exocorr}
\usepackage{hyperref}                           %Doit être appelé en dernier.

\usepackage[utf8]{inputenc}
\usepackage[T1]{fontenc}
\usepackage{textcomp}
\usepackage{lmodern}
\usepackage[a4paper,margin=2cm]{geometry} 
\usepackage[english,frenchb]{babel}



\begin{document}

\corrPosition{0}

\thispagestyle{empty}

\large
\begin{center}
    Feuille d'exercices numéro 4
\end{center}


\tiny
\begin{center}
    Toutes les réponses doivent être justifiées soit par un calcul soit par un raisonnement clairement rédigé.
\end{center}
\normalsize

\Exo{Premiere-0015}                                               
\Exo{Premiere-0016}                                                                       
\Exo{Premiere-0017}                                                                                                                      

\end{document}

