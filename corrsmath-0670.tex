% This is part of Un soupçon de mathématique sans être agressif pour autant
% Copyright (c) 2014
%   Laurent Claessens
% See the file fdl-1.3.txt for copying conditions.

\begin{corrige}{smath-0670}

    \begin{enumerate}
        \item
            Pour vérifier si le quadrilatère \( ABCD\) est un parallélogramme, nous vérifions si \( \vect{ AB }=\vect{ DC }\). Calcul :
            \begin{equation}
                \vect{ AB }=\begin{pmatrix}
                    2-1    \\ 
                    -1-(-5)    
                \end{pmatrix}=\begin{pmatrix}
                    1    \\ 
                    4    
                \end{pmatrix}.
            \end{equation}
            et
            \begin{equation}
                \vect{ DC }=\begin{pmatrix}
                    1    \\ 
                    4    
                \end{pmatrix}.
            \end{equation}
            Nous avons donc un parallélogramme.
        \item
            Pour vérifier si nous avons un rectangle, nous vérifions si par exemple le triangle \( ABC\) serait rectangle en \( B\). Pour cela nous calculons les longueurs des trois côtés :
            \begin{subequations}
                \begin{align}
                    AB^2&=1^2+4^2=17\\
                    BC^2&=0^2+20^2=400\\
                    AC^2&=1^2+24^2=576.
                \end{align}
            \end{subequations}
            Vu que \( AC^2\neq AB^2+BC^2\), le triangle \( ABC\) n'est pas rectangle et le parallélogramme \( ABCD\) n'est pas un rectangle.

        \item
            Nous commençons par calculer les coordonnées du vecteur \( \vect{ AB }+\vect{ CB }\) :
            \begin{equation}
                \vect{ AB }+\vect{ CB }=\begin{pmatrix}
                    1    \\ 
                    4    
                \end{pmatrix}+\begin{pmatrix}
                    0    \\ 
                    -20
                \end{pmatrix}=\begin{pmatrix}
                    1    \\ 
                    -16    
                \end{pmatrix}.
            \end{equation}
            Nous devons donc placer le point \( E\) de telle sorte que \( \vect{ AE }\) soit \( \begin{pmatrix}
                1    \\ 
                -16    
            \end{pmatrix}\). Le point \( E\) est donc le translaté de \( A\) par le vecteur \( \begin{pmatrix}
                1    \\ 
                -16    
            \end{pmatrix}\) et a donc pour coordonnées
            \begin{equation}
                E=(1+1;-5-16)=(2;-21).
            \end{equation}
        \item
            Pour prouver que \( \vect{ BC }=\vect{ EB }\), il suffit de calculer les coordonnées de ces deux vecteurs avec les nombres que l'on a trouvé; nous obtenons
            \begin{equation}
                \vect{ BC }=\vect{ EB }=\begin{pmatrix}
                    0    \\ 
                    20    
                \end{pmatrix}.
            \end{equation}
            
            Une façon alternative de prouver \( \vect{ BC }=\vect{ EB }\) est de se souvenir de la définition de \( E\) : \( \vect{ AE }=\vect{ AB }+\vect{ CB }\). En passant \( \vect{ AB } \) à gauche :
            \begin{subequations}
                \begin{align}
                    \vect{ AE }=\vect{ AB }+\vect{ CB }\\
                    \vect{ AE }-\vect{ AB }=\vect{ CB }\\
                    \vect{ AE }+\vect{ BA }=\vect{ CB }\\
                    \vect{ BE }=\vect{ CB }.
                \end{align}
            \end{subequations}
            Pour la dernière ligne nous avons utilisé les relations de Chasles.
        \item
            Étant donné que \( \vect{ EB }=\vect{ BC }\), le «trajet» de \( E\) à $B$ est le même que le «trajet» de \( B\) à \( C\). Donc \( B\) est le milieu de \( [EC]\).
    \end{enumerate}

\end{corrige}
