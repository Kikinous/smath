% This is part of Un soupçon de mathématique sans être agressif pour autant
% Copyright (c) 2014
%   Laurent Claessens
% See the file fdl-1.3.txt for copying conditions.

% NOTE : cet exercice a changé son nombre de points d'un DS à l'autre.
\begin{exercice}[\ldots/4]\label{exosmath-0597}

    Un bûcheron veut faire entrer un tronc d'arbre de \unit{15}{\meter} de long dans son camion. Ce dernier est un parallélépipède rectangle dont les dimensions de la base sont \unit{13}{\meter}$\times$\unit{5}{\meter}.

    \begin{enumerate}
        \item
            Est-ce qu'il pourra y mettre le tronc d'arbre à plat ?
        \item
            Quitte à ne pas mettre le tronc à plat, est-ce que le bûcheron peut le rentrer dans son camion sachant qu'il fait \unit{6}{\meter} de hauteur ? Sinon, quelle longueur devra-t-il couper ?
    \end{enumerate}

\corrref{smath-0597}
\end{exercice}
