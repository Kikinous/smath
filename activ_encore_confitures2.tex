% This is part of Un soupçon de mathématique sans être agressif pour autant
% Copyright (c) 2015
%   Laurent Claessens
% See the file fdl-1.3.txt for copying conditions.

%--------------------------------------------------------------------------------------------------------------------------- 
\subsection*{Activité : encore des confitures}
%---------------------------------------------------------------------------------------------------------------------------

Voici quelque ingrédients utilisés pour des confitures.
\begin{center}
    \begin{tabular}[]{|c|c|}
        \hline
        Confiture d'abricots& «\SI{500}{\gram} de sucre et \SI{500}{\gram} d'abricots» \\
        \hline
        Confiture de fraises&«\SI{450}{\gram} de sucre et \SI{750}{\gram} de fraises» \\
        \hline
        Confiture de cerises&  «\SI{800}{\gram} de sucre et \SI{2400}{\gram} de cerises» \\ 
        \hline
    \end{tabular}
\end{center}
Est-ce que la quantité de sucre ajoutée est proportionnelle à la quantité de fruits ?

%--------------------------------------------------------------------------------------------------------------------------- 
\subsection*{Activité : prix en solde}
%---------------------------------------------------------------------------------------------------------------------------

Un magasin fait des soldes «\( -20\%\)» et donne le tableau suivant pour aider le clients à savoir les prix soldés en fonction du prix de base :
\begin{equation*}
    \begin{array}[]{|c||c|c|c|c|}
        \hline
        \text{prix}&50&70&130&200\\
        \hline\hline
        \text{prix soldé}&40&56&104&160\\
        \hline
    \end{array}
\end{equation*}
\begin{enumerate}
    \item
        Le prix soldé est-il proportionnel au prix non soldé ?
    \item
        Combien coûterait un article dont le prix non soldé est de \( 150\)€ ?
\end{enumerate}
