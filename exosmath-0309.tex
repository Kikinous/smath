% This is part of Un soupçon de mathématique sans être agressif pour autant
% Copyright (c) 2013
%   Laurent Claessens
% See the file fdl-1.3.txt for copying conditions.

\begin{exercice}[\cite{PRhtJYn}]\label{exosmath-0309}

Calculer :
\begin{enumerate}
    \item
Le terme d'indice $10$ de la suite $(u_n)$ définie par $u_n=10n-5$. 
\item
  Le terme d'indice $4$ de la suite $(u_n)$ définie par $u_n=\sqrt{n^2+9}4$. 
  \item
  Le rang pour lequel la suite $(u_n)$ définie par $u_n=n^2-3$ prend la valeur 22.
\end{enumerate}

\corrref{smath-0309}
\end{exercice}
