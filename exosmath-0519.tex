% This is part of Un soupçon de mathématique sans être agressif pour autant
% Copyright (c) 2013
%   Laurent Claessens
% See the file fdl-1.3.txt for copying conditions.

\begin{exercice}\label{exosmath-0519}

    Pour chacune des questions suivantes, une seule proposition est correcte. Dire laquelle en justifiant brièvement.

    \begin{multicols}{2}
    \begin{enumerate}
        \item
        Pour l'inéquation \( -x+2>3\)
        \begin{enumerate}
            \item
                
        \( x=-1\) est une solution\item\( x=1000\) est une solution\item\( x=-2\) est une solution \\
        \end{enumerate}
    \item 
        La fonction \( x\mapsto -2x-7\)
        \begin{enumerate}
                
            \item est croissante\item est décroissante\item est linéaire\\
        \end{enumerate}
    \item
        L'intervalle \( \mathopen] -12 , 15 \mathclose]\)
        \begin{enumerate}
                
    \item contient \( -12\)\item contient \( 18\)\item contient \( 0\)\\
        \end{enumerate}
            
    \end{enumerate}
    \end{multicols}

\corrref{smath-0519}
\end{exercice}
