% This is part of Un soupçon de mathématique sans être agressif pour autant
% Copyright (c) 2013
%   Laurent Claessens
% See the file fdl-1.3.txt for copying conditions.

\begin{exercice}\label{exosmath-0511}

\begin{wrapfigure}{r}{7.0cm}
   \vspace{-1.5cm}        % à adapter.
   \centering
   \input{Fig_ZQfGNsD.pstricks}
\end{wrapfigure}

À partir du dessin ci-contre,
\begin{enumerate}
    \item
        lire les valeurs \( f(0)\), \( f(1)\), \( f(2)\) et \( f(3)\);
    \item
        lire les valeurs \( g(0)\), \( g(1)\), \( g(2)\) et \( g(3)\);
    \item
        donner le coefficient directeur et l'ordonnée à l'origine des deux droites représentées;
    \item
       sur le même graphique, dessiner le graphe de la fonction affine donnée par la formule \( h(x)=4x+2\).
\end{enumerate}

\corrref{smath-0511}
\end{exercice}

\vspace{2.5cm}  % Cet espace sert pour la correction.

