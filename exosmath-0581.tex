% This is part of Un soupçon de mathématique sans être agressif pour autant
% Copyright (c) 2013
%   Laurent Claessens
% See the file fdl-1.3.txt for copying conditions.

\begin{exercice}\label{exosmath-0581}

    Dans cet exercice nous voulons traiter la question suivante : nous tirons deux nombres au hasard entre \( 1\) et \( 100\). Quelle est la probabilité que la somme soit divisible par \( 4\) ?

    Aide : pour tester si un nombre \( x\) est divisible par \( 4\) il faut faire : \info{SI \( x\%4==0 \) ALORS}.

    \begin{enumerate}
        \item
            Écrire un programme qui choisit deux nombres au hasard entre \( 1\) et \( 100\), calcule la somme et dit si il est divisible par \( 4\).
        \item
            Modifier le programme pour faire \( 1000\) essais et compter combien de fois la somme a été divisible par \( 4\).
        \item
            Aller lire l'aide d'Algobox pour comprendre pourquoi la condition \info{SI \( x\%4==0 \) ALORS} teste bien la divisibilité par \( 4\).
    \end{enumerate}

\corrref{smath-0581}
\end{exercice}
