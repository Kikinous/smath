% This is part of Un soupçon de mathématique sans être agressif pour autant
% Copyright (c) 2012-2015
%   Laurent Claessens
% See the file fdl-1.3.txt for copying conditions.

Début de match sur un terrain faisant \SI{50}{\meter} de large et \SI{100}{\meter} de long. Le joueur \( A\) fait une passe à son attaquant \( B\). Ce dernier est situé \SI{10}{\meter} au dessus de la ligne médiane et à \SI{5}{\meter} du bord du terrain.

\begin{center}
   \input{Fig_EGDFJAT.pstricks}
\end{center}

\begin{enumerate}
    \item
        Compléter le dessin (le rayon du cercle central est de \SI{10}{\meter}).
    \item
        Quelle est la longueur de la passe ?
    \item
        L'attaquant veut directement tirer vers le but. À quelle distance est-il ?
\end{enumerate}

%+++++++++++++++++++++++++++++++++++++++++++++++++++++++++++++++++++++++++++++++++++++++++++++++++++++++++++++++++++++++++++
\section{Repères et coordonnées}
%+++++++++++++++++++++++++++++++++++++++++++++++++++++++++++++++++++++++++++++++++++++++++++++++++++++++++++++++++++++++++++

\begin{definition}
    Un \defe{repère orthonormé}{repère!orthonormé} du plan est la donné de trois points \( O\), \( I\), \( J\) non alignés formant un triangle rectangle isocèle en  \( O\).
\end{definition}

La façon dont nous associons à chaque point \( M\) du plan ses coordonnées dans le repère \( O\), \( I\), \( J\) est donnée à la figure \ref{LabelFigReperexjVyii}.
\newcommand{\CaptionFigReperexjVyii}{Lire les coordonnées du point \( M\) dans le repère \( OIJ\).}
\input{Fig_ReperexjVyii.pstricks}

Nous notons 
\begin{equation}
    M(x_M;y_M)
\end{equation}
pour dire que le point \( M\) a pour coordonnées \( (x_M;y_M)\) dans le repère \( (OIJ)\).

\begin{Aretenir}
    Le nombre \( x_M\) est l'\defe{abscisse}{abscisse} du point \( M\) et le nombre \( y_M\) en l'est l'\defe{ordonnée}{ordonnée}.
\end{Aretenir}


Nous avons
\begin{enumerate}
    \item
       \( I(1;0)\),
   \item
       \( J(0;1)\),
   \item
       \( O(0;0)\).
\end{enumerate}



%\begin{wrapfigure}{r}{20.cm}
%   \vspace{-0.5cm}        % à adapter.
%   \centering
%   \input{Fig_EvZZys.pstricks}
%\end{wrapfigure}

\vspace{1cm}

\begin{center}
    \input{Fig_EvZZys.pstricks}
\end{center}

    \begin{enumerate}
        \item
            Placer les points \( A=(1;3)\), \( B=(-1;0)\), \( C=(1,\frac{ 3 }{2})\).
        \item
            Donner les coordonnées des points \( X\), \( Y\) et \( Z\).
    \end{enumerate}
    

%+++++++++++++++++++++++++++++++++++++++++++++++++++++++++++++++++++++++++++++++++++++++++++++++++++++++++++++++++++++++++++
\section{Distance entre deux points}
%+++++++++++++++++++++++++++++++++++++++++++++++++++++++++++++++++++++++++++++++++++++++++++++++++++++++++++++++++++++++++++

En pratique, il faut ajouter un point pour se faire son petit triangle rectangle. Si \( A(3;6)\) et \( B(-2;10)\), nous ajoutons le point \( X(-2:6)\).

\begin{Aretenir}

\begin{wrapfigure}{r}{7.0cm}
   \vspace{-1.5cm}        % à adapter.
   \centering
   \input{Fig_PythagoreeBqLDU.pstricks}
\end{wrapfigure}

        Soient les points \( A\) et \( B\) de coordonnées \( A=(x_A,y_A)\) et \( B=(x_B,y_B)\) dans un repère orthonormé. La distance entre \( A\) et \( B\) est donnée par la formule
        \begin{equation*}
            AB =\sqrt{(x_B-x_A)^2+(y_B-y_A)^2}.
        \end{equation*}
    \end{Aretenir}

\begin{proof}
    Nous allons utiliser le théorème de Pythagore. Pour cela nous construisons le triangle rectangle construit sur les points \( A\) et \( B\) comme indiqué sur le dessin. Le point \( C\) est le point de même abscisse que \( B\) et de même ordonnée que \( A\), c'est à dire que \( C\) est le point
    \begin{equation}
        C=(x_B,y_A).
    \end{equation}
    La droite \( AC\) est parallèle à l'axe des abscisses et la droite \( BC\) est parallèle à l'axe des ordonnées; elles sont donc perpendiculaires et le triangle \( ABC\) est un triangle rectangle en \( C\). Le théorème de Pythagore s'applique :
    \begin{equation}    \label{EqjLjEKr}
       AB =\sqrt{ AC^2+BC^2}.
    \end{equation}
    Il reste à déterminer les longueurs \(  AC \) et \(  BC \). Le segment \( [AC]\) est horizontal et s'étend de l'abscisse \( x_A\) à l'abscisse \( x_B\), donc il est de longueur soit \( x_A-x_B\) soit \( x_B-x_A\), mais dans les deux cas nous avons
    \begin{equation}
         AC^2=(x_B-x_A)^2.
    \end{equation}
    De la même façon nous avons 
    \begin{equation}
         BC^2=(y_B-y_A)^2.
    \end{equation}
    En remplaçant dans \eqref{EqjLjEKr}, nous obtenons le résultat annoncé.
\end{proof}

%+++++++++++++++++++++++++++++++++++++++++++++++++++++++++++++++++++++++++++++++++++++++++++++++++++++++++++++++++++++++++++
\section{Milieu d'un segment}
%+++++++++++++++++++++++++++++++++++++++++++++++++++++++++++++++++++++++++++++++++++++++++++++++++++++++++++++++++++++++++++


    Une piscine a pour dimension \SI{25}{\meter} fois \SI{50}{\meter} et \( 4\) couloirs numérotés de \( 0\) à \( 3\). Un nageur parcours la piscine dans le sens de la diagonale. Dans quel couloir sera-t-il lorsqu'il aura parcouru la moitié de son parcours ?

    Même question pour une piscine olympique comprenant \( 10\) couloirs numérotés de \( 0\) à \( 9\).

    
    \vspace{1cm}

\begin{Aretenir}
    Soient les points \( A\) et \( B\) de coordonnées \( A=(x_A;y_A)\) et \( B=(x_B;y_B)\) dans un repère. Alors le milieu du segment \( [AB]\) est le point de coordonnées
    \begin{equation*}
            \left( \frac{ x_A+x_B }{ 2 }\,;\,\frac{ y_A+y_B }{2} \right).
    \end{equation*}
\end{Aretenir}


%+++++++++++++++++++++++++++++++++++++++++++++++++++++++++++++++++++++++++++++++++++++++++++++++++++++++++++++++++++++++++++
\section{Compléments}
%+++++++++++++++++++++++++++++++++++++++++++++++++++++++++++++++++++++++++++++++++++++++++++++++++++++++++++++++++++++++++++

\begin{definition}
    Un \defe{repère}{repère} (quelconque) du plan est la donnée de trois points \( (O,I,J)\) non alignés.

    Les deux axes sont les droites \( (OI)\) et \( (OJ)\). Les longueurs \( OI\) et \( OJ\) servent de graduation.
\end{definition}
