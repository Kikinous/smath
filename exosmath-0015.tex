% This is part of Un soupçon de mathématique sans être agressif pour autant
% Copyright (c) 2012-2013
%   Laurent Claessens
% See the file fdl-1.3.txt for copying conditions.

\begin{exercice}\label{exosmath-0015}

    \begin{multicols}{2}

        Répondre aux questions à partir du graphe ci-contre. Les réponses peuvent être approximatives. La fonction \( g\) est celle en trait discontinu.
        \begin{enumerate}
            \item
                Quel est l'ensemble de définition de \( f\) ?
            \item
                Combien vaut \( f(0)\) ?
            \item
                Quels sont les antécédents de \( -1\) par \( f\) ?
            \item
                Résoudre l'inéquation \( f(x)\geq 3\).
            \item
                Donner un antécédent de \( 8\).
            \item
                Résoudre \( f(x)=g(x)\).
            \item
                Résoudre \( f(x)<g(x)\).
            \item
                Combien vaut \( f(3)\) ?
        \end{enumerate}
        
        \columnbreak

%        The result is on figure \ref{LabelFigFnInterrobgepC}. % From file FnInterrobgepC
%\newcommand{\CaptionFigFnInterrobgepC}{<+Type your caption here+>}
        \begin{center}
\input{Fig_FnInterrobgepC.pstricks}
        \end{center}

    \end{multicols}

\corrref{smath-0015}
\end{exercice}
