% This is part of Un soupçon de mathématique sans être agressif pour autant
% Copyright (c) 2013
%   Laurent Claessens
% See the file fdl-1.3.txt for copying conditions.

\begin{exercice}\label{exosmath-0505}

    Dessiner le graphe d'une fonction \( f\) définie sur l'intervalle \( \mathopen[ -5 ;5 \mathclose]\)  satisfaisant aux caractéristiques suivantes :
    \begin{enumerate}
        \item
            \( f\) est croissante sur \( \mathopen[ -5; 0 \mathclose]\),
        \item
            \( f(-1)=-2\),
        \item
            \( f(5)=6\).
        \item
            le maximum de la fonction \( f\) sur l'intervalle $\mathopen[ 2 , 4 \mathclose]$ est atteint pour \( x=3\),
        \item
            \( f\) est décroissante sur \( \mathopen[ 3 , 4 \mathclose]\),
    \end{enumerate}
    

\corrref{smath-0505}
\end{exercice}
