% This is part of Un soupçon de mathématique sans être agressif pour autant
% Copyright (c) 2012
%   Laurent Claessens
% See the file fdl-1.3.txt for copying conditions.

\begin{exercice}\label{exosmath-0025}

    \begin{enumerate}
        \item
            Déterminer pour quelle(s) valeur(s) de \( m\) les droites d'équations \( d_1\equiv 3x-my+7=0\) et \( (m+3)x-6y-7=0\) sont parallèles.
        \item
            Pour chacune des valeurs de \( m\) trouvées, dire si les droits sont confondues.
    \end{enumerate}

\corrref{smath-0025}
\end{exercice}
