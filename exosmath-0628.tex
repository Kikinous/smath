% This is part of Un soupçon de mathématique sans être agressif pour autant
% Copyright (c) 2014
%   Laurent Claessens
% See the file fdl-1.3.txt for copying conditions.

\begin{exercice}\label{exosmath-0628}

\begin{wrapfigure}{r}{10.cm}
   \vspace{-0.5cm}        % à adapter.
   \centering
   \input{Fig_LectureGraphnrkEEM.pstricks}
\end{wrapfigure}

    À partir du graphe ci-contre de la fonction \( f\), répondre aux questions suivantes.
    \begin{enumerate}
        \item
            Donner les solutions de l'équation \( f(x)=0\).
        \item
            Sur quel intervalle la fonction est-elle croissante ?
        \item 
            Quel est le maximum de la fonction sur l'intervalle \( \mathopen[ -1 , \frac{ 1 }{2} \mathclose]\) ?
        \item 
            Quel est le maximum de la fonction sur l'intervalle \( \mathopen[ \frac{ 1 }{2} , 2 \mathclose]\) ?
        \item
            Étudier les variations de cette fonction. 
        \item
            Donner les antécédents de \( 2\).
        \item
            Quelle est l'image de \( -1\) par \( f\) ?
    \end{enumerate}


\corrref{smath-0628}
\end{exercice}
