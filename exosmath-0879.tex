% This is part of Un soupçon de mathématique sans être agressif pour autant
% Copyright (c) 2014
%   Laurent Claessens
% See the file fdl-1.3.txt for copying conditions.

\begin{exercice}\label{exosmath-0879}

    Vrai ou faux ?
    \begin{enumerate}
        \item
            Multiplier par \( \dfrac{ 1 }{ 2 }\) revient à diviser par \( 0.5\).
        \item
            L'aire d'un rectangle exprimée en \si{\centi\meter\squared}\ est toujours plus grande que son périmètre exprimé en \si{\centi\meter}.
        \item
            Multiplier par \( \dfrac{ 1 }{ 3 }\) revient à multiplier par \( 0.33\).
        \item
            Un mètre est la fraction \( \dfrac{ 1 }{ 1000 }\) d'un kilomètre.
        \item
            La moitié d'un nombre positif est toujours plus petit que le nombre.
        \item
            Multiplier par \( \dfrac{ 1 }{ 4 }\) revient à diviser par \( 4 \).
    \end{enumerate}

\corrref{smath-0879}
\end{exercice}
