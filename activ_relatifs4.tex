% This is part of Un soupçon de mathématique sans être agressif pour autant
% Copyright (c) 2014-2015
%   Laurent Claessens
% See the file fdl-1.3.txt for copying conditions.

%--------------------------------------------------------------------------------------------------------------------------- 
\subsection*{Activité : multiplication de nombres relatifs}
%---------------------------------------------------------------------------------------------------------------------------

Nous considérons le nombre \( B=(-2)+(-2)+(-2)+(-2)+(-2)\).

\begin{enumerate}
    \item
        Combien vaut \( B\) ?
    \item
        Écrire \( B\) sous forme d'un produit.
    \item
        Écrire sous forme d'une somme et calculer :
        \begin{enumerate}
            \item \( (-6)\times 4\)
            \item \( (-21)\times 5\)
            \item\( (-1.5)\times 3\).
        \end{enumerate}
\end{enumerate}

Compléter le tableau de produits suivant:
\begin{equation*}
    \begin{array}[]{|c||c|c|c|c|c|c|c|c|c|}
        \hline
        \times&-4&-3&-2&-1&\hphantom{-}0&1\hphantom{-}&2\hphantom{-}&3\hphantom{-}&4\hphantom{-}\\
        \hline\hline
        -4&&&&&0&&&&\\
        \hline
        -3&&&&&&&&&\\
        \hline
        -2&&&&&&&&&\\
        \hline
        -1&&&&&0&&&&\\
        \hline
        0&&&&&&&&0&\\
        \hline
        1&&&&&&&&&\\
        \hline
        2&&&&&&&&&\\
        \hline
        3&&&&&&&&&12\\
        \hline
        4&&&&&&4&&&\\
        \hline
    \end{array}
\end{equation*}
