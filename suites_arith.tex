% This is part of Un soupçon de mathématique sans être agressif pour autant
% Copyright (c) 2013
%   Laurent Claessens
% See the file fdl-1.3.txt for copying conditions.

%+++++++++++++++++++++++++++++++++++++++++++++++++++++++++++++++++++++++++++++++++++++++++++++++++++++++++++++++++++++++++++ 
\section{Suites arithmétiques}
%+++++++++++++++++++++++++++++++++++++++++++++++++++++++++++++++++++++++++++++++++++++++++++++++++++++++++++++++++++++++++++

\begin{definition}
    Une suite est \defe{arithmétique}{suite!arithmétique} quand on ajoute toujours le même nombre pour passer d'un terme au suivant.

Une suite arithmétique est donc définie par :
\begin{enumerate}
    \item
la donnée de son premier terme \( u_0\),
\item
une relation de récurrence de la forme : $u_{n+1}=u_n+r$.
\end{enumerate}
Le nombre $r$ qui permet de passer d'un terme au suivant s'appelle la raison de la suite \( (u_n)\).
\end{definition}

\begin{example}
    Une chaîne de supermarché a déjà \( 200\) points de vente. Chaque année elle en ouvre \( 6\) de plus.
\end{example}

\begin{Aretenir}
    Sens de variation.
    \begin{enumerate}
        \item
            Si \( a>0\) alors la suite est croissante.
        \item
            Si \( a<0\) alors la suite est décroissante.
        \item
            Si \( a=0\) alors la suite est constante.
    \end{enumerate}
\end{Aretenir}

\begin{Aretenir}
Placés sur un graphique, les éléments d'une suite arithmétique se mettent sur une droite.
\end{Aretenir}

%+++++++++++++++++++++++++++++++++++++++++++++++++++++++++++++++++++++++++++++++++++++++++++++++++++++++++++++++++++++++++++ 
\section{Exercices}
%+++++++++++++++++++++++++++++++++++++++++++++++++++++++++++++++++++++++++++++++++++++++++++++++++++++++++++++++++++++++++++

\Exo{smath-0163}
\Exo{smath-0164}
\Exo{smath-0309}
\Exo{smath-0165}
\Exo{smath-0166}
\Exo{smath-0211}
\Exo{smath-0167}
\Exo{smath-0169}
\Exo{smath-0170}
\Exo{smath-0303}
