% This is part of Un soupçon de mathématique sans être agressif pour autant
% Copyright (c) 2012
%   Laurent Claessens
% See the file fdl-1.3.txt for copying conditions.

\begin{exercice}\label{exoSeconde-0062}

    Soit \( ABC\) un triangle équilatéral et \( G\), son centre de gravité. Nous notons \( I\), le milieu du segment \( [BG]\). Calculer les coordonnées de \( I\) dans les repères \( (A,B,C)\), \( (G,B,C)\) et \( (A,I,B)\).

    Faire des dessins et penser à Thalès.

\corrref{Seconde-0062}
\end{exercice}
