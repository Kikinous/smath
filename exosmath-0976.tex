% This is part of Un soupçon de mathématique sans être agressif pour autant
% Copyright (c) 2014
%   Laurent Claessens
% See the file fdl-1.3.txt for copying conditions.

\begin{exercice}[\ldots\ldots/4]\label{exosmath-0976}

    Norbert le propriétaire possède un terrain de \SI{3}{\meter} de la large et \SI{10}{\meter} de long. Il décide de consacrer \( 6\) mètres à une terrasse et le reste à une piscine. 

\begin{center}
   \input{Fig_XUBNooNxZEmS.pstricks}
\end{center}

Répondre en justifiant aux questions suivantes.
\begin{enumerate}
    \item
        Exercice de lecture de l'énoncé : sur le dessin, la piscine est-elle la partie blanche ou hachurée ? Justifier en citant le texte.
    \item
        Laquelle des expressions suivantes donne l'aire de la piscine ?
        \begin{enumerate}
            \item
                \( 30-6\)
            \item
                \( 300+18\)
            \item
                \( 3\times (10-6)\).
        \end{enumerate}
\end{enumerate}

\corrref{smath-0976}
\end{exercice}
