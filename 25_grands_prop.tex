% This is part of Un soupçon de mathématique sans être agressif pour autant
% Copyright (c) 2014
%   Laurent Claessens
% See the file fdl-1.3.txt for copying conditions.

% This is part of Un soupçon de mathématique sans être agressif pour autant
% Copyright (c) 2014
%   Laurent Claessens
% See the file fdl-1.3.txt for copying conditions.

%--------------------------------------------------------------------------------------------------------------------------- 
\subsection*{Tableaux et représentation graphique}
%---------------------------------------------------------------------------------------------------------------------------

Associer à chaque tableau le graphique qui correspond. Quels sont ceux qui correspondent à des situations de proportionnalité ?

\begin{equation}
    \input{Fig_ARKZooKZOuAkconvprix.latex}\quad
    \input{ARKZooKZOuAkfar.latex}
\end{equation}
\begin{equation}
    \input{ARKZooKZOuAkmilm.latex}\quad
    \input{ARKZooKZOuAkctlibre.latex}
\end{equation}

\begin{center}
   \input{Fig_ARKZooKZOuAk0.pstricks}
   \input{Fig_ARKZooKZOuAk3.pstricks}
   \input{Fig_ARKZooKZOuAk2.pstricks}
   \input{Fig_ARKZooKZOuAk1.pstricks}
\end{center}


%+++++++++++++++++++++++++++++++++++++++++++++++++++++++++++++++++++++++++++++++++++++++++++++++++++++++++++++++++++++++++++ 
\section{Représentation graphique}
%+++++++++++++++++++++++++++++++++++++++++++++++++++++++++++++++++++++++++++++++++++++++++++++++++++++++++++++++++++++++++++

\begin{propriete}
    Une situation est représentée dans un repère par des points alignés avec l'origine si et seulement si c'est une situation de proportionnalité.
\end{propriete}

\begin{example}
    Le périmètre d'un carré est proportionnel à son côté parce que \( p=4\times c\). Représenter graphiquement le périmètre en fonction du côté.

    \begin{enumerate}
        \item
            On choisit quelque valeurs pour le côté \( c\)
        \item
            On calcule les valeurs correspondantes pour le périmètre \( p\) :
            \begin{equation*}
                \begin{array}[]{|c||c|c|c|c|}
                    \hline
                    \text{côté}&1&2&3&4\\
                    \hline\hline
                    \text{périmètre}&4&8&12&16\\
                    \hline
                \end{array}
            \end{equation*}
            La deuxième ligne est quatre fois la première.
    \end{enumerate}
    <++>

\end{example}
<++>

%+++++++++++++++++++++++++++++++++++++++++++++++++++++++++++++++++++++++++++++++++++++++++++++++++++++++++++++++++++++++++++ 
\section{Quatrième proportionnelle}
%+++++++++++++++++++++++++++++++++++++++++++++++++++++++++++++++++++++++++++++++++++++++++++++++++++++++++++++++++++++++++++

<++>
