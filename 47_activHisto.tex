% This is part of Un soupçon de mathématique sans être agressif pour autant
% Copyright (c) 2015
%   Laurent Claessens
% See the file fdl-1.3.txt for copying conditions.

%--------------------------------------------------------------------------------------------------------------------------- 
\subsection*{Activité : voyages en bus}
%---------------------------------------------------------------------------------------------------------------------------

Un chauffeur de bus effectue la navette entre le camping et la plage. Il est chargé de prendre note du nombre de passagers transporté à chaque voyage, afin que la compagnie de bus puisse savoir le nombre de bus sur la ligne est adéquat. Voici ses résultats :
\begin{equation}
\begin{array}[]{ccccccccccc}
52& 46& 32& 47& 20& 31& 26& 32& 40& 31& 57\\
33& 41& 17& 44& 39& 7& 36& 43& 51& 24& 23\\
44& 51& 34& 44& 54& 35& 49& 30& 56&
\end{array}
\end{equation}

\begin{enumerate}
    \item
 Combien de voyages a-t-il effectués au total ?
 \item
 Quel est le nombre minimum de passagers transportés pour un voyage ? Le maximum ?
\item
 Combien de voyages a-t-il effectués avec un nombre de passagers compris entre 41 (inclus) et 50 ?
\item
 Pour présenter le résultat de son travail à son patron, il aimerait réaliser un tableau et un histogramme permettant de visualiser facilement la répartition globale. Comment faire ?
\end{enumerate}
