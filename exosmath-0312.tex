% This is part of Un soupçon de mathématique sans être agressif pour autant
% Copyright (c) 2013
%   Laurent Claessens
% See the file fdl-1.3.txt for copying conditions.

\begin{exercice}\label{exosmath-0312}

    Le but de cet exercice est de trouver quel taux d'évolution est nécessaire pour garantir le doublement de la population d'une ville en 15 ans. Rendez-vous dans le fichier \info{TD\_suites\_stmg.ods}, onglet «doublement de population». Nous y avons considéré une ville de un million et demi d'habitants en $1980$ et l'évolution de sa population jusqu'en \( 1995\) en supposant un taux de \( 5\%\) par an.

    \begin{enumerate}
        \item
            Expliquer la formule donnant le cellule B6. Quel est son rapport avec une évolution de \( 5\%\) ?
        \item
            En modifiant la cellule B1, chercher un taux qui donne un doublement de la population en 15 ans.
        \item
            Tracer un graphique donnant la population en fonction de l'année. pour le taux d'évolution trouvé.
    \end{enumerate}

\corrref{smath-0312}
\end{exercice}
