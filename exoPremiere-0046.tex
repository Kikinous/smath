% This is part of Un soupçon de mathématique sans être agressif pour autant
% Copyright (c) 2012
%   Laurent Claessens
% See the file fdl-1.3.txt for copying conditions.

\begin{exercice}\label{exoPremiere-0046}
    
    Trouver le racines des polynômes suivants.
    \begin{multicols}{3}
        \begin{enumerate}
            \item
                $f_1(x)=x(x-5)$
            \item
                \( f_2(x)=x^2+x-2\)
            \item
                \( f_3(x)=x^2+\frac{ 5x }{ 4 }+\frac{ 3 }{ 8 }\)
            \item
                \( f_4(x)=16x^2+20x+6\)
            \item
                \( f_5(x)=-x^2-2x-1\)
            \item
                \( f_6(x)=-x^2-7x\)
            \item
                \( f_7(x)=(x+1)^2\)
            \item
                \( f_8(x)=(x-1)^2\)
            \item
                \( f_9(x)=2x^2-\frac{ 1 }{2}\)
        \end{enumerate}
    \end{multicols}

    Ce sont ceux de l'exercice \ref{exoPremiere-0040}.

\corrref{Premiere-0046}
\end{exercice}
