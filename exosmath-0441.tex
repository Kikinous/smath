% This is part of Un soupçon de mathématique sans être agressif pour autant
% Copyright (c) 2013
%   Laurent Claessens
% See the file fdl-1.3.txt for copying conditions.

\begin{exercice}[\cite{oklaEg}]\label{exosmath-0441}


Lors d'une élection, un sondage portant sur un échantillon aléatoire de 1000 personnes donne 400 votants en faveur d’un candidat L. Au risque d'erreur de 5\%, quelle information peut-on obtenir sur la proportion réelle d'électeurs envisageant de voter pour L ?

Quel conseil donneriez-vous à quelqu'un qui voudrait parier un gros paquet sur la victoire de L ?

\corrref{smath-0441}
\end{exercice}
