% This is part of Un soupçon de mathématique sans être agressif pour autant
% Copyright (c) 2015
%   Laurent Claessens
% See the file fdl-1.3.txt for copying conditions.

\begin{exercice}\label{exo2smath-0251}

    Vrai ou faux ? (justifier)
    \begin{enumerate}
        \item
            Le petit chaperon rouge traverse la forêt et prend des notes sur son avancement :
            \begin{center}
                \begin{tabular}[]{|c||c|c|c|}
                    \hline
                    temps de écoulé&30 minutes&60 minutes&deux heures\\
                    \hline
                    Distance parcourue&\SI{2}{\kilo\meter}&\SI{4}{\kilo\meter}&\SI{6}{\kilo\meter}\\ 
                    \hline
                \end{tabular}
            \end{center}
                Est-ce qu'elle avance à vitesse constante ?
        \item
            Si \( ABC\) est un triangle équilatéral, la somme des angles \( \widehat{ABC}\) et \( \widehat{BCA}\) vaut \(\SI{80}{\degree}\).
        \item
            Si \( x\) est un nombre quelconque alors \( x+5\) est plus grand que \( 4\).
    \end{enumerate}

\corrref{2smath-0251}
\end{exercice}
