% This is part of Un soupçon de mathématique sans être agressif pour autant
% Copyright (c) 2013
%   Laurent Claessens
% See the file fdl-1.3.txt for copying conditions.

\begin{exercice}[\cite{GZAtGXY}]\label{exosmath-0397}

La figure \ref{LabelFigKQluTdN} donne le graphique de la fonction \( f(x)=-x^2+2x+2\) ainsi que ses tangentes aux abscisses \( -1\) et \( 2\). Attention : remarquez que le graphe n'est pas dessiné dans un repère orthonormé.

\newcommand{\CaptionFigKQluTdN}{Le graphe de la fonction \( f(x)=-x^2+2x+2\) pour l'exercice \ref{exosmath-0397}.}
\input{Fig_KQluTdN.pstricks}

\begin{enumerate}
    \item
        Déterminer graphiquement les équations des deux tangentes tracées.
    \item
        En déduire la valeur de la dérivée de \( f\) aux abscisses \( -1\) et \( 2\).
    \item
        Retrouver les résultats par le calcul.
\end{enumerate}

\corrref{smath-0397}
\end{exercice}
