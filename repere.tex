%+++++++++++++++++++++++++++++++++++++++++++++++++++++++++++++++++++++++++++++++++++++++++++++++++++++++++++++++++++++++++++
\section{Repères, distances, milieu}
%+++++++++++++++++++++++++++++++++++++++++++++++++++++++++++++++++++++++++++++++++++++++++++++++++++++++++++++++++++++++++++

%---------------------------------------------------------------------------------------------------------------------------
\subsection{Activité}
%---------------------------------------------------------------------------------------------------------------------------

Si nous plaçons un point sur le tableau, comment faire pour mettre un point au même endroit sur le tableau de la classe d'à côté ?

%---------------------------------------------------------------------------------------------------------------------------
\subsection{Repère}
%---------------------------------------------------------------------------------------------------------------------------

\begin{definition}
    Un \defe{repère orthonormé}{repère!orthonormé} du plan est la donné de trois points \( O\), \( I\), \( J\) formant un triangle rectangle isocèle en  \( O\).
\end{definition}

%---------------------------------------------------------------------------------------------------------------------------
\subsection{Exercices}
%---------------------------------------------------------------------------------------------------------------------------

\Exo{Seconde-0001}
\Exo{Seconde-0002}
\Exo{Seconde-0003}
\Exo{Seconde-0004}
\Exo{Seconde-0005}
\Exo{Seconde-0006}
\Exo{Seconde-0007}
\Exo{Seconde-0008}
\Exo{Seconde-0009}
\Exo{Seconde-0010}
\Exo{Seconde-0011}
\Exo{Seconde-0012}
\Exo{Seconde-0013}
\Exo{Seconde-0021}


