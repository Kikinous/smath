%+++++++++++++++++++++++++++++++++++++++++++++++++++++++++++++++++++++++++++++++++++++++++++++++++++++++++++++++++++++++++++
\section{Repères, distances, milieu}
%+++++++++++++++++++++++++++++++++++++++++++++++++++++++++++++++++++++++++++++++++++++++++++++++++++++++++++++++++++++++++++

%---------------------------------------------------------------------------------------------------------------------------
\subsection{Activité}
%---------------------------------------------------------------------------------------------------------------------------

Si nous plaçons un point sur le tableau, comment faire pour mettre un point au même endroit sur le tableau de la classe d'à côté ?

%---------------------------------------------------------------------------------------------------------------------------
\subsection{Repère}
%---------------------------------------------------------------------------------------------------------------------------

\begin{definition}
    Un \defe{repère orthonormé}{repère!orthonormé} du plan est la donné de trois points \( O\), \( I\), \( J\) non alignés formant un triangle rectangle isocèle en  \( O\).
\end{definition}

La façon dont nous associons à chaque point \( M\) du plan ses coordonnées dans le repère \( O\), \( I\), \( J\) est donnée à la figure \ref{LabelFigReperexjVyii}.
\newcommand{\CaptionFigReperexjVyii}{Lire les coordonnées du point \( M\) dans le repère \( OIJ\).}
\input{Fig_ReperexjVyii.pstricks}

%---------------------------------------------------------------------------------------------------------------------------
\subsection{Milieu d'un segment}
%---------------------------------------------------------------------------------------------------------------------------

\begin{example}
    Le tableau suivant recense différentes situations de points \( A\) et \( B\) dans le plan. Pour chaque cas, dessiner les points et trouver le milieu.

    \begin{center}
        \begin{tabular}[h]{|c||c|c|c|c|c|}
            \hline
            \( A\)&\( (3;0)\)&\( (1;1)\)&\( 0;0\)&\( (-1,3)\)&\( (7;-8)\)\\
            \hline
            \( B\)&\( (7,0)\)&\( (3,3)\)&\( (3,4)\)&\( (1,-5)\)&\( (-6,1)\)\\
            \hline\hline
            milieu&&&&&\\
            \hline
        \end{tabular}
    \end{center}
\end{example}

\begin{Aretenir}
    Soient les points \( A\) et \( B\) de coordonnées \( A=(x_A,y_A)\) et \( B=(x_B,y_B)\) dans un repère. Alors le milieu du segment \( [AB]\) est le point de coordonnées
    \begin{equation}
        \begin{aligned}[]
            x_M&=\frac{ x_A+x_B }{ 2 },&y_M&=\frac{ y_A+y_B }{2}.
        \end{aligned}
    \end{equation}
\end{Aretenir}

%+++++++++++++++++++++++++++++++++++++++++++++++++++++++++++++++++++++++++++++++++++++++++++++++++++++++++++++++++++++++++++
\section{Exercices de géométrie repérée}
%+++++++++++++++++++++++++++++++++++++++++++++++++++++++++++++++++++++++++++++++++++++++++++++++++++++++++++++++++++++++++++

%---------------------------------------------------------------------------------------------------------------------------
\subsection{Géométrie brute}
%---------------------------------------------------------------------------------------------------------------------------


\Exo{Seconde-0045}

%---------------------------------------------------------------------------------------------------------------------------
\subsection{Placer et lire des coordonnées dans un repère}
%---------------------------------------------------------------------------------------------------------------------------

\Exo{Seconde-0001}
\Exo{Seconde-0002}
\Exo{Seconde-0007}

%---------------------------------------------------------------------------------------------------------------------------
\subsection{Milieu de segments}
%---------------------------------------------------------------------------------------------------------------------------

\Exo{Seconde-0056}
\Exo{Seconde-0012}
\Exo{Seconde-0055}
\Exo{Seconde-0011}

%---------------------------------------------------------------------------------------------------------------------------
\subsection{Longueur de segments}
%---------------------------------------------------------------------------------------------------------------------------

\Exo{Seconde-0008}
\Exo{Seconde-0003}
\Exo{Seconde-0004}
\Exo{Seconde-0005}
\Exo{Seconde-0006}
\Exo{Seconde-0009}
\Exo{Seconde-0010}
\Exo{Seconde-0013}
\Exo{Seconde-0021}
\Exo{Seconde-0019}
\Exo{Seconde-0020}

