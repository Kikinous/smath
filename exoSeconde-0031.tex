% This is part of Un soupçon de mathématique sans être agressif pour autant
% Copyright (c) 2012
%   Laurent Claessens
% See the file fdl-1.3.txt for copying conditions.

\begin{exercice}\label{exoSeconde-0031}

    Résoudre les équations suivantes :
    \begin{subequations}
        \begin{align}
            110&=105+105\frac{ x }{ 100 }\\
            90&=110+110\frac{ x }{ 100 }
        \end{align}
    \end{subequations}
    Si le prix d'un article passe de \( 105\) à \( 110\), quel est le pourcentage d'augmentation ?

    Si le prix passe de \( 110\) à \( 90\), ce n'est pas une augmentation mais une diminution. De quel pourcentage ?
\corrref{Seconde-0031}
\end{exercice}
