% This is part of Un soupçon de mathématique sans être agressif pour autant
% Copyright (c) 2014
%   Laurent Claessens
% See the file fdl-1.3.txt for copying conditions.

\begin{exercice}[\cite{NRHooXFvgpp4}]\label{exo2smath-0037}

Compléter les pointillés
\begin{center}
   \input{Fig_UKRHooEvocBg.pstricks}
\end{center}
\begin{enumerate}
    \item
        La partie grisée représente \( \dfrac{ 2 }{ \ldots }\) de l'aire totale.
    \item
        La partie hachurée représente \( \dfrac{ 1 }{ \ldots }\) de l'aire totale.
    \item
        En s'aidant du dessin, calculer la somme
        \begin{equation}
            \frac{ 2 }{ 3 }+\frac{1}{ 4 }=\ldots
        \end{equation}
\end{enumerate}

\corrref{2smath-0037}
\end{exercice}
