% This is part of Un soupçon de mathématique sans être agressif pour autant
% Copyright (c) 2014
%   Laurent Claessens
% See the file fdl-1.3.txt for copying conditions.

\begin{exercice}[\ldots\ldots/4]\label{exosmath-0885}

    Un biscuit coûte \( 1\) euro et une bouteille d'eau coûte \( 2.5\) euros. Sarah souhaite constituer \( 12\) sacs contenant chacun deux biscuits et une bouteille d'eau. 

            Lesquelles parmi les expressions suivantes donnent le bon résultat ? Justifier vos choix.
            \begin{enumerate}
                \item
                    \( A=12\times (   2+2.5 )\)
                \item
                    \( B=12+24\times 2.5\)
                \item
                    \( C=12\times 4.5\).
                \item
                    \( D=12\times 2.5+2\)
            \end{enumerate}
    Expliquer brièvement vos choix.

\corrref{smath-0885}
\end{exercice}
