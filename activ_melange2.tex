% This is part of Un soupçon de mathématique sans être agressif pour autant
% Copyright (c) 2015
%   Laurent Claessens
% See the file fdl-1.3.txt for copying conditions.

%--------------------------------------------------------------------------------------------------------------------------- 
\subsection*{Activité : des mélanges}
%---------------------------------------------------------------------------------------------------------------------------

Un restaurateur prépare un mélange de jus de fruits : deux litres de jus d'orange pour trois litres de jus de pomme. 
\begin{enumerate}
    \item
        Compléter ce tableau :
        \begin{equation*}
            \begin{array}[]{|c|c|c|}
                \hline
                \text{Volume de jus d'orange}&\SI{2}{\liter}&\ldots\ldots\\
                \hline
                \text{Volume de mélange}&\ldots\ldots&\SI{100}{\liter}\\
                \hline
            \end{array}
        \end{equation*}
    \item
        Exprimer la proportion de jus de pomme dans le mélange sous forme d'une fraction ayant \( 100\) comme dénominateur.
\end{enumerate}
