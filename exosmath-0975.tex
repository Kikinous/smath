% This is part of Un soupçon de mathématique sans être agressif pour autant
% Copyright (c) 2014
%   Laurent Claessens
% See the file fdl-1.3.txt for copying conditions.

\begin{exercice}[\ldots\ldots/4]\label{exosmath-0975}

    Pour chacune des deux situations suivantes, tracer le triangle \( ABC\), ou expliquer pourquoi vous n'y parvenez pas. Laisser les traits de construction.
    \begin{enumerate}
        \item
            le triangle \( ABC\) dont les mesures sont \( AB=\SI{5}{\centi\meter}\), \( BC=\SI{10}{\centi\meter}\) et \( AC=\SI{7}{\centi\meter}\).
        \item
            le triangle \( KLM\) dont les mesures sont \( KL=\SI{5}{\centi\meter}\), \( LM=\SI{15}{\centi\meter}\) et \( KM=\SI{7}{\centi\meter}\).
    \end{enumerate}

\corrref{smath-0975}
\end{exercice}
