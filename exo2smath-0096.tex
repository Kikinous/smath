% This is part of Un soupçon de mathématique sans être agressif pour autant
% Copyright (c) 2015
%   Laurent Claessens
% See the file fdl-1.3.txt for copying conditions.

\begin{exercice}\label{exo2smath-0096}

    Un train avance à vitesse constante (la distance parcourue est donc proportionnelle au temps de voyage). 
    
    \begin{enumerate}
        \item
            
    Lequel de ces trois graphiques correspond à cette situation (les durées sont données en minutes, et les distances en kilomètres) ?

   \input{Fig_QXYWooOWHshl0.pstricks}
   \input{Fig_QXYWooOWHshl1.pstricks}
   \input{Fig_QXYWooOWHshl2.pstricks}

 \item

Compléter le tableau suivant donnant l'avancée du train en fonction du temps :
\begin{center}
$
    \begin{array}[]{|c|c|c|c|}
          \hline
          \text{Durée (\si{\minute})}&27&&\\
        \hline
        \text{Distance (\si{\kilo\meter})}&&100&130\\
          \hline
    \end{array}
    $
\end{center}
    \end{enumerate}

\corrref{2smath-0096}
\end{exercice}
