% This is part of Un soupçon de mathématique sans être agressif pour autant
% Copyright (c) 2014
%   Laurent Claessens
% See the file fdl-1.3.txt for copying conditions.

\begin{MentalActivity}
    \begin{mental}

        \begin{center}
   \input{Fig_RYDooDLpToB.pstricks}
        \end{center}
        
        Le périmètre du rectangle est
        \begin{enumerate}
            \item
                \( 2\times 15+\ell\)
            \item
                \( 15\times \ell\)
            \item
                \( 2\times (15+\ell)\)
        \end{enumerate}

    \end{mental}

    \begin{mental}
        \begin{enumerate}
            \item
                $\SI{250}{\centi\meter}=\ldots\si{\meter}$
            \item
                $\SI{1}{\liter}=\ldots\si{\deci\cubic\meter}$
            \item
                \( \SI{1}{\hour}=\ldots\si{\second}\)
        \end{enumerate}
    \end{mental}

    \begin{mental}
        Vrai ou faux ?
        \begin{enumerate}
            \item
                Si un stylo coûte \( 1.5\) euros, alors trois stylos coûtent \( 4.5\) euros.
            \item
                Le quadruple d'un nombre \( x\) s'écrit \( 4\times x\).
            \item
                Si j'ai \( x\) pièces de \( 1\) centime, alors j'ai \( x-100\) euros.
        \end{enumerate}
    \end{mental}

    \begin{mental}
        Calculs 
        \begin{enumerate}
            \item
                Combien vaut \( 7\times a\) si \( a=9\) ?
            \item
                Combien vaut \( 2\times b+5\) si \( b=3\) ?
            \item
                Combien vaut \( 7\times a+b\) si \( a=2\) et \( b=6\) ?
        \end{enumerate}
    \end{mental}


\end{MentalActivity}

%%%%%%%%%%%%%%%%%%%%%%%%%%%%%%%%%%%%%%%%%%%%%%%%%%%%%%%%%%%%
\begin{MentalActivity}

    \begin{mental}
        Quelle est la circonférence d'un cercle de rayon \( \SI{2}{\centi\meter}\) ? (rappel : la formule est \( 2\times\pi\times R\)) ?
    \end{mental}

    \begin{mental}
        \begin{enumerate}
            \item
                L'aire de ce rectangle est de \(\SI{20}{\centi\meter\squared}\).
        \begin{center}
           \input{Fig_LCUooNGZJFk.pstricks}
        \end{center}
        Que vaut sa hauteur \( h\) ?
            \item
                L'aire de ce rectangle est de \(\SI{20}{\centi\meter\squared}\).
                \begin{center}
                    \input{Fig_KXXooKBoqAY.pstricks}
                \end{center}
                Que vaut sa hauteur \( h\) en fonction de sa longueur \( l\) ?
        \end{enumerate}
    \end{mental}

    \begin{mental}
        Vrai ou faux ?
        \begin{enumerate}
            \item
                \( 8.7^2>87\)
            \item
                Pour tout entier \( n\), le nombre \( 3n\) est dans la table de \( 6\).
            \item
                \begin{equation}
                    \frac{1}{ -2 }=-\frac{1}{ 2 }.
                \end{equation}
        \end{enumerate}
    \end{mental}
    
\end{MentalActivity}
%%%%%%%%%%%%%%%%%%%%%%%%%%%%%%%%%%%%%%%%%%%%%%%%%%%%%%%%%%
\begin{MentalActivity}
    \begin{mental}
        Calculer
        \begin{enumerate}
            \item
                \( 30\times 22\)
            \item
                \( 32\times 22\)
        \end{enumerate}
    \end{mental}
    \begin{mental}
        Quelle est l'aire de cette figure ?
        \begin{center}
        \large
            \input{Fig_KUMFooSHPFkG.pstricks}
        \end{center}
    \end{mental}
    \begin{mental}
        Déterminer l'aire du rectangle \( ABCD\) en fonction de \( x\) :
\begin{center}
    \large
   \input{Fig_YIGPooZbaCAI.pstricks}
\end{center}
    \end{mental}
\end{MentalActivity}


\begin{MentalActivity}
% This is part of Un soupçon de mathématique sans être agressif pour autant
% Copyright (c) 2015
%   Laurent Claessens
% See the file fdl-1.3.txt for copying conditions.

\begin{mental}
    \begin{center}
        \large
\input{Fig_VXUCooAxCIuP.pstricks}
    \end{center}
    Compléter les pointillés :
    \begin{enumerate}
        \item
            \( \dfrac{ \ldots }{ AE }=\dfrac{ AC }{ AB }\)
        \item
        $CD\times AE=\ldots \times AD$
    \end{enumerate}
\end{mental}
<++>

% This is part of Un soupçon de mathématique sans être agressif pour autant
% Copyright (c) 2015
%   Laurent Claessens
% See the file fdl-1.3.txt for copying conditions.

\begin{mental}
    Développer :
    \begin{enumerate}
        \item
            \( 3\times (a+2)\)
        \item
            \( 7\times (1-4a)\)
        \item
            \( (a+1)\times (a+2)\)
    \end{enumerate}
\end{mental}

% This is part of Un soupçon de mathématique sans être agressif pour autant
% Copyright (c) 2015
%   Laurent Claessens
% See the file fdl-1.3.txt for copying conditions.

\begin{mental}
L'aire l'un cercle de rayon \( R\) est de \SI{20}{\centi\meter\squared}. Quelle formule est correcte ?
\begin{enumerate}
    \item
        \( 20=2\pi R\)
    \item
        \( \pi=\dfrac{ 20 }{ R^2 }\)
    \item
        \( R=20\times \pi\)
\end{enumerate}
\end{mental}

\end{MentalActivity}

% 20 février 2015
\begin{MentalActivity}
% This is part of Un soupçon de mathématique sans être agressif pour autant
% Copyright (c) 2015
%   Laurent Claessens
% See the file fdl-1.3.txt for copying conditions.

\begin{mental}
    Factoriser :
    \begin{enumerate}
        \item
            \( 5a+10\)
        \item
            \( 3a^2+9a\)
    \end{enumerate}
\end{mental}

% This is part of Un soupçon de mathématique sans être agressif pour autant
% Copyright (c) 2015
%   Laurent Claessens
% See the file fdl-1.3.txt for copying conditions.

\begin{mental}
    
Quelle est la valeur de 
\begin{equation}
    s\times (s+9)
\end{equation}
si \( s=-3\) ?
\end{mental}


% This is part of Un soupçon de mathématique sans être agressif pour autant
% Copyright (c) 2015
%   Laurent Claessens
% See the file fdl-1.3.txt for copying conditions.

\begin{mental}
    
Vrai ou faux ?

Si \( ABC\) est un triangle rectangle en \( A\), et si \( AB=4\) et \( BC=10\) alors \( AC>8\).
\end{mental}


% This is part of Un soupçon de mathématique sans être agressif pour autant
% Copyright (c) 2015
%   Laurent Claessens
% See the file fdl-1.3.txt for copying conditions.


\begin{mental}

\( (ML)\parallel (KJ)\).


\begin{center}
   \input{Fig_VXDJooBGRpbL.pstricks}
\end{center}

\begin{enumerate}
    \item
        \( IK=\ldots\)
    \item
        \( KM=\ldots\)
\end{enumerate}

\end{mental}
<++>

\end{MentalActivity}

% Il faut des patrons


%%%%%%%%%%%%%%%%%%%%%%%%%%%%%%%%%%%%%%%%%%%%%%%%%%%%%%%%%%

% Les exercices suivants contienent des vrai/faux sur le second degré. À mettre dans les prochains calculs mental.
% M'est avis qu'il faut les ajouter aussi à ``autres exercices de seconde''.
%\Exo{smath-0652}    % Haag; c'est mon vrai ou faux en vrac à compléter.
%\Exo{smath-0254}
