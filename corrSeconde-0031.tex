% This is part of Un soupçon de mathématique sans être agressif pour autant
% Copyright (c) 2012
%   Laurent Claessens
% See the file fdl-1.3.txt for copying conditions.

\begin{corrige}{Seconde-0031}

    \begin{enumerate}
        \item
            La première chose à faire est de séparer les termes en \( x\) de termes sans \( x\) :
            \begin{equation}
                110-105=105\frac{ x }{ 100 }.
            \end{equation}
            Le terme \( 105\frac{ x }{ 100 }\) s'écrit tout aussi bien \( \frac{ 105 }{ 100 }x\) et nous avons donc l'équation
            \begin{equation}
                5=\frac{ 105 }{ 100 }x.
            \end{equation}
            Pour isoler \( x\) il suffit alors de multiplier les deux membres par l'inverse de \( \frac{ 105 }{ 100 }\), c'est à dire par \( \frac{ 100 }{ 105 }\) :
            \begin{equation}
                5\times \frac{ 100 }{ 105 }=x.
            \end{equation}
            Il ne s'agit plus maintenant que de simplifier la fraction. En simplifiant le \( 5\) avec le \( 105\),
            \begin{equation}
                x=\frac{ 100 }{ 21 }.
            \end{equation}
            Il vaut mieux s'abstenir de donner une approximation numérique lorsque vous n'avez pas de contexte !

        \item
            Le même schéma de résolution s'applique :
            \begin{subequations}
                \begin{align}
                    90&=110+110\frac{ x }{ 100 }\\
                    90-110&=\frac{ 110 }{ 100 }x\\
                    -20&=\frac{ 11 }{ 10 }x\\
                    x&=-\frac{ 200 }{ 11 }.
                \end{align}
            \end{subequations}
    \end{enumerate}

\end{corrige}
