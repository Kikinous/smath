% This is part of Un soupçon de mathématique sans être agressif pour autant
% Copyright (c) 2014
%   Laurent Claessens
% See the file fdl-1.3.txt for copying conditions.

\begin{exercice}\label{exosmath-0704}

    Nous considérons les points \( A(1;2)\), \( B(-2;4)\) et \( C(-1;1)\) dans un repère orthonormé. Soit \( F(x;x+1)\); déterminer pour quelle valeur de \( x\) les droites \( (AB)\) et \( (CF)\) sont parallèles.

\corrref{smath-0704}
\end{exercice}
