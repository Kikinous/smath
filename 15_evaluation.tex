% This is part of Un soupçon de mathématique sans être agressif pour autant
% Copyright (c) 2014
%   Laurent Claessens
% See the file fdl-1.3.txt for copying conditions.

% LES ĚVALUATIONS DE CINQUIÈME

\begin{center}
    Ceci sont les énoncés et les corrections des questions posées en évaluation. Attention : certaines questions ont pu être modifiées par rapport à l'énoncé exact donné aux élèves; cela pour clarifier certaines questions qui n'ont pas été bien comprises.
\end{center}

%\begin{minipage}{0.485\textwidth}
    \enteteInterro{Le 19 septembre 2014}{1}{A}

    Calculer :
    \begin{enumerate}
        \item
            \( 8-3\times 2=\ldots\)
        \item
            \( \dfrac{ 16+4 }{ 5 }=\ldots\)
        \item
            \( \ldots \times 4+12=40\)
        \item
            \( (8-3)\times 2=\ldots\)
        \item
            \(  4\times 3-4=\ldots \)
    \end{enumerate}

\end{minipage}
\begin{minipage}{0.485\textwidth}
    \enteteInterro{Le 19 septembre 2014}{1}{B}

    Calculer :
    \begin{enumerate}
        \item
            \( 17-4\times 4=\ldots\)
        \item
            \( \dfrac{ 12+6 }{ 2 }=\ldots\)
        \item
            \( \ldots \times 6+5=41\)
        \item
            \( (10-6)\times 9=\ldots\)
        \item
            \(  9\times 3-7=\ldots \)
    \end{enumerate}

\end{minipage}
      % C'est ce fichier qui demandait exosmath-0813 et exosmath-0814

%+++++++++++++++++++++++++++++++++++++++++++++++++++++++++++++++++++++++++++++++++++++++++++++++++++++++++++++++++++++++++++ 
\section{29 septembre 2014}
%+++++++++++++++++++++++++++++++++++++++++++++++++++++++++++++++++++++++++++++++++++++++++++++++++++++++++++++++++++++++++++

\Exo{smath-0813}
\Exo{smath-0814}
\Exo{smath-0829}
\Exo{smath-0830}

%+++++++++++++++++++++++++++++++++++++++++++++++++++++++++++++++++++++++++++++++++++++++++++++++++++++++++++++++++++++++++++ 
\section{3 octobre 2014}
%+++++++++++++++++++++++++++++++++++++++++++++++++++++++++++++++++++++++++++++++++++++++++++++++++++++++++++++++++++++++++++

\Exo{smath-0820}
\Exo{smath-0822}
\Exo{smath-0823}
\Exo{smath-0824}
\Exo{smath-0847}
\Exo{smath-0846}
\Exo{smath-0845}

%+++++++++++++++++++++++++++++++++++++++++++++++++++++++++++++++++++++++++++++++++++++++++++++++++++++++++++++++++++++++++++ 
\section{17 octobre 2014}
%+++++++++++++++++++++++++++++++++++++++++++++++++++++++++++++++++++++++++++++++++++++++++++++++++++++++++++++++++++++++++++

\Exo{smath-0886}
\Exo{smath-0885}
\Exo{smath-0893}    % gâteau au chocolat. Il faut un peu le modifier avant de le reposer
                    % parce qu'il serait mieux qu'il n'y ait pas deux ingrédients ayant les mêmes quantités (100g de chocolat et de lait).
\Exo{smath-0907}
\Exo{smath-0889}
\Exo{smath-0903}

%+++++++++++++++++++++++++++++++++++++++++++++++++++++++++++++++++++++++++++++++++++++++++++++++++++++++++++++++++++++++++++ 
\section{Devoir maison}
%+++++++++++++++++++++++++++++++++++++++++++++++++++++++++++++++++++++++++++++++++++++++++++++++++++++++++++++++++++++++++++

\Exo{smath-0841}

%+++++++++++++++++++++++++++++++++++++++++++++++++++++++++++++++++++++++++++++++++++++++++++++++++++++++++++++++++++++++++++ 
\section{Devoir surveillé, 25 novembre 2014}
%+++++++++++++++++++++++++++++++++++++++++++++++++++++++++++++++++++++++++++++++++++++++++++++++++++++++++++++++++++++++++++

\Exo{smath-0971}
\Exo{smath-0972}
\Exo{smath-0976}
\Exo{smath-0974}
\Exo{smath-0975}
\Exo{smath-0978}
\Exo{smath-0977}

%+++++++++++++++++++++++++++++++++++++++++++++++++++++++++++++++++++++++++++++++++++++++++++++++++++++++++++++++++++++++++++ 
\section{Interrogation du 9 décembre 2014}
%+++++++++++++++++++++++++++++++++++++++++++++++++++++++++++++++++++++++++++++++++++++++++++++++++++++++++++++++++++++++++++

\Exo{2smath-0001}   % Attention : la réponse de celui-ci est négative !! à ne pas refaire.
\Exo{2smath-0002}  

\corrPosition{1}

%+++++++++++++++++++++++++++++++++++++++++++++++++++++++++++++++++++++++++++++++++++++++++++++++++++++++++++++++++++++++++++ 
\section{Devoir surveillé, 16 décembre 2014}
%+++++++++++++++++++++++++++++++++++++++++++++++++++++++++++++++++++++++++++++++++++++++++++++++++++++++++++++++++++++++++++

\Exo{smath-0973}    % Ce serait bien de changer les nombres pour que la part totale ne soit pas un demi.
\Exo{2smath-0009}  
\Exo{2smath-0005}  
\Exo{2smath-0006} 
\Exo{2smath-0007}  
\Exo{2smath-0003}  
\Exo{2smath-0017}

