% This is part of Un soupçon de mathématique sans être agressif pour autant
% Copyright (c) 2013
%   Laurent Claessens
% See the file fdl-1.3.txt for copying conditions.

\begin{exercice}\label{exosmath-0567}

\begin{wrapfigure}{r}{5.0cm}
   \vspace{-1cm}        % à adapter.
   \centering
   \input{Fig_QLbEnnx.pstricks}
\end{wrapfigure}

Sur la figure ci-contre, \( ABCD\) est un tétraèdre. Le point \( J\) est sur le segment \( [AD]\) et le point \( I\) est sur le segment \( [BC]\).
\begin{enumerate}
    \item
        Donner la droite d'intersection des plans \( (DIJ)\) et \( (BCD)\).
    \item
        Donner la droite d'intersection des plans \( (DIJ)\) et \( (ABD)\).
\end{enumerate}

\corrref{smath-0567}
\end{exercice}
