% This is part of Un soupçon de mathématique sans être agressif pour autant
% Copyright (c) 2014
%   Laurent Claessens
% See the file fdl-1.3.txt for copying conditions.

\begin{corrige}{smath-0831}

    Pour un contre-exemple, il faut trouver un nombre qui est divisible en même temps par \( 3\) et \( 7\) et qui n'est pourtant pas un multiple de \( 42\). C'est le cas de \( 21\) et de \( 63\).

    Note : le critère de divisibilité par \( 3\) est que la somme des chiffres du nombre doit être divisible par \( 3\).

\end{corrige}
