% This is part of Un soupçon de mathématique sans être agressif pour autant
% Copyright (c) 2014
%   Laurent Claessens
% See the file fdl-1.3.txt for copying conditions.

\begin{corrige}{smath-0970}

    Le dessin très schématique (et pas spécialement à l'échelle) est comme ceci :
\begin{center}
   \input{Fig_MEAQooWoXbHZ.pstricks}
\end{center}
Le côté \( [CA]\) est la muraille, et \( [AB]\) est la douve. La question revient à demander si \( [CB]\) est ou non plus long que \SI{7}{\meter}.

L'égalité de Pythagore dans le triangle rectangle \( ABC\) est :
\begin{equation}
    CB^2=2^2+6^2=4+36=40.
\end{equation}
Vu que \( 6^2=36\) et \( 7^2=49\), la longueur \( CB\) est entre \( 6\) et \( 7\). Par conséquent une échelle de \SI{7}{\meter} suffit.


Note : pour cette question, il est également possible de répondre en faisant un dessin à l'échelle (par exemple \SI{1}{\centi\meter} représente \SI{1}{\meter}). Il n'est par contre pas possible de répondre par un dessin à la question «quelle est la longueur de \( [AB]\)» parce que le dessin est assez précis pour voir que cette longueur est plus grande que \SI{7}{\centi\meter}, mais il n'est pas assez précis pour donner une valeur exacte.

\end{corrige}
