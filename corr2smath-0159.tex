% This is part of Un soupçon de mathématique sans être agressif pour autant
% Copyright (c) 2015
%   Laurent Claessens
% See the file fdl-1.3.txt for copying conditions.

\begin{corrige}{2smath-0159}

    Les égalités de Thalès sont :
    \begin{equation}
        \frac{ AO }{ OD }=\frac{ AB }{ DC }=\frac{ OB }{ OC }.
    \end{equation}
    Si \( OD=9\) alors ces égalités donnent
    \begin{equation}
        \frac{ OA }{ 9 }=\frac{ 7 }{ 3 }
    \end{equation}
    et donc \( 3\times OA=9\times 7\), ce qui donne \( OA=21\).

    Pour des valeurs possibles à donner à \( OC\) et \( CB\), rien n'empèche de prendre les mêmes que \( OD\) et \( DA\). Tous les choix sont possibles tant que les deux conditions suivantes soient respectées :
    \begin{itemize}
        \item \( \frac{ OB }{ OC }=\frac{ 7 }{ 3 }\)
        \item les inégalités triangulaires dans \( ODC\) et \( OBA\).
    \end{itemize}

\end{corrige}
