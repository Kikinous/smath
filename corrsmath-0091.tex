% This is part of Un soupçon de mathématique sans être agressif pour autant
% Copyright (c) 2012
%   Laurent Claessens
% See the file fdl-1.3.txt for copying conditions.

\begin{corrige}{smath-0091}

    \begin{enumerate}
        \item
            La formule est \( x_0=-\frac{ b }{ 2a }\).
        \item
            Ici nous avons \( a=1\), \( b=-3\) et \( c=5\), donc le sommet est en
            \begin{equation}
                x_0=-\frac{ -3 }{ 2\times 1 }=\frac{ 3 }{ 2 }.
            \end{equation}
            Cela est l'abscisse. Pour trouver l'ordonnée du sommet, il faut encore calculer \( y=f(x_0)=\frac{ 11 }{ 2 }\). Donc le sommet a pour coordonnées
            \begin{equation}
                \big( \frac{ 3 }{ 2 },\frac{ 11 }{2} \big).
            \end{equation}
    \end{enumerate}

\end{corrige}
