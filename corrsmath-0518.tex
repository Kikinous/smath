% This is part of Un soupçon de mathématique sans être agressif pour autant
% Copyright (c) 2013
%   Laurent Claessens
% See the file fdl-1.3.txt for copying conditions.

\begin{corrige}{smath-0518}

\begin{wrapfigure}{r}{8.0cm}
   \vspace{-0.5cm}        % à adapter.
   \centering
   \input{Fig_YBxSjFS.pstricks}
\end{wrapfigure}

    Il y a de nombreuses façons de procéder. En voici une particulièrement facile composée de deux morceaux de droite.
    \begin{enumerate}
        \item
            Cette fonction passe par les points \( (0;0)\) et \( A(-3;6)\), ce qui donne bien \( f(0)=0\) et \( f(-3)=6\).
        \item
            Elle est décroissante sur \( \mathopen[ 0 ;2 \mathclose]\). Et même sur \( \mathopen[ -3 ;2 \mathclose]\), mais ce n'est pas important.
        \item
            De tous les points d'abscisses dans \( \mathopen[ 2 ; 10 \mathclose]\), le plus haut est celui d'abscisse \( 10\), c'est le point \( C\).
    \end{enumerate}

\end{corrige}
