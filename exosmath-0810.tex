% This is part of Un soupçon de mathématique sans être agressif pour autant
% Copyright (c) 2014
%   Laurent Claessens
% See the file fdl-1.3.txt for copying conditions.

\begin{exercice}[Le tricercle de Mohr\cite{NRHooXFvgpp4}]\label{exosmath-0810}

La figure ci-dessus est constituée de trois demi-cercles dont les centres appartiennent au segment $[AB]$.

RKSooSVlhiF


\begin{enumerate}
    \item
Réalise cette figure pour $x = 3$. Dans ce cas-là, calcule la longueur de chacun des trois demi-cercles (tu donneras la valeur arrondie des résultats au dixième).  Quel est alors le périmètre de la figure bleue délimitée par les trois demi-cercles ?
\item
 Même question pour
 $x = 8$.
 \item
 Que remarques-tu ?
\item
 Exprime, en fonction de $x$ et de \( \pi\), la longueur de chacun des trois demi-cercles.
\item
 Déduis-en une expression du périmètre de la figure bleue en fonction de $x$ et de \( \pi\).  Que peux-tu dire de ce périmètre ? Justifie.
\item
 Utilise le résultat de la question précédente pour déterminer le périmètre de la figure bleue lorsque $x = 1$, puis pour $x = 5$ et enfin pour $x = 8,7$.
\end{enumerate}

\corrref{smath-0810}
\end{exercice}
