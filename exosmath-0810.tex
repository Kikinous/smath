% This is part of Un soupçon de mathématique sans être agressif pour autant
% Copyright (c) 2014
%   Laurent Claessens
% See the file fdl-1.3.txt for copying conditions.

\begin{exercice}[Le tricercle de Mohr\cite{NRHooXFvgpp4}]\label{exosmath-0810}

    \begin{wrapfigure}[4]{r}{3.5cm}
   \vspace{-0.5cm}        % à adapter.
   \centering
   \input{Fig_RKSooSVlhiF.pstricks}
\end{wrapfigure}

Cette figure est constituée de trois demi-cercles dont les centres appartiennent au segment $[AB]$.

\begin{enumerate}
    \item
Réaliser cette figure pour $x = 3$ et calculer la longueur de chacun des trois demi-cercles. Quel est alors le périmètre de la figure grise délimitée par les trois demi-cercles ?
\item
 Même question pour $x = 8$.  Que remarques-tu ?
\item
 Exprime, en fonction de $x$ et de \( \pi\), la longueur de chacun des trois demi-cercles.
\item
 Déduis-en une expression du périmètre de la figure grisée en fonction de $x$ et de \( \pi\).  Que peux-tu dire de ce périmètre ? Justifie.
\item
 Utilise le résultat de la question précédente pour déterminer le périmètre de la figure bleue lorsque $x = 1$, puis pour $x = 5$ et enfin pour $x = 8,7$.
\end{enumerate}

\corrref{smath-0810}
\end{exercice}
