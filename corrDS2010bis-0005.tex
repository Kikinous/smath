% This is part of Exercices de mathématique pour SVT
% Copyright (C) 2010
%   Laurent Claessens et Carlotta Donadello
% See the file fdl-1.3.txt for copying conditions.

\begin{corrige}{DS2010bis-0005}


	\begin{enumerate}
		\item
			Pour rappel, 
			\begin{equation}
				(-1)^n=\begin{cases}
					1	&	\text{si $n$ est pair}\\
					-1	&	 \text{si $n$ est impair.}
				\end{cases}
			\end{equation}
			Les premiers termes sont donc
			\begin{equation}
				\begin{aligned}[]
					u_1&=\frac{(-1)^1 1 }{2}=-\frac{ 1 }{2}\\
					u_2&=\frac{(-1)^2 2 }{3}=\frac{ 2 }{ 3 }\\
					u_3&=\frac{(-1)^33}{ 4 }=-\frac{ 3 }{ 4 }\\
					u_4&=\frac{(-1)^4 4}{ 5 }=\frac{ 4 }{ 5 }.
				\end{aligned}
			\end{equation}
		\item
			Nous avons
			\begin{equation}
				\left| \frac{(-1)^nn}{ n+1 } \right|=\frac{n}{ n+1 }.
			\end{equation}
			Or $\frac{n}{ n+1 }$ est inférieur à $1$ pour tout $n$ entier positif. Ce qui fait que $\frac{(-1)^nn}{ n+1 }$ est bornée entre $-1$ et $1$. 
		\item
			La suite est croissante entre $u_1$ et $u_2$, et décroissante entre $u_2$ et $u_3$. Plus généralement, les termes pairs seront positifs tandis que les termes impairs seront négatifs; elle n'arrête donc pas de monter et descendre au-dessus de zéro et en-dessous de zéro.
		\item
			La suite $(| u_n |)$ converge vers $1$. Pour le voir on peut calculer la limite de la fonction $f(x)=\frac{x}{ x+1 }$ lorsque $x$ tend vers $+\infty$.  
		\item
			\begin{equation}
				\begin{aligned}[]
					u_2&=\frac{ 2 }{3},	&u_1&=-\frac{1}{2},\\
					u_4&=\frac{ 4 }{5},	&u_3&=-\frac{ 3 }{ 4 },\\
					u_6&=\frac{ 6 }{7},	&u_5&=-\frac{ 5 }{ 6 }.
				\end{aligned}
			\end{equation}
		\item
			La suite des termes pairs est (rappel : pour tout $n$, le nombre $2n$ est pair et donc $(-1)^{2n}=1$)
			\begin{equation}
				u_{2n}=\frac{2n}{ 2n+1 }.
			\end{equation}
			Cette suite tend vers $1$.

			La suite des termes impairs par contre vaut
			\begin{equation}
				u_{2n+1}=-\frac{ 2n+1 }{ 2n+2 },
			\end{equation}
			et cette suite tend vers $-1$.

			Nous avons donc trouvé deux sous-suites de $(u_n)$ qui tendent vers des limites différentes. La suite $(u_n)$ elle-même ne converge donc pas.
	\end{enumerate}

\end{corrige}