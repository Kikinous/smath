% This is part of Un soupçon de mathématique sans être agressif pour autant
% Copyright (c) 2014
%   Laurent Claessens
% See the file fdl-1.3.txt for copying conditions.

\begin{exercice}\label{exo2smath-0005}

    L'affirmation suivante est fausse : « Dans un triangle \( ABC\), la médiatrice du côté \( [BC]\) passe par le sommet \( A\) ».
    \begin{enumerate}
        \item   \label{itemVEDWooDxgxnoi}
            Dessiner un contre-exemple à l'affirmation.
        \item   \label{itemVEDWooDxgxnoii}
            Tracer un triangle \( ABC\) pour lequel la médiatrice du côté \( [BC]\) passe par le sommet \( A\).
        \item
            Tracer, en laissant les traits de construction, le cercle circonscrit au triangle tracé pour la question \ref{itemVEDWooDxgxnoi} ou \ref{itemVEDWooDxgxnoii} au choix.
    \end{enumerate}

\corrref{2smath-0005}
\end{exercice}
