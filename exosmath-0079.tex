% This is part of Un soupçon de mathématique sans être agressif pour autant
% Copyright (c) 2012
%   Laurent Claessens
% See the file fdl-1.3.txt for copying conditions.

\begin{exercice}\label{exosmath-0079}

    Dans l'exercice suivant, lorsqu'on demande les «positions relatives» de deux droites ou plans, on demande si ils sont sécants ou parallèles, ou ni l'un ni l'autre.
    \begin{multicols}{2}
        \begin{enumerate}
            \item
        Donner une droite coplanaire à \( (BH)\) passant par \( E\).
    \item
        Donner une droite non coplanaire à \( (BH)\) passant par \( G\).
    \item
        Quelle est la position relative des droites \( (DC)\) et \( (EF)\) ?
    \item
        Quelle est la position relative des droites \( (DC)\) et \( (EF)\) ?
    \item
        Quelle est la position relative des plans \( (ABF)\) et \( (HGA)\) ?
    \item
        Quelle est la position relative des plans \( (AHC)\) et \( (EBG)\) ?
                
        \end{enumerate}
        \columnbreak
%The result is on figure \ref{LabelFigPositionRelativexpwkEJ}. % From file PositionRelativexpwkEJ
%\newcommand{\CaptionFigPositionRelativexpwkEJ}{<+Type your caption here+>}
        \begin{center}
\input{Fig_PositionRelativexpwkEJ.pstricks}
        \end{center}
    \end{multicols}

\corrref{smath-0079}
\end{exercice}
