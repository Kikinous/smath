% This is part of Un soupçon de mathématique sans être agressif pour autant
% Copyright (c) 2013
%   Laurent Claessens
% See the file fdl-1.3.txt for copying conditions.

\begin{exercice}\label{exosmath-0417}

    Un sondage portant sur \( 500\) personnes donne \( 31\%\) des voix au candidat J. et \( 37\%\) au candidat K. Le jour du scrutin, monsieur J. prend \(36.5\% \) et monsieur K. prend \( 36\%\). C'est donc monsieur J. qui remporte l'élection.
    \begin{enumerate}
        \item
            Est-ce que le candidat K. peut être étonné de n'obtenir que \( 36\%\) alors que les sondages lui donnaient \( 37\%\) ?
        \item
            En tant qu'expert statisticien, que pensez-vous de la différence entre les \( 31\%\) annoncés et les \( 36.5\%\) obtenus par monsieur J. ?
    \end{enumerate}

\corrref{smath-0417}
\end{exercice}
