% This is part of Un soupçon de mathématique sans être agressif pour autant
% Copyright (c) 2013
%   Laurent Claessens
% See the file fdl-1.3.txt for copying conditions.

\begin{exercice}\label{exosmath-0575}

    Une agence de sécurité nationale d'un certain pays lointain intercepte des courriers électroniques et en étudie la longueur (en caractères). Voici un extrait des résultats :
    \begin{equation*}
        \begin{array}[]{|c||c|c|c|c|c|c|c|c|c|c|c|}
            \hline
            \text{Nombre de caractères}&200&500&1000&1500&2000&2500&3000&3500&4000&4500&5000\\
            \hline\hline
            \text{Effectifs}&50&20&65&40&15&12&20&17&28&13&5\\
            \hline
            \text{FCC}&&&&&&&&&&&\\
            \hline
        \end{array}
    \end{equation*}
    \begin{enumerate}
        \item
            Combien de courriers sont répertoriés dans cet extrait ?
        \item
            Compléter la ligne des fréquences cumulées croissantes.
        \item
            En déduire les quartiles et la médiane de la série.
    \end{enumerate}


\corrref{smath-0575}
\end{exercice}
