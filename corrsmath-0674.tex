% This is part of Un soupçon de mathématique sans être agressif pour autant
% Copyright (c) 2014
%   Laurent Claessens
% See the file fdl-1.3.txt for copying conditions.

\begin{corrige}{smath-0674}

    \begin{enumerate}
        \item
            N'importe quelle fonction affine dont le coefficient directeur est \( -2\) fait l'affaire. Par exemple \( f(x)=-2x+7\), \( h(x)=-2x-12\), \( k(x)=-2x+1\), etc.
        \item
            Il y a moyen de simplifier par \( 3\) parce que tant \( 9x\) que \( 6b\) sont divisibles par \( 3\). Le résultat est :
            \begin{equation}
                \frac{ 9x+6b }{ 3 }=3x+2b
            \end{equation}
        \item
            La «distance horizontale» est de \( 10\) à \( 15\), donc \( 5\) et la «distance verticale» est de \( 20\) à \( 25\), donc \( 5\) aussi. La longueur \( AB\) vaut donc
            \begin{equation}
                AB=\sqrt{25+25}=\sqrt{50}=2\sqrt{5}.
            \end{equation}
        \item
            Il suffit de remplacer \( x\) par \( 3\) dans la fonction : 
            \begin{equation}
                f(3)=-15-10=-25,
            \end{equation}
            donc le point \( K(3,-25) \) est sur le graphe de \( f\). La réponse est \( y=3\).
        \item
            Il y a énormément de possibilités. Une des plus simples serait \( 0,1,1,1,1\). La médiane est \( 1\) (troisième valeur) tandis que la moyenne est un peu plus petite que \( 1\) vu qu'il y a un zéro et tout des \( 1\).
    \end{enumerate}
    <++>

\end{corrige}
