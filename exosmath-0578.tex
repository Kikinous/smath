% This is part of Un soupçon de mathématique sans être agressif pour autant
% Copyright (c) 2013
%   Laurent Claessens
% See the file fdl-1.3.txt for copying conditions.

\begin{exercice}\label{exosmath-0578}

\begin{fmpage}{0.9\linewidth}

    entrer \( xA\), \( yA\), \( xB\), \( yB\), \( xC\), \( yC\), \( x_D\), \( y_D\)

    \( e\) prend la valeur \ldots\ldots\ldots\ldots

    \( f\) prend la valeur \ldots\ldots\ldots\ldots
 
    \( g\) prend la valeur \ldots\ldots\ldots\ldots
    
    \( h\) prend la valeur \ldots\ldots\ldots\ldots

    si \( e=g\) et \( f=h\)   alors

    \hspace{0.7cm} afficher « \( ABCD\) est un parallélogramme » 

    sinon

    \hspace{0.7cm} afficher \ldots\ldots\ldots\ldots\ldots\ldots\ldots\ldots\ldots

\end{fmpage}


\corrref{smath-0578}
\end{exercice}
