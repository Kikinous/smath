% This is part of Un soupçon de mathématique sans être agressif pour autant
% Copyright (c) 2012-2013
%   Laurent Claessens
% See the file fdl-1.3.txt for copying conditions.

\begin{exercice}\label{exosmath-0189}

    Nous tirons une boule au hasard d'une urne en contenant \( 24\) dont \( 9\)  jaunes, \( 7\) rouges, \( 5\) vertes, et \( 3\) bleues. Nous regardons la couleur de la boule tirée.
    \begin{enumerate}
        \item
            Combien de résultats sont possibles ?
        \item
            Donner la probabilité des événements suivants :
            \begin{enumerate}
                \item
                    «La boule tirée est rouge».
                \item
                    «La boule tirée est rouge ou bleue» 
                \item
                    «La boule tirée n'est pas verte» 
                \item
                    «La boule tirée n'est pas noire» 
                \item
                    «La boule tirée n'est ni jaune ni rouge»
            \end{enumerate}
    \end{enumerate}

\corrref{smath-0189}
\end{exercice}
