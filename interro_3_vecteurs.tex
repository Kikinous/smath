% This is part of Un soupçon de mathématique sans être agressif pour autant
% Copyright (c) 2012-2014
%   Laurent Claessens
% See the file fdl-1.3.txt for copying conditions.

\vbox{Numéro 1.
\emph{Toutes les réponses doivent être justifiées par un calcul accompagné d'un raisonnement.}
\begin{enumerate}\item
Soient les points $A(4;-7)$, $ B(5,10)$ et $ C(1;-6)$. 
    \begin{enumerate}
    \item
    
    Calculer les coordonnées des vecteurs \( \vect{ AB }\) et \( \vect{ AB }+\vect{ BC }\). 

\item
    Donner les coordonnées du point \( X\) tel que \( \vect{ AX }=\vect{ BC }\) (méthode au choix)
    \end{enumerate}
    
    



    $\vect{ AB }=(1;17)$

    $\vect{ AB }+\vect{ BC }=(-3;1)$

    $X=(0;-23)$
    \item

    Soient les points $D(1;-3)$, $L(4;0)$ et $B(-4;7)$. Donner les coordonnées du point $K$ tel que $DLBK$ soit un parallélogramme (méthode au choix).
    

$K=(-7;4)$
\end{enumerate}
}
\vbox{Numéro 2.
\emph{Toutes les réponses doivent être justifiées par un calcul accompagné d'un raisonnement.}
\begin{enumerate}\item
Soient les points $A(8;-1)$, $ B(4,-7)$ et $ C(8;-5)$. 
    \begin{enumerate}
    \item
    
    Calculer les coordonnées des vecteurs \( \vect{ AB }\) et \( \vect{ AB }+\vect{ BC }\). 

\item
    Donner les coordonnées du point \( X\) tel que \( \vect{ AX }=\vect{ BC }\) (méthode au choix)
    \end{enumerate}
    
    



    $\vect{ AB }=(-4;-6)$

    $\vect{ AB }+\vect{ BC }=(0;-4)$

    $X=(12;1)$
    \item

    Soient les points $E(-10;2)$, $B(8;-5)$ et $D(3;10)$. Donner les coordonnées du point $F$ tel que $EBDF$ soit un parallélogramme (méthode au choix).
    

$F=(-15;17)$
\end{enumerate}
}
\vbox{Numéro 3.
\emph{Toutes les réponses doivent être justifiées par un calcul accompagné d'un raisonnement.}
\begin{enumerate}\item

    Soient les points $F(9;0)$, $L(6;3)$ et $E(1;6)$. Donner les coordonnées du point $M$ tel que $FLEM$ soit un parallélogramme (méthode au choix).
    

$M=(4;3)$\item
Soient les points $A(1;-5)$, $ B(7,-6)$ et $ C(1;-4)$. 
    \begin{enumerate}
    \item
    
    Calculer les coordonnées des vecteurs \( \vect{ AB }\) et \( \vect{ AB }+\vect{ BC }\). 

\item
    Donner les coordonnées du point \( X\) tel que \( \vect{ AX }=\vect{ BC }\) (méthode au choix)
    \end{enumerate}
    
    



    $\vect{ AB }=(6;-1)$

    $\vect{ AB }+\vect{ BC }=(0;1)$

    $X=(-5;-3)$
    
\end{enumerate}
}
\vbox{Numéro 4.
\emph{Toutes les réponses doivent être justifiées par un calcul accompagné d'un raisonnement.}
\begin{enumerate}\item
Soient les points $A(-8;9)$, $ B(-3,-7)$ et $ C(-9;-8)$. 
    \begin{enumerate}
    \item
    
    Calculer les coordonnées des vecteurs \( \vect{ AB }\) et \( \vect{ AB }+\vect{ BC }\). 

\item
    Donner les coordonnées du point \( X\) tel que \( \vect{ AX }=\vect{ BC }\) (méthode au choix)
    \end{enumerate}
    
    



    $\vect{ AB }=(5;-16)$

    $\vect{ AB }+\vect{ BC }=(-1;-17)$

    $X=(-14;8)$
    \item

    Soient les points $L(-7;7)$, $A(7;-8)$ et $F(-3;-8)$. Donner les coordonnées du point $D$ tel que $LAFD$ soit un parallélogramme (méthode au choix).
    

$D=(-17;7)$
\end{enumerate}
}
\vbox{Numéro 5.
\emph{Toutes les réponses doivent être justifiées par un calcul accompagné d'un raisonnement.}
\begin{enumerate}\item

    Soient les points $K(0;8)$, $A(0;-7)$ et $D(3;4)$. Donner les coordonnées du point $B$ tel que $KADB$ soit un parallélogramme (méthode au choix).
    

$B=(3;19)$\item
Soient les points $A(-4;3)$, $ B(8,-6)$ et $ C(-5;-7)$. 
    \begin{enumerate}
    \item
    
    Calculer les coordonnées des vecteurs \( \vect{ AB }\) et \( \vect{ AB }+\vect{ BC }\). 

\item
    Donner les coordonnées du point \( X\) tel que \( \vect{ AX }=\vect{ BC }\) (méthode au choix)
    \end{enumerate}
    
    



    $\vect{ AB }=(12;-9)$

    $\vect{ AB }+\vect{ BC }=(-1;-10)$

    $X=(-17;2)$
    
\end{enumerate}
}
\vbox{Numéro 6.
\emph{Toutes les réponses doivent être justifiées par un calcul accompagné d'un raisonnement.}
\begin{enumerate}\item

    Soient les points $F(10;-5)$, $M(-7;-5)$ et $B(5;-7)$. Donner les coordonnées du point $D$ tel que $FMBD$ soit un parallélogramme (méthode au choix).
    

$D=(22;-7)$\item
Soient les points $A(1;-5)$, $ B(9,10)$ et $ C(-6;3)$. 
    \begin{enumerate}
    \item
    
    Calculer les coordonnées des vecteurs \( \vect{ AB }\) et \( \vect{ AB }+\vect{ BC }\). 

\item
    Donner les coordonnées du point \( X\) tel que \( \vect{ AX }=\vect{ BC }\) (méthode au choix)
    \end{enumerate}
    
    



    $\vect{ AB }=(8;15)$

    $\vect{ AB }+\vect{ BC }=(-7;8)$

    $X=(-14;-12)$
    
\end{enumerate}
}
\vbox{Numéro 7.
\emph{Toutes les réponses doivent être justifiées par un calcul accompagné d'un raisonnement.}
\begin{enumerate}\item
Soient les points $A(-9;-2)$, $ B(1,10)$ et $ C(1;4)$. 
    \begin{enumerate}
    \item
    
    Calculer les coordonnées des vecteurs \( \vect{ AB }\) et \( \vect{ AB }+\vect{ BC }\). 

\item
    Donner les coordonnées du point \( X\) tel que \( \vect{ AX }=\vect{ BC }\) (méthode au choix)
    \end{enumerate}
    
    



    $\vect{ AB }=(10;12)$

    $\vect{ AB }+\vect{ BC }=(10;6)$

    $X=(-9;-8)$
    \item

    Soient les points $M(3;7)$, $B(-6;-5)$ et $A(1;10)$. Donner les coordonnées du point $L$ tel que $MBAL$ soit un parallélogramme (méthode au choix).
    

$L=(10;22)$
\end{enumerate}
}
\vbox{Numéro 8.
\emph{Toutes les réponses doivent être justifiées par un calcul accompagné d'un raisonnement.}
\begin{enumerate}\item

    Soient les points $A(-10;-7)$, $F(-5;0)$ et $D(3;4)$. Donner les coordonnées du point $E$ tel que $AFDE$ soit un parallélogramme (méthode au choix).
    

$E=(-2;-3)$\item
Soient les points $A(8;2)$, $ B(-5,-1)$ et $ C(4;-9)$. 
    \begin{enumerate}
    \item
    
    Calculer les coordonnées des vecteurs \( \vect{ AB }\) et \( \vect{ AB }+\vect{ BC }\). 

\item
    Donner les coordonnées du point \( X\) tel que \( \vect{ AX }=\vect{ BC }\) (méthode au choix)
    \end{enumerate}
    
    



    $\vect{ AB }=(-13;-3)$

    $\vect{ AB }+\vect{ BC }=(-4;-11)$

    $X=(17;-6)$
    
\end{enumerate}
}
\vbox{Numéro 9.
\emph{Toutes les réponses doivent être justifiées par un calcul accompagné d'un raisonnement.}
\begin{enumerate}\item

    Soient les points $B(10;0)$, $K(6;-7)$ et $D(10;-8)$. Donner les coordonnées du point $L$ tel que $BKDL$ soit un parallélogramme (méthode au choix).
    

$L=(14;-1)$\item
Soient les points $A(-1;4)$, $ B(-7,2)$ et $ C(5;5)$. 
    \begin{enumerate}
    \item
    
    Calculer les coordonnées des vecteurs \( \vect{ AB }\) et \( \vect{ AB }+\vect{ BC }\). 

\item
    Donner les coordonnées du point \( X\) tel que \( \vect{ AX }=\vect{ BC }\) (méthode au choix)
    \end{enumerate}
    
    



    $\vect{ AB }=(-6;-2)$

    $\vect{ AB }+\vect{ BC }=(6;1)$

    $X=(11;7)$
    
\end{enumerate}
}
\vbox{Numéro 10.
\emph{Toutes les réponses doivent être justifiées par un calcul accompagné d'un raisonnement.}
\begin{enumerate}\item
Soient les points $A(-2;-1)$, $ B(-10,-1)$ et $ C(-7;-7)$. 
    \begin{enumerate}
    \item
    
    Calculer les coordonnées des vecteurs \( \vect{ AB }\) et \( \vect{ AB }+\vect{ BC }\). 

\item
    Donner les coordonnées du point \( X\) tel que \( \vect{ AX }=\vect{ BC }\) (méthode au choix)
    \end{enumerate}
    
    



    $\vect{ AB }=(-8;0)$

    $\vect{ AB }+\vect{ BC }=(-5;-6)$

    $X=(1;-7)$
    \item

    Soient les points $A(2;4)$, $F(-9;-9)$ et $B(-3;3)$. Donner les coordonnées du point $L$ tel que $AFBL$ soit un parallélogramme (méthode au choix).
    

$L=(8;16)$
\end{enumerate}
}
\vbox{Numéro 11.
\emph{Toutes les réponses doivent être justifiées par un calcul accompagné d'un raisonnement.}
\begin{enumerate}\item

    Soient les points $F(-7;6)$, $D(-3;-5)$ et $M(-4;10)$. Donner les coordonnées du point $L$ tel que $FDML$ soit un parallélogramme (méthode au choix).
    

$L=(-8;21)$\item
Soient les points $A(2;-3)$, $ B(0,-6)$ et $ C(3;-2)$. 
    \begin{enumerate}
    \item
    
    Calculer les coordonnées des vecteurs \( \vect{ AB }\) et \( \vect{ AB }+\vect{ BC }\). 

\item
    Donner les coordonnées du point \( X\) tel que \( \vect{ AX }=\vect{ BC }\) (méthode au choix)
    \end{enumerate}
    
    



    $\vect{ AB }=(-2;-3)$

    $\vect{ AB }+\vect{ BC }=(1;1)$

    $X=(5;1)$
    
\end{enumerate}
}
\vbox{Numéro 12.
\emph{Toutes les réponses doivent être justifiées par un calcul accompagné d'un raisonnement.}
\begin{enumerate}\item

    Soient les points $F(9;3)$, $L(10;4)$ et $A(-5;-6)$. Donner les coordonnées du point $D$ tel que $FLAD$ soit un parallélogramme (méthode au choix).
    

$D=(-6;-7)$\item
Soient les points $A(10;2)$, $ B(7,-9)$ et $ C(-1;-8)$. 
    \begin{enumerate}
    \item
    
    Calculer les coordonnées des vecteurs \( \vect{ AB }\) et \( \vect{ AB }+\vect{ BC }\). 

\item
    Donner les coordonnées du point \( X\) tel que \( \vect{ AX }=\vect{ BC }\) (méthode au choix)
    \end{enumerate}
    
    



    $\vect{ AB }=(-3;-11)$

    $\vect{ AB }+\vect{ BC }=(-11;-10)$

    $X=(2;3)$
    
\end{enumerate}
}
\vbox{Numéro 13.
\emph{Toutes les réponses doivent être justifiées par un calcul accompagné d'un raisonnement.}
\begin{enumerate}\item
Soient les points $A(-1;-10)$, $ B(7,4)$ et $ C(-5;-10)$. 
    \begin{enumerate}
    \item
    
    Calculer les coordonnées des vecteurs \( \vect{ AB }\) et \( \vect{ AB }+\vect{ BC }\). 

\item
    Donner les coordonnées du point \( X\) tel que \( \vect{ AX }=\vect{ BC }\) (méthode au choix)
    \end{enumerate}
    
    



    $\vect{ AB }=(8;14)$

    $\vect{ AB }+\vect{ BC }=(-4;0)$

    $X=(-13;-24)$
    \item

    Soient les points $E(-10;-5)$, $L(8;-3)$ et $D(3;2)$. Donner les coordonnées du point $A$ tel que $ELDA$ soit un parallélogramme (méthode au choix).
    

$A=(-15;0)$
\end{enumerate}
}
\vbox{Numéro 14.
\emph{Toutes les réponses doivent être justifiées par un calcul accompagné d'un raisonnement.}
\begin{enumerate}\item
Soient les points $A(5;-10)$, $ B(-7,-10)$ et $ C(9;6)$. 
    \begin{enumerate}
    \item
    
    Calculer les coordonnées des vecteurs \( \vect{ AB }\) et \( \vect{ AB }+\vect{ BC }\). 

\item
    Donner les coordonnées du point \( X\) tel que \( \vect{ AX }=\vect{ BC }\) (méthode au choix)
    \end{enumerate}
    
    



    $\vect{ AB }=(-12;0)$

    $\vect{ AB }+\vect{ BC }=(4;16)$

    $X=(21;6)$
    \item

    Soient les points $F(-9;8)$, $K(-8;-3)$ et $L(-2;-2)$. Donner les coordonnées du point $A$ tel que $FKLA$ soit un parallélogramme (méthode au choix).
    

$A=(-3;9)$
\end{enumerate}
}
\vbox{Numéro 15.
\emph{Toutes les réponses doivent être justifiées par un calcul accompagné d'un raisonnement.}
\begin{enumerate}\item

    Soient les points $D(7;-6)$, $L(3;-7)$ et $K(0;-3)$. Donner les coordonnées du point $A$ tel que $DLKA$ soit un parallélogramme (méthode au choix).
    

$A=(4;-2)$\item
Soient les points $A(7;-9)$, $ B(-6,-2)$ et $ C(-7;10)$. 
    \begin{enumerate}
    \item
    
    Calculer les coordonnées des vecteurs \( \vect{ AB }\) et \( \vect{ AB }+\vect{ BC }\). 

\item
    Donner les coordonnées du point \( X\) tel que \( \vect{ AX }=\vect{ BC }\) (méthode au choix)
    \end{enumerate}
    
    



    $\vect{ AB }=(-13;7)$

    $\vect{ AB }+\vect{ BC }=(-14;19)$

    $X=(6;3)$
    
\end{enumerate}
}
\vbox{Numéro 16.
\emph{Toutes les réponses doivent être justifiées par un calcul accompagné d'un raisonnement.}
\begin{enumerate}\item

    Soient les points $F(-5;8)$, $M(6;5)$ et $D(5;10)$. Donner les coordonnées du point $K$ tel que $FMDK$ soit un parallélogramme (méthode au choix).
    

$K=(-6;13)$\item
Soient les points $A(9;-8)$, $ B(9,0)$ et $ C(0;-3)$. 
    \begin{enumerate}
    \item
    
    Calculer les coordonnées des vecteurs \( \vect{ AB }\) et \( \vect{ AB }+\vect{ BC }\). 

\item
    Donner les coordonnées du point \( X\) tel que \( \vect{ AX }=\vect{ BC }\) (méthode au choix)
    \end{enumerate}
    
    



    $\vect{ AB }=(0;8)$

    $\vect{ AB }+\vect{ BC }=(-9;5)$

    $X=(0;-11)$
    
\end{enumerate}
}
\vbox{Numéro 17.
\emph{Toutes les réponses doivent être justifiées par un calcul accompagné d'un raisonnement.}
\begin{enumerate}\item
Soient les points $A(-4;4)$, $ B(-6,-8)$ et $ C(10;6)$. 
    \begin{enumerate}
    \item
    
    Calculer les coordonnées des vecteurs \( \vect{ AB }\) et \( \vect{ AB }+\vect{ BC }\). 

\item
    Donner les coordonnées du point \( X\) tel que \( \vect{ AX }=\vect{ BC }\) (méthode au choix)
    \end{enumerate}
    
    



    $\vect{ AB }=(-2;-12)$

    $\vect{ AB }+\vect{ BC }=(14;2)$

    $X=(12;18)$
    \item

    Soient les points $K(-7;-10)$, $M(-7;5)$ et $E(9;2)$. Donner les coordonnées du point $D$ tel que $KMED$ soit un parallélogramme (méthode au choix).
    

$D=(9;-13)$
\end{enumerate}
}
\vbox{Numéro 18.
\emph{Toutes les réponses doivent être justifiées par un calcul accompagné d'un raisonnement.}
\begin{enumerate}\item

    Soient les points $L(4;-7)$, $F(5;0)$ et $A(2;8)$. Donner les coordonnées du point $M$ tel que $LFAM$ soit un parallélogramme (méthode au choix).
    

$M=(1;1)$\item
Soient les points $A(-10;-3)$, $ B(10,-2)$ et $ C(7;6)$. 
    \begin{enumerate}
    \item
    
    Calculer les coordonnées des vecteurs \( \vect{ AB }\) et \( \vect{ AB }+\vect{ BC }\). 

\item
    Donner les coordonnées du point \( X\) tel que \( \vect{ AX }=\vect{ BC }\) (méthode au choix)
    \end{enumerate}
    
    



    $\vect{ AB }=(20;1)$

    $\vect{ AB }+\vect{ BC }=(17;9)$

    $X=(-13;5)$
    
\end{enumerate}
}
\vbox{Numéro 19.
\emph{Toutes les réponses doivent être justifiées par un calcul accompagné d'un raisonnement.}
\begin{enumerate}\item

    Soient les points $B(1;-9)$, $E(8;0)$ et $L(10;-4)$. Donner les coordonnées du point $M$ tel que $BELM$ soit un parallélogramme (méthode au choix).
    

$M=(3;-13)$\item
Soient les points $A(2;-9)$, $ B(5,5)$ et $ C(4;-10)$. 
    \begin{enumerate}
    \item
    
    Calculer les coordonnées des vecteurs \( \vect{ AB }\) et \( \vect{ AB }+\vect{ BC }\). 

\item
    Donner les coordonnées du point \( X\) tel que \( \vect{ AX }=\vect{ BC }\) (méthode au choix)
    \end{enumerate}
    
    



    $\vect{ AB }=(3;14)$

    $\vect{ AB }+\vect{ BC }=(2;-1)$

    $X=(1;-24)$
    
\end{enumerate}
}
\vbox{Numéro 20.
\emph{Toutes les réponses doivent être justifiées par un calcul accompagné d'un raisonnement.}
\begin{enumerate}\item
Soient les points $A(9;3)$, $ B(8,-2)$ et $ C(-6;-3)$. 
    \begin{enumerate}
    \item
    
    Calculer les coordonnées des vecteurs \( \vect{ AB }\) et \( \vect{ AB }+\vect{ BC }\). 

\item
    Donner les coordonnées du point \( X\) tel que \( \vect{ AX }=\vect{ BC }\) (méthode au choix)
    \end{enumerate}
    
    



    $\vect{ AB }=(-1;-5)$

    $\vect{ AB }+\vect{ BC }=(-15;-6)$

    $X=(-5;2)$
    \item

    Soient les points $L(0;-6)$, $F(-6;9)$ et $E(7;-5)$. Donner les coordonnées du point $D$ tel que $LFED$ soit un parallélogramme (méthode au choix).
    

$D=(13;-20)$
\end{enumerate}
}
\vbox{Numéro 21.
\emph{Toutes les réponses doivent être justifiées par un calcul accompagné d'un raisonnement.}
\begin{enumerate}\item
Soient les points $A(9;-4)$, $ B(-8,9)$ et $ C(-10;9)$. 
    \begin{enumerate}
    \item
    
    Calculer les coordonnées des vecteurs \( \vect{ AB }\) et \( \vect{ AB }+\vect{ BC }\). 

\item
    Donner les coordonnées du point \( X\) tel que \( \vect{ AX }=\vect{ BC }\) (méthode au choix)
    \end{enumerate}
    
    



    $\vect{ AB }=(-17;13)$

    $\vect{ AB }+\vect{ BC }=(-19;13)$

    $X=(7;-4)$
    \item

    Soient les points $D(-9;1)$, $E(-6;-3)$ et $A(-9;-3)$. Donner les coordonnées du point $L$ tel que $DEAL$ soit un parallélogramme (méthode au choix).
    

$L=(-12;1)$
\end{enumerate}
}
\vbox{Numéro 22.
\emph{Toutes les réponses doivent être justifiées par un calcul accompagné d'un raisonnement.}
\begin{enumerate}\item
Soient les points $A(-4;-3)$, $ B(-8,-9)$ et $ C(9;5)$. 
    \begin{enumerate}
    \item
    
    Calculer les coordonnées des vecteurs \( \vect{ AB }\) et \( \vect{ AB }+\vect{ BC }\). 

\item
    Donner les coordonnées du point \( X\) tel que \( \vect{ AX }=\vect{ BC }\) (méthode au choix)
    \end{enumerate}
    
    



    $\vect{ AB }=(-4;-6)$

    $\vect{ AB }+\vect{ BC }=(13;8)$

    $X=(13;11)$
    \item

    Soient les points $F(1;-10)$, $B(-8;-10)$ et $L(-3;3)$. Donner les coordonnées du point $M$ tel que $FBLM$ soit un parallélogramme (méthode au choix).
    

$M=(6;3)$
\end{enumerate}
}
\vbox{Numéro 23.
\emph{Toutes les réponses doivent être justifiées par un calcul accompagné d'un raisonnement.}
\begin{enumerate}\item

    Soient les points $A(8;-8)$, $B(-6;5)$ et $L(5;7)$. Donner les coordonnées du point $E$ tel que $ABLE$ soit un parallélogramme (méthode au choix).
    

$E=(19;-6)$\item
Soient les points $A(1;3)$, $ B(-4,5)$ et $ C(10;-4)$. 
    \begin{enumerate}
    \item
    
    Calculer les coordonnées des vecteurs \( \vect{ AB }\) et \( \vect{ AB }+\vect{ BC }\). 

\item
    Donner les coordonnées du point \( X\) tel que \( \vect{ AX }=\vect{ BC }\) (méthode au choix)
    \end{enumerate}
    
    



    $\vect{ AB }=(-5;2)$

    $\vect{ AB }+\vect{ BC }=(9;-7)$

    $X=(15;-6)$
    
\end{enumerate}
}
\vbox{Numéro 24.
\emph{Toutes les réponses doivent être justifiées par un calcul accompagné d'un raisonnement.}
\begin{enumerate}\item

    Soient les points $F(-5;3)$, $K(1;3)$ et $A(7;6)$. Donner les coordonnées du point $L$ tel que $FKAL$ soit un parallélogramme (méthode au choix).
    

$L=(1;6)$\item
Soient les points $A(2;0)$, $ B(-2,5)$ et $ C(-6;-7)$. 
    \begin{enumerate}
    \item
    
    Calculer les coordonnées des vecteurs \( \vect{ AB }\) et \( \vect{ AB }+\vect{ BC }\). 

\item
    Donner les coordonnées du point \( X\) tel que \( \vect{ AX }=\vect{ BC }\) (méthode au choix)
    \end{enumerate}
    
    



    $\vect{ AB }=(-4;5)$

    $\vect{ AB }+\vect{ BC }=(-8;-7)$

    $X=(-2;-12)$
    
\end{enumerate}
}
\vbox{Numéro 25.
\emph{Toutes les réponses doivent être justifiées par un calcul accompagné d'un raisonnement.}
\begin{enumerate}\item
Soient les points $A(-8;-8)$, $ B(0,1)$ et $ C(9;3)$. 
    \begin{enumerate}
    \item
    
    Calculer les coordonnées des vecteurs \( \vect{ AB }\) et \( \vect{ AB }+\vect{ BC }\). 

\item
    Donner les coordonnées du point \( X\) tel que \( \vect{ AX }=\vect{ BC }\) (méthode au choix)
    \end{enumerate}
    
    



    $\vect{ AB }=(8;9)$

    $\vect{ AB }+\vect{ BC }=(17;11)$

    $X=(1;-6)$
    \item

    Soient les points $L(-7;4)$, $M(9;-1)$ et $F(-3;-8)$. Donner les coordonnées du point $A$ tel que $LMFA$ soit un parallélogramme (méthode au choix).
    

$A=(-19;-3)$
\end{enumerate}
}
\vbox{Numéro 26.
\emph{Toutes les réponses doivent être justifiées par un calcul accompagné d'un raisonnement.}
\begin{enumerate}\item
Soient les points $A(0;2)$, $ B(-10,-2)$ et $ C(2;-3)$. 
    \begin{enumerate}
    \item
    
    Calculer les coordonnées des vecteurs \( \vect{ AB }\) et \( \vect{ AB }+\vect{ BC }\). 

\item
    Donner les coordonnées du point \( X\) tel que \( \vect{ AX }=\vect{ BC }\) (méthode au choix)
    \end{enumerate}
    
    



    $\vect{ AB }=(-10;-4)$

    $\vect{ AB }+\vect{ BC }=(2;-5)$

    $X=(12;1)$
    \item

    Soient les points $A(-9;-9)$, $K(-2;9)$ et $E(-7;1)$. Donner les coordonnées du point $B$ tel que $AKEB$ soit un parallélogramme (méthode au choix).
    

$B=(-14;-17)$
\end{enumerate}
}
\vbox{Numéro 27.
\emph{Toutes les réponses doivent être justifiées par un calcul accompagné d'un raisonnement.}
\begin{enumerate}\item
Soient les points $A(6;7)$, $ B(9,7)$ et $ C(9;5)$. 
    \begin{enumerate}
    \item
    
    Calculer les coordonnées des vecteurs \( \vect{ AB }\) et \( \vect{ AB }+\vect{ BC }\). 

\item
    Donner les coordonnées du point \( X\) tel que \( \vect{ AX }=\vect{ BC }\) (méthode au choix)
    \end{enumerate}
    
    



    $\vect{ AB }=(3;0)$

    $\vect{ AB }+\vect{ BC }=(3;-2)$

    $X=(6;5)$
    \item

    Soient les points $A(-7;-6)$, $D(2;-2)$ et $M(10;-4)$. Donner les coordonnées du point $K$ tel que $ADMK$ soit un parallélogramme (méthode au choix).
    

$K=(1;-8)$
\end{enumerate}
}
\vbox{Numéro 28.
\emph{Toutes les réponses doivent être justifiées par un calcul accompagné d'un raisonnement.}
\begin{enumerate}\item

    Soient les points $F(-4;2)$, $E(2;-1)$ et $B(2;-10)$. Donner les coordonnées du point $L$ tel que $FEBL$ soit un parallélogramme (méthode au choix).
    

$L=(-4;-7)$\item
Soient les points $A(-5;2)$, $ B(-6,8)$ et $ C(-7;-1)$. 
    \begin{enumerate}
    \item
    
    Calculer les coordonnées des vecteurs \( \vect{ AB }\) et \( \vect{ AB }+\vect{ BC }\). 

\item
    Donner les coordonnées du point \( X\) tel que \( \vect{ AX }=\vect{ BC }\) (méthode au choix)
    \end{enumerate}
    
    



    $\vect{ AB }=(-1;6)$

    $\vect{ AB }+\vect{ BC }=(-2;-3)$

    $X=(-6;-7)$
    
\end{enumerate}
}
\vbox{Numéro 29.
\emph{Toutes les réponses doivent être justifiées par un calcul accompagné d'un raisonnement.}
\begin{enumerate}\item

    Soient les points $L(-4;-4)$, $A(-7;-6)$ et $K(9;-7)$. Donner les coordonnées du point $D$ tel que $LAKD$ soit un parallélogramme (méthode au choix).
    

$D=(12;-5)$\item
Soient les points $A(8;-2)$, $ B(9,5)$ et $ C(9;6)$. 
    \begin{enumerate}
    \item
    
    Calculer les coordonnées des vecteurs \( \vect{ AB }\) et \( \vect{ AB }+\vect{ BC }\). 

\item
    Donner les coordonnées du point \( X\) tel que \( \vect{ AX }=\vect{ BC }\) (méthode au choix)
    \end{enumerate}
    
    



    $\vect{ AB }=(1;7)$

    $\vect{ AB }+\vect{ BC }=(1;8)$

    $X=(8;-1)$
    
\end{enumerate}
}
\vbox{Numéro 30.
\emph{Toutes les réponses doivent être justifiées par un calcul accompagné d'un raisonnement.}
\begin{enumerate}\item
Soient les points $A(4;-9)$, $ B(10,4)$ et $ C(10;-8)$. 
    \begin{enumerate}
    \item
    
    Calculer les coordonnées des vecteurs \( \vect{ AB }\) et \( \vect{ AB }+\vect{ BC }\). 

\item
    Donner les coordonnées du point \( X\) tel que \( \vect{ AX }=\vect{ BC }\) (méthode au choix)
    \end{enumerate}
    
    



    $\vect{ AB }=(6;13)$

    $\vect{ AB }+\vect{ BC }=(6;1)$

    $X=(4;-21)$
    \item

    Soient les points $K(-5;4)$, $A(-4;-5)$ et $E(5;4)$. Donner les coordonnées du point $B$ tel que $KAEB$ soit un parallélogramme (méthode au choix).
    

$B=(4;13)$
\end{enumerate}
}
\vbox{Numéro 31.
\emph{Toutes les réponses doivent être justifiées par un calcul accompagné d'un raisonnement.}
\begin{enumerate}\item
Soient les points $A(-4;1)$, $ B(2,-9)$ et $ C(-3;-2)$. 
    \begin{enumerate}
    \item
    
    Calculer les coordonnées des vecteurs \( \vect{ AB }\) et \( \vect{ AB }+\vect{ BC }\). 

\item
    Donner les coordonnées du point \( X\) tel que \( \vect{ AX }=\vect{ BC }\) (méthode au choix)
    \end{enumerate}
    
    



    $\vect{ AB }=(6;-10)$

    $\vect{ AB }+\vect{ BC }=(1;-3)$

    $X=(-9;8)$
    \item

    Soient les points $B(-1;3)$, $K(-3;-9)$ et $A(-10;-6)$. Donner les coordonnées du point $D$ tel que $BKAD$ soit un parallélogramme (méthode au choix).
    

$D=(-8;6)$
\end{enumerate}
}
\vbox{Numéro 32.
\emph{Toutes les réponses doivent être justifiées par un calcul accompagné d'un raisonnement.}
\begin{enumerate}\item

    Soient les points $A(-8;8)$, $L(9;3)$ et $D(3;-6)$. Donner les coordonnées du point $M$ tel que $ALDM$ soit un parallélogramme (méthode au choix).
    

$M=(-14;-1)$\item
Soient les points $A(9;0)$, $ B(-1,4)$ et $ C(6;-4)$. 
    \begin{enumerate}
    \item
    
    Calculer les coordonnées des vecteurs \( \vect{ AB }\) et \( \vect{ AB }+\vect{ BC }\). 

\item
    Donner les coordonnées du point \( X\) tel que \( \vect{ AX }=\vect{ BC }\) (méthode au choix)
    \end{enumerate}
    
    



    $\vect{ AB }=(-10;4)$

    $\vect{ AB }+\vect{ BC }=(-3;-4)$

    $X=(16;-8)$
    
\end{enumerate}
}
\vbox{Numéro 33.
\emph{Toutes les réponses doivent être justifiées par un calcul accompagné d'un raisonnement.}
\begin{enumerate}\item
Soient les points $A(1;-9)$, $ B(3,6)$ et $ C(8;6)$. 
    \begin{enumerate}
    \item
    
    Calculer les coordonnées des vecteurs \( \vect{ AB }\) et \( \vect{ AB }+\vect{ BC }\). 

\item
    Donner les coordonnées du point \( X\) tel que \( \vect{ AX }=\vect{ BC }\) (méthode au choix)
    \end{enumerate}
    
    



    $\vect{ AB }=(2;15)$

    $\vect{ AB }+\vect{ BC }=(7;15)$

    $X=(6;-9)$
    \item

    Soient les points $A(6;-5)$, $D(7;-7)$ et $B(-2;-7)$. Donner les coordonnées du point $M$ tel que $ADBM$ soit un parallélogramme (méthode au choix).
    

$M=(-3;-5)$
\end{enumerate}
}
\vbox{Numéro 34.
\emph{Toutes les réponses doivent être justifiées par un calcul accompagné d'un raisonnement.}
\begin{enumerate}\item
Soient les points $A(-1;1)$, $ B(0,-7)$ et $ C(5;-1)$. 
    \begin{enumerate}
    \item
    
    Calculer les coordonnées des vecteurs \( \vect{ AB }\) et \( \vect{ AB }+\vect{ BC }\). 

\item
    Donner les coordonnées du point \( X\) tel que \( \vect{ AX }=\vect{ BC }\) (méthode au choix)
    \end{enumerate}
    
    



    $\vect{ AB }=(1;-8)$

    $\vect{ AB }+\vect{ BC }=(6;-2)$

    $X=(4;7)$
    \item

    Soient les points $F(-1;3)$, $M(2;8)$ et $E(0;-3)$. Donner les coordonnées du point $K$ tel que $FMEK$ soit un parallélogramme (méthode au choix).
    

$K=(-3;-8)$
\end{enumerate}
}
\vbox{Numéro 35.
\emph{Toutes les réponses doivent être justifiées par un calcul accompagné d'un raisonnement.}
\begin{enumerate}\item

    Soient les points $F(9;-4)$, $L(5;-5)$ et $E(-3;-9)$. Donner les coordonnées du point $D$ tel que $FLED$ soit un parallélogramme (méthode au choix).
    

$D=(1;-8)$\item
Soient les points $A(-6;2)$, $ B(-5,-3)$ et $ C(-6;-1)$. 
    \begin{enumerate}
    \item
    
    Calculer les coordonnées des vecteurs \( \vect{ AB }\) et \( \vect{ AB }+\vect{ BC }\). 

\item
    Donner les coordonnées du point \( X\) tel que \( \vect{ AX }=\vect{ BC }\) (méthode au choix)
    \end{enumerate}
    
    



    $\vect{ AB }=(1;-5)$

    $\vect{ AB }+\vect{ BC }=(0;-3)$

    $X=(-7;4)$
    
\end{enumerate}
}
\vbox{Numéro 36.
\emph{Toutes les réponses doivent être justifiées par un calcul accompagné d'un raisonnement.}
\begin{enumerate}\item
Soient les points $A(3;-5)$, $ B(7,-10)$ et $ C(-9;6)$. 
    \begin{enumerate}
    \item
    
    Calculer les coordonnées des vecteurs \( \vect{ AB }\) et \( \vect{ AB }+\vect{ BC }\). 

\item
    Donner les coordonnées du point \( X\) tel que \( \vect{ AX }=\vect{ BC }\) (méthode au choix)
    \end{enumerate}
    
    



    $\vect{ AB }=(4;-5)$

    $\vect{ AB }+\vect{ BC }=(-12;11)$

    $X=(-13;11)$
    \item

    Soient les points $M(10;-8)$, $F(-5;-4)$ et $L(0;6)$. Donner les coordonnées du point $K$ tel que $MFLK$ soit un parallélogramme (méthode au choix).
    

$K=(15;2)$
\end{enumerate}
}
\vbox{Numéro 37.
\emph{Toutes les réponses doivent être justifiées par un calcul accompagné d'un raisonnement.}
\begin{enumerate}\item

    Soient les points $F(10;-3)$, $B(4;-7)$ et $M(10;1)$. Donner les coordonnées du point $L$ tel que $FBML$ soit un parallélogramme (méthode au choix).
    

$L=(16;5)$\item
Soient les points $A(-1;9)$, $ B(1,6)$ et $ C(0;-1)$. 
    \begin{enumerate}
    \item
    
    Calculer les coordonnées des vecteurs \( \vect{ AB }\) et \( \vect{ AB }+\vect{ BC }\). 

\item
    Donner les coordonnées du point \( X\) tel que \( \vect{ AX }=\vect{ BC }\) (méthode au choix)
    \end{enumerate}
    
    



    $\vect{ AB }=(2;-3)$

    $\vect{ AB }+\vect{ BC }=(1;-10)$

    $X=(-2;2)$
    
\end{enumerate}
}
\vbox{Numéro 38.
\emph{Toutes les réponses doivent être justifiées par un calcul accompagné d'un raisonnement.}
\begin{enumerate}\item
Soient les points $A(-1;4)$, $ B(1,-4)$ et $ C(4;-6)$. 
    \begin{enumerate}
    \item
    
    Calculer les coordonnées des vecteurs \( \vect{ AB }\) et \( \vect{ AB }+\vect{ BC }\). 

\item
    Donner les coordonnées du point \( X\) tel que \( \vect{ AX }=\vect{ BC }\) (méthode au choix)
    \end{enumerate}
    
    



    $\vect{ AB }=(2;-8)$

    $\vect{ AB }+\vect{ BC }=(5;-10)$

    $X=(2;2)$
    \item

    Soient les points $K(-5;6)$, $L(-6;-2)$ et $E(-5;-1)$. Donner les coordonnées du point $M$ tel que $KLEM$ soit un parallélogramme (méthode au choix).
    

$M=(-4;7)$
\end{enumerate}
}
\vbox{Numéro 39.
\emph{Toutes les réponses doivent être justifiées par un calcul accompagné d'un raisonnement.}
\begin{enumerate}\item

    Soient les points $E(-4;9)$, $L(-5;-2)$ et $B(-6;10)$. Donner les coordonnées du point $D$ tel que $ELBD$ soit un parallélogramme (méthode au choix).
    

$D=(-5;21)$\item
Soient les points $A(6;-8)$, $ B(10,10)$ et $ C(1;1)$. 
    \begin{enumerate}
    \item
    
    Calculer les coordonnées des vecteurs \( \vect{ AB }\) et \( \vect{ AB }+\vect{ BC }\). 

\item
    Donner les coordonnées du point \( X\) tel que \( \vect{ AX }=\vect{ BC }\) (méthode au choix)
    \end{enumerate}
    
    



    $\vect{ AB }=(4;18)$

    $\vect{ AB }+\vect{ BC }=(-5;9)$

    $X=(-3;-17)$
    
\end{enumerate}
}
\vbox{Numéro 40.
\emph{Toutes les réponses doivent être justifiées par un calcul accompagné d'un raisonnement.}
\begin{enumerate}\item
Soient les points $A(-7;4)$, $ B(4,-10)$ et $ C(-2;-1)$. 
    \begin{enumerate}
    \item
    
    Calculer les coordonnées des vecteurs \( \vect{ AB }\) et \( \vect{ AB }+\vect{ BC }\). 

\item
    Donner les coordonnées du point \( X\) tel que \( \vect{ AX }=\vect{ BC }\) (méthode au choix)
    \end{enumerate}
    
    



    $\vect{ AB }=(11;-14)$

    $\vect{ AB }+\vect{ BC }=(5;-5)$

    $X=(-13;13)$
    \item

    Soient les points $F(8;9)$, $K(-9;-7)$ et $D(-8;7)$. Donner les coordonnées du point $B$ tel que $FKDB$ soit un parallélogramme (méthode au choix).
    

$B=(9;23)$
\end{enumerate}
}
\vbox{Numéro 41.
\emph{Toutes les réponses doivent être justifiées par un calcul accompagné d'un raisonnement.}
\begin{enumerate}\item

    Soient les points $A(-10;9)$, $F(4;-6)$ et $E(-10;0)$. Donner les coordonnées du point $M$ tel que $AFEM$ soit un parallélogramme (méthode au choix).
    

$M=(-24;15)$\item
Soient les points $A(-1;1)$, $ B(6,-8)$ et $ C(-4;-4)$. 
    \begin{enumerate}
    \item
    
    Calculer les coordonnées des vecteurs \( \vect{ AB }\) et \( \vect{ AB }+\vect{ BC }\). 

\item
    Donner les coordonnées du point \( X\) tel que \( \vect{ AX }=\vect{ BC }\) (méthode au choix)
    \end{enumerate}
    
    



    $\vect{ AB }=(7;-9)$

    $\vect{ AB }+\vect{ BC }=(-3;-5)$

    $X=(-11;5)$
    
\end{enumerate}
}
\vbox{Numéro 42.
\emph{Toutes les réponses doivent être justifiées par un calcul accompagné d'un raisonnement.}
\begin{enumerate}\item

    Soient les points $M(-8;5)$, $E(-5;10)$ et $L(-6;1)$. Donner les coordonnées du point $B$ tel que $MELB$ soit un parallélogramme (méthode au choix).
    

$B=(-9;-4)$\item
Soient les points $A(0;-5)$, $ B(-6,8)$ et $ C(-3;-1)$. 
    \begin{enumerate}
    \item
    
    Calculer les coordonnées des vecteurs \( \vect{ AB }\) et \( \vect{ AB }+\vect{ BC }\). 

\item
    Donner les coordonnées du point \( X\) tel que \( \vect{ AX }=\vect{ BC }\) (méthode au choix)
    \end{enumerate}
    
    



    $\vect{ AB }=(-6;13)$

    $\vect{ AB }+\vect{ BC }=(-3;4)$

    $X=(3;-14)$
    
\end{enumerate}
}
\vbox{Numéro 43.
\emph{Toutes les réponses doivent être justifiées par un calcul accompagné d'un raisonnement.}
\begin{enumerate}\item

    Soient les points $E(-2;-6)$, $D(-3;6)$ et $A(1;7)$. Donner les coordonnées du point $B$ tel que $EDAB$ soit un parallélogramme (méthode au choix).
    

$B=(2;-5)$\item
Soient les points $A(-1;-5)$, $ B(0,-3)$ et $ C(-4;0)$. 
    \begin{enumerate}
    \item
    
    Calculer les coordonnées des vecteurs \( \vect{ AB }\) et \( \vect{ AB }+\vect{ BC }\). 

\item
    Donner les coordonnées du point \( X\) tel que \( \vect{ AX }=\vect{ BC }\) (méthode au choix)
    \end{enumerate}
    
    



    $\vect{ AB }=(1;2)$

    $\vect{ AB }+\vect{ BC }=(-3;5)$

    $X=(-5;-2)$
    
\end{enumerate}
}
\vbox{Numéro 44.
\emph{Toutes les réponses doivent être justifiées par un calcul accompagné d'un raisonnement.}
\begin{enumerate}\item
Soient les points $A(9;5)$, $ B(-9,-8)$ et $ C(6;-7)$. 
    \begin{enumerate}
    \item
    
    Calculer les coordonnées des vecteurs \( \vect{ AB }\) et \( \vect{ AB }+\vect{ BC }\). 

\item
    Donner les coordonnées du point \( X\) tel que \( \vect{ AX }=\vect{ BC }\) (méthode au choix)
    \end{enumerate}
    
    



    $\vect{ AB }=(-18;-13)$

    $\vect{ AB }+\vect{ BC }=(-3;-12)$

    $X=(24;6)$
    \item

    Soient les points $L(7;9)$, $A(-7;9)$ et $K(10;-8)$. Donner les coordonnées du point $E$ tel que $LAKE$ soit un parallélogramme (méthode au choix).
    

$E=(24;-8)$
\end{enumerate}
}
\vbox{Numéro 45.
\emph{Toutes les réponses doivent être justifiées par un calcul accompagné d'un raisonnement.}
\begin{enumerate}\item
Soient les points $A(2;-10)$, $ B(-8,7)$ et $ C(-7;5)$. 
    \begin{enumerate}
    \item
    
    Calculer les coordonnées des vecteurs \( \vect{ AB }\) et \( \vect{ AB }+\vect{ BC }\). 

\item
    Donner les coordonnées du point \( X\) tel que \( \vect{ AX }=\vect{ BC }\) (méthode au choix)
    \end{enumerate}
    
    



    $\vect{ AB }=(-10;17)$

    $\vect{ AB }+\vect{ BC }=(-9;15)$

    $X=(3;-12)$
    \item

    Soient les points $L(-5;-5)$, $F(-10;3)$ et $K(-10;9)$. Donner les coordonnées du point $B$ tel que $LFKB$ soit un parallélogramme (méthode au choix).
    

$B=(-5;1)$
\end{enumerate}
}
\vbox{Numéro 46.
\emph{Toutes les réponses doivent être justifiées par un calcul accompagné d'un raisonnement.}
\begin{enumerate}\item

    Soient les points $M(10;3)$, $B(-3;1)$ et $K(-10;-5)$. Donner les coordonnées du point $E$ tel que $MBKE$ soit un parallélogramme (méthode au choix).
    

$E=(3;-3)$\item
Soient les points $A(4;-6)$, $ B(-2,4)$ et $ C(8;-10)$. 
    \begin{enumerate}
    \item
    
    Calculer les coordonnées des vecteurs \( \vect{ AB }\) et \( \vect{ AB }+\vect{ BC }\). 

\item
    Donner les coordonnées du point \( X\) tel que \( \vect{ AX }=\vect{ BC }\) (méthode au choix)
    \end{enumerate}
    
    



    $\vect{ AB }=(-6;10)$

    $\vect{ AB }+\vect{ BC }=(4;-4)$

    $X=(14;-20)$
    
\end{enumerate}
}
\vbox{Numéro 47.
\emph{Toutes les réponses doivent être justifiées par un calcul accompagné d'un raisonnement.}
\begin{enumerate}\item

    Soient les points $L(-4;6)$, $F(4;9)$ et $M(-10;-3)$. Donner les coordonnées du point $D$ tel que $LFMD$ soit un parallélogramme (méthode au choix).
    

$D=(-18;-6)$\item
Soient les points $A(-6;-1)$, $ B(10,-2)$ et $ C(-8;-1)$. 
    \begin{enumerate}
    \item
    
    Calculer les coordonnées des vecteurs \( \vect{ AB }\) et \( \vect{ AB }+\vect{ BC }\). 

\item
    Donner les coordonnées du point \( X\) tel que \( \vect{ AX }=\vect{ BC }\) (méthode au choix)
    \end{enumerate}
    
    



    $\vect{ AB }=(16;-1)$

    $\vect{ AB }+\vect{ BC }=(-2;0)$

    $X=(-24;0)$
    
\end{enumerate}
}
\vbox{Numéro 48.
\emph{Toutes les réponses doivent être justifiées par un calcul accompagné d'un raisonnement.}
\begin{enumerate}\item

    Soient les points $D(-9;9)$, $F(7;3)$ et $B(6;10)$. Donner les coordonnées du point $E$ tel que $DFBE$ soit un parallélogramme (méthode au choix).
    

$E=(-10;16)$\item
Soient les points $A(4;-1)$, $ B(-10,7)$ et $ C(10;5)$. 
    \begin{enumerate}
    \item
    
    Calculer les coordonnées des vecteurs \( \vect{ AB }\) et \( \vect{ AB }+\vect{ BC }\). 

\item
    Donner les coordonnées du point \( X\) tel que \( \vect{ AX }=\vect{ BC }\) (méthode au choix)
    \end{enumerate}
    
    



    $\vect{ AB }=(-14;8)$

    $\vect{ AB }+\vect{ BC }=(6;6)$

    $X=(24;-3)$
    
\end{enumerate}
}
\vbox{Numéro 49.
\emph{Toutes les réponses doivent être justifiées par un calcul accompagné d'un raisonnement.}
\begin{enumerate}\item

    Soient les points $D(-4;-9)$, $L(1;6)$ et $K(-4;-4)$. Donner les coordonnées du point $E$ tel que $DLKE$ soit un parallélogramme (méthode au choix).
    

$E=(-9;-19)$\item
Soient les points $A(6;-2)$, $ B(2,-6)$ et $ C(-6;-1)$. 
    \begin{enumerate}
    \item
    
    Calculer les coordonnées des vecteurs \( \vect{ AB }\) et \( \vect{ AB }+\vect{ BC }\). 

\item
    Donner les coordonnées du point \( X\) tel que \( \vect{ AX }=\vect{ BC }\) (méthode au choix)
    \end{enumerate}
    
    



    $\vect{ AB }=(-4;-4)$

    $\vect{ AB }+\vect{ BC }=(-12;1)$

    $X=(-2;3)$
    
\end{enumerate}
}
\vbox{Numéro 50.
\emph{Toutes les réponses doivent être justifiées par un calcul accompagné d'un raisonnement.}
\begin{enumerate}\item

    Soient les points $F(-10;2)$, $B(5;3)$ et $L(-8;-10)$. Donner les coordonnées du point $E$ tel que $FBLE$ soit un parallélogramme (méthode au choix).
    

$E=(-23;-11)$\item
Soient les points $A(10;-8)$, $ B(3,-2)$ et $ C(0;-8)$. 
    \begin{enumerate}
    \item
    
    Calculer les coordonnées des vecteurs \( \vect{ AB }\) et \( \vect{ AB }+\vect{ BC }\). 

\item
    Donner les coordonnées du point \( X\) tel que \( \vect{ AX }=\vect{ BC }\) (méthode au choix)
    \end{enumerate}
    
    



    $\vect{ AB }=(-7;6)$

    $\vect{ AB }+\vect{ BC }=(-10;0)$

    $X=(7;-14)$
    
\end{enumerate}
}
\vbox{Numéro 51.
\emph{Toutes les réponses doivent être justifiées par un calcul accompagné d'un raisonnement.}
\begin{enumerate}\item

    Soient les points $E(-6;10)$, $K(7;-1)$ et $A(1;-2)$. Donner les coordonnées du point $L$ tel que $EKAL$ soit un parallélogramme (méthode au choix).
    

$L=(-12;9)$\item
Soient les points $A(-7;1)$, $ B(6,5)$ et $ C(2;4)$. 
    \begin{enumerate}
    \item
    
    Calculer les coordonnées des vecteurs \( \vect{ AB }\) et \( \vect{ AB }+\vect{ BC }\). 

\item
    Donner les coordonnées du point \( X\) tel que \( \vect{ AX }=\vect{ BC }\) (méthode au choix)
    \end{enumerate}
    
    



    $\vect{ AB }=(13;4)$

    $\vect{ AB }+\vect{ BC }=(9;3)$

    $X=(-11;0)$
    
\end{enumerate}
}
\vbox{Numéro 52.
\emph{Toutes les réponses doivent être justifiées par un calcul accompagné d'un raisonnement.}
\begin{enumerate}\item
Soient les points $A(6;-5)$, $ B(-4,6)$ et $ C(-4;-6)$. 
    \begin{enumerate}
    \item
    
    Calculer les coordonnées des vecteurs \( \vect{ AB }\) et \( \vect{ AB }+\vect{ BC }\). 

\item
    Donner les coordonnées du point \( X\) tel que \( \vect{ AX }=\vect{ BC }\) (méthode au choix)
    \end{enumerate}
    
    



    $\vect{ AB }=(-10;11)$

    $\vect{ AB }+\vect{ BC }=(-10;-1)$

    $X=(6;-17)$
    \item

    Soient les points $M(-8;-3)$, $K(9;-10)$ et $D(-4;-6)$. Donner les coordonnées du point $B$ tel que $MKDB$ soit un parallélogramme (méthode au choix).
    

$B=(-21;1)$
\end{enumerate}
}
\vbox{Numéro 53.
\emph{Toutes les réponses doivent être justifiées par un calcul accompagné d'un raisonnement.}
\begin{enumerate}\item

    Soient les points $D(8;9)$, $A(-2;2)$ et $B(1;-8)$. Donner les coordonnées du point $M$ tel que $DABM$ soit un parallélogramme (méthode au choix).
    

$M=(11;-1)$\item
Soient les points $A(-5;6)$, $ B(-2,0)$ et $ C(4;10)$. 
    \begin{enumerate}
    \item
    
    Calculer les coordonnées des vecteurs \( \vect{ AB }\) et \( \vect{ AB }+\vect{ BC }\). 

\item
    Donner les coordonnées du point \( X\) tel que \( \vect{ AX }=\vect{ BC }\) (méthode au choix)
    \end{enumerate}
    
    



    $\vect{ AB }=(3;-6)$

    $\vect{ AB }+\vect{ BC }=(9;4)$

    $X=(1;16)$
    
\end{enumerate}
}
\vbox{Numéro 54.
\emph{Toutes les réponses doivent être justifiées par un calcul accompagné d'un raisonnement.}
\begin{enumerate}\item

    Soient les points $A(4;-6)$, $B(-5;3)$ et $K(-3;-8)$. Donner les coordonnées du point $D$ tel que $ABKD$ soit un parallélogramme (méthode au choix).
    

$D=(6;-17)$\item
Soient les points $A(1;3)$, $ B(0,6)$ et $ C(3;-4)$. 
    \begin{enumerate}
    \item
    
    Calculer les coordonnées des vecteurs \( \vect{ AB }\) et \( \vect{ AB }+\vect{ BC }\). 

\item
    Donner les coordonnées du point \( X\) tel que \( \vect{ AX }=\vect{ BC }\) (méthode au choix)
    \end{enumerate}
    
    



    $\vect{ AB }=(-1;3)$

    $\vect{ AB }+\vect{ BC }=(2;-7)$

    $X=(4;-7)$
    
\end{enumerate}
}
\vbox{Numéro 55.
\emph{Toutes les réponses doivent être justifiées par un calcul accompagné d'un raisonnement.}
\begin{enumerate}\item
Soient les points $A(-7;7)$, $ B(-2,-8)$ et $ C(-10;-7)$. 
    \begin{enumerate}
    \item
    
    Calculer les coordonnées des vecteurs \( \vect{ AB }\) et \( \vect{ AB }+\vect{ BC }\). 

\item
    Donner les coordonnées du point \( X\) tel que \( \vect{ AX }=\vect{ BC }\) (méthode au choix)
    \end{enumerate}
    
    



    $\vect{ AB }=(5;-15)$

    $\vect{ AB }+\vect{ BC }=(-3;-14)$

    $X=(-15;8)$
    \item

    Soient les points $M(-8;1)$, $B(-5;9)$ et $L(3;9)$. Donner les coordonnées du point $E$ tel que $MBLE$ soit un parallélogramme (méthode au choix).
    

$E=(0;1)$
\end{enumerate}
}
\vbox{Numéro 56.
\emph{Toutes les réponses doivent être justifiées par un calcul accompagné d'un raisonnement.}
\begin{enumerate}\item
Soient les points $A(-1;0)$, $ B(-6,8)$ et $ C(4;-10)$. 
    \begin{enumerate}
    \item
    
    Calculer les coordonnées des vecteurs \( \vect{ AB }\) et \( \vect{ AB }+\vect{ BC }\). 

\item
    Donner les coordonnées du point \( X\) tel que \( \vect{ AX }=\vect{ BC }\) (méthode au choix)
    \end{enumerate}
    
    



    $\vect{ AB }=(-5;8)$

    $\vect{ AB }+\vect{ BC }=(5;-10)$

    $X=(9;-18)$
    \item

    Soient les points $D(1;-7)$, $E(6;2)$ et $B(-4;-9)$. Donner les coordonnées du point $F$ tel que $DEBF$ soit un parallélogramme (méthode au choix).
    

$F=(-9;-18)$
\end{enumerate}
}
\vbox{Numéro 57.
\emph{Toutes les réponses doivent être justifiées par un calcul accompagné d'un raisonnement.}
\begin{enumerate}\item

    Soient les points $E(-2;-4)$, $B(-6;-10)$ et $A(-6;-7)$. Donner les coordonnées du point $L$ tel que $EBAL$ soit un parallélogramme (méthode au choix).
    

$L=(-2;-1)$\item
Soient les points $A(-8;4)$, $ B(8,-7)$ et $ C(8;6)$. 
    \begin{enumerate}
    \item
    
    Calculer les coordonnées des vecteurs \( \vect{ AB }\) et \( \vect{ AB }+\vect{ BC }\). 

\item
    Donner les coordonnées du point \( X\) tel que \( \vect{ AX }=\vect{ BC }\) (méthode au choix)
    \end{enumerate}
    
    



    $\vect{ AB }=(16;-11)$

    $\vect{ AB }+\vect{ BC }=(16;2)$

    $X=(-8;17)$
    
\end{enumerate}
}
\vbox{Numéro 58.
\emph{Toutes les réponses doivent être justifiées par un calcul accompagné d'un raisonnement.}
\begin{enumerate}\item

    Soient les points $L(1;-4)$, $D(0;8)$ et $K(7;-9)$. Donner les coordonnées du point $B$ tel que $LDKB$ soit un parallélogramme (méthode au choix).
    

$B=(8;-21)$\item
Soient les points $A(2;-6)$, $ B(-3,3)$ et $ C(-8;8)$. 
    \begin{enumerate}
    \item
    
    Calculer les coordonnées des vecteurs \( \vect{ AB }\) et \( \vect{ AB }+\vect{ BC }\). 

\item
    Donner les coordonnées du point \( X\) tel que \( \vect{ AX }=\vect{ BC }\) (méthode au choix)
    \end{enumerate}
    
    



    $\vect{ AB }=(-5;9)$

    $\vect{ AB }+\vect{ BC }=(-10;14)$

    $X=(-3;-1)$
    
\end{enumerate}
}
\vbox{Numéro 59.
\emph{Toutes les réponses doivent être justifiées par un calcul accompagné d'un raisonnement.}
\begin{enumerate}\item
Soient les points $A(2;6)$, $ B(-8,-3)$ et $ C(3;-9)$. 
    \begin{enumerate}
    \item
    
    Calculer les coordonnées des vecteurs \( \vect{ AB }\) et \( \vect{ AB }+\vect{ BC }\). 

\item
    Donner les coordonnées du point \( X\) tel que \( \vect{ AX }=\vect{ BC }\) (méthode au choix)
    \end{enumerate}
    
    



    $\vect{ AB }=(-10;-9)$

    $\vect{ AB }+\vect{ BC }=(1;-15)$

    $X=(13;0)$
    \item

    Soient les points $A(7;0)$, $D(7;6)$ et $L(-8;-1)$. Donner les coordonnées du point $F$ tel que $ADLF$ soit un parallélogramme (méthode au choix).
    

$F=(-8;-7)$
\end{enumerate}
}
\vbox{Numéro 60.
\emph{Toutes les réponses doivent être justifiées par un calcul accompagné d'un raisonnement.}
\begin{enumerate}\item
Soient les points $A(-2;8)$, $ B(1,10)$ et $ C(2;-7)$. 
    \begin{enumerate}
    \item
    
    Calculer les coordonnées des vecteurs \( \vect{ AB }\) et \( \vect{ AB }+\vect{ BC }\). 

\item
    Donner les coordonnées du point \( X\) tel que \( \vect{ AX }=\vect{ BC }\) (méthode au choix)
    \end{enumerate}
    
    



    $\vect{ AB }=(3;2)$

    $\vect{ AB }+\vect{ BC }=(4;-15)$

    $X=(-1;-9)$
    \item

    Soient les points $K(2;-5)$, $L(-7;9)$ et $E(6;-10)$. Donner les coordonnées du point $B$ tel que $KLEB$ soit un parallélogramme (méthode au choix).
    

$B=(15;-24)$
\end{enumerate}
}
\vbox{Numéro 61.
\emph{Toutes les réponses doivent être justifiées par un calcul accompagné d'un raisonnement.}
\begin{enumerate}\item

    Soient les points $F(9;-5)$, $A(9;-8)$ et $D(-2;4)$. Donner les coordonnées du point $M$ tel que $FADM$ soit un parallélogramme (méthode au choix).
    

$M=(-2;7)$\item
Soient les points $A(-6;1)$, $ B(-1,1)$ et $ C(6;-2)$. 
    \begin{enumerate}
    \item
    
    Calculer les coordonnées des vecteurs \( \vect{ AB }\) et \( \vect{ AB }+\vect{ BC }\). 

\item
    Donner les coordonnées du point \( X\) tel que \( \vect{ AX }=\vect{ BC }\) (méthode au choix)
    \end{enumerate}
    
    



    $\vect{ AB }=(5;0)$

    $\vect{ AB }+\vect{ BC }=(12;-3)$

    $X=(1;-2)$
    
\end{enumerate}
}
\vbox{Numéro 62.
\emph{Toutes les réponses doivent être justifiées par un calcul accompagné d'un raisonnement.}
\begin{enumerate}\item

    Soient les points $E(0;-1)$, $L(1;-5)$ et $F(-9;5)$. Donner les coordonnées du point $K$ tel que $ELFK$ soit un parallélogramme (méthode au choix).
    

$K=(-10;9)$\item
Soient les points $A(3;-6)$, $ B(0,-10)$ et $ C(-8;-2)$. 
    \begin{enumerate}
    \item
    
    Calculer les coordonnées des vecteurs \( \vect{ AB }\) et \( \vect{ AB }+\vect{ BC }\). 

\item
    Donner les coordonnées du point \( X\) tel que \( \vect{ AX }=\vect{ BC }\) (méthode au choix)
    \end{enumerate}
    
    



    $\vect{ AB }=(-3;-4)$

    $\vect{ AB }+\vect{ BC }=(-11;4)$

    $X=(-5;2)$
    
\end{enumerate}
}
\vbox{Numéro 63.
\emph{Toutes les réponses doivent être justifiées par un calcul accompagné d'un raisonnement.}
\begin{enumerate}\item
Soient les points $A(5;-6)$, $ B(-9,7)$ et $ C(-10;-4)$. 
    \begin{enumerate}
    \item
    
    Calculer les coordonnées des vecteurs \( \vect{ AB }\) et \( \vect{ AB }+\vect{ BC }\). 

\item
    Donner les coordonnées du point \( X\) tel que \( \vect{ AX }=\vect{ BC }\) (méthode au choix)
    \end{enumerate}
    
    



    $\vect{ AB }=(-14;13)$

    $\vect{ AB }+\vect{ BC }=(-15;2)$

    $X=(4;-17)$
    \item

    Soient les points $M(5;-3)$, $L(-6;-3)$ et $B(-2;-6)$. Donner les coordonnées du point $E$ tel que $MLBE$ soit un parallélogramme (méthode au choix).
    

$E=(9;-6)$
\end{enumerate}
}
\vbox{Numéro 64.
\emph{Toutes les réponses doivent être justifiées par un calcul accompagné d'un raisonnement.}
\begin{enumerate}\item
Soient les points $A(-7;1)$, $ B(-6,2)$ et $ C(-7;-5)$. 
    \begin{enumerate}
    \item
    
    Calculer les coordonnées des vecteurs \( \vect{ AB }\) et \( \vect{ AB }+\vect{ BC }\). 

\item
    Donner les coordonnées du point \( X\) tel que \( \vect{ AX }=\vect{ BC }\) (méthode au choix)
    \end{enumerate}
    
    



    $\vect{ AB }=(1;1)$

    $\vect{ AB }+\vect{ BC }=(0;-6)$

    $X=(-8;-6)$
    \item

    Soient les points $L(5;5)$, $M(7;10)$ et $E(8;-2)$. Donner les coordonnées du point $F$ tel que $LMEF$ soit un parallélogramme (méthode au choix).
    

$F=(6;-7)$
\end{enumerate}
}
\vbox{Numéro 65.
\emph{Toutes les réponses doivent être justifiées par un calcul accompagné d'un raisonnement.}
\begin{enumerate}\item

    Soient les points $M(-3;-9)$, $D(-2;0)$ et $A(5;1)$. Donner les coordonnées du point $F$ tel que $MDAF$ soit un parallélogramme (méthode au choix).
    

$F=(4;-8)$\item
Soient les points $A(-4;9)$, $ B(0,0)$ et $ C(-6;-7)$. 
    \begin{enumerate}
    \item
    
    Calculer les coordonnées des vecteurs \( \vect{ AB }\) et \( \vect{ AB }+\vect{ BC }\). 

\item
    Donner les coordonnées du point \( X\) tel que \( \vect{ AX }=\vect{ BC }\) (méthode au choix)
    \end{enumerate}
    
    



    $\vect{ AB }=(4;-9)$

    $\vect{ AB }+\vect{ BC }=(-2;-16)$

    $X=(-10;2)$
    
\end{enumerate}
}
\vbox{Numéro 66.
\emph{Toutes les réponses doivent être justifiées par un calcul accompagné d'un raisonnement.}
\begin{enumerate}\item
Soient les points $A(2;-10)$, $ B(1,-1)$ et $ C(6;6)$. 
    \begin{enumerate}
    \item
    
    Calculer les coordonnées des vecteurs \( \vect{ AB }\) et \( \vect{ AB }+\vect{ BC }\). 

\item
    Donner les coordonnées du point \( X\) tel que \( \vect{ AX }=\vect{ BC }\) (méthode au choix)
    \end{enumerate}
    
    



    $\vect{ AB }=(-1;9)$

    $\vect{ AB }+\vect{ BC }=(4;16)$

    $X=(7;-3)$
    \item

    Soient les points $K(-7;-1)$, $B(6;-2)$ et $M(-5;2)$. Donner les coordonnées du point $E$ tel que $KBME$ soit un parallélogramme (méthode au choix).
    

$E=(-18;3)$
\end{enumerate}
}
\vbox{Numéro 67.
\emph{Toutes les réponses doivent être justifiées par un calcul accompagné d'un raisonnement.}
\begin{enumerate}\item

    Soient les points $L(-10;10)$, $F(0;1)$ et $E(0;6)$. Donner les coordonnées du point $K$ tel que $LFEK$ soit un parallélogramme (méthode au choix).
    

$K=(-10;15)$\item
Soient les points $A(-5;10)$, $ B(7,-5)$ et $ C(7;5)$. 
    \begin{enumerate}
    \item
    
    Calculer les coordonnées des vecteurs \( \vect{ AB }\) et \( \vect{ AB }+\vect{ BC }\). 

\item
    Donner les coordonnées du point \( X\) tel que \( \vect{ AX }=\vect{ BC }\) (méthode au choix)
    \end{enumerate}
    
    



    $\vect{ AB }=(12;-15)$

    $\vect{ AB }+\vect{ BC }=(12;-5)$

    $X=(-5;20)$
    
\end{enumerate}
}
\vbox{Numéro 68.
\emph{Toutes les réponses doivent être justifiées par un calcul accompagné d'un raisonnement.}
\begin{enumerate}\item

    Soient les points $D(-7;-2)$, $F(-6;2)$ et $K(9;-8)$. Donner les coordonnées du point $E$ tel que $DFKE$ soit un parallélogramme (méthode au choix).
    

$E=(8;-12)$\item
Soient les points $A(7;-6)$, $ B(6,-2)$ et $ C(-9;0)$. 
    \begin{enumerate}
    \item
    
    Calculer les coordonnées des vecteurs \( \vect{ AB }\) et \( \vect{ AB }+\vect{ BC }\). 

\item
    Donner les coordonnées du point \( X\) tel que \( \vect{ AX }=\vect{ BC }\) (méthode au choix)
    \end{enumerate}
    
    



    $\vect{ AB }=(-1;4)$

    $\vect{ AB }+\vect{ BC }=(-16;6)$

    $X=(-8;-4)$
    
\end{enumerate}
}
\vbox{Numéro 69.
\emph{Toutes les réponses doivent être justifiées par un calcul accompagné d'un raisonnement.}
\begin{enumerate}\item
Soient les points $A(-1;-9)$, $ B(5,-8)$ et $ C(4;8)$. 
    \begin{enumerate}
    \item
    
    Calculer les coordonnées des vecteurs \( \vect{ AB }\) et \( \vect{ AB }+\vect{ BC }\). 

\item
    Donner les coordonnées du point \( X\) tel que \( \vect{ AX }=\vect{ BC }\) (méthode au choix)
    \end{enumerate}
    
    



    $\vect{ AB }=(6;1)$

    $\vect{ AB }+\vect{ BC }=(5;17)$

    $X=(-2;7)$
    \item

    Soient les points $E(-3;3)$, $F(-4;5)$ et $D(-3;10)$. Donner les coordonnées du point $A$ tel que $EFDA$ soit un parallélogramme (méthode au choix).
    

$A=(-2;8)$
\end{enumerate}
}
\vbox{Numéro 70.
\emph{Toutes les réponses doivent être justifiées par un calcul accompagné d'un raisonnement.}
\begin{enumerate}\item

    Soient les points $K(-2;-6)$, $F(8;9)$ et $B(5;10)$. Donner les coordonnées du point $M$ tel que $KFBM$ soit un parallélogramme (méthode au choix).
    

$M=(-5;-5)$\item
Soient les points $A(5;8)$, $ B(-10,-6)$ et $ C(2;6)$. 
    \begin{enumerate}
    \item
    
    Calculer les coordonnées des vecteurs \( \vect{ AB }\) et \( \vect{ AB }+\vect{ BC }\). 

\item
    Donner les coordonnées du point \( X\) tel que \( \vect{ AX }=\vect{ BC }\) (méthode au choix)
    \end{enumerate}
    
    



    $\vect{ AB }=(-15;-14)$

    $\vect{ AB }+\vect{ BC }=(-3;-2)$

    $X=(17;20)$
    
\end{enumerate}
}
\vbox{Numéro 71.
\emph{Toutes les réponses doivent être justifiées par un calcul accompagné d'un raisonnement.}
\begin{enumerate}\item
Soient les points $A(-10;8)$, $ B(-9,10)$ et $ C(-6;-1)$. 
    \begin{enumerate}
    \item
    
    Calculer les coordonnées des vecteurs \( \vect{ AB }\) et \( \vect{ AB }+\vect{ BC }\). 

\item
    Donner les coordonnées du point \( X\) tel que \( \vect{ AX }=\vect{ BC }\) (méthode au choix)
    \end{enumerate}
    
    



    $\vect{ AB }=(1;2)$

    $\vect{ AB }+\vect{ BC }=(4;-9)$

    $X=(-7;-3)$
    \item

    Soient les points $K(9;1)$, $L(-2;-8)$ et $F(-6;-7)$. Donner les coordonnées du point $A$ tel que $KLFA$ soit un parallélogramme (méthode au choix).
    

$A=(5;2)$
\end{enumerate}
}
\vbox{Numéro 72.
\emph{Toutes les réponses doivent être justifiées par un calcul accompagné d'un raisonnement.}
\begin{enumerate}\item

    Soient les points $M(8;10)$, $E(-9;-6)$ et $L(-4;2)$. Donner les coordonnées du point $F$ tel que $MELF$ soit un parallélogramme (méthode au choix).
    

$F=(13;18)$\item
Soient les points $A(5;10)$, $ B(-7,-10)$ et $ C(1;8)$. 
    \begin{enumerate}
    \item
    
    Calculer les coordonnées des vecteurs \( \vect{ AB }\) et \( \vect{ AB }+\vect{ BC }\). 

\item
    Donner les coordonnées du point \( X\) tel que \( \vect{ AX }=\vect{ BC }\) (méthode au choix)
    \end{enumerate}
    
    



    $\vect{ AB }=(-12;-20)$

    $\vect{ AB }+\vect{ BC }=(-4;-2)$

    $X=(13;28)$
    
\end{enumerate}
}
\vbox{Numéro 73.
\emph{Toutes les réponses doivent être justifiées par un calcul accompagné d'un raisonnement.}
\begin{enumerate}\item
Soient les points $A(1;2)$, $ B(-4,9)$ et $ C(-10;4)$. 
    \begin{enumerate}
    \item
    
    Calculer les coordonnées des vecteurs \( \vect{ AB }\) et \( \vect{ AB }+\vect{ BC }\). 

\item
    Donner les coordonnées du point \( X\) tel que \( \vect{ AX }=\vect{ BC }\) (méthode au choix)
    \end{enumerate}
    
    



    $\vect{ AB }=(-5;7)$

    $\vect{ AB }+\vect{ BC }=(-11;2)$

    $X=(-5;-3)$
    \item

    Soient les points $A(0;7)$, $K(-5;-1)$ et $B(-3;6)$. Donner les coordonnées du point $F$ tel que $AKBF$ soit un parallélogramme (méthode au choix).
    

$F=(2;14)$
\end{enumerate}
}
\vbox{Numéro 74.
\emph{Toutes les réponses doivent être justifiées par un calcul accompagné d'un raisonnement.}
\begin{enumerate}\item

    Soient les points $M(-1;-6)$, $F(-7;8)$ et $A(10;-5)$. Donner les coordonnées du point $E$ tel que $MFAE$ soit un parallélogramme (méthode au choix).
    

$E=(16;-19)$\item
Soient les points $A(7;-3)$, $ B(-9,2)$ et $ C(-3;9)$. 
    \begin{enumerate}
    \item
    
    Calculer les coordonnées des vecteurs \( \vect{ AB }\) et \( \vect{ AB }+\vect{ BC }\). 

\item
    Donner les coordonnées du point \( X\) tel que \( \vect{ AX }=\vect{ BC }\) (méthode au choix)
    \end{enumerate}
    
    



    $\vect{ AB }=(-16;5)$

    $\vect{ AB }+\vect{ BC }=(-10;12)$

    $X=(13;4)$
    
\end{enumerate}
}




