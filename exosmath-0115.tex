% This is part of Un soupçon de mathématique sans être agressif pour autant
% Copyright (c) 2012-2013,2015
%   Laurent Claessens
% See the file fdl-1.3.txt for copying conditions.

\begin{exercice}\label{exosmath-0115}

\begin{wrapfigure}{r}{5.0cm}
   \vspace{-0.5cm}        % à adapter.
   \centering
   \input{Fig_figureXQZwoWu.pstricks}
\end{wrapfigure}

    Sur le dessin ci-contre, nous nommons \( c\) la longueur des côtés du cube. Le point \( I\) est le milieu de \( [AF]\). Les réponses aux questions suivantes dépendent de \( c\).

        \begin{enumerate}
 %           \item
 %               Quelle est la longueur de \( [AI]\) ?
            \item
                Quel est le périmètre du triangle \( AFC\) ?
            \item
                Quelle est la longueur de la hauteur \( [IC]\) ?
            \item
                Calculer l'aire du triangle \( AFC\).
            \item
                Quelle est la nature du triangle \( AEF\) ?
            \item
                Calculer l'aire du triangle \( AEF\). Comparer à l'aire du triangle \( ADC\).
        \end{enumerate}


\corrref{smath-0115}
\end{exercice}
