% This is part of Un soupçon de mathématique sans être agressif pour autant
% Copyright (c) 2012
%   Laurent Claessens
% See the file fdl-1.3.txt for copying conditions.

\begin{corrige}{smath-0073}

    \begin{multicols}{2}
        Soit \( M\) le point d'intersection de deux diamètres. Le point \( M\) est le centre du cercle et donc le milieu de \( [AC]\) et de \( [BD]\). Par conséquent le quadrilatère \( ADCB\) est un parallélogramme et nous avons les égalités vectorielles \( \vect{ AD }=\vect{ BC }\) et \( \vect{ AB }=\vect{ DC }\). Donc
        \begin{equation}
            \vect{ AD }+\vect{ AB }=\vect{ BC }+\vect{ AB }=\vect{ AC }.
        \end{equation}

    \columnbreak

%The result is on figure \ref{LabelFigfigureKHUxoaG}. % From file figureKHUxoaG
%\newcommand{\CaptionFigfigureKHUxoaG}{<+Type your caption here+>}
    \begin{center}
\input{Fig_figureKHUxoaG.pstricks}
    \end{center}

    \end{multicols}


\end{corrige}
