% This is part of Un soupçon de mathématique sans être agressif pour autant
% Copyright (c) 2014
%   Laurent Claessens
% See the file fdl-1.3.txt for copying conditions.

\begin{corrige}{smath-0671}

    \begin{enumerate}
        \item
            Vu que la droite doit passer par l'origine (\( (0;0)\)), la fonction que nous cherchons est une fonction linéaire, donc de la forme \( f(x)=mx\). Le fait que le graphe de cette droite passe par le point \( (6;3)\) nous dit que
            \begin{equation}
                f(6)=3,
            \end{equation}
            donc \( m\times 6=3\) et donc \( m=\frac{ 1 }{2}\). La fonction cherchée est donc
            \begin{equation}
                f(x)=\frac{ x }{2}.
            \end{equation}
        \item
            Il suffit de calculer \( f(20)=10\), donc \( A\) est sur le graphe; \( f(2)=1\) donc \( B\) n'y est pas (c'est le point \( (2;1)\) qui y est) et \( f(0)=0\) donc le point \( C\) n'est pas non plus sur le graphe de \( f\).
        \item
            Le dessin est ici :
            \begin{center}
   \input{Fig_TCpyyhO.pstricks}
            \end{center}
        \item
            L'équation \( g(x)=h(x)\) n'est autre que \( 2x+1=3x-4\). Pour la résolution, c'est 
            \begin{subequations}
                \begin{align}
                    2x+1&=3x-4\\
                    -x&=-5\\
                    x&=5
                \end{align}
            \end{subequations}
        \item
            La solution de l'équation \( g(x)-h(x)\) donne l'abscisse du point d'intersection. Le point d'intersection est donc de la forme \( I=(5;y)\). Pour trouver \( y\) il suffit de calculer \( y=g(5)\) ou \( y=h(5)\); dans les deux cas nous trouvons \( I=(5;11)\).

            Notons que le dessin confirme le point d'intersection en  \( (5;11)\).
    \end{enumerate}

\end{corrige}
