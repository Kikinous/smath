% This is part of Un soupçon de mathématique sans être agressif pour autant
% Copyright (c) 2014
%   Laurent Claessens
% See the file fdl-1.3.txt for copying conditions.

\begin{corrige}{smath-0975}

    \begin{enumerate}
        \item
            Le plus simple est toujours de commencer par tracer le plus long côté (mais si on veut, on peut commencer par un autre, si on veut). Voici la procédure à suivre :
            \begin{enumerate}
                \item
                    Tracer \( [BC]\) avec une longueur de \SI{10}{\centi\meter}. Notez qu'il est conseillé de tracer ce segment horizontalement, en suivant les lignes du cahier.
                \item
                    Tracer un arc de cercle de rayon \SI{5}{\centi\meter} centré en \( B\).
                \item
                    Tracer un arc de cercle de rayon \SI{7}{\centi\meter} centré en \( C\).
                \item
                    Le point \( A\) est placé sur l'intersection.
            \end{enumerate}

\begin{center}
   \input{Fig_TFUFooJKyBhZ.pstricks}
\end{center}

\item

    Le second triangle ne peut pas être tracé parce que les mesures qui sont demandées ne respectent pas l'inégalité triangulaire : la somme des deux petites longueurs fait
    \begin{equation}
        5+7=12
    \end{equation}
    alors que le plus grand côté fait \( 15\).
    \end{enumerate}

\end{corrige}
