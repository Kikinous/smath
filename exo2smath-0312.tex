% This is part of Un soupçon de mathématique sans être agressif pour autant
% Copyright (c) 2015
%   Laurent Claessens
% See the file fdl-1.3.txt for copying conditions.

\begin{exercice}\label{exo2smath-0312}

    Dans un texte écrit en français, on observe les nombres d'apparitions de lettres suivants :
    \begin{center}
    \begin{tabular}[]{|c||c|}
        \hline
        lettre&nombre d'apparitions\\
        \hline\hline
        a&$2326$\\
        \hline
        e&$3982$\\
        \hline
        w&$28$\\
        \hline
        autres&$21262$\\
        \hline
    \end{tabular}
    \end{center}
    \begin{enumerate}
        \item
            Quelle est la proportion de «e» dans ce texte ? Donner une approximation en pourcentage.
        \item
            Quelle est la proportion de «w» dans ce texte ? Donner une approximation en pourcentage.
        \item
            Après ajout d'un nouveau chapitre, l'auteur a ajouté \( 1000\) lettres dont \( 200\) «e». Quelle est la proportion de lettres \emph{autre} que «e»  dans l'ajout ?
        \item
            Quelle est la proportion de «e» dans le texte complet ?
    \end{enumerate}


\corrref{2smath-0312}
\end{exercice}
