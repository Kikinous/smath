% This is part of Un soupçon de mathématique sans être agressif pour autant
% Copyright (c) 2012-2013
%   Laurent Claessens
% See the file fdl-1.3.txt for copying conditions.

\begin{exercice}\label{exosmath-0061}

    Soit le vecteur \( \vect{ u }=\begin{pmatrix}
        3    \\ 
        5    
    \end{pmatrix}\).
    \begin{enumerate}
        \item
            Donner la valeur de \( x\) telle que le vecteur \( \vect{ v }= \begin{pmatrix}
                x    \\ 
                3    
            \end{pmatrix}\) soit colinéaire à \( \vect{ u }\).
        \item
            Dessiner les vecteurs \( \vect{ u }\) et \( \vect{ v }\) dans un repère orthonormé.
        \item
            Déterminer graphiquement et par le calcul les cordonnées du vecteur \( 2\vect{ v }-\vect{ u }\).
        \item
            Soit le point \( A=(-2;7)\). 
            \begin{enumerate}
                \item
            Déterminer le point \( B\) tel que \( \frac{ 1 }{2}\vect{ AB }=\vect{ u }\).
        \item
            Donner un point \( C\) tel que le vecteur \( \vect{ AC }\) soit un vecteur directeur de la droite d'équation \( y=2x+11\).
            \end{enumerate}
    \end{enumerate}

\corrref{smath-0061}
\end{exercice}
