% This is part of Un soupçon de mathématique sans être agressif pour autant
% Copyright (c) 2014
%   Laurent Claessens
% See the file fdl-1.3.txt for copying conditions.

%--------------------------------------------------------------------------------------------------------------------------- 
\subsection*{Confiture sucrée}
%---------------------------------------------------------------------------------------------------------------------------

Après un bel été bien ensoleillé, Philippe souhaite faire de la confiture pas trop sucrée. En regardant sur l'internet, il trouve trois recettes.

\begin{center}
    \begin{tabular}[]{|c|c|}
        \hline
        Confiture de fraises&«\SI{450}{\gram} de sucre pour \SI{750}{\gram} de fraises» \\
        \hline
        Confiture d'abricots& «\SI{500}{\gram} de sucre pour \SI{500}{\kilo\gram} d'abricots» \\
        \hline
        Confiture de cerises&  «\SI{800}{\gram} de sucre pour \SI{2400}{\gram} de cerises» \\ 
        \hline
    \end{tabular}
\end{center}

\begin{enumerate}
    \item
Pour chaque recette, exprimer la proportion de sucre ajouté dans la confiture sous forme de fraction.
\item
    Simplifier le plus possible les fractions obtenues à la question précédente.
\end{enumerate}

Philippe cherche à savoir quelle est la recette avec le moins de sucre ajouté. Il fait le raisonnement suivant : « C'est dans la confiture de fraises qu'on retrouve la masse de sucre ajouté la moins importante (\SI{450}{\gram}), c'est donc dans la confiture de fraises qu'il y a le moins de sucre ajouté. ». 

\begin{enumerate}
    \item
Que penser de ce raisonnement ?
\item
Quelle est la confiture qui contient le moins de sucre ajouté en proportion ?
\end{enumerate}
