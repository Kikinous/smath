% This is part of Un soupçon de mathématique sans être agressif pour autant
% Copyright (c) 2015
%   Laurent Claessens
% See the file fdl-1.3.txt for copying conditions.

\begin{exercice}\label{exo2smath-0188}

% Il ne faut pas spécialement détruire cet exercice pour le faire rentrer dans sa colonne : lorsqu'il est sur la colonne de droite, ça va bien.
% Attention : il y a des corrections de cet exercice. Il faut donc les modifier aussi.

    Donner le volume des solides décrits ci-dessous
    \begin{enumerate}
        \item
            Une pyramide à case carré dont le côté mesure \SI{4}{\milli\meter} et la hauteur \SI{10}{\milli\meter}.
        \item\label{ItemGMVUooTdUUJG}
            Un cône dont le rayon de la base est \SI{4}{\centi\meter} et dont la hauteur est \SI{9}{\centi\meter}
        \item
            Une pyramide dont la base est un triangle rectangle de mesures \SI{13}{\centi\meter}, \SI{84}{\centi\meter} et \SI{85}{\centi\meter} et dont la hauteur est \SI{17}{\centi\meter}.
        \item
            Le même cône que dans \ref{ItemGMVUooTdUUJG}, mais à qui on aurait multiplié le rayon de la base par \( k\).
        \item   \label{ItemGMVUooTdUUJGe}
            \begin{multicols}{2}
            Sur ce dessin \( (SI)\) est perpendiculaire à la base, \( IB=15\) et \( SB=17\). La base est carrée.
            \columnbreak
\begin{center}                                                                                     
   \input{Fig_WUEUooYWwanB.pstricks}                                                 
\end{center}


            \end{multicols}
    \end{enumerate}

\corrref{2smath-0188}
\end{exercice}
