% This is part of Un soupçon de mathématique sans être agressif pour autant
% Copyright (c) 2014
%   Laurent Claessens
% See the file fdl-1.3.txt for copying conditions.

\begin{corrige}{smath-0976}

    \begin{enumerate}
        \item
            La piscine est la partie hachurée. En effet il est dit dans l'énoncé que la terrasse ferait \SI{6}{\meter}; la partie blanche est donc la terrasse.
        \item
            L'aire d'un rectangle se calcul avec la formule 
            \begin{equation}
                \text{longueur}\times\text{largeur}.
            \end{equation}
            Dans notre cas, la longueur est \( 10-6\) et la largeur est \( 3\). L'aire est donc donnée par \( (10-6)\times 3\).
    \end{enumerate}

\end{corrige}
