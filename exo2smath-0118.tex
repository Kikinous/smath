% This is part of Un soupçon de mathématique sans être agressif pour autant
% Copyright (c) 2015
%   Laurent Claessens
% See the file fdl-1.3.txt for copying conditions.

\begin{exercice}\label{exo2smath-0118}

    Voici les quantités conseillées de sirop de menthe à ajouter en fonction de la quantité d'eau :
    \begin{equation*}
        \begin{array}[]{|c||c|c|c|c|c|}
            \hline
            \text{quantité d'eau (\si{\centi\liter})}&15&30&45&50&100\\
              \hline\hline
              \text{sirop de menthe (\si{\centi\liter})}&2&4&6&6.66&13.33\\ 
              \hline 
               \end{array}
     \end{equation*}
     Est-ce une situation de proportionnalité ? Si oui, quelle quantité de sirop de menthe devrait-on mettre pour deux litres d'eau ?
           

\corrref{2smath-0118}
\end{exercice}
