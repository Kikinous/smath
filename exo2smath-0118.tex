% This is part of Un soupçon de mathématique sans être agressif pour autant
% Copyright (c) 2015
%   Laurent Claessens
% See the file fdl-1.3.txt for copying conditions.

\begin{exercice}\label{exo2smath-0118}

    Lorsqu'on met une résistance sur un circuit électrique, le circuit chauffe. Voici la chaleur émise (en joule par seconde) par un multiprise sur lequel nous avons connecté de plus en plus d'ampoules (identiques) :
    \begin{equation*}
        \begin{array}[]{|c||c|c|c|c|}
            \hline
            \text{nombre d'ampoules}&2&3&5&10\\
            \hline\hline
            \text{chaleur émise}&<++>&<++>&<++>&<++>\\
            \hline
        \end{array}
    \end{equation*}
    Est-ce une situation de proportionnalité ? Que se passe-t-il si on branche trop d'objets sur une même prise ?

\corrref{2smath-0118}
\end{exercice}
