% This is part of Un soupçon de mathématique sans être agressif pour autant
% Copyright (c) 2013
%   Laurent Claessens
% See the file fdl-1.3.txt for copying conditions.

\begin{exercice}\label{exosmath-0542}

\begin{wrapfigure}{r}{4.0cm}
   \vspace{-0.5cm}        % à adapter.
   \centering
   \input{Fig_HMNXfhy.pstricks}
\end{wrapfigure}

    Un jardinier est responsable d'un jardin carré qu'il veut diviser en \( 6\) massifs de fleurs différentes. La division est faite comme sur la figure ci-contre (qui n'est pas à l'échelle) en \( 3\) petits carrés, \( 3\) carrés moyens et un grand rectangle.

    Si l'aire du rectangle est de \unit{168}{\meter\squared}, quelle est l'aire du jardin ?

    Même question si le rectangle mesure \unit{210}{\meter\squared}.

\corrref{smath-0542}
\end{exercice}
