% This is part of Un soupçon de mathématique sans être agressif pour autant
% Copyright (c) 2012-2013
%   Laurent Claessens
% See the file fdl-1.3.txt for copying conditions.

%+++++++++++++++++++++++++++++++++++++++++++++++++++++++++++++++++++++++++++++++++++++++++++++++++++++++++++++++++++++++++++
\section*{Sources et remerciements}
%+++++++++++++++++++++++++++++++++++++++++++++++++++++++++++++++++++++++++++++++++++++++++++++++++++++++++++++++++++++++++++

Ces notes sont rédigées pour mes classes de seconde, de première STMG et de terminale STL du lycée Jacques Duhamel.

\begin{enumerate}
    \item
        Pour la partie statistiques et pour les fonctions, j'ai pompé énormément à Pauline Klein\footnote{Je tiens à préciser que sa mise en page était plus belle que celle que celle que vous trouverez ici.}. Je lui ai copié de nombreux exercices dans à peu près tous les chapitres.
    \item
        Certaines idées et définitions proviennent de \cite{oklaEg}.
    \item
        Certains exercices pour les STMG ont profité des exemples donnés par madame Colette Perrot (merci) 
    \item
        J'ai prix les fiches \cite{qyKnLf} sur les statistiques descriptives trouvées sur le site de l'université de Montpelier 3 (Paul Valéry). Je les copie ici avec l'aimable autorisation de Richard Varro.
    \item
        J'ai pris de nombreux exercices de questions posées sur le forum de l'\href{http://www.ilemaths.net/forum_lycee.php}{île des mathématiques}.
    \item
        Pas mal d'exercices proviennent de \href{http://tehessin.tuxfamily.org/?page=35}{Guillaume Connan} qui a aimablement tout mis sous une \href{http://guilde.jeunes-chercheurs.org/Guilde/Licence/ldl.html}{licence de documentation libre}.
    \item
        Les \href{http://eduscol.education.fr/cid49154/mentions-legales.html}{mentions légales} d'éduscol donnent les documents sous licence \href{http://creativecommons.org/licenses/by-nc-sa/2.0/fr/}{Créative Commons}. Je ne m'en suis pas privé.
\end{enumerate}

%+++++++++++++++++++++++++++++++++++++++++++++++++++++++++++++++++++++++++++++++++++++++++++++++++++++++++++++++++++++++++++ 
\section*{Note à propos des calculatrices}
%+++++++++++++++++++++++++++++++++++++++++++++++++++++++++++++++++++++++++++++++++++++++++++++++++++++++++++++++++++++++++++

Est-ce que quelqu'un pourrait me dire si la proposition suivante est vraie ou fausse ?
\begin{quote}
    L'usage des calculatrices en classe est uniquement possible parce que TI et Casio sont en état de monopole. Si dix entreprises fournissaient dix modèles différents avec des modes opératoires différents, les cours à la calculatrices ne seraient pas possibles. En particulier, TI et Casio prennent l'éducation nationale comme un substitut à écrire des modes d'emploi lisibles, et vivent d'une sorte de taxe sur l'enseignement.

    En un mot : elles sont des entreprises parasites de l'éducation nationale. (sans compter les déchets que génèrent toutes ces calculatrices)
\end{quote}
