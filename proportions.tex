% This is part of Un soupçon de mathématique sans être agressif pour autant
% Copyright (c) 2012
%   Laurent Claessens
% See the file fdl-1.3.txt for copying conditions.

%+++++++++++++++++++++++++++++++++++++++++++++++++++++++++++++++++++++++++++++++++++++++++++++++++++++++++++++++++++++++++++
\section{Effectifs}
%+++++++++++++++++++++++++++++++++++++++++++++++++++++++++++++++++++++++++++++++++++++++++++++++++++++++++++++++++++++++++++

Nous disons qu'un gaz est en concentration de une \defe{partie par million}{partie par million} si un million de grammes d'air contient un gramme du gaz. Voici quelque chiffre concernant l'évolution de la concentration de \( CO_2\) dans l'atmosphère; les chiffres sont en \( \unit{}{ppm}\) :
\begin{center}
\begin{tabular}{|c|c|c|}
    \hline
    1750    &   2005    &   1012\\
    \hline
    280&380&395\\
    \hline
\end{tabular}
\end{center}
Pour information, cette concentration n'a pas dépassé les \unit{300}{ppm} depuis au moins \( 600.000\) ans.

\begin{enumerate}
    \item
        De combien de pourcent la concentration de \( CO_2\) a augmenté entre 1750 et 2005 ?
    \item
        Sur un kilo d'air, combien de grammes de \( CO_2\) ?
    \item 
        Quelle est la vitesse (en ppm par an) d'augmentation de la concentration entre 1750 et 2005 ? Même question entre 2005 et 2012.
\end{enumerate}

\begin{definition}
    Une \defe{population}{population} est un ensemble fini. Si \( E\) est une population, une \defe{sous-population}{sous-population} est un sous-ensemble \( A\subset E\). L'\defe{effectif}{effectif} d'une population est le nombre de ses éléments.

    La \defe{proportion}{proportion} de \( A\) dans \( E\) est le rapport
    \begin{equation}
        p_A=\frac{ n_A }{ n_E }
    \end{equation}
    où \( n_A\) et \( n_E\) sont les effectifs de \( A\) et \( E\).
\end{definition}
Note : une proportion est un nombre compris entre zéro et un.

\begin{example}
    Demander combien il y a de gauchers dans la classe. Quelle en est la proportion ?
\end{example}

\begin{example}
    Diviser la classe en \( 4\) groupes suivant que l'élève habite ou non à Dole et qu'il utilise ou non un cahier. Remplir le tableau suivant :

    \begin{center}
    \begin{tabular}{|l||c|c||c|}
        \hline\hline
        & habite à Dole&n'habite pas à Dole&total\\
        \hline
        utilise un cahier&&&\\
        \hline
        n'utilise pas de cahier&&&\\
        \hline\hline
        total&&&\\
        \hline
    \end{tabular}
    \end{center}

    Soit \( E\) la population totale : toute la classe; soit $A$ la population de ceux qui vivent à Dole; et \( B\) celle de ceux qui utilisent un cahier.
    \begin{enumerate}
        \item
            Trouver les effectifs des populations \( A\cup B\) et \( A\cap B\).
        \item
            Quelle est la proportion de \( A\) dans \( E\) ? Et celle du complémentaire \( \bar A\) ?
        \item
            Exprimer les proportions \( p_{A\cap B}\) et \( p_{A\cup B}\).
    \end{enumerate}
\end{example}

Nous avons l'égalité
\begin{equation}
    n_{A\cup B}=n_A+n_B-n_{A\cap B}
\end{equation}
parce que dans le compte \( n_A+n_B\), nous comptons deux fois les individus qui sont dans \( A\) et dans \( B\). En passant aux proportions (c'est à dire en divisant tout par \( n_E\)), nous avons la formule
\begin{equation}
    p_{A\cup B}=p_A+p_B-p_{A\cap B}.
\end{equation}

\begin{remark}
    Si les populations \( A\) et \( B\) sont disjointes, alors \( n_{A\cap B}=0\) et nous trouvons la formule
    \begin{equation}
        p_{A\cup B}=p_A+p_B.
    \end{equation}
\end{remark}

\begin{example} \label{ExscIpur}
    Dans une station de montagne, \( 35\%\) des personnes font du ski de fond et \( 50\%\) font de la marche.
    \begin{enumerate}
        \item
            Si nous supposons que personne ne fait les deux, quelle est la proportion des personnes ne faisant ni l'un ni l'autre ?

            Sur \( 100\) personnes, \( 35\) font du ski de fond, \( 50\) font de la marche , donc \( 85\) font l'un ou l'autre et par conséquent \( 15\) ne font rien. Il y a donc \( 15\%\) des personnes de la station de montagne ne faisant ni ski de fond mi marche.
        \item
            Si \( 10\%\) des personnes font le ski de fond et la marche, quelle est la proportion des personnes ne faisant ni l'un ni l'autre ?

            Lorsqu'on a une étude de plusieurs critères pour la même population, avec intersection, le plus simple est de faire un arbre, et de le remplir avec les données.
            \begin{equation}
            \xymatrix{%
                &&&\fbox{100}\ar[lld]_{\text{marche}}\ar[rrd]^{\text{pas marche}}\\
                &\fbox{50}\ar[ld]_{\text{ski}}\ar[rd]^{\text{pas ski}}&&&&\fbox{?A?}\ar[ld]_{\text{ski}}\ar[rd]^{\text{pas ski}}\\
                \fbox{10}&&\fbox{?B?}&&\fbox{?C?}&&\fbox{?D?}
               }
            \end{equation}
            Dans ce diagramme, le \( 100\) indique l'ensemble de la station, le \( 50\) indique les \( 50\%\) qui font de la marche et le \( 10\) indique les \( 10\%\) de personnes qui font les deux. En effet : la case en bas à gauche est la case dans laquelle sont les personnes qui font la marche et le ski de fond.

            Les autres cases peuvent maintenant être remplies. On commence par la case \( ?B?\). Étant donné que le groupe de la marche fait \( 50\) et que le sous-groupe de la marche et du ski fait \( 10\), le sous-groupe de la «marche+pas ski» doit faire \( 40\). 

            L'énoncé dit que \( 35\) font du ski, donc la somme des «ski+marche» et «ski+pas marche» doit faire \( 35\). Vu qu'on en a déjà \( 10\) dans «ski+marche», il y en a \( 25\) dans «ski+pas marche», c'est à dire dans la case \( ?C?\).

            Nous pouvons déjà écrire ceci :
            \begin{equation}
            \xymatrix{%
                &&&\fbox{100}\ar[lld]_{\text{marche}}\ar[rrd]^{\text{pas marche}}\\
                &\fbox{50}\ar[ld]_{\text{ski}}\ar[rd]^{\text{pas ski}}&&&&\fbox{?A?}\ar[ld]_{\text{ski}}\ar[rd]^{\text{pas ski}}\\
                \fbox{10}&&\fbox{40}&&\fbox{25}&&\fbox{?D?}
               }
            \end{equation}
            Étant donné que \( 50\%\) font de la marche, \( 50\%\) n'en font pas, donc la case \( ?A?\) vaut \( 50\). Et pour compléter, la case \( ?D?\) doit faire \( 25\) parce que la case «pas marche+ski» fait \( 25\).

            La tableau final est :
            \begin{equation}
            \xymatrix{%
                &&&\fbox{100}\ar[lld]_{\text{marche}}\ar[rrd]^{\text{pas marche}}\\
                &\fbox{50}\ar[ld]_{\text{ski}}\ar[rd]^{\text{pas ski}}&&&&\fbox{50}\ar[ld]_{\text{ski}}\ar[rd]^{\text{pas ski}}\\
                \fbox{10}&&\fbox{40}&&\fbox{25}&&\fbox{25}
               }
            \end{equation}
    \end{enumerate}
\end{example}

%+++++++++++++++++++++++++++++++++++++++++++++++++++++++++++++++++++++++++++++++++++++++++++++++++++++++++++++++++++++++++++
\section{Évolution}
%+++++++++++++++++++++++++++++++++++++++++++++++++++++++++++++++++++++++++++++++++++++++++++++++++++++++++++++++++++++++++++

Quelle est la plus lourde augmentation ? Un ordinateur dont le prix passe de \( 600\)€ à \( 605\)€ euros ou bien un pain au chocolat qui passe de \( 0.8\)€ à \( 0.9\)€ ?

Dans toute cette partie nous considérons une valeur (prix, effectifs, capacité d'une salle de spectacle, \ldots) évoluant d'une \emph{valeur initiale} \( V_I\) à une \emph{valeur finale} \( V_F\).

\begin{definition}
    Soit une valeur (prix, effectifs, capacité d'une salle de spectacle, \ldots) évoluant d'une \emph{valeur initiale} \( V_I\) à une \emph{valeur finale} \( V_F\). Le \defe{coefficient multiplicateur}{coefficient!multiplicateur} est la nombre par lequel \( V_I\) est multiplié pour obtenir \( V_F\) :
    \begin{equation}
        V_F=C\times V_I.
    \end{equation}
    Ici, le coefficient multiplicateur est le \( C\).
\end{definition}

Un coefficient multiplicateur de hausse est un nombre plus grand que \( 1\). Un coefficient multiplicateur de baisse est un nombre entre \( 0\) et \( 1\).

\begin{definition}
    Soit une valeur (prix, effectifs, capacité d'une salle de spectacle, \ldots) évoluant d'une \emph{valeur initiale} \( V_I\) à une \emph{valeur finale} \( V_F\). La \defe{variation absolue}{variation!absolue} est la différence :
    \begin{equation}
        \VarAbs=V_F-V_I
    \end{equation}
    et la \defe{variation relative}{variation!relative} est le rapport entre la variation absolue et la valeur initiale :
    \begin{equation}
        \VarRel=\frac{ V_F-V_I }{ V_I }.
    \end{equation}
    Cette dernière peut être exprimée sous forme de pourcentage.
\end{definition}

\begin{Aretenir}
    \begin{enumerate}
        \item
            Pour diminuer une valeur de \( x\%\), il faut la multiplier par 
            \begin{equation}
                1-\frac{ x }{ 100 }.
            \end{equation}
        \item
            Le coefficient multiplicateur d'une augmentation de \( x\%\) est
            \begin{equation}
                1+\frac{ x }{ 100 }.
            \end{equation}
    \end{enumerate}
\end{Aretenir}

En effet si nous notons \( V_I\) la valeur initiale et \( V_F\) la valeur finale (qui est une augmentation de \( x\%\) par rapport à \( V_I\)), alors nous avons
\begin{equation}
    V_F=V_I+\underbrace{V_I\times \frac{ x }{ 100 }}_{x\% \text{de \( V_I\)}}.
\end{equation}
En mettant \( V_I\) en facteur,
\begin{equation}
    V_F=V_I\times \left( 1+\frac{ x }{ 100 } \right).
\end{equation}

%+++++++++++++++++++++++++++++++++++++++++++++++++++++++++++++++++++++++++++++++++++++++++++++++++++++++++++++++++++++++++++
\section{Exercices}
%+++++++++++++++++++++++++++++++++++++++++++++++++++++++++++++++++++++++++++++++++++++++++++++++++++++++++++++++++++++++++++

%---------------------------------------------------------------------------------------------------------------------------
\subsection{Effectifs et pourcentage}
%---------------------------------------------------------------------------------------------------------------------------

\Exo{Seconde-0022}
\Exo{Seconde-0030}
\Exo{Premiere-0001}
\Exo{Premiere-0002}
\Exo{Premiere-0003}
\Exo{Premiere-0004}
\Exo{Premiere-0005}
\Exo{Premiere-0006}
\Exo{Premiere-0007}
\Exo{Premiere-0008}
\Exo{Premiere-0009}
\Exo{Premiere-0010}
\Exo{Premiere-0011}
\Exo{Premiere-0012}
\Exo{Seconde-0026}
\Exo{Seconde-0024}
\Exo{Premiere-0014}                                                                                                                                     
\Exo{Premiere-0015}                                               
\Exo{Premiere-0016}                                                                       
\Exo{Premiere-0017}                                                                                                                      
\Exo{Seconde-0034}
\Exo{Premiere-0023}                                                                                                                                    

\Exo{Premiere-0065}

%---------------------------------------------------------------------------------------------------------------------------
\subsection{Évolution : introduction}
%---------------------------------------------------------------------------------------------------------------------------

\Exo{smath-0038}
\Exo{smath-0039}
\Exo{smath-0040}

%---------------------------------------------------------------------------------------------------------------------------
\subsection{Évolution : autres}
%---------------------------------------------------------------------------------------------------------------------------

\Exo{smath-0031}                                                                                                                            
\Exo{smath-0032}
\Exo{smath-0033}
\Exo{smath-0034}
\Exo{smath-0035}
\Exo{smath-0036}
\Exo{smath-0041}
\Exo{smath-0080}
