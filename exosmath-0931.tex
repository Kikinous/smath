% This is part of Un soupçon de mathématique sans être agressif pour autant
% Copyright (c) 2014
%   Laurent Claessens
% See the file fdl-1.3.txt for copying conditions.

\begin{exercice}[\cite{JGWooXAVokw}]\label{exosmath-0931}

    \begin{multicols}{2}

\begin{center}
   \input{Fig_ZFNXooSjzTPJ.pstricks}
\end{center}

\columnbreak

    \begin{enumerate}
        \item
            Que représente la droite \( (EP)\) dans le triangle \( EFG\) ?
        \item
            Que représente la droite \( (FP)\) dans le triangle \( EFG\) ?
        \item
            En déduire que la droite \( (PG)\) est perpendiculaire à la droite \( (FE)\).
    \end{enumerate}

    \end{multicols}
\corrref{smath-0931}
\end{exercice}
