% This is part of Un soupçon de mathématique sans être agressif pour autant
% Copyright (c) 2012
%   Laurent Claessens
% See the file fdl-1.3.txt for copying conditions.

\begin{exercice}\label{exoPremiere-0068}

    Nous allons concevoir une fonction qui donne le milieu et la longueur d'un segment. Pour cela nous allons considérer qu'un point est donné (en coordonnées) par une liste \info{[x,y]} à deux éléments.

    \begin{enumerate}
        \item
            Écrire une fonction  qui prend comme argument deux listes \info{A} et \info{B} et qui en affiche le contenu, c'est à dire \info{A[0]}, \info{A[1]}, \info{B[0]}, \info{B[1]}.
        \item
            Modifier la fonction pour qu'elle calcule deux variables intermédiaires \info{DX} et \info{DY} qui contiennent la «somme» des deux, c'est à dire \info{DX=A[0]+B[0]} et idem pour la seconde composante.
    \end{enumerate}
    <++>

\corrref{Premiere-0068}
\end{exercice}
