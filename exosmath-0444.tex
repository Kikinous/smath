% This is part of Un soupçon de mathématique sans être agressif pour autant
% Copyright (c) 2013
%   Laurent Claessens
% See the file fdl-1.3.txt for copying conditions.

\begin{exercice}\label{exosmath-0444}

    Soit un triangle \( ABC\) équilatéral de côté \( \ell\). Nous notons \( H\) le milieu du côté \( [AB]\).
    \begin{enumerate}
        \item
            Exprimer la longueur \( h\) de la hauteur en fonction de \( \ell\) en utilisant le théorème de Pythagore.
        \item
            Exprimer la longueur de la hauteur en fonction de \( \ell\) en utilisant le sinus de l'angle \( \widehat{HAB}\).
        \item
            À partir des deux expressions trouvées de \( h\), déterminer la valeur de \( \sin( \SI{60}{\degree}  )\).
    \end{enumerate}

\corrref{smath-0444}
\end{exercice}
