% This is part of Un soupçon de mathématique sans être agressif pour autant
% Copyright (c) 2014
%   Laurent Claessens
% See the file fdl-1.3.txt for copying conditions.

\begin{corrige}{smath-0634}

    \begin{enumerate}
        \item
            Les points \( A\), \( J\), \( E\) et \( F\) sont dans le plan de la face supérieure du cube. Donc les droites \( (AJ)\) et \( (EF)\) sont coplanaires. Elles ne sont par ailleurs pas parallèles (par exemple parce que la droite parallèle à \( (EF)\) dans le plan \( (AEF)\) passant par \( A\) est \( (AB)\) et non \( (AJ)\)) et sont donc sécantes.
        \item
            Sur le dessin, oui : les traits se coupent. Mais dans la réalité les droites ne se coupent pas; c'est une illusion. Dans la réalité les droites \( (IH)\) et \( (AD)\) ne se coupent pas parce que \( (IH)\) est contenue dans le plan de la face arrière tandis que la droite \( (AD)\) est dans le plan de la face avant. Ces deux plans étant parallèles, les droites ne peuvent pas se couper.
        \item
            Oui, ils sont sécants parce que \( A\) est dans les deux plans (cela est directement visible du nom des plans). Vu que deux plans sécants (non confondus) se coupent en une droite, l'intersection \( (ABF)\cap (HGA)\) doit être une droite. Nous savons déjà que le point \( A\) est sur cette droite et nous devons donc encore trouver un point dans l'intersection pour caractériser la droite.

            Le point \( B\) est dans \( (ABF)\) parce que \( (ABF)\) est le plan de la face supérieure. Mais il est aussi dans le plan \( (HGA)\) parce que la droite parallèle à \( (HG)\) passant par \( A\) est dans le plan \( (HGA)\); cette droite est \( (AB)\). Donc \( (AB)\) est la droite d'intersection cherchée.
        \item
            Voir la correction du point \ref{ItemECveWvo} de l'exercice \ref{exosmath-0600}.
    \end{enumerate}

\end{corrige}
