% This is part of Un soupçon de mathématique sans être agressif pour autant
% Copyright (c) 2012
%   Laurent Claessens
% See the file fdl-1.3.txt for copying conditions.

Pour ce chapitre, nous suivons entre autres \cite{oklaEg}.

Ce chapitre ne contient pas de théorie particulière. Seulement de la pratique. Le exemples seront écrits en python; en dehors des séances effectuées sur ordinateur, les élèves ne sont pas obligés de maîtriser la syntaxe exacte.




%--------------------------------------------------------------------------------------------------------------------------- 
\subsection{Choses basiques}
%---------------------------------------------------------------------------------------------------------------------------

\Exo{smath-0118}
\Exo{Seconde-0018}
\Exo{smath-0037}

%--------------------------------------------------------------------------------------------------------------------------- 
\subsection{Instructions conditionnelles}
%---------------------------------------------------------------------------------------------------------------------------

\Exo{smath-0119}
\Exo{smath-0171}
\Exo{smath-0172}
\Exo{smath-0173}

%--------------------------------------------------------------------------------------------------------------------------- 
\subsection{Boucles}
%---------------------------------------------------------------------------------------------------------------------------


\Exo{smath-0181}
\Exo{smath-0182}
\Exo{smath-0183}

%--------------------------------------------------------------------------------------------------------------------------- 
\subsection{Autres}
%---------------------------------------------------------------------------------------------------------------------------

\Exo{smath-0002}
\Exo{smath-0003}
\Exo{Seconde-0086}
\Exo{smath-0007}
\Exo{smath-0008}

