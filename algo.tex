% This is part of Un soupçon de mathématique sans être agressif pour autant
% Copyright (c) 2012
%   Laurent Claessens
% See the file fdl-1.3.txt for copying conditions.

Ce chapitre ne contient pas de théorie particulière. Seulement de la pratique. Le exemples seront écrits en python; en dehors des séances effectuées sur ordinateur, les élèves ne sont pas obligés de maîtriser la syntaxe exacte.

Commencer par l'exercice \ref{exoSeconde-0018}.

%---------------------------------------------------------------------------------------------------------------------------
\subsection{Fonctions}
%---------------------------------------------------------------------------------------------------------------------------

Le programme suivant définit la fonction \info{f} qui prend deux arguments et en retourne la somme. La variable \info{k} prend la valeur \info{f(2,3)} et enfin on affiche \( \info{k}\), c'est à dire \( 5\).
\lstinputlisting{ex_algo1.py}

Pour rappel, le mot «\emph{print}» en anglais signifie «imprimer», «afficher». D'où le mot «\emph{printer}» pour «imprimante».

À votre avis, qu'affiche le programme suivant ?
\lstinputlisting{ex_algo2.py}

Les fonctions peuvent faire plus de travail avant de retourner un nombre.

\lstinputlisting{ex_algo3.py}

%---------------------------------------------------------------------------------------------------------------------------
\subsection{Instruction conditionnelle}
%---------------------------------------------------------------------------------------------------------------------------

%+++++++++++++++++++++++++++++++++++++++++++++++++++++++++++++++++++++++++++++++++++++++++++++++++++++++++++++++++++++++++++
\section{Exercices}
%+++++++++++++++++++++++++++++++++++++++++++++++++++++++++++++++++++++++++++++++++++++++++++++++++++++++++++++++++++++++++++


\Exo{Seconde-0018}
