% This is part of Un soupçon de mathématique sans être agressif pour autant
% Copyright (c) 2015
%   Laurent Claessens
% See the file fdl-1.3.txt for copying conditions.

\begin{exercice}[\cite{NRHooXFvgpp5}]\label{exo2smath-0291}


 Pour chaque question, tracer un prisme droit en perspective cavalière, décrire les faces, et tracer un patron.
 \begin{enumerate}
     \item
         
         Il a cinq faces dont une est un rectangle de \SI{6}{\centi\meter} sur \SI{4}{\centi\meter} et une autre est un triangle de côtés \SI{3}{\centi\meter}, \SI{4}{\centi\meter} et \SI{5}{\centi\meter}.

\item
    Il a six faces dont une est un parallélogramme de côtés \SI{5}{\centi\meter} et \SI{7}{\centi\meter}, et dont une autre est un carré de \SI{5}{\centi\meter} de côté.
\item

    Il a huit faces dont six d'entre elles sont des rectangles de \SI{3}{\centi\meter} sur \SI{4}{\centi\meter} et un côté de la base mesure \SI{3}{\centi\meter}.
 \end{enumerate}

\corrref{2smath-0291}
\end{exercice}
