% This is part of Un soupçon de mathématique sans être agressif pour autant
% Copyright (c) 2013
%   Laurent Claessens
% See the file fdl-1.3.txt for copying conditions.

\begin{corrige}{smath-0510}

    C'est du calcul et de la résolution d'équations.
    \begin{enumerate}
        \item
            \( f(1)=-3+7=4\) et \( f(\frac{ 4 }{ 9 })=\frac{ 17 }{ 3 }\).
        \item
            Il faut trouver une valeur de \( x\) pour laquelle nous avons \( f(x)=19\), c'est à dire résoudre l'équation
            \begin{equation}
                -3x+7=19,
            \end{equation}
            dont la solution est \( x=-4\).
        \item
            Il faut résoudre l'équation \( -3x+7=-32\). La résolution se fait comme suit :
            \begin{subequations}
                \begin{align}
                    -3x+7=-32\\
                    -3x=-32-7=-39\\
                    x=\frac{ 39 }{ 3 }=13.
                \end{align}
            \end{subequations}
            
    \end{enumerate}

\end{corrige}
