% This is part of Un soupçon de mathématique sans être agressif pour autant
% Copyright (c) 2012
%   Laurent Claessens
% See the file fdl-1.3.txt for copying conditions.

\begin{exercice}\label{exosmath-0016}

    Soient les fonctions \( f(x)=3x-1\) et \( g(x)=-x+3\), et les points \( A=(0;-1)\), \( B=(\frac{ 1 }{2};\frac{ 5 }{2})\), \( C=(-2;5)\) et \( D=(-2;-7)\).
    \begin{multicols}{2}
        \begin{enumerate}
            \item
                Parmi les points \( A\), \( B\), \( C\) et \( D\), dire lesquels sont sur le graphe de \( f\).
            \item
                Parmi les points \( A\), \( B\), \( C\) et \( D\), dire lesquels sont sur le graphe de \( g\).
            \item
                Tracer les graphes de \( f\) et \( g\). Indice : vous pouvez vous aider des points trouvés.
            \item
                Résoudre graphiquement ou algébriquement \( f(x)=g(x)\).
        \end{enumerate}
    \end{multicols}

\corrref{smath-0016}
\end{exercice}
