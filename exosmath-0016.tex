% This is part of Un soupçon de mathématique sans être agressif pour autant
% Copyright (c) 2012-2013
%   Laurent Claessens
% See the file fdl-1.3.txt for copying conditions.

\begin{exercice}\label{exosmath-0016}

    Soient les fonctions \( f\colon x\mapsto 3x-1\) et \(g\colon x\mapsto -x+3\), et les points \( A=(0;-1)\), \( B=(\frac{ 1 }{2};\frac{ 5 }{2})\), \( C=(-2;5)\) et \( D=(-2;-7)\).
        \begin{enumerate}
            \item
                Parmi les points \( A\), \( B\), \( C\) et \( D\), dire lesquels sont sur le graphe de \( f\).
            \item
                Parmi les points \( A\), \( B\), \( C\) et \( D\), dire lesquels sont sur le graphe de \( g\).
            \item
                Tracer les graphes de \( f\) et \( g\).
            \item
                Résoudre graphiquement ou algébriquement \( f(x)=g(x)\).
            \item
                Donner le point d'intersection des graphes de \( f\) et de \( g\).
        \end{enumerate}

\corrref{smath-0016}
\end{exercice}
