% This is part of Un soupçon de mathématique sans être agressif pour autant
% Copyright (c) 2015
%   Laurent Claessens
% See the file fdl-1.3.txt for copying conditions.

%+++++++++++++++++++++++++++++++++++++++++++++++++++++++++++++++++++++++++++++++++++++++++++++++++++++++++++++++++++++++++++ 
\section{Opposé d'une somme}
%+++++++++++++++++++++++++++++++++++++++++++++++++++++++++++++++++++++++++++++++++++++++++++++++++++++++++++++++++++++++++++

% This is part of Un soupçon de mathématique sans être agressif pour autant
% Copyright (c) 2015
%   Laurent Claessens
% See the file fdl-1.3.txt for copying conditions.

%--------------------------------------------------------------------------------------------------------------------------- 
\subsection%{Activité : un peu de calcul mental}
%---------------------------------------------------------------------------------------------------------------------------
Activité à ne pas distribuer : écrire au tableau au fur et à mesure

Calculer mentalement :
\begin{enumerate}
    \item
        \( 200-31\)
    \item
        \( 100-34\)
    \item
        \( 500-49\)
    \item
        \( 330-31\)
\end{enumerate}


% This is part of Un soupçon de mathématique sans être agressif pour autant
% Copyright (c) 2015
%   Laurent Claessens
% See the file fdl-1.3.txt for copying conditions.

%--------------------------------------------------------------------------------------------------------------------------- 
\subsection*{Activité : soustraire une somme}
%---------------------------------------------------------------------------------------------------------------------------

Sur la figure suivante nous avons tracé un triangle \( ABC\) et la droite parallèle à \( (AB)\) passant par \( C\). Parmi les trois angles que l'on voit au point \( C\), lequel fait \SI{34}{\degree} ? Est-il possible de déterminer la mesure des deux autres ?

\begin{center}
   \input{Fig_UZOQooTSAQcl.pstricks}
\end{center}

% ATTENTION : ces deux figures sont reprise dans une autre activité (Activité : soustraire une somme). Ne pas les changer sans les doubler.

Sur le dessin suivant, indiquer quels sont les angles égaux à \( a\) et à \( b\). 
\begin{center}
    \input{Fig_QZABooEsqWaq.pstricks}
\end{center}
Si \( a\) et \( b\) étaient connus, comment feriez-vous pour calculer l'angle \( c\) ?


\begin{Aretenir}

    Ajouter une somme revient à ajouter chacun des termes :
    \begin{equation}
        a+(b+c)=a+b+c
    \end{equation}
    pour tout nombres \( a\), \( b\) et \( c\).

    Soustraire une somme revient à ajouter l'opposé de chacun des termes :
    \begin{equation}
        a-(b+c)=a-b-c.
    \end{equation}
\end{Aretenir}

Attention : 
\begin{equation}
        a+(b-c)=a+b-c
\end{equation}
et
\begin{equation}
    a-(b-c)=a-b+c.
\end{equation}


\begin{example}
    \begin{enumerate}
        \item
            \( -(4+7)=-(11)\) et \( -4-7=-11\).
        \item
            \( 12-(3+4)=12-3-4=9-4=5\).
        \item
            \( 3x-(5+x)=3x-5-x=2x-5\).
    \end{enumerate}
\end{example}

\begin{example}
    \begin{equation}
        15-(a+b)=15-a-b.
    \end{equation}
\end{example}

% This is part of Un soupçon de mathématique sans être agressif pour autant
% Copyright (c) 2015
%   Laurent Claessens
% See the file fdl-1.3.txt for copying conditions.

%--------------------------------------------------------------------------------------------------------------------------- 
\subsection*{Activité : boîtes de bonbons et de biscuits}
%---------------------------------------------------------------------------------------------------------------------------

\begin{enumerate}
    \item   \label{ItemILDXooQXINUca}
        
On possède \( 36\) bonbons et \( 18\) biscuits, et on veut les emballer dans des boîtes identiques. Quel serait le contenu d'une boîte, et combien de telles boîtes pourra-t-on faire ?

\item
    Parmi les expressions suivantes, lesquelles sont égales à \( 18a+36b\) ?
    \begin{enumerate}
        \item
            \( 6(6a+3b)\). 
        \item
            \( 18(a+b)\)
        \item
            \( 18a\times (a+2b)\)
        \item
            \( 18(a+2b)\)
    \end{enumerate}
\end{enumerate}


%+++++++++++++++++++++++++++++++++++++++++++++++++++++++++++++++++++++++++++++++++++++++++++++++++++++++++++++++++++++++++++ 
\section{Factorisation}
%+++++++++++++++++++++++++++++++++++++++++++++++++++++++++++++++++++++++++++++++++++++++++++++++++++++++++++++++++++++++++++

% This is part of Un soupçon de mathématique sans être agressif pour autant
% Copyright (c) 2015
%   Laurent Claessens
% See the file fdl-1.3.txt for copying conditions.

%--------------------------------------------------------------------------------------------------------------------------- 
\subsection*{Activité : un peu de calcul mental}
%---------------------------------------------------------------------------------------------------------------------------


Activité à ne pas imprimer, mais à écrire au fur et à mesure au tableau :

Calculer
\begin{enumerate}
    \item
        \( 30\times 21\)
    \item
        \( 32\times 21\)
\end{enumerate}


Cette activité est à faire en groupe :
% This is part of Un soupçon de mathématique sans être agressif pour autant
% Copyright (c) 2015
%   Laurent Claessens
% See the file fdl-1.3.txt for copying conditions.

%--------------------------------------------------------------------------------------------------------------------------- 
\subsection*{Activité : le produit du milieu}
%---------------------------------------------------------------------------------------------------------------------------

Soit le petit programme de calcul
\begin{itemize}
    \item
        Écrire quatre nombres entiers consécutifs,
    \item
        calculer le produit du plus grand par le plus petit,
    \item
        calculer que le produit des deux nombres du milieu,
    \item
        soustraire le premier produit au second.
\end{itemize}
\begin{enumerate}
    \item
        Effectuer ce programme pour quelque choix de nombres. 
    \item
        Émettre une conjecture.
    \item
        Démontrer.
\end{enumerate}

de \cite{ZGRKooSUSPix}

%+++++++++++++++++++++++++++++++++++++++++++++++++++++++++++++++++++++++++++++++++++++++++++++++++++++++++++++++++++++++++++ 
\section{Double distributivité}
%+++++++++++++++++++++++++++++++++++++++++++++++++++++++++++++++++++++++++++++++++++++++++++++++++++++++++++++++++++++++++++

\begin{propriete}
    Si \( a\), \( b\), \( c\) et \( d\) sont des nombres, alors
    \begin{equation}
        (a+b)(c+d)=ac+ad\quad+\quad bc+bd.
    \end{equation}
\end{propriete}

\begin{proof}
    Il s'agit d'utiliser plusieurs fois la formule de distribution :
    \begin{subequations}
        \begin{align}
            (a+b)\times \boxed{(c+d)}&=a\times \boxed{(c+d)}&&+&& b\times \boxed{(c+d)}\\
            &=ac+ad  &&+&&bc+bd.
        \end{align}
    \end{subequations}
\end{proof}

\begin{example}
    Pour calculer \( 32\times 23 \) on peut faire
    \begin{subequations}
        \begin{align}
                32\times 21&=(30+2)\times (20+3)\\
                &=30\times 20+30\times 3+2\times 20+2\times 3\\
                &=600+90+40+6\\
                &=736
        \end{align}
    \end{subequations}
\end{example}

%+++++++++++++++++++++++++++++++++++++++++++++++++++++++++++++++++++++++++++++++++++++++++++++++++++++++++++++++++++++++++++ 
\section{Élargissement de rectangle}
%+++++++++++++++++++++++++++++++++++++++++++++++++++++++++++++++++++++++++++++++++++++++++++++++++++++++++++++++++++++++++++

Parler ici du fameux dessin de rectangle élargi dans les deux sens sur lequel on voit les quatre termes de la double distributivité.


