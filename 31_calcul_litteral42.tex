% This is part of Un soupçon de mathématique sans être agressif pour autant
% Copyright (c) 2015
%   Laurent Claessens
% See the file fdl-1.3.txt for copying conditions.

%+++++++++++++++++++++++++++++++++++++++++++++++++++++++++++++++++++++++++++++++++++++++++++++++++++++++++++++++++++++++++++ 
\section{Opposé d'une somme}
%+++++++++++++++++++++++++++++++++++++++++++++++++++++++++++++++++++++++++++++++++++++++++++++++++++++++++++++++++++++++++++

% This is part of Un soupçon de mathématique sans être agressif pour autant
% Copyright (c) 2015
%   Laurent Claessens
% See the file fdl-1.3.txt for copying conditions.

%--------------------------------------------------------------------------------------------------------------------------- 
\subsection%{Activité : un peu de calcul mental}
%---------------------------------------------------------------------------------------------------------------------------
Activité à ne pas distribuer : écrire au tableau au fur et à mesure

Calculer mentalement :
\begin{enumerate}
    \item
        \( 200-31\)
    \item
        \( 100-34\)
    \item
        \( 500-49\)
    \item
        \( 330-31\)
\end{enumerate}


% This is part of Un soupçon de mathématique sans être agressif pour autant
% Copyright (c) 2015
%   Laurent Claessens
% See the file fdl-1.3.txt for copying conditions.

%--------------------------------------------------------------------------------------------------------------------------- 
\subsection*{Activité : soustraire une somme}
%---------------------------------------------------------------------------------------------------------------------------

Sur la figure suivante nous avons tracé un triangle \( ABC\) et la droite parallèle à \( (AB)\) passant par \( C\). Parmi les trois angles que l'on voit au point \( C\), lequel fait \SI{34}{\degree} ? Est-il possible de déterminer la mesure des deux autres ?

\begin{center}
   \input{Fig_UZOQooTSAQcl.pstricks}
\end{center}

% ATTENTION : ces deux figures sont reprise dans une autre activité (Activité : soustraire une somme). Ne pas les changer sans les doubler.

Sur le dessin suivant, indiquer quels sont les angles égaux à \( a\) et à \( b\). 
\begin{center}
    \input{Fig_QZABooEsqWaq.pstricks}
\end{center}
Si \( a\) et \( b\) étaient connus, comment feriez-vous pour calculer l'angle \( c\) ?


\begin{Aretenir}

    Ajouter une somme revient à ajouter chacun des termes :
    \begin{equation}
        a+(b+c)=a+b+c
    \end{equation}
    pour tout nombres \( a\), \( b\) et \( c\).

    Soustraire une somme revient à ajouter l'opposé de chacun des termes :
    \begin{equation}
        a-(b+c)=a-b-c.
    \end{equation}
\end{Aretenir}

Attention : 
\begin{equation}
        a+(b-c)=a+b-c
\end{equation}
et
\begin{equation}
    a-(b-c)=a-b+c.
\end{equation}


\begin{example}
    \begin{enumerate}
        \item
            \( -(4+7)=-(11)\) et \( -4-7=-11\).
        \item
            \( 12-(3+4)=12-3-4=9-4=5\).
        \item
            \( 3x-(5+x)=3x-5-x=2x-5\).
    \end{enumerate}
\end{example}

\begin{example}
    \begin{equation}
        15-(a+b)=15-a-b.
    \end{equation}
\end{example}

%+++++++++++++++++++++++++++++++++++++++++++++++++++++++++++++++++++++++++++++++++++++++++++++++++++++++++++++++++++++++++++ 
\section{Factorisation}
%+++++++++++++++++++++++++++++++++++++++++++++++++++++++++++++++++++++++++++++++++++++++++++++++++++++++++++++++++++++++++++

% This is part of Un soupçon de mathématique sans être agressif pour autant
% Copyright (c) 2015
%   Laurent Claessens
% See the file fdl-1.3.txt for copying conditions.

%--------------------------------------------------------------------------------------------------------------------------- 
\subsection*{Activité : boîtes de bonbons et de biscuits}
%---------------------------------------------------------------------------------------------------------------------------

\begin{enumerate}
    \item   \label{ItemILDXooQXINUca}
        
On possède \( 36\) bonbons et \( 18\) biscuits, et on veut les emballer dans des boîtes identiques. Quel serait le contenu d'une boîte, et combien de telles boîtes pourra-t-on faire ?

\item
    Parmi les expressions suivantes, lesquelles sont égales à \( 18a+36b\) ?
    \begin{enumerate}
        \item
            \( 6(6a+3b)\). 
        \item
            \( 18(a+b)\)
        \item
            \( 18a\times (a+2b)\)
        \item
            \( 18(a+2b)\)
    \end{enumerate}
\end{enumerate}


%+++++++++++++++++++++++++++++++++++++++++++++++++++++++++++++++++++++++++++++++++++++++++++++++++++++++++++++++++++++++++++ 
\section{Double distributivité}
%+++++++++++++++++++++++++++++++++++++++++++++++++++++++++++++++++++++++++++++++++++++++++++++++++++++++++++++++++++++++++++

% This is part of Un soupçon de mathématique sans être agressif pour autant
% Copyright (c) 2015
%   Laurent Claessens
% See the file fdl-1.3.txt for copying conditions.

%--------------------------------------------------------------------------------------------------------------------------- 
\subsection*{Activité : un peu de calcul mental}
%---------------------------------------------------------------------------------------------------------------------------


Activité à ne pas imprimer, mais à écrire au fur et à mesure au tableau :

Calculer
\begin{enumerate}
    \item
        \( 30\times 21\)
    \item
        \( 32\times 21\)
\end{enumerate}


% This is part of Un soupçon de mathématique sans être agressif pour autant
% Copyright (c) 2015
%   Laurent Claessens
% See the file fdl-1.3.txt for copying conditions.

%--------------------------------------------------------------------------------------------------------------------------- 
\subsection*{Activité : une proposition douteuse}
%---------------------------------------------------------------------------------------------------------------------------

Julien possède un champ rectangulaire. Un mystérieux personnage lui propose de lui offrir deux mètres de longueur supplémentaire en échange de un mètre de largeur (c'est à dire qu'on allonge le champ de deux mètres dans un sens et on le raccourcis de un mètre dans l'autre sens).

Julien fait ses calculs et accepte l'échange.

On fait la même proposition à Paul, qui refuse, ayant calculé qu'il allait y perdre.

Quelles sont les dimensions possibles des champs de Paul et Julien ?



\begin{propriete}
    Si \( a\), \( b\), \( c\) et \( d\) sont des nombres, alors
    \begin{equation}
        (a+b)(c+d)=ac+ad\quad+\quad bc+bd.
    \end{equation}
\end{propriete}

\begin{proof}
    Il s'agit d'utiliser plusieurs fois la formule de distribution :
    \begin{subequations}
        \begin{align}
            (a+b)\times \boxed{(c+d)}&=a\times \boxed{(c+d)}&&+&& b\times \boxed{(c+d)}\\
            &=ac+ad  &&+&&bc+bd.
        \end{align}
    \end{subequations}
\end{proof}

\begin{example}
    Pour calculer \( 32\times 23 \) on peut faire
    \begin{subequations}
        \begin{align}
                32\times 21&=(30+2)\times (20+3)\\
                &=30\times 20+30\times 3+2\times 20+2\times 3\\
                &=600+90+40+6\\
                &=736
        \end{align}
    \end{subequations}
\end{example}


