% This is part of Un soupçon de mathématique sans être agressif pour autant
% Copyright (c) 2015
%   Laurent Claessens
% See the file fdl-1.3.txt for copying conditions.

%+++++++++++++++++++++++++++++++++++++++++++++++++++++++++++++++++++++++++++++++++++++++++++++++++++++++++++++++++++++++++++ 
\section{Opposé d'une somme}
%+++++++++++++++++++++++++++++++++++++++++++++++++++++++++++++++++++++++++++++++++++++++++++++++++++++++++++++++++++++++++++

% This is part of Un soupçon de mathématique sans être agressif pour autant
% Copyright (c) 2015
%   Laurent Claessens
% See the file fdl-1.3.txt for copying conditions.

%--------------------------------------------------------------------------------------------------------------------------- 
\subsection*{Activité : Encore du carrelage}
%---------------------------------------------------------------------------------------------------------------------------



% ATTENTION : cette activité est reprise en exercice, d'où la petite taille de la figure.


Christoffer carreleur veut réaliser une fresque \( 10\times 20\) comme ceci :
\begin{center}
   \input{Fig_CMUTooFyLisx.pstricks}
\end{center}
Il s'agit d'un rectangle de carreaux blancs, dont les quatre coins sont rouges (carreaux grisés) et contenant deux bandes bleues au milieu.

\begin{enumerate}
    \item
        Combien de carreaux de chaque couleur faut-il ?
    \item
        Sur ce dessin, chacune des bandes bleues a une longueur de \( 4\) carreaux. Christoffer veut savoir combien de carreaux de chaque couleur il aurait besoin si il faisait varier cette longueur. Donner le nombre de carreaux de chaque couleur en fonction de la longueur de la bande bleue.
\end{enumerate}


\begin{Aretenir}
    L'opposé d'une somme est la somme des opposés. En formule :
    \begin{equation}
        -(a+b)=-a-b
    \end{equation}
    pour tout nombres \( a\) et \( b\).

    À l'intérieur d'une expression littérale on peut supprimer une parenthèse précédée d'un signe moins en changeant tous les signes à l'intérieur de la parenthèse :
    \begin{equation}
        15-(a+b)=15-a-b.
    \end{equation}
\end{Aretenir}

\begin{example}
    \begin{enumerate}
        \item
            \( -(4+7)=-(11)\) et \( -4-7=-11\).
        \item
            \( 12-(3+4)=12-3-4=9-4=5\).
        \item
            \( 3x-(5+x)=3x-5-x=2x-5\).
    \end{enumerate}
\end{example}

%+++++++++++++++++++++++++++++++++++++++++++++++++++++++++++++++++++++++++++++++++++++++++++++++++++++++++++++++++++++++++++ 
\section{Double distributivité}
%+++++++++++++++++++++++++++++++++++++++++++++++++++++++++++++++++++++++++++++++++++++++++++++++++++++++++++++++++++++++++++

Activité à ne pas imprimer, mais à écrire au fur et à mesure au tableau :
% This is part of Un soupçon de mathématique sans être agressif pour autant
% Copyright (c) 2015
%   Laurent Claessens
% See the file fdl-1.3.txt for copying conditions.

%--------------------------------------------------------------------------------------------------------------------------- 
\subsection*{Activité : un peu de calcul mental}
%---------------------------------------------------------------------------------------------------------------------------


Activité à ne pas imprimer, mais à écrire au fur et à mesure au tableau :

Calculer
\begin{enumerate}
    \item
        \( 30\times 21\)
    \item
        \( 32\times 21\)
\end{enumerate}


% This is part of Un soupçon de mathématique sans être agressif pour autant
% Copyright (c) 2015
%   Laurent Claessens
% See the file fdl-1.3.txt for copying conditions.

%--------------------------------------------------------------------------------------------------------------------------- 
\subsection*{Activité : une proposition douteuse}
%---------------------------------------------------------------------------------------------------------------------------

Julien possède un champ rectangulaire. Un mystérieux personnage lui propose de lui offrir deux mètres de longueur supplémentaire en échange de un mètre de largeur (c'est à dire qu'on allonge le champ de deux mètres dans un sens et on le raccourcis de un mètre dans l'autre sens).

Julien fait ses calculs et accepte l'échange.

On fait la même proposition à Paul, qui refuse, ayant calculé qu'il allait y perdre.

Quelles sont les dimensions possibles des champs de Paul et Julien ?




