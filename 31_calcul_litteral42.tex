% This is part of Un soupçon de mathématique sans être agressif pour autant
% Copyright (c) 2015
%   Laurent Claessens
% See the file fdl-1.3.txt for copying conditions.

% This is part of Un soupçon de mathématique sans être agressif pour autant
% Copyright (c) 2015
%   Laurent Claessens
% See the file fdl-1.3.txt for copying conditions.

%--------------------------------------------------------------------------------------------------------------------------- 
\subsection{Activité : des parenthèses}
%---------------------------------------------------------------------------------------------------------------------------

kmlkmlklm


sdsfdsf


%+++++++++++++++++++++++++++++++++++++++++++++++++++++++++++++++++++++++++++++++++++++++++++++++++++++++++++++++++++++++++++ 
\section{Opposé d'une somme}
%+++++++++++++++++++++++++++++++++++++++++++++++++++++++++++++++++++++++++++++++++++++++++++++++++++++++++++++++++++++++++++

\begin{Aretenir}
    L'opposé d'une somme est la somme des opposés. En formule :
    \begin{equation}
        -(a+b)=-a-b
    \end{equation}
    pour tout nombres \( a\) et \( b\).
\end{Aretenir}

\begin{example}
    \begin{enumerate}
        \item
            \( -(4+7)=-(11)\) et \( -4-7=-11\).
        \item
            \( 12-(3+4)=12-3-4=9-4=5\).
    \end{enumerate}
\end{example}
