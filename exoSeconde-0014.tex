% This is part of Un soupçon de mathématique sans être agressif pour autant
% Copyright (c) 2012-2013
%   Laurent Claessens
% See the file fdl-1.3.txt for copying conditions.

\begin{exercice}\label{exoSeconde-0014}

    Voici un relevé de notes à un devoir :
    \begin{center}
        \begin{tabular}{|l||c|c|c|c|c|c|c|c|c|c|c|c|c|c|c|c|c|c|c|c|c|c||c|}
            \hline
            note \( x_i\)&0&1&2&3&4&5&6&7&8&9&10&11&12&13&14&15&16&17&18&19&20&total\\
            \hline
            effectifs \( n_i\)&0&0&1&2&2&4&6&5&4&5&11&5&6&6&5&4&3&5&2&1&0&77\\
            \hline
            fréquence&&&&&&&&&&&&&&&&&&&&&&\\
            \hline
        \end{tabular}
    \end{center}
    \begin{multicols}{2}
        \begin{enumerate}
            \item
                Remplir les cases vides.
            \item
                Donner la moyenne.
            \item
                Combien de candidats ont obtenu moins de \( 5\) ?
            \item
                Déterminer la médiane et les quartiles.
            \item
                Tracer le graphique des fréquences cumulées.
        \end{enumerate}
    \end{multicols}

\corrref{Seconde-0014}
\end{exercice}
