% This is part of Un soupçon de mathématique sans être agressif pour autant
% Copyright (c) 2013
%   Laurent Claessens
% See the file fdl-1.3.txt for copying conditions.

\begin{exercice}[\cite{FGMWzeK}]\label{exosmath-0292}

En utilisant les propriétés de la fonction inverse, comparer les nombres suivants.
    \begin{enumerate}
        \item
            \( \frac{1}{ \sqrt{6}+2 }\) et \( \frac{1}{ \sqrt{6}-3 }\).
        \item
            \( \frac{1}{ x^2 }\) et \( \frac{1}{ x^2+2 }\) 
    \end{enumerate}
    Note : nous ne supposons rien sur la valeur de \( x\). Discuter si nécessaire.

\corrref{smath-0292}
\end{exercice}
