% This is part of Un soupçon de mathématique sans être agressif pour autant
% Copyright (c) 2014-2015
%   Laurent Claessens
% See the file fdl-1.3.txt for copying conditions.

%--------------------------------------------------------------------------------------------------------------------------- 
\subsection*{Activité : secteurs d'émissions}
%---------------------------------------------------------------------------------------------------------------------------

Sur cette planète, un cinquième des émissions de dioxyde de carbone sont dus aux processus industriels, un autre cinquième aux transports et un dixième aux bâtiments. 
\begin{enumerate}
    \item
Colorier l'égalité suivante :
\begin{equation*}
    \begin{array}[]{ccccccc}
    \input{Fig_GLVooNqioToooZERO.pstricks}&+&\input{Fig_GLVooNqioToooONE.pstricks}&+&\input{Fig_GLVooNqioToooTWO.pstricks}&=&\input{Fig_GLVooNqioToooTHREE.pstricks}\\
    \text{industries}&&\text{transports}&&\text{bâtiments}&&\text{total des \( 3\)}
    \end{array}
\end{equation*}
\item
Quelle fraction du total des émissions de \( CO_2\) est due à ces trois activités ?
\item
Les centrales énergétiques sont responsables d'un tiers des émissions de dioxyde de carbone. Quel est le total de ces quatre activités ?
\end{enumerate}

\noindent {\scriptsize Pour plus d'informations, voir\ \url{http://savoirsenmultimedia.ens.fr/uploads/videos//diffusion/2012_02_09_jancovici.mp4}}
