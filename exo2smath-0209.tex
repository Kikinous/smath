% This is part of Un soupçon de mathématique sans être agressif pour autant
% Copyright (c) 2015
%   Laurent Claessens
% See the file fdl-1.3.txt for copying conditions.

\begin{exercice}[\cite{NRHooXFvgpp4}]\label{exo2smath-0209}

    La masse d’un atome de carbone est égale à \SI[parse-numbers=true]{1.99e-26 }{\kilo\gram}. Les chimistes considèrent des paquets contenant \SI[parse-numbers=true]{6.022e23} atomes. 
    \begin{enumerate}
        \item
            Calculer la masse en grammes d’un tel paquet d’atomes de carbone. 
        \item
            Donner une valeur arrondie de cette masse à un gramme près.
    \end{enumerate}

\corrref{2smath-0209}
\end{exercice}
