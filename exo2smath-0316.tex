% This is part of Un soupçon de mathématique sans être agressif pour autant
% Copyright (c) 2015
%   Laurent Claessens
% See the file fdl-1.3.txt for copying conditions.

\begin{exercice}\label{exo2smath-0316}

    Vrai ou faux ? (justifier)
    \begin{enumerate}
        \item
            Si \( x+y=17\) alors il est possible que \( y=20\).
        \item
            Si Léa possède \( x\) bonbons et Jean en possède \( y\), et si \( x+y=17\), alors il est possible que \( y=20\).
        \item
            Si \( F\) appartient au cercle de diamètre \( [GH]\) alors \( \widehat{FGH}+\widehat{FHG}=\SI{90}{\degree}\).
        \item
            Si \( EDF\) est un triangle rectangle en \( D\) et si \( O\) est le milieu de \( [EF]\) alors \( OE=\frac{ EF }{ 2 }\).
    \end{enumerate}

\corrref{2smath-0316}
\end{exercice}
