% This is part of Un soupçon de mathématique sans être agressif pour autant
% Copyright (c) 2012
%   Laurent Claessens
% See the file fdl-1.3.txt for copying conditions.

\begin{corrige}{Premiere-0045}

    \begin{enumerate}
        \item
            Il suffit de remplacer \( x\) par \( -1\) dans l'expression donnée et de voir que l'on obtient zéro.
        \item
            Vu que \( -1\) est racine et que le coefficient de \( x^2\) est \( (m-2)\), le polynôme peut s'écrire sous la forme
            \begin{equation}
                (m-2)(x+1)(x+A)
            \end{equation}
            pour un certain \( A\). Nous devons donc fixer \( A\) de telle façon à avoir
            \begin{equation}
                (m-2)\big( x^2+(A+1)x+A \big)=(m-2)x^2+5x+7-m.
            \end{equation}
            En égalisant les termes de même degré nous trouvons les équations
            \begin{subequations}
                \begin{align}
                    m-2&=m-2\\
                    (m-2)(A+1)&=5\\
                    (m-2)A&=7-m.
                \end{align}
            \end{subequations}
            La première est automatiquement satisfaite. La troisième donne
            \begin{equation}
                A=\frac{ 7-m }{ m-2 }.
            \end{equation}
            Nous vérifions que ce \( A\) fonctionne aussi pour la seconde équation :
            \begin{equation}
                (m-2)\left( \frac{ 7-m }{ m-2 }+1 \right)=5.
            \end{equation}
            Un peu de calcul montre que c'est bon. Donc la seconde racine est
            \begin{equation}
                -A=-\frac{ 7-m }{ m-2 }.
            \end{equation}
        \item
            llmklm
    \end{enumerate}
    <++>

\end{corrige}
