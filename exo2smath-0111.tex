% This is part of Un soupçon de mathématique sans être agressif pour autant
% Copyright (c) 2015
%   Laurent Claessens
% See the file fdl-1.3.txt for copying conditions.

\begin{exercice}\label{exo2smath-0111}

    Barnabé le cuisinier connaît une recette immanquable pour faire du spaghetti à la bolognaise. Pour un kilo de préparation, il utilise \SI{275}{\gram} de spaghetti, \SI{270}{\gram} de viande de bœuf et \SI{455}{\gram} de tomates fraîches.

    \begin{enumerate}
        \item
            De quelle quantité de chaque ingrédient a-t-il besoin pour préparer \( x\) kilos de pâtes bolognaise ?
        \item
            Barnabé dispose d'une quantité suffisante de tomates et de pâtes, mais ne dispose que de \( \SI{x}{\gram}\) de viande. Quelle quantité de tomates et de spaghetti pourra-t-il utiliser, et quelle poids total de préparation pourra-t-il faire ?
    \end{enumerate}

\corrref{2smath-0111}
\end{exercice}
