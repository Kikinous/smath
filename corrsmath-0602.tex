% This is part of Un soupçon de mathématique sans être agressif pour autant
% Copyright (c) 2014
%   Laurent Claessens
% See the file fdl-1.3.txt for copying conditions.

\begin{corrige}{smath-0602}

    \begin{enumerate}
        \item

    Voici le tableau complété avec la ligne des effectifs cumulés croissants :
    \begin{center}
\begin{tabular}{|c||c|c|c|c|c|c|c|c|c|c|c|}
  \hline 
  \textbf{Salaire} & \ 900&1100&1300&1500&1700&1900&2100&2500&3100&4500&\text{total}\\
  \hline 
  \textbf{Effectif} & 12&10&20&18&8&8&5&5&2&1&89\\
  \hline
  \textbf{ECC}&12 &22&42&60&68&76&81&86&88&89&\\
  \hline
\end{tabular}
        
    \end{center}
    
\item

    Le salaire moyen s'obtient comme moyenne pondérée par les effectifs :
    \begin{equation}
        m=\frac{ 900\times 12+1100\times 10+\ldots +2500\times 5+3100\times 2+4500 }{ 89 }=\frac{ 137300 }{ 89 }\simeq 1542.7.
    \end{equation}
\item
    Il y a \(60 \) personnes qui gagnent moins que la moyenne, c'est donc plus de la moitié des salariés.
\item
    Le premier quartile vaut \( 1300\) et le second vaut \( 1700\). Il y a donc \( 20+18+8=46\) personnes gagnant un salaire dans l'intervalle interquartile. En pourcentage, cela fait
    \begin{equation}
        \frac{ 46 }{ 89 }\times 100\simeq 51.68
    \end{equation}
    Notons que le fait que cela soit très proche de \( 50\%\) est logique : environ la moitié de la population est entre le premier et le troisième quartile.

\item
    En ce qui concerne la moyenne, elle augmente parce le changement revient à augmenter le salaire de \( 10\) personnes de \( 1500\) à \( 1700\).

    La médiane ne change pas parce que les \( 44\)\ieme\ et \( 45\)\ieme\ valeurs restent à \( 1500\).
            
    \end{enumerate}

\end{corrige}
