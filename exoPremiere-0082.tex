% This is part of Un soupçon de mathématique sans être agressif pour autant
% Copyright (c) 2012
%   Laurent Claessens
% See the file fdl-1.3.txt for copying conditions.

\begin{exercice}\label{exoPremiere-0082}

    Dans un verger de cinq mille arbres contenant \( 150\) pommes chacun, une pomme sur vingt est mangée par une chenille. Nous en cueillons une centaine.
    \begin{enumerate}
        \item
            Combien de pommes mangées par une chenille devrions nous obtenir environ ?
        \item
            Est-ce que la situation peut être assimilée à un schéma de Bernoulli de paramètres \( p=0.05\) et \( n=100\) ? En particulier, que pensez-vous de l'hypothèse de tirage sans remise qui se trouve dans la définition du schéma de Bernoulli ?
    \end{enumerate}

\corrref{Premiere-0082}
\end{exercice}
