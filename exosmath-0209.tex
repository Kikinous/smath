% This is part of Un soupçon de mathématique sans être agressif pour autant
% Copyright (c) 2012-2013
%   Laurent Claessens
% See the file fdl-1.3.txt for copying conditions.

\begin{exercice}\label{exosmath-0209}

    Soit une fonction définie sur \( \mathopen[ -20 , 20 \mathclose]\) dont le tableau de variation est
\begin{equation*}
    \begin{array}[]{c|ccccccc}
        x&-20&&-3&&10&&20\\
        \hline
        &&&-6&&&&1\\
        f(x)&&\nearrow&&\searrow&&\nearrow\\
        &-10&&&&-15
    \end{array}
\end{equation*}
\begin{enumerate}
    \item
        Comparer \( f(5)\) et \( f(9)\).
    \item
        Comparer \( f(-5)\) et \( f(0)\).
    \item
        Dessiner une forme possible du graphe de cette fonction.
\end{enumerate}

\corrref{smath-0209}
\end{exercice}
