% This is part of Un soupçon de mathématique sans être agressif pour autant
% Copyright (c) 2013
%   Laurent Claessens
% See the file fdl-1.3.txt for copying conditions.

\begin{corrige}{smath-0410}

    Pour plus de facilité, nous nommons les points \( A=(1;1)\), \( B=(4;-1)\) et \( C=(-1;5)\). Pour savoir si ils forment un triangle isocèle, il suffit de calculer les longueurs des trois côtés et de voir si deux sont identiques.
    \begin{subequations}
        \begin{align}
            AB^2=(4-1)^2+(-1-1)^2=9+4=13.\\
            AC^2=(-1-1)^2+(5-1)^2=4+16=20\\
            BC^2=(-1-4)^2+(5+1)^2=25+36=61.
        \end{align}
    \end{subequations}
    Donc les trois longueurs sont \( \sqrt{13}\), \( \sqrt{20}\) et \( \sqrt{61}\). Ce triangle n'est donc pas isocèle.

\end{corrige}
