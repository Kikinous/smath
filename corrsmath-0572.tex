% This is part of Un soupçon de mathématique sans être agressif pour autant
% Copyright (c) 2013
%   Laurent Claessens
% See the file fdl-1.3.txt for copying conditions.

\begin{corrige}{smath-0572}

    \begin{enumerate}
        \item
            Nous cherchons une fonction du type \( f(x)=mx+p\). Vu que le coefficient directeur doit être \( 4\), nous avons \( m=4\). Pour que \( 3\) soit antécédent de \( 10\), il faut \( f(3)=10\), ou encore
            \begin{equation}
                4\times 3+p=10.
            \end{equation}
            Cela demande \( p=-2\). La réponse est donc \( f(x)=4x-2\).
        \item
            La fonction proposée est affine et son graphe est donc une droite. Pour tracer une droite il faut savoir deux points. Le plus simple est de commencer par \( g(0)=-2\); le graphe passe donc par le point \( A(0;-2)\). Un autre facile est par exemple \( g(2)=6\) et donc le point \( B(2;6)\). Le résultat est le suivant :
\begin{center}
   \input{Fig_ILauamX.pstricks}
\end{center}

        \item

            Pour le tableau de signe d'une fonction affine, le plus compliqué est de trouver pour quelle valeur de \( x\) la fonction s'annule; dans notre cas, cela se découvre en résolvant \( 4x-2=0\); la solution est \( x=\frac{ 1 }{2}\). Le tableau de signe est :
                \begin{equation*}
                    \begin{array}[]{c|ccc}
                        x&&\frac{ 1 }{2}&\\
                          \hline
                          g(x)&-&0&+\\ 
                           \end{array}
                       \end{equation*}

                   \item

                       Les solutions de \( g(x)<0\) se lisent dans le tableau de signe; ce sont tous les nombres jusqu'à \( \frac{ 1 }{2}\) non compris : \( x\in\mathopen] -\infty , \frac{ 1 }{2} \mathclose[\).
            
    \end{enumerate}

\end{corrige}
