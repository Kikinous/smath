%This is part of Un soupçon de mathématique sans être agressif pour autant
% Copyright (c) 2012-2013
%   Laurent Claessens
% See the file fdl-1.3.txt for copying conditions.

%+++++++++++++++++++++++++++++++++++++++++++++++++++++++++++++++++++++++++++++++++++++++++++++++++++++++++++++++++++++++++++ 
\section{Introduction}
%+++++++++++++++++++++++++++++++++++++++++++++++++++++++++++++++++++++++++++++++++++++++++++++++++++++++++++++++++++++++++++

Le but de ce chapitre est de voir la notion de fonction affine.
\begin{enumerate}
    \item
        La forme \( x\mapsto mx+p\).
    \item
        Représentation graphique : signe de \( m\), coefficient directeur et ordonnée à l'origine.
    \item
        Tableau de signe.
    \item
        Résolution des équations correspondantes.
\end{enumerate}
Nous ne parlerons pas d'équation de droite à proprement parler. Donc pas de graphes de fonctions passant par deux points donné.

%+++++++++++++++++++++++++++++++++++++++++++++++++++++++++++++++++++++++++++++++++++++++++++++++++++++++++++++++++++++++++++ 
\section{L'activité}
%+++++++++++++++++++++++++++++++++++++++++++++++++++++++++++++++++++++++++++++++++++++++++++++++++++++++++++++++++++++++++++

%This is part of Un soupçon de mathématique sans être agressif pour autant
% Copyright (c) 2012-2013
%   Laurent Claessens
% See the file fdl-1.3.txt for copying conditions.

    Le taxi Besacdanslesac divise son prix en deux parties : $0.2$ euros de frais de prise en charge plus un euro par km parcouru. Le taxi Ledoubstoudoux par contre divise son prix en $1$ euro de frais de prise en charge plus $0.8$ euros par kilomètre parcouru.

    \begin{enumerate}
        \item
            Combien coûte un trajet de \SI{5}{\kilo\meter} avec Besacdanslesac ?
        \item
            Donner une expression algébrique du prix d'une course en fonction du nombre de kilomètres parcourus.
        \item
            Combien de kilomètres peut-t-on effectuer dans Ledoubstoudoux avec \( 10\) euros ?
        \item
            Exprimer les prix en fonction du nombre de kilomètres parcourus sur un graphique (les deux taxis sur le même graphique).
        \item
            À partir de combien de kilomètres parcourus vaut-il mieux prendre Ledoubstoudoux ?
    \end{enumerate}



%+++++++++++++++++++++++++++++++++++++++++++++++++++++++++++++++++++++++++++++++++++++++++++++++++++++++++++++++++++++++++++ 
\section{Définitions}
%+++++++++++++++++++++++++++++++++++++++++++++++++++++++++++++++++++++++++++++++++++++++++++++++++++++++++++++++++++++++++++

\begin{definition}
    Une \defe{fonction affine}{affine}\index{fonction!affine} est une fonction définie sur \( \eR\) par
    \begin{equation}
        f(x)=mx+p
    \end{equation}
    où \( m\) et \( p\) sont deux nombres réels fixés.
\end{definition}

\begin{Aretenir}
    Le graphe de la fonction \( x\mapsto mx+p\) passe par le point de coordonnées \( (0;p)\).
    \begin{enumerate}
        \item
            \( p\) est l'\defe{ordonnées à l'origine}{ordonnées!à l'origine} de la fonction affine \( mx+p\).
        \item
            \( m\) est le \defe{coefficient directeur}{coefficient directeur} de la fonction affine \( mx+p\).
    \end{enumerate}
\end{Aretenir}

Cas particuliers :
\begin{enumerate}
    \item
        Si \( p=0\) alors \( f(x)=mx\) et nous disons que \( f\) est une fonction \defe{linéaire}{fonction!linéaire}. Son graphe passe par l'origine \( (0;0)\).
    \item
        Si \( m=0\) alors \( f(x)=p\) et nous disons que \( f\) est une fonction \defe{constante}{fonction!constante}.
\end{enumerate}

%+++++++++++++++++++++++++++++++++++++++++++++++++++++++++++++++++++++++++++++++++++++++++++++++++++++++++++++++++++++++++++ 
\section{Représentation graphique}
%+++++++++++++++++++++++++++++++++++++++++++++++++++++++++++++++++++++++++++++++++++++++++++++++++++++++++++++++++++++++++++

\begin{Aretenir}
    La représentation graphique de la fonction \( f(x)=mx+p\) est une droite non parallèle à l'axe des ordonnées. Réciproquement toute droite non parallèle est la représentation graphique d'une fonction affine.
\end{Aretenir}


Un exemple des éléments de \( mx+p\) est donné à la figure \ref{LabelFigfigureUERGVgS}. % From file figureUERGVgS
La droite «commence» à la hauteur \( p\) et ensuite monte encore de \( m\) vers le haut à chaque pas de \( 1\) vers la droite.
\newcommand{\CaptionFigfigureUERGVgS}{Une droite et quelque éléments de son équation.}
\input{Fig_figureUERGVgS.pstricks}


\begin{multicols}{2}
    Sur la figure ci-contre, nous avons tracé les graphes des fonctions données par
    \begin{enumerate}
        \item
            \( f(x)=2x+1\)
        \item
            \( g(x)=2x-2\)
        \item
            \( h(x)=3x-1\)
    \end{enumerate}

    \columnbreak

%The result is on figure \ref{LabelFigfigureCFoZCYe}. % From file figureCFoZCYe
%\newcommand{\CaptionFigfigureCFoZCYe}{<+Type your caption here+>}
    \begin{center}
\input{Fig_figureCFoZCYe.pstricks}
    \end{center}

\end{multicols}

Nous constatons que
\begin{enumerate}
    \item
        Les droites représentatives de \( f\) et \( g\) sont parallèles.
    \item
        La droite \( h\) est plus pendue que les deux autres.
\end{enumerate}


\begin{propriete}
    Soit \( f(x)=ax+b\) une fonction affine.
    \begin{enumerate}
        \item
            Si \( a>0\) alors \( f\) est croissante sur \( \eR\).
        \item
            Si \( a<0\) alors \( f\) est décroissante sur \( \eR\).
        \item
            Si \( a=0\) alors \( f\) est constante sur \( \eR\) (et vaut \( b\)).
    \end{enumerate}
\end{propriete}


\begin{theorem}
    Soit la fonction affine \( f(x)=ax+b\). Alors pour tout nombres distincts \( u\) et \( v\) nous avons
    \begin{equation}
        \frac{ f(u)-f(v) }{ u-v }=a.
    \end{equation}
\end{theorem}

\begin{proof}
    Soient \( u\) et \( v\), deux réels distincts. Nous calculons
    \begin{equation}
        f(u)-f(v)=au+b-(av+b)=au-av=a(u-v).
    \end{equation}
    En divisant par \( u-v\) on trouve le résultat annoncé.
\end{proof}

Cela donne une méthode graphique pour déterminer le coefficient directeur d'une droite. Si les points \( A=(x_A;y_A)\) et \( B=(x_B;y_B)\) sont sur la droite \( d\) d'équation \( y=ax+b\), alors
\begin{equation}
    a=\frac{ y_B-y_A }{ x_B-x_A }.
\end{equation}
Notez que cela vient du vecteur directeur
\begin{equation}
    \begin{pmatrix}
        x_B-x_A    \\ 
        y_B-y_A    
    \end{pmatrix}
\end{equation}


\subsection{Sens de variation d'une fonction affine}

\begin{Aretenir}
      Soit $f$ la fonction affine $x\mapsto ax+b$.
      \begin{itemize}
      \item Si $a>0$, alors $f$ est croissante sur $\eR$.
      \item Si $a<0$, alors $f$ est décroissante sur $\eR$.
      \item Si $a=0$, alors $f$ est constante sur $\eR$
        (et $f(x)=b$ pour tout $x\in\eR$).
      \end{itemize}
\end{Aretenir}
Nous savons que si \( f(x)=ax+b\), alors \( f\) s'annule en \( x=-\frac{ b }{ a }\). Cela nous sert à écrire un tableau de signe de \( f\).



Les figures \ref{LabelFigFnAffineipcEQfssLabelSubFigFnAffineipcEQf0} et \ref{LabelFigFnAffineipcEQfssLabelSubFigFnAffineipcEQf1} montrent des droites affines. Lorsque \( a>0\), la droite monte; lorsque \( a<0\) elle descend. La pente est d'autant plus forte que \( a\) est grand.
\newcommand{\CaptionFigFnAffineipcEQf}{Deux droites affines.}
\input{Fig_FnAffineipcEQf.pstricks}


\begin{Aretenir}
      Règle du signe de $ax+b$ \\
      
      \begin{tabular}{cc}
        $a<0$ & $a>0$ \\
        & \\
        $\begin{array}{|c|ccccc|}
          \hline
          x & -\infty & & -\dfrac{b}{a} & & +\infty \\[2ex]
          \hline
          ax+b & & + \quad \ & 0 & \quad - & \\[1ex]
          \hline
        \end{array}$
        &
        $\begin{array}{|c|ccccc|}
          \hline
          x & -\infty & & -\dfrac{b}{a} & & +\infty \\[2ex]
          \hline
          ax+b & & - \quad \ & 0 & \quad + & \\[1ex]
          \hline
        \end{array}$ \\
      \end{tabular}  \\[1ex] 
\end{Aretenir}



%+++++++++++++++++++++++++++++++++++++++++++++++++++++++++++++++++++++++++++++++++++++++++++++++++++++++++++++++++++++++++++ 
\section{Tableau de variation}
%+++++++++++++++++++++++++++++++++++++++++++++++++++++++++++++++++++++++++++++++++++++++++++++++++++++++++++++++++++++++++++

Nous savons qu'une fonction affine est croissante ou décroissante suivant le signe de \( a\). Il y a donc deux tableaux possibles. Les tableaux de signes s'en déduisent immédiatement.

\begin{minipage}{0.485\textwidth}
    \begin{center}

        Si \( a<0\)
        \vspace{5mm}

               \input{Fig_OSQOqJN.pstricks}   

               \begin{equation*}
                   \begin{array}[]{c|ccccc}
                        x&-\infty&&-b/a&&\infty\\
                         \hline
                         ax+b&&-&0&+&\\ 
                          \end{array}
                      \end{equation*}
    \end{center}
\end{minipage}
\hspace{1mm}
\begin{minipage}{0.485\textwidth}
    \begin{center}
        Si \( a>0\)
        \vspace{5mm}

                    \input{Fig_BZqEWco.pstricks}

               \begin{equation*}
                   \begin{array}[]{c|ccccc}
                        x&-\infty&&-b/a&&\infty\\
                         \hline
                         ax+b&&+&0&-&\\ 
                          \end{array}
                      \end{equation*}
    \end{center}
\end{minipage}


%+++++++++++++++++++++++++++++++++++++++++++++++++++++++++++++++++++++++++++++++++++++++++++++++++++++++++++++++++++++++++++ 
\section{Droites parallèles et droites sécantes}
%+++++++++++++++++++++++++++++++++++++++++++++++++++++++++++++++++++++++++++++++++++++++++++++++++++++++++++++++++++++++++++

%--------------------------------------------------------------------------------------------------------------------------- 
\subsection{Critère avec le coefficient directeur}
%---------------------------------------------------------------------------------------------------------------------------

\begin{example}
    Les droites \( y=-3x+4\) et \( y=2x+1\) sont sécantes, comme nous pouvons le voir sur un dessin. Comment trouver les coordonnées du point d'intersection ?

    Le point d'intersection \( (x_0;y_0)\) doit satisfaire \( y_0=-3x_0+4\) et \( y_0=2x_0+1\) en même temps. Donc
    \begin{equation}
        -3x_0+4=2x_0+1,
    \end{equation}
    ce qui donne \( x_0=\frac{ 3 }{ 5 }\). Pour trouver \( y_0\) il suffit de remplacer dans l'une ou l'autre équation :
    \begin{equation}
        y_0=-3x_0+4=-3\frac{ 3 }{ 5 }+4=\frac{ 11 }{ 5 }.
    \end{equation}
    Vérification : dans l'autre on obtient le même résultat :
    \begin{equation}
        y_0=2x_0+1=2\frac{ 3 }{ 5 }+1=\frac{ 11 }{ 5 }.
    \end{equation}
    Donc le point d'intersection des deux droites est le point de coordonnées
    \begin{equation}
        \big( \frac{ 3 }{ 5 };\frac{ 11 }{ 5 } \big).
    \end{equation}
\end{example}

\begin{theorem}
    Dans un repère, les droites d'équations \( y=ax+b\) et \( y=a'x+b'\) sont parallèles si et seulement si elles ont même coefficient directeur, c'est à dire si et seulement si \( a=a'\).

    Les droites sont sécantes si et seulement si \( a\neq a'\).
\end{theorem}

\begin{proof}
    Nous allons seulement démontrer que si \( a\neq a'\), alors les droites s'intersectent. Pour cela nous allons montrer qu'il existe un point \( (x_0,y_0)\) qui se trouve sur le graphe des deux fonctions, c'est à dire qui vérifie
    \begin{subequations}
        \begin{numcases}{}
            y_0=ax_0+b\\
            y_0=a'x_0+b'.
        \end{numcases}
    \end{subequations}
    Le nombre \( x_0\) doit donc satisfaire
    \begin{equation}
        ax_0+b=a'x_0+b',
    \end{equation}
    c'est à dire \( (a-a')x_0=b'-b\) et donc
    \begin{equation}
        x_0=\frac{ b'-b }{ a-a' }.
    \end{equation}
\end{proof}

%--------------------------------------------------------------------------------------------------------------------------- 
\subsection{Et le vecteur directeur}
%---------------------------------------------------------------------------------------------------------------------------

Si \( A\) et \( B\) sont des points du plan, nous avons déjà vu que le vecteur \( \vect{ AB }\) était un critère de parallélisme de la droite \( AB\). Plus précisément, nous avons vu qu'une droite \( (CD)\) était parallèle à \( AB\) si et seulement si \( \vect{ CD }\) est un multiple de \( \vect{ AB }\). Voyons cela de plus près.

Soit la droite \( y=ax+b\). Les points \( A=(0;b)\) et \( B=(1,a+b)\) sont sur cette droite et donc le vecteur directeur est
\begin{equation}
    \vect{ AB }=\begin{pmatrix}
        1    \\ 
        a    
    \end{pmatrix}
\end{equation}
Si les points \( C\) et \( D\) sont tels que \( (CD)\parallel (AB)\), alors
\begin{equation}
    \vect{ CD }=\lambda\begin{pmatrix}
        1    \\ 
        a    
    \end{pmatrix}=\begin{pmatrix}
        \lambda    \\ 
            \lambda a
    \end{pmatrix}.
\end{equation}
Si le point \( C\) a pour coordonnées \( C=(x_C;y_C)\), alors le point \( D\) doit avoir comme coordonnées
\begin{equation}
    D=\big( x_C+\lambda;y_C+\lambda a \big).
\end{equation}
Le coefficient directeur de la droite \( (CD)\) est alors
\begin{equation}
    \frac{ (y_C+\lambda a)-y_C }{ (x_C+\lambda)-x_C }=a.
\end{equation}
Nous avons donc confirmé que le critère du vecteur proportionnel et le critère du coefficient directeur sont en réalité les mêmes.

%+++++++++++++++++++++++++++++++++++++++++++++++++++++++++++++++++++++++++++++++++++++++++++++++++++++++++++++++++++++++++++ 
\section{Colinéarité}
%+++++++++++++++++++++++++++++++++++++++++++++++++++++++++++++++++++++++++++++++++++++++++++++++++++++++++++++++++++++++++++

\begin{theorem}
    Trois points distincts \( A\), \( B\) et \( C\) sont alignés si et seulement si les droites \( (AB)\) et \( (AC)\) ont même coefficient directeur.
\end{theorem}

\begin{proof}
    Si les droites \( (AB)\) et \( (AC)\) ont même coefficient directeur, alors nous pouvons écrire
    \begin{equation}
        (AB)\equiv y=ax+b
    \end{equation}
    et
    \begin{equation}
        (AC)\equiv ax+b'.
    \end{equation}
    Cependant \( A\) est un point commun aux deux droites : \( y_A=ax_A+b=ax_A+b'\) et donc \( b=b'\), ce qui signifie que les droites \( (AB)\) et \( (AC)\) sont les mêmes.
\end{proof}

\begin{example}
    Vérifions que les points \( A=(1;-1)\), \( B=(3;5)\) et \( C=(4;8)\) sont alignés.

    D'abord le coefficient directeur de la droite \( (AB)\) est 
    \begin{equation}
        \frac{ y_B-y_A }{ x_B-x_A }=\frac{ 5-(-1) }{ 3-1 }=3
    \end{equation}
    ensuite le coefficient directeur de la droite \( (AC)\) est
    \begin{equation}
        \frac{ y_C-y_A }{ x_C-x_A }=\frac{ 8+1 }{ 4-1 }=3
    \end{equation}
    Donc les points \( A\), \( B\) et \( C\) sont alignés.
\end{example}

