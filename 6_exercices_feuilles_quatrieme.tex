% This is part of Un soupçon de mathématique sans être agressif pour autant
% Copyright (c) 2014
%   Laurent Claessens
% See the file fdl-1.3.txt for copying conditions.

Avertissement : ce chapitre contient les exercices des feuilles distribuées en classe, \emph{à quelque variations près}; notamment dans la numérotation.

%+++++++++++++++++++++++++++++++++++++++++++++++++++++++++++++++++++++++++++++++++++++++++++++++++++++++++++++++++++++++++++ 
\section{Opérations sur les nombres relatifs}
%+++++++++++++++++++++++++++++++++++++++++++++++++++++++++++++++++++++++++++++++++++++++++++++++++++++++++++++++++++++++++++

\Exo{smath-0726}
\Exo{smath-0722}     % à supprimer pour l'année prochaine.
\Exo{smath-0724}
\Exo{smath-0725}
\Exo{smath-0747}
\Exo{smath-0727}
\Exo{smath-0743}
\Exo{smath-0744}
\Exo{smath-0723}

\Exo{smath-0745}
\Exo{smath-0746}

\Exo{smath-0751}
\Exo{smath-0748}
\Exo{smath-0749}
\Exo{smath-0801}
\Exo{smath-0750}
\Exo{smath-0752}
\Exo{smath-0753}
\Exo{smath-0754}
\Exo{smath-0755}
\Exo{smath-0756}
\Exo{smath-0757}
\Exo{smath-0759}
\Exo{smath-0760}
\Exo{smath-0821}

%+++++++++++++++++++++++++++++++++++++++++++++++++++++++++++++++++++++++++++++++++++++++++++++++++++++++++++++++++++++++++++ 
\section{Initiation à la démonstration}
%+++++++++++++++++++++++++++++++++++++++++++++++++++++++++++++++++++++++++++++++++++++++++++++++++++++++++++++++++++++++++++

\Exo{smath-0780}
\Exo{smath-0864}
\Exo{smath-0863}
\Exo{smath-0840}
\Exo{smath-0862}
\Exo{smath-0784}
\Exo{smath-0865}
\Exo{smath-0861}
\Exo{smath-0839}
\Exo{smath-0788}
\Exo{smath-0860}
\Exo{smath-0859}
\Exo{smath-0858}

%+++++++++++++++++++++++++++++++++++++++++++++++++++++++++++++++++++++++++++++++++++++++++++++++++++++++++++++++++++++++++++ 
\section{Calcul littéral}
%+++++++++++++++++++++++++++++++++++++++++++++++++++++++++++++++++++++++++++++++++++++++++++++++++++++++++++++++++++++++++++

\Exo{smath-0804}
\Exo{smath-0805}
\Exo{smath-0806}
\Exo{smath-0807}
\Exo{smath-0808}
\Exo{smath-0809}
\Exo{smath-0810}

%+++++++++++++++++++++++++++++++++++++++++++++++++++++++++++++++++++++++++++++++++++++++++++++++++++++++++++++++++++++++++++ 
\section{Égalité de Pythagore}
%+++++++++++++++++++++++++++++++++++++++++++++++++++++++++++++++++++++++++++++++++++++++++++++++++++++++++++++++++++++++++++

\Exo{smath-0842}
\Exo{smath-0843}
\Exo{smath-0909}
\Exo{smath-0833}
\Exo{smath-0786}
\Exo{smath-0832}
\Exo{smath-0835}
\Exo{smath-0836}

%+++++++++++++++++++++++++++++++++++++++++++++++++++++++++++++++++++++++++++++++++++++++++++++++++++++++++++++++++++++++++++ 
\section{Grandeurs proportionnelles}
%+++++++++++++++++++++++++++++++++++++++++++++++++++++++++++++++++++++++++++++++++++++++++++++++++++++++++++++++++++++++++++

\Exo{smath-0941}
\Exo{smath-0948}
\Exo{smath-0949}
\Exo{smath-0944}
\Exo{smath-0950}
\Exo{smath-0945}

\Exo{smath-0952}
\Exo{smath-0951}
\Exo{smath-0943}
\Exo{smath-0946}
\Exo{smath-0947}
\Exo{smath-0957}

%+++++++++++++++++++++++++++++++++++++++++++++++++++++++++++++++++++++++++++++++++++++++++++++++++++++++++++++++++++++++++++ 
\section{Droite des milieux}
%+++++++++++++++++++++++++++++++++++++++++++++++++++++++++++++++++++++++++++++++++++++++++++++++++++++++++++++++++++++++++++

\Exo{smath-0996}
\Exo{smath-0993}
\Exo{smath-0998}   % Cet exercice est à garder parce qu'il est la preuve du théorème des milieux.
\Exo{smath-0997}
\Exo{smath-0999}
\Exo{smath-1001}
\Exo{smath-1000}

%+++++++++++++++++++++++++++++++++++++++++++++++++++++++++++++++++++++++++++++++++++++++++++++++++++++++++++++++++++++++++++ 
\section{Opérations en écriture fractionnaire}
%+++++++++++++++++++++++++++++++++++++++++++++++++++++++++++++++++++++++++++++++++++++++++++++++++++++++++++++++++++++++++++

\Exo{2smath-0038}
\Exo{2smath-0039}
\Exo{2smath-0040}
\Exo{2smath-0023}
\Exo{2smath-0025}
\Exo{2smath-0026}
\Exo{2smath-0030}
\Exo{2smath-0027}
\Exo{2smath-0028}
\Exo{2smath-0029}
\Exo{2smath-0031}
\Exo{2smath-0032}
\Exo{2smath-0033}
\Exo{2smath-0034}
\Exo{2smath-0035}

%+++++++++++++++++++++++++++++++++++++++++++++++++++++++++++++++++++++++++++++++++++++++++++++++++++++++++++++++++++++++++++ 
\section{Théorème de Thalès}
%+++++++++++++++++++++++++++++++++++++++++++++++++++++++++++++++++++++++++++++++++++++++++++++++++++++++++++++++++++++++++++

\Exo{2smath-0010}  
\Exo{2smath-0060}
\Exo{2smath-0058}
\Exo{2smath-0057}
\Exo{2smath-0056}
\Exo{2smath-0061}
\Exo{2smath-0152}

%+++++++++++++++++++++++++++++++++++++++++++++++++++++++++++++++++++++++++++++++++++++++++++++++++++++++++++++++++++++++++++ 
\section{Manipulations littérales}
%+++++++++++++++++++++++++++++++++++++++++++++++++++++++++++++++++++++++++++++++++++++++++++++++++++++++++++++++++++++++++++

\Exo{2smath-0104}
\Exo{2smath-0103}
\Exo{2smath-0102}
\Exo{2smath-0105}
\Exo{2smath-0106}
\Exo{2smath-0111}
\Exo{2smath-0107}
\Exo{2smath-0108}
\Exo{2smath-0109}
\Exo{2smath-0110}
\Exo{2smath-0176}

%+++++++++++++++++++++++++++++++++++++++++++++++++++++++++++++++++++++++++++++++++++++++++++++++++++++++++++++++++++++++++++ 
\section{Cosinus d'un angle aigu}
%+++++++++++++++++++++++++++++++++++++++++++++++++++++++++++++++++++++++++++++++++++++++++++++++++++++++++++++++++++++++++++

\Exo{2smath-0114}
\Exo{2smath-0146}
\Exo{2smath-0144}
\Exo{2smath-0147}
\Exo{2smath-0148}
\Exo{2smath-0149}
\Exo{2smath-0150}
\Exo{2smath-0151}

%+++++++++++++++++++++++++++++++++++++++++++++++++++++++++++++++++++++++++++++++++++++++++++++++++++++++++++++++++++++++++++ 
\section{Puissances (4A)}
%+++++++++++++++++++++++++++++++++++++++++++++++++++++++++++++++++++++++++++++++++++++++++++++++++++++++++++++++++++++++++++

\Exo{2smath-0201}
\Exo{2smath-0199}
\Exo{2smath-0202}
\Exo{2smath-0205}
\Exo{2smath-0203}
\Exo{2smath-0204}
\Exo{2smath-0206}
\Exo{2smath-0207}
\Exo{2smath-0208}
\Exo{2smath-0209}
\Exo{2smath-0229}

%+++++++++++++++++++++++++++++++++++++++++++++++++++++++++++++++++++++++++++++++++++++++++++++++++++++++++++++++++++++++++++ 
\section{Pyramides et cônes (4A)}
%+++++++++++++++++++++++++++++++++++++++++++++++++++++++++++++++++++++++++++++++++++++++++++++++++++++++++++++++++++++++++++

%\corrDraft

    % Patrons
\Exo{2smath-0184}
\Exo{2smath-0190}
    % Dans les volumes
\Exo{2smath-0180}
\Exo{2smath-0182}
\Exo{2smath-0181}
    %Volumes
\Exo{2smath-0188}
\Exo{2smath-0185}
\Exo{2smath-0186}
\Exo{2smath-0189}
    % Encore dans les volumes
\Exo{2smath-0187}
\Exo{2smath-0183}

%+++++++++++++++++++++++++++++++++++++++++++++++++++++++++++++++++++++++++++++++++++++++++++++++++++++++++++++++++++++++++++ 
\section{Triangles rectangles}
%+++++++++++++++++++++++++++++++++++++++++++++++++++++++++++++++++++++++++++++++++++++++++++++++++++++++++++++++++++++++++++

\Exo{2smath-0221}
\Exo{2smath-0258}
\Exo{2smath-0259}
\Exo{2smath-0260}
\Exo{2smath-0261}
\Exo{2smath-0262}
\Exo{2smath-0263}
\Exo{2smath-0264}

%+++++++++++++++++++++++++++++++++++++++++++++++++++++++++++++++++++++++++++++++++++++++++++++++++++++++++++++++++++++++++++ 
\section{Équations}
%+++++++++++++++++++++++++++++++++++++++++++++++++++++++++++++++++++++++++++++++++++++++++++++++++++++++++++++++++++++++++++

\Exo{2smath-0277}
\Exo{2smath-0282}
\Exo{2smath-0285}
\Exo{2smath-0279}
\Exo{2smath-0287}
\Exo{2smath-0288}
\Exo{2smath-0281}
\Exo{2smath-0280}
\Exo{2smath-0286}
\Exo{2smath-0278}
\Exo{2smath-0289}
