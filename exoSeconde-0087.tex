% This is part of Un soupçon de mathématique sans être agressif pour autant
% Copyright (c) 2012-2013
%   Laurent Claessens
% See the file fdl-1.3.txt for copying conditions.

\begin{exercice}\label{exoSeconde-0087}

\begin{wrapfigure}{r}{5.0cm}
    \vspace{-1cm}
    \centering
    \input{Fig_LignesCubeshBfjxk.pstricks}
\end{wrapfigure}

        Le solide ci-contre est un cube. 
        \begin{enumerate}
            \item
                Pourquoi peut-on affirmer que \( H\) est dans le plan \( (EBC)\) ?
            \item
                Est-ce que \( (BG)\) et \( (CF)\) sont perpendiculaires.
            \item
                Est-ce que \( (BH)\) et \( (CE)\) sont sécantes ? Perpendiculaires ?
            % La question suivante est supprimée parce que l'orthogonalité n'est pas au programme.
            %\item
               % Est-ce que les droites \( (CH)\) et \( (EH)\) sont perpendiculaires ?
            \item
                Si ce cube avait un côté de \unit{5}{\centi\meter}, quelle serait la longueur de \( [CH]\) ?
        \end{enumerate}

        Faire à la maison l'exercice 13 de la page 235.

\corrref{Seconde-0087}
\end{exercice}
