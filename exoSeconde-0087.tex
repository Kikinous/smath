% This is part of Un soupçon de mathématique sans être agressif pour autant
% Copyright (c) 2012
%   Laurent Claessens
% See the file fdl-1.3.txt for copying conditions.

\begin{exercice}\label{exoSeconde-0087}

    \begin{multicols}{2}
            Le solide ci-contre est un cube. 
        \begin{enumerate}
            \item
        Comment sont les droites rouges ?
    \item
        Est-ce que les droites bleus sont sécantes ? Perpendiculaires ?
    \item
        Est-ce que les droites \( (CH)\) et \( (EH)\) sont perpendiculaires ?
                
        \end{enumerate}


    Est-ce que les segments rouges sont perpendiculaires ? Et les bleus ?

\columnbreak

%    The result is on figure \ref{LabelFigLignesCubeshBfjxk}. % From file LignesCubeshBfjxk
%\newcommand{\CaptionFigLignesCubeshBfjxk}{<+Type your caption here+>}
\input{Fig_LignesCubeshBfjxk.pstricks}

    \end{multicols}

\corrref{Seconde-0087}
\end{exercice}
