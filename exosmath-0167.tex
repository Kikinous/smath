% This is part of Un soupçon de mathématique sans être agressif pour autant
% Copyright (c) 2012
%   Laurent Claessens
% See the file fdl-1.3.txt for copying conditions.

\begin{exercice}\label{exosmath-0167}

    Nous considérons la suite \( (u_n)\) définie par \( u_n=-\frac{ 1 }{2}n+2\).
    \begin{enumerate}
        \item
            Est-ce une suite arithmétique ? Si oui, quelle est sa raison ?
        \item
            Donner les valeurs de \( u_0\), \( u_1\), \( u_2\), \( u_3\) et \( u_{153}\).
        \item
            Dessiner quelque termes sur un graphique.
    \end{enumerate}

\corrref{smath-0167}
\end{exercice}
