% This is part of Un soupçon de mathématique sans être agressif pour autant
% Copyright (c) 2014
%   Laurent Claessens
% See the file fdl-1.3.txt for copying conditions.

%--------------------------------------------------------------------------------------------------------------------------- 
\subsection*{Activité : somme de fractions}
%---------------------------------------------------------------------------------------------------------------------------

Le but de cet exercice est de calculer la somme
\begin{equation}
    \frac{1}{ 4 }+\frac{ 2 }{ 3 }.
\end{equation}
en trouvant un dénominateur commun.

\begin{enumerate}
    \item
        Tracer un quadrillage \( 3\times 4\),
    \item
        Combien de cases il y a dans un quart du quadrillage ?
    \item
        Colorier un quart du quadrillage.
    \item
        Combien de cases il y a dans un tiers de ce quadrillage ?
    \item
        Colorier les deux tiers du quadrillage.
    \item
        Combien de cases sont coloriées ?
    \item
        Calculer la somme demandée au début.
\end{enumerate}

Autres questions :
\begin{enumerate}
    \item
        
Calculer de la même façon la somme \( \dfrac{ 1 }{ 5 }+\dfrac{ 9 }{ 2 }\) et la différence \( \dfrac{ 1 }{ 2 }-\dfrac{ 1 }{ 3 }\).
 \item
     Comment calculer \( \dfrac{ 15 }{ 37 }+\dfrac{ 7 }{ 10 }\) ?
\end{enumerate}
