% This is part of Un soupçon de mathématique sans être agressif pour autant
% Copyright (c) 2013
%   Laurent Claessens
% See the file fdl-1.3.txt for copying conditions.

\begin{exercice}[\cite{QTRwmtM}]\label{exosmath-0530}

    Le nombre de SMS reçus par les élèves d'une classe au cours d'une semaine se répartit comme suit :
    \begin{equation*}
        \begin{array}[]{|c||c|c|c|c|c|c|}
            \hline
            \text{Nombre de SMS}&0&5&12&15&18&50\\
            \hline
            \text{Effectifs}&10&6&8&6&4&1\\
            \hline
        \end{array}
    \end{equation*}
    Vrai ou faux (justifier) ?
    \begin{enumerate}
        \item
            L'effectif total de cette série est \( 100\).
        \item
            Le nombre moyen de SMS reçu par un élève est égal à \( 16.7\).
        \item
            L'écart interquartile \( Q_3-Q_1\) de cette série est égal à \( 18\).
        \item
            Six élèves on reçu au moins \( 15\) SMS.
        \item
            Le nombre médian de SMS reçus est égal à \( 12\).
    \end{enumerate}

\corrref{smath-0530}
\end{exercice}
