% This is part of Un soupçon de mathématique sans être agressif pour autant
% Copyright (c) 2014
%   Laurent Claessens
% See the file fdl-1.3.txt for copying conditions.

% This is part of Un soupçon de mathématique sans être agressif pour autant
% Copyright (c) 2014
%   Laurent Claessens
% See the file fdl-1.3.txt for copying conditions.



\begin{wrapfigure}{r}{2.5cm}
   \vspace{-0.5cm}        % à adapter.
   \centering
   \input{Fig_HHOooQUedri.pstricks}
\end{wrapfigure}

Tentatives de construire des triangles de longueurs imposées.

\begin{enumerate}
    \item
        
Choisir trois nombres compris entre $2$ et $15$ et tenter de tracer un triangle dont les côtés ont ces mesures (règle, rapporteur, compas, équerre).

\item

    Voyant le triangle ci-contre, Louise s'est exclamée «il est complètement faux !». Pourquoi ? Essayer de dessiner un triangle correct ayant ces mesures.

\end{enumerate}


%+++++++++++++++++++++++++++++++++++++++++++++++++++++++++++++++++++++++++++++++++++++++++++++++++++++++++++++++++++++++++++ 
\section{Constructions de triangles}
%+++++++++++++++++++++++++++++++++++++++++++++++++++++++++++++++++++++++++++++++++++++++++++++++++++++++++++++++++++++++++++

%--------------------------------------------------------------------------------------------------------------------------- 
\subsection{Connaissant les trois longueurs}
%---------------------------------------------------------------------------------------------------------------------------

Nous voulons construire un triangle \(ABC\) dont les côtés sont de longueurs sont \( AB=\unit{3}{\centi\meter}\), \( BC=\unit{4}{\centi\meter}\) et \( AC=\unit{2}{\centi\meter}\).

% Les dessins viennent de phystricksIPAooQliVZD.py

\begin{enumerate}
    \item
        Tracer un segment \( [AB]\) de longueur \unit{3}{\centi\meter}.
\begin{center}
   \input{Fig_HZPooATdojo.pstricks}
\end{center}
    \item
        Vu que le point \( C\) est à distance \unit{2}{\centi\meter} de \( A\), tracer un cercle de centre \( A\) et de rayon \unit{2}{\centi\meter}.
    \item
        Vu que le point \( C\) est à distance \unit{4}{\centi\meter} de \( B\), tracer un cercle de centre \( B\) et de rayon \unit{4}{\centi\meter}.
\begin{center}
   \input{Fig_JUZooFbAOlx.pstricks}
\end{center}
    \item
        Les intersections des deux cercles est en même temps à distance \unit{2}{\centi\meter} de \( A\) et \unit{4}{\centi\meter} de \( B\); le point \( C\) peut y être placé. Dans le cas présenté ici, il y a deux possibilités :

\begin{center}
   \input{Fig_VJZooKxwzhE.pstricks}
\end{center}

\end{enumerate}

Combien d'intersections sont possibles ?
\begin{enumerate}
    \item
        Deux. Il y a alors deux possibilités pour construire le triangle.
    \item
        Une. Les côtés \( [AC]\) et \( [BC]\) sont «tout juste» suffisant pour aller de \( A\) à \( B\). C'est le cas d'égalité
        \begin{equation}
            AB=AC+BC.
        \end{equation}
    \item 
        Aucune. Les côtés \( [AC]\) et \( [BC]\) ne sont pas assez longs pour joindre \( A\) à \( B\). Cas d'inégalité
        \begin{equation}
            AB>AC+BC.
        \end{equation}
\end{enumerate}

\begin{Aretenir}
Dans un triangle, la longueur d'un côté est toujours inférieure à la somme des longueurs des deux autres côtés.

Lorsqu'il y a égalité, les trois points sont alignés.

Cela est l'\defe{inégalité triangulaire}{inégalité!triangulaire}.
\end{Aretenir}

En pratique faut seulement vérifier que le plus long côté est plus petit que la somme des deux autres.

\begin{example}
    Est-il possible de construire un triangle \( COR\) dont les côtés ont pour mesure \( CO=\unit{5}{\centi\meter}\), \( OR=\unit{6}{\centi\meter}\) et \( RC=\unit{4}{\centi\meter}\) ?

    Le plus long côté est \( OR=6\). Mais \( CO+RC=5+4=9\), donc \( OR<CO+RC\) et il est possible de créer un tel triangle.
\end{example}

\begin{example}
    Est-il possible de construire un triangle \( KLM\) dont les côtés ont pour mesure \( KL=\unit{10}{\centi\meter}\), \( KM=\unit{1}{\centi\meter}\) et \( LM=\unit{1}{\centi\meter}\) ?

    Le plus long côté est \( KL\); les deux autres ont pour somme \( 2\) et sont donc trop petits. Il n'est pas possible de dessiner créer ce triangle.

\begin{center}
   \input{Fig_UKTooJhUzKU.pstricks}
\end{center}

\end{example}

\begin{example}
    Essayer de créer un triangle avec une règle de \unit{40}{\centi\meter} et deux stylos.
\end{example}

%--------------------------------------------------------------------------------------------------------------------------- 
\subsection{Connaissant une longueur et deux angles adjacents}
%---------------------------------------------------------------------------------------------------------------------------

% This is part of Un soupçon de mathématique sans être agressif pour autant
% Copyright (c) 2014
%   Laurent Claessens
% See the file fdl-1.3.txt for copying conditions.

%--------------------------------------------------------------------------------------------------------------------------- 
\subsection*{Activité : mesure astronomique}
%---------------------------------------------------------------------------------------------------------------------------

Des astronomes veulent mesurer la distance approximative entre la Terre et une comète. La technique consiste à mesurer à \( 6\) mois d'écart l'angle formé entre les droites Terre-comète et Terre-Soleil. Pour la facilité (nous ne sommes pas des astronomes professionnels) nous allons supposer que la comète ne se soit pas beaucoup déplacée en \( 6\) mois%\footnote{En réalité la technique décrite ici est utilisée pour mesurer des distance avec des étoiles proches, mais les angles ne sont alors pas possible à dessiner avec un rapporteur, étant de l'ordre de \unit{89.999772}{\degree}. Les nombres donnés ici sont choisis pour que l'exercice soit possible, plutôt que pour le réalisme.}. 
(En réalité la technique décrite ici est utilisée pour mesurer des distance avec des étoiles proches, mais les angles ne sont alors pas possible à dessiner avec un rapporteur, étant de l'ordre de \unit{89.999772}{\degree}. Les nombres donnés ici sont choisis pour que l'exercice soit possible, plutôt que pour le réalisme.)

Lors de la première mesure, l'angle obtenu est \unit{80}{\degree} tandis que la seconde mesure a donné \unit{60}{\degree}. Quelle est la distance entre la comète et la Terre ?


\vspace{2cm}

Comment tracer un triangle \( ABC\) sachant que \( AB=\unit{10}{\centi\meter}\), \( \widehat{CAB}=\unit{40}{\degree}\) et \( \widehat{CBA}=\unit{75}{\degree}\) ?



% Les figures viennent de CKOooBQtves
\begin{enumerate}
    \item
        Tracer un segment $[AB]$ de la bonne longueur, en prolongeant des deux côtés pour plus de précision.

\begin{center}
   \input{Fig_EIQooXptWUa.pstricks}
\end{center}
\item
    Avec le rapporteur, tracer les droites partant de \( A\) et \( B\) avec les angles donnés par rapport à \( (AB)\).

\begin{center}
   \input{Fig_GUEooKQcWgv.pstricks}
\end{center}

\item

    Le point d'intersection des deux droites est le troisième point du triangle.

\begin{center}
   \input{Fig_MQJooVtXTde.pstricks}
\end{center}



\end{enumerate}
<++>

