% This is part of Un soupçon de mathématique sans être agressif pour autant
% Copyright (c) 2015
%   Laurent Claessens
% See the file fdl-1.3.txt for copying conditions.

\begin{corrige}{2smath-0311}

    \begin{enumerate}
        \item
            

    L'aire du rectangle est \( \SI{8}{\centi\meter}\times \SI{3}{\centi\meter}=\SI{24}{\centi\meter}\).

\item

    Il faut soustraire l'aire du demi-cercle blanc. Le cercle complet aurait pour aire :
    \begin{equation}
        \pi\times R\times R=\pi\times 3\times 3=9\pi.
    \end{equation}
    Donc le demi-cercle a pour aire : \( \SI{4.5\pi}{\centi\meter\squared}\).

\item

    Au final l'aire grisée est
    \begin{equation}
        \SI{24}{\centi\meter\squared}-\SI{4.5\pi}{\centi\meter\squared}\simeq \SI{9.86}{\centi\meter\squared}.
    \end{equation}

    \end{enumerate}

\end{corrige}
