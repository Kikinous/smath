% This is part of Un soupçon de mathématique sans être agressif pour autant
% Copyright (c) 2013
%   Laurent Claessens
% See the file fdl-1.3.txt for copying conditions.

\begin{corrige}{smath-0113}

    La boîte doit faire \unit{3.2}{\centi\meter} de rayon pour \( 3\times 6.4\) centimètres de hauteur, c'est à dire \unit{19.2}{\centi\meter}. Le volume de cette boîte est donné par la formule du volume du cylindre \( V=\pi r^2h\) :
    \begin{equation}
        V=\pi\times(3.2)^2\times 19.2\simeq\unit{617.66}{\centi\cubic\meter}.
    \end{equation}
    Ici nous avons un peu arrondi, mais c'est raisonnable : nous ne sommes pas à un dixième de centimètre cube près lorsqu'on parle de mettre de l'eau dans une boîte de \( 600\) centimètres cubes.

    Le volume d'une balle est donné par le volume d'une sphère :
    \begin{equation}
        V_b=\frac{ 4 }{ 3 }\pi r^3=\frac{ 4 }{ 3 }\pi(3.2)^3=\unit{137.26}{\centi\cubic\meter}.
    \end{equation}
    Les trois balles ensemble prennent alors \unit{411.77}{\centi\cubic\meter} et l'espace vide restant dans la boîte est de
    \begin{equation}
        V_{vide}=617.66-411.77=\unit{205.88}{\centi\cubic\meter}.
    \end{equation}
    
    Combien de litres d'eau nous avons dans \( 205\) centimètres cube ? Un lite c'est \unit{1}{\deci\cubic\meter} et donc \unit{1000}{\centi\cubic\meter}. Pour remplir \unit{205}{\centi\cubic\meter} il nous faut donc \unit{0.205}{\liter}.


\end{corrige}
