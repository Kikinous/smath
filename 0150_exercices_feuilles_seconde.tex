% This is part of Un soupçon de mathématique sans être agressif pour autant
% Copyright (c) 2014
%   Laurent Claessens
% See the file fdl-1.3.txt for copying conditions.

%+++++++++++++++++++++++++++++++++++++++++++++++++++++++++++++++++++++++++++++++++++++++++++++++++++++++++++++++++++++++++++ 
    \section{Repères, distance et milieu}
%+++++++++++++++++++++++++++++++++++++++++++++++++++++++++++++++++++++++++++++++++++++++++++++++++++++++++++++++++++++++++++
    \justification

% Placer
\Exo{smath-0480}
\Exo{smath-0481}
\Exo{smath-0482}
\Exo{Seconde-0007}

% Distance
\Exo{Seconde-0008}
\Exo{smath-0483}
\Exo{smath-0475}
\Exo{smath-0627}

\Exo{smath-0293}
\Exo{Seconde-0006}
\Exo{Seconde-0009}


% Milieu
\Exo{smath-0484}
\Exo{smath-0485}
\Exo{smath-0019}
\Exo{smath-0486}

\Exo{smath-0487}
\Exo{smath-0624}
\Exo{Seconde-0012}
\Exo{smath-0488}
\Exo{Seconde-0004}
\Exo{Seconde-0056}
\Exo{smath-0476}
\Exo{smath-0478}


\Exo{Seconde-0055}
\Exo{Seconde-0011}

\Exo{smath-0125}
\Exo{smath-0410}

\Exo{smath-0124}
\Exo{Seconde-0005}
\Exo{Seconde-0010}
\Exo{Seconde-0020}


% Repère pas ON
\Exo{smath-0020}


%+++++++++++++++++++++++++++++++++++++++++++++++++++++++++++++++++++++++++++++++++++++++++++++++++++++++++++++++++++++++++++ 
\section{Fonctions linéaires et affines}
%+++++++++++++++++++++++++++++++++++++++++++++++++++++++++++++++++++++++++++++++++++++++++++++++++++++++++++++++++++++++++++

    
%\corrDraft{1}

\Exo{smath-0514}
    \Exo{smath-0149}
\Exo{smath-0513}
    \Exo{smath-0500}
    \Exo{smath-0494}
    \Exo{smath-0491}
    \Exo{smath-0127}
    \Exo{smath-0490}
    \Exo{smath-0016}
    \Exo{smath-0135}
    \Exo{smath-0148}
    \Exo{smath-0001}
    \Exo{smath-0502}
\Exo{smath-0625}
    \Exo{smath-0492}
    \Exo{smath-0497}
    \Exo{smath-0496}
\Exo{smath-0626}

    \Exo{smath-0499}
    \Exo{smath-0498}
    \Exo{smath-0501}
    \Exo{smath-0504}
    \Exo{smath-0503}
    \Exo{smath-0134}


    \clearpage
%+++++++++++++++++++++++++++++++++++++++++++++++++++++++++++++++++++++++++++++++++++++++++++++++++++++++++++++++++++++++++++ 
\section{Représentation graphique de fonctions}
%+++++++++++++++++++++++++++++++++++++++++++++++++++++++++++++++++++++++++++++++++++++++++++++++++++++++++++++++++++++++++++


\Exo{smath-0322}    % Exercices sur les intervalles
 %   \corrDraft{0}
\Exo{Seconde-0072}
\Exo{smath-0629}
\Exo{Seconde-0070}


\vspace{3cm}

    \Exo{smath-0209}
    \Exo{Seconde-0049}
    \Exo{smath-0298}

    \Exo{Seconde-0043}
    \Exo{smath-0489}
    \Exo{Seconde-0075}
    \Exo{smath-0439}
\Exo{smath-0505}
\Exo{Seconde-0069}

    \clearpage
%+++++++++++++++++++++++++++++++++++++++++++++++++++++++++++++++++++++++++++++++++++++++++++++++++++++++++++++++++++++++++++ 
\section{Statistique descriptive}
%+++++++++++++++++++++++++++++++++++++++++++++++++++++++++++++++++++++++++++++++++++++++++++++++++++++++++++++++++++++++++++


%\corrDraft{1}

\Exo{smath-0534}    %DS
\Exo{smath-0536}
\Exo{smath-0537}
\Exo{Seconde-0035}
\Exo{smath-0538}
\Exo{Seconde-0028}
\Exo{Seconde-0073}

\Exo{smath-0532}        %DS
\Exo{smath-0533}    %DS
    \Exo{smath-0530}    %DS
    \Exo{smath-0535}
\Exo{Seconde-0014}
\Exo{Seconde-0037}


    \clearpage
%+++++++++++++++++++++++++++++++++++++++++++++++++++++++++++++++++++++++++++++++++++++++++++++++++++++++++++++++++++++++++++ 
\section{Géométrie dans l'espace}
%+++++++++++++++++++++++++++++++++++++++++++++++++++++++++++++++++++++++++++++++++++++++++++++++++++++++++++++++++++++++++++

% TODO : il faut plus d'exercices avec des droites à prolonger hors du cube. Et aussi avec des volumes qui ne sont pas des cubes, genre des
%           pavés pas droits.

   %\corrDraft{1}

% Intersections de plans et co

\Exo{smath-0081}
\Exo{Seconde-0087}
\Exo{smath-0525}
\Exo{smath-0079}
\Exo{smath-0115}
\Exo{smath-0523}

\Exo{smath-0524}
\Exo{smath-0114}
\Exo{smath-0009}
\Exo{smath-0113}


\Exo{smath-0094}
\Exo{smath-0529}

\clearpage

%+++++++++++++++++++++++++++++++++++++++++++++++++++++++++++++++++++++++++++++++++++++++++++++++++++++++++++++++++++++++++++ 
\section{Variations de fonctions}
%+++++++++++++++++++++++++++++++++++++++++++++++++++++++++++++++++++++++++++++++++++++++++++++++++++++++++++++++++++++++++++

   %\corrDraft{1}

\Exo{smath-0547}    % De DS des autres 
\vspace{0.5cm}
\Exo{smath-0552}    % De DS des autres
\vspace{2cm}
\Exo{smath-0548}    % De DS des autres 
\Exo{smath-0549}    % De DS des autres 
\Exo{smath-0546}    % De DS des autres
\Exo{smath-0550}    % De DS des autres
\Exo{smath-0553}    % DS
\Exo{smath-0554}

%+++++++++++++++++++++++++++++++++++++++++++++++++++++++++++++++++++++++++++++++++++++++++++++++++++++++++++++++++++++++++++ 
\section{Vecteurs}
%+++++++++++++++++++++++++++++++++++++++++++++++++++++++++++++++++++++++++++++++++++++++++++++++++++++++++++++++++++++++++++

\Exo{smath-0594}
\Exo{smath-0593}

\Exo{smath-0592}
\Exo{smath-0588}    % de Haag
\Exo{smath-0595}
\Exo{smath-0590}    % Haag          % Celui-ci est un exercice à découper en deux.
\Exo{smath-0103}
\Exo{smath-0332}
\Exo{smath-0067}

\Exo{smath-0596}
\Exo{smath-0591}    % Haag
\Exo{smath-0106}
\Exo{smath-0063}
\Exo{smath-0064}
\Exo{smath-0111}
\Exo{smath-0142}
\Exo{smath-0085}
\Exo{smath-0073}
\Exo{smath-0661}    % Celui-ci n'est pas des feuilles, mais il faut le considérer parce qu'il vient d'une formation.

%+++++++++++++++++++++++++++++++++++++++++++++++++++++++++++++++++++++++++++++++++++++++++++++++++++++++++++++++++++++++++++ 
\section{Équation de droite}
%+++++++++++++++++++++++++++++++++++++++++++++++++++++++++++++++++++++++++++++++++++++++++++++++++++++++++++++++++++++++++++

\Exo{smath-0607}
\Exo{smath-0613}
\Exo{smath-0608}
\Exo{smath-0620}
\Exo{smath-0130}
\Exo{smath-0615}
\Exo{smath-0616}
\Exo{smath-0614}
    \Exo{smath-0603}    % Haag
\Exo{smath-0605}    % Haag
\Exo{smath-0200}


\Exo{smath-0606}    % Haag
\vspace{1cm}
\Exo{smath-0609}    % Haag
\Exo{smath-0610}    % Haag
\Exo{smath-0622}
\Exo{smath-0611}    % Haag
\Exo{smath-0612}    % Haag
\Exo{smath-0617}
\Exo{smath-0618}
\Exo{smath-0110}
\Exo{smath-0084}
\Exo{smath-0623}

\Exo{smath-0619}
\Exo{smath-0621}
\Exo{smath-0451}
\Exo{smath-0450}
\Exo{smath-0234}
\Exo{smath-0649}
\Exo{smath-0413}

%+++++++++++++++++++++++++++++++++++++++++++++++++++++++++++++++++++++++++++++++++++++++++++++++++++++++++++++++++++++++++++ 
\section{Intervalle de confiance et de fluctuation (2A)}
%+++++++++++++++++++++++++++++++++++++++++++++++++++++++++++++++++++++++++++++++++++++++++++++++++++++++++++++++++++++++++++

% TODO : décommenter les citations vers oklaEg, et QSfMxlk

 %  \corrDraft{1}

\Exo{smath-0348}
    \Exo{smath-0329}
\Exo{smath-0328}
\Exo{smath-0376}
\Exo{smath-0576}
\Exo{smath-0349}
\Exo{smath-0702}
\Exo{smath-0681}
\Exo{smath-0441}
\Exo{smath-0417}

%+++++++++++++++++++++++++++++++++++++++++++++++++++++++++++++++++++++++++++++++++++++++++++++++++++++++++++++++++++++++++++ 
\section{Vecteurs : colinéarité (2D)}
%+++++++++++++++++++++++++++++++++++++++++++++++++++++++++++++++++++++++++++++++++++++++++++++++++++++++++++++++++++++++++++

\Exo{smath-0686}
\Exo{smath-0651}    % Haag; c'est mon vrai ou faux en vrac à compléter.
\Exo{smath-0689}
\Exo{smath-0687}
\Exo{smath-0688}
\Exo{smath-0692}
\Exo{smath-0690}    % Haag
\Exo{smath-0691}    % Haag
\Exo{smath-0653}
\Exo{smath-0663}    % Haag
\Exo{smath-0664}    % Haag
\Exo{smath-0705}

\Exo{smath-0665}    % Haag
\Exo{smath-0667}    % Haag
\Exo{smath-0704}    % Haag
\Exo{smath-0706}


%+++++++++++++++++++++++++++++++++++++++++++++++++++++++++++++++++++++++++++++++++++++++++++++++++++++++++++++++++++++++++++ 
\section{Le second degré}
%+++++++++++++++++++++++++++++++++++++++++++++++++++++++++++++++++++++++++++++++++++++++++++++++++++++++++++++++++++++++++++

   %\corrDraft{1}

\Exo{smath-0662}    % D'une formation; trajectoire d'un boulet de canon
\Exo{smath-0255}
\Exo{smath-0650}    % Haag
\Exo{smath-0656}

\Exo{smath-0655}    % Haag
\Exo{smath-0657}    % Haag
\Exo{smath-0658}    % Haag; l'oiseau qui plonge
\Exo{smath-0180}

\Exo{smath-0659}    % Haag
\Exo{smath-0660}    % Haag
\Exo{smath-0666}

% À partir d'ici ce sont les miens.
\Exo{smath-0139}
\Exo{smath-0177}
\Exo{smath-0141}
\Exo{smath-0252}
\Exo{smath-0253}

%Équations

\Exo{smath-0176}

% Inéquations
\Exo{smath-0175}
\Exo{smath-0140}

% Problèmes plus longs

\Exo{smath-0654}    % Haag; il faut le simplifier en ne demandant par exemple que où placer \( x\) pour que l'aire soit maximale.
                    % Mais attention : une correction est déjà tapée.

%+++++++++++++++++++++++++++++++++++++++++++++++++++++++++++++++++++++++++++++++++++++++++++++++++++++++++++++++++++++++++++ 
\section{Probabilités}
%+++++++++++++++++++++++++++++++++++++++++++++++++++++++++++++++++++++++++++++++++++++++++++++++++++++++++++++++++++++++++++
\justification
%\corrDraft{1}


\Exo{smath-0187}
\Exo{smath-0359}
\Exo{smath-0396}
\Exo{smath-0712}
\Exo{smath-0710}    % Haag
\Exo{smath-0708}    % Haag
\Exo{smath-0709}    % Haag
\Exo{smath-0214}
\Exo{smath-0280}
\Exo{smath-0287}
\Exo{smath-0347}
\Exo{smath-0358}
\Exo{smath-0357}
\Exo{smath-0192}
\Exo{smath-0198}
\Exo{smath-0188}
\Exo{smath-0281}
\Exo{smath-0282}
\Exo{smath-0279}

%+++++++++++++++++++++++++++++++++++++++++++++++++++++++++++++++++++++++++++++++++++++++++++++++++++++++++++++++++++++++++++ 
\section{Fonctions homographiques}
%+++++++++++++++++++++++++++++++++++++++++++++++++++++++++++++++++++++++++++++++++++++++++++++++++++++++++++++++++++++++++++
\justification
%\corrDraft{1}

\Exo{smath-0334}
\Exo{smath-0379}  
\Exo{smath-0368}
\Exo{smath-0316}
\Exo{smath-0377}
\Exo{smath-0137}
\Exo{smath-0144}
\Exo{smath-0335}
\Exo{smath-0336}
%inéquations
\Exo{smath-0341}
\Exo{smath-0324}
\Exo{smath-0343}
\Exo{smath-0340}
\Exo{smath-0325}
\Exo{smath-0319}
\Exo{smath-0408}
\Exo{smath-0320}
\Exo{smath-0342}
\Exo{smath-0352}
\Exo{smath-0353}
\Exo{smath-0370}
%\Exo{smath-0371}
\Exo{smath-0711}    % Haag
\Exo{smath-0372}


