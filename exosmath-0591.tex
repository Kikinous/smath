% This is part of Un soupçon de mathématique sans être agressif pour autant
% Copyright (c) 2013-2014
%   Laurent Claessens
% See the file fdl-1.3.txt for copying conditions.

\begin{exercice}\label{exosmath-0591}

    Placer les points \( A=(-1;-1)\), \( B=(3;1)\), \( C=(2;3)\) et \( D=(-1;1)\) dans un repère.
    \begin{enumerate}
        \item
            Est-ce que \( ABCD\) est un parallélogramme ?
        \item
            Construire le point \( E\) tel que \( \vect{ AE }=\vect{ AB }+\vect{ CB }\), et donner les coordonnées du point \( E\).
        \item
            Démontrer que \( \vect{ CB }=\vect{ BE } \). Qu'en déduire sur le point \( B\) par rapport au segment \( [BC]\) ?
    \end{enumerate}

    Faire l'exercice 39, page 214.

\corrref{smath-0591}
\end{exercice}
