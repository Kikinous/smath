% This is part of Un soupçon de mathématique sans être agressif pour autant
% Copyright (c) 2014
%   Laurent Claessens
% See the file fdl-1.3.txt for copying conditions.

\begin{exercice}\label{exosmath-0653}

    Nous considérons les points \( A(-8;-1)\), \( B(-4;7)\), \( C(5;8)\) et \( D(1;0)\).
        \begin{enumerate}
            \item
                Déterminer les coordonnées des vecteurs \( \vect{ AB }\) et \( \vect{ DC }\).
            \item
                Que peut-on en déduire à propos du quadrilatère \( ABCD\) ?
            \item
                Est-ce que \( ABCD\) est un losange ? Justifier.
            \item
                Nous définissons le point \( L\) par l'égalité 
                \begin{equation*}
                    \vect{ AL }=\frac{1}{ 3 }\vect{ AC }.
                \end{equation*}
                Montrer que les coordonnées de \( L\) sont \( \big( -\frac{ 11 }{ 3 };2 \big)\).
            \item
                Calculer les coordonnées du milieu \( J\) de \( [AB]\).
            \item
                Les ponts \( J\), \( L\) et \( D\) sont-ils alignés ?
        \end{enumerate}

\corrref{smath-0653}
\end{exercice}
