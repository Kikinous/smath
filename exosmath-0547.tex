% This is part of Un soupçon de mathématique sans être agressif pour autant
% Copyright (c) 2013
%   Laurent Claessens
% See the file fdl-1.3.txt for copying conditions.

\begin{exercice}\label{exosmath-0547}

%\begin{wrapfigure}{r}{8.0cm}
%   \vspace{-0.5cm}        % à adapter.
%   \centering
    \begin{center}
   \input{Fig_YGlrtNX.pstricks}
    \end{center}
%\end{wrapfigure}

Nous considérons la fonction \( f\) dont la représentation graphique dans un repère orthonormal est donnée ci-contre 
\begin{enumerate}
    \item
        Déterminer l'ensemble de définition de \( f\). Nous noterons \( D_f\) cet ensemble.
    \item
        Déterminer \( f(2)\).
    \item
        Déterminer les éventuels antécédents de \( 0\) par la fonction \( f\).
    \item
        Donner un réel ayant exactement deux antécédents par \( f\).
    \item
        Dresser son tableau de variations.
    \item
        Déterminer les valeurs de \( x\) pour lesquelles \( 0\leq f(x)\leq 1\).
    \item
        Donner le minimum et le maximum de \( f\) sur \( D_f\) ainsi que les valeurs de \( x\) pour lesquels ils sont atteints; même question sur \( \mathopen[ -3 ; 1 \mathclose]\).
    \item
        Dresser le tableau de signe de \( f\) sur \( D_f\).
\end{enumerate}

\corrref{smath-0547}
\end{exercice}
