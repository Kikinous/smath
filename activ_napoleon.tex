% This is part of Un soupçon de mathématique sans être agressif pour autant
% Copyright (c) 2014
%   Laurent Claessens
% See the file fdl-1.3.txt for copying conditions.

%--------------------------------------------------------------------------------------------------------------------------- 
\subsection*{Activité : Napoléon se place}
%---------------------------------------------------------------------------------------------------------------------------

Les Anglais et les Autrichiens ont pris position en deux points \( A\) et \( B\) distants de \( \SI{10}{\kilo\meter}\). Napoléon veut pouvoir être à même de les attaquer tous les deux d'égale manière et décide donc de se positionner en un point $N$ qui serait à égale distance de \( A\) que de \( B\).

Bien entendu il pourrait se placer au milieu du segment \( [AB]\), mais ainsi il se ferait trop facilement attaquer des deux côtés à la fois (pas fou le Corse!). Où peut-il se placer ? Faire un dessin pour l'aider.
