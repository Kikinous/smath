% This is part of Un soupçon de mathématique sans être agressif pour autant
% Copyright (c) 2013
%   Laurent Claessens
% See the file fdl-1.3.txt for copying conditions.

\begin{exercice}\label{exosmath-0488}

    Dans un repère orthonormé nous considérons les points \( A=(-1;2)\), \( B=(3;3)\), \( C=(1,-1)\) et \( D=(-1;-2)\).
    \begin{enumerate}
        \item
            Calculer coordonnées des milieux des côtés du quadrilatère \( QBCD\). Nous les nommons \( K\), \( L\), \( M\) et \( N\).
        \item
            Quelle est la nature du quadrilatère \( KLMN\) ?
    \end{enumerate}
    

\corrref{smath-0488}
\end{exercice}
