\vbox{1
\emph{Toutes les réponses doivent être justifiées par un calcul accompagné d'un raisonnement.}
\begin{enumerate}\item
Placer les points \( A=(-5;3)\), \( B=(1;9)\), \( C=(-9;8)\) et \( D=(6;-1)\) dans un repère orthonormé. Calculer la longueur du segment \( [AB]\) et les coordonnées du milieu du segment \( [CD]\).

 $l^2=72$,$l=6*sqrt(2)$,$M=(-1.5,3.5)$\item
Est-ce que le triangle formé par les points \( A(2;6)\), \( B(6;2)\) et \( C(17;7)\) est isocèle ?

False
\end{enumerate}
} 
\vspace{2cm}
\vbox{2
\emph{Toutes les réponses doivent être justifiées par un calcul accompagné d'un raisonnement.}
\begin{enumerate}\item
Placer les points \( A=(-5;3)\), \( B=(-6;-10)\), \( C=(10;-7)\) et \( D=(4;1)\) dans un repère orthonormé. Calculer la longueur du segment \( [AB]\) et les coordonnées du milieu du segment \( [CD]\).

 $l^2=170$,$l=sqrt(170)$,$M=(7.0,-3.0)$\item
Est-ce que le triangle formé par les points \( A(1;-1)\), \( B(-5;5)\) et \( C(2;6)\) est isocèle ?

True
\end{enumerate}
} 
\vspace{2cm}
\vbox{3
\emph{Toutes les réponses doivent être justifiées par un calcul accompagné d'un raisonnement.}
\begin{enumerate}\item
Placer les points \( A=(0;-9)\), \( B=(6;9)\), \( C=(-1;9)\) et \( D=(8;10)\) dans un repère orthonormé. Calculer la longueur du segment \( [AB]\) et les coordonnées du milieu du segment \( [CD]\).

 $l^2=360$,$l=6*sqrt(10)$,$M=(3.5,9.5)$\item
Est-ce que le triangle formé par les points \( A(-6;-2)\), \( B(-11;3)\) et \( C(1;-5)\) est rectangle ?

False
\end{enumerate}
} 
\vspace{2cm}
\vbox{4
\emph{Toutes les réponses doivent être justifiées par un calcul accompagné d'un raisonnement.}
\begin{enumerate}\item
Placer les points \( A=(4;-6)\), \( B=(-7;-2)\), \( C=(-8;-8)\) et \( D=(-4;-2)\) dans un repère orthonormé. Calculer la longueur du segment \( [AB]\) et les coordonnées du milieu du segment \( [CD]\).

 $l^2=137$,$l=sqrt(137)$,$M=(-6.0,-5.0)$\item
Est-ce que le triangle formé par les points \( A(-3;-10)\), \( B(-7;-6)\) et \( C(4;-9)\) est isocèle ?

False
\end{enumerate}
} 
\vspace{2cm}
\vbox{5
\emph{Toutes les réponses doivent être justifiées par un calcul accompagné d'un raisonnement.}
\begin{enumerate}\item
Placer les points \( A=(6;6)\), \( B=(-10;-9)\), \( C=(-4;9)\) et \( D=(-4;2)\) dans un repère orthonormé. Calculer la longueur du segment \( [AB]\) et les coordonnées du milieu du segment \( [CD]\).

 $l^2=481$,$l=sqrt(481)$,$M=(-4.0,5.5)$\item
Est-ce que le triangle formé par les points \( A(7;10)\), \( B(1;16)\) et \( C(18;17)\) est isocèle ?

False
\end{enumerate}
} 
\vspace{2cm}
\vbox{6
\emph{Toutes les réponses doivent être justifiées par un calcul accompagné d'un raisonnement.}
\begin{enumerate}\item
Placer les points \( A=(-7;7)\), \( B=(-4;0)\), \( C=(4;-2)\) et \( D=(10;-8)\) dans un repère orthonormé. Calculer la longueur du segment \( [AB]\) et les coordonnées du milieu du segment \( [CD]\).

 $l^2=58$,$l=sqrt(58)$,$M=(7.0,-5.0)$\item
Est-ce que le triangle formé par les points \( A(-2;10)\), \( B(-1;9)\) et \( C(-4;8)\) est rectangle ?

True
\end{enumerate}
} 
\vspace{2cm}
\vbox{7
\emph{Toutes les réponses doivent être justifiées par un calcul accompagné d'un raisonnement.}
\begin{enumerate}\item
Placer les points \( A=(8;2)\), \( B=(7;5)\), \( C=(-10;5)\) et \( D=(1;10)\) dans un repère orthonormé. Calculer la longueur du segment \( [AB]\) et les coordonnées du milieu du segment \( [CD]\).

 $l^2=10$,$l=sqrt(10)$,$M=(-4.5,7.5)$\item
Est-ce que le triangle formé par les points \( A(8;-1)\), \( B(4;3)\) et \( C(10;1)\) est rectangle ?

True
\end{enumerate}
} 
\vspace{2cm}
\vbox{8
\emph{Toutes les réponses doivent être justifiées par un calcul accompagné d'un raisonnement.}
\begin{enumerate}\item
Placer les points \( A=(-4;-8)\), \( B=(0;8)\), \( C=(-2;1)\) et \( D=(0;-3)\) dans un repère orthonormé. Calculer la longueur du segment \( [AB]\) et les coordonnées du milieu du segment \( [CD]\).

 $l^2=272$,$l=4*sqrt(17)$,$M=(-1.0,-1.0)$\item
Est-ce que le triangle formé par les points \( A(4;-4)\), \( B(1;-1)\) et \( C(7;-1)\) est rectangle ?

True
\end{enumerate}
} 
\vspace{2cm}
\vbox{9
\emph{Toutes les réponses doivent être justifiées par un calcul accompagné d'un raisonnement.}
\begin{enumerate}\item
Placer les points \( A=(8;-5)\), \( B=(8;-2)\), \( C=(8;2)\) et \( D=(0;10)\) dans un repère orthonormé. Calculer la longueur du segment \( [AB]\) et les coordonnées du milieu du segment \( [CD]\).

 $l^2=9$,$l=3$,$M=(4.0,6.0)$\item
Est-ce que le triangle formé par les points \( A(-6;6)\), \( B(-1;1)\) et \( C(3;5)\) est rectangle ?

False
\end{enumerate}
} 
\vspace{2cm}
\vbox{10
\emph{Toutes les réponses doivent être justifiées par un calcul accompagné d'un raisonnement.}
\begin{enumerate}\item
Placer les points \( A=(4;-2)\), \( B=(7;9)\), \( C=(-10;-9)\) et \( D=(5;-2)\) dans un repère orthonormé. Calculer la longueur du segment \( [AB]\) et les coordonnées du milieu du segment \( [CD]\).

 $l^2=130$,$l=sqrt(130)$,$M=(-2.5,-5.5)$\item
Est-ce que le triangle formé par les points \( A(-1;-2)\), \( B(-4;1)\) et \( C(1;0)\) est rectangle ?

True
\end{enumerate}
} 
\vspace{2cm}
\vbox{11
\emph{Toutes les réponses doivent être justifiées par un calcul accompagné d'un raisonnement.}
\begin{enumerate}\item
Placer les points \( A=(5;9)\), \( B=(-7;5)\), \( C=(7;-1)\) et \( D=(9;-6)\) dans un repère orthonormé. Calculer la longueur du segment \( [AB]\) et les coordonnées du milieu du segment \( [CD]\).

 $l^2=160$,$l=4*sqrt(10)$,$M=(8.0,-3.5)$\item
Est-ce que le triangle formé par les points \( A(3;9)\), \( B(-3;15)\) et \( C(8;10)\) est isocèle ?

False
\end{enumerate}
} 
\vspace{2cm}
\vbox{12
\emph{Toutes les réponses doivent être justifiées par un calcul accompagné d'un raisonnement.}
\begin{enumerate}\item
Placer les points \( A=(-10;-5)\), \( B=(-10;1)\), \( C=(-5;5)\) et \( D=(0;-3)\) dans un repère orthonormé. Calculer la longueur du segment \( [AB]\) et les coordonnées du milieu du segment \( [CD]\).

 $l^2=36$,$l=6$,$M=(-2.5,1.0)$\item
Est-ce que le triangle formé par les points \( A(6;-1)\), \( B(12;-7)\) et \( C(13;0)\) est isocèle ?

True
\end{enumerate}
} 
\vspace{2cm}
\vbox{13
\emph{Toutes les réponses doivent être justifiées par un calcul accompagné d'un raisonnement.}
\begin{enumerate}\item
Placer les points \( A=(10;6)\), \( B=(-5;-7)\), \( C=(-1;7)\) et \( D=(-8;5)\) dans un repère orthonormé. Calculer la longueur du segment \( [AB]\) et les coordonnées du milieu du segment \( [CD]\).

 $l^2=394$,$l=sqrt(394)$,$M=(-4.5,6.0)$\item
Est-ce que le triangle formé par les points \( A(-10;-10)\), \( B(-16;-4)\) et \( C(0;-4)\) est isocèle ?

False
\end{enumerate}
} 
\vspace{2cm}
\vbox{14
\emph{Toutes les réponses doivent être justifiées par un calcul accompagné d'un raisonnement.}
\begin{enumerate}\item
Placer les points \( A=(1;-7)\), \( B=(9;-3)\), \( C=(5;7)\) et \( D=(-6;-5)\) dans un repère orthonormé. Calculer la longueur du segment \( [AB]\) et les coordonnées du milieu du segment \( [CD]\).

 $l^2=80$,$l=4*sqrt(5)$,$M=(-0.5,1.0)$\item
Est-ce que le triangle formé par les points \( A(-3;-8)\), \( B(-9;-2)\) et \( C(-9;-8)\) est isocèle ?

True
\end{enumerate}
} 
\vspace{2cm}
\vbox{15
\emph{Toutes les réponses doivent être justifiées par un calcul accompagné d'un raisonnement.}
\begin{enumerate}\item
Placer les points \( A=(-7;-7)\), \( B=(-1;-9)\), \( C=(10;-8)\) et \( D=(-9;4)\) dans un repère orthonormé. Calculer la longueur du segment \( [AB]\) et les coordonnées du milieu du segment \( [CD]\).

 $l^2=40$,$l=2*sqrt(10)$,$M=(0.5,-2.0)$\item
Est-ce que le triangle formé par les points \( A(2;-10)\), \( B(7;-15)\) et \( C(1;-11)\) est rectangle ?

True
\end{enumerate}
} 
\vspace{2cm}
\vbox{16
\emph{Toutes les réponses doivent être justifiées par un calcul accompagné d'un raisonnement.}
\begin{enumerate}\item
Placer les points \( A=(-3;-10)\), \( B=(-10;-9)\), \( C=(-2;-4)\) et \( D=(-7;3)\) dans un repère orthonormé. Calculer la longueur du segment \( [AB]\) et les coordonnées du milieu du segment \( [CD]\).

 $l^2=50$,$l=5*sqrt(2)$,$M=(-4.5,-0.5)$\item
Est-ce que le triangle formé par les points \( A(-5;-2)\), \( B(0;-7)\) et \( C(-8;-5)\) est rectangle ?

True
\end{enumerate}
} 
\vspace{2cm}
\vbox{17
\emph{Toutes les réponses doivent être justifiées par un calcul accompagné d'un raisonnement.}
\begin{enumerate}\item
Placer les points \( A=(3;6)\), \( B=(7;8)\), \( C=(6;9)\) et \( D=(8;-1)\) dans un repère orthonormé. Calculer la longueur du segment \( [AB]\) et les coordonnées du milieu du segment \( [CD]\).

 $l^2=20$,$l=2*sqrt(5)$,$M=(7.0,4.0)$\item
Est-ce que le triangle formé par les points \( A(6;7)\), \( B(8;5)\) et \( C(21;10)\) est isocèle ?

False
\end{enumerate}
} 
\vspace{2cm}
\vbox{18
\emph{Toutes les réponses doivent être justifiées par un calcul accompagné d'un raisonnement.}
\begin{enumerate}\item
Placer les points \( A=(-1;-10)\), \( B=(4;-4)\), \( C=(-4;-1)\) et \( D=(-5;1)\) dans un repère orthonormé. Calculer la longueur du segment \( [AB]\) et les coordonnées du milieu du segment \( [CD]\).

 $l^2=61$,$l=sqrt(61)$,$M=(-4.5,0.0)$\item
Est-ce que le triangle formé par les points \( A(-10;-4)\), \( B(-11;-3)\) et \( C(-1;-5)\) est rectangle ?

False
\end{enumerate}
} 
\vspace{2cm}
\vbox{19
\emph{Toutes les réponses doivent être justifiées par un calcul accompagné d'un raisonnement.}
\begin{enumerate}\item
Placer les points \( A=(-9;0)\), \( B=(-4;-1)\), \( C=(3;-2)\) et \( D=(7;-6)\) dans un repère orthonormé. Calculer la longueur du segment \( [AB]\) et les coordonnées du milieu du segment \( [CD]\).

 $l^2=26$,$l=sqrt(26)$,$M=(5.0,-4.0)$\item
Est-ce que le triangle formé par les points \( A(10;-6)\), \( B(13;-9)\) et \( C(23;-3)\) est rectangle ?

False
\end{enumerate}
} 
\vspace{2cm}
\vbox{20
\emph{Toutes les réponses doivent être justifiées par un calcul accompagné d'un raisonnement.}
\begin{enumerate}\item
Placer les points \( A=(-5;0)\), \( B=(6;8)\), \( C=(4;5)\) et \( D=(-4;-6)\) dans un repère orthonormé. Calculer la longueur du segment \( [AB]\) et les coordonnées du milieu du segment \( [CD]\).

 $l^2=185$,$l=sqrt(185)$,$M=(0.0,-0.5)$\item
Est-ce que le triangle formé par les points \( A(1;-7)\), \( B(5;-11)\) et \( C(5;-7)\) est isocèle ?

True
\end{enumerate}
} 
\vspace{2cm}
\vbox{21
\emph{Toutes les réponses doivent être justifiées par un calcul accompagné d'un raisonnement.}
\begin{enumerate}\item
Placer les points \( A=(-4;5)\), \( B=(0;-10)\), \( C=(7;-4)\) et \( D=(4;9)\) dans un repère orthonormé. Calculer la longueur du segment \( [AB]\) et les coordonnées du milieu du segment \( [CD]\).

 $l^2=241$,$l=sqrt(241)$,$M=(5.5,2.5)$\item
Est-ce que le triangle formé par les points \( A(5;-4)\), \( B(1;0)\) et \( C(17;-2)\) est rectangle ?

False
\end{enumerate}
} 
\vspace{2cm}
\vbox{22
\emph{Toutes les réponses doivent être justifiées par un calcul accompagné d'un raisonnement.}
\begin{enumerate}\item
Placer les points \( A=(8;8)\), \( B=(-3;-2)\), \( C=(3;2)\) et \( D=(-8;9)\) dans un repère orthonormé. Calculer la longueur du segment \( [AB]\) et les coordonnées du milieu du segment \( [CD]\).

 $l^2=221$,$l=sqrt(221)$,$M=(-2.5,5.5)$\item
Est-ce que le triangle formé par les points \( A(1;-1)\), \( B(-1;1)\) et \( C(1;1)\) est isocèle ?

True
\end{enumerate}
} 
\vspace{2cm}
\vbox{23
\emph{Toutes les réponses doivent être justifiées par un calcul accompagné d'un raisonnement.}
\begin{enumerate}\item
Placer les points \( A=(8;10)\), \( B=(-7;1)\), \( C=(-9;5)\) et \( D=(-5;8)\) dans un repère orthonormé. Calculer la longueur du segment \( [AB]\) et les coordonnées du milieu du segment \( [CD]\).

 $l^2=306$,$l=3*sqrt(34)$,$M=(-7.0,6.5)$\item
Est-ce que le triangle formé par les points \( A(0;9)\), \( B(-4;13)\) et \( C(7;6)\) est rectangle ?

False
\end{enumerate}
} 
\vspace{2cm}
\vbox{24
\emph{Toutes les réponses doivent être justifiées par un calcul accompagné d'un raisonnement.}
\begin{enumerate}\item
Placer les points \( A=(4;-2)\), \( B=(8;1)\), \( C=(4;-3)\) et \( D=(-3;6)\) dans un repère orthonormé. Calculer la longueur du segment \( [AB]\) et les coordonnées du milieu du segment \( [CD]\).

 $l^2=25$,$l=5$,$M=(0.5,1.5)$\item
Est-ce que le triangle formé par les points \( A(1;4)\), \( B(-4;9)\) et \( C(9;2)\) est rectangle ?

False
\end{enumerate}
} 
\vspace{2cm}
\vbox{25
\emph{Toutes les réponses doivent être justifiées par un calcul accompagné d'un raisonnement.}
\begin{enumerate}\item
Placer les points \( A=(2;-5)\), \( B=(-9;4)\), \( C=(5;4)\) et \( D=(-7;4)\) dans un repère orthonormé. Calculer la longueur du segment \( [AB]\) et les coordonnées du milieu du segment \( [CD]\).

 $l^2=202$,$l=sqrt(202)$,$M=(-1.0,4.0)$\item
Est-ce que le triangle formé par les points \( A(-4;3)\), \( B(-2;1)\) et \( C(-3;4)\) est rectangle ?

True
\end{enumerate}
} 
\vspace{2cm}
\vbox{26
\emph{Toutes les réponses doivent être justifiées par un calcul accompagné d'un raisonnement.}
\begin{enumerate}\item
Placer les points \( A=(1;3)\), \( B=(-4;-8)\), \( C=(-6;9)\) et \( D=(-7;9)\) dans un repère orthonormé. Calculer la longueur du segment \( [AB]\) et les coordonnées du milieu du segment \( [CD]\).

 $l^2=146$,$l=sqrt(146)$,$M=(-6.5,9.0)$\item
Est-ce que le triangle formé par les points \( A(-5;3)\), \( B(-11;9)\) et \( C(-6;8)\) est isocèle ?

True
\end{enumerate}
} 
\vspace{2cm}
\vbox{27
\emph{Toutes les réponses doivent être justifiées par un calcul accompagné d'un raisonnement.}
\begin{enumerate}\item
Placer les points \( A=(2;-2)\), \( B=(8;-5)\), \( C=(-1;1)\) et \( D=(10;2)\) dans un repère orthonormé. Calculer la longueur du segment \( [AB]\) et les coordonnées du milieu du segment \( [CD]\).

 $l^2=45$,$l=3*sqrt(5)$,$M=(4.5,1.5)$\item
Est-ce que le triangle formé par les points \( A(1;7)\), \( B(-2;10)\) et \( C(14;10)\) est rectangle ?

False
\end{enumerate}
} 
\vspace{2cm}
\vbox{28
\emph{Toutes les réponses doivent être justifiées par un calcul accompagné d'un raisonnement.}
\begin{enumerate}\item
Placer les points \( A=(7;5)\), \( B=(1;0)\), \( C=(10;1)\) et \( D=(1;-9)\) dans un repère orthonormé. Calculer la longueur du segment \( [AB]\) et les coordonnées du milieu du segment \( [CD]\).

 $l^2=61$,$l=sqrt(61)$,$M=(5.5,-4.0)$\item
Est-ce que le triangle formé par les points \( A(-9;-4)\), \( B(-12;-1)\) et \( C(-11;-6)\) est rectangle ?

True
\end{enumerate}
} 
\vspace{2cm}
\vbox{29
\emph{Toutes les réponses doivent être justifiées par un calcul accompagné d'un raisonnement.}
\begin{enumerate}\item
Placer les points \( A=(-2;-3)\), \( B=(-7;4)\), \( C=(-10;-6)\) et \( D=(-10;-3)\) dans un repère orthonormé. Calculer la longueur du segment \( [AB]\) et les coordonnées du milieu du segment \( [CD]\).

 $l^2=74$,$l=sqrt(74)$,$M=(-10.0,-4.5)$\item
Est-ce que le triangle formé par les points \( A(9;6)\), \( B(10;5)\) et \( C(18;5)\) est rectangle ?

False
\end{enumerate}
} 
\vspace{2cm}
\vbox{30
\emph{Toutes les réponses doivent être justifiées par un calcul accompagné d'un raisonnement.}
\begin{enumerate}\item
Placer les points \( A=(6;9)\), \( B=(1;9)\), \( C=(3;9)\) et \( D=(1;8)\) dans un repère orthonormé. Calculer la longueur du segment \( [AB]\) et les coordonnées du milieu du segment \( [CD]\).

 $l^2=25$,$l=5$,$M=(2.0,8.5)$\item
Est-ce que le triangle formé par les points \( A(10;4)\), \( B(8;6)\) et \( C(19;3)\) est rectangle ?

False
\end{enumerate}
} 
\vspace{2cm}
\vbox{31
\emph{Toutes les réponses doivent être justifiées par un calcul accompagné d'un raisonnement.}
\begin{enumerate}\item
Placer les points \( A=(-1;7)\), \( B=(3;-9)\), \( C=(7;-1)\) et \( D=(-7;-10)\) dans un repère orthonormé. Calculer la longueur du segment \( [AB]\) et les coordonnées du milieu du segment \( [CD]\).

 $l^2=272$,$l=4*sqrt(17)$,$M=(0.0,-5.5)$\item
Est-ce que le triangle formé par les points \( A(-9;3)\), \( B(-5;-1)\) et \( C(-4;4)\) est isocèle ?

True
\end{enumerate}
} 
\vspace{2cm}
\vbox{32
\emph{Toutes les réponses doivent être justifiées par un calcul accompagné d'un raisonnement.}
\begin{enumerate}\item
Placer les points \( A=(7;8)\), \( B=(-5;3)\), \( C=(1;7)\) et \( D=(-10;-10)\) dans un repère orthonormé. Calculer la longueur du segment \( [AB]\) et les coordonnées du milieu du segment \( [CD]\).

 $l^2=169$,$l=13$,$M=(-4.5,-1.5)$\item
Est-ce que le triangle formé par les points \( A(-4;5)\), \( B(-7;8)\) et \( C(5;4)\) est rectangle ?

False
\end{enumerate}
} 
\vspace{2cm}
\vbox{33
\emph{Toutes les réponses doivent être justifiées par un calcul accompagné d'un raisonnement.}
\begin{enumerate}\item
Placer les points \( A=(-1;4)\), \( B=(10;8)\), \( C=(4;-8)\) et \( D=(5;7)\) dans un repère orthonormé. Calculer la longueur du segment \( [AB]\) et les coordonnées du milieu du segment \( [CD]\).

 $l^2=137$,$l=sqrt(137)$,$M=(4.5,-0.5)$\item
Est-ce que le triangle formé par les points \( A(-9;6)\), \( B(-14;11)\) et \( C(-11;4)\) est rectangle ?

True
\end{enumerate}
} 
\vspace{2cm}
\vbox{34
\emph{Toutes les réponses doivent être justifiées par un calcul accompagné d'un raisonnement.}
\begin{enumerate}\item
Placer les points \( A=(-5;-2)\), \( B=(9;10)\), \( C=(7;0)\) et \( D=(-6;8)\) dans un repère orthonormé. Calculer la longueur du segment \( [AB]\) et les coordonnées du milieu du segment \( [CD]\).

 $l^2=340$,$l=2*sqrt(85)$,$M=(0.5,4.0)$\item
Est-ce que le triangle formé par les points \( A(-6;-7)\), \( B(-10;-3)\) et \( C(3;-4)\) est isocèle ?

False
\end{enumerate}
} 
\vspace{2cm}
\vbox{35
\emph{Toutes les réponses doivent être justifiées par un calcul accompagné d'un raisonnement.}
\begin{enumerate}\item
Placer les points \( A=(-4;-9)\), \( B=(-6;6)\), \( C=(-6;8)\) et \( D=(8;-8)\) dans un repère orthonormé. Calculer la longueur du segment \( [AB]\) et les coordonnées du milieu du segment \( [CD]\).

 $l^2=229$,$l=sqrt(229)$,$M=(1.0,0.0)$\item
Est-ce que le triangle formé par les points \( A(-6;-4)\), \( B(-9;-1)\) et \( C(2;-6)\) est rectangle ?

False
\end{enumerate}
} 
\vspace{2cm}
\vbox{36
\emph{Toutes les réponses doivent être justifiées par un calcul accompagné d'un raisonnement.}
\begin{enumerate}\item
Placer les points \( A=(1;-4)\), \( B=(9;1)\), \( C=(-8;-1)\) et \( D=(-1;7)\) dans un repère orthonormé. Calculer la longueur du segment \( [AB]\) et les coordonnées du milieu du segment \( [CD]\).

 $l^2=89$,$l=sqrt(89)$,$M=(-4.5,3.0)$\item
Est-ce que le triangle formé par les points \( A(2;-10)\), \( B(8;-16)\) et \( C(11;-17)\) est isocèle ?

False
\end{enumerate}
} 
\vspace{2cm}
\vbox{37
\emph{Toutes les réponses doivent être justifiées par un calcul accompagné d'un raisonnement.}
\begin{enumerate}\item
Placer les points \( A=(0;5)\), \( B=(7;4)\), \( C=(-2;-6)\) et \( D=(3;-6)\) dans un repère orthonormé. Calculer la longueur du segment \( [AB]\) et les coordonnées du milieu du segment \( [CD]\).

 $l^2=50$,$l=5*sqrt(2)$,$M=(0.5,-6.0)$\item
Est-ce que le triangle formé par les points \( A(9;-3)\), \( B(15;-9)\) et \( C(15;-3)\) est isocèle ?

True
\end{enumerate}
} 
\vspace{2cm}
\vbox{38
\emph{Toutes les réponses doivent être justifiées par un calcul accompagné d'un raisonnement.}
\begin{enumerate}\item
Placer les points \( A=(1;0)\), \( B=(6;6)\), \( C=(7;3)\) et \( D=(1;8)\) dans un repère orthonormé. Calculer la longueur du segment \( [AB]\) et les coordonnées du milieu du segment \( [CD]\).

 $l^2=61$,$l=sqrt(61)$,$M=(4.0,5.5)$\item
Est-ce que le triangle formé par les points \( A(0;-3)\), \( B(-3;0)\) et \( C(-2;-5)\) est rectangle ?

True
\end{enumerate}
} 
\vspace{2cm}
\vbox{39
\emph{Toutes les réponses doivent être justifiées par un calcul accompagné d'un raisonnement.}
\begin{enumerate}\item
Placer les points \( A=(-4;-4)\), \( B=(0;-3)\), \( C=(-6;1)\) et \( D=(-10;9)\) dans un repère orthonormé. Calculer la longueur du segment \( [AB]\) et les coordonnées du milieu du segment \( [CD]\).

 $l^2=17$,$l=sqrt(17)$,$M=(-8.0,5.0)$\item
Est-ce que le triangle formé par les points \( A(0;-6)\), \( B(2;-8)\) et \( C(5;-3)\) est isocèle ?

True
\end{enumerate}
} 
\vspace{2cm}
\vbox{40
\emph{Toutes les réponses doivent être justifiées par un calcul accompagné d'un raisonnement.}
\begin{enumerate}\item
Placer les points \( A=(6;-6)\), \( B=(6;-10)\), \( C=(8;-6)\) et \( D=(2;5)\) dans un repère orthonormé. Calculer la longueur du segment \( [AB]\) et les coordonnées du milieu du segment \( [CD]\).

 $l^2=16$,$l=4$,$M=(5.0,-0.5)$\item
Est-ce que le triangle formé par les points \( A(10;3)\), \( B(8;5)\) et \( C(7;2)\) est isocèle ?

True
\end{enumerate}
} 
\vspace{2cm}
\vbox{41
\emph{Toutes les réponses doivent être justifiées par un calcul accompagné d'un raisonnement.}
\begin{enumerate}\item
Placer les points \( A=(0;0)\), \( B=(6;-2)\), \( C=(-4;-3)\) et \( D=(8;7)\) dans un repère orthonormé. Calculer la longueur du segment \( [AB]\) et les coordonnées du milieu du segment \( [CD]\).

 $l^2=40$,$l=2*sqrt(10)$,$M=(2.0,2.0)$\item
Est-ce que le triangle formé par les points \( A(0;-5)\), \( B(1;-6)\) et \( C(9;-6)\) est rectangle ?

False
\end{enumerate}
} 
\vspace{2cm}
\vbox{42
\emph{Toutes les réponses doivent être justifiées par un calcul accompagné d'un raisonnement.}
\begin{enumerate}\item
Placer les points \( A=(-9;-8)\), \( B=(-5;8)\), \( C=(3;8)\) et \( D=(-4;5)\) dans un repère orthonormé. Calculer la longueur du segment \( [AB]\) et les coordonnées du milieu du segment \( [CD]\).

 $l^2=272$,$l=4*sqrt(17)$,$M=(-0.5,6.5)$\item
Est-ce que le triangle formé par les points \( A(2;-4)\), \( B(0;-2)\) et \( C(0;-4)\) est isocèle ?

True
\end{enumerate}
} 
\vspace{2cm}
\vbox{43
\emph{Toutes les réponses doivent être justifiées par un calcul accompagné d'un raisonnement.}
\begin{enumerate}\item
Placer les points \( A=(-7;-4)\), \( B=(7;-1)\), \( C=(6;-2)\) et \( D=(-9;2)\) dans un repère orthonormé. Calculer la longueur du segment \( [AB]\) et les coordonnées du milieu du segment \( [CD]\).

 $l^2=205$,$l=sqrt(205)$,$M=(-1.5,0.0)$\item
Est-ce que le triangle formé par les points \( A(1;-6)\), \( B(6;-11)\) et \( C(3;-4)\) est rectangle ?

True
\end{enumerate}
} 
\vspace{2cm}
\vbox{44
\emph{Toutes les réponses doivent être justifiées par un calcul accompagné d'un raisonnement.}
\begin{enumerate}\item
Placer les points \( A=(6;-3)\), \( B=(-9;-8)\), \( C=(8;-9)\) et \( D=(-6;-7)\) dans un repère orthonormé. Calculer la longueur du segment \( [AB]\) et les coordonnées du milieu du segment \( [CD]\).

 $l^2=250$,$l=5*sqrt(10)$,$M=(1.0,-8.0)$\item
Est-ce que le triangle formé par les points \( A(8;1)\), \( B(13;-4)\) et \( C(21;4)\) est rectangle ?

False
\end{enumerate}
} 
\vspace{2cm}
\vbox{45
\emph{Toutes les réponses doivent être justifiées par un calcul accompagné d'un raisonnement.}
\begin{enumerate}\item
Placer les points \( A=(7;1)\), \( B=(9;5)\), \( C=(-8;6)\) et \( D=(-3;-5)\) dans un repère orthonormé. Calculer la longueur du segment \( [AB]\) et les coordonnées du milieu du segment \( [CD]\).

 $l^2=20$,$l=2*sqrt(5)$,$M=(-5.5,0.5)$\item
Est-ce que le triangle formé par les points \( A(-6;-1)\), \( B(-2;-5)\) et \( C(7;2)\) est rectangle ?

False
\end{enumerate}
} 
\vspace{2cm}
\vbox{46
\emph{Toutes les réponses doivent être justifiées par un calcul accompagné d'un raisonnement.}
\begin{enumerate}\item
Placer les points \( A=(8;9)\), \( B=(5;-9)\), \( C=(-9;-5)\) et \( D=(-1;7)\) dans un repère orthonormé. Calculer la longueur du segment \( [AB]\) et les coordonnées du milieu du segment \( [CD]\).

 $l^2=333$,$l=3*sqrt(37)$,$M=(-5.0,1.0)$\item
Est-ce que le triangle formé par les points \( A(-3;8)\), \( B(-5;10)\) et \( C(9;12)\) est isocèle ?

False
\end{enumerate}
} 
\vspace{2cm}
\vbox{47
\emph{Toutes les réponses doivent être justifiées par un calcul accompagné d'un raisonnement.}
\begin{enumerate}\item
Placer les points \( A=(-8;-2)\), \( B=(10;-8)\), \( C=(0;-6)\) et \( D=(8;4)\) dans un repère orthonormé. Calculer la longueur du segment \( [AB]\) et les coordonnées du milieu du segment \( [CD]\).

 $l^2=360$,$l=6*sqrt(10)$,$M=(4.0,-1.0)$\item
Est-ce que le triangle formé par les points \( A(10;-10)\), \( B(12;-12)\) et \( C(9;-11)\) est rectangle ?

True
\end{enumerate}
} 
\vspace{2cm}
\vbox{48
\emph{Toutes les réponses doivent être justifiées par un calcul accompagné d'un raisonnement.}
\begin{enumerate}\item
Placer les points \( A=(4;-1)\), \( B=(-1;-8)\), \( C=(2;6)\) et \( D=(-5;-7)\) dans un repère orthonormé. Calculer la longueur du segment \( [AB]\) et les coordonnées du milieu du segment \( [CD]\).

 $l^2=74$,$l=sqrt(74)$,$M=(-1.5,-0.5)$\item
Est-ce que le triangle formé par les points \( A(-3;-2)\), \( B(-9;4)\) et \( C(0;-3)\) est isocèle ?

False
\end{enumerate}
} 
\vspace{2cm}
\vbox{49
\emph{Toutes les réponses doivent être justifiées par un calcul accompagné d'un raisonnement.}
\begin{enumerate}\item
Placer les points \( A=(9;9)\), \( B=(10;-3)\), \( C=(9;0)\) et \( D=(9;-9)\) dans un repère orthonormé. Calculer la longueur du segment \( [AB]\) et les coordonnées du milieu du segment \( [CD]\).

 $l^2=145$,$l=sqrt(145)$,$M=(9.0,-4.5)$\item
Est-ce que le triangle formé par les points \( A(3;-9)\), \( B(7;-13)\) et \( C(2;-10)\) est rectangle ?

True
\end{enumerate}
} 
\vspace{2cm}
\vbox{50
\emph{Toutes les réponses doivent être justifiées par un calcul accompagné d'un raisonnement.}
\begin{enumerate}\item
Placer les points \( A=(5;7)\), \( B=(-7;0)\), \( C=(-7;1)\) et \( D=(0;9)\) dans un repère orthonormé. Calculer la longueur du segment \( [AB]\) et les coordonnées du milieu du segment \( [CD]\).

 $l^2=193$,$l=sqrt(193)$,$M=(-3.5,5.0)$\item
Est-ce que le triangle formé par les points \( A(-6;-3)\), \( B(-11;2)\) et \( C(-4;-1)\) est rectangle ?

True
\end{enumerate}
} 
\vspace{2cm}
\vbox{51
\emph{Toutes les réponses doivent être justifiées par un calcul accompagné d'un raisonnement.}
\begin{enumerate}\item
Placer les points \( A=(-6;1)\), \( B=(-10;-3)\), \( C=(8;6)\) et \( D=(-8;2)\) dans un repère orthonormé. Calculer la longueur du segment \( [AB]\) et les coordonnées du milieu du segment \( [CD]\).

 $l^2=32$,$l=4*sqrt(2)$,$M=(0.0,4.0)$\item
Est-ce que le triangle formé par les points \( A(-7;7)\), \( B(-6;6)\) et \( C(-4;10)\) est rectangle ?

True
\end{enumerate}
} 
\vspace{2cm}
\vbox{52
\emph{Toutes les réponses doivent être justifiées par un calcul accompagné d'un raisonnement.}
\begin{enumerate}\item
Placer les points \( A=(3;-2)\), \( B=(5;8)\), \( C=(-3;10)\) et \( D=(4;4)\) dans un repère orthonormé. Calculer la longueur du segment \( [AB]\) et les coordonnées du milieu du segment \( [CD]\).

 $l^2=104$,$l=2*sqrt(26)$,$M=(0.5,7.0)$\item
Est-ce que le triangle formé par les points \( A(-4;8)\), \( B(-8;12)\) et \( C(-5;11)\) est isocèle ?

True
\end{enumerate}
} 
\vspace{2cm}
\vbox{53
\emph{Toutes les réponses doivent être justifiées par un calcul accompagné d'un raisonnement.}
\begin{enumerate}\item
Placer les points \( A=(10;-2)\), \( B=(-10;-4)\), \( C=(6;1)\) et \( D=(5;-4)\) dans un repère orthonormé. Calculer la longueur du segment \( [AB]\) et les coordonnées du milieu du segment \( [CD]\).

 $l^2=404$,$l=2*sqrt(101)$,$M=(5.5,-1.5)$\item
Est-ce que le triangle formé par les points \( A(-7;7)\), \( B(-3;3)\) et \( C(1;1)\) est isocèle ?

False
\end{enumerate}
} 
\vspace{2cm}
\vbox{54
\emph{Toutes les réponses doivent être justifiées par un calcul accompagné d'un raisonnement.}
\begin{enumerate}\item
Placer les points \( A=(-3;9)\), \( B=(5;-6)\), \( C=(-9;-8)\) et \( D=(-10;-4)\) dans un repère orthonormé. Calculer la longueur du segment \( [AB]\) et les coordonnées du milieu du segment \( [CD]\).

 $l^2=289$,$l=17$,$M=(-9.5,-6.0)$\item
Est-ce que le triangle formé par les points \( A(2;6)\), \( B(1;7)\) et \( C(14;8)\) est rectangle ?

False
\end{enumerate}
} 
\vspace{2cm}
\vbox{55
\emph{Toutes les réponses doivent être justifiées par un calcul accompagné d'un raisonnement.}
\begin{enumerate}\item
Placer les points \( A=(-9;-9)\), \( B=(-6;-2)\), \( C=(-1;-4)\) et \( D=(7;10)\) dans un repère orthonormé. Calculer la longueur du segment \( [AB]\) et les coordonnées du milieu du segment \( [CD]\).

 $l^2=58$,$l=sqrt(58)$,$M=(3.0,3.0)$\item
Est-ce que le triangle formé par les points \( A(1;10)\), \( B(2;9)\) et \( C(8;7)\) est rectangle ?

False
\end{enumerate}
} 
\vspace{2cm}
\vbox{56
\emph{Toutes les réponses doivent être justifiées par un calcul accompagné d'un raisonnement.}
\begin{enumerate}\item
Placer les points \( A=(6;-6)\), \( B=(9;-8)\), \( C=(9;7)\) et \( D=(-9;4)\) dans un repère orthonormé. Calculer la longueur du segment \( [AB]\) et les coordonnées du milieu du segment \( [CD]\).

 $l^2=13$,$l=sqrt(13)$,$M=(0.0,5.5)$\item
Est-ce que le triangle formé par les points \( A(-2;7)\), \( B(2;3)\) et \( C(3;8)\) est isocèle ?

True
\end{enumerate}
} 
\vspace{2cm}
\vbox{57
\emph{Toutes les réponses doivent être justifiées par un calcul accompagné d'un raisonnement.}
\begin{enumerate}\item
Placer les points \( A=(-9;4)\), \( B=(-2;9)\), \( C=(8;0)\) et \( D=(-10;-2)\) dans un repère orthonormé. Calculer la longueur du segment \( [AB]\) et les coordonnées du milieu du segment \( [CD]\).

 $l^2=74$,$l=sqrt(74)$,$M=(-1.0,-1.0)$\item
Est-ce que le triangle formé par les points \( A(-4;7)\), \( B(-8;11)\) et \( C(3;8)\) est isocèle ?

False
\end{enumerate}
} 
\vspace{2cm}
\vbox{58
\emph{Toutes les réponses doivent être justifiées par un calcul accompagné d'un raisonnement.}
\begin{enumerate}\item
Placer les points \( A=(-1;2)\), \( B=(-7;-5)\), \( C=(1;1)\) et \( D=(-10;7)\) dans un repère orthonormé. Calculer la longueur du segment \( [AB]\) et les coordonnées du milieu du segment \( [CD]\).

 $l^2=85$,$l=sqrt(85)$,$M=(-4.5,4.0)$\item
Est-ce que le triangle formé par les points \( A(-4;2)\), \( B(-2;0)\) et \( C(9;5)\) est rectangle ?

False
\end{enumerate}
} 
\vspace{2cm}
\vbox{59
\emph{Toutes les réponses doivent être justifiées par un calcul accompagné d'un raisonnement.}
\begin{enumerate}\item
Placer les points \( A=(6;3)\), \( B=(7;-7)\), \( C=(10;-1)\) et \( D=(3;0)\) dans un repère orthonormé. Calculer la longueur du segment \( [AB]\) et les coordonnées du milieu du segment \( [CD]\).

 $l^2=101$,$l=sqrt(101)$,$M=(6.5,-0.5)$\item
Est-ce que le triangle formé par les points \( A(-10;10)\), \( B(-6;6)\) et \( C(6;12)\) est isocèle ?

False
\end{enumerate}
} 
\vspace{2cm}
\vbox{60
\emph{Toutes les réponses doivent être justifiées par un calcul accompagné d'un raisonnement.}
\begin{enumerate}\item
Placer les points \( A=(3;-7)\), \( B=(-9;7)\), \( C=(-8;-5)\) et \( D=(-7;-5)\) dans un repère orthonormé. Calculer la longueur du segment \( [AB]\) et les coordonnées du milieu du segment \( [CD]\).

 $l^2=340$,$l=2*sqrt(85)$,$M=(-7.5,-5.0)$\item
Est-ce que le triangle formé par les points \( A(5;7)\), \( B(4;8)\) et \( C(12;4)\) est rectangle ?

False
\end{enumerate}
} 
\vspace{2cm}
\vbox{61
\emph{Toutes les réponses doivent être justifiées par un calcul accompagné d'un raisonnement.}
\begin{enumerate}\item
Placer les points \( A=(10;-5)\), \( B=(-5;-9)\), \( C=(4;2)\) et \( D=(-4;-3)\) dans un repère orthonormé. Calculer la longueur du segment \( [AB]\) et les coordonnées du milieu du segment \( [CD]\).

 $l^2=241$,$l=sqrt(241)$,$M=(0.0,-0.5)$\item
Est-ce que le triangle formé par les points \( A(2;3)\), \( B(-2;7)\) et \( C(-1;4)\) est isocèle ?

True
\end{enumerate}
} 
\vspace{2cm}
\vbox{62
\emph{Toutes les réponses doivent être justifiées par un calcul accompagné d'un raisonnement.}
\begin{enumerate}\item
Placer les points \( A=(-6;1)\), \( B=(-1;-3)\), \( C=(-8;-3)\) et \( D=(8;-6)\) dans un repère orthonormé. Calculer la longueur du segment \( [AB]\) et les coordonnées du milieu du segment \( [CD]\).

 $l^2=41$,$l=sqrt(41)$,$M=(0.0,-4.5)$\item
Est-ce que le triangle formé par les points \( A(-3;7)\), \( B(-5;9)\) et \( C(10;10)\) est rectangle ?

False
\end{enumerate}
} 
\vspace{2cm}
\vbox{63
\emph{Toutes les réponses doivent être justifiées par un calcul accompagné d'un raisonnement.}
\begin{enumerate}\item
Placer les points \( A=(-7;-4)\), \( B=(-9;-8)\), \( C=(-5;2)\) et \( D=(-3;-1)\) dans un repère orthonormé. Calculer la longueur du segment \( [AB]\) et les coordonnées du milieu du segment \( [CD]\).

 $l^2=20$,$l=2*sqrt(5)$,$M=(-4.0,0.5)$\item
Est-ce que le triangle formé par les points \( A(-1;10)\), \( B(1;8)\) et \( C(9;8)\) est isocèle ?

False
\end{enumerate}
} 
\vspace{2cm}
\vbox{64
\emph{Toutes les réponses doivent être justifiées par un calcul accompagné d'un raisonnement.}
\begin{enumerate}\item
Placer les points \( A=(-9;-6)\), \( B=(10;-3)\), \( C=(7;-10)\) et \( D=(7;-1)\) dans un repère orthonormé. Calculer la longueur du segment \( [AB]\) et les coordonnées du milieu du segment \( [CD]\).

 $l^2=370$,$l=sqrt(370)$,$M=(7.0,-5.5)$\item
Est-ce que le triangle formé par les points \( A(1;-4)\), \( B(5;-8)\) et \( C(10;-5)\) est rectangle ?

False
\end{enumerate}
} 
\vspace{2cm}
\vbox{65
\emph{Toutes les réponses doivent être justifiées par un calcul accompagné d'un raisonnement.}
\begin{enumerate}\item
Placer les points \( A=(-3;9)\), \( B=(7;6)\), \( C=(-4;3)\) et \( D=(1;-1)\) dans un repère orthonormé. Calculer la longueur du segment \( [AB]\) et les coordonnées du milieu du segment \( [CD]\).

 $l^2=109$,$l=sqrt(109)$,$M=(-1.5,1.0)$\item
Est-ce que le triangle formé par les points \( A(5;-7)\), \( B(3;-5)\) et \( C(13;-7)\) est isocèle ?

False
\end{enumerate}
} 
\vspace{2cm}
\vbox{66
\emph{Toutes les réponses doivent être justifiées par un calcul accompagné d'un raisonnement.}
\begin{enumerate}\item
Placer les points \( A=(-8;-7)\), \( B=(0;1)\), \( C=(8;2)\) et \( D=(-8;8)\) dans un repère orthonormé. Calculer la longueur du segment \( [AB]\) et les coordonnées du milieu du segment \( [CD]\).

 $l^2=128$,$l=8*sqrt(2)$,$M=(0.0,5.0)$\item
Est-ce que le triangle formé par les points \( A(-1;-7)\), \( B(3;-11)\) et \( C(12;-4)\) est rectangle ?

False
\end{enumerate}
} 
\vspace{2cm}
\vbox{67
\emph{Toutes les réponses doivent être justifiées par un calcul accompagné d'un raisonnement.}
\begin{enumerate}\item
Placer les points \( A=(3;-4)\), \( B=(5;-2)\), \( C=(-1;9)\) et \( D=(5;-6)\) dans un repère orthonormé. Calculer la longueur du segment \( [AB]\) et les coordonnées du milieu du segment \( [CD]\).

 $l^2=8$,$l=2*sqrt(2)$,$M=(2.0,1.5)$\item
Est-ce que le triangle formé par les points \( A(-7;4)\), \( B(-10;7)\) et \( C(1;2)\) est rectangle ?

False
\end{enumerate}
} 
\vspace{2cm}
\vbox{68
\emph{Toutes les réponses doivent être justifiées par un calcul accompagné d'un raisonnement.}
\begin{enumerate}\item
Placer les points \( A=(9;2)\), \( B=(-6;6)\), \( C=(-1;-8)\) et \( D=(-7;9)\) dans un repère orthonormé. Calculer la longueur du segment \( [AB]\) et les coordonnées du milieu du segment \( [CD]\).

 $l^2=241$,$l=sqrt(241)$,$M=(-4.0,0.5)$\item
Est-ce que le triangle formé par les points \( A(5;-6)\), \( B(9;-10)\) et \( C(20;-5)\) est isocèle ?

False
\end{enumerate}
} 
\vspace{2cm}
\vbox{69
\emph{Toutes les réponses doivent être justifiées par un calcul accompagné d'un raisonnement.}
\begin{enumerate}\item
Placer les points \( A=(-1;2)\), \( B=(0;-4)\), \( C=(-9;-4)\) et \( D=(8;5)\) dans un repère orthonormé. Calculer la longueur du segment \( [AB]\) et les coordonnées du milieu du segment \( [CD]\).

 $l^2=37$,$l=sqrt(37)$,$M=(-0.5,0.5)$\item
Est-ce que le triangle formé par les points \( A(7;-9)\), \( B(13;-15)\) et \( C(9;-13)\) est isocèle ?

True
\end{enumerate}
} 
\vspace{2cm}
\vbox{70
\emph{Toutes les réponses doivent être justifiées par un calcul accompagné d'un raisonnement.}
\begin{enumerate}\item
Placer les points \( A=(-1;0)\), \( B=(-1;2)\), \( C=(-2;5)\) et \( D=(9;-8)\) dans un repère orthonormé. Calculer la longueur du segment \( [AB]\) et les coordonnées du milieu du segment \( [CD]\).

 $l^2=4$,$l=2$,$M=(3.5,-1.5)$\item
Est-ce que le triangle formé par les points \( A(9;-2)\), \( B(6;1)\) et \( C(22;1)\) est rectangle ?

False
\end{enumerate}
} 
\vspace{2cm}
\vbox{71
\emph{Toutes les réponses doivent être justifiées par un calcul accompagné d'un raisonnement.}
\begin{enumerate}\item
Placer les points \( A=(5;-7)\), \( B=(8;6)\), \( C=(-8;9)\) et \( D=(-5;-6)\) dans un repère orthonormé. Calculer la longueur du segment \( [AB]\) et les coordonnées du milieu du segment \( [CD]\).

 $l^2=178$,$l=sqrt(178)$,$M=(-6.5,1.5)$\item
Est-ce que le triangle formé par les points \( A(6;-4)\), \( B(9;-7)\) et \( C(17;-3)\) est rectangle ?

False
\end{enumerate}
} 
\vspace{2cm}
\vbox{72
\emph{Toutes les réponses doivent être justifiées par un calcul accompagné d'un raisonnement.}
\begin{enumerate}\item
Placer les points \( A=(1;-6)\), \( B=(-1;-1)\), \( C=(-4;10)\) et \( D=(-2;-1)\) dans un repère orthonormé. Calculer la longueur du segment \( [AB]\) et les coordonnées du milieu du segment \( [CD]\).

 $l^2=29$,$l=sqrt(29)$,$M=(-3.0,4.5)$\item
Est-ce que le triangle formé par les points \( A(-5;9)\), \( B(-3;7)\) et \( C(0;12)\) est isocèle ?

True
\end{enumerate}
} 
\vspace{2cm}
\vbox{73
\emph{Toutes les réponses doivent être justifiées par un calcul accompagné d'un raisonnement.}
\begin{enumerate}\item
Placer les points \( A=(2;3)\), \( B=(0;4)\), \( C=(-6;-8)\) et \( D=(6;-4)\) dans un repère orthonormé. Calculer la longueur du segment \( [AB]\) et les coordonnées du milieu du segment \( [CD]\).

 $l^2=5$,$l=sqrt(5)$,$M=(0.0,-6.0)$\item
Est-ce que le triangle formé par les points \( A(7;-9)\), \( B(6;-8)\) et \( C(20;-6)\) est rectangle ?

False
\end{enumerate}
} 
\vspace{2cm}
\vbox{74
\emph{Toutes les réponses doivent être justifiées par un calcul accompagné d'un raisonnement.}
\begin{enumerate}\item
Placer les points \( A=(-1;1)\), \( B=(1;7)\), \( C=(-7;-7)\) et \( D=(8;-2)\) dans un repère orthonormé. Calculer la longueur du segment \( [AB]\) et les coordonnées du milieu du segment \( [CD]\).

 $l^2=40$,$l=2*sqrt(10)$,$M=(0.5,-4.5)$\item
Est-ce que le triangle formé par les points \( A(6;-6)\), \( B(10;-10)\) et \( C(20;-6)\) est isocèle ?

False
\end{enumerate}
} 
\vspace{2cm}
\vbox{75
\emph{Toutes les réponses doivent être justifiées par un calcul accompagné d'un raisonnement.}
\begin{enumerate}\item
Placer les points \( A=(2;-7)\), \( B=(-1;-9)\), \( C=(-4;10)\) et \( D=(4;-4)\) dans un repère orthonormé. Calculer la longueur du segment \( [AB]\) et les coordonnées du milieu du segment \( [CD]\).

 $l^2=13$,$l=sqrt(13)$,$M=(0.0,3.0)$\item
Est-ce que le triangle formé par les points \( A(10;0)\), \( B(4;6)\) et \( C(13;-1)\) est isocèle ?

False
\end{enumerate}
} 
\vspace{2cm}
\vbox{76
\emph{Toutes les réponses doivent être justifiées par un calcul accompagné d'un raisonnement.}
\begin{enumerate}\item
Placer les points \( A=(-2;-8)\), \( B=(-10;-5)\), \( C=(3;-10)\) et \( D=(-8;7)\) dans un repère orthonormé. Calculer la longueur du segment \( [AB]\) et les coordonnées du milieu du segment \( [CD]\).

 $l^2=73$,$l=sqrt(73)$,$M=(-2.5,-1.5)$\item
Est-ce que le triangle formé par les points \( A(6;-5)\), \( B(10;-9)\) et \( C(11;-4)\) est isocèle ?

True
\end{enumerate}
} 
\vspace{2cm}
\vbox{77
\emph{Toutes les réponses doivent être justifiées par un calcul accompagné d'un raisonnement.}
\begin{enumerate}\item
Placer les points \( A=(9;-1)\), \( B=(-10;-1)\), \( C=(-3;4)\) et \( D=(6;4)\) dans un repère orthonormé. Calculer la longueur du segment \( [AB]\) et les coordonnées du milieu du segment \( [CD]\).

 $l^2=361$,$l=19$,$M=(1.5,4.0)$\item
Est-ce que le triangle formé par les points \( A(0;-9)\), \( B(4;-13)\) et \( C(1;-12)\) est isocèle ?

True
\end{enumerate}
} 
\vspace{2cm}
\vbox{78
\emph{Toutes les réponses doivent être justifiées par un calcul accompagné d'un raisonnement.}
\begin{enumerate}\item
Placer les points \( A=(-7;-3)\), \( B=(7;-5)\), \( C=(3;-6)\) et \( D=(0;-3)\) dans un repère orthonormé. Calculer la longueur du segment \( [AB]\) et les coordonnées du milieu du segment \( [CD]\).

 $l^2=200$,$l=10*sqrt(2)$,$M=(1.5,-4.5)$\item
Est-ce que le triangle formé par les points \( A(6;-9)\), \( B(9;-12)\) et \( C(4;-11)\) est rectangle ?

True
\end{enumerate}
} 
\vspace{2cm}
\vbox{79
\emph{Toutes les réponses doivent être justifiées par un calcul accompagné d'un raisonnement.}
\begin{enumerate}\item
Placer les points \( A=(0;-8)\), \( B=(8;0)\), \( C=(-8;0)\) et \( D=(-7;6)\) dans un repère orthonormé. Calculer la longueur du segment \( [AB]\) et les coordonnées du milieu du segment \( [CD]\).

 $l^2=128$,$l=8*sqrt(2)$,$M=(-7.5,3.0)$\item
Est-ce que le triangle formé par les points \( A(10;8)\), \( B(6;12)\) et \( C(17;5)\) est rectangle ?

False
\end{enumerate}
} 
\vspace{2cm}
\vbox{80
\emph{Toutes les réponses doivent être justifiées par un calcul accompagné d'un raisonnement.}
\begin{enumerate}\item
Placer les points \( A=(-2;2)\), \( B=(-3;-1)\), \( C=(-9;4)\) et \( D=(-1;5)\) dans un repère orthonormé. Calculer la longueur du segment \( [AB]\) et les coordonnées du milieu du segment \( [CD]\).

 $l^2=10$,$l=sqrt(10)$,$M=(-5.0,4.5)$\item
Est-ce que le triangle formé par les points \( A(9;-10)\), \( B(5;-6)\) et \( C(18;-7)\) est isocèle ?

False
\end{enumerate}
} 
\vspace{2cm}
\vbox{81
\emph{Toutes les réponses doivent être justifiées par un calcul accompagné d'un raisonnement.}
\begin{enumerate}\item
Placer les points \( A=(10;3)\), \( B=(10;-8)\), \( C=(-2;2)\) et \( D=(-9;7)\) dans un repère orthonormé. Calculer la longueur du segment \( [AB]\) et les coordonnées du milieu du segment \( [CD]\).

 $l^2=121$,$l=11$,$M=(-5.5,4.5)$\item
Est-ce que le triangle formé par les points \( A(-10;7)\), \( B(-12;9)\) et \( C(-12;5)\) est rectangle ?

True
\end{enumerate}
} 
\vspace{2cm}
\vbox{82
\emph{Toutes les réponses doivent être justifiées par un calcul accompagné d'un raisonnement.}
\begin{enumerate}\item
Placer les points \( A=(-8;-6)\), \( B=(-9;-8)\), \( C=(-6;8)\) et \( D=(6;0)\) dans un repère orthonormé. Calculer la longueur du segment \( [AB]\) et les coordonnées du milieu du segment \( [CD]\).

 $l^2=5$,$l=sqrt(5)$,$M=(0.0,4.0)$\item
Est-ce que le triangle formé par les points \( A(6;-9)\), \( B(3;-6)\) et \( C(7;-8)\) est rectangle ?

True
\end{enumerate}
} 
\vspace{2cm}
\vbox{83
\emph{Toutes les réponses doivent être justifiées par un calcul accompagné d'un raisonnement.}
\begin{enumerate}\item
Placer les points \( A=(0;-9)\), \( B=(9;-1)\), \( C=(2;8)\) et \( D=(10;-8)\) dans un repère orthonormé. Calculer la longueur du segment \( [AB]\) et les coordonnées du milieu du segment \( [CD]\).

 $l^2=145$,$l=sqrt(145)$,$M=(6.0,0.0)$\item
Est-ce que le triangle formé par les points \( A(2;4)\), \( B(4;2)\) et \( C(9;-1)\) est isocèle ?

False
\end{enumerate}
} 
\vspace{2cm}
\vbox{84
\emph{Toutes les réponses doivent être justifiées par un calcul accompagné d'un raisonnement.}
\begin{enumerate}\item
Placer les points \( A=(1;-4)\), \( B=(7;-5)\), \( C=(-4;-4)\) et \( D=(-8;4)\) dans un repère orthonormé. Calculer la longueur du segment \( [AB]\) et les coordonnées du milieu du segment \( [CD]\).

 $l^2=37$,$l=sqrt(37)$,$M=(-6.0,0.0)$\item
Est-ce que le triangle formé par les points \( A(-10;3)\), \( B(-4;-3)\) et \( C(-6;1)\) est isocèle ?

True
\end{enumerate}
} 
\vspace{2cm}
