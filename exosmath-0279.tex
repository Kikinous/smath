% This is part of Un soupçon de mathématique sans être agressif pour autant
% Copyright (c) 2013
%   Laurent Claessens
% See the file fdl-1.3.txt for copying conditions.

\begin{exercice}\label{exosmath-0279}

    Nous tirons un point \( A\) au hasard parmi tous les points \( (x,y)\) de coordonnées entières telles que \( 0< x\leq 10\) et \( 0< y\leq 10\).
    \begin{enumerate}
        \item
            Quelle est la probabilité que le point \( A\) soit le point \( (3;7)\) ?
        \item
            Quelle est la probabilité que le point \( A\) soit sur la droite \( y=x+5\) ?
        \item
            Soit \( d\) la droite contenant le point \( (0;0)\) et le point \( A\). Quelle est la probabilité que le point \( (1;2)\) soit sur la droite \( d\) ?
    \end{enumerate}

\corrref{smath-0279}
\end{exercice}
