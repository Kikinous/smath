% This is part of Un soupçon de mathématique sans être agressif pour autant
% Copyright (c) 2013-2014
%   Laurent Claessens
% See the file fdl-1.3.txt for copying conditions.

%%%%%%%%%%%%%%%%%%%%%%%%%%%%%%%%%%%%%%%%%%%%%%%%%%%%%%%%%%%%
\begin{MentalActivity}

\begin{mental}
    
            Simplifier : \( \sqrt{12}\)

\end{mental}

\begin{mental}
            Les coordonnées du milieu entre \( (1;-2)\) et \( (6;4)\)
\end{mental}

\begin{mental}
            Simplifier 
            \begin{equation*}
                \frac{ 2xy }{ 4 }
            \end{equation*}
\end{mental}

\begin{mental}
            Simplifier
            \begin{equation*}
                \frac{ 8x+4a }{ 2 }
            \end{equation*}
\end{mental}

\begin{mental}
            Résoudre : \( 3x=12\).
\end{mental}

\begin{mental}
            Résoudre : \( 2x-6=4\).


\end{mental}

    %%%%%%%%%%%%%%%%%%%%%%%%%%%%%%%%%%%%%%%%%%%%%%%%%%%%%%%%%%%%

\end{MentalActivity}
\begin{MentalActivity}

\begin{mental}
            Simplifier \( \sqrt{72}\)
\end{mental}

\begin{mental}
            Calculer la distance entre \( (0;10)\) et \( (2;0)\).
\end{mental}

\begin{mental}
            Résoudre \( \frac{1}{ x }=\frac{ 2 }{ 3 }\).
\end{mental}

\begin{mental}
            Simplifier 
            \begin{equation*}
                \frac{ 12x+3b }{ 21 }.
            \end{equation*}
\end{mental}

\begin{mental}
            Calculer le milieu du segment entre \( (1;3)\) et \( (-3;8)\).
\end{mental}

\begin{mental}
            Résoudre
            \begin{equation*}
                (x+2)(x-1)=0.
            \end{equation*}
\end{mental}

    %%%%%%%%%%%%%%%%%%%%%%%%%%%%%%%%%%%%%%%%%%%%%%%%%%%%%%%%%%%%

\end{MentalActivity}
\begin{MentalActivity}

\begin{mental}
            Résoudre \( (x+1)(x-17)=0\)
\end{mental}

\begin{mental}
            Simplifier
            \begin{equation*}
                \frac{ 4x+8a }{ 2 }.
            \end{equation*}
\end{mental}

\begin{mental}
            Si \( f(x)=x^2-x\), que vaut \( f(4)\) ?
\end{mental}

\begin{mental}
            Quelle est la distance entre \( (0;4)\) et \( (2;-2)\) ?
\end{mental}

\begin{mental}
            La droite représentative de la fonction \( f(x)=3x+2\) est-elle parallèle à celle de \( g(x)=-3x+7\) ?
\end{mental}

    %%%%%%%%%%%%%%%%%%%%%%%%%%%%%%%%%%%%%%%%%%%%%%%%%%%%%%%%%%%%

\end{MentalActivity}
\begin{MentalActivity}

\begin{mental}

    Ceci est un cube.
                
                \input{Fig_MUriGyU.pstricks}
                
                Quelle est la nature du triangle \( AHD\) ?



\end{mental}

\begin{mental}
            Écrire sous forme d'une somme : $(x+3)^2$.
\end{mental}

\begin{mental}
            L'aire d'un carré est de \unit{11}{\centi\meter\squared}. Quelle est la longueur exacte de son côté ?
\end{mental}

\begin{mental}
            Simplifier 
            \begin{equation*}
                \frac{ 8ax+a^2 }{ 2a }.
            \end{equation*}
            

\end{mental}

    %%%%%%%%%%%%%%%%%%%%%%%%%%%%%%%%%%%%%%%%%%%%%%%%%%%%%%%%%%%%

\end{MentalActivity}
\begin{MentalActivity}

\begin{mental}
            Écrire sous forme d'une somme : \( (x-4)^2\)

\end{mental}

\begin{mental}
            Simplifier :
            \begin{equation*}
                \frac{ 4bx+b^2 }{ 2b }.
            \end{equation*}
\end{mental}

\begin{mental}

            Résoudre : \( (x+2)(x-1)=0\).

\end{mental}

\begin{mental}
            Quelle est la longueur du segment joignant \( A(3;7)\) et \( B(4;10)\) ?
\end{mental}

\begin{mental}
            Si \( f(x)=x^2-2\), que vaut \( f(-1)\) ?
            

    

\end{mental}

    %%%%%%%%%%%%%%%%%%%%%%%%%%%%%%%%%%%%%%%%%%%%%%%%%%%%%%%%%%%%

\end{MentalActivity}
\begin{MentalActivity}

\begin{mental}
            Résoudre \( \frac{1}{ x }=4\)
\end{mental}

\begin{mental}
            Simplifier
            \begin{equation*}
                \frac{ x+2 }{ (x+2)(x-3) }.
            \end{equation*}
\end{mental}

\begin{mental}
            Qu'affiche l'algorithme suivant si on entre la valeur \( 4\) ?

\begin{fmpage}{0.9\linewidth}

    Demander \( x\)

    Si \( x < 2\) alors

    \hspace{1cm} \( P\) prend la valeur \( 2x-7 \)

    Si \( x >= 2 \) alors

    \hspace{1cm} \( P\) prend la valeur \( \frac{ 3 }{ 8 }x\)

    Afficher \( P\)

\end{fmpage}
\end{mental}

\begin{mental}
                Résoudre \( 2x+60=4\).


\end{mental}

    %%%%%%%%%%%%%%%%%%%%%%%%%%%%%%%%%%%%%%%%%%%%%%%%%%%%%%%%%%%%

\end{MentalActivity}
\begin{MentalActivity}

\begin{mental}
            Résoudre \( -3x+7=-8\)
\end{mental}

\begin{mental}
            Quel est le signe de \( (x-3)(x+4)\) lorsque \( x=-10\) ?
\end{mental}

\begin{mental}
            Donner un encadrement de \( f(5)\).
    \begin{equation*}
        \begin{array}[]{|c||ccccccccc|}
            \hline
            x&-10&&-3&&4&&6&&10\\
            \hline\hline
            &&&0&&&&2&&\\
            f(x)&&\nearrow&&\searrow&&\nearrow&&\searrow&\\
            &1&&&&-12&&&&2\\
            \hline
        \end{array}
    \end{equation*}
\end{mental}

\begin{mental}
                Qu'affiche le programme suivant ?

\begin{fmpage}{0.9\linewidth}

    $S = 0$

    Pour \( i\) allant de \( 1\) à \( 4\) :

    \hspace{1cm} \( S=S+i\)

    Afficher \( S\)

\end{fmpage}

\end{mental}

    %%%%%%%%%%%%%%%%%%%%%%%%%%%%%%%%%%%%%%%%%%%%%%%%%%%%%%%%%%%%

\end{MentalActivity}
\begin{MentalActivity}


\begin{mental}

        \begin{center}
   \input{Fig_UZlaYZ.pstricks}
        \end{center}

                Le vecteur \( \vect{ AB }\) a pour coordonnées 
                \begin{enumerate}
                    \item
        $                
    \begin{pmatrix}
       -3 \\ 
       -8 
   \end{pmatrix}$
   \item
   \( \begin{pmatrix}
       -3 \\ 
       8 
   \end{pmatrix}\)
   \item
   \( \begin{pmatrix}
       3 \\ 
       -8 
   \end{pmatrix}\)
                \end{enumerate}
            
\end{mental}

\begin{mental}

            Si \( A\) est le point de coordonnées \( (3;-1)\) et \( X\) celui de coordonnées \( (7;-4)\), quelles sont les coordonnées de \( \vect{ AX }\) ?

\end{mental}

\begin{mental}

    Calculer
    \begin{equation*}
        \frac{ 12\times 50+5 }{ 5 }
    \end{equation*}


\end{mental}

\begin{mental}

    Pour quel \( y\) le point \( (4;y)\) est-il sur le graphe de la fonction \( f(x)=3x-2\) ?

\end{mental}

    %%%%%%%%%%%%%%%%%%%%%%%%%%%%%%%%%%%%%%%%%%%%%%%%%%%%%%%%%%%%

\end{MentalActivity}
\begin{MentalActivity}

\begin{mental}
            QCM (une seule réponse correcte). Soient \( A(2;-1)\) et \( B(-1;-1)\). La droite \( (AB)\) \ldots
            \begin{itemize}
                \item
                    a pour équation \( x=-1\)
                \item
                    a pour équation \( y=-1\)
                \item
                    a pour équation \( y=-x\).
            \end{itemize}
\end{mental}

\begin{mental}
    \begin{enumerate}
        \item
            Calculer 
            \begin{equation*}
                1- \frac{1}{ 3 }
            \end{equation*}
        \item
            Calculer 
            \begin{equation*}
                1- \frac{1}{ a }
            \end{equation*}
    \end{enumerate}

\end{mental}


\begin{mental}
    \begin{enumerate}
        \item
            Dire si \( x=3\) est solution de l'inéquation
            \begin{equation*}
                \frac{1}{ x }>10
            \end{equation*}
        \item
            Dire si \( x=5\) est solution de l'inéquation
            \begin{equation*}
                \frac{ 6 }{ x }<3
            \end{equation*}
    \end{enumerate}
\end{mental}

\begin{mental}
    \begin{enumerate}
        \item
            Une boîte contient \( 6\) bonbons et \( 9\) pommes. Que contient un tiers de la boîte ?
        \item
            Simplifier
            \begin{equation*}
                \frac{ 6x+9b }{ 3 }.
            \end{equation*}
    \end{enumerate}
\end{mental}

\begin{mental}
        Ceci est un carré :

        \begin{center}
   \input{Fig_GUEjmmR.pstricks}
        \end{center}

        \begin{itemize}
            \item Si \( AB=3\), que vaut \( AC\) ?
            \item Si \( AB=x\), que vaut \( AC\) ?
            \item Pour quelle valeur de \( AB\) a-t-on \( AC=1\) ?
        \end{itemize}
\end{mental}
\end{MentalActivity}


\begin{MentalActivity}
\begin{mental}
            Calculer
            \begin{equation*}
                52\times 3+12\times 3.
            \end{equation*}
\end{mental}

\begin{mental}
        \begin{itemize}
            \item
    Développer et réduire $(x+1)(x-1)$, 
\item
    Factoriser $x^2-1$.
        \end{itemize}
\end{mental}


\begin{mental}
            Soit la fonction \( f(x)=3x^2-7\) et \( \mC\) la courbe représentative de \( f\).
            \begin{itemize}
            \item Donner les coordonnées d'un point de \( \mC\)
            \item Donner les coordonnées d'un point d'abscisse \( 3\) sur \( \mC\).
            \item Donner les coordonnées d'un point d'ordonnée \( 5\) sur \( \mC\).
            \end{itemize}
\end{mental}

\begin{mental}
    \begin{enumerate}
        \item
            Calculer
            \begin{equation*}
                \frac{1}{ 4 }+\frac{1}{ 3 }
            \end{equation*}
        \item
            Calculer 
            \begin{equation*}
                \frac{1}{ 4 }+\frac{1}{ a }
            \end{equation*}
        \item
            Calculer
            \begin{equation*}
                \frac{1}{ b }+\frac{1}{ a }
            \end{equation*}
    \end{enumerate}
\end{mental}

\end{MentalActivity}


\begin{MentalActivity}

    \begin{mental}
        Vrai ou faux.
        \begin{itemize}
            \item Le point \( A(1;4)\) est sur le graphe de \( f\colon x\mapsto 5x-1\).
            \item Le point \( B(2;0)\) est sur le graphe de \( g\colon x\mapsto x^2-2\).
            \item Le point \( C(-1;10)\) est à l'intersection  des graphes de \( h(x)=3x^2-2x+5\) et \( k(x)=7x+17\).
        \end{itemize}
    \end{mental}

    \begin{mental}
    Factoriser
    \begin{enumerate}
        \item
            \begin{equation*}
                a^2+3a.
            \end{equation*}
        \item
            \begin{equation*}
                (x+1)(x+2)+(x+1)(x-2)
            \end{equation*}
        \item 
    \begin{equation*}
        x+3-(x^2-1)(x+3)
    \end{equation*}
    \end{enumerate}
    \end{mental}

    \begin{mental}
       
        Quelle est l'aire du quadrilatère suivant ?

        \begin{center}
            \input{Fig_IBmsroy.pstricks}
        \end{center}

    \end{mental}

\end{MentalActivity}
