% This is part of Un soupçon de mathématique sans être agressif pour autant
% Copyright (c) 2012
%   Laurent Claessens
% See the file fdl-1.3.txt for copying conditions.

\begin{exercice}\label{exoSeconde-0077}

    \begin{enumerate}
        \item
            Soient les points \( A=(-2;4)\) et \( B=(-1;-1)\). Donner un point \( C\) tel que le triangle \( ABC\) soit rectangle en \( C\).
        \item
            Soient les points \( A=(0;1)\) et \( B=(0;-3)\). Donner un point \( C\) tel que le triangle \( ABC\) soit rectangle en \( A\)
        \item
            Soient les points \( A=(3;1)\) et \( B=(3;7)\). Donner un point \( C\) tel que le triangle \( ABC\) soit rectangle (ici nous n'imposons pas en quel point; c'est au choix).
    \end{enumerate}

\corrref{Seconde-0077}
\end{exercice}
